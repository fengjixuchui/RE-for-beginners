% !TEX program = XeLaTeX
% !TEX encoding = UTF-8
\documentclass[UTF8,nofonts]{ctexart}
\setCJKmainfont[BoldFont=STHeiti,ItalicFont=STKaiti]{STSong}
\setCJKsansfont[BoldFont=STHeiti]{STXihei}
\setCJKmonofont{STFangsong}

\begin{document}

%daveti: translated on Dec 25, 2016 (Merry Xmas!)
%NOTE: above is needed for MacTex.

\subsection*{迷你答疑 mini-FAQ}

% TBT
%\par Q: Is this book simpler/easier than others?
%\par A: No, it is at about the same level as other books of this subject.

% TBT
%\par Q: I'm too frightened to start reading this book, there are more than 1000 pages.
%\par A: All sorts of listings are the bulk of the book.

\par Q: 阅读本书的预备知识是什么?
\par A: 最好对C/C++有基本了解。
%TBT
\par Q: 我能买到俄语或者英语的纸板印刷品吗?
\par A: 抱歉,买不到,因为没有出版商对俄语或者英语版本有兴趣。当然,你可以自己找地方打印装订成书。

\par Q: 有epub/mobi版本吗?
\par A: 本书是用TeX/LaTeX编写和编译, 所以转换成HTML (epub/mobi是一种HTMLs)不是那么容易。\footnote{译者语:国内有一个早先的翻译版本,可以从这个链接下载 (\url{https://github.com/veficos/reverse-engineering-for-beginners})。注意,该版本基于某个较老版本翻译,已经不再和目前版本同步。}


\par Q: 这年头为啥还要学汇编呢?
\par A: 除非你是 \ac{OS} 程序员, 否则你基本不需要写汇编\textemdash{}最新的编译器 (2010s) 已经能产生比手动汇编优化更好的优化代码。 \footnote{推荐一个相关扩展阅读: \InSqBrackets{\AgnerFog}}.

而且,最新的 \ac{CPU}s 相当复杂,汇编知识已经不能帮助理解其内部构造。

然而懂汇编至少对2个领域依然有帮助:
第一个也是最重要的一个,安全/病毒研究 (security/malware research)。另外,在调试程序的时候,对汇编的理解也能帮助理解编译器产生的代码。
因此,本书着重于帮助读者理解汇编语言而不是手动写汇编语言。
这也是为什么本书含有大量的编译器产生的汇编代码。

\par Q: 我点击了一个PDF里的链接,然后如何回到PDF?
\par A: 在Adobe Acrobat Reader点击Alt+LeftArrow。在Evince点击 ``<'' 按键。

\par Q: 我能打印本书或者作为教材教课吗?
\par A: 必须的!本书使用Creative Commons license (CC BY-SA 4.0)就是为了这个目的。

\par Q: 为啥本书免费?这本书很牛比,而且免费,让人生疑。
\par A:  根据我的经验,技术类书籍作者通常通过免费来打广告。而且通常写这类书赚不到什么银子。

\par Q: 如何能找到一个逆向工程师的工作呢?How does one get a job in reverse engineering?
\par A: Reddit上有一个专门的用来招聘逆向工程师的论坛\FNURLREDDIT{}。去那儿看看。

另一个相关的招聘信息可以在\q{netsec} subreddit找到。

% TBT
%\par Q: Compilers' versions in the book are outdated already...
%\par A: No need to follow all steps precisely.
%Use the compilers you already have installed on your \ac{OS}.
%Also, there is: \href{https://godbolt.org/}{Compiler Explorer}.

\par Q: 我想提问。。。
\par A: 给俺发邮件 (\EMAIL)。

\end{document}

