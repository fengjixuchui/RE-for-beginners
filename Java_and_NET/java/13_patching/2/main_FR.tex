% TODO proof-reading
\subsubsection{Second exemple}

Un autre exemple simple de crackme:

\begin{lstlisting}[style=customjava]
public class password
{
	public static void main(String[] args)
	{
		System.out.println("Please enter the password");
		String input = System.console().readLine();
		if (input.equals("secret"))
			System.out.println("password is correct");
		else
			System.out.println("password is not correct");
	}
}
\end{lstlisting}

Chargeons le dans \IDA:

\begin{figure}[H]
\centering
\myincludegraphics{Java_and_NET/java/13_patching/2/1.png}
\caption{IDA}
\end{figure}

Nous voyons ici l'instruction \TT{ifeq} qui effectue le travail.

Son nom signifie \emph{if equal} (si égal), et c'est mal-nommé, un meilleur nom serait
\TT{ifz} (\emph{if zero}, si zéro), i.e, si la valeur sur le \ac{TOS} est zéro, alors
effectuer le saut.

Dans notre exemple, le saut est fait si le mot de passe est incorrect (la méthode
\TT{equals} renvoie \TT{False}, qui est 0).

La toute première idée est de patcher cette instruction.

Il y a deux octets dans l'opcode de \TT{ifeq}, qui encode l'offset du saut.

Pour transformer cette instruction en NOP, nous devons mettre la valeur 3 dans le
3ème octet (car ajouter 3 à l'adresse courante sautera toujours à l'instruction
suivante, puisque la longueur de l'instruction \TT{ifeq} est de 3 octets):

\begin{figure}[H]
\centering
\myincludegraphics{Java_and_NET/java/13_patching/2/2.png}
\caption{IDA}
\end{figure}

Ceci ne fonctionne pas (JRE 1.7):

\begin{lstlisting}
Exception in thread "main" java.lang.VerifyError: Expecting a stackmap frame at branch target 24
Exception Details:
  Location:
    password.main([Ljava/lang/String;)V @21: ifeq
  Reason:
    Expected stackmap frame at this location.
  Bytecode:
    0000000: b200 0212 03b6 0004 b800 05b6 0006 4c2b
    0000010: 1207 b600 0899 0003 b200 0212 09b6 0004
    0000020: a700 0bb2 0002 120a b600 04b1
  Stackmap Table:
    append_frame(@35,Object[#20])
    same_frame(@43)

        at java.lang.Class.getDeclaredMethods0(Native Method)
        at java.lang.Class.privateGetDeclaredMethods(Class.java:2615)
        at java.lang.Class.getMethod0(Class.java:2856)
        at java.lang.Class.getMethod(Class.java:1668)
        at sun.launcher.LauncherHelper.getMainMethod(LauncherHelper.java:494)
        at sun.launcher.LauncherHelper.checkAndLoadMain(LauncherHelper.java:486)
\end{lstlisting}

Mais il faut mentionner que ceci fonctionnait avec le JRE 1.6.

Nous pouvons aussi essayer de remplacer les 3 octets de l'opcode de \TT{ifeq} par
des octets à zéro (\ac{NOP}), mais ça ne fonctionne toujours pas.

Il semble qu'il y a plus de tests de l'état de la pile en JRE 1.7.

OK, remplaçons tout l'appel à la méthode \TT{equals} avec l'instruction \TT{iconst\_1}
et quelques \ac{NOP}s:

\begin{figure}[H]
\centering
\myincludegraphics{Java_and_NET/java/13_patching/2/3.png}
\caption{IDA}
\end{figure}

Il doit toujours y avoir 1 sur le \ac{TOS} lorsque l'instruction \TT{ifeq} est exécutée,
ainsi \TT{ifeq} ne fera jamais le saut.

Ceci fonctionne.
