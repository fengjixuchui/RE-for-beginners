% TODO proof-reading
\subsubsection{Tableaux bi-dimensionnels}

Les tableaux bidimensionnels en Java sont juste des tableaux unidimensionnel de
\emph{références} sur d'autres tableaux uni-dimensionnels.

Créons un tableau bi-dimensionnel:

\begin{lstlisting}[style=customjava]
	public static void main(String[] args)
	{
		int[][] a = new int[5][10];
		a[1][2]=3;
	}
\end{lstlisting}

\begin{lstlisting}
  public static void main(java.lang.String[]);
    flags: ACC_PUBLIC, ACC_STATIC
    Code:
      stack=3, locals=2, args_size=1
         0: iconst_5
         1: bipush        10
         3: multianewarray #2,  2      // class "[[I"
         7: astore_1
         8: aload_1
         9: iconst_1
        10: aaload
        11: iconst_2
        12: iconst_3
        13: iastore
        14: return
\end{lstlisting}

Il est créé en utilisant l'instruction \TT{multianewarray}: le type de l'objet et
ses dimensions sont passés comme opérandes.

La taille du tableau (10*5) est laissée dans la pile (en utilisant les instructions
\TT{iconst\_5} et \TT{bipush}).

Une \emph{référence} à la line \#1 est chargée à l'offset 10 (\TT{iconst\_1} et \TT{aaload}).

La colonne est choisie en utilisant \TT{iconst\_2} à l'offset 11.

La valeur à écrire est mise à l'offset 12.

\TT{iastore} en 13 écrit l'élément du tableau.

Comment un élément est-il accédé?

\begin{lstlisting}[style=customjava]
	public static int get12 (int[][] in)
	{
		return in[1][2];
	}
\end{lstlisting}

\begin{lstlisting}
  public static int get12(int[][]);
    flags: ACC_PUBLIC, ACC_STATIC
    Code:
      stack=2, locals=1, args_size=1
         0: aload_0
         1: iconst_1
         2: aaload
         3: iconst_2
         4: iaload
         5: ireturn
\end{lstlisting}

Un \emph{référence} sur la ligne du tableau est chargée à l'offset 2, la colonne
est mise à l'offset 3, puis \TT{iaload} charge l'élément du tableau.

