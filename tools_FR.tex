\chapter{Outils}

\epigraph{Maintenant que Dennis Yurichev a réalisé ce livre gratuit, il s'agit d'une contribution au monde de la connaissance et de l'éducation gratuite.
Cependant, pour l'amour de la liberté, nous avons besoin d'outils de rétro-ingénierie (libres) afin de remplacer les outils propriétaires mentionnés dans ce livre.}{Richard M. Stallman}

\mysection{Analyse statique}

Outils à utiliser lorsqu'aucun processus n'est en cours d'exécution.

\myindex{Hiew}
\myindex{UNIX!strings}
\myindex{UNIX!xxd}
\myindex{UNIX!od}

\begin{itemize}
\item
(Gratuit, open-source) \emph{ent}\footnote{\url{http://www.fourmilab.ch/random/}}: outil d'analyse d'entropie.
En savoir plus sur l'entropie : \myref{entropy}.

\item
\label{Hiew}
\emph{Hiew}\footnote{\href{http://go.yurichev.com/17035}{hiew.ru}}:
pour de petites modifications de code dans les fichiers binaires.
Inclut un assembleur/désassembleur.

\item {Libre, open-source} \emph{GHex}\footnote{\url{https://wiki.gnome.org/Apps/Ghex}}: éditeur
hexadécimal simple pour Linux.

\item (Libre, open-source) \emph{xxd} et \emph{od}: utilitaires standards UNIX pour réaliser un dump.

\item (Libre, open-source) \emph{strings}: outil *NIX pour rechercher des chaînes ASCII dans des fichiers binaires, fichiers exécutables inclus.
Sysinternals ont une alternative \footnote{\url{https://technet.microsoft.com/en-us/sysinternals/strings}}
qui supporte les larges chaînes de caractères (UTF-16, très utilisé dans Windows).

\item (Libre, open-source) \emph{Binwalk}\footnote{\url{http://binwalk.org/}}: analyser les images firmware.

\item
\myindex{binary grep}
(Libre, open-source) \emph{binary grep}:
un petit utilitaire pour rechercher une séquence d'octets dans un paquet de fichiers,
incluant ceux non exécutables : \BGREPURL.
\myindex{rafind2}
Il y a aussi rafind2 dans rada.re pour le même usage.
\end{itemize}

\subsection{Désassembleurs}

\myindex{IDA}
\myindex{Binary Ninja}
\myindex{BinNavi}
\myindex{objdump}

\begin{itemize}
\item \emph{IDA}. Une ancienne version Freeware est disponible via téléchargement
\footnote{\href{http://go.yurichev.com/17031}{hex-rays.com/products/ida/support/download\_Freeware.shtml}}.
\ShortHotKeyCheatsheet: \myref{sec:IDA_cheatsheet}

% TBT \item \emph{Ghidra}\footnote{\url{https://ghidra-sre.org/}} --- free alternative to IDA from \ac{NSA}.

\item \emph{Binary Ninja}\footnote{\url{http://binary.ninja/}}

\item (Gratuit, open-source) \emph{zynamics BinNavi}\footnote{\url{https://www.zynamics.com/binnavi.html}}

\item (Gratuit, open-source) \emph{objdump}: simple utilitaire en ligne de commandes pour désassembler et réaliser des dumps.

\item (Gratuit, open-source) \emph{readelf}\footnote{\url{https://sourceware.org/binutils/docs/binutils/readelf.html}}:
réaliser des dumps d'informations sur des fichiers ELF.
\end{itemize}

\subsection{Décompilateurs}

Il n'existe qu'un seul décompilateur connu en C, d'excellente qualité et disponible au public \emph{Hex-Rays}:\\
\href{http://go.yurichev.com/17033}{hex-rays.com/products/decompiler/}

Pour en savoir plus: \myref{hex_rays}.
% TBT: to be synced with EN version

\subsection{Comparaison de versions}

Vous pouvez éventuellement les utiliser lorsque vous comparez la version originale d'un exécutable et une version remaniée, pour déterminer ce qui a été corrigé et en déterminer la raison.

\begin{itemize}
\item (Gratuit) \emph{zynamics BinDiff}\footnote{\url{https://www.zynamics.com/software.html}}

\item (Gratuit, open-source) \emph{Diaphora}\footnote{\url{https://github.com/joxeankoret/diaphora}}
\end{itemize}

\mysection{Analyse dynamique}

Outils à utiliser lorsque que le système est en cours d'exploitation ou lorsqu'un processus est en cours d'exécution.

\subsection{Débogueurs}

\myindex{\olly}
\myindex{Radare}
\myindex{GDB}
\myindex{tracer}
\myindex{LLDB}
\myindex{WinDbg}

\begin{itemize}
\item (Gratuit) \emph{OllyDbg}.
Débogueur Win32 très populaire \footnote{\href{http://go.yurichev.com/17032}{ollydbg.de}}.
\ShortHotKeyCheatsheet: \myref{sec:Olly_cheatsheet}

\item (Gratuit, open-source) \emph{GDB}.
Débogueur peu populaire parmi les ingénieurs en rétro-ingénierie, car il est principalement destiné aux programmeurs.
Quelques commandes : \myref{sec:GDB_cheatsheet}.
Il y a une interface graphique pour GDB, ``GDB dashboard''\footnote{\url{https://github.com/cyrus-and/gdb-dashboard}}.

\item (Gratuit, open-source) \emph{LLDB}\footnote{\url{http://lldb.llvm.org/}}.

\item \emph{WinDbg}\footnote{\url{https://developer.microsoft.com/en-us/windows/hardware/windows-driver-kit}}:
débogueur pour le noyau Windows.

\item (Gratuit, open-source) \emph{Radare} \ac{AKA} rada.re \ac{AKA} r2\footnote{\url{http://rada.re/r/}}.
Une interface graphique existe aussi : \emph{ragui}\footnote{\url{http://radare.org/ragui/}}.

\item (Gratuit, open-source) \emph{tracer}.
\label{tracer}
L'auteur utilise souvent \emph{tracer}
\footnote{\href{http://go.yurichev.com/17338}{yurichev.com}}
au lieu d'un débogueur.

L'auteur de ces lignes a finalement arrêté d'utiliser un débogueur, depuis que tout ce dont il a besoin est de repérer les arguments d'une fonction lorsque cette dernière est exécutée, ou l'état des registres à un instant donné.
Le temps de chargement d'un débogueur étant trop long, un petit utilitaire sous le nom de \emph{tracer} a été conçu.
Il fonctionne depuis la ligne de commandes, permettant d'intercepter l'exécution d'une fonction,
en plaçant des breakpoints à des endroits définis, en lisant et en changeant l'état des registres, etc...

N.B.: \emph{tracer} n'évolue pas, parce qu'il a été développé en tant qu'outil de démonstration pour ce livre, et non pas comme un outil dont on se servirait au quotidien.
\end{itemize}

\subsection{Tracer les appels de librairies}

\emph{ltrace}\footnote{\url{http://www.ltrace.org/}}.

\subsection{Tracer les appels système}

\label{strace}
\myindex{strace}
\myindex{dtruss}
\subsubsection{strace / dtruss}

\myindex{syscall}
Montre les appels système (syscalls(\myref{syscalls})) effectués dans l'immédiat.

Par exemple:

\begin{lstlisting}
# strace df -h

...

access("/etc/ld.so.nohwcap", F_OK)      = -1 ENOENT (No such file or directory)
open("/lib/i386-linux-gnu/libc.so.6", O_RDONLY|O_CLOEXEC) = 3
read(3, "\177ELF\1\1\1\0\0\0\0\0\0\0\0\0\3\0\3\0\1\0\0\0\220\232\1\0004\0\0\0"..., 512) = 512
fstat64(3, {st_mode=S_IFREG|0755, st_size=1770984, ...}) = 0
mmap2(NULL, 1780508, PROT_READ|PROT_EXEC, MAP_PRIVATE|MAP_DENYWRITE, 3, 0) = 0xb75b3000
\end{lstlisting}

\myindex{\MacOSX}
\MacOSX a dtruss pour faire la même chose.

\myindex{Cygwin}
Cygwin a également strace, mais de ce que je sais, cela ne fonctionne que pour les fichiers .exe
compilés pour l'environnement Cygwin lui-même.

\subsection{Sniffer le réseau}

\emph{Sniffer} signifie intercepter des informations qui peuvent vous intéresser.

(Gratuit, open-source) \emph{Wireshark}\footnote{\url{https://www.wireshark.org/}} pour sniffer le réseau.
Peut également sniffer les protocoles USB \footnote{\url{https://wiki.wireshark.org/CaptureSetup/USB}}.

Wireshark a un petit (ou vieux) frère \emph{tcpdump}\footnote{\url{http://www.tcpdump.org/}}, outil simple en ligne de commandes.

\subsection{Sysinternals}

\myindex{Sysinternals}
(Gratuit) Sysinternals (développé par Mark Russinovich)
\footnote{\url{https://technet.microsoft.com/en-us/sysinternals/bb842062}}.
Ces outils sont importants et valent la peine d'être étudiés : Process Explorer, Handle, VMMap, TCPView, Process Monitor.

\subsection{Valgrind}

(Gratuit, open-source) un puissant outil pour détecter les fuites mémoire : \url{http://valgrind.org/}.
Grâce à ses puissants mécanismes \ac{JIT} ("Just In Time"), Valgrind est utilisé comme un framework pour d'autres outils.

% TODO network fuzzing

\subsection{Emulateurs}

\begin{itemize}
\item (Gratuit, open-source) \emph{QEMU}\footnote{\url{http://qemu.org}}: émulateur pour différents CPUs et architectures.

\item (Gratuit, open-source) \emph{DosBox}\footnote{\url{https://www.dosbox.com/}}: émulateur MS-DOS, principalement utilisé pour le rétro-gaming.

\item (Gratuit, open-source) \emph{SimH}\footnote{\url{http://simh.trailing-edge.com/}}: émulateur d'anciens ordinateurs, unités centrales, etc...
\end{itemize}

\mysection{Autres outils}

\emph{Microsoft Visual Studio Express}
\footnote{\href{http://go.yurichev.com/17034}{visualstudio.com/en-US/products/visual-studio-express-vs}}:
Version gratuite simplifiée de Visual Studio, pratique pour des études de cas simples.

Quelques options utiles : \myref{sec:MSVC_options}.

Il y a un site web appelé ``Compiler Explorer'', permettant de compiler des petits
morceaux de code et de voir le résultat avec des versions variées de GCC et d'architectures
(au moins x86, ARM, MIPS): \url{http://godbolt.org/}---Je l'aurais utilisé
pour le livre si je l'avais connu!

% TBT
%\subsection{SMT solvers}

%From the reverse engineer's perspective, SMT solvers are used when dealing with
%amateur cryptography,
%symbolic/concolic execution,
%ROP chains generation.

%For more information, read: \url{https://yurichev.com/writings/SAT_SMT_by_example.pdf}.

\subsection{Calculatrices}

Une bonne calculatrice pour les besoins des rétro-ingénieurs doit au moins supporter
les bases décimale, hexadécimale et binaire, ainsi que plusieurs opérations importantes
comme XOR et les décalages.

\begin{itemize}

\item IDA possède une calculatrice intégrée (``?'').

\item rada.re a \emph{rax2}.

\item \url{https://github.com/DennisYurichev/progcalc}

\item En dernier recours, la calculatrice standard de Windows dispose d'un mode
programmeur.

\end{itemize}

\mysection{Un outil manquant ?}

Si vous connaissez un bon outil non listé précédemment, n'hésitez pas à m'en faire la remarque : \\
\TT{\EMAIL}.

