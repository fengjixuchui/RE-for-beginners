\subsubsection{x86}

\myparagraph{MSVC}

Here is what we get after compiling with MSVC 2010:

\lstinputlisting[style=customasmx86]{patterns/04_scanf/1_simple/ex1_MSVC_EN.asm}

\TT{x} is a local variable.

According to the \CCpp standard it must be visible only in this function and not from any other external scope. 
Traditionally, local variables are stored on the stack. 
There are probably other ways to allocate them, but in x86 that is the way it is.

\myindex{x86!\Instructions!PUSH}
The goal of the instruction following the function prologue, \TT{PUSH ECX}, is not to save the \ECX state 
(notice the absence of corresponding \TT{POP ECX} at the function's end).

In fact it allocates 4 bytes on the stack for storing the \TT{x} variable.

\label{stack_frame}
\myindex{\Stack!Stack frame}
\myindex{x86!\Registers!EBP}
\TT{x} is to be accessed with the assistance of the \TT{\_x\$} macro (it equals to -4) and the \EBP register pointing to the current frame.

Over the span of the function's execution, \EBP is pointing to the current \gls{stack frame}
making it possible to access local variables and function arguments via \TT{EBP+offset}.

\myindex{x86!\Registers!ESP}
It is also possible to use \ESP for the same purpose, although that is not very convenient since it changes frequently.
The value of the \EBP could be perceived as a \emph{frozen state} of the value in \ESP at the start of the function's execution.

% FIXME1 это уже было в 02_stack?
Here is a typical \gls{stack frame} layout in 32-bit environment:

\begin{center}
\begin{tabular}{ | l | l | }
\hline
\dots & \dots \\
\hline
EBP-8 & local variable \#2, \MarkedInIDAAs{} \TT{var\_8} \\
\hline
EBP-4 & local variable \#1, \MarkedInIDAAs{} \TT{var\_4} \\
\hline
EBP & saved value of \EBP \\
\hline
EBP+4 & return address \\
\hline
EBP+8 & \argument \#1, \MarkedInIDAAs{} \TT{arg\_0} \\
\hline
EBP+0xC & \argument \#2, \MarkedInIDAAs{} \TT{arg\_4} \\
\hline
EBP+0x10 & \argument \#3, \MarkedInIDAAs{} \TT{arg\_8} \\
\hline
\dots & \dots \\
\hline
\end{tabular}
\end{center}

The \scanf function in our example has two arguments.

The first one is a pointer to the string containing \TT{\%d} and the second is the address of the \TT{x} variable.

\myindex{x86!\Instructions!LEA}
First, the \TT{x} variable's address is loaded into the \EAX register by the \\
\TT{lea eax, DWORD PTR \_x\$[ebp]} instruction.

\LEA stands for \emph{load effective address}, and is often used for forming an address ~(\myref{sec:LEA}).

We could say that in this case \LEA simply stores the sum of the \EBP register value and the \TT{\_x\$} macro in the \EAX register.

This is the same as \INS{lea eax, [ebp-4]}.

So, 4 is being subtracted from the \EBP register value and the result is loaded in the \EAX register.
Next the \EAX register value is pushed into the stack and \scanf is being called.

\printf is being called after that with its first argument --- a pointer to the string:
\TT{You entered \%d...\textbackslash{}n}.

The second argument is prepared with: \TT{mov ecx, [ebp-4]}.
The instruction stores the \TT{x} variable value and not its address, in the \ECX register.

Next the value in the \ECX is stored on the stack and the last \printf is being called.

\input{patterns/04_scanf/1_simple/olly.tex}

\myparagraph{GCC}

Let's try to compile this code in GCC 4.4.1 under Linux:

\lstinputlisting[style=customasmx86]{patterns/04_scanf/1_simple/ex1_GCC.asm}

\myindex{puts() instead of printf()}
GCC replaced the \printf call with call to \puts. The reason for this was explained in ~(\myref{puts}).

% TODO: rewrite
%\RU{Почему \scanf переименовали в \TT{\_\_\_isoc99\_scanf}, я честно говоря, пока не знаю.}
%\EN{Why \scanf is renamed to \TT{\_\_\_isoc99\_scanf}, I do not know yet.}
% 
% Apparently it has to do with the ISO c99 standard compliance. By default GCC allows specifying a standard to adhere to.
% For example if you compile with -std=c89 the outputted assmebly file will contain scanf and not __isoc99__scanf. I guess current GCC version adhares to c99 by default.
% According to my understanding the two implementations differ in the set of suported modifyers (See printf man page)

As in the MSVC example---the arguments are placed on the stack using the \MOV instruction.

\myparagraph{By the way}

This simple example is a demonstration of the fact that the compiler translates
a list of expressions in \CCpp-block into a sequential list of instructions.
There is nothing in between expressions in \CCpp, and so in the resulting machine code, 
there are nothing between, the control flow slips from one expression to the next one.

