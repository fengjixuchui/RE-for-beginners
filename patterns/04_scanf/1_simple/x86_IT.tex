\subsubsection{x86}

\myparagraph{MSVC}

Questo è ciò che si ottiene dopo la compilazione con MSVC 2010:

\lstinputlisting[style=customasmx86]{patterns/04_scanf/1_simple/ex1_MSVC_IT.asm}

\TT{x} è una variabile locale.

In base allo standard \CCpp deve essere visibile soltanto in questa funzione e non in altri ambiti (esterni alla funzione).
Tradizionalmente, le variabili locali sono memorizzate sullo stack. 
Ci sono probabilmente altri modi per allocarle, ma in x86 è così.

\myindex{x86!\Instructions!PUSH}
Lo scopo dell'istruzione che segue il prologo della funzione, \TT{PUSH ECX}, non è quello di salvare lo stato di \ECX  
(si noti infatti l'assenza della corrispondente istruzione \TT{POP ECX} alla fine della funzione).

Infatti alloca 4 byte sullo stack per memorizzare la variabile \TT{x}.

\label{stack_frame}
\myindex{\Stack!Stack frame}
\myindex{x86!\Registers!EBP}
\TT{x} sarà acceduta con l'aiuto della macro \TT{\_x\$} (che è uguale a -4) ed il registro \EBP che punta al frame corrente.

Durante l'esecuzione delle funziona, \EBP punta allo \gls{stack frame} corrente 
rendendo possibile accedere alle variabili locali ed agli argomenti della funzione attraverso \TT{EBP+offset}.

\myindex{x86!\Registers!ESP}
E' anche possibile usare \ESP per lo stesso scopo, tuttavia non è molto conveniente poichè cambia di frequente.
Il valore di \EBP può essere pensato come uno \emph{stato congelato} del valore in \ESP all'inizio dell'esecuzione della funzione.

% FIXME1 это уже было в 02_stack?
Questo è un tipico layout di uno \gls{stack frame} in un ambiente a 32-bit:

\begin{center}
\begin{tabular}{ | l | l | }
\hline
\dots & \dots \\
\hline
EBP-8 & local variable \#2, \MarkedInIDAAs{} \TT{var\_8} \\
\hline
EBP-4 & local variable \#1, \MarkedInIDAAs{} \TT{var\_4} \\
\hline
EBP & saved value of \EBP \\
\hline
EBP+4 & return address \\
\hline
EBP+8 & \argument \#1, \MarkedInIDAAs{} \TT{arg\_0} \\
\hline
EBP+0xC & \argument \#2, \MarkedInIDAAs{} \TT{arg\_4} \\
\hline
EBP+0x10 & \argument \#3, \MarkedInIDAAs{} \TT{arg\_8} \\
\hline
\dots & \dots \\
\hline
\end{tabular}
\end{center}

La funzione \scanf nel nostro esempio ha due argomenti.
Il primo è un puntatore alla stringa contenente \TT{\%d} e il secondo è l'indirizzo della variabile \TT{x}.

\myindex{x86!\Instructions!LEA}
Per prima cosa l'indirizzo della variabile \TT{x} è caricato nel registro \EAX dall'istruzione \TT{lea eax, DWORD PTR \_x\$[ebp]}.

\LEA sta per \emph{load effective address}, ed è spesso usata per formare un indirizzo ~(\myref{sec:LEA}).

Potremmo dire che in questo caso \LEA memorizza semplicemente la somma del valore nel registro \EBP e della macro \TT{\_x\$} nel registro \EAX.

E' l'equivalente di \INS{lea eax, [ebp-4]}.

Quindi, 4 viene sottratto dal valore del registro \EBP ed il risultato è memorizzato nel registro \EAX.
Successivamente il registro \EAX è messo sullo stack (push) e \scanf viene chiamata.

\printf viene chiamata subito dopo con il suo primo argomento --- un puntatore alla stringa:
\TT{You entered \%d...\textbackslash{}n}.

Il secondo argomento è preparato con: \TT{mov ecx, [ebp-4]}.
L'istruzione memorizza il valore della variabile \TT{x},  non il suo indirizzo, nel registro \ECX.

Successivamente il valore in \ECX è memorizzato sullo stack e l'ultima \printf viene chiamata.

\input{patterns/04_scanf/1_simple/olly.tex}

\myparagraph{GCC}

Proviamo a compilare questo codice con GCC 4.4.1 su Linux:

\lstinputlisting[style=customasmx86]{patterns/04_scanf/1_simple/ex1_GCC.asm}

\myindex{puts() instead of printf()}
GCC ha sostituito la chiamata a \printf con \puts. La ragione per cui ciò avviene è stata spiegata in ~(\myref{puts}).

% TODO: rewrite
%\RU{Почему \scanf переименовали в \TT{\_\_\_isoc99\_scanf}, я честно говоря, пока не знаю.}
%\EN{Why \scanf is renamed to \TT{\_\_\_isoc99\_scanf}, I do not know yet.}
% 
% Apparently it has to do with the ISO c99 standard compliance. By default GCC allows specifying a standard to adhere to.
% For example if you compile with -std=c89 the outputted assmebly file will contain scanf and not __isoc99__scanf. I guess current GCC version adhares to c99 by default.
% According to my understanding the two implementations differ in the set of suported modifyers (See printf man page)

Come nell'esempio compilato con MSVC ---gli argomenti sono messi sullo stack utilizzando l'istruzione \MOV.

\myparagraph{A proposito}

Questo semplice esempio è la dimostrazione del fatto che il compilatore traduce una lista di espressioni in blocchi \CCpp in una lista sequenziale di istruzioni.
Non c'è nulla tra le espressioni in \CCpp, e di conseguenza, nemmeno nel codice macchina risultate. Il flusso di controllo dunque passa da un'istruzione alla successiva.
