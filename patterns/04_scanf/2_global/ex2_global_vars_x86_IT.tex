\subsubsection{MSVC: x86}

\lstinputlisting[style=customasmx86]{patterns/04_scanf/2_global/ex2_MSVC.asm}

In questo caso la variabile \TT{x} è definita nel segmento \TT{\_DATA} e per essa non viene allocata alcuna memoria nello stack locale. Viene acceduta direttamente, non attraverso lo stack.
Le variabili globali non inizializzate non occupano spazio nel file eseguibile 
(che motivo ci sarebbe di allocare spazio per varibaili inizialmente settate a zero?), 
ma quando qualcuno accede al loro indirizzo, l' \ac{OS} allocherà un bloco di zeri al loro posto \footnote{Questo è il modo in cui vunziona una \ac{VM} }.

Adesso assegnamo esplicitamente un valore alla variabile:

\lstinputlisting[style=customc]{patterns/04_scanf/2_global/default_value_IT.c}

Otteniamo:

\begin{lstlisting}[style=customasmx86]
_DATA	SEGMENT
_x	DD	0aH

...
\end{lstlisting}

Vediamo qui un valore \TT{0xA} di tipo DWORD (DD sta per DWORD = 32 bit) per questa variabile.

Analizzando con \IDA il file .exe compilato, notiamo che la variabile \emph{x} è collocata all'inizio del segmento \TT{\_DATA}, e dopo di essa vediamo le stringhe testuali.

Analizzando l'eseguibile dell'esempio precedente con \IDA, vedremo qualcosa del genere dove il valore di \emph{x} non era stato impostato:

\lstinputlisting[caption=\IDA,style=customasmx86]{patterns/04_scanf/2_global/IDA.lst}

\label{BSSClearedByCStd}
\TT{\_x} è contrassegnata con il simbolo \TT{?} insieme al resto delle variabili che non necessitano di essere inizializzate.
Ciò implica che dopo il caricamento del .exe in memoria, verrà allocato spazio riempito di zeri per tutte queste variabili \InSqBrackets{\CNineNineStd 6.7.8p10}.
Ma nel file .exe tutte le variabili non inizializzate non occupano alcuno spazio.
Questo risulta molto conveniente ad esempio nel caso di array molto grandi.

\input{patterns/04_scanf/2_global/olly.tex}

\subsubsection{GCC: x86}

\myindex{ELF}
La situazione in Linux è pressoché identica, con la differenza che le variabili non inizializzate sono collocate nel segmento \TT{\_bss}. 
In un file \ac{ELF} questo segmento ha i seguenti attributi:

\begin{lstlisting}
; Segment type: Uninitialized
; Segment permissions: Read/Write
\end{lstlisting}

Se invece si inizializza la variabile con un qualunque valore, es. 10, 
sarà collocata nel segmento \TT{\_data}, che ha i seguenti attributi:

\begin{lstlisting}
; Segment type: Pure data
; Segment permissions: Read/Write
\end{lstlisting}

\subsubsection{MSVC: x64}

\lstinputlisting[caption=MSVC 2012 x64,style=customasmx86]{patterns/04_scanf/2_global/ex2_MSVC_x64_IT.asm}

Il codice è pressoché identico a quello in x86.
Si noti che l'indirizzo della variabile $x$ è passato a \TT{scanf()} usando un'istruzione \LEA ,
mentre il valore della variabile è passato alla seconda \printf usando un'istruzione \MOV.
\TT{DWORD PTR}--- è parte del linguaggio assembly (non ha a che vedere con il codice macchina),
indica che la dimensione del dato della variabile è 32-bit e l'istruzione \MOV deve essere codificata in accordo alla dimensione.

