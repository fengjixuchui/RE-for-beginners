\subsection{scanf()}

Come già detto in precedenza, usare \scanf oggi è un pò antiquato.
Se proprio dobbiamo, è necessario almeno controllare se \scanf termina correttamente senza errori.

\lstinputlisting[style=customc]{patterns/04_scanf/3_checking_retval/ex3.c}

Per standard, la funzione \scanf\footnote{scanf, wscanf: \href{http://go.yurichev.com/17255}{MSDN}} restituisce il numero di campi che è riuscita a leggere con successo.
Nel nostro caso, se tutto va bene e l'utente inserisce un numero, \scanf restituisce 1, oppure 0 (o \ac{EOF}) in caso di errore. 

Aggiungiamo un po' di codice C per controllare che \scanf restituisca un valore e stampi un messaggio in caso di errore.

Funziona come ci si aspetta:

\begin{lstlisting}
C:\...>ex3.exe
Enter X:
123
You entered 123...

C:\...>ex3.exe
Enter X:
ouch
What you entered? Huh?
\end{lstlisting}

% subsections
\input{patterns/04_scanf/3_checking_retval/x86}
\input{patterns/04_scanf/3_checking_retval/x64}
\input{patterns/04_scanf/3_checking_retval/ARM}
\subsubsection{MIPS}

\lstinputlisting[caption=\Optimizing GCC 4.4.5 (IDA),style=customasmMIPS]{patterns/04_scanf/3_checking_retval/MIPS_O3_IDA.lst}

\myindex{MIPS!\Instructions!BEQ}

\EN{\input{patterns/04_scanf/3_checking_retval/MIPS_EN}}
\RU{\input{patterns/04_scanf/3_checking_retval/MIPS_RU}}
\IT{\scanf restituisce il risultato del suo lavoro nel registro \$V0. Ciò viene controllato all'indirizzo 0x004006E4
confrontando il valore in \$V0 con quello in \$V1 (1 era stato memorizzato in \$V1 precedentemente, a 0x004006DC).
\INS{BEQ} sta per \q{Branch Equal}.
Se i due valori sono uguali (cioè \scanf è terminata con successo), l'esecuzione salta all'indirizzo 0x0040070C.

}
\JA{\input{patterns/04_scanf/3_checking_retval/MIPS_JA}}
\FR{\scanf renvoie le résultat de son traitement dans le registre \$V0. Il est testé à l'adresse 0x004006E4
en comparant la valeur dans \$V0 avec celle dans \$V1 (1 a été stocké dans \$V1 plus tôt, en 0x004006DC).
\INS{BEQ} signifie \q{Branch Equal} (branchement si égal).
Si les deux valeurs sont égales (i.e., succès), l'exécution saute à l'adresse 0x0040070C.
}



\subsubsection{\Exercise}

\myindex{x86!\Instructions!Jcc}
\myindex{ARM!\Instructions!Bcc}
Come possiamo vedere, le istruzioni \INS{JNE}/\INS{JNZ} possono essere scambiate con \INS{JE}/\INS{JZ} e viceversa.
(lo stesso vale per \INS{BNE} e \INS{BEQ}).
Ma se ciò avviene i blocchi base devono anch'essi essere scambiati. Provate a farlo in qualche esempio.

