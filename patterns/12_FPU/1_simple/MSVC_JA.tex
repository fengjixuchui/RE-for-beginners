\myparagraph{MSVC}

MSVC 2010でコンパイルしましょう。

\lstinputlisting[caption=MSVC 2010: \ttf{},style=customasmx86]{patterns/12_FPU/1_simple/MSVC_JA.asm}

\FLD はスタックから8バイトを取り出し、その数値を\ST{0}レジスタにロードし、内部80ビットフォーマット
(\emph{拡張精度})に自動的に変換します。

\myindex{x86!\Instructions!FDIV}

\FDIV は、\ST{0}の値をアドレス\GTT{\_\_real@40091eb851eb851f}~に格納された数値で除算します。
値3.14はそこにエンコードされます。
アセンブリ構文は浮動小数点数をサポートしていないので、64ビットIEEE 754形式での3.14の16進表現です。

\FDIV \ST{0}の実行後に\gls{quotient}が保持されます。

\myindex{x86!\Instructions!FDIVP}

ちなみに、 \FDIVP 命令もあります。これは、\ST{1}を\ST{0}で除算し、
これらの値をスタックからポップし、その結果をプッシュします。
あなたがForth言語 を知っていれば、
すぐにこれがスタックマシン であることがわかります。

後続の \FLD 命令は、 $b$ の値をスタックにプッシュします。

その後、商は\ST{1}に置かれ、\ST{0}は $b$ の値を持ちます。

\myindex{x86!\Instructions!FMUL}

次の \FMUL 命令は乗算を行います。\ST{0}の $b$ は\GTT{\_\_real@4010666666666666}
(そこには4.1が入る)の値で乗算され、結果は\ST{0}レジスタに残ります。

\myindex{x86!\Instructions!FADDP}

最後の \FADDP 命令は、スタックの先頭に2つの値を加算し、結果を\ST{1}に格納した後、
\ST{0}の値をポップし、\ST{0}のスタックの先頭に結果を残します。

関数はその結果を\ST{0}レジスタに戻す必要があるため、
\FADDP 後の関数エピローグ以外の命令はありません。

\input{patterns/12_FPU/1_simple/olly_JA.tex}
