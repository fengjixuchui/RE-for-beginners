\mysection{\FPUChapterName}
\label{sec:FPU}

\ac{FPU}~--- блок в процессоре работающий с числами с плавающей запятой.

Раньше он назывался \q{сопроцессором} и он стоит немного в стороне от \ac{CPU}.

\subsection{IEEE 754}

Число с плавающей точкой в формате IEEE 754 состоит из \emph{знака}, \emph{мантиссы}\footnote{\emph{significand} или \emph{fraction} 
в англоязычной литературе} и \emph{экспоненты}.

\subsection{x86}

Перед изучением \ac{FPU} в x86 полезно ознакомиться с тем как работают стековые машины
или ознакомиться с основами языка Forth.

\myindex{Intel!80486}
\myindex{Intel!FPU}
Интересен факт, что в свое время (до 80486) сопроцессор был отдельным чипом на материнской плате, 
и вследствие его высокой цены, он не всегда присутствовал. Его можно было докупить и установить отдельно
\footnote{Например, Джон Кармак использовал в своей игре Doom числа с фиксированной запятой, хранящиеся
в обычных 32-битных \ac{GPR} (16 бит на целую часть и 16 на дробную),
чтобы Doom работал на 32-битных компьютерах без FPU, т.е. 80386 и 80486 SX.}.
Начиная с 80486 DX в состав процессора всегда входит FPU.

\myindex{x86!\Instructions!FWAIT}
Этот факт может напоминать такой рудимент как наличие инструкции \INS{FWAIT}, которая заставляет
\ac{CPU} ожидать, пока \ac{FPU} закончит работу.
Другой рудимент это тот факт, что опкоды \ac{FPU}-инструкций начинаются с т.н. \q{escape}-опкодов 
(\GTT{D8..DF}) как опкоды, передающиеся в отдельный сопроцессор.

\myindex{IEEE 754}
\label{FPU_is_stack}
FPU имеет стек из восьми 80-битных регистров: \ST{0}..\ST{7}.
Для краткости, \IDA и \olly отображают \ST{0} как \GTT{ST},
что в некоторых учебниках и документациях означает \q{Stack Top} (\q{вершина стека}).
Каждый регистр может содержать число в формате IEEE 754.

\subsection{ARM, MIPS, x86/x64 SIMD}

В ARM и MIPS FPU это не стек, а просто набор регистров, к которым можно обращаться произвольно, как к \ac{GPR}.

Такая же идеология применяется в расширениях SIMD в процессорах x86/x64.

\subsection{\CCpp}

\myindex{float}
\myindex{double}
В стандартных \CCpp имеются два типа для работы с числами с плавающей запятой: 
\Tfloat (\emph{число одинарной точности}, 32 бита)
\footnote{Формат представления чисел с плавающей точкой одинарной точности затрагивается в разделе 
\emph{\WorkingWithFloatAsWithStructSubSubSectionName}~(\myref{sec:floatasstruct}).}
и \Tdouble (\emph{число двойной точности}, 64 бита).

В \InSqBrackets{\TAOCPvolII 246} мы можем найти что \emph{single-precision} означает, что значение с плавающей точкой может быть
помещено в одно [32-битное] машинное слово, а \emph{doulbe-precision} означает, что оно размещено в двух словах (64 бита).

\myindex{long double}
GCC также поддерживает тип \emph{long double} (\emph{extended precision}, 80 бит), но MSVC~--- нет.

Несмотря на то, что \Tfloat занимает столько же места, сколько и \Tint на 32-битной архитектуре, 
представление чисел, разумеется, совершенно другое.

\EN{% TODO translate
\mysection{Breaking simple executable cryptor}

I've got an executable file which is encrypted by relatively simple encryption.
\href{\GitHubBlobMasterURL/examples/simple_exec_crypto/files/cipher.bin}{Here is it} (only executable section is left here).

First, all encryption function does is just adds number of position in buffer to the byte.
Here is how this can be encoded in Python:

\begin{lstlisting}[caption=Python script,style=custompy]
#!/usr/bin/env python
def e(i, k):
    return chr ((ord(i)+k) % 256)

def encrypt(buf):
    return e(buf[0], 0)+ e(buf[1], 1)+ e(buf[2], 2) + e(buf[3], 3)+ e(buf[4], 4)+ e(buf[5], 5)+ e(buf[6], 6)+ e(buf[7], 7)+
           e(buf[8], 8)+ e(buf[9], 9)+ e(buf[10], 10)+ e(buf[11], 11)+ e(buf[12], 12)+ e(buf[13], 13)+ e(buf[14], 14)+ e(buf[15], 15)
\end{lstlisting}

Hence, if you encrypt buffer with 16 zeros, you'll get \emph{0, 1, 2, 3 ... 12, 13, 14, 15}.

\myindex{Propagating Cipher Block Chaining}
Propagating Cipher Block Chaining (PCBC) is also used, here is how it works:

\begin{figure}[H]
\centering
\myincludegraphics{examples/simple_exec_crypto/601px-PCBC_encryption.png}
\caption{Propagating Cipher Block Chaining encryption (image is taken from Wikipedia article)}
\end{figure}

The problem is that it's too boring to recover IV (Initialization Vector) each time.
Brute-force is also not an option, because IV is too long (16 bytes).
Let's see, if it's possible to recover IV for arbitrary encrypted executable file?

Let's try simple frequency analysis.
This is 32-bit x86 executable code, so let's gather statistics about most frequent bytes and opcodes.
I tried huge oracle.exe file from Oracle RDBMS version 11.2 for windows x86 and I've found that the most frequent byte (no surprise) is zero (~10\%).
The next most frequent byte is (again, no surprise) 0xFF (~5\%).
The next is 0x8B (~5\%).

\myindex{x86!\Instructions!MOV}
0x8B is opcode for \INS{MOV}, this is indeed one of the most busy x86 instructions.
Now what about popularity of zero byte?
If compiler needs to encode value bigger than 127, it has to use 32-bit displacement instead of 8-bit one, but large values are very rare,
so it is padded by zeros.
\myindex{x86!\Instructions!LEA}
\myindex{x86!\Instructions!PUSH}
\myindex{x86!\Instructions!CALL}
This is at least in \INS{LEA}, \INS{MOV}, \INS{PUSH}, \INS{CALL}.

For example:

\begin{lstlisting}[style=customasmx86]
8D B0 28 01 00 00                 lea     esi, [eax+128h]
8D BF 40 38 00 00                 lea     edi, [edi+3840h]
\end{lstlisting}

Displacements bigger than 127 are very popular, but they are rarely exceeds 0x10000
(indeed, such large memory buffers/structures are also rare).

Same story with \INS{MOV}, large constants are rare, the most heavily used are 0, 1, 10, 100, $2^n$, and so on.
Compiler has to pad small constants by zeros to represent them as 32-bit values:

\begin{lstlisting}[style=customasmx86]
BF 02 00 00 00                    mov     edi, 2
BF 01 00 00 00                    mov     edi, 1
\end{lstlisting}

Now about 00 and FF bytes combined: jumps (including conditional) and calls can pass execution flow forward or backwards, but very often,
within the limits of the current executable module.
If forward, displacement is not very big and also padded with zeros.
If backwards, displacement is represented as negative value, so padded with FF bytes.
For example, transfer execution flow forward:

\begin{lstlisting}[style=customasmx86]
E8 43 0C 00 00                    call    _function1
E8 5C 00 00 00                    call    _function2
0F 84 F0 0A 00 00                 jz      loc_4F09A0
0F 84 EB 00 00 00                 jz      loc_4EFBB8
\end{lstlisting}

Backwards:

\begin{lstlisting}[style=customasmx86]
E8 79 0C FE FF                    call    _function1
E8 F4 16 FF FF                    call    _function2
0F 84 F8 FB FF FF                 jz      loc_8212BC
0F 84 06 FD FF FF                 jz      loc_FF1E7D
\end{lstlisting}

FF byte is also very often occurred in negative displacements like these:

\begin{lstlisting}[style=customasmx86]
8D 85 1E FF FF FF                 lea     eax, [ebp-0E2h]
8D 95 F8 5C FF FF                 lea     edx, [ebp-0A308h]
\end{lstlisting}

So far so good. Now we have to try various 16-byte keys, decrypt executable section and measure how often 00, FF and 8B bytes are occurred.
Let's also keep in sight how PCBC decryption works:

\begin{figure}[H]
\centering
\myincludegraphics{examples/simple_exec_crypto/640px-PCBC_decryption.png}
\caption{Propagating Cipher Block Chaining decryption (image is taken from Wikipedia article)}
\end{figure}

The good news is that we don't really have to decrypt whole piece of data, but only slice by slice, this is exactly how I did in my previous example: \myref{XOR_mask_2}.

Now I'm trying all possible bytes (0..255) for each byte in key and just pick the byte producing maximal amount of 00/FF/8B bytes in a decrypted slice:

\begin{lstlisting}[style=custompy]
#!/usr/bin/env python
import sys, hexdump, array, string, operator

KEY_LEN=16

def chunks(l, n):
    # split n by l-byte chunks
    # https://stackoverflow.com/q/312443
    n = max(1, n)
    return [l[i:i + n] for i in range(0, len(l), n)]

def read_file(fname):
    file=open(fname, mode='rb')
    content=file.read()
    file.close()
    return content

def decrypt_byte (c, key):
    return chr((ord(c)-key) % 256)

def XOR_PCBC_step (IV, buf, k):
    prev=IV
    rt=""
    for c in buf:
	new_c=decrypt_byte(c, k)
        plain=chr(ord(new_c)^ord(prev))
	prev=chr(ord(c)^ord(plain))
	rt=rt+plain
    return rt

each_Nth_byte=[""]*KEY_LEN

content=read_file(sys.argv[1])
# split input by 16-byte chunks:
all_chunks=chunks(content, KEY_LEN)
for c in all_chunks:
    for i in range(KEY_LEN):
        each_Nth_byte[i]=each_Nth_byte[i] + c[i]

# try each byte of key
for N in range(KEY_LEN):
    print "N=", N
    stat={}
    for i in range(256):
        tmp_key=chr(i)
	tmp=XOR_PCBC_step(tmp_key,each_Nth_byte[N], N)
        # count 0, FFs and 8Bs in decrypted buffer:
	important_bytes=tmp.count('\x00')+tmp.count('\xFF')+tmp.count('\x8B')
	stat[i]=important_bytes
    sorted_stat = sorted(stat.iteritems(), key=operator.itemgetter(1), reverse=True)
    print sorted_stat[0]
\end{lstlisting}

(Source code can be downloaded \href{\GitHubBlobMasterURL/examples/simple_exec_crypto/files/decrypt.py}{here}.)

I run it and here is a key for which 00/FF/8B bytes presence in decrypted buffer is maximal:

\begin{lstlisting}
N= 0
(147, 1224)
N= 1
(94, 1327)
N= 2
(252, 1223)
N= 3
(218, 1266)
N= 4
(38, 1209)
N= 5
(192, 1378)
N= 6
(199, 1204)
N= 7
(213, 1332)
N= 8
(225, 1251)
N= 9
(112, 1223)
N= 10
(143, 1177)
N= 11
(108, 1286)
N= 12
(10, 1164)
N= 13
(3, 1271)
N= 14
(128, 1253)
N= 15
(232, 1330)
\end{lstlisting}

Let's write decryption utility with the key we got:

\begin{lstlisting}[style=custompy]
#!/usr/bin/env python
import sys, hexdump, array

def xor_strings(s,t):
    # \verb|https://en.wikipedia.org/wiki/XOR_cipher#Example_implementation|
    """xor two strings together"""
    return "".join(chr(ord(a)^ord(b)) for a,b in zip(s,t))

IV=array.array('B', [147, 94, 252, 218, 38, 192, 199, 213, 225, 112, 143, 108, 10, 3, 128, 232]).tostring()

def chunks(l, n):
    n = max(1, n)
    return [l[i:i + n] for i in range(0, len(l), n)]

def read_file(fname):
    file=open(fname, mode='rb')
    content=file.read()
    file.close()
    return content

def decrypt_byte(i, k):
    return chr ((ord(i)-k) % 256)

def decrypt(buf):
    return "".join(decrypt_byte(buf[i], i) for i in range(16))

fout=open(sys.argv[2], mode='wb')

prev=IV
content=read_file(sys.argv[1])
tmp=chunks(content, 16)
for c in tmp:
    new_c=decrypt(c)
    p=xor_strings (new_c, prev)
    prev=xor_strings(c, p)
    fout.write(p)
fout.close()
\end{lstlisting}

(Source code can be downloaded \href{\GitHubBlobMasterURL/examples/simple_exec_crypto/files/decrypt2.py}{here}.)

Let's check resulting file:

\lstinputlisting{examples/simple_exec_crypto/objdump_result.txt}

Yes, this is seems correctly disassembled piece of x86 code.
The whole decryped file can be downloaded \href{\GitHubBlobMasterURL/examples/simple_exec_crypto/files/decrypted.bin}{here}.

In fact, this is text section from regedit.exe from Windows 7.
But this example is based on a real case I encountered, so just executable is different (and key), algorithm is the same.

\subsection{Other ideas to consider}

What if I would fail with such simple frequency analysis?
There are other ideas on how to measure correctness of decrypted/decompressed x86 code:

\begin{itemize}

\item Many modern compilers aligns functions on 0x10 border.
So the space left before is filled with NOPs (0x90) or other NOP instructions with known opcodes: \myref{sec:npad}.

\item Perhaps, the most frequent pattern in any assembly language is function call:\\
\TT{PUSH chain / CALL / ADD ESP, X}.
This sequence can easily detected and found.
I've even gathered statistics about average number of function arguments: \myref{args_stat}.
(Hence, this is average length of PUSH chain.)

\end{itemize}

Read more about incorrectly/correctly disassembled code: \myref{ISA_detect}.
}%
\FR{\mysection{Une fonction vide: redux}

Revenons sur l'exemple de la fonction vide \myref{empty_func}.
Maintenant que nous connaissons le prologue et l'épilogue de fonction, ceci est
une fonction vide \myref{lst:empty_func} compilée par GCC sans optimisation:

\lstinputlisting[caption=GCC 8.2 x64 \NonOptimizing (\assemblyOutput),style=customasmx86]{patterns/016_empty_redux/1.s}

C'est \INS{RET}, mais le prologue et l'épilogue de la fonction, probablement, n'ont
pas été optimisés et laissés tels quels.
\INS{NOP} semble être un autre artefact du compilateur.
De toutes façons, la seule instruction effective ici est \INS{RET}.
Toutes les autres instructions peuvent être supprimées (ou optimisées).

}

\EN{% TODO translate
\mysection{Breaking simple executable cryptor}

I've got an executable file which is encrypted by relatively simple encryption.
\href{\GitHubBlobMasterURL/examples/simple_exec_crypto/files/cipher.bin}{Here is it} (only executable section is left here).

First, all encryption function does is just adds number of position in buffer to the byte.
Here is how this can be encoded in Python:

\begin{lstlisting}[caption=Python script,style=custompy]
#!/usr/bin/env python
def e(i, k):
    return chr ((ord(i)+k) % 256)

def encrypt(buf):
    return e(buf[0], 0)+ e(buf[1], 1)+ e(buf[2], 2) + e(buf[3], 3)+ e(buf[4], 4)+ e(buf[5], 5)+ e(buf[6], 6)+ e(buf[7], 7)+
           e(buf[8], 8)+ e(buf[9], 9)+ e(buf[10], 10)+ e(buf[11], 11)+ e(buf[12], 12)+ e(buf[13], 13)+ e(buf[14], 14)+ e(buf[15], 15)
\end{lstlisting}

Hence, if you encrypt buffer with 16 zeros, you'll get \emph{0, 1, 2, 3 ... 12, 13, 14, 15}.

\myindex{Propagating Cipher Block Chaining}
Propagating Cipher Block Chaining (PCBC) is also used, here is how it works:

\begin{figure}[H]
\centering
\myincludegraphics{examples/simple_exec_crypto/601px-PCBC_encryption.png}
\caption{Propagating Cipher Block Chaining encryption (image is taken from Wikipedia article)}
\end{figure}

The problem is that it's too boring to recover IV (Initialization Vector) each time.
Brute-force is also not an option, because IV is too long (16 bytes).
Let's see, if it's possible to recover IV for arbitrary encrypted executable file?

Let's try simple frequency analysis.
This is 32-bit x86 executable code, so let's gather statistics about most frequent bytes and opcodes.
I tried huge oracle.exe file from Oracle RDBMS version 11.2 for windows x86 and I've found that the most frequent byte (no surprise) is zero (~10\%).
The next most frequent byte is (again, no surprise) 0xFF (~5\%).
The next is 0x8B (~5\%).

\myindex{x86!\Instructions!MOV}
0x8B is opcode for \INS{MOV}, this is indeed one of the most busy x86 instructions.
Now what about popularity of zero byte?
If compiler needs to encode value bigger than 127, it has to use 32-bit displacement instead of 8-bit one, but large values are very rare,
so it is padded by zeros.
\myindex{x86!\Instructions!LEA}
\myindex{x86!\Instructions!PUSH}
\myindex{x86!\Instructions!CALL}
This is at least in \INS{LEA}, \INS{MOV}, \INS{PUSH}, \INS{CALL}.

For example:

\begin{lstlisting}[style=customasmx86]
8D B0 28 01 00 00                 lea     esi, [eax+128h]
8D BF 40 38 00 00                 lea     edi, [edi+3840h]
\end{lstlisting}

Displacements bigger than 127 are very popular, but they are rarely exceeds 0x10000
(indeed, such large memory buffers/structures are also rare).

Same story with \INS{MOV}, large constants are rare, the most heavily used are 0, 1, 10, 100, $2^n$, and so on.
Compiler has to pad small constants by zeros to represent them as 32-bit values:

\begin{lstlisting}[style=customasmx86]
BF 02 00 00 00                    mov     edi, 2
BF 01 00 00 00                    mov     edi, 1
\end{lstlisting}

Now about 00 and FF bytes combined: jumps (including conditional) and calls can pass execution flow forward or backwards, but very often,
within the limits of the current executable module.
If forward, displacement is not very big and also padded with zeros.
If backwards, displacement is represented as negative value, so padded with FF bytes.
For example, transfer execution flow forward:

\begin{lstlisting}[style=customasmx86]
E8 43 0C 00 00                    call    _function1
E8 5C 00 00 00                    call    _function2
0F 84 F0 0A 00 00                 jz      loc_4F09A0
0F 84 EB 00 00 00                 jz      loc_4EFBB8
\end{lstlisting}

Backwards:

\begin{lstlisting}[style=customasmx86]
E8 79 0C FE FF                    call    _function1
E8 F4 16 FF FF                    call    _function2
0F 84 F8 FB FF FF                 jz      loc_8212BC
0F 84 06 FD FF FF                 jz      loc_FF1E7D
\end{lstlisting}

FF byte is also very often occurred in negative displacements like these:

\begin{lstlisting}[style=customasmx86]
8D 85 1E FF FF FF                 lea     eax, [ebp-0E2h]
8D 95 F8 5C FF FF                 lea     edx, [ebp-0A308h]
\end{lstlisting}

So far so good. Now we have to try various 16-byte keys, decrypt executable section and measure how often 00, FF and 8B bytes are occurred.
Let's also keep in sight how PCBC decryption works:

\begin{figure}[H]
\centering
\myincludegraphics{examples/simple_exec_crypto/640px-PCBC_decryption.png}
\caption{Propagating Cipher Block Chaining decryption (image is taken from Wikipedia article)}
\end{figure}

The good news is that we don't really have to decrypt whole piece of data, but only slice by slice, this is exactly how I did in my previous example: \myref{XOR_mask_2}.

Now I'm trying all possible bytes (0..255) for each byte in key and just pick the byte producing maximal amount of 00/FF/8B bytes in a decrypted slice:

\begin{lstlisting}[style=custompy]
#!/usr/bin/env python
import sys, hexdump, array, string, operator

KEY_LEN=16

def chunks(l, n):
    # split n by l-byte chunks
    # https://stackoverflow.com/q/312443
    n = max(1, n)
    return [l[i:i + n] for i in range(0, len(l), n)]

def read_file(fname):
    file=open(fname, mode='rb')
    content=file.read()
    file.close()
    return content

def decrypt_byte (c, key):
    return chr((ord(c)-key) % 256)

def XOR_PCBC_step (IV, buf, k):
    prev=IV
    rt=""
    for c in buf:
	new_c=decrypt_byte(c, k)
        plain=chr(ord(new_c)^ord(prev))
	prev=chr(ord(c)^ord(plain))
	rt=rt+plain
    return rt

each_Nth_byte=[""]*KEY_LEN

content=read_file(sys.argv[1])
# split input by 16-byte chunks:
all_chunks=chunks(content, KEY_LEN)
for c in all_chunks:
    for i in range(KEY_LEN):
        each_Nth_byte[i]=each_Nth_byte[i] + c[i]

# try each byte of key
for N in range(KEY_LEN):
    print "N=", N
    stat={}
    for i in range(256):
        tmp_key=chr(i)
	tmp=XOR_PCBC_step(tmp_key,each_Nth_byte[N], N)
        # count 0, FFs and 8Bs in decrypted buffer:
	important_bytes=tmp.count('\x00')+tmp.count('\xFF')+tmp.count('\x8B')
	stat[i]=important_bytes
    sorted_stat = sorted(stat.iteritems(), key=operator.itemgetter(1), reverse=True)
    print sorted_stat[0]
\end{lstlisting}

(Source code can be downloaded \href{\GitHubBlobMasterURL/examples/simple_exec_crypto/files/decrypt.py}{here}.)

I run it and here is a key for which 00/FF/8B bytes presence in decrypted buffer is maximal:

\begin{lstlisting}
N= 0
(147, 1224)
N= 1
(94, 1327)
N= 2
(252, 1223)
N= 3
(218, 1266)
N= 4
(38, 1209)
N= 5
(192, 1378)
N= 6
(199, 1204)
N= 7
(213, 1332)
N= 8
(225, 1251)
N= 9
(112, 1223)
N= 10
(143, 1177)
N= 11
(108, 1286)
N= 12
(10, 1164)
N= 13
(3, 1271)
N= 14
(128, 1253)
N= 15
(232, 1330)
\end{lstlisting}

Let's write decryption utility with the key we got:

\begin{lstlisting}[style=custompy]
#!/usr/bin/env python
import sys, hexdump, array

def xor_strings(s,t):
    # \verb|https://en.wikipedia.org/wiki/XOR_cipher#Example_implementation|
    """xor two strings together"""
    return "".join(chr(ord(a)^ord(b)) for a,b in zip(s,t))

IV=array.array('B', [147, 94, 252, 218, 38, 192, 199, 213, 225, 112, 143, 108, 10, 3, 128, 232]).tostring()

def chunks(l, n):
    n = max(1, n)
    return [l[i:i + n] for i in range(0, len(l), n)]

def read_file(fname):
    file=open(fname, mode='rb')
    content=file.read()
    file.close()
    return content

def decrypt_byte(i, k):
    return chr ((ord(i)-k) % 256)

def decrypt(buf):
    return "".join(decrypt_byte(buf[i], i) for i in range(16))

fout=open(sys.argv[2], mode='wb')

prev=IV
content=read_file(sys.argv[1])
tmp=chunks(content, 16)
for c in tmp:
    new_c=decrypt(c)
    p=xor_strings (new_c, prev)
    prev=xor_strings(c, p)
    fout.write(p)
fout.close()
\end{lstlisting}

(Source code can be downloaded \href{\GitHubBlobMasterURL/examples/simple_exec_crypto/files/decrypt2.py}{here}.)

Let's check resulting file:

\lstinputlisting{examples/simple_exec_crypto/objdump_result.txt}

Yes, this is seems correctly disassembled piece of x86 code.
The whole decryped file can be downloaded \href{\GitHubBlobMasterURL/examples/simple_exec_crypto/files/decrypted.bin}{here}.

In fact, this is text section from regedit.exe from Windows 7.
But this example is based on a real case I encountered, so just executable is different (and key), algorithm is the same.

\subsection{Other ideas to consider}

What if I would fail with such simple frequency analysis?
There are other ideas on how to measure correctness of decrypted/decompressed x86 code:

\begin{itemize}

\item Many modern compilers aligns functions on 0x10 border.
So the space left before is filled with NOPs (0x90) or other NOP instructions with known opcodes: \myref{sec:npad}.

\item Perhaps, the most frequent pattern in any assembly language is function call:\\
\TT{PUSH chain / CALL / ADD ESP, X}.
This sequence can easily detected and found.
I've even gathered statistics about average number of function arguments: \myref{args_stat}.
(Hence, this is average length of PUSH chain.)

\end{itemize}

Read more about incorrectly/correctly disassembled code: \myref{ISA_detect}.
}%
\FR{\mysection{Une fonction vide: redux}

Revenons sur l'exemple de la fonction vide \myref{empty_func}.
Maintenant que nous connaissons le prologue et l'épilogue de fonction, ceci est
une fonction vide \myref{lst:empty_func} compilée par GCC sans optimisation:

\lstinputlisting[caption=GCC 8.2 x64 \NonOptimizing (\assemblyOutput),style=customasmx86]{patterns/016_empty_redux/1.s}

C'est \INS{RET}, mais le prologue et l'épilogue de la fonction, probablement, n'ont
pas été optimisés et laissés tels quels.
\INS{NOP} semble être un autre artefact du compilateur.
De toutes façons, la seule instruction effective ici est \INS{RET}.
Toutes les autres instructions peuvent être supprimées (ou optimisées).

}

\EN{% TODO translate
\mysection{Breaking simple executable cryptor}

I've got an executable file which is encrypted by relatively simple encryption.
\href{\GitHubBlobMasterURL/examples/simple_exec_crypto/files/cipher.bin}{Here is it} (only executable section is left here).

First, all encryption function does is just adds number of position in buffer to the byte.
Here is how this can be encoded in Python:

\begin{lstlisting}[caption=Python script,style=custompy]
#!/usr/bin/env python
def e(i, k):
    return chr ((ord(i)+k) % 256)

def encrypt(buf):
    return e(buf[0], 0)+ e(buf[1], 1)+ e(buf[2], 2) + e(buf[3], 3)+ e(buf[4], 4)+ e(buf[5], 5)+ e(buf[6], 6)+ e(buf[7], 7)+
           e(buf[8], 8)+ e(buf[9], 9)+ e(buf[10], 10)+ e(buf[11], 11)+ e(buf[12], 12)+ e(buf[13], 13)+ e(buf[14], 14)+ e(buf[15], 15)
\end{lstlisting}

Hence, if you encrypt buffer with 16 zeros, you'll get \emph{0, 1, 2, 3 ... 12, 13, 14, 15}.

\myindex{Propagating Cipher Block Chaining}
Propagating Cipher Block Chaining (PCBC) is also used, here is how it works:

\begin{figure}[H]
\centering
\myincludegraphics{examples/simple_exec_crypto/601px-PCBC_encryption.png}
\caption{Propagating Cipher Block Chaining encryption (image is taken from Wikipedia article)}
\end{figure}

The problem is that it's too boring to recover IV (Initialization Vector) each time.
Brute-force is also not an option, because IV is too long (16 bytes).
Let's see, if it's possible to recover IV for arbitrary encrypted executable file?

Let's try simple frequency analysis.
This is 32-bit x86 executable code, so let's gather statistics about most frequent bytes and opcodes.
I tried huge oracle.exe file from Oracle RDBMS version 11.2 for windows x86 and I've found that the most frequent byte (no surprise) is zero (~10\%).
The next most frequent byte is (again, no surprise) 0xFF (~5\%).
The next is 0x8B (~5\%).

\myindex{x86!\Instructions!MOV}
0x8B is opcode for \INS{MOV}, this is indeed one of the most busy x86 instructions.
Now what about popularity of zero byte?
If compiler needs to encode value bigger than 127, it has to use 32-bit displacement instead of 8-bit one, but large values are very rare,
so it is padded by zeros.
\myindex{x86!\Instructions!LEA}
\myindex{x86!\Instructions!PUSH}
\myindex{x86!\Instructions!CALL}
This is at least in \INS{LEA}, \INS{MOV}, \INS{PUSH}, \INS{CALL}.

For example:

\begin{lstlisting}[style=customasmx86]
8D B0 28 01 00 00                 lea     esi, [eax+128h]
8D BF 40 38 00 00                 lea     edi, [edi+3840h]
\end{lstlisting}

Displacements bigger than 127 are very popular, but they are rarely exceeds 0x10000
(indeed, such large memory buffers/structures are also rare).

Same story with \INS{MOV}, large constants are rare, the most heavily used are 0, 1, 10, 100, $2^n$, and so on.
Compiler has to pad small constants by zeros to represent them as 32-bit values:

\begin{lstlisting}[style=customasmx86]
BF 02 00 00 00                    mov     edi, 2
BF 01 00 00 00                    mov     edi, 1
\end{lstlisting}

Now about 00 and FF bytes combined: jumps (including conditional) and calls can pass execution flow forward or backwards, but very often,
within the limits of the current executable module.
If forward, displacement is not very big and also padded with zeros.
If backwards, displacement is represented as negative value, so padded with FF bytes.
For example, transfer execution flow forward:

\begin{lstlisting}[style=customasmx86]
E8 43 0C 00 00                    call    _function1
E8 5C 00 00 00                    call    _function2
0F 84 F0 0A 00 00                 jz      loc_4F09A0
0F 84 EB 00 00 00                 jz      loc_4EFBB8
\end{lstlisting}

Backwards:

\begin{lstlisting}[style=customasmx86]
E8 79 0C FE FF                    call    _function1
E8 F4 16 FF FF                    call    _function2
0F 84 F8 FB FF FF                 jz      loc_8212BC
0F 84 06 FD FF FF                 jz      loc_FF1E7D
\end{lstlisting}

FF byte is also very often occurred in negative displacements like these:

\begin{lstlisting}[style=customasmx86]
8D 85 1E FF FF FF                 lea     eax, [ebp-0E2h]
8D 95 F8 5C FF FF                 lea     edx, [ebp-0A308h]
\end{lstlisting}

So far so good. Now we have to try various 16-byte keys, decrypt executable section and measure how often 00, FF and 8B bytes are occurred.
Let's also keep in sight how PCBC decryption works:

\begin{figure}[H]
\centering
\myincludegraphics{examples/simple_exec_crypto/640px-PCBC_decryption.png}
\caption{Propagating Cipher Block Chaining decryption (image is taken from Wikipedia article)}
\end{figure}

The good news is that we don't really have to decrypt whole piece of data, but only slice by slice, this is exactly how I did in my previous example: \myref{XOR_mask_2}.

Now I'm trying all possible bytes (0..255) for each byte in key and just pick the byte producing maximal amount of 00/FF/8B bytes in a decrypted slice:

\begin{lstlisting}[style=custompy]
#!/usr/bin/env python
import sys, hexdump, array, string, operator

KEY_LEN=16

def chunks(l, n):
    # split n by l-byte chunks
    # https://stackoverflow.com/q/312443
    n = max(1, n)
    return [l[i:i + n] for i in range(0, len(l), n)]

def read_file(fname):
    file=open(fname, mode='rb')
    content=file.read()
    file.close()
    return content

def decrypt_byte (c, key):
    return chr((ord(c)-key) % 256)

def XOR_PCBC_step (IV, buf, k):
    prev=IV
    rt=""
    for c in buf:
	new_c=decrypt_byte(c, k)
        plain=chr(ord(new_c)^ord(prev))
	prev=chr(ord(c)^ord(plain))
	rt=rt+plain
    return rt

each_Nth_byte=[""]*KEY_LEN

content=read_file(sys.argv[1])
# split input by 16-byte chunks:
all_chunks=chunks(content, KEY_LEN)
for c in all_chunks:
    for i in range(KEY_LEN):
        each_Nth_byte[i]=each_Nth_byte[i] + c[i]

# try each byte of key
for N in range(KEY_LEN):
    print "N=", N
    stat={}
    for i in range(256):
        tmp_key=chr(i)
	tmp=XOR_PCBC_step(tmp_key,each_Nth_byte[N], N)
        # count 0, FFs and 8Bs in decrypted buffer:
	important_bytes=tmp.count('\x00')+tmp.count('\xFF')+tmp.count('\x8B')
	stat[i]=important_bytes
    sorted_stat = sorted(stat.iteritems(), key=operator.itemgetter(1), reverse=True)
    print sorted_stat[0]
\end{lstlisting}

(Source code can be downloaded \href{\GitHubBlobMasterURL/examples/simple_exec_crypto/files/decrypt.py}{here}.)

I run it and here is a key for which 00/FF/8B bytes presence in decrypted buffer is maximal:

\begin{lstlisting}
N= 0
(147, 1224)
N= 1
(94, 1327)
N= 2
(252, 1223)
N= 3
(218, 1266)
N= 4
(38, 1209)
N= 5
(192, 1378)
N= 6
(199, 1204)
N= 7
(213, 1332)
N= 8
(225, 1251)
N= 9
(112, 1223)
N= 10
(143, 1177)
N= 11
(108, 1286)
N= 12
(10, 1164)
N= 13
(3, 1271)
N= 14
(128, 1253)
N= 15
(232, 1330)
\end{lstlisting}

Let's write decryption utility with the key we got:

\begin{lstlisting}[style=custompy]
#!/usr/bin/env python
import sys, hexdump, array

def xor_strings(s,t):
    # \verb|https://en.wikipedia.org/wiki/XOR_cipher#Example_implementation|
    """xor two strings together"""
    return "".join(chr(ord(a)^ord(b)) for a,b in zip(s,t))

IV=array.array('B', [147, 94, 252, 218, 38, 192, 199, 213, 225, 112, 143, 108, 10, 3, 128, 232]).tostring()

def chunks(l, n):
    n = max(1, n)
    return [l[i:i + n] for i in range(0, len(l), n)]

def read_file(fname):
    file=open(fname, mode='rb')
    content=file.read()
    file.close()
    return content

def decrypt_byte(i, k):
    return chr ((ord(i)-k) % 256)

def decrypt(buf):
    return "".join(decrypt_byte(buf[i], i) for i in range(16))

fout=open(sys.argv[2], mode='wb')

prev=IV
content=read_file(sys.argv[1])
tmp=chunks(content, 16)
for c in tmp:
    new_c=decrypt(c)
    p=xor_strings (new_c, prev)
    prev=xor_strings(c, p)
    fout.write(p)
fout.close()
\end{lstlisting}

(Source code can be downloaded \href{\GitHubBlobMasterURL/examples/simple_exec_crypto/files/decrypt2.py}{here}.)

Let's check resulting file:

\lstinputlisting{examples/simple_exec_crypto/objdump_result.txt}

Yes, this is seems correctly disassembled piece of x86 code.
The whole decryped file can be downloaded \href{\GitHubBlobMasterURL/examples/simple_exec_crypto/files/decrypted.bin}{here}.

In fact, this is text section from regedit.exe from Windows 7.
But this example is based on a real case I encountered, so just executable is different (and key), algorithm is the same.

\subsection{Other ideas to consider}

What if I would fail with such simple frequency analysis?
There are other ideas on how to measure correctness of decrypted/decompressed x86 code:

\begin{itemize}

\item Many modern compilers aligns functions on 0x10 border.
So the space left before is filled with NOPs (0x90) or other NOP instructions with known opcodes: \myref{sec:npad}.

\item Perhaps, the most frequent pattern in any assembly language is function call:\\
\TT{PUSH chain / CALL / ADD ESP, X}.
This sequence can easily detected and found.
I've even gathered statistics about average number of function arguments: \myref{args_stat}.
(Hence, this is average length of PUSH chain.)

\end{itemize}

Read more about incorrectly/correctly disassembled code: \myref{ISA_detect}.
}%
\FR{\mysection{Une fonction vide: redux}

Revenons sur l'exemple de la fonction vide \myref{empty_func}.
Maintenant que nous connaissons le prologue et l'épilogue de fonction, ceci est
une fonction vide \myref{lst:empty_func} compilée par GCC sans optimisation:

\lstinputlisting[caption=GCC 8.2 x64 \NonOptimizing (\assemblyOutput),style=customasmx86]{patterns/016_empty_redux/1.s}

C'est \INS{RET}, mais le prologue et l'épilogue de la fonction, probablement, n'ont
pas été optimisés et laissés tels quels.
\INS{NOP} semble être un autre artefact du compilateur.
De toutes façons, la seule instruction effective ici est \INS{RET}.
Toutes les autres instructions peuvent être supprimées (ou optimisées).

}


\subsection{Некоторые константы}

В Wikipedia легко найти представление некоторых констант в IEEE 754.
Любопытно узнать, что 0.0 в IEEE 754 представляется как 32 нулевых бита (для одинарной точности) или 64 нулевых бита
(для двойной).
Так что, для записи числа 0.0 в переменную в памяти или регистр, можно пользоваться инструкцией \MOV, или \TT{XOR reg, reg}.
\myindex{\CStandardLibrary!memset()}
Это тем может быть удобно, что если в структуре есть много переменных разных типов, то обычной ф-ций memset()
можно установить все целочисленные переменные в 0, все булевы переменные в \emph{false}, все указатели в NULL,
и все переменные с плавающей точкой (любой точности) в 0.0.

\subsection{Копирование}

По инерции можно подумать, что для загрузки и сохранения (и, следовательно, копирования) чисел в формате
IEEE 754 нужно использовать пару инструкций \INS{FLD}/\INS{FST}.
Тем не менее, этого куда легче достичь используя обычную инструкцию \INS{MOV},
которая, конечно же, просто копирует значения побитово.

\subsection{Стек, калькуляторы и обратная польская запись}

\myindex{Обратная польская запись}
Теперь понятно, почему некоторые старые программируемые калькуляторы используют обратную польскую запись.

Например для сложения 12 и 34 нужно было набрать 12, потом 34, потом нажать знак \q{плюс}.

Это потому что старые калькуляторы просто реализовали стековую машину и это было куда проще, чем обрабатывать сложные выражения со скобками.

Подобный калькулятор все еще присутствует во многих Unix-дистрибутивах: \emph{dc}.

\subsection{80 бит?}

\myindex{Перфокарты}
Внутреннее представление чисел с FPU --- 80-битное.
Странное число, потому как не является числом вида $2^n$.
Имеется гипотеза, что причина, возможно, историческая --- стандартные IBM-овские перфокарты могли кодировать 12 строк по 80 бит.
Раньше было также популярно текстовое разрешение $80 \cdot 25$.

В Wikipedia есть еще одно объяснение: \url{https://en.wikipedia.org/wiki/Extended_precision}.

Если вы знаете более точную причину, просьба сообщить автору: \EMAIL{}.

\subsection{x64}

О том, как происходит работа с числами с плавающей запятой в x86-64, читайте здесь: \myref{floating_SIMD}.

% sections
\input{patterns/12_FPU/exercises}

