\mysection{Il metodo}

Quando l'autore di questo libro ha cominciato ad imparare il C e, successivamente, \Cpp, era solito scrivere piccoli pezzi di codice, compilarli e guardare l'output prodotto in linguaggio assembly. Questo procedimento ha facilitato la comprensione del comportamento del codice che aveva scritto.
\footnote{In effetti lo fa ancora oggi quando non riesce a capire cosa fa un particolare pezzo di codice.}.
Lo ha fatto talmente tante volte che la relazione tra il codice \CCpp e ciò che viene prodotto dal compilatore si è impressa profondamente nella sua mente. E' facile immaginare a colpo d'occhio un contorno della forma e della funzione di un dato codice C.
Magari questa tecnica può rivelarsi utile anche per gli altri.

%There are a lot of examples for both x86/x64 and ARM.
%Those who already familiar with one of architectures, may freely skim over pages.

Ad ogni modo, esiste un utile sito in cui puoi fare lo stesso, con diversi compilatori, invece di installarli sul tuo PC.
Puoi usarlo a questo indirizzo: \url{https://godbolt.org/}.

\section*{\Exercises}

Quando l'autore di questo libro studiava il linguaggio assembly, era solito anche compilare piccola funzioni C e riscriverle gradualmente in assembly tentando di restringere il codice il più possibile.
Questa pratica è oggi probabilmente inutile in uno scenario reale, in quanto è molto difficile competere, in termini di efficienza, con i moderni compilatori. Rappresenta comunque un ottimo modo di acquisire una migliore conoscenza dell'assembly.
Sentitevi quindi liberi di prendere qualunque pezzo di codice assembly da questo libro e cercare di renderlo più piccolo. Tuttavia non dimenticate di testare il vostro risultato.

\section*{Livelli di ottimizzazione e informazioni di debug}

Il codice sorgente può essere compilato da compilatori diversi e con vari livelli di ottimizzazione.
Un compilatore tipico ne prevede solitamente tre, dei quali il livello zero corrisponde a nessuna ottimizzazione (ottimizzazione disabilitata).

L'ottimizzazione può essere orientata verso la dimensione del codice o la sua velocità di esecuzione.
Un compilatore non ottimizzante è più veloce e produce codice più comprensibile (sebbene prolisso), mentre un compilatore ottimizzante è più lento e cerca di produrre codice più veloce in termini di performance (ma non necesariamente più compatto).
Oltre ai livelli di ottimizzazione, un compilatore può includere informazioni di debug nel file risultante, producendo quindi codice che può essere debuggato più facilmente.
Una delle caratteristiche più importanti del codice di ´debug' è che può contenere collegamenti tra ogni riga del codice sorgente e l'indirizzo del corrispondente codice macchina.
I compilatori ottimizzanti tendono invece a produrre output in cui intere righe di codice sorgente possono essere ottimizzate a tal punto da non essere neanche presenti nel codice macchina risultante.
I reverse engineers possono incontrare entrambe le versioni, semplicemente perchè alcuni sviluppatori utilizzano le opzioni di ottimizzazione dei compilatori ed altri no.
A causa di ciò negli esempi proveremo, quando possibile, a lavorare sia sulle versioni di debug che su quelle di release del codice illustrato in questo libro.

A volte in questo libro vengono utilizzate delle versioni di compilatori particolarmente vecchie, in modo da ottenere il più corto (o semplice) blocco di codice.
