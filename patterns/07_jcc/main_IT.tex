\mysection{Jump condizionali}
\label{sec:Jcc}
\myindex{\CLanguageElements!if}

% sections
\EN{% TODO translate
\mysection{Breaking simple executable cryptor}

I've got an executable file which is encrypted by relatively simple encryption.
\href{\GitHubBlobMasterURL/examples/simple_exec_crypto/files/cipher.bin}{Here is it} (only executable section is left here).

First, all encryption function does is just adds number of position in buffer to the byte.
Here is how this can be encoded in Python:

\begin{lstlisting}[caption=Python script,style=custompy]
#!/usr/bin/env python
def e(i, k):
    return chr ((ord(i)+k) % 256)

def encrypt(buf):
    return e(buf[0], 0)+ e(buf[1], 1)+ e(buf[2], 2) + e(buf[3], 3)+ e(buf[4], 4)+ e(buf[5], 5)+ e(buf[6], 6)+ e(buf[7], 7)+
           e(buf[8], 8)+ e(buf[9], 9)+ e(buf[10], 10)+ e(buf[11], 11)+ e(buf[12], 12)+ e(buf[13], 13)+ e(buf[14], 14)+ e(buf[15], 15)
\end{lstlisting}

Hence, if you encrypt buffer with 16 zeros, you'll get \emph{0, 1, 2, 3 ... 12, 13, 14, 15}.

\myindex{Propagating Cipher Block Chaining}
Propagating Cipher Block Chaining (PCBC) is also used, here is how it works:

\begin{figure}[H]
\centering
\myincludegraphics{examples/simple_exec_crypto/601px-PCBC_encryption.png}
\caption{Propagating Cipher Block Chaining encryption (image is taken from Wikipedia article)}
\end{figure}

The problem is that it's too boring to recover IV (Initialization Vector) each time.
Brute-force is also not an option, because IV is too long (16 bytes).
Let's see, if it's possible to recover IV for arbitrary encrypted executable file?

Let's try simple frequency analysis.
This is 32-bit x86 executable code, so let's gather statistics about most frequent bytes and opcodes.
I tried huge oracle.exe file from Oracle RDBMS version 11.2 for windows x86 and I've found that the most frequent byte (no surprise) is zero (~10\%).
The next most frequent byte is (again, no surprise) 0xFF (~5\%).
The next is 0x8B (~5\%).

\myindex{x86!\Instructions!MOV}
0x8B is opcode for \INS{MOV}, this is indeed one of the most busy x86 instructions.
Now what about popularity of zero byte?
If compiler needs to encode value bigger than 127, it has to use 32-bit displacement instead of 8-bit one, but large values are very rare,
so it is padded by zeros.
\myindex{x86!\Instructions!LEA}
\myindex{x86!\Instructions!PUSH}
\myindex{x86!\Instructions!CALL}
This is at least in \INS{LEA}, \INS{MOV}, \INS{PUSH}, \INS{CALL}.

For example:

\begin{lstlisting}[style=customasmx86]
8D B0 28 01 00 00                 lea     esi, [eax+128h]
8D BF 40 38 00 00                 lea     edi, [edi+3840h]
\end{lstlisting}

Displacements bigger than 127 are very popular, but they are rarely exceeds 0x10000
(indeed, such large memory buffers/structures are also rare).

Same story with \INS{MOV}, large constants are rare, the most heavily used are 0, 1, 10, 100, $2^n$, and so on.
Compiler has to pad small constants by zeros to represent them as 32-bit values:

\begin{lstlisting}[style=customasmx86]
BF 02 00 00 00                    mov     edi, 2
BF 01 00 00 00                    mov     edi, 1
\end{lstlisting}

Now about 00 and FF bytes combined: jumps (including conditional) and calls can pass execution flow forward or backwards, but very often,
within the limits of the current executable module.
If forward, displacement is not very big and also padded with zeros.
If backwards, displacement is represented as negative value, so padded with FF bytes.
For example, transfer execution flow forward:

\begin{lstlisting}[style=customasmx86]
E8 43 0C 00 00                    call    _function1
E8 5C 00 00 00                    call    _function2
0F 84 F0 0A 00 00                 jz      loc_4F09A0
0F 84 EB 00 00 00                 jz      loc_4EFBB8
\end{lstlisting}

Backwards:

\begin{lstlisting}[style=customasmx86]
E8 79 0C FE FF                    call    _function1
E8 F4 16 FF FF                    call    _function2
0F 84 F8 FB FF FF                 jz      loc_8212BC
0F 84 06 FD FF FF                 jz      loc_FF1E7D
\end{lstlisting}

FF byte is also very often occurred in negative displacements like these:

\begin{lstlisting}[style=customasmx86]
8D 85 1E FF FF FF                 lea     eax, [ebp-0E2h]
8D 95 F8 5C FF FF                 lea     edx, [ebp-0A308h]
\end{lstlisting}

So far so good. Now we have to try various 16-byte keys, decrypt executable section and measure how often 00, FF and 8B bytes are occurred.
Let's also keep in sight how PCBC decryption works:

\begin{figure}[H]
\centering
\myincludegraphics{examples/simple_exec_crypto/640px-PCBC_decryption.png}
\caption{Propagating Cipher Block Chaining decryption (image is taken from Wikipedia article)}
\end{figure}

The good news is that we don't really have to decrypt whole piece of data, but only slice by slice, this is exactly how I did in my previous example: \myref{XOR_mask_2}.

Now I'm trying all possible bytes (0..255) for each byte in key and just pick the byte producing maximal amount of 00/FF/8B bytes in a decrypted slice:

\begin{lstlisting}[style=custompy]
#!/usr/bin/env python
import sys, hexdump, array, string, operator

KEY_LEN=16

def chunks(l, n):
    # split n by l-byte chunks
    # https://stackoverflow.com/q/312443
    n = max(1, n)
    return [l[i:i + n] for i in range(0, len(l), n)]

def read_file(fname):
    file=open(fname, mode='rb')
    content=file.read()
    file.close()
    return content

def decrypt_byte (c, key):
    return chr((ord(c)-key) % 256)

def XOR_PCBC_step (IV, buf, k):
    prev=IV
    rt=""
    for c in buf:
	new_c=decrypt_byte(c, k)
        plain=chr(ord(new_c)^ord(prev))
	prev=chr(ord(c)^ord(plain))
	rt=rt+plain
    return rt

each_Nth_byte=[""]*KEY_LEN

content=read_file(sys.argv[1])
# split input by 16-byte chunks:
all_chunks=chunks(content, KEY_LEN)
for c in all_chunks:
    for i in range(KEY_LEN):
        each_Nth_byte[i]=each_Nth_byte[i] + c[i]

# try each byte of key
for N in range(KEY_LEN):
    print "N=", N
    stat={}
    for i in range(256):
        tmp_key=chr(i)
	tmp=XOR_PCBC_step(tmp_key,each_Nth_byte[N], N)
        # count 0, FFs and 8Bs in decrypted buffer:
	important_bytes=tmp.count('\x00')+tmp.count('\xFF')+tmp.count('\x8B')
	stat[i]=important_bytes
    sorted_stat = sorted(stat.iteritems(), key=operator.itemgetter(1), reverse=True)
    print sorted_stat[0]
\end{lstlisting}

(Source code can be downloaded \href{\GitHubBlobMasterURL/examples/simple_exec_crypto/files/decrypt.py}{here}.)

I run it and here is a key for which 00/FF/8B bytes presence in decrypted buffer is maximal:

\begin{lstlisting}
N= 0
(147, 1224)
N= 1
(94, 1327)
N= 2
(252, 1223)
N= 3
(218, 1266)
N= 4
(38, 1209)
N= 5
(192, 1378)
N= 6
(199, 1204)
N= 7
(213, 1332)
N= 8
(225, 1251)
N= 9
(112, 1223)
N= 10
(143, 1177)
N= 11
(108, 1286)
N= 12
(10, 1164)
N= 13
(3, 1271)
N= 14
(128, 1253)
N= 15
(232, 1330)
\end{lstlisting}

Let's write decryption utility with the key we got:

\begin{lstlisting}[style=custompy]
#!/usr/bin/env python
import sys, hexdump, array

def xor_strings(s,t):
    # \verb|https://en.wikipedia.org/wiki/XOR_cipher#Example_implementation|
    """xor two strings together"""
    return "".join(chr(ord(a)^ord(b)) for a,b in zip(s,t))

IV=array.array('B', [147, 94, 252, 218, 38, 192, 199, 213, 225, 112, 143, 108, 10, 3, 128, 232]).tostring()

def chunks(l, n):
    n = max(1, n)
    return [l[i:i + n] for i in range(0, len(l), n)]

def read_file(fname):
    file=open(fname, mode='rb')
    content=file.read()
    file.close()
    return content

def decrypt_byte(i, k):
    return chr ((ord(i)-k) % 256)

def decrypt(buf):
    return "".join(decrypt_byte(buf[i], i) for i in range(16))

fout=open(sys.argv[2], mode='wb')

prev=IV
content=read_file(sys.argv[1])
tmp=chunks(content, 16)
for c in tmp:
    new_c=decrypt(c)
    p=xor_strings (new_c, prev)
    prev=xor_strings(c, p)
    fout.write(p)
fout.close()
\end{lstlisting}

(Source code can be downloaded \href{\GitHubBlobMasterURL/examples/simple_exec_crypto/files/decrypt2.py}{here}.)

Let's check resulting file:

\lstinputlisting{examples/simple_exec_crypto/objdump_result.txt}

Yes, this is seems correctly disassembled piece of x86 code.
The whole decryped file can be downloaded \href{\GitHubBlobMasterURL/examples/simple_exec_crypto/files/decrypted.bin}{here}.

In fact, this is text section from regedit.exe from Windows 7.
But this example is based on a real case I encountered, so just executable is different (and key), algorithm is the same.

\subsection{Other ideas to consider}

What if I would fail with such simple frequency analysis?
There are other ideas on how to measure correctness of decrypted/decompressed x86 code:

\begin{itemize}

\item Many modern compilers aligns functions on 0x10 border.
So the space left before is filled with NOPs (0x90) or other NOP instructions with known opcodes: \myref{sec:npad}.

\item Perhaps, the most frequent pattern in any assembly language is function call:\\
\TT{PUSH chain / CALL / ADD ESP, X}.
This sequence can easily detected and found.
I've even gathered statistics about average number of function arguments: \myref{args_stat}.
(Hence, this is average length of PUSH chain.)

\end{itemize}

Read more about incorrectly/correctly disassembled code: \myref{ISA_detect}.
}%
\FR{\mysection{Une fonction vide: redux}

Revenons sur l'exemple de la fonction vide \myref{empty_func}.
Maintenant que nous connaissons le prologue et l'épilogue de fonction, ceci est
une fonction vide \myref{lst:empty_func} compilée par GCC sans optimisation:

\lstinputlisting[caption=GCC 8.2 x64 \NonOptimizing (\assemblyOutput),style=customasmx86]{patterns/016_empty_redux/1.s}

C'est \INS{RET}, mais le prologue et l'épilogue de la fonction, probablement, n'ont
pas été optimisés et laissés tels quels.
\INS{NOP} semble être un autre artefact du compilateur.
De toutes façons, la seule instruction effective ici est \INS{RET}.
Toutes les autres instructions peuvent être supprimées (ou optimisées).

}

\EN{% TODO translate
\mysection{Breaking simple executable cryptor}

I've got an executable file which is encrypted by relatively simple encryption.
\href{\GitHubBlobMasterURL/examples/simple_exec_crypto/files/cipher.bin}{Here is it} (only executable section is left here).

First, all encryption function does is just adds number of position in buffer to the byte.
Here is how this can be encoded in Python:

\begin{lstlisting}[caption=Python script,style=custompy]
#!/usr/bin/env python
def e(i, k):
    return chr ((ord(i)+k) % 256)

def encrypt(buf):
    return e(buf[0], 0)+ e(buf[1], 1)+ e(buf[2], 2) + e(buf[3], 3)+ e(buf[4], 4)+ e(buf[5], 5)+ e(buf[6], 6)+ e(buf[7], 7)+
           e(buf[8], 8)+ e(buf[9], 9)+ e(buf[10], 10)+ e(buf[11], 11)+ e(buf[12], 12)+ e(buf[13], 13)+ e(buf[14], 14)+ e(buf[15], 15)
\end{lstlisting}

Hence, if you encrypt buffer with 16 zeros, you'll get \emph{0, 1, 2, 3 ... 12, 13, 14, 15}.

\myindex{Propagating Cipher Block Chaining}
Propagating Cipher Block Chaining (PCBC) is also used, here is how it works:

\begin{figure}[H]
\centering
\myincludegraphics{examples/simple_exec_crypto/601px-PCBC_encryption.png}
\caption{Propagating Cipher Block Chaining encryption (image is taken from Wikipedia article)}
\end{figure}

The problem is that it's too boring to recover IV (Initialization Vector) each time.
Brute-force is also not an option, because IV is too long (16 bytes).
Let's see, if it's possible to recover IV for arbitrary encrypted executable file?

Let's try simple frequency analysis.
This is 32-bit x86 executable code, so let's gather statistics about most frequent bytes and opcodes.
I tried huge oracle.exe file from Oracle RDBMS version 11.2 for windows x86 and I've found that the most frequent byte (no surprise) is zero (~10\%).
The next most frequent byte is (again, no surprise) 0xFF (~5\%).
The next is 0x8B (~5\%).

\myindex{x86!\Instructions!MOV}
0x8B is opcode for \INS{MOV}, this is indeed one of the most busy x86 instructions.
Now what about popularity of zero byte?
If compiler needs to encode value bigger than 127, it has to use 32-bit displacement instead of 8-bit one, but large values are very rare,
so it is padded by zeros.
\myindex{x86!\Instructions!LEA}
\myindex{x86!\Instructions!PUSH}
\myindex{x86!\Instructions!CALL}
This is at least in \INS{LEA}, \INS{MOV}, \INS{PUSH}, \INS{CALL}.

For example:

\begin{lstlisting}[style=customasmx86]
8D B0 28 01 00 00                 lea     esi, [eax+128h]
8D BF 40 38 00 00                 lea     edi, [edi+3840h]
\end{lstlisting}

Displacements bigger than 127 are very popular, but they are rarely exceeds 0x10000
(indeed, such large memory buffers/structures are also rare).

Same story with \INS{MOV}, large constants are rare, the most heavily used are 0, 1, 10, 100, $2^n$, and so on.
Compiler has to pad small constants by zeros to represent them as 32-bit values:

\begin{lstlisting}[style=customasmx86]
BF 02 00 00 00                    mov     edi, 2
BF 01 00 00 00                    mov     edi, 1
\end{lstlisting}

Now about 00 and FF bytes combined: jumps (including conditional) and calls can pass execution flow forward or backwards, but very often,
within the limits of the current executable module.
If forward, displacement is not very big and also padded with zeros.
If backwards, displacement is represented as negative value, so padded with FF bytes.
For example, transfer execution flow forward:

\begin{lstlisting}[style=customasmx86]
E8 43 0C 00 00                    call    _function1
E8 5C 00 00 00                    call    _function2
0F 84 F0 0A 00 00                 jz      loc_4F09A0
0F 84 EB 00 00 00                 jz      loc_4EFBB8
\end{lstlisting}

Backwards:

\begin{lstlisting}[style=customasmx86]
E8 79 0C FE FF                    call    _function1
E8 F4 16 FF FF                    call    _function2
0F 84 F8 FB FF FF                 jz      loc_8212BC
0F 84 06 FD FF FF                 jz      loc_FF1E7D
\end{lstlisting}

FF byte is also very often occurred in negative displacements like these:

\begin{lstlisting}[style=customasmx86]
8D 85 1E FF FF FF                 lea     eax, [ebp-0E2h]
8D 95 F8 5C FF FF                 lea     edx, [ebp-0A308h]
\end{lstlisting}

So far so good. Now we have to try various 16-byte keys, decrypt executable section and measure how often 00, FF and 8B bytes are occurred.
Let's also keep in sight how PCBC decryption works:

\begin{figure}[H]
\centering
\myincludegraphics{examples/simple_exec_crypto/640px-PCBC_decryption.png}
\caption{Propagating Cipher Block Chaining decryption (image is taken from Wikipedia article)}
\end{figure}

The good news is that we don't really have to decrypt whole piece of data, but only slice by slice, this is exactly how I did in my previous example: \myref{XOR_mask_2}.

Now I'm trying all possible bytes (0..255) for each byte in key and just pick the byte producing maximal amount of 00/FF/8B bytes in a decrypted slice:

\begin{lstlisting}[style=custompy]
#!/usr/bin/env python
import sys, hexdump, array, string, operator

KEY_LEN=16

def chunks(l, n):
    # split n by l-byte chunks
    # https://stackoverflow.com/q/312443
    n = max(1, n)
    return [l[i:i + n] for i in range(0, len(l), n)]

def read_file(fname):
    file=open(fname, mode='rb')
    content=file.read()
    file.close()
    return content

def decrypt_byte (c, key):
    return chr((ord(c)-key) % 256)

def XOR_PCBC_step (IV, buf, k):
    prev=IV
    rt=""
    for c in buf:
	new_c=decrypt_byte(c, k)
        plain=chr(ord(new_c)^ord(prev))
	prev=chr(ord(c)^ord(plain))
	rt=rt+plain
    return rt

each_Nth_byte=[""]*KEY_LEN

content=read_file(sys.argv[1])
# split input by 16-byte chunks:
all_chunks=chunks(content, KEY_LEN)
for c in all_chunks:
    for i in range(KEY_LEN):
        each_Nth_byte[i]=each_Nth_byte[i] + c[i]

# try each byte of key
for N in range(KEY_LEN):
    print "N=", N
    stat={}
    for i in range(256):
        tmp_key=chr(i)
	tmp=XOR_PCBC_step(tmp_key,each_Nth_byte[N], N)
        # count 0, FFs and 8Bs in decrypted buffer:
	important_bytes=tmp.count('\x00')+tmp.count('\xFF')+tmp.count('\x8B')
	stat[i]=important_bytes
    sorted_stat = sorted(stat.iteritems(), key=operator.itemgetter(1), reverse=True)
    print sorted_stat[0]
\end{lstlisting}

(Source code can be downloaded \href{\GitHubBlobMasterURL/examples/simple_exec_crypto/files/decrypt.py}{here}.)

I run it and here is a key for which 00/FF/8B bytes presence in decrypted buffer is maximal:

\begin{lstlisting}
N= 0
(147, 1224)
N= 1
(94, 1327)
N= 2
(252, 1223)
N= 3
(218, 1266)
N= 4
(38, 1209)
N= 5
(192, 1378)
N= 6
(199, 1204)
N= 7
(213, 1332)
N= 8
(225, 1251)
N= 9
(112, 1223)
N= 10
(143, 1177)
N= 11
(108, 1286)
N= 12
(10, 1164)
N= 13
(3, 1271)
N= 14
(128, 1253)
N= 15
(232, 1330)
\end{lstlisting}

Let's write decryption utility with the key we got:

\begin{lstlisting}[style=custompy]
#!/usr/bin/env python
import sys, hexdump, array

def xor_strings(s,t):
    # \verb|https://en.wikipedia.org/wiki/XOR_cipher#Example_implementation|
    """xor two strings together"""
    return "".join(chr(ord(a)^ord(b)) for a,b in zip(s,t))

IV=array.array('B', [147, 94, 252, 218, 38, 192, 199, 213, 225, 112, 143, 108, 10, 3, 128, 232]).tostring()

def chunks(l, n):
    n = max(1, n)
    return [l[i:i + n] for i in range(0, len(l), n)]

def read_file(fname):
    file=open(fname, mode='rb')
    content=file.read()
    file.close()
    return content

def decrypt_byte(i, k):
    return chr ((ord(i)-k) % 256)

def decrypt(buf):
    return "".join(decrypt_byte(buf[i], i) for i in range(16))

fout=open(sys.argv[2], mode='wb')

prev=IV
content=read_file(sys.argv[1])
tmp=chunks(content, 16)
for c in tmp:
    new_c=decrypt(c)
    p=xor_strings (new_c, prev)
    prev=xor_strings(c, p)
    fout.write(p)
fout.close()
\end{lstlisting}

(Source code can be downloaded \href{\GitHubBlobMasterURL/examples/simple_exec_crypto/files/decrypt2.py}{here}.)

Let's check resulting file:

\lstinputlisting{examples/simple_exec_crypto/objdump_result.txt}

Yes, this is seems correctly disassembled piece of x86 code.
The whole decryped file can be downloaded \href{\GitHubBlobMasterURL/examples/simple_exec_crypto/files/decrypted.bin}{here}.

In fact, this is text section from regedit.exe from Windows 7.
But this example is based on a real case I encountered, so just executable is different (and key), algorithm is the same.

\subsection{Other ideas to consider}

What if I would fail with such simple frequency analysis?
There are other ideas on how to measure correctness of decrypted/decompressed x86 code:

\begin{itemize}

\item Many modern compilers aligns functions on 0x10 border.
So the space left before is filled with NOPs (0x90) or other NOP instructions with known opcodes: \myref{sec:npad}.

\item Perhaps, the most frequent pattern in any assembly language is function call:\\
\TT{PUSH chain / CALL / ADD ESP, X}.
This sequence can easily detected and found.
I've even gathered statistics about average number of function arguments: \myref{args_stat}.
(Hence, this is average length of PUSH chain.)

\end{itemize}

Read more about incorrectly/correctly disassembled code: \myref{ISA_detect}.
}%
\FR{\mysection{Une fonction vide: redux}

Revenons sur l'exemple de la fonction vide \myref{empty_func}.
Maintenant que nous connaissons le prologue et l'épilogue de fonction, ceci est
une fonction vide \myref{lst:empty_func} compilée par GCC sans optimisation:

\lstinputlisting[caption=GCC 8.2 x64 \NonOptimizing (\assemblyOutput),style=customasmx86]{patterns/016_empty_redux/1.s}

C'est \INS{RET}, mais le prologue et l'épilogue de la fonction, probablement, n'ont
pas été optimisés et laissés tels quels.
\INS{NOP} semble être un autre artefact du compilateur.
De toutes façons, la seule instruction effective ici est \INS{RET}.
Toutes les autres instructions peuvent être supprimées (ou optimisées).

}

\EN{% TODO translate
\mysection{Breaking simple executable cryptor}

I've got an executable file which is encrypted by relatively simple encryption.
\href{\GitHubBlobMasterURL/examples/simple_exec_crypto/files/cipher.bin}{Here is it} (only executable section is left here).

First, all encryption function does is just adds number of position in buffer to the byte.
Here is how this can be encoded in Python:

\begin{lstlisting}[caption=Python script,style=custompy]
#!/usr/bin/env python
def e(i, k):
    return chr ((ord(i)+k) % 256)

def encrypt(buf):
    return e(buf[0], 0)+ e(buf[1], 1)+ e(buf[2], 2) + e(buf[3], 3)+ e(buf[4], 4)+ e(buf[5], 5)+ e(buf[6], 6)+ e(buf[7], 7)+
           e(buf[8], 8)+ e(buf[9], 9)+ e(buf[10], 10)+ e(buf[11], 11)+ e(buf[12], 12)+ e(buf[13], 13)+ e(buf[14], 14)+ e(buf[15], 15)
\end{lstlisting}

Hence, if you encrypt buffer with 16 zeros, you'll get \emph{0, 1, 2, 3 ... 12, 13, 14, 15}.

\myindex{Propagating Cipher Block Chaining}
Propagating Cipher Block Chaining (PCBC) is also used, here is how it works:

\begin{figure}[H]
\centering
\myincludegraphics{examples/simple_exec_crypto/601px-PCBC_encryption.png}
\caption{Propagating Cipher Block Chaining encryption (image is taken from Wikipedia article)}
\end{figure}

The problem is that it's too boring to recover IV (Initialization Vector) each time.
Brute-force is also not an option, because IV is too long (16 bytes).
Let's see, if it's possible to recover IV for arbitrary encrypted executable file?

Let's try simple frequency analysis.
This is 32-bit x86 executable code, so let's gather statistics about most frequent bytes and opcodes.
I tried huge oracle.exe file from Oracle RDBMS version 11.2 for windows x86 and I've found that the most frequent byte (no surprise) is zero (~10\%).
The next most frequent byte is (again, no surprise) 0xFF (~5\%).
The next is 0x8B (~5\%).

\myindex{x86!\Instructions!MOV}
0x8B is opcode for \INS{MOV}, this is indeed one of the most busy x86 instructions.
Now what about popularity of zero byte?
If compiler needs to encode value bigger than 127, it has to use 32-bit displacement instead of 8-bit one, but large values are very rare,
so it is padded by zeros.
\myindex{x86!\Instructions!LEA}
\myindex{x86!\Instructions!PUSH}
\myindex{x86!\Instructions!CALL}
This is at least in \INS{LEA}, \INS{MOV}, \INS{PUSH}, \INS{CALL}.

For example:

\begin{lstlisting}[style=customasmx86]
8D B0 28 01 00 00                 lea     esi, [eax+128h]
8D BF 40 38 00 00                 lea     edi, [edi+3840h]
\end{lstlisting}

Displacements bigger than 127 are very popular, but they are rarely exceeds 0x10000
(indeed, such large memory buffers/structures are also rare).

Same story with \INS{MOV}, large constants are rare, the most heavily used are 0, 1, 10, 100, $2^n$, and so on.
Compiler has to pad small constants by zeros to represent them as 32-bit values:

\begin{lstlisting}[style=customasmx86]
BF 02 00 00 00                    mov     edi, 2
BF 01 00 00 00                    mov     edi, 1
\end{lstlisting}

Now about 00 and FF bytes combined: jumps (including conditional) and calls can pass execution flow forward or backwards, but very often,
within the limits of the current executable module.
If forward, displacement is not very big and also padded with zeros.
If backwards, displacement is represented as negative value, so padded with FF bytes.
For example, transfer execution flow forward:

\begin{lstlisting}[style=customasmx86]
E8 43 0C 00 00                    call    _function1
E8 5C 00 00 00                    call    _function2
0F 84 F0 0A 00 00                 jz      loc_4F09A0
0F 84 EB 00 00 00                 jz      loc_4EFBB8
\end{lstlisting}

Backwards:

\begin{lstlisting}[style=customasmx86]
E8 79 0C FE FF                    call    _function1
E8 F4 16 FF FF                    call    _function2
0F 84 F8 FB FF FF                 jz      loc_8212BC
0F 84 06 FD FF FF                 jz      loc_FF1E7D
\end{lstlisting}

FF byte is also very often occurred in negative displacements like these:

\begin{lstlisting}[style=customasmx86]
8D 85 1E FF FF FF                 lea     eax, [ebp-0E2h]
8D 95 F8 5C FF FF                 lea     edx, [ebp-0A308h]
\end{lstlisting}

So far so good. Now we have to try various 16-byte keys, decrypt executable section and measure how often 00, FF and 8B bytes are occurred.
Let's also keep in sight how PCBC decryption works:

\begin{figure}[H]
\centering
\myincludegraphics{examples/simple_exec_crypto/640px-PCBC_decryption.png}
\caption{Propagating Cipher Block Chaining decryption (image is taken from Wikipedia article)}
\end{figure}

The good news is that we don't really have to decrypt whole piece of data, but only slice by slice, this is exactly how I did in my previous example: \myref{XOR_mask_2}.

Now I'm trying all possible bytes (0..255) for each byte in key and just pick the byte producing maximal amount of 00/FF/8B bytes in a decrypted slice:

\begin{lstlisting}[style=custompy]
#!/usr/bin/env python
import sys, hexdump, array, string, operator

KEY_LEN=16

def chunks(l, n):
    # split n by l-byte chunks
    # https://stackoverflow.com/q/312443
    n = max(1, n)
    return [l[i:i + n] for i in range(0, len(l), n)]

def read_file(fname):
    file=open(fname, mode='rb')
    content=file.read()
    file.close()
    return content

def decrypt_byte (c, key):
    return chr((ord(c)-key) % 256)

def XOR_PCBC_step (IV, buf, k):
    prev=IV
    rt=""
    for c in buf:
	new_c=decrypt_byte(c, k)
        plain=chr(ord(new_c)^ord(prev))
	prev=chr(ord(c)^ord(plain))
	rt=rt+plain
    return rt

each_Nth_byte=[""]*KEY_LEN

content=read_file(sys.argv[1])
# split input by 16-byte chunks:
all_chunks=chunks(content, KEY_LEN)
for c in all_chunks:
    for i in range(KEY_LEN):
        each_Nth_byte[i]=each_Nth_byte[i] + c[i]

# try each byte of key
for N in range(KEY_LEN):
    print "N=", N
    stat={}
    for i in range(256):
        tmp_key=chr(i)
	tmp=XOR_PCBC_step(tmp_key,each_Nth_byte[N], N)
        # count 0, FFs and 8Bs in decrypted buffer:
	important_bytes=tmp.count('\x00')+tmp.count('\xFF')+tmp.count('\x8B')
	stat[i]=important_bytes
    sorted_stat = sorted(stat.iteritems(), key=operator.itemgetter(1), reverse=True)
    print sorted_stat[0]
\end{lstlisting}

(Source code can be downloaded \href{\GitHubBlobMasterURL/examples/simple_exec_crypto/files/decrypt.py}{here}.)

I run it and here is a key for which 00/FF/8B bytes presence in decrypted buffer is maximal:

\begin{lstlisting}
N= 0
(147, 1224)
N= 1
(94, 1327)
N= 2
(252, 1223)
N= 3
(218, 1266)
N= 4
(38, 1209)
N= 5
(192, 1378)
N= 6
(199, 1204)
N= 7
(213, 1332)
N= 8
(225, 1251)
N= 9
(112, 1223)
N= 10
(143, 1177)
N= 11
(108, 1286)
N= 12
(10, 1164)
N= 13
(3, 1271)
N= 14
(128, 1253)
N= 15
(232, 1330)
\end{lstlisting}

Let's write decryption utility with the key we got:

\begin{lstlisting}[style=custompy]
#!/usr/bin/env python
import sys, hexdump, array

def xor_strings(s,t):
    # \verb|https://en.wikipedia.org/wiki/XOR_cipher#Example_implementation|
    """xor two strings together"""
    return "".join(chr(ord(a)^ord(b)) for a,b in zip(s,t))

IV=array.array('B', [147, 94, 252, 218, 38, 192, 199, 213, 225, 112, 143, 108, 10, 3, 128, 232]).tostring()

def chunks(l, n):
    n = max(1, n)
    return [l[i:i + n] for i in range(0, len(l), n)]

def read_file(fname):
    file=open(fname, mode='rb')
    content=file.read()
    file.close()
    return content

def decrypt_byte(i, k):
    return chr ((ord(i)-k) % 256)

def decrypt(buf):
    return "".join(decrypt_byte(buf[i], i) for i in range(16))

fout=open(sys.argv[2], mode='wb')

prev=IV
content=read_file(sys.argv[1])
tmp=chunks(content, 16)
for c in tmp:
    new_c=decrypt(c)
    p=xor_strings (new_c, prev)
    prev=xor_strings(c, p)
    fout.write(p)
fout.close()
\end{lstlisting}

(Source code can be downloaded \href{\GitHubBlobMasterURL/examples/simple_exec_crypto/files/decrypt2.py}{here}.)

Let's check resulting file:

\lstinputlisting{examples/simple_exec_crypto/objdump_result.txt}

Yes, this is seems correctly disassembled piece of x86 code.
The whole decryped file can be downloaded \href{\GitHubBlobMasterURL/examples/simple_exec_crypto/files/decrypted.bin}{here}.

In fact, this is text section from regedit.exe from Windows 7.
But this example is based on a real case I encountered, so just executable is different (and key), algorithm is the same.

\subsection{Other ideas to consider}

What if I would fail with such simple frequency analysis?
There are other ideas on how to measure correctness of decrypted/decompressed x86 code:

\begin{itemize}

\item Many modern compilers aligns functions on 0x10 border.
So the space left before is filled with NOPs (0x90) or other NOP instructions with known opcodes: \myref{sec:npad}.

\item Perhaps, the most frequent pattern in any assembly language is function call:\\
\TT{PUSH chain / CALL / ADD ESP, X}.
This sequence can easily detected and found.
I've even gathered statistics about average number of function arguments: \myref{args_stat}.
(Hence, this is average length of PUSH chain.)

\end{itemize}

Read more about incorrectly/correctly disassembled code: \myref{ISA_detect}.
}%
\FR{\mysection{Une fonction vide: redux}

Revenons sur l'exemple de la fonction vide \myref{empty_func}.
Maintenant que nous connaissons le prologue et l'épilogue de fonction, ceci est
une fonction vide \myref{lst:empty_func} compilée par GCC sans optimisation:

\lstinputlisting[caption=GCC 8.2 x64 \NonOptimizing (\assemblyOutput),style=customasmx86]{patterns/016_empty_redux/1.s}

C'est \INS{RET}, mais le prologue et l'épilogue de la fonction, probablement, n'ont
pas été optimisés et laissés tels quels.
\INS{NOP} semble être un autre artefact du compilateur.
De toutes façons, la seule instruction effective ici est \INS{RET}.
Toutes les autres instructions peuvent être supprimées (ou optimisées).

}

\EN{% TODO translate
\mysection{Breaking simple executable cryptor}

I've got an executable file which is encrypted by relatively simple encryption.
\href{\GitHubBlobMasterURL/examples/simple_exec_crypto/files/cipher.bin}{Here is it} (only executable section is left here).

First, all encryption function does is just adds number of position in buffer to the byte.
Here is how this can be encoded in Python:

\begin{lstlisting}[caption=Python script,style=custompy]
#!/usr/bin/env python
def e(i, k):
    return chr ((ord(i)+k) % 256)

def encrypt(buf):
    return e(buf[0], 0)+ e(buf[1], 1)+ e(buf[2], 2) + e(buf[3], 3)+ e(buf[4], 4)+ e(buf[5], 5)+ e(buf[6], 6)+ e(buf[7], 7)+
           e(buf[8], 8)+ e(buf[9], 9)+ e(buf[10], 10)+ e(buf[11], 11)+ e(buf[12], 12)+ e(buf[13], 13)+ e(buf[14], 14)+ e(buf[15], 15)
\end{lstlisting}

Hence, if you encrypt buffer with 16 zeros, you'll get \emph{0, 1, 2, 3 ... 12, 13, 14, 15}.

\myindex{Propagating Cipher Block Chaining}
Propagating Cipher Block Chaining (PCBC) is also used, here is how it works:

\begin{figure}[H]
\centering
\myincludegraphics{examples/simple_exec_crypto/601px-PCBC_encryption.png}
\caption{Propagating Cipher Block Chaining encryption (image is taken from Wikipedia article)}
\end{figure}

The problem is that it's too boring to recover IV (Initialization Vector) each time.
Brute-force is also not an option, because IV is too long (16 bytes).
Let's see, if it's possible to recover IV for arbitrary encrypted executable file?

Let's try simple frequency analysis.
This is 32-bit x86 executable code, so let's gather statistics about most frequent bytes and opcodes.
I tried huge oracle.exe file from Oracle RDBMS version 11.2 for windows x86 and I've found that the most frequent byte (no surprise) is zero (~10\%).
The next most frequent byte is (again, no surprise) 0xFF (~5\%).
The next is 0x8B (~5\%).

\myindex{x86!\Instructions!MOV}
0x8B is opcode for \INS{MOV}, this is indeed one of the most busy x86 instructions.
Now what about popularity of zero byte?
If compiler needs to encode value bigger than 127, it has to use 32-bit displacement instead of 8-bit one, but large values are very rare,
so it is padded by zeros.
\myindex{x86!\Instructions!LEA}
\myindex{x86!\Instructions!PUSH}
\myindex{x86!\Instructions!CALL}
This is at least in \INS{LEA}, \INS{MOV}, \INS{PUSH}, \INS{CALL}.

For example:

\begin{lstlisting}[style=customasmx86]
8D B0 28 01 00 00                 lea     esi, [eax+128h]
8D BF 40 38 00 00                 lea     edi, [edi+3840h]
\end{lstlisting}

Displacements bigger than 127 are very popular, but they are rarely exceeds 0x10000
(indeed, such large memory buffers/structures are also rare).

Same story with \INS{MOV}, large constants are rare, the most heavily used are 0, 1, 10, 100, $2^n$, and so on.
Compiler has to pad small constants by zeros to represent them as 32-bit values:

\begin{lstlisting}[style=customasmx86]
BF 02 00 00 00                    mov     edi, 2
BF 01 00 00 00                    mov     edi, 1
\end{lstlisting}

Now about 00 and FF bytes combined: jumps (including conditional) and calls can pass execution flow forward or backwards, but very often,
within the limits of the current executable module.
If forward, displacement is not very big and also padded with zeros.
If backwards, displacement is represented as negative value, so padded with FF bytes.
For example, transfer execution flow forward:

\begin{lstlisting}[style=customasmx86]
E8 43 0C 00 00                    call    _function1
E8 5C 00 00 00                    call    _function2
0F 84 F0 0A 00 00                 jz      loc_4F09A0
0F 84 EB 00 00 00                 jz      loc_4EFBB8
\end{lstlisting}

Backwards:

\begin{lstlisting}[style=customasmx86]
E8 79 0C FE FF                    call    _function1
E8 F4 16 FF FF                    call    _function2
0F 84 F8 FB FF FF                 jz      loc_8212BC
0F 84 06 FD FF FF                 jz      loc_FF1E7D
\end{lstlisting}

FF byte is also very often occurred in negative displacements like these:

\begin{lstlisting}[style=customasmx86]
8D 85 1E FF FF FF                 lea     eax, [ebp-0E2h]
8D 95 F8 5C FF FF                 lea     edx, [ebp-0A308h]
\end{lstlisting}

So far so good. Now we have to try various 16-byte keys, decrypt executable section and measure how often 00, FF and 8B bytes are occurred.
Let's also keep in sight how PCBC decryption works:

\begin{figure}[H]
\centering
\myincludegraphics{examples/simple_exec_crypto/640px-PCBC_decryption.png}
\caption{Propagating Cipher Block Chaining decryption (image is taken from Wikipedia article)}
\end{figure}

The good news is that we don't really have to decrypt whole piece of data, but only slice by slice, this is exactly how I did in my previous example: \myref{XOR_mask_2}.

Now I'm trying all possible bytes (0..255) for each byte in key and just pick the byte producing maximal amount of 00/FF/8B bytes in a decrypted slice:

\begin{lstlisting}[style=custompy]
#!/usr/bin/env python
import sys, hexdump, array, string, operator

KEY_LEN=16

def chunks(l, n):
    # split n by l-byte chunks
    # https://stackoverflow.com/q/312443
    n = max(1, n)
    return [l[i:i + n] for i in range(0, len(l), n)]

def read_file(fname):
    file=open(fname, mode='rb')
    content=file.read()
    file.close()
    return content

def decrypt_byte (c, key):
    return chr((ord(c)-key) % 256)

def XOR_PCBC_step (IV, buf, k):
    prev=IV
    rt=""
    for c in buf:
	new_c=decrypt_byte(c, k)
        plain=chr(ord(new_c)^ord(prev))
	prev=chr(ord(c)^ord(plain))
	rt=rt+plain
    return rt

each_Nth_byte=[""]*KEY_LEN

content=read_file(sys.argv[1])
# split input by 16-byte chunks:
all_chunks=chunks(content, KEY_LEN)
for c in all_chunks:
    for i in range(KEY_LEN):
        each_Nth_byte[i]=each_Nth_byte[i] + c[i]

# try each byte of key
for N in range(KEY_LEN):
    print "N=", N
    stat={}
    for i in range(256):
        tmp_key=chr(i)
	tmp=XOR_PCBC_step(tmp_key,each_Nth_byte[N], N)
        # count 0, FFs and 8Bs in decrypted buffer:
	important_bytes=tmp.count('\x00')+tmp.count('\xFF')+tmp.count('\x8B')
	stat[i]=important_bytes
    sorted_stat = sorted(stat.iteritems(), key=operator.itemgetter(1), reverse=True)
    print sorted_stat[0]
\end{lstlisting}

(Source code can be downloaded \href{\GitHubBlobMasterURL/examples/simple_exec_crypto/files/decrypt.py}{here}.)

I run it and here is a key for which 00/FF/8B bytes presence in decrypted buffer is maximal:

\begin{lstlisting}
N= 0
(147, 1224)
N= 1
(94, 1327)
N= 2
(252, 1223)
N= 3
(218, 1266)
N= 4
(38, 1209)
N= 5
(192, 1378)
N= 6
(199, 1204)
N= 7
(213, 1332)
N= 8
(225, 1251)
N= 9
(112, 1223)
N= 10
(143, 1177)
N= 11
(108, 1286)
N= 12
(10, 1164)
N= 13
(3, 1271)
N= 14
(128, 1253)
N= 15
(232, 1330)
\end{lstlisting}

Let's write decryption utility with the key we got:

\begin{lstlisting}[style=custompy]
#!/usr/bin/env python
import sys, hexdump, array

def xor_strings(s,t):
    # \verb|https://en.wikipedia.org/wiki/XOR_cipher#Example_implementation|
    """xor two strings together"""
    return "".join(chr(ord(a)^ord(b)) for a,b in zip(s,t))

IV=array.array('B', [147, 94, 252, 218, 38, 192, 199, 213, 225, 112, 143, 108, 10, 3, 128, 232]).tostring()

def chunks(l, n):
    n = max(1, n)
    return [l[i:i + n] for i in range(0, len(l), n)]

def read_file(fname):
    file=open(fname, mode='rb')
    content=file.read()
    file.close()
    return content

def decrypt_byte(i, k):
    return chr ((ord(i)-k) % 256)

def decrypt(buf):
    return "".join(decrypt_byte(buf[i], i) for i in range(16))

fout=open(sys.argv[2], mode='wb')

prev=IV
content=read_file(sys.argv[1])
tmp=chunks(content, 16)
for c in tmp:
    new_c=decrypt(c)
    p=xor_strings (new_c, prev)
    prev=xor_strings(c, p)
    fout.write(p)
fout.close()
\end{lstlisting}

(Source code can be downloaded \href{\GitHubBlobMasterURL/examples/simple_exec_crypto/files/decrypt2.py}{here}.)

Let's check resulting file:

\lstinputlisting{examples/simple_exec_crypto/objdump_result.txt}

Yes, this is seems correctly disassembled piece of x86 code.
The whole decryped file can be downloaded \href{\GitHubBlobMasterURL/examples/simple_exec_crypto/files/decrypted.bin}{here}.

In fact, this is text section from regedit.exe from Windows 7.
But this example is based on a real case I encountered, so just executable is different (and key), algorithm is the same.

\subsection{Other ideas to consider}

What if I would fail with such simple frequency analysis?
There are other ideas on how to measure correctness of decrypted/decompressed x86 code:

\begin{itemize}

\item Many modern compilers aligns functions on 0x10 border.
So the space left before is filled with NOPs (0x90) or other NOP instructions with known opcodes: \myref{sec:npad}.

\item Perhaps, the most frequent pattern in any assembly language is function call:\\
\TT{PUSH chain / CALL / ADD ESP, X}.
This sequence can easily detected and found.
I've even gathered statistics about average number of function arguments: \myref{args_stat}.
(Hence, this is average length of PUSH chain.)

\end{itemize}

Read more about incorrectly/correctly disassembled code: \myref{ISA_detect}.
}
\RU{\subsection{Простое шифрование используя XOR-маску}
\label{XOR_mask_1}

Я нашел одну старую игру в стиле interactive fiction в архиве \emph{if-archive}\footnote{\url{http://www.ifarchive.org/}}:

\begin{lstlisting}
The New Castle v3.5 - Text/Adventure Game
in the style of the original Infocom (tm)
type games, Zork, Collosal Cave (Adventure),
etc.  Can you solve the mystery of the
abandoned castle?
Shareware from Software Customization.
Software Customization [ASP] Version 3.5 Feb. 2000
\end{lstlisting}

Можно скачать здесь: \url{\GitHubBlobMasterURL/ff/XOR/mask_1/files/newcastle.tgz}.

Там внутри есть файл (с названием \emph{castle.dbf}), который явно зашифрован, но не настоящим криптоалгоритмом,
и оне сжат, это что-то куда проще.
Я бы даже не стал измерять уровень энтропии (\myref{entropy}) этого файла, потому что я итак уверен, что он низкий.
Вот как он выглядит в Midnight Commander:

\begin{figure}[H]
\centering
\myincludegraphics{ff/XOR/mask_1/mc_encrypted.png}
\caption{Зашифрованный файл в Midnight Commander}
\end{figure}

Зашифрованный файл можно скачать здесь:
\url{\GitHubBlobMasterURL/ff/XOR/mask_1/files/castle.dbf.bz2}.

Можно ли расшифровать его без доступа к программе, используя просто этот файл?

Тут явно просматривается повторяющаяся строка. 
Если использовалось простое шифрование с XOR-маской, такие повторяющиеся строки это явное свидетельство,
потому что, вероятно, тут были длинные лакуны с нулевыми байтами, которые, в свою очередь, присутствуют
во мноигих исполняемых файлах, и в остальных бинарных файлах.

\myindex{UNIX!xxd}
Вот дам начала этого файла используя утилиту \emph{xxd} из UNIX:

\lstinputlisting{ff/XOR/mask_1/xxd_result.txt}

Давайте держаться за повторяющуюся строку \TT{iubgv}.
Глядя на этот дамп, мы можем легко увидеть, что период повторений этой строки это 0x51 или 81.
Вероятно, 81 это длина блока?
Длина файла 1658961, и она может быть поделена на 81 без остатка (и тогда там 20481 блоков).

Теперь я буду использовать Mathematica для анализа, есть ли тут повторяющиеся 81-байтные блоки в файле?
Я разделю входной файл на 81-байтные блоки и затем использую ф-цию
\emph{Tally[]}\footnote{\url{https://reference.wolfram.com/language/ref/Tally.html}}
которая просто считает, сколько раз каждый элемент встретился во входном списке.
Вывод Tally не отсортирован, так что я также добавлю ф-цию \emph{Sort[]} для сортировки его по кол-ву вхождений
в нисходящем порядке.

\begin{lstlisting}[style=custommath]
input = BinaryReadList["/home/dennis/.../castle.dbf"];

blocks = Partition[input, 81];

stat = Sort[Tally[blocks], #1[[2]] > #2[[2]] &]
\end{lstlisting}

И вот вывод:

\begin{lstlisting}[style=custommath]
{{{80, 103, 2, 116, 113, 102, 118, 25, 99, 8, 19, 23, 116, 125, 107, 
   25, 99, 109, 114, 102, 14, 121, 115, 31, 9, 117, 113, 111, 5, 4, 
   127, 28, 122, 101, 8, 110, 14, 18, 124, 106, 16, 20, 104, 119, 8, 
   109, 26, 106, 9, 97, 13, 99, 15, 119, 20, 105, 117, 98, 103, 118, 
   1, 126, 29, 97, 122, 17, 15, 114, 110, 3, 5, 125, 125, 99, 126, 
   119, 102, 30, 122, 2, 117}, 1739}, 
{{80, 100, 2, 116, 113, 102, 118, 25, 99, 8, 19, 23, 116, 
   125, 107, 25, 99, 109, 114, 102, 14, 121, 115, 31, 9, 117, 113, 
   111, 5, 4, 127, 28, 122, 101, 8, 110, 14, 18, 124, 106, 16, 20, 
   104, 119, 8, 109, 26, 106, 9, 97, 13, 99, 15, 119, 20, 105, 117, 
   98, 103, 118, 1, 126, 29, 97, 122, 17, 15, 114, 110, 3, 5, 125, 
   125, 99, 126, 119, 102, 30, 122, 2, 117}, 1422}, 
{{80, 101, 2, 116, 113, 102, 118, 25, 99, 8, 19, 23, 116, 
   125, 107, 25, 99, 109, 114, 102, 14, 121, 115, 31, 9, 117, 113, 
   111, 5, 4, 127, 28, 122, 101, 8, 110, 14, 18, 124, 106, 16, 20, 
   104, 119, 8, 109, 26, 106, 9, 97, 13, 99, 15, 119, 20, 105, 117, 
   98, 103, 118, 1, 126, 29, 97, 122, 17, 15, 114, 110, 3, 5, 125, 
   125, 99, 126, 119, 102, 30, 122, 2, 117}, 1012},
{{80, 120, 2, 116, 113, 102, 118, 25, 99, 8, 19, 23, 116, 
   125, 107, 25, 99, 109, 114, 102, 14, 121, 115, 31, 9, 117, 113, 
   111, 5, 4, 127, 28, 122, 101, 8, 110, 14, 18, 124, 106, 16, 20, 
   104, 119, 8, 109, 26, 106, 9, 97, 13, 99, 15, 119, 20, 105, 117, 
   98, 103, 118, 1, 126, 29, 97, 122, 17, 15, 114, 110, 3, 5, 125, 
   125, 99, 126, 119, 102, 30, 122, 2, 117}, 377},

...

{{80, 2, 74, 49, 113, 21, 62, 88, 39, 71, 68, 23, 63, 51, 36, 78, 48, 
   108, 114, 102, 14, 121, 115, 31, 9, 117, 113, 111, 5, 4, 127, 28, 
   122, 101, 8, 110, 14, 18, 124, 106, 16, 20, 104, 119, 8, 109, 26, 
   106, 9, 97, 13, 99, 15, 119, 20, 105, 117, 98, 103, 118, 1, 126, 
   29, 97, 122, 17, 15, 114, 110, 3, 5, 125, 125, 99, 126, 119, 102, 
   30, 122, 2, 117}, 1},
{{80, 1, 74, 59, 113, 45, 56, 86, 52, 91, 19, 64, 60, 60, 63, 
   25, 38, 59, 59, 42, 14, 53, 38, 77, 66, 38, 113, 38, 75, 4, 43, 84,
    63, 101, 64, 43, 79, 64, 40, 57, 16, 91, 46, 119, 69, 40, 84, 117,
    9, 97, 13, 99, 15, 119, 20, 105, 117, 98, 103, 118, 1, 126, 29, 
   97, 122, 17, 15, 114, 110, 3, 5, 125, 125, 99, 126, 119, 102, 30, 
   122, 2, 117}, 1},
{{80, 2, 74, 49, 113, 49, 51, 92, 39, 8, 92, 81, 116, 62, 57, 
   80, 46, 40, 114, 36, 75, 56, 33, 76, 9, 55, 56, 59, 81, 65, 45, 28,
    60, 55, 93, 39, 90, 28, 124, 106, 16, 20, 104, 119, 8, 109, 26, 
   106, 9, 97, 13, 99, 15, 119, 20, 105, 117, 98, 103, 118, 1, 126, 
   29, 97, 122, 17, 15, 114, 110, 3, 5, 125, 125, 99, 126, 119, 102, 
   30, 122, 2, 117}, 1}}
\end{lstlisting}

Вывод Tally это список пар, каждая пара это 81-байтный блок и количество раз, сколько он встретился в файле.
Мы видим, что наиболее частно встречающийся блок это первый, он встретился 1739 раз.
Второй встретился 1422 раза. Есть и другие: 1012 раза, 377 раз, итд.
81-байтные блоки, встреченные лишь один раз, находятся в конце вывода.

Попробуем сравнить эти блоки. Первый и второй.
Есть ли в Mathematica ф-ция для сравнения списков/массивов?
Наверняка есть, но в педагогических целях, я буду использоват операцию XOR для сравнения.
Действительно: если байты во входных массивах равны друг другу, результат операции XOR это 0.
Если не равны, результат будет ненулевой.

Сравним первый блок (встречается 1739 раз) и второй (встречается 1422 раз):

\begin{lstlisting}[style=custommath]
In[]:= BitXor[stat[[1]][[1]], stat[[2]][[1]]]
Out[]= {0, 3, 0, 0, 0, 0, 0, 0, 0, 0, 0, 0, 0, 0, 0, 0, 0, 0, 0, \
0, 0, 0, 0, 0, 0, 0, 0, 0, 0, 0, 0, 0, 0, 0, 0, 0, 0, 0, 0, 0, 0, 0, \
0, 0, 0, 0, 0, 0, 0, 0, 0, 0, 0, 0, 0, 0, 0, 0, 0, 0, 0, 0, 0, 0, 0, \
0, 0, 0, 0, 0, 0, 0, 0, 0, 0, 0, 0, 0, 0, 0, 0}
\end{lstlisting}

Они отличаются только вторым байтом.

Сравним второй блок (встречается 1422 раза) и третий (встречается 1012 раз):

\begin{lstlisting}[style=custommath]
In[]:= BitXor[stat[[2]][[1]], stat[[3]][[1]]]
Out[]= {0, 1, 0, 0, 0, 0, 0, 0, 0, 0, 0, 0, 0, 0, 0, 0, 0, 0, 0, \
0, 0, 0, 0, 0, 0, 0, 0, 0, 0, 0, 0, 0, 0, 0, 0, 0, 0, 0, 0, 0, 0, 0, \
0, 0, 0, 0, 0, 0, 0, 0, 0, 0, 0, 0, 0, 0, 0, 0, 0, 0, 0, 0, 0, 0, 0, \
0, 0, 0, 0, 0, 0, 0, 0, 0, 0, 0, 0, 0, 0, 0, 0}
\end{lstlisting}

Они тоже отличаются только вторым байтом.

Так или иначе, попробуем использовать самый встречающийся блок как XOR-ключ и попробуем расшифровать первые 4 81-байтных
блока в файле:

\begin{lstlisting}[style=custommath]
In[]:= key = stat[[1]][[1]]
Out[]= {80, 103, 2, 116, 113, 102, 118, 25, 99, 8, 19, 23, 116, \
125, 107, 25, 99, 109, 114, 102, 14, 121, 115, 31, 9, 117, 113, 111, \
5, 4, 127, 28, 122, 101, 8, 110, 14, 18, 124, 106, 16, 20, 104, 119, \
8, 109, 26, 106, 9, 97, 13, 99, 15, 119, 20, 105, 117, 98, 103, 118, \
1, 126, 29, 97, 122, 17, 15, 114, 110, 3, 5, 125, 125, 99, 126, 119, \
102, 30, 122, 2, 117}

In[]:= ToASCII[val_] := If[val == 0, " ", FromCharacterCode[val, "PrintableASCII"]]

In[]:= DecryptBlockASCII[blk_] := Map[ToASCII[#] &, BitXor[key, blk]]

In[]:= DecryptBlockASCII[blocks[[1]]]
Out[]= {" ", " ", " ", " ", " ", " ", " ", " ", " ", " ", " ", " \
", " ", " ", " ", " ", " ", " ", " ", " ", " ", " ", " ", " ", " ", " \
", " ", " ", " ", " ", " ", " ", " ", " ", " ", " ", " ", " ", " ", " \
", " ", " ", " ", " ", " ", " ", " ", " ", " ", " ", " ", " ", " ", " \
", " ", " ", " ", " ", " ", " ", " ", " ", " ", " ", " ", " ", " ", " \
", " ", " ", " ", " ", " ", " ", " ", " ", " ", " ", " ", " ", " "}

In[]:= DecryptBlockASCII[blocks[[2]]]
Out[]= {" ", "e", "H", "E", " ", "W", "E", "E", "D", " ", "O", \
"F", " ", "C", "R", "I", "M", "E", " ", "B", "E", "A", "R", "S", " ", \
"B", "I", "T", "T", "E", "R", " ", "F", "R", "U", "I", "T", "?", \
" ", " ", " ", " ", " ", " ", " ", " ", " ", " ", " ", " ", " ", " ", \
" ", " ", " ", " ", " ", " ", " ", " ", " ", " ", " ", " ", " ", " ", \
" ", " ", " ", " ", " ", " ", " ", " ", " ", " ", " ", " ", " ", " ", \
" "}

In[]:= DecryptBlockASCII[blocks[[3]]]
Out[]= {" ", "?", " ", " ", " ", " ", " ", " ", " ", " ", " \
", " ", " ", " ", " ", " ", " ", " ", " ", " ", " ", " ", " ", " ", " \
", " ", " ", " ", " ", " ", " ", " ", " ", " ", " ", " ", " ", " ", " \
", " ", " ", " ", " ", " ", " ", " ", " ", " ", " ", " ", " ", " ", " \
", " ", " ", " ", " ", " ", " ", " ", " ", " ", " ", " ", " ", " ", " \
", " ", " ", " ", " ", " ", " ", " ", " ", " ", " ", " ", " ", " ", " \
"}

In[]:= DecryptBlockASCII[blocks[[4]]]
Out[]= {" ", "f", "H", "O", " ", "K", "N", "O", "W", "S", " ", \
"W", "H", "A", "T", " ", "E", "V", "I", "L", " ", "L", "U", "R", "K", \
"S", " ", "I", "N", " ", "T", "H", "E", " ", "H", "E", "A", "R", "T", \
"S", " ", "O", "F", " ", "M", "E", "N", "?", " ", " ", " ", " ", \
" ", " ", " ", " ", " ", " ", " ", " ", " ", " ", " ", " ", " ", " ", \
" ", " ", " ", " ", " ", " ", " ", " ", " ", " ", " ", " ", " ", " ", \
" "}
\end{lstlisting}

(Я заменил непечатаемые символы на \q{?}.)

Мы видим что первый и третий блоки пустые (или почти пустые),
но второй и четвертый имеют ясно различимые английские слова/фразы.
Похоже что наше предположение насчет ключа верно (как минимум частично).
Это означает, что самый встречающийся 81-байтный блок в файле находится в местах лакун с нулевыми байтами
или что-то в этом роде.

Попробуем расшифровать весь файл:

\begin{lstlisting}[style=custommath]
DecryptBlock[blk_] := BitXor[key, blk]

decrypted = Map[DecryptBlock[#] &, blocks];

BinaryWrite["/home/dennis/.../tmp", Flatten[decrypted]]

Close["/home/dennis/.../tmp"]
\end{lstlisting}

\begin{figure}[H]
\centering
\myincludegraphics{ff/XOR/mask_1/mc_decrypted1.png}
\caption{Расшифрованный файл в Midnight Commander, первая попытка}
\end{figure}

Выглядит как английские фразы для какой-то игры, но что-то не так.
Прежде всего, регистр инвертирован: фразы и некоторые слова начинаются со строчных букв,
в то время как остальные буквы заглавные.
Также, некоторые фразы начинаются с не тех букв.
Посмотрите на самую первую фразу: \q{eHE WEED OF CRIME BEARS BITTER FRUIT}.
Что такое \q{eHE}? Разве не \q{tHE} тут должно быть?
Возможно ли что наш ключ для дешифрования имеет неверный байт в этом месте?

Посмотрим снова на второй блок в файле, на ключ и на результат дешифрования:

\begin{lstlisting}[style=custommath]
In[]:= blocks[[2]]
Out[]= {80, 2, 74, 49, 113, 49, 51, 92, 39, 8, 92, 81, 116, 62, \
57, 80, 46, 40, 114, 36, 75, 56, 33, 76, 9, 55, 56, 59, 81, 65, 45, \
28, 60, 55, 93, 39, 90, 28, 124, 106, 16, 20, 104, 119, 8, 109, 26, \
106, 9, 97, 13, 99, 15, 119, 20, 105, 117, 98, 103, 118, 1, 126, 29, \
97, 122, 17, 15, 114, 110, 3, 5, 125, 125, 99, 126, 119, 102, 30, \
122, 2, 117}

In[]:= key
Out[]= {80, 103, 2, 116, 113, 102, 118, 25, 99, 8, 19, 23, 116, \
125, 107, 25, 99, 109, 114, 102, 14, 121, 115, 31, 9, 117, 113, 111, \
5, 4, 127, 28, 122, 101, 8, 110, 14, 18, 124, 106, 16, 20, 104, 119, \
8, 109, 26, 106, 9, 97, 13, 99, 15, 119, 20, 105, 117, 98, 103, 118, \
1, 126, 29, 97, 122, 17, 15, 114, 110, 3, 5, 125, 125, 99, 126, 119, \
102, 30, 122, 2, 117}

In[]:= BitXor[key, blocks[[2]]]
Out[]= {0, 101, 72, 69, 0, 87, 69, 69, 68, 0, 79, 70, 0, 67, 82, \
73, 77, 69, 0, 66, 69, 65, 82, 83, 0, 66, 73, 84, 84, 69, 82, 0, 70, \
82, 85, 73, 84, 14, 0, 0, 0, 0, 0, 0, 0, 0, 0, 0, 0, 0, 0, 0, 0, 0, \
0, 0, 0, 0, 0, 0, 0, 0, 0, 0, 0, 0, 0, 0, 0, 0, 0, 0, 0, 0, 0, 0, 0, \
0, 0, 0, 0}
\end{lstlisting}

Зашифрованный байт это 2, байт из ключа это 103, $2 \oplus 103=101$ и 101 это ASCII-код символа \q{e}.
Чему должен равнятся этот байт ключа, чтобы ASCII-код был 116 (для символа  \q{t})?
$2 \oplus 116=118$, присвоим 118 второму байту в ключе \dots

\begin{lstlisting}[style=custommath]
key = {80, 118, 2, 116, 113, 102, 118, 25, 99, 8, 19, 23, 116, 125, 
  107, 25, 99, 109, 114, 102, 14, 121, 115, 31, 9, 117, 113, 111, 5, 
  4, 127, 28, 122, 101, 8, 110, 14, 18, 124, 106, 16, 20, 104, 119, 8,
   109, 26, 106, 9, 97, 13, 99, 15, 119, 20, 105, 117, 98, 103, 118, 
  1, 126, 29, 97, 122, 17, 15, 114, 110, 3, 5, 125, 125, 99, 126, 119,
   102, 30, 122, 2, 117}
\end{lstlisting}

\dots и снова дешифруем весь файл.

\begin{figure}[H]
\centering
\myincludegraphics{ff/XOR/mask_1/mc_decrypted2.png}
\caption{Дешифрованный файл в Midnight Commander, вторая попытка}
\end{figure}

Ух ты, теперь грамматика корректна, и все фразы начинаются с корректных букв.
Но все таки, регистр подозрителен.
С чего бы разработчику игры записывать их в такой манере?
Может быть наш ключ все еще неправилен?

% TODO ASCII table somewhere in the book
Изучая таблицу ASCII мы можем заметить что ASCII-коды для букв в верхнем и нижнем регистре отличаются только на один бит
(6-й бит, если считать с первого, 0b100000):

\begin{figure}[H]
\centering
\includegraphics[width=0.7\textwidth]{ascii.png}
\caption{7-битная таблица \ac{ASCII} в Emacs}
\end{figure}

6-й бит, выставленный в нулевом байте, В десятичном виде это будет 32.
Но 32 это ASCII-код пробела!

Действительно, можно менять регистр просто применяя XOR к ASCII-коду, с 32 (больше об этом: \myref{toupper_bit}).

Возможно ли, что пустые лакуны в файле это не нулевые байты, а скорее содержащие пробелы?
Еще раз модифицируем наш XOR-ключ (я про-XOR-ю каждый байт ключа с 32):

\begin{lstlisting}[style=custommath]
(* "32" это скаляр, и "key" это вектор, но это OK *)

In[]:= key3 = BitXor[32, key]
Out[]= {112, 86, 34, 84, 81, 70, 86, 57, 67, 40, 51, 55, 84, 93, 75, \
57, 67, 77, 82, 70, 46, 89, 83, 63, 41, 85, 81, 79, 37, 36, 95, 60, \
90, 69, 40, 78, 46, 50, 92, 74, 48, 52, 72, 87, 40, 77, 58, 74, 41, \
65, 45, 67, 47, 87, 52, 73, 85, 66, 71, 86, 33, 94, 61, 65, 90, 49, \
47, 82, 78, 35, 37, 93, 93, 67, 94, 87, 70, 62, 90, 34, 85}

In[]:= DecryptBlock[blk_] := BitXor[key3, blk]
\end{lstlisting}

И снова дешифруем входной файл:

\begin{figure}[H]
\centering
\myincludegraphics{ff/XOR/mask_1/mc_decrypted.png}
\caption{Дешифрованный файл в Midnight Commander, последняя попытка}
\end{figure}

(Расшифрованный файл доступен здесь:
\url{\GitHubBlobMasterURL/ff/XOR/mask_1/files/decrypted.dat.bz2}.)

Несомненно, это корректный исходный файл.
Да, и мы видим числа в начале каждого блока. Должно быть это и есть источник некорректного XOR-ключа.
Как выходит, самый встречающийся 81-байтный блок в файле это блок заполненный пробелами и содержащий символ \q{1} на месте
второго байта.
Действительно, как-то так получилось что многие блоки здесь перемежаются с этим блоком.
Может быть это что-то вроде выравнивания (padding) для коротких фраз/сообщений?
Другой часто встречающийся 81-байтный блок также заполнен пробелами, но с другой цифрой, следовательно,
они отличаются только вторым байтом.

Вот и всё! Теперь мы можем написать утилиту для зашифрования файла назад, и, может быть, модифицировать его перед этим

Файл для Mathematica можно скачать здесь:\\
\url{\GitHubBlobMasterURL/ff/XOR/mask_1/files/XOR_mask_1.nb}.

Итог: XOR-шифрование не надежно вообще. Вероятно, разработчик игры хотел просто скрыть внутренности игры от игрока,
ничего более серьезного.
Все же, шифрование вроде этого крайне популярно вследствии его простоты, так что многие реверс инженеры обычно хорошо
с этим знакомы.

}
\DE{\myparagraph{\NonOptimizing MSVC}

MSVC 2010 erzeugt den folgenden Code:

\lstinputlisting[caption=\NonOptimizing MSVC
2010,style=customasmx86]{patterns/12_FPU/3_comparison/x86/MSVC/MSVC_DE.asm}

\myindex{x86!\Instructions!FLD}

Der Befehl \FLD lädt \GTT{\_b} nach \ST{0}.

\label{Czero_etc}
\newcommand{\Czero}{\GTT{C0}\xspace}
\newcommand{\Ctwo}{\GTT{C2}\xspace}
\newcommand{\Cthree}{\GTT{C3}\xspace}
\newcommand{\CThreeBits}{\Cthree/\Ctwo/\Czero}

\myindex{x86!\Instructions!FCOMP}
\FCOMP verlgeicht den Wert in \ST{0} mit dem Wert, der sich in \GTT{\_a}
befindet und setzt die \CThreeBits im FPU Status Register entsprechend.
Das Statusregister ist ein 16-Bit-Register, das den aktueller Zustand der FPU
abbildet.

Nachdem die Bits gesetzt worden sind, nimmer der \FCOMP Befehl auch eine
Variable vom Stack. Dieses Verhalten unterscheidet ihn von \FCOM, der einfach
zwei Werte vergleicht und den Stack unangetastet lässt.

Leider verfügen CPUs vor Intel P6\footnote{Intel P6 ist Pentium Pro, Pentium II,
etc.}über keinerlei bedingte Sprungbefehle, die die \CThreeBits prüfen.

After the bits are set, the \FCOMP instruction also pops one variable from the stack. 
This is what distinguishes it from \FCOM, which is just compares values, leaving the stack in the same state.
Vielleicht ist diese Tatsache historisch begründet (man erinnere sich: die FPU
war früher ein eigener Chip).\\
Moderne CPUs, beginnend mit Intel P6 haben \FCOMI/\FCOMIP/\FUCOMI/\FUCOMIP
Befehle~--welche im Prinzip das gleiche tun, aber die \ZF/\PF/\CF Flags der CPU
verändern können.

\myindex{x86!\Instructions!FNSTSW}
Der \FNSTSW Befehl kopiert das FPU Statusregister nach \AX.
\CThreeBits werden an den Stellen 14/10/8 abgelegt, sie befinden sich im \AX
Register an den gleichen Stellen und sie werden alle in höherwertigen Teil von
\AX{}~---\AH{} abgelegt.

\begin{itemize}
\item Falls in unserem Beispiel $b>a$, dann werden die \CThreeBits Bits wie
folgt gesetzt: 0, 0, 0.
\item Falls $a>b$, dann ist das Bitmuster: 0, 0, 1.
\item Falls $a=b$, dann ist das Bitmuster: 1, 0, 0.
\item

Wenn das Ergebnis (z.B. im Fehlerfall) ungeordnet ist, dann werden die Bits wie
folgt gesetzt: 1,1,1.
\end{itemize}
% TODO: table here?
So werden die \CThreeBits Bits im \AX Register angeordnet:

\input{C3_in_AX}

So werden die \CThreeBits Bits im \AH Register angeordnet:

\input{C3_in_AH}
Nach der Ausführung von \INS{test ah, 5}\footnote{5=101b} werden nur die \Czero
und \Ctwo Bits (an den Stellen 0 und 2) betrachtet, alle übrigen Bits werden
einfach überlesen.

\label{parity_flag}
\myindex{x86!\Registers!\Flags!Parity flag}
Werfen wir nun einen Blick auf ein anderes bemerkenswertes historisches
Überbleibsel: das \emph{parity flag}.

Dieses Flag wird auf 1 gesetzt, falls die Anzahl der Einsen im Ergebnis der
letzten Berechnung gerade ist und auf 1, falls dies nicht der Fall ist.

Schlagen wir in der Wikipedia nach\footnote{\WikipediaParityFlag}:

%TODO Quotation has been translated from English wiki article, since the
% correspondig German article doesn't offer such information.
\begin{framed}
\begin{quotation}
Ein guter Grund das Parity Flag abzufragen, hat tatsächlich gar nichts mit
Parität zu tun. Die FPU hat vier Bedingungsflags (C0 bis C3), aber diese können
nicht direkt abgefragt werden, sondern müssen zunächst in das Flags Register
kopiert werden. Wenn dies geschieht, wird C0 im Carry Flag abgelegt, C2 im
Parity Flag und C3 im Zero Flag.
Das C2 Flag ist gesetzt, wenn z.B. unvergleichbare Fließkommawerte (NaN oder
nicht unterstütztes Format) über der \FUCOM Befehl miteinander verglichen
werden.\textit{(Übersetzung aus der englischen Wikipedia.)}
\end{quotation}
\end{framed}

Wie in der Wikipedia dargestellt wird das Parity Flag manchmal im FPU Code
verwendet; schauen wir uns genauer an wie das funktioniert.

\myindex{x86!\Instructions!JP}
Das \PF Flag wird auf 1 gesetzt, wenn sowohl \Czero als auch \Ctwo beide 0 oder
beide 1 sind. In diesem Fall wird der nachfolgende Sprung \JP(\emph{jump if
PF==1}) ausgeführt.
Wenn wir die Werte der \CThreeBits in den unterschiedlichen Fällen betrachten,
dann sehen wir, dass der bedingte Sprung \JP in zwei Fällen ausgeführt wird:
wenn $b>a$ oder wenn $a=b$ (das \Cthree Bit wird hier nicht betrachtet, da es
durch den Befehl \INS{test ah,5}) gelöscht wurde).

Der Rest ist leicht nachvollziehbar.
Denn der bedingte Sprung ausgeführt wurde, lädt \FLD den Wert von \GTT{\_b} nach
\ST{0} und wenn nicht, wird der Wert von \GTT{\_a} dorthin geladen.

\mysubparagraph{Was ist mit der Abfrage von \Ctwo?}
Das \Ctwo Flag wird im Fehlerfall (\gls{NaN}, etc.) gesetzt, aber unser Code
prüft dies nicht. 
Wenn sich der Programmierer für FPU Fehler interessiert, muss er zusätzliche
Abfragen hinzufügen.

\input{patterns/12_FPU/3_comparison/x86/MSVC/olly_DE.tex}
}
\FR{\mysection{Fonction presque vide}
\label{Boolector}
\myindex{Boolector}
\myindex{x86!\Instructions!JMP}

Ceci est un morceau de code réel que j'ai trouvé dans Boolector\footnote{\url{https://boolector.github.io/}}:

\lstinputlisting[style=customc]{patterns/025_almost_empty/boolectormain.c}

Pourquoi quelqu'un ferait-il comme ça?
Je ne sais pas mais mon hypothèse est que \verb|boolector_main()| peut être compilée
dans une sorte de DLL ou bibliothèque dynamique, et appelée depuis une suite de test.
Certainement qu'une suite de test peut préparer les variables argc/argv comme
le ferait \ac{CRT}.

Il est intéressant de voir comment c'est compilé:

\lstinputlisting[caption=GCC 8.2 x64 \NonOptimizing (\assemblyOutput),style=customasmx86]{patterns/025_almost_empty/boolectormain_O0.s}

Ceci est OK, le prologue (non optimisé) déplace inutilement deux arguments,
\INS{CALL}, épilogue, \INS{RET}.
Mais regardons la version optimisée:

\lstinputlisting[caption=GCC 8.2 x64 \Optimizing (\assemblyOutput),style=customasmx86]{patterns/025_almost_empty/boolectormain_O3.s}

Aussi simple que ça: la pile et les registres ne sont pas touchés et \verb|boolector_main()|
a le même ensemble d'arguments.
Donc, tout ce que nous avons à faire est de passer l'exécution à une autre adresse.

Ceci est proche d'une \glslink{thunk function}{fonction thunk}.

Nous verons queelque chose de plus avancé plus tard: \myref{ARM_B_to_printf}, \myref{JMP_instead_of_RET}.
}
\JA{\myparagraph{\NonOptimizing MSVC}

MSVC 2010は以下のコードを生成します。

\lstinputlisting[caption=\NonOptimizing MSVC 2010,style=customasmx86]{patterns/12_FPU/3_comparison/x86/MSVC/MSVC_JA.asm}

\myindex{x86!\Instructions!FLD}

\FLD は\GTT{\_b}を\ST{0}にロードします。

\label{Czero_etc}
\newcommand{\Czero}{\GTT{C0}\xspace}
\newcommand{\Ctwo}{\GTT{C2}\xspace}
\newcommand{\Cthree}{\GTT{C3}\xspace}
\newcommand{\CThreeBits}{\Cthree/\Ctwo/\Czero}

\myindex{x86!\Instructions!FCOMP}

\FCOMP は\ST{0}の値と\GTT{\_a}の値を比較し、
それに応じてFPUステータスワードレジスタの \CThreeBits ビットを設定します。
これは、FPUの現在の状態を反映する16ビットのレジスタです。

ビットがセットされると、 \FCOMP 命令はスタックから1つの変数もポップします。
これは、値を比較してスタックを同じ状態にしておく \FCOM とは区別されます。

残念ながら、インテルP6 
\footnote{インテルP6はPentium Pro、Pentium IIなどです。}
より前のCPUには、 \CThreeBits ビットをチェックする条件付きジャンプ命令はありません。
おそらく、それは歴史の問題です。(思い起こしてみてください:FPUは過去に別のチップでした)

インテルP6で始まる最新のCPUは、\FCOMI/\FCOMIP/\FUCOMI/\FUCOMIP 命令を持っていて、
同じことをしますが、 \ZF/\PF/\CF CPUフラグを変更します。

\myindex{x86!\Instructions!FNSTSW}

\FNSTSW 命令は状態レジスタであるFPUを \AX にコピーします。 
\CThreeBits ビットは14/10/8の位置に配置され、
\AX レジスタの同じ位置にあり、 \AX{}~---\AH{} の上位部分に配置されます。

\begin{itemize}
\item この例では $b>a$ の場合、 \CThreeBits ビットは0,0,0と設定します。
\item $a>b$ の場合、ビットは0,0,1です。
\item $a=b$ の場合、ビットは1,0,0です。
\item 結果が順序付けられていない場合(エラーの場合)、セットされたビットは1,1,1,1です。
\end{itemize}
% TODO: table here?

これは、 \CThreeBits ビットが \AX レジスタにどのように配置されるかを示しています。

\input{C3_in_AX}

これは、 \CThreeBits ビットが \AH レジスタにどのように配置されるかを示しています。

\input{C3_in_AH}

\INS{test ah, 5}\footnote{5=101b}の実行後、
\Czero と \Ctwo ビット(0と2の位置)のみが考慮され、他のビットはすべて無視されます。

\label{parity_flag}
\myindex{x86!\Registers!\Flags!Parity flag}

さて、\emph{パリティーフラグ}と注目すべきもう1つの歴史的基礎についてお話しましょう。

このフラグは、最後の計算結果の1の数が偶数の場合は1に設定され、奇数の場合は0に設定されます。

Wikipedia\footnote{\WikipediaParityFlag}を見てみましょう:

\begin{framed}
\begin{quotation}
パリティフラグをテストする一般的な理由の1つに、無関係なFPUフラグをチェックすることがあります。 FPUには4つの条件フラグ
(C0~C3)がありますが、直接テストすることはできず、最初にフラグレジスタにコピーする必要があります。 
これが起こると、C0はキャリーフラグに、C2はパリティフラグに、C3はゼロフラグに置かれます。 
C2フラグは、例えば比較できない浮動小数点値(NaNまたはサポートされていない形式)がFUCOM命令と比較されます。
\end{quotation}
\end{framed}

Wikipediaで述べられているように、パリティフラグはFPUコードで使用されることがあります。

\myindex{x86!\Instructions!JP}

\Czero と \Ctwo の両方が0に設定されている場合、 \PF フラグは1に設定されます。その場合、
後続の \JP (\emph{jump if PF==1})が実行されます。 
いろいろな場合の \CThreeBits の値を思い出すと、
条件ジャンプ \JP は、 $b>a$ または $a=b$ の場合に実行されます。
(\INS{test ah, 5}命令によってクリアされているので、 \Cthree ビットはここでは考慮されていません)

それ以降はすべて簡単です。 
条件付きジャンプが実行された場合、
\FLD は\ST{0}の\GTT{\_b}の値をロードし、
実行されていなければ\GTT{\_a}の値をロードします。

\mysubparagraph{\Ctwo? のチェックは?}

\Ctwo フラグはエラー(\gls{NaN}など)の場合に設定されますが、コードではチェックされません。

プログラマがFPUエラーを気にする場合は、チェックを追加する必要があります。

\input{patterns/12_FPU/3_comparison/x86/MSVC/olly_JA.tex}
}
\IT{\subsection{Ottenere i valori massimo e minimo}

\subsubsection{32-bit}

\lstinputlisting[style=customc]{patterns/07_jcc/minmax/minmax.c}

\lstinputlisting[caption=\NonOptimizing MSVC 2013,style=customasmx86]{patterns/07_jcc/minmax/minmax_MSVC_2013_IT.asm}

\myindex{x86!\Instructions!Jcc}

Queste due funzioni differiscono solo per l'istruzione di salto condizionale: 
\INS{JGE} (\q{Jump if Greater or Equal}) è usata nella prima
e \INS{JLE} (\q{Jump if Less or Equal}) nella seconda.

\myindex{\CompilerAnomaly}
\label{MSVC_double_JMP_anomaly}

In ciascuna funzione c'è un'istruzione \JMP non necessaria, che MSVC ha probabilmente lasciato per sbaglio.

\myparagraph{Branchless}

ARM in modalità Thumb ci ricorda molto codice x86:

\lstinputlisting[caption=\OptimizingKeilVI (\ThumbMode),style=customasmARM]{patterns/07_jcc/minmax/minmax_Keil_Thumb_O3_IT.s}

\myindex{ARM!\Instructions!Bcc}

Le funzioni differiscono per le istruzioni di branching: \INS{BGT} e \INS{BLT}.
Essendo possibile usare suffissi condizionali in modalità ARM, il codice è più conciso.

\myindex{ARM!\Instructions!MOVcc}
\INS{MOVcc} viene eseguita se la condizione è soddisfatta:

\lstinputlisting[caption=\OptimizingKeilVI (\ARMMode),style=customasmARM]{patterns/07_jcc/minmax/minmax_Keil_ARM_O3_IT.s}

\myindex{x86!\Instructions!CMOVcc}
\Optimizing, GCC 4.8.1 e MSVC 2013 possono usare l'istruzione \INS{CMOVcc}, che è analoga a \INS{MOVcc} in ARM:

\lstinputlisting[caption=\Optimizing MSVC 2013,style=customasmx86]{patterns/07_jcc/minmax/minmax_GCC481_O3_IT.s}

\subsubsection{64-bit}

\lstinputlisting[style=customc]{patterns/07_jcc/minmax/minmax64.c}

C'è un po' di spostamento di valori non necessario, ma il codice è comprensibile:

\lstinputlisting[caption=\NonOptimizing GCC 4.9.1 ARM64,style=customasmARM]{patterns/07_jcc/minmax/minmax64_GCC_49_ARM64_O0.s}

\myparagraph{Branchless}

Non è necessario caricare gli argomenti della funzione dallo stack poiché sono già nei registri:

\lstinputlisting[caption=\Optimizing GCC 4.9.1 x64,style=customasmx86]{patterns/07_jcc/minmax/minmax64_GCC_49_x64_O3_IT.s}

MSVC 2013 fa pressoché lo stesso.

\myindex{ARM!\Instructions!CSEL}

ARM64 ha l'istruzione \INS{CSEL}, che funziona esattamente come \INS{MOVcc} in ARM o \INS{CMOVcc} in x86, cambia soltanto il nome:
\q{Conditional SELect}.

\lstinputlisting[caption=\Optimizing GCC 4.9.1 ARM64,style=customasmARM]{patterns/07_jcc/minmax/minmax64_GCC_49_ARM64_O3_IT.s}

\subsubsection{MIPS}

Sfortunatamente GCC 4.4.5 per MIPS non è altrettanto bravo:

\lstinputlisting[caption=\Optimizing GCC 4.4.5 (IDA),style=customasmMIPS]{patterns/07_jcc/minmax/minmax_MIPS_O3_IDA_IT.lst}

Non ci dimentichiamo dei \emph{branch delay slot}: la prima \INS{MOVE} è eseguita \emph{prima} di \INS{BEQZ}, 
la seconda \INS{MOVE} viene eseguita solo se il branch non è stato seguito.

}


% Do not translate, this is macro:
\subsection{\Conclusion{}}

\subsubsection{x86}

La forma grezza di un jump condizionale è la seguente:

\begin{lstlisting}[caption=x86,style=customasmx86]
CMP register, register/value
Jcc true ; cc=condition code
false:
... codice da eseguire se il risultato del confronto è false ...
JMP exit 
true:
... codice da eseguire se il risultato del confronto è true ...
exit:
\end{lstlisting}

\subsubsection{ARM}

\begin{lstlisting}[caption=ARM,style=customasmARM]
CMP register, register/value
Bcc true ; cc=condition code
false:
... codice da eseguire se il risultato del confronto è false ...
JMP exit 
true:
... codice da eseguire se il risultato del confronto è true ...
exit:
\end{lstlisting}

\subsubsection{MIPS}

\begin{lstlisting}[caption=Check for zero,style=customasmMIPS]
BEQZ REG, label
...
\end{lstlisting}

\begin{lstlisting}[caption=Check for less than zero (using pseudoinstruction),style=customasmMIPS]
BLTZ REG, label
...
\end{lstlisting}

\begin{lstlisting}[caption=Check for equal values,style=customasmMIPS]
BEQ REG1, REG2, label
...
\end{lstlisting}

\begin{lstlisting}[caption=Check for non-equal values,style=customasmMIPS]
BNE REG1, REG2, label
...
\end{lstlisting}

\begin{lstlisting}[caption=Check for less than (signed),style=customasmMIPS]
SLT REG1, REG2, REG3
BEQ REG1, label
...
\end{lstlisting}

\begin{lstlisting}[caption=Check for less than (unsigned),style=customasmMIPS]
SLTU REG1, REG2, REG3
BEQ REG1, label
...
\end{lstlisting}

\subsubsection{Branchless}

\myindex{ARM!\Instructions!MOVcc}
\myindex{x86!\Instructions!CMOVcc}
\myindex{ARM!\Instructions!CSEL}
Se il corpo di uno statement condizionale è molto piccolo, può essere utilizzata l'istruzione "move" condizionale: 
\INS{MOVcc} in ARM (in ARM mode), \INS{CSEL} in ARM64, \INS{CMOVcc} in x86.

\myparagraph{ARM}

In ARM è possibile usare suffissi condizionali per alcune istruzioni:

\begin{lstlisting}[caption=ARM (\ARMMode),style=customasmARM]
CMP register, register/value
instr1_cc ; istruzione che sarà eseguita se il condition code è true
instr2_cc ; altra istruzione che sarà eseguita se il condition code è true
... etc...
\end{lstlisting}

Ovviamente non c'è limite al numero di istruzioni con il suffisso condizionale, a patto che le flag CPU non siano modificate da nessuna istruzione. 
% FIXME: list of such instructions or \myref{} to it

\myindex{ARM!\Instructions!IT}

La modalità Thumb ha l'istruzione \INS{IT}, che permette di aggiungere suffissi condizionali alle prossime quattro istruzioni.
Maggiori informazioni qui: \myref{ARM_Thumb_IT}.

\begin{lstlisting}[caption=ARM (\ThumbMode),style=customasmARM]
CMP register, register/value
ITEEE EQ ; imposta questi suffissi: if-then-else-else-else
instr1   ; istruzione da eseguire se la condizione è true
instr2   ; istruzione da eseguire se la condizione è false
instr3   ; istruzione da eseguire se la condizione è false
instr4   ; istruzione da eseguire se la condizione è false
\end{lstlisting}

% Do not translate, this is macro:
\subsection{\Exercise}

(ARM64) Prova a riscrivere il codice in \lstref{cond_ARM64} rimuovendo tutti i jump condizionali e usando al loro posto l'istruzione \TT{CSEL} instruction.
