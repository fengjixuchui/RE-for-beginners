\mysection{\Stack}
\label{sec:stack}
\myindex{\Stack}

Lo stack è una delle strutture dati più importanti in informatica
\footnote{\href{http://go.yurichev.com/17119}{wikipedia.org/wiki/Call\_stack}}.
\ac{AKA} \ac{LIFO}.

Tecnicamente, è soltanto un blocco di memoria nella memoria di un processo insieme al registro \ESP o \RSP in x86 o x64, o il registro \ac{SP} in ARM, come puntatore all'interno di quel blocco.

\myindex{ARM!\Instructions!PUSH}
\myindex{ARM!\Instructions!POP}
\myindex{x86!\Instructions!PUSH}
\myindex{x86!\Instructions!POP}
Le istruzioni di accesso allo stack più usate sono \PUSH e \POP (sia in x86 che in ARM Thumb-mode).
\PUSH sottrae da \ESP/\RSP/\ac{SP} 4 in modalità 32-bit (oppure 8 in modalità 64-bit) e scrive successivamente il contenuto del suo unico operando nell'indirizzo di memoria puntato da \ESP/\RSP/\ac{SP}.

\POP è l'operazione inversa: recupera il dato dalla memoria a cui punta \ac{SP}, lo carica nell'operando dell'istruzione (di solito un registro)
e successivamente aggiunge 4 (o 8) allo \gls{stack pointer}.

A seguito dell'allocazione dello stack, lo \gls{stack pointer} punta alla base (fondo) dello stack.
\PUSH decrementa lo \gls{stack pointer} e \POP lo incrementa.
La base dello stack è in realtà all'inizio del blocco di memoria allocato per lo stack. Sembra strano, ma è così.

ARM supporta sia stack decrescenti che crescenti.

\myindex{ARM!\Instructions!STMFD}
\myindex{ARM!\Instructions!LDMFD}
\myindex{ARM!\Instructions!STMED}
\myindex{ARM!\Instructions!LDMED}
\myindex{ARM!\Instructions!STMFA}
\myindex{ARM!\Instructions!LDMFA}
\myindex{ARM!\Instructions!STMEA}
\myindex{ARM!\Instructions!LDMEA}

Ad esempio le istruzioni \ac{STMFD}/\ac{LDMFD}, \ac{STMED}/\ac{LDMED} sono fatte per operare con uno stack decrescente (che cresce verso il basso, inizia con un indirizzo alto e prosegue verso il basso).
Le istruzioni \ac{STMFA}/\ac{LDMFA}, \ac{STMEA}/\ac{LDMEA} sono fatte per operare con uno stack crescente (che cresce verso l'alto, da un indirizzo basso verso uno più alto).

% It might be worth mentioning that STMED and STMEA write first,
% and then move the pointer,
% and that LDMED and LDMEA move the pointer first, and then read.
% In other words, ARM not only lets the stack grow in a non-standard direction,
% but also in a non-standard order.
% Maybe this can be in the glossary, which would explain why E stands for "empty".

\subsection{Perchè lo stack cresce al contrario?}
\label{stack_grow_backwards}

Intuitivamente potremmo pensare che lo stack cresca verso l'alto, ovvero verso indirizzi più alti, come qualunque altra struttura dati.

La ragione per cui lo stack cresce verso il basso è probabilmente di natura storica.
Quando i computer erano talmente grandi da occupare un'intera stanza, era facile dividere la memoria in due parti, una per lo
\gls{heap} e l'altra per lo stack.
Ovviamente non era possibile sapere a priori quanto sarebbero stati grandi lo stack e lo \gls{heap} durante l'esecuzione di un programma,
e questa soluzione era la più semplice.

\input{patterns/02_stack/stack_and_heap}

In \RitchieThompsonUNIX possiamo leggere:

\begin{framed}
\begin{quotation}
Il nucleo utente di una immagine è diviso in tre segmenti logici.
Il segmento text del programma inizia in posizione 0 nel virtual address space.
Durante l'esecuzione questo segmento viene protetto da scrittura, ed una sua singola copia viene condivisa tra i processi che eseguono lo stesso programma.
Al primo limite di 8K byte sopra il segmento text del programma, nel virtual address space comincia un segmento dati scrivibile, non condiviso, le cui dimensioni possono essere estese da una chiamata di sistema.A partire dall'indirizzo più alto nel virtual address space c'è lo stack segment, che automaticammente cresce verso il basso al variare dello stack pointer hardware.
\end{quotation}
\end{framed}

Questo ricorda molto come alcuni studenti utilizzino lo stesso quaderno per prendere appunti di due diverse materie:
gli appunti per la prima materia sono scritti normalmente, e quelli della seconda materia sono scritti a partire dalla fine del quaderno, capovolgendolo.
Le note si potrebbero "incontrare" da qualche parte in mezzo al quaderno, nel caso in cui non ci sia abbastanza spazio libero.

% I think if we want to expand on this analogy,
% one might remember that the line number increases as as you go down a page.
% So when you decrease the address when pushing to the stack, visually,
% the stack does grow upwards.
% Of course, the problem is that in most human languages,
% just as with computers,
% we write downwards, so this direction is what makes buffer overflows so messy.

\subsection{Per cosa viene usato lo stack?}

% subsections
\input{patterns/02_stack/01_saving_ret_addr_IT}
\subsubsection{Passaggio di argomenti alle funzioni}

Il modo più diffuso per passare parametri in x86 è detto \q{cdecl}:

\begin{lstlisting}[style=customasmx86]
push arg3
push arg2
push arg1
call f
add esp, 12 ; 4*3=12
\end{lstlisting}

La funzioni chiamate, \Gls{callee}, ricevono i propri argomenti tramite lo stack pointer.

Quindi è così che i valori degli argomenti sono posizionati nello stack prima dell'esecuzione della prima istruzione della funzione \ttf{}:

\begin{center}
\begin{tabular}{ | l | l | }
\hline
ESP & return address \\
\hline
ESP+4 & \argument \#1, \MarkedInIDAAs{} \TT{arg\_0} \\
\hline
ESP+8 & \argument \#2, \MarkedInIDAAs{} \TT{arg\_4} \\
\hline
ESP+0xC & \argument \#3, \MarkedInIDAAs{} \TT{arg\_8} \\
\hline
\dots & \dots \\
\hline
\end{tabular}
\end{center}

Per ulteriori informazioni su altri tipi di convenzioni di chiamata (calling conventions), fare riferimento alla sezione~(\myref{sec:callingconventions}).

\par
A proposito, la funzione \gls{callee}{chiamata} non possiede alcuna informazione su quanti argomenti sono stati passati.
Le funzioni C con un numero variabile di argomenti (come \printf) determinano il loro numero attraverso specificatori di formato stringa (che iniziano con il simbolo \%).

Se scriviamo qualcosa come:

\begin{lstlisting}
printf("%d %d %d", 1234);
\end{lstlisting}

\printf scriverà 1234, e successivamente due numeri casuali\footnote{Non casuali in senso stretto, ma piuttosto non predicibili: \myref{noise_in_stack}}, che si trovavano lì vicino nello stack.

\label{main_arguments}
\par
Per questo motivo non è molto importante come dichiariamo la funzione \main: come \main, \\
\TT{main(int argc, char *argv[])} oppure \TT{main(int argc, char *argv[], char *envp[])}.

Infatti, il codice \ac{CRT} sta chiamando \main circa in questo modo:

\begin{lstlisting}[style=customasmx86]
push envp
push argv
push argc
call main
...
\end{lstlisting}

Se dichiari \main come \main senza argomenti, questi sono, in ogni caso, ancora presenti nello stack, ma non vengono utilizzati.
Se dichiari \main come \TT{main(int argc, char *argv[])},
sarai in grado di utilizzare i primi due argomenti, ed il terzo rimarrà \q{invisibile} per la tua funzione.
In più, è possibile dichiarare \TT{main(int argc)}, e continuerà a funzionare.

% TBT Another related example: \ref{cdecl_DLL}.

\myparagraph{Metodi alternativi per passare argomenti}

Vale la pena notare che non c'è nulla che obbliga il programmatore a passare gli argomenti attraverso lo stack. Non è un requisito necessario.
Si potrebbe implementare un qualunque altro metodo anche senza usare per niente lo stack.

Un metodo abbastanza popolare tra chi inizia a programmare in linguaggio assembly language è di passare argomenti attraverso variabili globali, in questo modo:

\lstinputlisting[caption=Assembly code,style=customasmx86]{patterns/02_stack/global_args.asm}

Tuttavia questo metodo ha un limite evidente: la funzione \emph{do\_something()} non può richiamare sè stessa in modo ricorsivo (o attraverso un'altra funzione),
perchè deve cancellare i suoi stessi argomenti.
Lo stesso accade con le variabili locali: se le tieni in variabili globali, la funzione non può chiamare se stessa.
Inoltre questo non sarebbe thread-safe
\footnote{Implementato correttamente, ciascun thread avrebbe il suo proprio stack con i suoi argomenti/variabili.}.
Il metodo di memorizzare queste informazioni nello stack rende il tutto più semplice---può mantenere quanti argomenti di funzione e/o valori,
quanto spazio è disponibile.

\InSqBrackets{\TAOCPvolI{}, 189} menziona alcuni schemi ancora più strani e particolarmente convenienti su IBM System/360.

\myindex{MS-DOS}
\myindex{x86!\Instructions!INT}

MS-DOS utilizzava un modo per passare tutti gli argomenti di funzione via registri, ad esempio, in questo pezzo
di codice per MS-DOS a 16 bit scrive ``Hello, world!'':

\begin{lstlisting}[style=customasmx86]
mov  dx, msg      ; indirizzo del messaggio
mov  ah, 9        ; 9 indica la funzione "print string"
int  21h          ; "syscall" (chiamata di sistema) DOS

mov  ah, 4ch      ; funzione "termina il programma"
int  21h          ; "syscall" DOS

msg  db 'Hello, World!\$'
\end{lstlisting}

\myindex{fastcall}
Questo è abbastanza simile al metodo \myref{fastcall}.
Ed è inoltre molto simile alle chiamate syscalls in Linux (\myref{linux_syscall}) e Windows.

\myindex{x86!\Flags!CF}
Se una funzione MS-DOS restituisce un valore di tipo boolean (cioè, un singolo bit, di solito per indicare uno stato di errore),
il flag \TT{CF} era spesso utilizzato.

Ad esempio:

\begin{lstlisting}[style=customasmx86]
mov ah, 3ch       ; crea file
lea dx, filename
mov cl, 1
int 21h
jc  error
mov file_handle, ax
...
error:
...
\end{lstlisting}

In caso di errore, il flag \TT{CF} viene innalzato. Altrimenti, l'handle ad un nuovo file creato viene restituito attraverso \TT{AX}.

Questo metodo viene ancora utilizzato dai programmatori assembly.
Nel codice sorgente del Windows Research Kernel (che è abbastanza simile a Windows 2003) possiamo trovare qualcosa tipo:
(file \emph{base/ntos/ke/i386/cpu.asm}):

\begin{lstlisting}[style=customasmx86]
        public  Get386Stepping
Get386Stepping  proc

        call    MultiplyTest            ; Esegue test di moltiplicazione
        jnc     short G3s00             ; se nc, muttest è ok
        mov     ax, 0
        ret
G3s00:
        call    Check386B0              ; Verifica B0 stepping
        jnc     short G3s05             ; se nc, è B1/later
        mov     ax, 100h                ; è B0/earlier stepping
        ret

G3s05:
        call    Check386D1              ; Verifica D1 stepping
        jc      short G3s10             ; se c, non è D1
        mov     ax, 301h                ; è D1/later stepping
        ret

G3s10:
        mov     ax, 101h                ; suppone che sia B1 stepping
        ret

	...

MultiplyTest    proc

        xor     cx,cx                   ; 64K volte è un bel numero tondo
mlt00:  push    cx
        call    Multiply                ; la moltiplicazione funziona in questo chip?
        pop     cx
        jc      short mltx              ; se c, No, esci
        loop    mlt00                   ; se nc, Si, cicla per riprovare
        clc
mltx:
        ret

MultiplyTest    endp
\end{lstlisting}

\input{patterns/02_stack/03_local_vars_IT}
\EN{% TODO translate
\mysection{Breaking simple executable cryptor}

I've got an executable file which is encrypted by relatively simple encryption.
\href{\GitHubBlobMasterURL/examples/simple_exec_crypto/files/cipher.bin}{Here is it} (only executable section is left here).

First, all encryption function does is just adds number of position in buffer to the byte.
Here is how this can be encoded in Python:

\begin{lstlisting}[caption=Python script,style=custompy]
#!/usr/bin/env python
def e(i, k):
    return chr ((ord(i)+k) % 256)

def encrypt(buf):
    return e(buf[0], 0)+ e(buf[1], 1)+ e(buf[2], 2) + e(buf[3], 3)+ e(buf[4], 4)+ e(buf[5], 5)+ e(buf[6], 6)+ e(buf[7], 7)+
           e(buf[8], 8)+ e(buf[9], 9)+ e(buf[10], 10)+ e(buf[11], 11)+ e(buf[12], 12)+ e(buf[13], 13)+ e(buf[14], 14)+ e(buf[15], 15)
\end{lstlisting}

Hence, if you encrypt buffer with 16 zeros, you'll get \emph{0, 1, 2, 3 ... 12, 13, 14, 15}.

\myindex{Propagating Cipher Block Chaining}
Propagating Cipher Block Chaining (PCBC) is also used, here is how it works:

\begin{figure}[H]
\centering
\myincludegraphics{examples/simple_exec_crypto/601px-PCBC_encryption.png}
\caption{Propagating Cipher Block Chaining encryption (image is taken from Wikipedia article)}
\end{figure}

The problem is that it's too boring to recover IV (Initialization Vector) each time.
Brute-force is also not an option, because IV is too long (16 bytes).
Let's see, if it's possible to recover IV for arbitrary encrypted executable file?

Let's try simple frequency analysis.
This is 32-bit x86 executable code, so let's gather statistics about most frequent bytes and opcodes.
I tried huge oracle.exe file from Oracle RDBMS version 11.2 for windows x86 and I've found that the most frequent byte (no surprise) is zero (~10\%).
The next most frequent byte is (again, no surprise) 0xFF (~5\%).
The next is 0x8B (~5\%).

\myindex{x86!\Instructions!MOV}
0x8B is opcode for \INS{MOV}, this is indeed one of the most busy x86 instructions.
Now what about popularity of zero byte?
If compiler needs to encode value bigger than 127, it has to use 32-bit displacement instead of 8-bit one, but large values are very rare,
so it is padded by zeros.
\myindex{x86!\Instructions!LEA}
\myindex{x86!\Instructions!PUSH}
\myindex{x86!\Instructions!CALL}
This is at least in \INS{LEA}, \INS{MOV}, \INS{PUSH}, \INS{CALL}.

For example:

\begin{lstlisting}[style=customasmx86]
8D B0 28 01 00 00                 lea     esi, [eax+128h]
8D BF 40 38 00 00                 lea     edi, [edi+3840h]
\end{lstlisting}

Displacements bigger than 127 are very popular, but they are rarely exceeds 0x10000
(indeed, such large memory buffers/structures are also rare).

Same story with \INS{MOV}, large constants are rare, the most heavily used are 0, 1, 10, 100, $2^n$, and so on.
Compiler has to pad small constants by zeros to represent them as 32-bit values:

\begin{lstlisting}[style=customasmx86]
BF 02 00 00 00                    mov     edi, 2
BF 01 00 00 00                    mov     edi, 1
\end{lstlisting}

Now about 00 and FF bytes combined: jumps (including conditional) and calls can pass execution flow forward or backwards, but very often,
within the limits of the current executable module.
If forward, displacement is not very big and also padded with zeros.
If backwards, displacement is represented as negative value, so padded with FF bytes.
For example, transfer execution flow forward:

\begin{lstlisting}[style=customasmx86]
E8 43 0C 00 00                    call    _function1
E8 5C 00 00 00                    call    _function2
0F 84 F0 0A 00 00                 jz      loc_4F09A0
0F 84 EB 00 00 00                 jz      loc_4EFBB8
\end{lstlisting}

Backwards:

\begin{lstlisting}[style=customasmx86]
E8 79 0C FE FF                    call    _function1
E8 F4 16 FF FF                    call    _function2
0F 84 F8 FB FF FF                 jz      loc_8212BC
0F 84 06 FD FF FF                 jz      loc_FF1E7D
\end{lstlisting}

FF byte is also very often occurred in negative displacements like these:

\begin{lstlisting}[style=customasmx86]
8D 85 1E FF FF FF                 lea     eax, [ebp-0E2h]
8D 95 F8 5C FF FF                 lea     edx, [ebp-0A308h]
\end{lstlisting}

So far so good. Now we have to try various 16-byte keys, decrypt executable section and measure how often 00, FF and 8B bytes are occurred.
Let's also keep in sight how PCBC decryption works:

\begin{figure}[H]
\centering
\myincludegraphics{examples/simple_exec_crypto/640px-PCBC_decryption.png}
\caption{Propagating Cipher Block Chaining decryption (image is taken from Wikipedia article)}
\end{figure}

The good news is that we don't really have to decrypt whole piece of data, but only slice by slice, this is exactly how I did in my previous example: \myref{XOR_mask_2}.

Now I'm trying all possible bytes (0..255) for each byte in key and just pick the byte producing maximal amount of 00/FF/8B bytes in a decrypted slice:

\begin{lstlisting}[style=custompy]
#!/usr/bin/env python
import sys, hexdump, array, string, operator

KEY_LEN=16

def chunks(l, n):
    # split n by l-byte chunks
    # https://stackoverflow.com/q/312443
    n = max(1, n)
    return [l[i:i + n] for i in range(0, len(l), n)]

def read_file(fname):
    file=open(fname, mode='rb')
    content=file.read()
    file.close()
    return content

def decrypt_byte (c, key):
    return chr((ord(c)-key) % 256)

def XOR_PCBC_step (IV, buf, k):
    prev=IV
    rt=""
    for c in buf:
	new_c=decrypt_byte(c, k)
        plain=chr(ord(new_c)^ord(prev))
	prev=chr(ord(c)^ord(plain))
	rt=rt+plain
    return rt

each_Nth_byte=[""]*KEY_LEN

content=read_file(sys.argv[1])
# split input by 16-byte chunks:
all_chunks=chunks(content, KEY_LEN)
for c in all_chunks:
    for i in range(KEY_LEN):
        each_Nth_byte[i]=each_Nth_byte[i] + c[i]

# try each byte of key
for N in range(KEY_LEN):
    print "N=", N
    stat={}
    for i in range(256):
        tmp_key=chr(i)
	tmp=XOR_PCBC_step(tmp_key,each_Nth_byte[N], N)
        # count 0, FFs and 8Bs in decrypted buffer:
	important_bytes=tmp.count('\x00')+tmp.count('\xFF')+tmp.count('\x8B')
	stat[i]=important_bytes
    sorted_stat = sorted(stat.iteritems(), key=operator.itemgetter(1), reverse=True)
    print sorted_stat[0]
\end{lstlisting}

(Source code can be downloaded \href{\GitHubBlobMasterURL/examples/simple_exec_crypto/files/decrypt.py}{here}.)

I run it and here is a key for which 00/FF/8B bytes presence in decrypted buffer is maximal:

\begin{lstlisting}
N= 0
(147, 1224)
N= 1
(94, 1327)
N= 2
(252, 1223)
N= 3
(218, 1266)
N= 4
(38, 1209)
N= 5
(192, 1378)
N= 6
(199, 1204)
N= 7
(213, 1332)
N= 8
(225, 1251)
N= 9
(112, 1223)
N= 10
(143, 1177)
N= 11
(108, 1286)
N= 12
(10, 1164)
N= 13
(3, 1271)
N= 14
(128, 1253)
N= 15
(232, 1330)
\end{lstlisting}

Let's write decryption utility with the key we got:

\begin{lstlisting}[style=custompy]
#!/usr/bin/env python
import sys, hexdump, array

def xor_strings(s,t):
    # \verb|https://en.wikipedia.org/wiki/XOR_cipher#Example_implementation|
    """xor two strings together"""
    return "".join(chr(ord(a)^ord(b)) for a,b in zip(s,t))

IV=array.array('B', [147, 94, 252, 218, 38, 192, 199, 213, 225, 112, 143, 108, 10, 3, 128, 232]).tostring()

def chunks(l, n):
    n = max(1, n)
    return [l[i:i + n] for i in range(0, len(l), n)]

def read_file(fname):
    file=open(fname, mode='rb')
    content=file.read()
    file.close()
    return content

def decrypt_byte(i, k):
    return chr ((ord(i)-k) % 256)

def decrypt(buf):
    return "".join(decrypt_byte(buf[i], i) for i in range(16))

fout=open(sys.argv[2], mode='wb')

prev=IV
content=read_file(sys.argv[1])
tmp=chunks(content, 16)
for c in tmp:
    new_c=decrypt(c)
    p=xor_strings (new_c, prev)
    prev=xor_strings(c, p)
    fout.write(p)
fout.close()
\end{lstlisting}

(Source code can be downloaded \href{\GitHubBlobMasterURL/examples/simple_exec_crypto/files/decrypt2.py}{here}.)

Let's check resulting file:

\lstinputlisting{examples/simple_exec_crypto/objdump_result.txt}

Yes, this is seems correctly disassembled piece of x86 code.
The whole decryped file can be downloaded \href{\GitHubBlobMasterURL/examples/simple_exec_crypto/files/decrypted.bin}{here}.

In fact, this is text section from regedit.exe from Windows 7.
But this example is based on a real case I encountered, so just executable is different (and key), algorithm is the same.

\subsection{Other ideas to consider}

What if I would fail with such simple frequency analysis?
There are other ideas on how to measure correctness of decrypted/decompressed x86 code:

\begin{itemize}

\item Many modern compilers aligns functions on 0x10 border.
So the space left before is filled with NOPs (0x90) or other NOP instructions with known opcodes: \myref{sec:npad}.

\item Perhaps, the most frequent pattern in any assembly language is function call:\\
\TT{PUSH chain / CALL / ADD ESP, X}.
This sequence can easily detected and found.
I've even gathered statistics about average number of function arguments: \myref{args_stat}.
(Hence, this is average length of PUSH chain.)

\end{itemize}

Read more about incorrectly/correctly disassembled code: \myref{ISA_detect}.
}%
\FR{\mysection{Une fonction vide: redux}

Revenons sur l'exemple de la fonction vide \myref{empty_func}.
Maintenant que nous connaissons le prologue et l'épilogue de fonction, ceci est
une fonction vide \myref{lst:empty_func} compilée par GCC sans optimisation:

\lstinputlisting[caption=GCC 8.2 x64 \NonOptimizing (\assemblyOutput),style=customasmx86]{patterns/016_empty_redux/1.s}

C'est \INS{RET}, mais le prologue et l'épilogue de la fonction, probablement, n'ont
pas été optimisés et laissés tels quels.
\INS{NOP} semble être un autre artefact du compilateur.
De toutes façons, la seule instruction effective ici est \INS{RET}.
Toutes les autres instructions peuvent être supprimées (ou optimisées).

}

\input{patterns/02_stack/05_SEH}
\input{patterns/02_stack/06_BO_protection}

\subsubsection{Deallocazione automatica dei dati nello stack}

Probabilmente la ragione per cui si memorizzano nello stack le variabili locali e i record SEH deriva dal fatto che questi dati vengono "liberati" automaticamente all'uscita dalla funzione,
usando soltanto un'istruzione per correggere lo stack pointer (spesso è \ADD).
Si può dire che anche gli argomenti delle funzioni sono deallocati automaticamente alla fine della funzione.
Invece, qualunque altra cosa memorizzata nello \emph{heap} deve essere deallocata esplicitamente.

% subsections
\input{patterns/02_stack/07_layout_IT}
\EN{% TODO translate
\mysection{Breaking simple executable cryptor}

I've got an executable file which is encrypted by relatively simple encryption.
\href{\GitHubBlobMasterURL/examples/simple_exec_crypto/files/cipher.bin}{Here is it} (only executable section is left here).

First, all encryption function does is just adds number of position in buffer to the byte.
Here is how this can be encoded in Python:

\begin{lstlisting}[caption=Python script,style=custompy]
#!/usr/bin/env python
def e(i, k):
    return chr ((ord(i)+k) % 256)

def encrypt(buf):
    return e(buf[0], 0)+ e(buf[1], 1)+ e(buf[2], 2) + e(buf[3], 3)+ e(buf[4], 4)+ e(buf[5], 5)+ e(buf[6], 6)+ e(buf[7], 7)+
           e(buf[8], 8)+ e(buf[9], 9)+ e(buf[10], 10)+ e(buf[11], 11)+ e(buf[12], 12)+ e(buf[13], 13)+ e(buf[14], 14)+ e(buf[15], 15)
\end{lstlisting}

Hence, if you encrypt buffer with 16 zeros, you'll get \emph{0, 1, 2, 3 ... 12, 13, 14, 15}.

\myindex{Propagating Cipher Block Chaining}
Propagating Cipher Block Chaining (PCBC) is also used, here is how it works:

\begin{figure}[H]
\centering
\myincludegraphics{examples/simple_exec_crypto/601px-PCBC_encryption.png}
\caption{Propagating Cipher Block Chaining encryption (image is taken from Wikipedia article)}
\end{figure}

The problem is that it's too boring to recover IV (Initialization Vector) each time.
Brute-force is also not an option, because IV is too long (16 bytes).
Let's see, if it's possible to recover IV for arbitrary encrypted executable file?

Let's try simple frequency analysis.
This is 32-bit x86 executable code, so let's gather statistics about most frequent bytes and opcodes.
I tried huge oracle.exe file from Oracle RDBMS version 11.2 for windows x86 and I've found that the most frequent byte (no surprise) is zero (~10\%).
The next most frequent byte is (again, no surprise) 0xFF (~5\%).
The next is 0x8B (~5\%).

\myindex{x86!\Instructions!MOV}
0x8B is opcode for \INS{MOV}, this is indeed one of the most busy x86 instructions.
Now what about popularity of zero byte?
If compiler needs to encode value bigger than 127, it has to use 32-bit displacement instead of 8-bit one, but large values are very rare,
so it is padded by zeros.
\myindex{x86!\Instructions!LEA}
\myindex{x86!\Instructions!PUSH}
\myindex{x86!\Instructions!CALL}
This is at least in \INS{LEA}, \INS{MOV}, \INS{PUSH}, \INS{CALL}.

For example:

\begin{lstlisting}[style=customasmx86]
8D B0 28 01 00 00                 lea     esi, [eax+128h]
8D BF 40 38 00 00                 lea     edi, [edi+3840h]
\end{lstlisting}

Displacements bigger than 127 are very popular, but they are rarely exceeds 0x10000
(indeed, such large memory buffers/structures are also rare).

Same story with \INS{MOV}, large constants are rare, the most heavily used are 0, 1, 10, 100, $2^n$, and so on.
Compiler has to pad small constants by zeros to represent them as 32-bit values:

\begin{lstlisting}[style=customasmx86]
BF 02 00 00 00                    mov     edi, 2
BF 01 00 00 00                    mov     edi, 1
\end{lstlisting}

Now about 00 and FF bytes combined: jumps (including conditional) and calls can pass execution flow forward or backwards, but very often,
within the limits of the current executable module.
If forward, displacement is not very big and also padded with zeros.
If backwards, displacement is represented as negative value, so padded with FF bytes.
For example, transfer execution flow forward:

\begin{lstlisting}[style=customasmx86]
E8 43 0C 00 00                    call    _function1
E8 5C 00 00 00                    call    _function2
0F 84 F0 0A 00 00                 jz      loc_4F09A0
0F 84 EB 00 00 00                 jz      loc_4EFBB8
\end{lstlisting}

Backwards:

\begin{lstlisting}[style=customasmx86]
E8 79 0C FE FF                    call    _function1
E8 F4 16 FF FF                    call    _function2
0F 84 F8 FB FF FF                 jz      loc_8212BC
0F 84 06 FD FF FF                 jz      loc_FF1E7D
\end{lstlisting}

FF byte is also very often occurred in negative displacements like these:

\begin{lstlisting}[style=customasmx86]
8D 85 1E FF FF FF                 lea     eax, [ebp-0E2h]
8D 95 F8 5C FF FF                 lea     edx, [ebp-0A308h]
\end{lstlisting}

So far so good. Now we have to try various 16-byte keys, decrypt executable section and measure how often 00, FF and 8B bytes are occurred.
Let's also keep in sight how PCBC decryption works:

\begin{figure}[H]
\centering
\myincludegraphics{examples/simple_exec_crypto/640px-PCBC_decryption.png}
\caption{Propagating Cipher Block Chaining decryption (image is taken from Wikipedia article)}
\end{figure}

The good news is that we don't really have to decrypt whole piece of data, but only slice by slice, this is exactly how I did in my previous example: \myref{XOR_mask_2}.

Now I'm trying all possible bytes (0..255) for each byte in key and just pick the byte producing maximal amount of 00/FF/8B bytes in a decrypted slice:

\begin{lstlisting}[style=custompy]
#!/usr/bin/env python
import sys, hexdump, array, string, operator

KEY_LEN=16

def chunks(l, n):
    # split n by l-byte chunks
    # https://stackoverflow.com/q/312443
    n = max(1, n)
    return [l[i:i + n] for i in range(0, len(l), n)]

def read_file(fname):
    file=open(fname, mode='rb')
    content=file.read()
    file.close()
    return content

def decrypt_byte (c, key):
    return chr((ord(c)-key) % 256)

def XOR_PCBC_step (IV, buf, k):
    prev=IV
    rt=""
    for c in buf:
	new_c=decrypt_byte(c, k)
        plain=chr(ord(new_c)^ord(prev))
	prev=chr(ord(c)^ord(plain))
	rt=rt+plain
    return rt

each_Nth_byte=[""]*KEY_LEN

content=read_file(sys.argv[1])
# split input by 16-byte chunks:
all_chunks=chunks(content, KEY_LEN)
for c in all_chunks:
    for i in range(KEY_LEN):
        each_Nth_byte[i]=each_Nth_byte[i] + c[i]

# try each byte of key
for N in range(KEY_LEN):
    print "N=", N
    stat={}
    for i in range(256):
        tmp_key=chr(i)
	tmp=XOR_PCBC_step(tmp_key,each_Nth_byte[N], N)
        # count 0, FFs and 8Bs in decrypted buffer:
	important_bytes=tmp.count('\x00')+tmp.count('\xFF')+tmp.count('\x8B')
	stat[i]=important_bytes
    sorted_stat = sorted(stat.iteritems(), key=operator.itemgetter(1), reverse=True)
    print sorted_stat[0]
\end{lstlisting}

(Source code can be downloaded \href{\GitHubBlobMasterURL/examples/simple_exec_crypto/files/decrypt.py}{here}.)

I run it and here is a key for which 00/FF/8B bytes presence in decrypted buffer is maximal:

\begin{lstlisting}
N= 0
(147, 1224)
N= 1
(94, 1327)
N= 2
(252, 1223)
N= 3
(218, 1266)
N= 4
(38, 1209)
N= 5
(192, 1378)
N= 6
(199, 1204)
N= 7
(213, 1332)
N= 8
(225, 1251)
N= 9
(112, 1223)
N= 10
(143, 1177)
N= 11
(108, 1286)
N= 12
(10, 1164)
N= 13
(3, 1271)
N= 14
(128, 1253)
N= 15
(232, 1330)
\end{lstlisting}

Let's write decryption utility with the key we got:

\begin{lstlisting}[style=custompy]
#!/usr/bin/env python
import sys, hexdump, array

def xor_strings(s,t):
    # \verb|https://en.wikipedia.org/wiki/XOR_cipher#Example_implementation|
    """xor two strings together"""
    return "".join(chr(ord(a)^ord(b)) for a,b in zip(s,t))

IV=array.array('B', [147, 94, 252, 218, 38, 192, 199, 213, 225, 112, 143, 108, 10, 3, 128, 232]).tostring()

def chunks(l, n):
    n = max(1, n)
    return [l[i:i + n] for i in range(0, len(l), n)]

def read_file(fname):
    file=open(fname, mode='rb')
    content=file.read()
    file.close()
    return content

def decrypt_byte(i, k):
    return chr ((ord(i)-k) % 256)

def decrypt(buf):
    return "".join(decrypt_byte(buf[i], i) for i in range(16))

fout=open(sys.argv[2], mode='wb')

prev=IV
content=read_file(sys.argv[1])
tmp=chunks(content, 16)
for c in tmp:
    new_c=decrypt(c)
    p=xor_strings (new_c, prev)
    prev=xor_strings(c, p)
    fout.write(p)
fout.close()
\end{lstlisting}

(Source code can be downloaded \href{\GitHubBlobMasterURL/examples/simple_exec_crypto/files/decrypt2.py}{here}.)

Let's check resulting file:

\lstinputlisting{examples/simple_exec_crypto/objdump_result.txt}

Yes, this is seems correctly disassembled piece of x86 code.
The whole decryped file can be downloaded \href{\GitHubBlobMasterURL/examples/simple_exec_crypto/files/decrypted.bin}{here}.

In fact, this is text section from regedit.exe from Windows 7.
But this example is based on a real case I encountered, so just executable is different (and key), algorithm is the same.

\subsection{Other ideas to consider}

What if I would fail with such simple frequency analysis?
There are other ideas on how to measure correctness of decrypted/decompressed x86 code:

\begin{itemize}

\item Many modern compilers aligns functions on 0x10 border.
So the space left before is filled with NOPs (0x90) or other NOP instructions with known opcodes: \myref{sec:npad}.

\item Perhaps, the most frequent pattern in any assembly language is function call:\\
\TT{PUSH chain / CALL / ADD ESP, X}.
This sequence can easily detected and found.
I've even gathered statistics about average number of function arguments: \myref{args_stat}.
(Hence, this is average length of PUSH chain.)

\end{itemize}

Read more about incorrectly/correctly disassembled code: \myref{ISA_detect}.
}%
\FR{\mysection{Une fonction vide: redux}

Revenons sur l'exemple de la fonction vide \myref{empty_func}.
Maintenant que nous connaissons le prologue et l'épilogue de fonction, ceci est
une fonction vide \myref{lst:empty_func} compilée par GCC sans optimisation:

\lstinputlisting[caption=GCC 8.2 x64 \NonOptimizing (\assemblyOutput),style=customasmx86]{patterns/016_empty_redux/1.s}

C'est \INS{RET}, mais le prologue et l'épilogue de la fonction, probablement, n'ont
pas été optimisés et laissés tels quels.
\INS{NOP} semble être un autre artefact du compilateur.
De toutes façons, la seule instruction effective ici est \INS{RET}.
Toutes les autres instructions peuvent être supprimées (ou optimisées).

}

\input{patterns/02_stack/exercises}

