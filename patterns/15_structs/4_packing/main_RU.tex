\subsection{\StructurePackingSectionName}
\label{structure_packing}

Достаточно немаловажный момент, это упаковка полей в структурах.

Возьмем простой пример:

\lstinputlisting[style=customc]{patterns/15_structs/4_packing/packing.c}

Как видно, мы имеем два поля \Tchar (занимающий один байт) и еще два ~--- \Tint (по 4 байта).

% subsections:
\subsubsection{x86}

Компилируется это все в:

\lstinputlisting[caption=MSVC 2012 /GS- /Ob0,label=src:struct_packing_4,numbers=left,style=customasmx86]{patterns/15_structs/4_packing/packing_RU.asm}

Кстати, мы передаем всю структуру, но в реальности, как видно, структура в начале копируется
во временную структуру (выделение места под нее в стеке происходит в строке 10,
а все 4 поля, по одному, копируются в строках 12 \ldots\ 19), 
затем передается только указатель на нее (или адрес).

Структура копируется, потому что неизвестно, будет ли функция \ttf модифицировать структуру или нет.
И если да, то структура внутри \main должна остаться той же.

Мы могли бы использовать указатели на \CCpp, и итоговый код был бы почти такой же,
только копирования не было бы.

Мы видим здесь что адрес каждого поля в структуре выравнивается по 4-байтной границе. 
Так что каждый \Tchar здесь занимает те же 4 байта что и \Tint. Зачем? 
Затем что процессору удобнее обращаться по таким адресам и кэшировать данные из памяти.

Но это не экономично по размеру данных.

Попробуем скомпилировать тот же исходник с опцией (\TT{/Zp1}) 
(\emph{/Zp[n] pack structures on n-byte boundary}).

\lstinputlisting[caption=MSVC 2012 /GS- /Zp1,label=src:struct_packing_1,numbers=left,style=customasmx86]{patterns/15_structs/4_packing/packing_msvc_Zp1_RU.asm}

Теперь структура занимает 10 байт и все \Tchar занимают по байту. Что это дает? 
Экономию места. Недостаток ~--- процессор будет обращаться к этим полям не так эффективно 
по скорости, как мог бы.

\label{short_struct_copying_using_MOV}
Структура так же копируется в \main. Но не по одному полю, а 10 байт, при помощи трех
пар \MOV.

Почему не 4?
Компилятор рассудил, что будет лучше скопировать 10 байт
при помощи 3 пар \MOV, чем копировать два 32-битных слова и два байта при помощи 4 пар \MOV.

Кстати, подобная реализация копирования при помощи \MOV взамен вызова функции \TT{memcpy()}, например, это
очень распространенная практика, потому что это в любом случае работает быстрее чем вызов \TT{memcpy()} ---
если речь идет о коротких блоках, конечно: \myref{copying_short_blocks}.

Как нетрудно догадаться, если структура используется много в каких исходниках и объектных файлах, 
все они должны быть откомпилированы с одним и тем же соглашением об упаковке структур.

Помимо ключа MSVC \TT{/Zp}, указывающего, по какой границе упаковывать поля структур, есть также 
опция компилятора \TT{\#pragma pack}, её можно указывать прямо в исходнике. 
Это справедливо и для MSVC\FNURLMSDNZP и GCC\FNURLGCCPC{}.

Давайте теперь вернемся к \TT{SYSTEMTIME}, которая состоит из 16-битных полей. 
Откуда наш компилятор знает что их надо паковать по однобайтной границе?

В файле \TT{WinNT.h} попадается такое:

\begin{lstlisting}[caption=WinNT.h,style=customc]
#include "pshpack1.h"
\end{lstlisting}

И такое:

\begin{lstlisting}[caption=WinNT.h,style=customc]
#include "pshpack4.h"                   // 4 byte packing is the default
\end{lstlisting}

Сам файл PshPack1.h выглядит так:

\lstinputlisting[caption=PshPack1.h,style=customc]{patterns/15_structs/4_packing/tmp1.c}

Собственно, так и задается компилятору, как паковать объявленные после \TT{\#pragma pack} структуры.

\input{patterns/15_structs/4_packing/olly_RU.tex}

\input{patterns/15_structs/4_packing/ARM_RU}
\input{patterns/15_structs/4_packing/MIPS_RU}

\subsubsection{Еще кое-что}

Передача структуры как аргумент функции (вместо передачи указателя на структуру) это то же
что и передача всех полей структуры по одному.

Если поля в структуре пакуются по умолчанию, то функцию f() можно переписать так:

\begin{lstlisting}[style=customc]
void f(char a, int b, char c, int d)
{
    printf ("a=%d; b=%d; c=%d; d=%d\n", a, b, c, d);
};
\end{lstlisting}

И в итоге будет такой же код.
