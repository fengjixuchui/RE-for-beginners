\chapter{読むべき本/ブログ}

\mysection{本と他の資料}

\subsection{リバースエンジニアリング}

\input{RE_books}

% TBT
% (Outdated, but still interesting) Pavol Cerven, \emph{Crackproof Your Software: Protect Your Software Against Crackers}, (2002).

そして、Kris Kasperskyの本も。

\subsection{Windows}

\input{Win_reading}

\subsection{\CCpp}

\input{CCppBooks}

\subsection{x86 / x86-64}

\label{x86_manuals}
\begin{itemize}
\item Intelマニュアル\footnote{\AlsoAvailableAs \url{http://www.intel.com/content/www/us/en/processors/architectures-software-developer-manuals.html}}

\item AMDマニュアル\footnote{\AlsoAvailableAs \url{http://developer.amd.com/resources/developer-guides-manuals/}}

\item \AgnerFog{}\footnote{\AlsoAvailableAs \url{http://agner.org/optimize/microarchitecture.pdf}}

\item \AgnerFogCC{}\footnote{\AlsoAvailableAs \url{http://www.agner.org/optimize/calling_conventions.pdf}}

\item \IntelOptimization

\item \AMDOptimization
\end{itemize}

やや時代遅れですが、それでも興味深く読めます。

\MAbrash\footnote{\AlsoAvailableAs \url{https://github.com/jagregory/abrash-black-book}}
(彼は、Windows NT 3.1やid Quakeなどのプロジェクトのための低レベルの最適化に関する仕事で知られています。)

\subsection{ARM}

\begin{itemize}
\item ARMマニュアル\footnote{\AlsoAvailableAs \url{http://infocenter.arm.com/help/index.jsp?topic=/com.arm.doc.subset.architecture.reference/index.html}}

\item \ARMSevenRef

\item \ARMSixFourRefURL

\item \ARMCookBook\footnote{\AlsoAvailableAs \url{http://go.yurichev.com/17273}}
\end{itemize}

\subsection{アセンブリ言語}

Richard Blum --- Professional Assembly Language.

\subsection{Java}

\JavaBook.

\subsection{UNIX}

\TAOUP

\subsection{プログラミング一般}

\begin{itemize}

\item \RobPikePractice

\item \HenryWarren.
本からのトリックやハックは、分岐命令が高価である\ac{RISC} \ac{CPU}にのみ適していたので、
今日は関係ないと言う人もいます。 
それにもかかわらず、これらはブール代数とそれに近いすべての数学を理解するために非常に役立ちます。

\end{itemize}

% subsection:
\input{crypto_reading}
