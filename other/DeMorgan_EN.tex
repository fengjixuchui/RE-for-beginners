\subsection{De Morgan's laws and decompilation}
\myindex{De Morgan's laws}

Sometimes a compiler's optimizer can use De Morgan's laws to make code shorter/faster.

For example, this:

\begin{lstlisting}[style=customc]
void f(int a, int b, int c, int d)
{
	if (a>0 && b>0)
		printf ("both a and b are positive\n");
	else if (c>0 && d>0)
		printf ("both c and d are positive\n");
	else
		printf ("something else\n");
};
\end{lstlisting}

... looks pretty innocent, when compiled by optimizing GCC 5.4.0 x64:

\begin{lstlisting}[style=customasmx86]
; \verb|int __fastcall f(int a, int b, int c, int d)|
                public f
f               proc near
                test    edi, edi
                jle     short loc_8
                test    esi, esi
                jg      short loc_30

loc_8:
                test    edx, edx
                jle     short loc_20
                test    ecx, ecx
                jle     short loc_20
                mov     edi, offset s   ; "both c and d are positive"
                jmp     puts

loc_20:
                mov     edi, offset aSomethingElse ; "something else"
                jmp     puts

loc_30:
                mov     edi, offset aAAndBPositive ; "both a and b are positive"

loc_35:
                jmp     puts
f               endp
\end{lstlisting}

... also looks innocent, but Hex-Rays 2.2.0 cannot clearly see that both AND operations were actually used in the source code:

\begin{lstlisting}[style=customc]
int __fastcall f(int a, int b, int c, int d)
{
  int result;

  if ( a > 0 && b > 0 )
  {
    result = puts("both a and b are positive");
  }
  else if ( c <= 0 || d <= 0 )
  {
    result = puts("something else");
  }
  else
  {
    result = puts("both c and d are positive");
  }
  return result;
}
\end{lstlisting}

The \verb|c <= 0 || d <= 0| expression is inversion of \verb|c>0 && d>0| since 
$\overline{A \cup B} = \overline{A} \cap \overline{B}$ and
$\overline{A \cap B} = \overline{A} \cup \overline{B}$,
in other words,
\verb~!(cond1 || cond2) == !cond1 && !cond2~ and \verb~!(cond1 && cond2) == !cond1 || !cond2~.

These rules are worth to be kept in mind, since this compiler optimization is used heavily almost everywhere.

Sometimes it's good idea to invert a condition, in order to understand a code better.
This is a piece of a real code decompiled by Hex-Rays:

\begin{lstlisting}[style=customc]
	for (int i=0; i<12; i++)
	{
		if (v1[i-12] != 0.0 || v1[i] != 0.0)
		{
			v108=min(v108, (float)v0[i*24 -2]);
			v113=max(v113, (float)v0[i*24]);
		};
	}
\end{lstlisting}

... it can be rewritten like:

\begin{lstlisting}[style=customc]
	for (int i=0; i<12; i++)
	{
		if (v1[i-12] == 0.0 && v1[i] == 0.0)
			continue;

		v108=min(v108, (float)v0[i*24 -2]);
		v113=max(v113, (float)v0[i*24]);
	}
\end{lstlisting}

Which is better? I don't know yet, but for better understanding, it's great to take a look on both.

