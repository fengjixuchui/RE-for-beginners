\mysection{Arguments passing methods (calling conventions)}
\label{sec:callingconventions}

\subsection{cdecl}
\myindex{cdecl}
\label{cdecl}

This is the most popular method for passing arguments to functions in the \CCpp languages.

The gls{caller} also must return the value of the \gls{stack pointer} (\ESP) to its initial state after the \gls{callee} function exits.

\begin{lstlisting}[caption=cdecl,style=customasmx86]
push arg3
push arg2
push arg1
call function
add esp, 12 ; returns ESP
\end{lstlisting}

\subsection{stdcall}
\label{sec:stdcall}
\myindex{stdcall}

\newcommand{\SIZEOFINT}{The size of an \Tint type variable is 4 in x86 systems and 8 in x64 systems}

It's almost the same as \emph{cdecl}, with the exception that the \gls{callee} must set \ESP to the initial state by executing the \TT{RET x} instruction instead of \RET, \\
where \TT{x = arguments number * sizeof(int)\footnote{\SIZEOFINT}}.
The \gls{caller} is not adjusting the \gls{stack pointer}, 
there are no \TT{add esp, x} instruction.

\begin{lstlisting}[caption=stdcall,style=customasmx86]
push arg3
push arg2
push arg1
call function

function:
... do something ...
ret 12
\end{lstlisting}

The method is ubiquitous in win32 standard libraries, but not in win64 (see below about win64).

\par For example, we can take the function from \myref{src:passing_arguments_ex} and change it slightly by adding the \TT{\_\_stdcall} modifier:

\begin{lstlisting}[style=customc]
int __stdcall f2 (int a, int b, int c)
{
	return a*b+c;
};
\end{lstlisting}

It is to be compiled in almost the same way as \myref{src:passing_arguments_ex_MSVC_cdecl}, but you will see \TT{RET 12} instead of \TT{RET}.
\ac{SP} is not updated in the \gls{caller}.

As a consequence, 
the number of function arguments can be easily deduced from the \TT{RETN n} instruction: just divide $n$ by 4.

\lstinputlisting[caption=MSVC 2010,style=customasmx86]{OS/calling_conventions/stdcall_ex.asm}

\subsubsection{Functions with variable number of arguments}

\printf-like functions are, probably, the only case of functions with a variable number of arguments in \CCpp,
but it is easy to illustrate an important difference between \emph{cdecl} and \emph{stdcall} with their help.
Let's start with the idea that the compiler knows the argument count of each \printf function call.

However, the called \printf, which is already compiled and located in MSVCRT.DLL (if we talk about Windows),
does not have any information about how much arguments were passed, however it can determine it from the format string.

Thus, if \printf would be a \emph{stdcall} function and restored \gls{stack pointer} to its initial state by counting
the number of arguments in the format string, this could be a dangerous situation, when one programmer's typo can
provoke a sudden program crash.
Thus it is not suitable for such functions to use \emph{stdcall}, \emph{cdecl} is better.

\subsection{fastcall}
\label{fastcall}
\myindex{fastcall}

That's the general naming for the method of passing some arguments via registers and the 
rest via the stack. It worked faster than \emph{cdecl}/\emph{stdcall} on older CPUs 
(because of smaller stack pressure).
It may not help to gain any significant performance on latest (much more complex) \ac{CPU}s, however.

It is not standardized, so the various compilers can do it differently.
It's a well known caveat: if you have two DLLs and the one uses another one, and they are built by different compilers with 
different \emph{fastcall} calling conventions, you can expect problems.

Both MSVC and GCC pass the first and second arguments via \ECX and \EDX and the rest of the arguments via the stack.

The \gls{stack pointer} must be restored to its initial state by the \gls{callee} (like in \emph{stdcall}).

\begin{lstlisting}[caption=fastcall,style=customasmx86]
push arg3
mov edx, arg2
mov ecx, arg1
call function

function:
.. do something ..
ret 4
\end{lstlisting}

For example, we may take the function from \myref{src:passing_arguments_ex} and change it slightly by adding a \TT{\_\_fastcall} modifier:

\begin{lstlisting}[style=customc]
int __fastcall f3 (int a, int b, int c)
{
	return a*b+c;
};
\end{lstlisting}

Here is how it is to be compiled:

\lstinputlisting[caption=\Optimizing MSVC 2010 /Ob0,style=customasmx86]{OS/calling_conventions/fastcall_ex.asm}

We see that the \gls{callee} returns \ac{SP} by using the \TT{RETN} instruction with an operand.

Which implies that the number of arguments can be deduced easily here as well.

\subsubsection{GCC regparm}

\newcommand{\URLREGPARMM}{\url{http://go.yurichev.com/17040}}

It is the evolution of \emph{fastcall}\footnote{\URLREGPARMM} in some sense.
With the \TT{-mregparm} option it is possible to set how many arguments are to be passed via registers (3 is the maximum).
Thus, the \EAX, \EDX and \ECX registers are to be used.

Of course, if the number the of arguments is less than 3, not all 3 registers are to be used.

The \gls{caller} restores the \gls{stack pointer} to its initial state.

For example, see (\myref{regparm}).

\subsubsection{Watcom/OpenWatcom}
\myindex{OpenWatcom}

Here it is called \q{register calling convention}.
The first 4 arguments are passed via the \EAX, \EDX, \EBX and \ECX registers.
All the rest---via the stack.

These functions have an underscore appended to the function name in order to distinguish them from 
those having a different calling convention.

\subsection{thiscall}
\myindex{thiscall}

This is passing the object's \ITthis pointer to the function-method, in \Cpp.

In MSVC, \ITthis is usually passed in the \ECX register.

In GCC, the \ITthis pointer is passed as the first function-method argument.
Thus it will be visible that all functions in assembly code have an extra argument, in comparison with the source code.

For an example, see (\myref{thiscall}).

\subsection{x86-64}
\myindex{x86-64}

\subsubsection{Windows x64}
\label{sec:callingconventions_win64}

The method of for passing arguments in Win64 somewhat resembles \TT{fastcall}.
The first 4 arguments are passed via \RCX, \RDX, \Reg{8} and \Reg{9}, the rest---via the stack.
The \gls{caller} also must prepare space for 32 bytes or 4 64-bit values,
so then the \gls{callee} can save there the first 4 arguments.
Short functions may use the arguments' values just from the registers,
but larger ones may save their values for further use.

The \gls{caller} also must return the \gls{stack pointer} into its initial state.

This calling convention is also used in Windows x86-64 system DLLs 
(instead of \emph{stdcall} in win32).

Example:

\lstinputlisting[style=customc]{OS/calling_conventions/x64.c}

\lstinputlisting[caption=MSVC 2012 /0b,style=customasmx86]{OS/calling_conventions/x64_MSVC_Ob.asm}

\myindex{Scratch space}

Here we clearly see how 7 arguments are passed: 4 via registers and the remaining 3 via the stack.

The code of the f1() function's prologue saves the arguments in the \q{scratch space}---a space in the stack
intended exactly for this purpose.

This is arranged so because the compiler cannot be sure that there will be enough registers to use without these 4,
which will otherwise be occupied by the arguments until the function's execution end.

The \q{scratch space} allocation in the stack is the caller's duty.

\lstinputlisting[caption=\Optimizing MSVC 2012 /0b,style=customasmx86]{OS/calling_conventions/x64_MSVC_Ox_Ob.asm}

If we compile the example with optimizations, it is to be almost the same, 
but the \q{scratch space} will not be used, because it won't be needed.

\myindex{x86!\Instructions!LEA}
\label{using_MOV_and_pack_of_LEA_to_load_values}

Also take a look on how MSVC 2012 optimizes the loading of primitive values into registers by using \LEA (\myref{sec:LEA}).
\INS{MOV} would be 1 byte longer here (5 instead of 4).

Another example of such thing is: \myref{TaskMgr_LEA}.

\myparagraph{Windows x64: Passing \ITthis (\CCpp)}

The \ITthis pointer is passed in \RCX, the first argument of the method is in \RDX, etc.
For an example see: \myref{simple_CPP_MSVC_x64}.
 
\subsubsection{Linux x64}

The way arguments are passed in Linux for x86-64 is almost the same as in Windows, but 6 registers are
used instead of 4 (\RDI, \RSI, \RDX, \RCX, \Reg{8}, \Reg{9}) and there is no \q{scratch space}, 
although the \gls{callee} may save the register values in the stack, if it needs/wants to.

\lstinputlisting[caption=\Optimizing GCC 4.7.3,style=customasmx86]{OS/calling_conventions/x64_linux_O3.s}

\myindex{AMD}

N.B.: here the values are written into the 32-bit parts of the registers (e.g., EAX) but not in the whole 64-bit 
register (RAX).
This is because each write to the low 32-bit part of a register automatically clears the high 32 bits.
Supposedly, it was decided in AMD to do so to simplify porting code to x86-64.

\subsection{Return values of \Tfloat and \Tdouble type}
\myindex{float}
\myindex{double}

In all conventions except in Win64, the values of type \Tfloat or \Tdouble are returned via the FPU register \ST{0}.

In Win64, the values of \Tfloat and \Tdouble types are returned 
in the low 32 or 64 bits of the \XMM{0} register.

\subsection{Modifying arguments}

Sometimes, \CCpp{} programmers (not limited to these \ac{PL}s, though),
may ask, what can happen if they modify the arguments?

The answer is simple: the arguments are stored in the stack, 
that is where the modification takes place.

The calling functions is not using them after the \gls{callee}'s exit (the author of these lines has never seen any such case in his practice).

\lstinputlisting[style=customc]{OS/calling_conventions/change_arguments.c}

\lstinputlisting[caption=MSVC 2012,style=customasmx86]{OS/calling_conventions/change_arguments.asm}

% TODO (OllyDbg) пример как в стеке меняется $a$

So yes, one can modify the arguments easily.
Of course, if it is not \emph{references} in \Cpp{} (\myref{cpp_references}),
and if you don't modify data to which a pointer points to, 
then the effect will not propagate outside the current function.

Theoretically, after the \gls{callee}'s return, 
the \gls{caller} could get the modified argument and use it somehow.
Maybe if it is written directly in assembly language.

For example, code like this will be generated by usual \CCpp compiler:

\begin{lstlisting}[style=customasmx86]
	push	456	; will be b
	push	123	; will be a
	call	f	; f() modifies its first argument
	add	esp, 2*4
\end{lstlisting}

We can rewrite this code like:

\begin{lstlisting}[style=customasmx86]
	push	456	; will be b
	push	123	; will be a
	call	f	; f() modifies its first argument
	pop	eax
	add	esp, 4
	; EAX=1st argument of f() modified in f()
\end{lstlisting}

Hard to imagine, why anyone would need this, but this is possible in practice.
Nevertheless, the \CCpp languages standards don't offer any way to do so.

% subsections
% TODO translate
\mysection{Breaking simple executable cryptor}

I've got an executable file which is encrypted by relatively simple encryption.
\href{\GitHubBlobMasterURL/examples/simple_exec_crypto/files/cipher.bin}{Here is it} (only executable section is left here).

First, all encryption function does is just adds number of position in buffer to the byte.
Here is how this can be encoded in Python:

\begin{lstlisting}[caption=Python script,style=custompy]
#!/usr/bin/env python
def e(i, k):
    return chr ((ord(i)+k) % 256)

def encrypt(buf):
    return e(buf[0], 0)+ e(buf[1], 1)+ e(buf[2], 2) + e(buf[3], 3)+ e(buf[4], 4)+ e(buf[5], 5)+ e(buf[6], 6)+ e(buf[7], 7)+
           e(buf[8], 8)+ e(buf[9], 9)+ e(buf[10], 10)+ e(buf[11], 11)+ e(buf[12], 12)+ e(buf[13], 13)+ e(buf[14], 14)+ e(buf[15], 15)
\end{lstlisting}

Hence, if you encrypt buffer with 16 zeros, you'll get \emph{0, 1, 2, 3 ... 12, 13, 14, 15}.

\myindex{Propagating Cipher Block Chaining}
Propagating Cipher Block Chaining (PCBC) is also used, here is how it works:

\begin{figure}[H]
\centering
\myincludegraphics{examples/simple_exec_crypto/601px-PCBC_encryption.png}
\caption{Propagating Cipher Block Chaining encryption (image is taken from Wikipedia article)}
\end{figure}

The problem is that it's too boring to recover IV (Initialization Vector) each time.
Brute-force is also not an option, because IV is too long (16 bytes).
Let's see, if it's possible to recover IV for arbitrary encrypted executable file?

Let's try simple frequency analysis.
This is 32-bit x86 executable code, so let's gather statistics about most frequent bytes and opcodes.
I tried huge oracle.exe file from Oracle RDBMS version 11.2 for windows x86 and I've found that the most frequent byte (no surprise) is zero (~10\%).
The next most frequent byte is (again, no surprise) 0xFF (~5\%).
The next is 0x8B (~5\%).

\myindex{x86!\Instructions!MOV}
0x8B is opcode for \INS{MOV}, this is indeed one of the most busy x86 instructions.
Now what about popularity of zero byte?
If compiler needs to encode value bigger than 127, it has to use 32-bit displacement instead of 8-bit one, but large values are very rare,
so it is padded by zeros.
\myindex{x86!\Instructions!LEA}
\myindex{x86!\Instructions!PUSH}
\myindex{x86!\Instructions!CALL}
This is at least in \INS{LEA}, \INS{MOV}, \INS{PUSH}, \INS{CALL}.

For example:

\begin{lstlisting}[style=customasmx86]
8D B0 28 01 00 00                 lea     esi, [eax+128h]
8D BF 40 38 00 00                 lea     edi, [edi+3840h]
\end{lstlisting}

Displacements bigger than 127 are very popular, but they are rarely exceeds 0x10000
(indeed, such large memory buffers/structures are also rare).

Same story with \INS{MOV}, large constants are rare, the most heavily used are 0, 1, 10, 100, $2^n$, and so on.
Compiler has to pad small constants by zeros to represent them as 32-bit values:

\begin{lstlisting}[style=customasmx86]
BF 02 00 00 00                    mov     edi, 2
BF 01 00 00 00                    mov     edi, 1
\end{lstlisting}

Now about 00 and FF bytes combined: jumps (including conditional) and calls can pass execution flow forward or backwards, but very often,
within the limits of the current executable module.
If forward, displacement is not very big and also padded with zeros.
If backwards, displacement is represented as negative value, so padded with FF bytes.
For example, transfer execution flow forward:

\begin{lstlisting}[style=customasmx86]
E8 43 0C 00 00                    call    _function1
E8 5C 00 00 00                    call    _function2
0F 84 F0 0A 00 00                 jz      loc_4F09A0
0F 84 EB 00 00 00                 jz      loc_4EFBB8
\end{lstlisting}

Backwards:

\begin{lstlisting}[style=customasmx86]
E8 79 0C FE FF                    call    _function1
E8 F4 16 FF FF                    call    _function2
0F 84 F8 FB FF FF                 jz      loc_8212BC
0F 84 06 FD FF FF                 jz      loc_FF1E7D
\end{lstlisting}

FF byte is also very often occurred in negative displacements like these:

\begin{lstlisting}[style=customasmx86]
8D 85 1E FF FF FF                 lea     eax, [ebp-0E2h]
8D 95 F8 5C FF FF                 lea     edx, [ebp-0A308h]
\end{lstlisting}

So far so good. Now we have to try various 16-byte keys, decrypt executable section and measure how often 00, FF and 8B bytes are occurred.
Let's also keep in sight how PCBC decryption works:

\begin{figure}[H]
\centering
\myincludegraphics{examples/simple_exec_crypto/640px-PCBC_decryption.png}
\caption{Propagating Cipher Block Chaining decryption (image is taken from Wikipedia article)}
\end{figure}

The good news is that we don't really have to decrypt whole piece of data, but only slice by slice, this is exactly how I did in my previous example: \myref{XOR_mask_2}.

Now I'm trying all possible bytes (0..255) for each byte in key and just pick the byte producing maximal amount of 00/FF/8B bytes in a decrypted slice:

\begin{lstlisting}[style=custompy]
#!/usr/bin/env python
import sys, hexdump, array, string, operator

KEY_LEN=16

def chunks(l, n):
    # split n by l-byte chunks
    # https://stackoverflow.com/q/312443
    n = max(1, n)
    return [l[i:i + n] for i in range(0, len(l), n)]

def read_file(fname):
    file=open(fname, mode='rb')
    content=file.read()
    file.close()
    return content

def decrypt_byte (c, key):
    return chr((ord(c)-key) % 256)

def XOR_PCBC_step (IV, buf, k):
    prev=IV
    rt=""
    for c in buf:
	new_c=decrypt_byte(c, k)
        plain=chr(ord(new_c)^ord(prev))
	prev=chr(ord(c)^ord(plain))
	rt=rt+plain
    return rt

each_Nth_byte=[""]*KEY_LEN

content=read_file(sys.argv[1])
# split input by 16-byte chunks:
all_chunks=chunks(content, KEY_LEN)
for c in all_chunks:
    for i in range(KEY_LEN):
        each_Nth_byte[i]=each_Nth_byte[i] + c[i]

# try each byte of key
for N in range(KEY_LEN):
    print "N=", N
    stat={}
    for i in range(256):
        tmp_key=chr(i)
	tmp=XOR_PCBC_step(tmp_key,each_Nth_byte[N], N)
        # count 0, FFs and 8Bs in decrypted buffer:
	important_bytes=tmp.count('\x00')+tmp.count('\xFF')+tmp.count('\x8B')
	stat[i]=important_bytes
    sorted_stat = sorted(stat.iteritems(), key=operator.itemgetter(1), reverse=True)
    print sorted_stat[0]
\end{lstlisting}

(Source code can be downloaded \href{\GitHubBlobMasterURL/examples/simple_exec_crypto/files/decrypt.py}{here}.)

I run it and here is a key for which 00/FF/8B bytes presence in decrypted buffer is maximal:

\begin{lstlisting}
N= 0
(147, 1224)
N= 1
(94, 1327)
N= 2
(252, 1223)
N= 3
(218, 1266)
N= 4
(38, 1209)
N= 5
(192, 1378)
N= 6
(199, 1204)
N= 7
(213, 1332)
N= 8
(225, 1251)
N= 9
(112, 1223)
N= 10
(143, 1177)
N= 11
(108, 1286)
N= 12
(10, 1164)
N= 13
(3, 1271)
N= 14
(128, 1253)
N= 15
(232, 1330)
\end{lstlisting}

Let's write decryption utility with the key we got:

\begin{lstlisting}[style=custompy]
#!/usr/bin/env python
import sys, hexdump, array

def xor_strings(s,t):
    # \verb|https://en.wikipedia.org/wiki/XOR_cipher#Example_implementation|
    """xor two strings together"""
    return "".join(chr(ord(a)^ord(b)) for a,b in zip(s,t))

IV=array.array('B', [147, 94, 252, 218, 38, 192, 199, 213, 225, 112, 143, 108, 10, 3, 128, 232]).tostring()

def chunks(l, n):
    n = max(1, n)
    return [l[i:i + n] for i in range(0, len(l), n)]

def read_file(fname):
    file=open(fname, mode='rb')
    content=file.read()
    file.close()
    return content

def decrypt_byte(i, k):
    return chr ((ord(i)-k) % 256)

def decrypt(buf):
    return "".join(decrypt_byte(buf[i], i) for i in range(16))

fout=open(sys.argv[2], mode='wb')

prev=IV
content=read_file(sys.argv[1])
tmp=chunks(content, 16)
for c in tmp:
    new_c=decrypt(c)
    p=xor_strings (new_c, prev)
    prev=xor_strings(c, p)
    fout.write(p)
fout.close()
\end{lstlisting}

(Source code can be downloaded \href{\GitHubBlobMasterURL/examples/simple_exec_crypto/files/decrypt2.py}{here}.)

Let's check resulting file:

\lstinputlisting{examples/simple_exec_crypto/objdump_result.txt}

Yes, this is seems correctly disassembled piece of x86 code.
The whole decryped file can be downloaded \href{\GitHubBlobMasterURL/examples/simple_exec_crypto/files/decrypted.bin}{here}.

In fact, this is text section from regedit.exe from Windows 7.
But this example is based on a real case I encountered, so just executable is different (and key), algorithm is the same.

\subsection{Other ideas to consider}

What if I would fail with such simple frequency analysis?
There are other ideas on how to measure correctness of decrypted/decompressed x86 code:

\begin{itemize}

\item Many modern compilers aligns functions on 0x10 border.
So the space left before is filled with NOPs (0x90) or other NOP instructions with known opcodes: \myref{sec:npad}.

\item Perhaps, the most frequent pattern in any assembly language is function call:\\
\TT{PUSH chain / CALL / ADD ESP, X}.
This sequence can easily detected and found.
I've even gathered statistics about average number of function arguments: \myref{args_stat}.
(Hence, this is average length of PUSH chain.)

\end{itemize}

Read more about incorrectly/correctly disassembled code: \myref{ISA_detect}.

\input{OS/calling_conventions/ctypes_EN}
\subsection{Cdecl example: a DLL}
\label{cdecl_DLL}

Let's back to the fact that this is not very important how to declare the \verb|main()| function: \myref{main_arguments}.

This is a real story: once upon a time I wanted to replace an original DLL file in some software by mine.
First I enumerated names of all DLL exports and made a function in my own replacement DLL for each function in the original DLL, like:

\begin{lstlisting}[style=customc]
void function1 ()
{
	write_to_log ("function1() called\n");
};
\end{lstlisting}

I wanted to see, which functions are called during run, and when.
However, I was in hurry and had no time to deduce arguments count for each function, let alone data types.
So each function in my replacement DLL had no argumnts whatsoever.
But everything worked, because all functions had \emph{cdecl} calling convention.
(It wouldn't work if functions had \emph{stdcall} calling convention.)
It also worked for x64 version.

And then I did a next step: I deduced argument types for some functions.
But I made several mistakes, for example, the original function took 3 arguments, but I knew only about 2, etc.

Still, it worked.
At the beginning, my replacement DLL just ignored all arguments.
Then, it ignored the 3rd argument.



