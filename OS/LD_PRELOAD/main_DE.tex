\subsection{\emph{LD\_PRELOAD}-Hack in Linux}

\myindex{LD\_PRELOAD}
\label{ld_preload}

Diese Technik erlaubt es eigene, dynamische Bibliotheken vor anderen zu laden\dots{}
sogar vor denen des Systems, wie libc.so.6.

Dies wiederum erlaubt es die eigenen Funktionen für die des Systems zu \q{ersetzen}.
Es ist zum Beispiel einfach alle Aufrufe zu time(), read(), write(), usw. abzufangen.

\myindex{uptime}
Sehen wir uns einmal an, wie das Tool \emph{uptime} ausgetrickst werden kann.
Wie bekannt ist, zeigt dieses Programm an, wie lange der Computer schon arbeitet.
\myindex{strace}
Mithilfe von strace(\myref{strace}), ist es möglich zu sehen, dass das Tool die
Informationen aus der Datei \TT{/proc/uptime} ausliest:

\begin{lstlisting}
$ strace uptime 
...
open("/proc/uptime", O_RDONLY)          = 3
lseek(3, 0, SEEK_SET)                   = 0
read(3, "416166.86 414629.38\n", 2047)  = 20
...
\end{lstlisting}

Es handelt sich dabei nicht um eine reale Datei auf der Festplatte sondern um eine
virtuelle, bei der die Dateien on-the-fly im Linux Kernel erstellt werden.
Es gibt zwei Zahlen:

\begin{lstlisting}
$ cat /proc/uptime
416690.91 415152.03
\end{lstlisting}

Nachfolgend, der englischen Wikipedia\footnote{\url{https://en.wikipedia.org/wiki/Uptime}}:

\begin{framed}
\begin{quotation}
The first number is the total number of seconds the system has been up.
The second number is how much of that time the machine has spent idle, in seconds.
\end{quotation}
\end{framed}

\myindex{\CStandardLibrary!open()}
\myindex{\CStandardLibrary!read()}
\myindex{\CStandardLibrary!close()}

Versuchen wir eine eigene dynamische Bibliothek mit den Funktionen open(), read()
und close() zu schreiben die so funktionieren wie wir es gerne hätten.

Zunächst wird unsere open()-Funktion den Namen der zu öffnenden Datei mit dem was
wir brauchen vergleichen. Ist dies der Fall, soll der Deskriptor der geöffnete Datei
geschrieben werden.

Als zweites read(): Wenn diese Funktion für den Datei-Deskriptor aufgerufen wird,
soll die Ausgaben ersetzt werden und der Rest dem original read() aus libc.so.6
entsprechen.
close() wird eine Meldung geben wenn die Datei der zur Zeit gefolgt wird geschlossen
wurde.

\myindex{dlopen()}
\myindex{dlsym()}

Wir werden die dlopen()- und dlsym()-Funktionen nutzen, um die Adressen der Original-Funktionen
in libc.so.6 herauszufinden.

Diese werden benötigt, weil die Ausführkontrolle wieder an die \q{realen} Funktionen
übergeben werden müssen.

\myindex{\CStandardLibrary!strcmp()}

Auf der anderen Seite: wenn wir strcmp() unterbrechen und jeden einzelnen Vergleich
von Zeichenketten im Programm untersuchen, müssten wir eine eigene strcmp()-Variante
schreiben und nicht die Original-Funktion
nutzen\footnote{Als Beispiel, wie einfach strcmp()-Unterbrechung funktioniert
\footnote{\href{http://go.yurichev.com/17143}{yurichev.com}} von Yong Huang}, was
einfacher wäre.

\lstinputlisting[style=customc]{OS/LD_PRELOAD/fool_uptime.c}
( \href{\GitHubBlobMasterURL/OS/LD_PRELOAD/fool_uptime.c}{Quellcode auf GitHub} )
% FIXME go.yurichev.com...

Kompilieren wir den Code als gemeinsame, dynamische Bibliothek:

\begin{lstlisting}
gcc -fpic -shared -Wall -o fool_uptime.so fool_uptime.c -ldl
\end{lstlisting}

Jetzt starten wir \emph{uptime} während unsere Bibliothek vor den anderen geladen wird:

\begin{lstlisting}
LD_PRELOAD=`pwd`/fool_uptime.so uptime
\end{lstlisting}

Und wir sehen:

\begin{lstlisting}
 01:23:02 up 24855 days,  3:14,  3 users,  load average: 0.00, 0.01, 0.05
\end{lstlisting}

Wenn die \emph{LD\_PRELOAD}-Umgebungsvariable immer auf den Dateinamen und -pfad unserer
Bibliothek zeigt, wird diese vor allen anderen gestarteten Programmen geladen.

Weitere Beispiele:

\begin{itemize}

\item
Sehr einfache Unterbrechung von strcmp() (Yong Huang)
\url{http://go.yurichev.com/17143}

\item
Kevin Pulo---Fun with LD\_PRELOAD. Viele Beispiele und Ideen.
\href{http://go.yurichev.com/17145}{yurichev.com}

\item
Datei-Funktionen unterbrechen beim Komprimieren/Entkomprimieren on-the-fly (zlibc). \url{http://go.yurichev.com/17146}
\end{itemize}
