\subsubsection{Utilisation de pointeurs indirects}

\begin{lstlisting}[style=customasmx86]
dummy_data1	db	100h dup (0)
message1	db	'hello world',0

dummy_data2	db	200h dup (0)
message2	db	'another message',0

func		proc
		...
		mov	eax, offset dummy_data1 ; PE or ELF reloc here
		add	eax, 100h
		push	eax
		call	dump_string
		...
		mov	eax, offset dummy_data2 ; PE or ELF reloc here
		add	eax, 200h
		push	eax
		call	dump_string
		...
func		endp
\end{lstlisting}

\IDA{} montrera seulement les références à \TT{dummy\_data1} et \TT{dummy\_data2},
mais pas aux chaînes de textes.

Les variables globales et même les fonctions peuvent être accédées comme ça.

Maintenant, quelque chose de légèrement plus avancé.

Franchement, je ne connais pas con nom exact, mais je vais l'appeler \textit{pointeur décalés}.
Cette technique est assez commune, au moins dans les systèmes de protection contre
la copie.

En bref: lorsque vous écrivez une valeur dans la mémoire globale, vous utilisez
une adresse, mais lorsque vous lisez, vous utilisez la somme d'une (autre) adresse,
ou peut-être une différence.
Le but est de cacher l'adresse réelle au rétro-ingénieur qui débogue le code ou
l'explore dans \IDA (ou un autre désassembleur).

Ceci peut être pénible.

\lstinputlisting[style=customc]{\CURPATH/indirect.c}

Compiler avec MSVC 2015 sans optimisation:

\lstinputlisting[style=customasmx86]{\CURPATH/indirect.asm}

Vous voyez, \IDA obtient seulement deux adresses: \verb|secret_array[]| (début du
tableau) et \verb|point_passed_to_get_byte_at_0x6000|.

Comment s'y prendre: vous pouvez utiliser les point d'arrêt matériel sur les opérations
d'accès en mémoire, \tracer possède l'option \TT{BPMx}) ou des moteurs d'exécution
symbolique ou peut-être écrire un module pour \IDA\dots

C'est sûr, un tableau peut être utiliser pour des nombreuses valeurs, non limité
aux booléennes\dots

N.B.: MSVC 2015 avec optimisation est assez malin pour optimiser la fonction \verb|get_byte_at_0x6000()|.

