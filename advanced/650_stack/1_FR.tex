\subsection{Accéder aux arguments/variables locales de l'appelant}

Des bases de \CCpp, nous savons qu'il est impossible à une fonction d'accéder aux
arguments de la fonction appelante ou à ses variables locales.

Néanmoins, c'est possible en utilisant des astuces tordues.
Par exemple:

\begin{lstlisting}[style=customc]
#include <stdio.h>

void f(char *text)
{
	// print stack
	int *tmp=&text;
	for (int i=0; i<20; i++)
	{
		printf ("0x%x\n", *tmp);
		tmp++;
	};
};

void draw_text(int X, int Y, char* text)
{
	f(text);

	printf ("We are going to draw [%s] at %d:%d\n", text, X, Y);
};

int main()
{
	printf ("address of main()=0x%x\n", &main);
	printf ("address of draw_text()=0x%x\n", &draw_text);
	draw_text(100, 200, "Hello!");
};
\end{lstlisting}

Sur Ubuntu 32-bit avec GCC 5.4.0, j'obtiens ceci:

\begin{lstlisting}
address of main()=0x80484f8
address of draw_text()=0x80484cb
0x8048645	first argument to f()
0x8048628
0xbfd8ab98
0xb7634590
0xb779eddc
0xb77e4918
0xbfd8aba8
0x8048547	return address into the middle of main()
0x64		first argument to draw_text()
0xc8		second argument to draw_text()
0x8048645	third argument to draw_text()
0x8048581
0xb779d3dc
0xbfd8abc0
0x0
0xb7603637
0xb779d000
0xb779d000
0x0
0xb7603637
\end{lstlisting}

(Les commentaires sont miens.)

Puisque \emph{f()} commence à énumérer les éléments de la pile à son premier argument,
le premier élément de la pile est en effet un pointeur sur la chaîne \q{Hello!}.
Nous voyons que son adresse est aussi utilisée comme troisième argument de la fonction
\emph{draw\_text()}.

Dans \emph{f()} nous pouvons lire tous les arguments des fonctions et les variables
locales si nous connaissons exactement l'agencement de la pile, mais ça change toujours
d'un compilateur à l'autre.
Des niveaux d'optimisation différents modifient grandement la structure de la pile.

Mais, si nous pouvons d'une manière ou d'une autre détecter l'information dont nous
avons besoin, nous pouvons l'utiliser et même la modifier.
À titre d'exemple, j'ai retravaillé la fonction \emph{f()}:

\begin{lstlisting}[style=customc]
void f(char *text)
{
	...

	// find 100, 200 values pair and modify the second on
	tmp=&text;
	for (int i=0; i<20; i++)
	{
		if (*tmp==100 && *(tmp+1)==200)
		{
			printf ("found\n");
			*(tmp+1)=210; // change 200 to 210
			break;
		};
		tmp++;
	};
};
\end{lstlisting}

Hé mais, ça fonctionne:

\begin{lstlisting}
found
We are going to draw [Hello!] at 100:210
\end{lstlisting}

\myparagraph{Résumé}

C'est vraiment un sale hack, dont le but est de montrer l'intérieur de la pile.
Je n'ai jamais vu ni entendu dire que quelqu'un a utilisé ceci dans du code réel.
Mais encore, ceci est un bon exemple.

\myparagraph{\Exercise}

L'exemple a été compilé sans optimisation sur Ubuntu 32-bit avec GCC 5.4.0 et il
fonctionne.
Mais lorsque j'active l'optimisation maximum \TT{-O3}, ça plante.
Essayez de trouver pourquoi.

Utilisez votre compilateur et OS favori, essayez différents niveaux d'optimisation,
trouvez si ça fonctionne et si ça ne fonctionne pas, trouvez pourquoi.
