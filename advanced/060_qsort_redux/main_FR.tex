\section{qsort() revisité}

(Revenons au fait que l'instruction \INS{CMP} fonctionne comme \INS{SUB}: \myref{CMPandSUB}.)

\myindex{\CStandardLibrary!qsort()}
\myindex{x86!\Instructions!CMP}
\myindex{x86!\Instructions!SUB}
Maintenant que vous êtes déjà familier avec la fonction qsort() (\myref{qsort}),
voici un bel exemple où l'opération de comparaison (\INS{CMP}) peut être remplacée
par l'opération de soustraction (\INS{SUB}).

\begin{lstlisting}[style=customc]
/* fonction de comparaison qsort int */ 
int int_cmp(const void *a, const void *b) 
{ 
    const int *ia = (const int *)a; // casting de types de pointeur
    const int *ib = (const int *)b;
    return *ia  - *ib; 
	/* comparaison d'entier: renvoie négatif si if b > a 
	et positif si a > b */ 
} 
\end{lstlisting}

( \url{http://www.anyexample.com/programming/c/qsort__sorting_array_of_strings__integers_and_structs.xml} \url{http://archive.is/Hh3jz} )

\myindex{\CStandardLibrary!strcmp()}
Aussi, une implémentation typique de \verb|strcmp()| (tiré d'OpenBSD):

\begin{lstlisting}[style=customc]
int
strcmp(const char *s1, const char *s2)
{
	while (*s1 == *s2++)
		if (*s1++ == 0)
			return (0);
	return (*(unsigned char *)s1 - *(unsigned char *)--s2);
}
\end{lstlisting}
