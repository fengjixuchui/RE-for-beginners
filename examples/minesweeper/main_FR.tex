\mysection{\MinesweeperWinXPExampleChapterName}
\label{minesweeper_winxp}
\myindex{Windows!Windows XP}

Pour ceux qui ne sont pas très bons avec le jeu démineur, nous pouvons essayer de
révéler les mines cachées dans le débogueur.

\myindex{\CStandardLibrary!rand()}
\myindex{Windows!PDB}

Comme on le sait, le démineur place des mines aléatoirement, donc il doit y avoir
une sorte de générateur de nombre aléatoire ou un appel à la fonction C standard
\TT{rand()}.

Ce qui est vraiment cool en rétro-ingénierant des produits Microsoft c'est qu'il
y a les fichiers \gls{PDB} avec les symboles (nom de fonctions, etc.)
Lorsque nous chargeons \TT{winmine.exe} dans \IDA, il télécharge le fichier \gls{PDB}
exact pour cet exécutable et affiche tous les noms.

Donc le voici, le seul appel à \TT{rand()} est cette fonction:

\lstinputlisting[style=customasmx86]{examples/minesweeper/tmp1.lst}

\IDA l'a appelé ainsi, et c'est le nom que lui ont donné les développeurs du démineur.

La fonction est très simple:

\begin{lstlisting}[style=customc]
int Rnd(int limit)
{
    return rand() % limit;
};
\end{lstlisting}

(Il n'y a pas de nom \q{limit} dans le fichier \gls{PDB}; nous avons nommé manuellement
les arguments comme ceci.)

Donc elle renvoie une valeur aléatoire entre 0 et la limite spécifiée.

\TT{Rnd()} est appelée depuis un seul endroit, la fonction appelée \TT{StartGame()},
et il semble bien que ce soit exactement le code qui place les mines:

\begin{lstlisting}[style=customasmx86]
.text:010036C7                 push    _xBoxMac
.text:010036CD                 call    _Rnd@4          ; Rnd(x)
.text:010036D2                 push    _yBoxMac
.text:010036D8                 mov     esi, eax
.text:010036DA                 inc     esi
.text:010036DB                 call    _Rnd@4          ; Rnd(x)
.text:010036E0                 inc     eax
.text:010036E1                 mov     ecx, eax
.text:010036E3                 shl     ecx, 5          ; ECX=ECX*32
.text:010036E6                 test    _rgBlk[ecx+esi], 80h
.text:010036EE                 jnz     short loc_10036C7
.text:010036F0                 shl     eax, 5          ; EAX=EAX*32
.text:010036F3                 lea     eax, _rgBlk[eax+esi]
.text:010036FA                 or      byte ptr [eax], 80h
.text:010036FD                 dec     _cBombStart
.text:01003703                 jnz     short loc_10036C7
\end{lstlisting}

Le démineur vous permet de définir la taille du plateau, donc les dimensions X (xBoxMac)
et Y (yBoxMac) du plateau sont des variables globales.
Elles sont passées à \TT{Rnd()} et des coordonnées aléatoires sont générées.
Une mine est placée par l'instruction \TT{OR} en \TT{0x010036FA}.
Et si une mine y a déjà été placée avant (il est possible que la fonction \TT{Rnd()}
génère une paire de coordonnées qui a déjà été générée), alors les instructions \TT{TEST}
et \TT{JNZ} en \TT{0x010036E6} bouclent sur la routine de génération.

\TT{cBombStart} est la variable globale contenant le nombre total de mines. Donc
ceci est une boucle.

La largeur du tableau est 32 (nous pouvons conclure ceci en regardant l'instruction
\TT{SHL}, qui multiplie l'une des coordonnées par 32).

La taille du tableau global \TT{rgBlk} peut facilement être déduite par la différence
entre le label \TT{rgBlk} dans le segment de données et le label suivant. Il s'agit
de 0x360 (864):

\begin{lstlisting}[style=customasmx86]
.data:01005340 _rgBlk          db 360h dup(?)          ; DATA XREF: MainWndProc(x,x,x,x)+574
.data:01005340                                         ; DisplayBlk(x,x)+23
.data:010056A0 _Preferences    dd ?                    ; DATA XREF: FixMenus()+2
...
\end{lstlisting}

$864/32=27$.

Donc, la taille du tableau est-elle $27*32$?
C'est proche de ce que nous savons: lorsque nous essayons de définir la taille du
plateau à $100*100$ dans les préférences du démineur, il corrige à une taille de
plateau de $24*30$.
Donc ceci est la taille maximale du plateau.
Et le tableau a une taille fixe, pour toutes les tailles de plateau.

REgardons tout ceci dans \olly.
Nous allons lancer le démineur, lui attacher \olly et nous allons pouvoir voir le
contenu de la mémoire à l'adresse du tableau \TT{rgBlk} (\TT{0x01005340})\footnote{Toutes
les adresses ici sont pour le démineur de Windows XP SP3 English. Elles peuvent être
différentes pour d'autres services packs.}.
Donc nous avons ceci à l'emplacement mémoire du tableau:

\lstinputlisting[style=customasmx86]{examples/minesweeper/1.lst}

\olly, comme tout autre éditeur hexadécimal, affiche 16 octets par ligne.
Donc chaque ligne de tableau de 32-octet occupe exactement 2 lignes ici.

Ceci est le niveau débutant (plateau de 9*9).

Il y a une sorte de structure carré que l'on voit ici (octets 0x10).

Nous cliquons \q{Run} dans \olly pour débloquer le processus du démineur, puis nous
cliquons au hasard dans la fenêtre du démineur et nous tombons sur une mine, mais
maintenant, toutes les mines sont visibles:

\begin{figure}[H]
\centering
\myincludegraphicsSmall{examples/minesweeper/1.png}
\caption{Mines}
\label{fig:minesweeper1}
\end{figure}

En comparant les emplacements des mines et le dump, nous pouvons en conclure que
0x10 correspond au bord, 0x0F---bloc vide, 0x8F---mine.
Peut-être que 0x10 est simplement une \emph{valeur sentinelle}.

Maintenant nous allons ajouter des commentaires et aussi mettre tous les octets à
0x8F entre parenthèses droites:

\lstinputlisting[style=customasmx86]{examples/minesweeper/2.lst}

Maintenant nous allons supprimer tous les \emph{octet de bord} (0x10) et ce qu'il
y a après:

\lstinputlisting[style=customasmx86]{examples/minesweeper/3.lst}

Oui, ce sont des mines, maintenant ça peut être vu clairement et comparé avec la
copie d'écran.

\clearpage
Ce qui est intéressant, c'est que nous pouvons modifier le tableau directement dans
\olly.
Nous pouvons supprimer toutes les mines en changeant les octets à 0x8F par 0x0F,
et voici ce que nous obtenons dans le démineur:

\begin{figure}[H]
\centering
\myincludegraphicsSmall{examples/minesweeper/3.png}
\caption{Toutes les mines sont supprimées depuis le débogueur}
\label{fig:minesweeper3}
\end{figure}

Nous pouvons aussi toutes les déplacer à la première ligne:

\begin{figure}[H]
\centering
\myincludegraphicsSmall{examples/minesweeper/2.png}
\caption{Mines mises dans le débogueur}
\label{fig:minesweeper2}
\end{figure}

Bon, le débogueur n'est pas très pratique pour espionner (ce qui est notre but),
donc nous allons écrire un petit utilitaire pour afficher le contenu du plateau:

\lstinputlisting[style=customc]{examples/minesweeper/minesweeper_cheater.c}

Simplement donner le \ac{PID}
\footnote{Le PID peut être vu dans le Task Manager
(l'activer avec \q{View $\rightarrow$ Select Columns})}
et l'adresse du tableau (\TT{0x01005340} pour Windows XP SP3 English)
et il l'affichera
\footnote{L'exécutable compilé est ici:
\href{http://go.yurichev.com/17165}{beginners.re}}.

Il s'attache à un processus win32 par le \ac{PID} et lit la mémoire du processus
à l'adresse.

\subsection{Trouver la grille automatiquement}

C'est pénible de mettre l'adresse à chaque fois que nous lançons notre utilitaire.
Aussi, différentes versions du démineur peuvent avoir le tableau à des adresses différentes.
Sachant qu'il a toujours un bord (octets 0x10), nous pouvons le trouver facilement
en mémoire:

\lstinputlisting[style=customc]{examples/minesweeper/cheater2_fragment.c}

Code source complet: \url{\RepoURL/examples/minesweeper/minesweeper_cheater2.c}.

\subsection{\Exercises}

\begin{itemize}

\item 
Pourquoi est-ce que les \emph{octets de bord} (ou \emph{valeurs sentinelles}) (0x10)
existent dans le tableau?

À quoi servent-elles si elles ne sont pas visibles dans l'interface du démineur?
Comment est-ce qu'il pourrait fonctionner sans elles?

\item 
Comme on s'en doute, il y a plus de valeurs possible (pour les blocs ouverts, ceux
flagués par l'utilisateur, etc.).
Essayez de trouver la signification de chacune d'elles.

\item 
Modifiez mon utilitaire afin qu'il puisse supprimer toutes les mines ou qu'il les
place suivant un schéma fixé de votre choix dans le démineur.

\end{itemize}
