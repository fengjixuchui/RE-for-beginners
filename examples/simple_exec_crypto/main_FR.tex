\mysection{Casser le simple exécutable cryptor}

J'ai un fichier exécutable qui est chiffré par un chiffrement relativement simple.
\href{\GitHubBlobMasterURL/examples/simple_exec_crypto/files/cipher.bin}{Il est ici}
(seule la section exécutable est laissée ici).

Tout d'abord, tout ce que fait la fonction de chiffrement, c'est d'ajouter l'index
de la position dans le buffer à l'octet.
Voici comment ça peut être implémenté en Python:

\begin{lstlisting}[caption=Python script,style=custompy]
#!/usr/bin/env python
def e(i, k):
    return chr ((ord(i)+k) % 256)

def encrypt(buf):
    return e(buf[0], 0)+ e(buf[1], 1)+ e(buf[2], 2) + e(buf[3], 3)+ e(buf[4], 4)+ e(buf[5], 5)+ e(buf[6], 6)+ e(buf[7], 7)+
           e(buf[8], 8)+ e(buf[9], 9)+ e(buf[10], 10)+ e(buf[11], 11)+ e(buf[12], 12)+ e(buf[13], 13)+ e(buf[14], 14)+ e(buf[15], 15)
\end{lstlisting}

Ainsi, si vous chiffrez un buffer avec 16 zéros, vous obtiendrez \emph{0, 1, 2, 3 ... 12, 13, 14, 15}.

\myindex{Propagating Cipher Block Chaining}
La Propagating Cipher Block Chaining (PCBC) est aussi utilisée, voici comment elle
fonctionne:

\begin{figure}[H]
\centering
\myincludegraphics{examples/simple_exec_crypto/601px-PCBC_encryption.png}
\caption{Chiffrement avec Propagating Cipher Block Chaining (l'image provient d'un article Wikipédia)}
\end{figure}

Le problème est qu'il est trop ennuyant de retrouver l'IV (Initialization Vector)
à chaque fois.
La force brute n'est pas une option, car l'IV est trop long (16 octets).
Voyons s'il est possible de recouvrer l'IV pour un fichier binaire exécutable arbitraire?

Essayons la simple analyse de fréquence.
Ceci est du code exécutable 32-bit x86, donc collectons des statistiques sur les
octets et les opcodes les plus fréquents.
J'ai essayé le fichier géant oracle.exe d' Oracle RDBMS version 11.2 pour windows
x86 et j'ai trouvé que l'octet le plus fréquent (pas de surprise) est zéro (~10\%).
L'octet suivant le plus fréquent est (encore une fois, sans surprise) 0xFF (~5\%).
Le suivant est 0x8B (~5\%).

\myindex{x86!\Instructions!MOV}
0x8B est l'opcode de \INS{MOV}, ceci est en effet l'une des instructions x86 les
plus fréquentes.
Maintenant, que dire de la popularité de l'octet zéro?
Si le compilateur doit encoder une valeur plus grande que 127, il doit utiliser un
déplacement 32-bit au lieu d'un de 8-bit, mais les grandes valeurs sont très rares,
donc il est complété par des zéros.
\myindex{x86!\Instructions!LEA}
\myindex{x86!\Instructions!PUSH}
\myindex{x86!\Instructions!CALL}
C'est le cas au moins avec \INS{LEA}, \INS{MOV}, \INS{PUSH}, \INS{CALL}.

Par exemple:

\begin{lstlisting}[style=customasmx86]
8D B0 28 01 00 00                 lea     esi, [eax+128h]
8D BF 40 38 00 00                 lea     edi, [edi+3840h]
\end{lstlisting}

Les déplacements plus grand que 127 sont très fréquents, mais ils excèdent rarement
0x10000 (en effet, des buffers mémoire/structures aussi grands sont aussi rares).

Même chose avec \INS{MOV}, les grandes constantes sont rares, les plus utilisées sont
0, 1, 10, 100, $2^n$, et ainsi de suite.
Le compilateur doit compléter les petites constantes avec des zéros pour les encoder
comme des valeurs 32-bit:

\begin{lstlisting}[style=customasmx86]
BF 02 00 00 00                    mov     edi, 2
BF 01 00 00 00                    mov     edi, 1
\end{lstlisting}

Maintenant parlons des octets 00 et FF combinés: les sauts (conditionnels inclus)
et appels peuvent transférer le flux d'exécution en avant ou en arrière, mais très
souvent, dans les limites du module exécutable courant.
Si c'est en avant, le déplacement n'est pas très grand et il y a des zéros ajoutés.
Si c'est en arrière, le déplacement est représenté par une valeur négative, donc
complétée par des octets FF.
Par exemple, transfert du flux d'exécution en avant:

\begin{lstlisting}[style=customasmx86]
E8 43 0C 00 00                    call    _function1
E8 5C 00 00 00                    call    _function2
0F 84 F0 0A 00 00                 jz      loc_4F09A0
0F 84 EB 00 00 00                 jz      loc_4EFBB8
\end{lstlisting}

En arrière:

\begin{lstlisting}[style=customasmx86]
E8 79 0C FE FF                    call    _function1
E8 F4 16 FF FF                    call    _function2
0F 84 F8 FB FF FF                 jz      loc_8212BC
0F 84 06 FD FF FF                 jz      loc_FF1E7D
\end{lstlisting}

L'octet FF se rencontre aussi très souvent dans des déplacements négatifs, comme
ceux-ci:

\begin{lstlisting}[style=customasmx86]
8D 85 1E FF FF FF                 lea     eax, [ebp-0E2h]
8D 95 F8 5C FF FF                 lea     edx, [ebp-0A308h]
\end{lstlisting}

Jusqu'ici, tout va bien. Maintenant nous devons essayer diverses clefs 16-octet, déchiffrer
la section exécutable et mesurer les occurences des octets 00, FF et 8B.
Gardons en vue la façon dont le déchiffrement PCBC fonctionne:

\begin{figure}[H]
\centering
\myincludegraphics{examples/simple_exec_crypto/640px-PCBC_decryption.png}
\caption{Propagating Cipher Block Chaining decryption (l'image provient d'un article Wikipédia)}
\end{figure}

La bonne nouvelle est que nous n'avons pas vraiment besoin de déchiffrer l'ensemble
des données, mais seulement slice par slice, ceci est exactement comment j'ai procédé
dans mon exemple précédent: \myref{XOR_mask_2}.

Maintenant j'essaye tous les octets possible (0..255) pour chaque octet dans la clef
et je prends l'octet produisant le plus grande nombre d'octets 00/FF/8B dans le slice
déchiffré:

\begin{lstlisting}[style=custompy]
#!/usr/bin/env python
import sys, hexdump, array, string, operator

KEY_LEN=16

def chunks(l, n):
    # split n by l-byte chunks
    # https://stackoverflow.com/q/312443
    n = max(1, n)
    return [l[i:i + n] for i in range(0, len(l), n)]

def read_file(fname):
    file=open(fname, mode='rb')
    content=file.read()
    file.close()
    return content

def decrypt_byte (c, key):
    return chr((ord(c)-key) % 256)

def XOR_PCBC_step (IV, buf, k):
    prev=IV
    rt=""
    for c in buf:
	new_c=decrypt_byte(c, k)
        plain=chr(ord(new_c)^ord(prev))
	prev=chr(ord(c)^ord(plain))
	rt=rt+plain
    return rt

each_Nth_byte=[""]*KEY_LEN

content=read_file(sys.argv[1])
# split input by 16-byte chunks:
all_chunks=chunks(content, KEY_LEN)
for c in all_chunks:
    for i in range(KEY_LEN):
        each_Nth_byte[i]=each_Nth_byte[i] + c[i]

# try each byte of key
for N in range(KEY_LEN):
    print "N=", N
    stat={}
    for i in range(256):
        tmp_key=chr(i)
	tmp=XOR_PCBC_step(tmp_key,each_Nth_byte[N], N)
        # count 0, FFs and 8Bs in decrypted buffer:
	important_bytes=tmp.count('\x00')+tmp.count('\xFF')+tmp.count('\x8B')
	stat[i]=important_bytes
    sorted_stat = sorted(stat.iteritems(), key=operator.itemgetter(1), reverse=True)
    print sorted_stat[0]
\end{lstlisting}

(Le code source peut être téléchargé
\href{\GitHubBlobMasterURL/examples/simple_exec_crypto/files/decrypt.py}{ici}.)

Je le lance et voici une clef pour laquelle le nombre d'octets 00/FF/8B dans le buffer
déchiffré est maximum:

\begin{lstlisting}
N= 0
(147, 1224)
N= 1
(94, 1327)
N= 2
(252, 1223)
N= 3
(218, 1266)
N= 4
(38, 1209)
N= 5
(192, 1378)
N= 6
(199, 1204)
N= 7
(213, 1332)
N= 8
(225, 1251)
N= 9
(112, 1223)
N= 10
(143, 1177)
N= 11
(108, 1286)
N= 12
(10, 1164)
N= 13
(3, 1271)
N= 14
(128, 1253)
N= 15
(232, 1330)
\end{lstlisting}

Écrivons un utilitaire de déchiffrement avec la clef obtenue:

\begin{lstlisting}[style=custompy]
#!/usr/bin/env python
import sys, hexdump, array

def xor_strings(s,t):
    # \verb|https://en.wikipedia.org/wiki/XOR_cipher#Example_implementation|
    """xor two strings together"""
    return "".join(chr(ord(a)^ord(b)) for a,b in zip(s,t))

IV=array.array('B', [147, 94, 252, 218, 38, 192, 199, 213, 225, 112, 143, 108, 10, 3, 128, 232]).tostring()

def chunks(l, n):
    n = max(1, n)
    return [l[i:i + n] for i in range(0, len(l), n)]

def read_file(fname):
    file=open(fname, mode='rb')
    content=file.read()
    file.close()
    return content

def decrypt_byte(i, k):
    return chr ((ord(i)-k) % 256)

def decrypt(buf):
    return "".join(decrypt_byte(buf[i], i) for i in range(16))

fout=open(sys.argv[2], mode='wb')

prev=IV
content=read_file(sys.argv[1])
tmp=chunks(content, 16)
for c in tmp:
    new_c=decrypt(c)
    p=xor_strings (new_c, prev)
    prev=xor_strings(c, p)
    fout.write(p)
fout.close()
\end{lstlisting}

(Le code source peut être téléchargé
\href{\GitHubBlobMasterURL/examples/simple_exec_crypto/files/decrypt2.py}
{ici}.)

Vérifions le fichier résultant:

\lstinputlisting{examples/simple_exec_crypto/objdump_result.txt}

Oui, ceci semble être un morceau correctement désassemblé de code x86.
Le fichier déchiffré entier peut être téléchargé
\href{\GitHubBlobMasterURL/examples/simple_exec_crypto/files/decrypted.bin}{ici}.

En fait, ceci est la section text du regedit.exe de Windows 7.
Mais cet exemple est basé sur un cas réel que j'ai rencontré, seul l'exécutable est
différent (et la clef), l'algorithme est le même.

\subsection{Autres idées à prendre en considération}

Et si j'avais échoué avec cette simple analyse des fréquences?
Il y a d'autres idées sur la façon de mesurer l'exactitude de code x86 déchiffré/décompressé:

\begin{itemize}

\item Les compilateurs modernes alignent les fonctions sur une limite de 0x10.
Donc l'espace libre avant est rempli avec de NOPs (0x90) ou d'autres instructions
avec des opcodes connus: \myref{sec:npad}.

\item Peut-être que le pattern le plus fréquent dans tout langage d'assemblage est
l'appel de fonction:\\
\TT{PUSH chain / CALL / ADD ESP, X}.
Cette séquence peut facilement être détectée et trouvée.
J'ai même collecté des statistiques sur le nombre moyen d'arguments des fonctions: \myref{args_stat}.
(Ainsi, ceci est la longueur moyenne d'une chaîne PUSH.)

\end{itemize}

En savoir plus sur le code désassemblé incorrectement/correctement: \myref{ISA_detect}.

