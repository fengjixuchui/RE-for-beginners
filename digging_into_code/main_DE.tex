\chapter{Finden von wichtigen / interessanten Stellen im Code}

Minimalismus ist kein beliebtes Feature moderner Software.

\myindex{\Cpp!STL}

Aber nicht weil die Programmierer so viel Code schreiben, sondern weil die Libaries
allgemein statisch zu ausf\"uhrbaren Dateien gelinkt werden. Wenn alle externen
Libraries in externe DLL Dateien verschoben werden w\"urden, w\"are die Welt ein
anderer Ort. (Ein weiterer Grund f\"ur C++ sind die \ac{STL} und andere Template-Libraries.)

\newcommand{\FOOTNOTEBOOST}{\footnote{\url{http://go.yurichev.com/17036}}}
\newcommand{\FOOTNOTELIBPNG}{\footnote{\url{http://go.yurichev.com/17037}}}

Deshalb ist es sehr wichtig den Ursprung einer Funktion zu bestimmen, wenn die
Funktion aus einer Standard-Library oder aus einer sehr bekannten Library stammt
(wie z.B Boost\FOOTNOTEBOOST, libpng\FOOTNOTELIBPNG), oder ob die Funktion sich
auf das bezieht was wir im Code versuchen zu finden.

Es ist ein wenig absurd s\"amtlichen Code in \CCpp neu zu schreiben, um das zu
finden was wir suchen.

Eine der Hauptaufgaben eines Reverse Enigneers ist es schnell Code zu finden den
er/sie sucht.

\myindex{\GrepUsage}

Der \IDA-Disassembler erlaubt es durch Textstrings, Byte-Sequenzen und Konstanten
zu suchen.  Es ist sogar m\"oglich den Code in .lst oder .asm Text Dateien zu
exportieren und diese mit \TT{grep}, \TT{awk}, etc. zu untersuchen.

Wenn man versucht zu verstehen wie ein bestimmter Code funktioniert, kann auch
eine einfache Open-Source-Library wie libpng als Beispiel dienen.
Wenn man also eine Konstante oder Textstrings findet die vertraut erscheinen, ist
es immer einen Versuch wert diese zu \emph{google}n .
Und wenn man ein Opensource Projekt findet in dem diese Funktion benutzt wird, 
reicht es meist aus diese Funktionen miteinander zu vergleichen.
Es k\"onnte helfen Teile des Problems zu l\"osen.

% When you try to understand what some code is doing, this easily could be some open-source library like libpng.
% So when you see some constants or text strings which look familiar, it is always worth to \emph{google} them.
% And if you find the opensource project where they are used, 
% then it's enough just to compare the functions.
% It may solve some part of the problem.

Zum Beispiel, wenn ein Programm XML Dateien benutzt, w\"are der erste Schritt zu ermitteln welche
XML-Library benutzt wird f\"ur die Verarbeitung, da die Standard (oder am weitesten verbreitete) libraries
normal benutzt werden anstatt selbst geschriebene librarys.

\myindex{SAP}
\myindex{Windows!PDB}

Zum Beispiel, der Autor dieser Zeilen wollte verstehen wie die Kompression/Dekompression von Netzwerkpaketen in SAP 6.0 funktioniert.
SAP ist ein gewaltiges St\"uck Software, aber detaillierte -\gls{PDB} Dateien mit Debug Informationen sind vorhanden, was sehr praktisch 
ist. Der Autor hat schließlich eine Ahnung gehabt, das eine Funktion genannt \emph{CsDecomprLZC} die Dekompression der Netzwerkpakete \"ubernahm.
Er hat nach dem Namen der Funktion auf google gesucht und ist schnell zum schluss gekommen das diese Funktion in 
MaxDB benutzt wurde (Das ist ein Open-Source SAP Projekt) \footnote{Mehr dar\"uber in der relevanten Sektion~(\myref{sec:SAPGbUI})}. 

\url{http://www.google.com/search?q=CsDecomprLZC}

Erstaunlich, das MaxDB und die SAP 6.0 Software den selben Code geteilt haben f\"ur die Kompression/Dekompression der Netzwerkpakete.

\input{digging_into_code/identification/exec_DE} 

\mysection{Kommunikation mit der außen Welt (Funktion Level)} 
Oft ist es empfehlenswert die Funktionsargumente und die R\"uckgabewerte im
Debugger oder \ac{DBI} zu \"uberwachen. Zum Beispiel hat der Autor einmal
versucht die Bedeutung einer obskuren Funktion zu verstehen, die einen inkorrekten
Bubblesort-Algorithmus implementiert hatte\footnote{\url{https://yurichev.com/blog/weird_sort_KLEE/}}
(Sie hat funktioniert, jedoch viel langsamer als normal). Die Eingaben und Ausgaben zur Laufzeit 
der Funktion zu \"uberwachen hilft sofort zu verstehen was die Funktion tut.

% TBT

% sections:
\input{digging_into_code/communication_win32_DE}
\input{digging_into_code/strings_DE}
\input{digging_into_code/assert_DE}
\mysection{Konstanten}

Menschen, Programmierer eingeschlossen, neigen dazu Zahlen zu runden wie z.B 10, 100, 1000,
im realen Leben so wie in ihrem Code.

Der angehende Reverse Engineer kennt diese Werte und ihre hexadezimale Repr\"asentation sehr gut:
10=0xA, 100=0x64, 1000=0x3E8, 10000=0x2710.

Die Konstanten \TT{0xAAAAAAAA} (0b10101010101010101010101010101010) und \\
\TT{0x55555555} (0b01010101010101010101010101010101) sind auch sehr popul\"ar---
sie sind zusammengesetzt aus ver\"andernden Bits. % <-- Findest vielleicht noch ne bessere Bezeichnung

Dies hilft Signale voneinander zu unterscheiden bei denen alle Bits eingeschaltet (0b1111 \dots) oder ausgeschaltet (0b0000 \dots) werden .
Zum Beispiel wird die Konstante \TT{0x55AA} beim Boot Sektor, \ac{MBR},
und im \ac{ROM} von IBM-Kompatiblen Erweiterung Karten benutzt.

Manche Algorithmen, speziell die Kryptografischen benutzen eindeutige Konstanten, die mit der Hilfe von \IDA einfach im Code zu finden sind.

\myindex{MD5}

Zum Beispiel, der MD5 Algorithmus initialisiert seine Internen Variablen wie folgt:

\begin{verbatim}
var int h0 := 0x67452301
var int h1 := 0xEFCDAB89
var int h2 := 0x98BADCFE
var int h3 := 0x10325476
\end{verbatim}

Wenn man diese vier Konstanten im Code hintereinander benutzt findet, dann ist die Wahrscheinlichkeit das diese Funktion 
sich auf MD5 bezieht.

\par Ein weiteres Beispiel sind die CRC16/CRC32 Algorithmen,
ihre Berechnungs Algorithmen benutzen oft vorberechnete Tabellen wie diese:

\begin{lstlisting}[caption=linux/lib/crc16.c,style=customc]
/** CRC table for the CRC-16. The poly is 0x8005 (x^16 + x^15 + x^2 + 1) */
u16 const crc16_table[256] = {
	0x0000, 0xC0C1, 0xC181, 0x0140, 0xC301, 0x03C0, 0x0280, 0xC241,
	0xC601, 0x06C0, 0x0780, 0xC741, 0x0500, 0xC5C1, 0xC481, 0x0440,
	0xCC01, 0x0CC0, 0x0D80, 0xCD41, 0x0F00, 0xCFC1, 0xCE81, 0x0E40,
	...
\end{lstlisting}

Man beachte auch die vorberechnete Tabelle f\"ur CRC32: \myref{sec:CRC32}.

In tabellenlosen CRC-Algorithmen werden bekannte Polynome benutzt, zum Beispiel, 0xEDB88320 f\"ur CRC32.

\subsection{Magic numbers}
\label{magic_numbers}

Viele Datei-Formate definieren einen Standard-Dateiheader in dem eine \emph{magic number(s)} benutzt wird, einzelne oder sogar mehrere. 

\myindex{MS-DOS}

Zum Beispiel, alle Win32 und MS-DOS executable starten mit zwei Zeichen \q{MZ}.

\myindex{MIDI}

Am Anfang einer MIDI Datei muss die \q{MThd} Signatur vorhanden sein.
Wenn wir ein Programm haben das auf MIDI Dateien zugreift um sonst was zu machen,
ist es sehr wahrscheinlich das das Programm die Datei validieren muss in dem es
mindestens die ersten 4 Bytes pr\"uft.

Das kann man wie folgt realisieren:
(\emph{buf} Zeigt auf den Anfang der geladenen Datei im Speicher) 

\begin{lstlisting}[style=customasmx86]
cmp [buf], 0x6468544D ; "MThd"
jnz _error_not_a_MIDI_file
\end{lstlisting}

\myindex{\CStandardLibrary!memcmp()}
\myindex{x86!\Instructions!CMPSB}

\dots oder durch das Aufrufen der Funktion f\"ur das vergleichen von Speicherbl\"ocken wie z.B \TT{memcmp()} oder 
beliebigen anderen Code bis hin zu einer \TT{CMPSB} (\myref{REPE_CMPSx}) Instruktion.

Wenn man so einen Punkt findet kann man bereits sagen das eine MIDI Datei geladen wird, % <-- \"Andern?
wir k\"onnen auch sehen wo der Puffer mit den Inhalten der MIDI Datei liegt und was/wie aus diesem
Puffer verwendet wird.

\subsubsection{Daten}

\myindex{UFS2}
\myindex{FreeBSD}
\myindex{HASP}

Oft findet man auch nur eine Zahl wie \TT{0x19870116}, was ganz klar nach einem Jahres Datum aussieht (Tag 16,  1 Monat (Januar),  Jahr 1987).
Das ist vielleicht das Geburtsdatum von jemandem (ein Programmierer. ihre/seine bekannte, Kind), oder ein anderes wichtiges Datum.
Das Datum kann auch in umgekehrter folge auftreten, wie z.B \TT{0x16011987}. 
Datums angaben im Amerikanischen-Stil sind auch weit verbreitet wie \TT{0x01161987}.

Ein ziemlich bekanntes Beispiel ist  \TT{0x19540119} (magic number wird in der UFS2 Superblock Struktur benutzt), das 
Geburtsdatum von Marschall Kirk McKusick ist, einem Prominenten FreeBSD Entwickler. 


\myindex{Stuxnet}
Stuxnet benutzt die Zahl ``19790509'' (nicht als 32-Bit Zahl, aber als String), was zu Spekulationen gef\"uhrt hat
weil die malware Verbindungen nach Israel aufzeigt%
\footnote{Das ist das Datum der Hinrichtung von Habib Elghanian, persischer Jude.}.

Solche Zahlen sind auch sehr beliebt in Amateur Kryptografie, zum Beispiel, ein Ausschnitt aus den \emph{secret function} Interna aus dem HASP3 Dongle %  <-- Vielleicht bessere formulierung?
\footnote{\url{https://web.archive.org/web/20160311231616/http://www.woodmann.com/fravia/bayu3.htm}}:

\begin{lstlisting}[style=customc]
void xor_pwd(void) 
{ 
	int i; 
	
	pwd^=0x09071966;
	for(i=0;i<8;i++) 
	{ 
		al_buf[i]= pwd & 7; pwd = pwd >> 3; 
	} 
};

void emulate_func2(unsigned short seed)
{ 
	int i, j; 
	for(i=0;i<8;i++) 
	{ 
		ch[i] = 0; 
		
		for(j=0;j<8;j++)
		{ 
			seed *= 0x1989; 
			seed += 5; 
			ch[i] |= (tab[(seed>>9)&0x3f]) << (7-j); 
		}
	} 
}
\end{lstlisting}

\subsubsection{DHCP}

Das Trifft auf Netzwerk Protokolle ebenso zu. 
Zum Beispiel, die Pakete des DHCP Protokoll's beinhalten so genannte \emph{magic cookie}: \TT{0x63538263}.
Jeder Code der ein DHCP Pakete generiert, muss diese Konstante in das Pakete einbetten.
Wenn wir diesen Code finden, wissen wir auch wo es passiert und nicht nur was passiert.
Jedes Programm das DHCP Pakete empfangen kann muss verifizieren das der \emph{magic cookie} mit der Konstante 
\"ubereinstimmt. 

Zum Beispiel, lasst uns die dhcpcore.dll Datei aus Windows 7 x64 analysieren die nach der Konstante suchen.
Wir k\"onnen die Konstante zweimal finden:
Es sieht danach aus als w\"are die Konstante in zwei Funktionen benutzt mit dem selbst redenden Namen\\
\TT{DhcpExtractOptionsForValidation()} und \TT{DhcpExtractFullOptions()}:

\begin{lstlisting}[caption=dhcpcore.dll (Windows 7 x64),style=customasmx86]
.rdata:000007FF6483CBE8 dword_7FF6483CBE8 dd 63538263h          ; DATA XREF: DhcpExtractOptionsForValidation+79
.rdata:000007FF6483CBEC dword_7FF6483CBEC dd 63538263h          ; DATA XREF: DhcpExtractFullOptions+97
\end{lstlisting}

Und hier die (Speicher) Orte an denen auf die Konstante zugegriffen wird:

\begin{lstlisting}[caption=dhcpcore.dll (Windows 7 x64),style=customasmx86]
.text:000007FF6480875F  mov     eax, [rsi]
.text:000007FF64808761  cmp     eax, cs:dword_7FF6483CBE8
.text:000007FF64808767  jnz     loc_7FF64817179
\end{lstlisting}

Und:

\begin{lstlisting}[caption=dhcpcore.dll (Windows 7 x64),style=customasmx86]
.text:000007FF648082C7  mov     eax, [r12]
.text:000007FF648082CB  cmp     eax, cs:dword_7FF6483CBEC
.text:000007FF648082D1  jnz     loc_7FF648173AF
\end{lstlisting}

\subsection{Spezifische Konstanten}

Manchmal, gibt es spezifische Konstanten f\"ur gewissen Code % <-- Besser? 
Zum Beispiel, einmal hat der Autor sich in ein St\"uck Code gegraben wo die Nummer 12 verd\"achtig
oft vor kam. Arrays haben oft eine Gr\"oße von 12 oder ein vielfaches von 12 (24, etc). 
Wie sich raus stellte, hat der Code eine 12-Kanal Audiodatei an der Eingabe entgegen genommen und
sie verarbeitet.

Und umgekehrt: zum Beispiel, wenn ein Programm ein Textfeld verarbeitet das eine L\"ange von 120 Bytes hat,
dann gibt es auch eine Konstante 120 oder 119 irgendwo im Code.
Wenn UTF-16 Benutzt wird, dann $2 \cdot 120$. Wenn Code mit Netzwerkpaketen arbeitet die von fester Gr\"oße
sind, ist es eine gute Idee nach dieser Konstante im Code zu suchen.

Das trifft auch auf Amateur Kryptografie zu (Lizenz Schl\"ussel, etc). 
Bei einem verschl\"usselten Block von $n$ Bytes, will man versuchen die vorkommen dieser Nummer im Code zu suchen,
auch, wenn man ein St\"uck Code sieht der sich $n$ mal w\"ahrend einer Schleifen Ausf\"uhrung wiederholt, ist das vielleicht
eine ver-/Entschl\"usselung Routine.

\subsection{Nach Konstanten suchen}

Das ist einfach mit \IDA: Alt-B oder Alt-I.
\myindex{bin\"ar grep}
Und f\"ur das suchen von Konstanten in einem Haufen großer Dateien, oder f\"ur das suchen in nicht ausf\"uhrbaren Dateien,
gibt es ein kleines Utility genannt \emph{binary grep}\footnote{\BGREPURL}.

\input{digging_into_code/instructions_DE}
\mysection{Verd\"achtige Code muster}

\subsection{XOR Instruktionen}
\myindex{x86!\Instructions!XOR}

Instruktionen wie \TT{XOR op, op} (zum Beispiel, \TT{XOR EAX, EAX})
werden normal daf\"ur benutzt Register Werte auf Null zu setzen, wenn jedoch
einer der Operanden sich unterscheidet wird die \q{exclusive or} Operation 
ausgef\"uhrt.

Diese Operation wird allgemeinen selten benutzt beim programmieren, aber ist
weit verbreitet in der Kryptografie, besonders bei Amateuren der Kryptografie.
Sowas ist besonders Verd\"achtig wenn der zweite Operand eine große Zahl ist.

Das k\"onnte ein Hinweis sein das etwas ver-/entschl\"usselt wird oder Checksumme berechnet werden, etc.

Eine Ausnahme dieser Beobachtung ist der \q{canary} (\myref{subsec:BO_protection}). 
Die Generierung und das pr\"ufen des \q{canary} werden oft mit Hilfe der \XOR Instruktion gemacht. 

\myindex{AWK}

Dieses AWK Skript kann benutzt werden um \IDA{} listing (.lst) Dateien zu parsen:

\lstinputlisting{digging_into_code/awk.sh}

Es sollte auch noch erw\"ahnt werden das diese Art von Skript in der Lage ist inkorrekt disassemblierten Code zu erkennen
(\myref{sec:incorrectly_disasmed_code}).

\subsection{Hand geschriebener Assembler code}

\myindex{Function prologue}
\myindex{Function epilogue}
\myindex{x86!\Instructions!LOOP}
\myindex{x86!\Instructions!RCL}

Moderne Compiler benutzen keine \TT{LOOP} und \TT{RCL} Instruktionen.
Auf der anderen Seite sind diese Instruktionen sehr beliebt bei Programmieren die Code direkt in Assembler schreiben.
Wenn man diese Instruktionen sieht, kann man mit hoher Sicherheit sagen das dieses Code Fragment h\"andisch geschrieben wurde.,
Diese Instruktionen sind in der Instruktionsliste im Anhang mit (M) markiert: \myref{sec:x86_instructions}.

\par Die Funktions Prolog und Epilog sind allgemein nicht vorhanden bei handgeschriebenen Assembler Code.

\par Tats\"achlich gibt es kein bestimmtes System um Argumente an Funktionen zu \"ubergeben wenn der Code handgeschrieben wurde. 

\par Beispiel aus dem Windows 2003 Kernel (ntoskrnl.exe file):

\lstinputlisting[style=customasmx86]{digging_into_code/ntoskrnl.lst}

Tats\"achlich, wenn wir in den \ac{WRK} v1.2 source code schauen, kann dieser Code einfach in der Datei
\emph{WRK-v1.2\textbackslash{}base\textbackslash{}ntos\textbackslash{}ke\textbackslash{}i386\textbackslash{}cpu.asm} gefunden werden.

% TBT
%\par 
%As of \INS{RCL}, I could find it in ntoskrnl.exe file from Windows 2003 x86 (MS Visual C compiler).
%It is occurred only once, in \TT{RtlExtendedLargeIntegerDivide()} function, and this might be inline assembler code case.

\input{digging_into_code/magic_numbers_tracing_DE}
\input{digging_into_code/loops_DE}
% TODO move section...

\subsection{ Muster in Bin\"ardatein finden}

Alle Beispiele hier wurden vorbereitet mit Windows mit aktiver Code Page 437
in der Konsole.
Bin\"ar Dateien sehen intern etwas anders aus wenn eine andere Code page gesetzt ist.

\clearpage
\subsubsection{Arrays}

Manchmal kann man klar ein Array von 16/32/64-Bit Werten mit bloßem Auge im hex Editor erkennen.

Hier ist ein Beispiel eines 16-Bit Wertes.
Wir sehen das das erste Byte ein paar aus 7 oder 8 ist und das zweite sieht
zuf\"allig aus:

\begin{figure}[H]
\centering
\myincludegraphics{digging_into_code/binary/16bit_array.png}
\caption{FAR: array von 16-Bit Werten}
\end{figure}

Ich habe eine Datei benutzt die ein 12 Kanal Signal digitalisiert mit 16-Bit nutzt \ac{ADC}.

\clearpage
\myindex{MIPS}
\par Und hier ist ein Beispiel von einem Typischen MIPS Code.

Wie wir uns vielleicht erinnern, jede MIPS ( also auch ARM in ARM Mode oder ARM64 ) Instruktion hat eine Gr\"oße von 32 Bits (oder 4 Bytes),
also ist solcher Code ein Array von 32-Bit Werten. 

Wenn man den Screenshot anschaut, sehen wir eine Art Muster.

Vertikale und rote Linien wurden zur besseren Lesbarkeit eingef\"ugt:

\begin{figure}[H]
\centering
\myincludegraphics{digging_into_code/binary/typical_MIPS_code.png}
\caption{Hiew: sehr typischer MIPS code}
\end{figure}

Ein weiteres Beispiel eines solchen Musters ist Buch:
\myref{Oracle_SYM_files_example}.

\clearpage
\subsubsection{Sparse Dateien} 

Diese d\"urftige Datei mit zerstreuten Daten inmitten einer fast leeren Datei.
Jedes Space Zeichen hier ist in der tat ein Zero Byte (das wie ein space aussieht). % <-- findet man sicher was besseres
Das ist eine Datei mit der ein FPGA Programmiert wird (Ein Altera Stratix GX Ger\"at).
Sicher k\"onnen Dateien wie diese einfach Komprimiert werden, aber diese Formate sind in 
der Wissenschaft und im Ingenieurs Wesen so wie in der Softwareentwicklung sehr verbreitet.
Wo es oft um effizienten Zugriff geht und weniger um die Komprimierung der Daten.

% This is sparse file with data scattered amidst almost empty file.
% Each space character here is in fact zero byte (which is looks like space).
% This is a file to program FPGA (Altera Stratix GX device).
% Of course, files like these can be compressed easily, but formats like this one are very popular in scientific and engineering software where efficient access is important while compactness is not.

\begin{figure}[H]
\centering
\myincludegraphics{digging_into_code/binary/sparse_FPGA.png}
\caption{FAR: Sparse file}
\end{figure}

\clearpage
\subsubsection{Komprimierte Dateien}

% FIXME \ref{} ->
Diese Datei ist einfach ein komprimiertes Archiv. 
Es hat eine relativ hohe Entropie und visuell betrachtet sieht es 
eher Chaotisch aus. So sehen komprimierte oder verschl\"usselte Dateien aus.

\begin{figure}[H]
\centering
\myincludegraphics{digging_into_code/binary/compressed.png}
\caption{FAR: Komprimierte Datei}
\end{figure}

\clearpage
\subsubsection{\ac{CDFS}}

\ac{OS} Installationen werden \"ublicherweise als ISO Datei bereit gestellt, die Kopien von CD/DVD Disks sind. 
Das Dateisystem das benutzt wird heißt \ac{cdfs}, hier sieht man wie Dateinamen mit zus\"atzlichen Daten vermischt sind.
Das k\"onnen Datei Gr\"oßen, Pointer auf andere Verzeichnisse, Datei Attribute und anderes sein. 
So sehen Dateisysteme typischerweise auch von innen aus.

\begin{figure}[H]
\centering
\myincludegraphics{digging_into_code/binary/cdfs.png}
\caption{FAR: ISO file: Ubuntu 15 Installation \ac{CD}}
\end{figure}

\clearpage
\subsubsection{32-bit x86 ausf\"uhrbarer Code} 

So sieht 32-Bit x86 ausf\"uhrbarer Code aus. 
Der Code hat nicht wirklich viel Entropie, weil manche Bytes \"ofters vorkommen als andere.

\begin{figure}[H]
\centering
\myincludegraphics{digging_into_code/binary/x86_32.png}
\caption{FAR: Executable 32-bit x86 code}
\end{figure}

% TODO: Read more about x86 statistics: \ref{}. % FIXME blog post about decryption...

\clearpage
\subsubsection{BMP graphics files}

% TODO: bitmap, bit, group of bits...

BMP Dateien sind nicht komprimiert, also ist jedes Byte ( oder Gruppen von Bytes ) beschrieben als
ein Pixel. Diese Bild habe ich irgendwo in meiner Windows 8.1 Installation gefunden: 

\begin{figure}[H]
\centering
\myincludegraphicsSmall{digging_into_code/binary/bmp.png}
\caption{Example picture}
\end{figure}

Man kann sehen das dieses Bild Pixel hat, die nicht wirklich gut komprimiert werden k\"onne (um das Zentrum herum),
aber es sind lange ein-Farben Linien am Anfang und am ende der Datei. Tats\"achlich Linien wie diese sehen wie Linien aus
wenn man sich die Datei anschaut:

\begin{figure}[H]
\centering
\myincludegraphics{digging_into_code/binary/bmp_FAR.png}
\caption{BMP file fragment}
\end{figure}


% FIXME comparison!
\subsection{Memory \q{snapshots} comparing}
\label{snapshots_comparing}

Die Technik zwei Memory Snapshots zu vergleichen ist recht einfach, das hat man auch oft benutzt um 8-Bit Computerspiele und
\q{high score}'s  zu hacken.

Zum Beispiel, wenn man ein geladenes Spiel auf einem 8-Bit Computer hat ( auf den Maschinen ist nicht viel Speicher 
vorhanden, jedoch braucht das Spiel noch weniger Speicher) und du weißt was du im Spiel hast, sagen wir 100 Patronen, 
nun kann man einen \q{snapshot} vom gesamten Speicher machen und diesen Irgendwohin speichern. Dann verschiesst man 
eine Patrone, dann geht der Patronen Z\"ahler auf 99, nun erstellt man den zweiten Snapshot und Vergleich die beiden: 
Nun muss es irgendwo ein Byte geben das vorher 100 war und jetzt 99 ist. 

Betrachtet man den Fakt das diese 8-Bit Spiele oftmals in Assembler geschrieben wurden und diese Variablen meist global 
waren, konnte man ziemlich einfach bestimmen welche Adressen im Speicher den Kugelz\"ahler beinhalten. Wenn man nach allen 
Referenzen der Adresse im dissassembelten Spiel code sucht, ist es nicht schwer den Code \glslink{decrement}{decrementing} 
zu finden und dann eine \gls{NOP} Instruktion an diese Stelle zu schreiben, oder gar mehrere \gls{NOP}-s, und dann hat man 
ein Spiel bei dem man f\"ur immer 100 Kugeln hat. %<-- das kacke der ganze block
\myindex{BASIC!POKE}
Spiele auf 8-Bit Computern wurden allgemein an konstanten Adressen geladen, zus\"atzlich gab es nicht viele unterschiedliche
Versionen des Spiels (  Es war meist eine Version f\"ur lange Zeit popul\"ar ), dadurch wussten enthusiastische Gamer welche
Bytes (durch das benutzen von Basic Instruktionen wie \gls{POKE}) \"uberschrieben werden mussten um das Spiel zu hacken. 
Das hat wiederum zu \q{cheat} listen gef\"uhrt die in Magazinen f\"ur 8-Bit Games erschienen, die dann \gls{POKE} Instruktionen enthielten.

% Considering the fact that these 8-bit games were often written in assembly language and such variables were global,
% it can be said for sure which address in memory has holding the bullet count. If you searched for all references to the
% address in the disassembled game code, it was not very hard to find a piece of code \glslink{decrement}{decrementing} the bullet count,
% then to write a \gls{NOP} instruction there, or a couple of \gls{NOP}-s, 
% and then have a game with 100 bullets forever.
% \myindex{BASIC!POKE}
% Games on these 8-bit computers were commonly loaded at the constant
% address, also, there were not much different versions of each game (commonly just one version was popular for a long span of time),
% so enthusiastic gamers knew which bytes must be overwritten (using the BASIC's instruction \gls{POKE}) at which address in
% order to hack it. This led to \q{cheat} lists that contained \gls{POKE} instructions, published in magazines related to
% 8-bit games.

\myindex{MS-DOS}

Es ist auch einfach \q{high score} Dateien zu modifizieren, das funktioniert nicht nur bei 8-Bit Spielen. Man achte 
auf seinen Highscore Z\"ahler, dann macht man ein Backup der Datei. Wenn sich der \q{high score} Z\"ahler \"andert, vergleicht man die 
zwei Dateien miteinander, das kann man sogar mit dem DOS Tool FC\footnote{MS-DOS Utility zum vergleichen von  Dateien} (\q{high score} Dateien,
sind oft in Bin\"arer Form). 

Es wird beim Vergleichen der Dateien einen Punkt geben wo einige Bytes sich unterscheiden und 
es wird leicht sein, die Punkte zu sehen die die Bytes des Punktez\"ahler beinhalten. 
Jedoch sind sich die Spiele Entwickler solcher Tricks bewusst und bauen Wege ein um das Programm
vor solchen Manipulationen zu sch\"utzen. 

Ein \"ahnliches Beispiel findet man auch in dem Buch \myref{Millenium_DOS_game}.

% TODO: пример с какой-то простой игрушкой?

% TBT 

\subsubsection{Windows registry}

Es ist auch m\"oglich die Windows Regestry zu vergleichen vor und nach der Programm Installation.

Es ist eine sehr popul\"are Methode Regestry Elemente zu finden die vom Programm benutzt werden.
Vielleicht ist das auch der Grund warum die \q{windows regestry cleaner} Shareware so popul\"ar ist.

% TBT

\subsubsection{Blink-comparator}

Der Vergleich von Datei- oder Speichersnapshots erinnert ein wenig an einen Blinkkomparator
\footnote{\url{http://go.yurichev.com/17348}}
ein Ger\"at das in der Vergangenheit von Astronomen benutzt wurde, um sich bewegende Astronomische
Objekte zu finden.

Ein Blinkkomperator erlaubt es schnell zwischen Photographie zu wechseln die zu unterschiedlicher
Zeit aufgenommen wurden, so kann ein Astronom Unterschiede zwischen Fotografien visuell erkennen.

Ach \"ubrigens, Pluto wurde durch einen solchen Blink-Komparator 1930 entdeckt.

% TBT \input{digging_into_code/ISA_detect_DE}

\mysection{Andere Dinge}

\subsection{Die Idee}  

Ein Reverse Engineer sollte versuchen so oft wie m\"oglich in den Schuhen des
Programmierers zu laufen. Um ihren/seinen Standpunkt zu betrachten uns sich
selbst zu Fragen wie man einen Task in spezifischen F\"allen l\"osen w\"urde.

\subsection{Anordnung von Funktionen in Bin\"ar Code}  

S\"amtliche Funktionen die in einer einzelnen .c oder .cpp-Datei gefunden werden,
werden zu den entsprechenden Objekt Dateien (.o) kompiliert. Sp\"ater, f\"ugt
der Linker alle Objektdatein die er braucht zusammen, ohne die Reihenfolge oder
die Funktionen in Ihnen zu ver\"andern. Als eine Konsequenz, ergibt sich daraus
wenn man zwei oder mehr aufeinander folgende Funktionen sieht, bedeutet dass das
sie in der gleichen Source Code Datei platziert waren (Außer nat\"urlich man bewegt
sich an der Grenze zwischen zwei Dateien.). Das bedeutet das diese Funktionen etwas
gemeinsam haben, das sie aus dem gleichen \ac{API}-Level stammen oder aus der
gleichen Library, etc.

% TBT
%\myindex{CryptoPP}
%This is a real story from practice: once upon a time, the author searched for Twofish-related functions in
%a program with CryptoPP library linked, especially encryption/decryption functions.\\
%I found the \verb|Twofish::Base::UncheckedSetKey()| function, but not others.
%After peeking into the \verb|twofish.cpp| source code
%\footnote{\url{https://github.com/weidai11/cryptopp/blob/b613522794a7633aa2bd81932a98a0b0a51bc04f/twofish.cpp}}, it became clear that all functions are located in one module (\verb|twofish.cpp|).\\
%So I tried all function that followed \verb|Twofish::Base::UncheckedSetKey()|---as it happened,\\
%one was \verb|Twofish::Enc::ProcessAndXorBlock()|, another---\verb|Twofish::Dec::ProcessAndXorBlock()|.

\subsection{kleine Funktionen} 

Sehr kleine oder leere Funktionen  (\myref{empty_func})
oder Funktionen die nur ``true'' (1) oder ``false'' (0) (\myref{ret_val_func}) sind weit verbreitet,
und fast jeder ordentlicher Compiler tendiert dazu nur solche Funktionen in den resultierenden ausf\"uhrbaren Code zu stecken,
sogar wenn es mehrere gleiche Funktionen im Source Code bereits gibt. 
Also, wann immer man solche kleinen Funktionen sieht die z.B nur aus \TT{mov eax, 1 / ret} bestehen und von mehreren 
Orten aus referenziert werden (und aufgerufen werden k\"onnen), und scheinbar keine Verbindung zu einander haben, dann 
ist das wahrscheinlich das Ergebnis einer Optimierung. 

\subsection{\Cpp}

\ac{RTTI}~(\myref{RTTI})-data ist vielleicht auch n\"utzlich f\"ur die \Cpp Klassen Identifikation.
