\mysection{Détection de l'\ac{ISA}}
\label{ISA_detect}

Souvent, vous avez à faire à un binaire avec un \ac{ISA} inconnu.
Peut-être que la manière la plus facile de détecter l'\ac{ISA} est d'en essayer plusieurs
dans \IDA, objdump ou un autre désassembleur.

Pour réussir ceci, il faut comprendre la différence entre du code incorrectement
et celui correctement désassemblé.

% subsection:
\renewcommand{\CURPATH}{digging_into_code/incorrect_disassembly}
\mysection{Fonction presque vide}
\label{Boolector}
\myindex{Boolector}
\myindex{x86!\Instructions!JMP}

Ceci est un morceau de code réel que j'ai trouvé dans Boolector\footnote{\url{https://boolector.github.io/}}:

\lstinputlisting[style=customc]{patterns/025_almost_empty/boolectormain.c}

Pourquoi quelqu'un ferait-il comme ça?
Je ne sais pas mais mon hypothèse est que \verb|boolector_main()| peut être compilée
dans une sorte de DLL ou bibliothèque dynamique, et appelée depuis une suite de test.
Certainement qu'une suite de test peut préparer les variables argc/argv comme
le ferait \ac{CRT}.

Il est intéressant de voir comment c'est compilé:

\lstinputlisting[caption=GCC 8.2 x64 \NonOptimizing (\assemblyOutput),style=customasmx86]{patterns/025_almost_empty/boolectormain_O0.s}

Ceci est OK, le prologue (non optimisé) déplace inutilement deux arguments,
\INS{CALL}, épilogue, \INS{RET}.
Mais regardons la version optimisée:

\lstinputlisting[caption=GCC 8.2 x64 \Optimizing (\assemblyOutput),style=customasmx86]{patterns/025_almost_empty/boolectormain_O3.s}

Aussi simple que ça: la pile et les registres ne sont pas touchés et \verb|boolector_main()|
a le même ensemble d'arguments.
Donc, tout ce que nous avons à faire est de passer l'exécution à une autre adresse.

Ceci est proche d'une \glslink{thunk function}{fonction thunk}.

Nous verons queelque chose de plus avancé plus tard: \myref{ARM_B_to_printf}, \myref{JMP_instead_of_RET}.


\subsection{Code désassemblé correctement}
\label{correctly_disasmed_code}

Chaque \ac{ISA} a une douzaine d'instructions les plus utilisées, toutes les autres
le sont beaucoup moins souvent.

Concernant le x86, il est intéressant de savoir le fait que les instructions d'appel
de fonctions (\PUSH/\CALL/\ADD) et \MOV sont les morceaux de code les plus fréquemment
exécutées dans presque tous les programmes que nous utilisons.
Autrement dit, le \ac{CPU} est très occupé à passer de l'information entre les niveaux
d'abstraction, ou, on peut dire qu'il est très occupé à commuter entre ces niveaux.
Indépendamment du type d'\ac{ISA}.
Ceci a un coût de diviser les problèmes entre plusieurs niveaux d'abstraction (ainsi
les êtres humain peuvent travailler plus facilement avec).

