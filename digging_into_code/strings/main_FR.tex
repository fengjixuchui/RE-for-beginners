\subsection{Chaînes de texte}

\subsubsection{\CCpp}

\label{C_strings}
Les chaînes C normales sont terminées par un zéro (chaînes \ac{ASCIIZ}).

La raison pour laquelle le format des chaînes C est ce qu'il est (terminé par zéro)
est apparemment historique:
Dans [Dennis M. Ritchie, \emph{The Evolution of the Unix Time-sharing System}, (1979)]
nous lisons:

\begin{framed}
\begin{quotation}
A minor difference was that the unit of I/O was the word, not the byte, because the PDP-7 was a word-addressed
machine. In practice this meant merely that all programs dealing with character streams ignored null
characters, because null was used to pad a file to an even number of characters.
\end{quotation}
\end{framed}
Une différence mineure était que l'unité d'E/S était le mot, pas l'octet, car le PDP-7 était une machine
adressée par mot. En pratique, cela signifiait que tous les programmes ayant à faire
avec des flux de caractères ignoraient le caractère nul, car nul était utilisé pour compléter un fichier
ayant un nombre impair de caractères.

\myindex{Hiew}

Dans Hiew ou FAR Manager ces chaînes ressemblent à ceci:

\begin{lstlisting}[style=customc]
int main()
{
	printf ("Hello, world!\n");
};
\end{lstlisting}

\begin{figure}[H]
\centering
\includegraphics[width=0.6\textwidth]{digging_into_code/strings/C-string.png}
\caption{Hiew}
\end{figure}

% FIXME видно \n в конце, потом пробел

\subsubsection{Borland Delphi}
\myindex{Pascal}
\myindex{Borland Delphi}

Une chaîne en Pascal et en Delphi de Borland est précédée par sa longueur sur 8-bit
ou 32-bit.

Par exemple:

\begin{lstlisting}[caption=Delphi,style=customasmx86]
CODE:00518AC8                 dd 19h
CODE:00518ACC aLoading___Plea db 'Loading... , please wait.',0

...

CODE:00518AFC                 dd 10h
CODE:00518B00 aPreparingRun__ db 'Preparing run...',0
\end{lstlisting}

\subsubsection{Unicode}

\myindex{Unicode}

Souvent, ce qui est appelé Unicode est la méthode pour encoder des chaînes où chaque
caractère occupe 2 octets ou 16 bits.
Ceci est une erreur de terminologie répandue.
Unicode est un standard pour assigner un nombre à chaque caractère dans un des nombreux
systèmes d'écriture dans le monde, mais ne décrit pas la méthode d'encodage.

\myindex{UTF-8}
\myindex{UTF-16LE}
Les méthodes d'encodage les plus répandues sont: UTF-8 (est répandue sur Internet
et les systèmes *NIX) et UTF-16LE (est utilisé dans Windows).

\myparagraph{UTF-8}

\myindex{UTF-8}
UTF-8 est l'une des méthodes les plus efficace pour l'encodage des caractères.
Tous les symboles Latin sont encodés comme en ASCII, et les symboles après la table
ASCII sont encodés en utilisant quelques octets.
0 est encodé comme avant, donc toutes les fonctions C de chaîne standard fonctionnent
avec des chaînes UTF-8 comme avec tout autre chaîne.

Voyons comment les symboles de divers langages sont encodés en UTF-8 et de quoi ils
ont l'air en FAR, en utilisant la page de code 437%
\footnote{L'exemple et les traductions ont été pris d'ici:
\url{http://go.yurichev.com/17304}}:

\begin{figure}[H]
\centering
\includegraphics[width=0.6\textwidth]{digging_into_code/strings/multilang_sampler.png}
\end{figure}

% FIXME: cut it
\begin{figure}[H]
\centering
\myincludegraphics{digging_into_code/strings/multilang_sampler_UTF8.png}
\caption{FAR: UTF-8}
\end{figure}

Comme vous le voyez, la chaîne en anglais est la même qu'en ASCII.

Le hongrois utilise certains symboles Latin et des symboles avec des signes diacritiques.

Ces symboles sont encodés en utilisant plusieurs octets, qui sont soulignés en rouge.
C'est le même principe avec l'islandais et le polonais.

Il y a aussi le symbole de l'\q{Euro} au début, qui est encodé avec 3 octets.

Les autres systèmes d'écritures n'ont de point commun avec Latin.

Au moins en russe, arabe hébreux et hindi, nous pouvons voir des octets récurrents,
et ce n'est pas une surprise: tous les symboles d'un système d'écriture sont en général
situés dans la même table Unicode, donc leur code débute par le même nombre.

Au début, avant la chaîne \q{How much?}, nous voyons 3 octets, qui sont en fait le
\ac{BOM}. Le \ac{BOM} défini le système d'encodage à utiliser.

\myparagraph{UTF-16LE}

\myindex{UTF-16LE}
\myindex{Windows!Win32}
De nombreuses fonctions win32 de Windows ont le suffixes \TT{-A} et \TT{-W}.
Le premier type de fonctions fonctionne avec les chaînes normales, l'autre, avec
des chaîne UTF-16LE (\emph{large}).

Dans le second cas, chaque symbole est en général stocké dans une valeur 16-bit de
type \emph{short}.

Les symboles Latin dans les chaînes UTF-16 dans Hiew ou FAR semblent être séparés
avec un octet zéro:

\begin{lstlisting}[style=customc]
int wmain()
{
	wprintf (L"Hello, world!\n");
};
\end{lstlisting}

\begin{figure}[H]
\centering
\includegraphics[width=0.6\textwidth]{digging_into_code/strings/UTF16-string.png}
\caption{Hiew}
\end{figure}

Nous voyons souvent ceci dans les fichiers système de \gls{Windows NT}:

\begin{figure}[H]
\centering
\includegraphics[width=0.6\textwidth]{digging_into_code/strings/ntoskrnl_UTF16.png}
\caption{Hiew}
\end{figure}

\myindex{IDA}
Les chaînes avec des caractères qui occupent exactement 2 octets sont appelées \q{Unicode}
dans \IDA:

\begin{lstlisting}[style=customasmx86]
.data:0040E000 aHelloWorld:
.data:0040E000                 unicode 0, <Hello, world!>
.data:0040E000                 dw 0Ah, 0
\end{lstlisting}

Voici comment une chaîne en russe est encodée en UTF-16LE:

\begin{figure}[H]
\centering
\includegraphics[width=0.6\textwidth]{digging_into_code/strings/russian_UTF16.png}
\caption{Hiew: UTF-16LE}
\end{figure}

Ce que nous remarquons facilement, c'est que les symboles sont intercalés par le
caractère diamant (qui a le code ASCII 4). En effet, les symboles cyrilliques sont
situés dans le quatrième plan Unicode.
Ainsi, tous les symboles cyrillique en UTF-16LE sont situés dans l'intervalle \TT{0x400-0x4FF}.

Retournons à l'exemple avec la chaîne écrite dans de multiple langages.
Voici à quoi elle ressemble en UTF-16LE.

% FIXME: cut it
\begin{figure}[H]
\centering
\myincludegraphics{digging_into_code/strings/multilang_sampler_UTF16.png}
\caption{FAR: UTF-16LE}
\end{figure}

Ici nous pouvons aussi voir le \ac{BOM} au début.
Tous les caractères Latin sont intercalés avec un octet à zéro.

Certains caractères avec signe diacritique (hongrois et islandais) sont aussi soulignés en rouge.

% subsection:
\input{digging_into_code/strings/base64_FR}

