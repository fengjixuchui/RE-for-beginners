\section*{Préface}

\subsection*{C'est quoi ces deux titres?}
\label{TwoTitles}

Le livre a été appelé ``Reverse Engineering for Beginners'' en 2014-2018, mais j'ai
toujours suspecté que ça rendait son audience trop réduite.

Les gens de l'infosec connaissent le ``reverse engineering'', mais j'ai rarement
entendu le mot ``assembleur'' de leur part.

De même, le terme ``reverse engineering'' est quelque peu cryptique pour une audience
générale de programmeurs, mais qui ont des connaissances à propos de l'``assembleur''.

En juillet 2018, à titre d'expérience, j'ai changé le titre en ``Assembly Language for Beginners''
et publié le lien sur le site Hacker News\footnote{\url{https://news.ycombinator.com/item?id=17549050}},
et le livre a été plutôt bien accueilli.

Donc, c'est ainsi que le livre a maintenant deux titres.

Toutefois, j'ai changé le second titre à ``Understanding Assembly Language'', car
quelqu'un a déjà écrit le livre ``Assembly Language for Beginners''.
De même, des gens disent que ``for Beginners'' sonne sarcastique pour un livre de
\textasciitilde{}1000 pages.

Les deux livres diffèrent seulement par le titre, le nom du fichier (UAL-XX.pdf versus
RE4B-XX.pdf), l'URL et quelques-une des première pages.

\subsection*{À propos de la rétro-ingénierie}

Il existe plusieurs définitions pour l'expression \q{ingénierie inverse ou rétro-ingénierie \gls{reverse engineering}} :

1) L'ingénierie inverse de logiciels : examiner des programmes compilés;

2) Le balayage des structures en 3D et la manipulation numérique nécessaire afin de les reproduire;

3) Recréer une structure de base de données.

Ce livre concerne la première définition.

\subsection*{Prérequis}

Connaissance basique du C \ac{PL}.
Il est recommandé de lire: \myref{CCppBooks}.

\subsection*{Exercices et tâches}

\dots
ont été déplacés sur un site différent : \url{http://challenges.re}.

\subsection*{A propos de l'auteur}
\begin{tabularx}{\textwidth}{ l X }

\raisebox{-\totalheight}{
\includegraphics[scale=0.60]{Dennis_Yurichev.jpg}
}

&
Dennis Yurichev est un ingénieur expérimenté en rétro-ingénierie et un programmeur.
Il peut être contacté par email : \textbf{\EMAIL{}}.

% FIXME: no link. \tablefootnote doesn't work
\end{tabularx}

% subsections:
\input{praise}
\ifdefined\RUSSIAN
\newcommand{\PeopleMistakesInaccuraciesRusEng}{Станислав \q{Beaver} Бобрицкий, Александр Лысенко, Александр \q{Solar Designer} Песляк, Федерико Рамондино, Марк Уилсон, Ксения Галинская, Разихова Мейрамгуль Кайратовна, Анатолий Прокофьев, Костя Бегунец, Валентин ``netch'' Нечаев, Александр Плахов, Артем Метла, Александр Ястребов, Влад Головкин\footnote{goto-vlad@github}, Евгений Прошин, Александр Мясников}
\else
\newcommand{\PeopleMistakesInaccuraciesRusEng}{Stanislav \q{Beaver} Bobrytskyy, Alexander Lysenko, Alexander \q{Solar Designer} Peslyak, Federico Ramondino, Mark Wilson, Xenia Galinskaya, Razikhova Meiramgul Kayratovna, Anatoly Prokofiev, Kostya Begunets, Valentin ``netch'' Nechayev, Aleksandr Plakhov, Artem Metla, Alexander Yastrebov, Vlad Golovkin\footnote{goto-vlad@github}, Evgeny Proshin, Alexander Myasnikov}
\fi

\newcommand{\PeopleMistakesInaccuracies}{\PeopleMistakesInaccuraciesRusEng{}, Zhu Ruijin, Changmin Heo, Vitor Vidal, Stijn Crevits, Jean-Gregoire Foulon\footnote{\url{https://github.com/pixjuan}}, Ben L., Etienne Khan, Norbert Szetei\footnote{\url{https://github.com/73696e65}}, Marc Remy, Michael Hansen, Derk Barten, The Renaissance\footnote{\url{https://github.com/TheRenaissance}}, Hugo Chan, Emil Mursalimov, Tanner Hoke, Tan90909090@GitHub, Ole Petter Orhagen, Sourav Punoriyar, Vitor Oliveira, Alexis Ehret, Maxim Shlochiski.}

\newcommand{\PeopleItalianTranslators}{Federico Ramondino\footnote{\url{https://github.com/pinkrab}},
Paolo Stivanin\footnote{\url{https://github.com/paolostivanin}}, twyK, Fabrizio Bertone, Matteo Sticco}

\newcommand{\PeopleFrenchTranslators}{Florent Besnard\footnote{\url{https://github.com/besnardf}}, Marc Remy\footnote{\url{https://github.com/mremy}}, Baudouin Landais, Téo Dacquet\footnote{\url{https://github.com/T30rix}}, BlueSkeye@GitHub\footnote{\url{https://github.com/BlueSkeye}}}

\newcommand{\PeopleGermanTranslators}{Dennis Siekmeier\footnote{\url{https://github.com/DSiekmeier}},
Julius Angres\footnote{\url{https://github.com/JAngres}}, Dirk Loser\footnote{\url{https://github.com/PolymathMonkey}}, Clemens Tamme}

\newcommand{\PeopleSpanishTranslators}{Diego Boy, Luis Alberto Espinosa Calvo, Fernando Guida, Diogo Mussi, Patricio Galdames}

\newcommand{\PeoplePTBRTranslators}{Thales Stevan de A. Gois, Diogo Mussi, Luiz Filipe}

\newcommand{\PeoplePolishTranslators}{Kateryna Rozanova, Aleksander Mistewicz, Wiktoria Lewicka}

\newcommand{\PeopleJapaneseTranslators}{shmz@github\footnote{\url{https://github.com/shmz}}}

\newcommand{\FNGithubContributors}{\footnote{\url{\GitHubURL/graphs/contributors}}}

\EN{\subsection*{Thanks}

For patiently answering all my questions: Slava \q{Avid} Kazakov, SkullC0DEr.

For sending me notes about mistakes and inaccuracies: \PeopleMistakesInaccuracies{}

For helping me in other ways:
Andrew Zubinski,
Arnaud Patard (rtp on \#debian-arm IRC),
noshadow on \#gcc IRC,
Aliaksandr Autayeu,
Mohsen Mostafa Jokar,
Peter Sovietov,
Misha ``tiphareth'' Verbitsky.

For translating the book into Simplified Chinese:
Antiy Labs (\href{http://antiy.cn}{antiy.cn}), Archer.

For translating the book into Korean: Byungho Min.

For translating the book into Dutch: Cedric Sambre (AKA Midas).

For translating the book into Spanish: \PeopleSpanishTranslators{}.

For translating the book into Portuguese: \PeoplePTBRTranslators{}.

For translating the book into Italian: \PeopleItalianTranslators{}.

For translating the book into French: \PeopleFrenchTranslators{}.

For translating the book into German: \PeopleGermanTranslators{}.

For translating the book into Polish: \PeoplePolishTranslators{}.

For translating the book into Japanese: \PeopleJapaneseTranslators{}.

For proofreading:
Alexander \q{Lstar} Chernenkiy,
Vladimir Botov,
Andrei Brazhuk,
Mark ``Logxen'' Cooper, Yuan Jochen Kang, Mal Malakov, Lewis Porter, Jarle Thorsen, Hong Xie.

Vasil Kolev\footnote{\url{https://vasil.ludost.net/}} did a great amount of work in proofreading and correcting many mistakes.

Thanks also to all the folks on github.com who have contributed notes and corrections\FNGithubContributors{}.

Many \LaTeX\ packages were used: I would like to thank the authors as well.

\subsubsection*{Donors}

Those who supported me during the time when I wrote significant part of the book:

\input{donors}

Thanks a lot to every donor!
}
\ES{\subsection*{Agradecimientos}

Por contestar pacientemente a todas mis preguntas: Slava \q{Avid} Kazakov, SkullC0DEr.

Por enviarme notas acerca de errores e inexactitudes: \PeopleMistakesInaccuracies{}.

Por ayudarme de otras formas:
Andrew Zubinski,
Arnaud Patard (rtp en \#debian-arm IRC),
noshadow en \#gcc IRC,
Aliaksandr Autayeu,
Mohsen Mostafa Jokar,
Peter Sovietov,
Misha ``tiphareth'' Verbitsky.

Por traducir el libro a Chino Simplificado:
Antiy Labs (\href{http://antiy.cn}{antiy.cn}), Archer.

Por traducir el libro a Coreano: Byungho Min.

\ESph{}: Cedric Sambre (AKA Midas).

\ESph{}: \PeopleSpanishTranslators{}.

\ESph{}: \PeoplePTBRTranslators{}.

\ESph{}: \PeopleItalianTranslators{}.

\ESph{}: \PeopleFrenchTranslators{}.

\DEph{}: \PeopleGermanTranslators{}.

\ac{TBT}: \PeoplePolishTranslators{}.

% TBT
%For translating the book into Japanese: \PeopleJapaneseTranslators{}.

\ES{Por correcci\'on de pruebas}%
Alexander \q{Lstar} Chernenkiy,
Vladimir Botov,
Andrei Brazhuk,
Mark ``Logxen'' Cooper, Yuan Jochen Kang, Mal Malakov, Lewis Porter, Jarle Thorsen, Hong Xie.

Vasil Kolev\footnote{\url{https://vasil.ludost.net/}} realiz\'o una gran cantidad de trabajo en correcci\'on de pruebas y correcci\'on de muchos errores.

Gracias a toda la gente en github.com que ha contribuido con notas y correcciones\FNGithubContributors{}.

Muchos paquetes de \LaTeX\ fueron utiliados: quiero agradecer tambi\'en a sus autores.

\subsubsection*{Donadores}

Aquellos que me apoyaron durante el tiempo que escrib\'i una parte significativa del libro:

\input{donors}

!`Gracias a cada donante!

}
\NL{\subsection*{Dankwoord}

Voor al mijn vragen geduldig te beantwoorden: Slava \q{Avid} Kazakov, SkullC0DEr.

Om me nota\'s over fouten en onnauwkeurigheden toe te sturen: \PeopleMistakesInaccuracies{}.

Om me te helpen op andere manieren:
Andrew Zubinski,
Arnaud Patard (rtp op \#debian-arm IRC),
noshadow op \#gcc IRC,
Aliaksandr Autayeu, Mohsen Mostafa Jokar,
Peter Sovietov,
Misha ``tiphareth'' Verbitsky.

Om het boek te vertalen naar het Vereenvoudigd Chinees:
Antiy Labs (\href{http://antiy.cn}{antiy.cn}), Archer.

Om dit boek te vertalen in het Koreaans: Byungho Min.

\NLph{}: Cedric Sambre (AKA Midas).

\NLph{}: \PeopleSpanishTranslators{}.

\NLph{}: \PeoplePTBRTranslators{}.

\NLph{}: \PeopleItalianTranslators{}.

\NLph{}: \PeopleFrenchTranslators{}.

\NLph{}: \PeopleGermanTranslators{}.

\ac{TBT}: \PeoplePolishTranslators{}.

% TBT
%For translating the book into Japanese: \PeopleJapaneseTranslators{}.

Voor proofreading:
Alexander \q{Lstar} Chernenkiy,
Vladimir Botov,
Andrei Brazhuk,
Mark ``Logxen'' Cooper, Yuan Jochen Kang, Mal Malakov, Lewis Porter, Jarle Thorsen, Hong Xie.

Vasil Kolev\footnote{\url{https://vasil.ludost.net/}}, voor het vele werk in proofreading en het verbeteren van vele fouten.

Dank aan al de mensen op github.com die hebben nota\'s en correcties hebben bijgedragen.

Veel \LaTeX\ packages zijn gebruikt. Ik zou de auteurs hiervan ook graag bedanken.

\subsubsection*{Donaties}

Zij die me gesteund hebben tijdens het schrijven van een groot deel van dit boek:

\input{donors}

Veel dank aan elke donor!
}
\RU{\subsection*{Благодарности}

Тем, кто много помогал мне отвечая на массу вопросов: Слава \q{Avid} Казаков, SkullC0DEr.

Тем, кто присылал замечания об ошибках и неточностях: \PeopleMistakesInaccuracies{}.

Просто помогали разными способами:
Андрей Зубинский,
Arnaud Patard (rtp на \#debian-arm IRC),
noshadow на \#gcc IRC,
Александр Автаев,
Mohsen Mostafa Jokar,
Пётр Советов,
Миша ``tiphareth'' Вербицкий.

Переводчикам на китайский язык:
Antiy Labs (\href{http://antiy.cn}{antiy.cn}), Archer.

Переводчику на корейский язык: Byungho Min.

Переводчику на голландский язык: Cedric Sambre (AKA Midas).

Переводчикам на испанский язык: \PeopleSpanishTranslators{}.

Переводчикам на португальский язык: \PeoplePTBRTranslators{}.

Переводчикам на итальянский язык: \PeopleItalianTranslators{}.

Переводчикам на французский язык: \PeopleFrenchTranslators{}.

Переводчикам на немецкий язык: \PeopleGermanTranslators{}.

Переводчикам на польский язык: \PeoplePolishTranslators{}.

Переводчикам на японский язык: \PeopleJapaneseTranslators{}.

Корректорам:
Александр \q{Lstar} Черненький,
Владимир Ботов,
Андрей Бражук,
Марк ``Logxen'' Купер, Yuan Jochen Kang, Mal Malakov, Lewis Porter, Jarle Thorsen, Hong Xie.

Васил Колев\footnote{\url{https://vasil.ludost.net/}} сделал очень много исправлений и указал на многие ошибки.

И ещё всем тем на github.com кто присылал замечания и исправления\FNGithubContributors{}.

Было использовано множество пакетов \LaTeX. Их авторов я также хотел бы поблагодарить.

\subsubsection*{Жертвователи}

Тем, кто поддерживал меня во время написания этой книги:

\input{donors}

Огромное спасибо каждому!

}
\IT{\subsection*{Ringraziamenti}

Per aver pazientemente risposto a tutte le mie domande: Slava \q{Avid} Kazakov, SkullC0DEr.

Per avermi inviato note riguardo i miei errori e le inaccuratezze: \PeopleMistakesInaccuracies{}.

Per avermi aiutato in altri modi:
Andrew Zubinski,
Arnaud Patard (rtp on \#debian-arm IRC),
noshadow on \#gcc IRC,
Aliaksandr Autayeu,
Mohsen Mostafa Jokar,
Peter Sovietov,
Misha ``tiphareth'' Verbitsky.

Per aver tradotto il libro in Cinese Semplificato:
Antiy Labs (\href{http://antiy.cn}{antiy.cn}), Archer.

Per la traduzione Coreana: Byungho Min.

Per la traduzione in Olandese: Cedric Sambre (AKA Midas).

Per la traduzione in Spagnolo: \PeopleSpanishTranslators{}.

Per la traduzione in Portoghese: \PeoplePTBRTranslators{}.

Per la traduzione Italiana: \PeopleItalianTranslators{}.

Per la traduzione in Francese: \PeopleFrenchTranslators{}.

Per la traduzione in Tedesco: \PeopleGermanTranslators{}.

Per la traduzione in Polacco: \PeoplePolishTranslators{}.

Per la traduzione in Giapponese: \PeopleJapaneseTranslators{}.

Per la revisione:
Alexander \q{Lstar} Chernenkiy,
Vladimir Botov,
Andrei Brazhuk,
Mark ``Logxen'' Cooper, Yuan Jochen Kang, Mal Malakov, Lewis Porter, Jarle Thorsen, Hong Xie.

Vasil Kolev\footnote{\url{https://vasil.ludost.net/}} ha speso una notevole quantità di tempo per la revisione e la correzione di molti errori.

Grazie inoltre a tutti quelli su github.com che hanno contribuito a note e correzioni\FNGithubContributors{}.

Sono stati usati molti pacchetti \LaTeX\ : vorrei ringraziare tutti gli autori di tali moduli.

\subsubsection*{Donatori}

Tutti quelli che mi hanno supportato durante il tempo in cui ho scritto la parte più significativa del libro:

\input{donors}

Grazie di cuore a tutti i donatori!
}
\FR{\subsection*{Remerciements}

Pour avoir patiemment répondu à toutes mes questions : Slava \q{Avid} Kazakov, SkullC0DEr.

Pour m'avoir fait des remarques par rapport à mes erreurs ou manques de précision : \PeopleMistakesInaccuracies{}.

Pour m'avoir aidé de toute autre manière :
Andrew Zubinski,
Arnaud Patard (rtp on \#debian-arm IRC),
noshadow on \#gcc IRC,
Aliaksandr Autayeu,
Mohsen Mostafa Jokar,
Peter Sovietov,
Misha ``tiphareth'' Verbitsky.

Pour avoir traduit le livre en chinois simplifié :
Antiy Labs (\href{http://antiy.cn}{antiy.cn}), Archer.

Pour avoir traduit le livre en coréen : Byungho Min.

Pour avoir traduit le livre en néerlandais : Cedric Sambre (AKA Midas).

Pour avoir traduit le livre en espagnol : \PeopleSpanishTranslators{}.

Pour avoir traduit le livre en portugais : \PeoplePTBRTranslators{}.

Pour avoir traduit le livre en italien : \PeopleItalianTranslators{}.

Pour avoir traduit le livre en français : \PeopleFrenchTranslators{}.

Pour avoir traduit le livre en allemand : \PeopleGermanTranslators{}.

Pour avoir traduit le livre en polonais: \PeoplePolishTranslators{}.

Pour avoir traduit le livre en japonais: \PeopleJapaneseTranslators{}.

Pour la relecture :
Alexander \q{Lstar} Chernenkiy,
Vladimir Botov,
Andrei Brazhuk,
Mark ``Logxen'' Cooper, Yuan Jochen Kang, Mal Malakov, Lewis Porter, Jarle Thorsen, Hong Xie.

Vasil Kolev\footnote{\url{https://vasil.ludost.net/}} a réalisé un gros travail de relecture et a corrigé beaucoup d'erreurs.

Merci également à toutes les personnes sur github.com qui ont contribué aux remarques et aux corrections\FNGithubContributors{}.

De nombreux packages \LaTeX\ ont été utilisé : j'aimerais également remercier leurs auteurs.

\subsubsection*{Donateurs}

Ceux qui m'ont soutenu lorsque j'écrivais le livre :

\input{donors}

Un énorme merci à chaque donateur !
}
\DE{\subsection*{Danksagung}

Für das geduldige Beantworten aller meiner Fragen: Slava \q{Avid} Kazakov, SkullC0DEr.

Für Anmerkungen über Fehler und Unstimmigkeiten: \PeopleMistakesInaccuracies{}.

Für die Hilfe in anderen Dingen:
Andrew Zubinski,
Arnaud Patard (rtp on \#debian-arm IRC),
noshadow on \#gcc IRC,
Aliaksandr Autayeu,
Mohsen Mostafa Jokar,
Peter Sovietov,
Misha ``tiphareth'' Verbitsky.

Für die Übersetzung des Buchs ins Vereinfachte Chinesisch:
Antiy Labs (\href{http://antiy.cn}{antiy.cn}), Archer.

Für die Übersetzung des Buchs ins Koreanische: Byungho Min.

Für die Übersetzung des Buchs ins Niederländische: Cedric Sambre (AKA Midas).

Für die Übersetzung des Buchs ins Spanische: \PeopleSpanishTranslators{}.

Für die Übersetzung des Buchs ins Portugiesische: \PeoplePTBRTranslators{}.

Für die Übersetzung des Buchs ins Italienische: \PeopleItalianTranslators{}.

Für die Übersetzung des Buchs ins Französische: \PeopleFrenchTranslators{}.

Für die Übersetzung des Buchs ins Deutsche: \PeopleGermanTranslators{}.

Für die Übersetzung des Buchs ins Polnische: \PeoplePolishTranslators{}.

Für die Übersetzung des Buchs ins Japanische: \PeopleJapaneseTranslators{}.

Für das Korrekturlesen:
Alexander \q{Lstar} Chernenkiy,
Vladimir Botov,
Andrei Brazhuk,
Mark ``Logxen'' Cooper, Yuan Jochen Kang, Mal Malakov, Lewis Porter, Jarle Thorsen, Hong Xie.

Vasil Kolev\footnote{\url{https://vasil.ludost.net/}} der unglaublich viel Arbeit in die Korrektur vieler Fehler investiert hat.

Danke auch an alle, die auf github.com Anmerkungen und Korrekturen eingebracht haben.

Es wurden viele \LaTeX\-Pakete genutzt: Vielen Dank an deren Autoren.

\subsubsection*{Donors}

Dank an diejenigen die mich während der Zeit in der ich wichtige Teile des Buchs geschrieben habe
unterstützt haben:

\input{donors}

Vielen Dank an alle Spender!
}
%\CN{% !TEX program = XeLaTeX
% !TEX encoding = UTF-8
\documentclass[UTF8,nofonts]{ctexart}
\setCJKmainfont[BoldFont=STHeiti,ItalicFont=STKaiti]{STSong}
\setCJKsansfont[BoldFont=STHeiti]{STXihei}
\setCJKmonofont{STFangsong}

\begin{document}

%daveti: translated on Dec 28, 2016
%NOTE: above is needed for MacTex.

\subsection*{感谢 Thanks}

耐心回答我每一个问题的人: Slava \q{Avid} Kazakov, SkullC0DEr.

为本书勘误的人: \PeopleMistakesInaccuracies{}.

其他形式帮忙的人:
Andrew Zubinski,
Arnaud Patard (rtp on \#debian-arm IRC),
noshadow on \#gcc IRC,
Aliaksandr Autayeu,
Mohsen Mostafa Jokar,
Peter Sovietov,
Misha ``tiphareth'' Verbitsky.

中文翻译~\footnote{译者语:可能是之前的那个epub版本,反正不是俺}:
Antiy Labs (\href{http://antiy.cn}{antiy.cn}), Archer.

韩语翻译: Byungho Min.

荷兰语翻译: Cedric Sambre (AKA Midas).

西班牙语翻译: \PeopleSpanishTranslators{}.

葡萄牙语翻译: \PeoplePTBRTranslators{}.

意大利语翻译: \PeopleItalianTranslators{}.

法语翻译: \PeopleFrenchTranslators{}.

德语翻译: \PeopleGermanTranslators{}.

\ac{TBT}: \PeoplePolishTranslators{}.

% TBT
%For translating the book into Japanese: \PeopleJapaneseTranslators{}.

试读的人:
Alexander \q{Lstar} Chernenkiy,
Vladimir Botov,
Andrei Brazhuk,
Mark ``Logxen'' Cooper, Yuan Jochen Kang, Mal Malakov, Lewis Porter, Jarle Thorsen, Hong Xie.

Vasil Kolev\footnote{\url{https://vasil.ludost.net/}}做了大量的试读和纠错工作。

感谢github.com所有提出建议或者纠错的人\FNGithubContributors{}.

本书使用了很多\LaTeX\ 包: 我也感谢这些软件的作者.

\subsubsection*{捐献者 Donors}

那些在我写书时提供各种帮助的人:

\input{donors}

感谢你们的每一份捐助!

\end{document}
}
\JA{\subsection*{謝辞}

忍耐強く質問に答えてくれた方々:Slava \q{Avid} Kazakov, SkullC0DEr

ミスや不正確な記述を指摘してくれた方々:\PeopleMistakesInaccuracies{}

他に手助けしてくれた方々:
Andrew Zubinski,
Arnaud Patard (rtp on \#debian-arm IRC),
noshadow on \#gcc IRC,
Aliaksandr Autayeu,
Mohsen Mostafa Jokar,
Peter Sovietov,
Misha ``tiphareth'' Verbitsky.

簡体字中国語への翻訳:Antiy Labs(\href{http://antiy.cn}{antiy.cn})、Archer

韓国語への翻訳:Byungho Min

オランダ語への翻訳:Cedric Sambre(AKA Midas)

スペイン語への翻訳: \PeopleSpanishTranslators{}

ポルトガル語への翻訳:\PeoplePTBRTranslators{}

イタリア語への翻訳:\PeopleItalianTranslators{}

フランス語への翻訳:\PeopleFrenchTranslators{}

ドイツ語への翻訳:\PeopleGermanTranslators{}

ポーランド語への翻訳:\PeoplePolishTranslators{}

日本語への翻訳:\PeopleJapaneseTranslators{}.

校正者:
Alexander \q{Lstar} Chernenkiy,
Vladimir Botov,
Andrei Brazhuk,
Mark ``Logxen'' Cooper, Yuan Jochen Kang, Mal Malakov, Lewis Porter, Jarle Thorsen, Hong Xie.

Vasil Kolev\footnote{\url{https://vasil.ludost.net/}} は大量の校正とミスを訂正してくれました。

github.comでフォークして指摘や訂正してくれたすべての方々に感謝します.

\LaTeX\ パッケージをたくさん使っています:パッケージの作者にも合わせて感謝します

\subsubsection*{ドナー}

本の大半を執筆中にサポートしてくれた方々:

\input{donors}

ドナーの方すべてに感謝します!
}

\subsection*{mini-FAQ}

\par Q: Est-ce que ce livre est plus simple/facile que les autres?
\par R: Non, c'est à peu près le même niveau que les autres livres sur ce sujet.

\par Q: Quels sont les pré-requis nécessaires avant de lire ce livre ?
\par R: Une compréhension de base du C/C++ serait l'idéal.

\par Q: Dois-je apprendre x86/x64/ARM et MIPS en même temps ? N'est-ce pas un peu trop ?
\par R: Je pense que les débutants peuvent seulement lire les parties x86/x64, tout en passant/feuilletant celles ARM/MIPS.

\par Q: Puis-je acheter une version papier du livre en russe / anglais ?
\par R: Malheureusement non, aucune maison d'édition n'a été intéressée pour publier une version en russe ou en anglais du livre jusqu'à présent.
Cependant, vous pouvez demander à votre imprimerie préférée de l'imprimer et de le relier.

\par Q: Y a-il une version ePub/Mobi ?
\par R: Le livre dépend majoritairement de TeX/LaTeX, il n'est donc pas évident de le convertir en version ePub/Mobi.

\par Q: Pourquoi devrait-on apprendre l'assembleur de nos jours ?
\par R: A moins d'être un développeur d'\ac{OS}, vous n'aurez probablement pas besoin d'écrire en assembleur\textemdash{}les derniers compilateurs (ceux de notre décennie) sont meilleurs que les êtres humains en terme d'optimisation. \footnote{Un très bon article à ce sujet : \InSqBrackets{\AgnerFog}}.

De plus, les derniers \ac{CPU}s sont des appareils complexes et la connaissance de l'assembleur n'aide pas vraiment à comprendre leurs mécanismes internes.

Cela dit, il existe au moins deux domaines dans lesquels une bonne connaissance de l'assembleur peut être utile : 
Tout d'abord, pour de la recherche en sécurité ou sur des malwares. C'est également un bon moyen de comprendre un code compilé lorsqu'on le debug.
Ce livre est donc destiné à ceux qui veulent comprendre l'assembleur plutôt que d'écrire en assembleur, ce qui explique pourquoi il y a de nombreux exemples de résultats issus de compilateurs dans ce livre. 

\par Q: J'ai cliqué sur un lien dans le document PDF, comment puis-je retourner en arrière ?
\par R: Dans Adobe Acrobat Reader, appuyez sur Alt + Flèche gauche. Dans Evince, appuyez sur le bouton ``<''.

\par Q: Puis-je imprimer ce livre / l'utiliser pour de l'enseignement ?
\par R: Bien sûr ! C'est la raison pour laquelle le livre est sous licence Creative Commons (CC BY-SA 4.0).

\par Q: Pourquoi ce livre est-il gratuit ? Vous avez fait du bon boulot. C'est suspect, comme nombre de choses gratuites.
\par R: D'après ma propre expérience, les auteurs d'ouvrages techniques font cela pour l'auto-publicité. Il n'est pas possible de se faire beaucoup d'argent d'une telle manière.

\par Q: Comment trouver du travail dans le domaine de la rétro-ingénierie ?
\par R: Il existe des sujets d'embauche qui apparaissent de temps en temps sur Reddit, dédiés à la rétro-ingénierie (cf. reverse engineering ou RE)\FNURLREDDIT{}.
Jetez un \oe{}il ici.

Un sujet d'embauche quelque peu lié peut être trouvé dans le subreddit \q{netsec}.

\par Q: J'ai une question...
\par R: Envoyez-la moi par email (\EMAIL).



\subsection*{À propos de la traduction en Coréen}

En Janvier 2015, la maison d'édition Acorn (\href{http://www.acornpub.co.kr}{www.acornpub.co.kr}) en Corée du Sud a réalisé un énorme travail en traduisant et en publiant mon livre (dans son état en Août 2014) en Coréen.

Il est désormais disponible sur \href{http://go.yurichev.com/17343}{leur site web}.

\iffalse
\begin{figure}[H]
\centering
\includegraphics[scale=0.3]{acorn_cover.jpg}
\end{figure}
\fi

Le traducteur est Byungho Min (\href{http://go.yurichev.com/17344}{twitter/tais9}).
L'illustration de couverture a été réalisée l'artiste, Andy Nechaevsky, un ami de l'auteur:
\href{http://go.yurichev.com/17023}{facebook/andydinka}.
Ils détiennent également les droits d'auteurs sur la traduction coréenne.

Donc si vous souhaitez avoir un livre \emph{réel} en coréen sur votre étagère et que vous souhaitez soutenir ce travail, il est désormais disponible à l'achat.

\subsection*{Á propos de la traduction en Farsi/Perse}

En 2016, ce livre a été traduit par Mohsen Mostafa Jokar (qui est aussi connu dans
la communauté iranienne pour sa traduction du manuel de Radare\footnote{\url{http://rada.re/get/radare2book-persian.pdf}}).
Il est disponible sur le site web de l'éditeur\footnote{\url{http://goo.gl/2Tzx0H}}
(Pendare Pars).

Extrait de 40 pages: \url{https://beginners.re/farsi.pdf}.

Enregistrement du livre à la Bibliothèque Nationale d'Iran: \url{http://opac.nlai.ir/opac-prod/bibliographic/4473995}.

\subsection*{Á propos de la traduction en Chinois}

En avril 2017, la traduction en Chinois a été terminée par Chinese PTPress. Ils sont
également les détenteurs des droits de la traduction en Chinois.

La version chinoise est disponible à l'achat ici: \url{http://www.epubit.com.cn/book/details/4174}.
Une revue partielle et l'historique de la traduction peut être trouvé ici: \url{http://www.cptoday.cn/news/detail/3155}.


Le traducteur principal est Archer, à qui je dois beaucoup. Il a été très méticuleux
(dans le bon sens du terme) et a signalé la plupart des erreurs et bugs connus, ce
qui est très important dans le genre de littérature de ce livre.
Je recommanderais ses services à tout autre auteur!

Les gens de \href{http://www.antiy.net/}{Antiy Labs} ont aussi aidé pour la traduction.
\href{http://www.epubit.com.cn/book/onlinechapter/51413}{Voici la préface} écrite par eux.

