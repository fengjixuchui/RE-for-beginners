\documentclass[a4paper,oneside]{book}

% http://www.tex.ac.uk/FAQ-noroom.html
\usepackage{etex}

\usepackage[table,usenames,dvipsnames]{xcolor}

\usepackage{fontspec}
% fonts
%\setmonofont{DroidSansMono}
%\setmainfont[Ligatures=TeX]{PT Sans}
%\setmainfont{DroidSans}
\setmainfont{DejaVu Sans}
\setmonofont{DejaVu Sans Mono}
\usepackage{polyglossia}
\defaultfontfeatures{Scale=MatchLowercase} % ensure all fonts have the same 1ex
\usepackage{ucharclasses}
\usepackage{csquotes}

\ifdefined\ENGLISH
%\wlog{main ENGLISH defined OK}
\setmainlanguage{english}
\setotherlanguage{russian}
\fi

\ifdefined\RUSSIAN
\setmainlanguage{russian}
%\newfontfamily\cyrillicfont{LiberationSans}
%\newfontfamily\cyrillicfonttt{LiberationMono}
%\newfontfamily\cyrillicfontsf{lmsans10-regular.otf}
\setotherlanguage{english}
\fi

\ifdefined\GERMAN
%\wlog{main GERMAN defined OK}
\setmainlanguage{german}
\setotherlanguage{english}
\fi

\ifdefined\SPANISH
\setmainlanguage{spanish}
\setotherlanguage{english}
\fi

\ifdefined\ITALIAN
\setmainlanguage{italian}
\setotherlanguage{english}
\fi

\ifdefined\BRAZILIAN
\setmainlanguage{portuges}
\setotherlanguage{english}
\fi

\ifdefined\POLISH
\setmainlanguage{polish}
\setotherlanguage{english}
\fi

\ifdefined\DUTCH
\setmainlanguage{dutch}
\setotherlanguage{english}
\fi

\ifdefined\TURKISH
\setmainlanguage{turkish}
\setotherlanguage{english}
\fi

\ifdefined\THAI
\setmainlanguage{thai}
%\usepackage[thai]{babel}
%\usepackage{fonts-tlwg}
\setmainfont[Script=Thai]{TH SarabunPSK}
\newfontfamily{\thaifont}[Script=Thai]{TH SarabunPSK}
\let\thaifonttt\ttfamily
\setotherlanguage{english}
\fi

\ifdefined\FRENCH
\setmainlanguage{french}
\setotherlanguage{english}
\fi

\ifdefined\JAPANESE
\usepackage{xeCJK}
\xeCJKallowbreakbetweenpuncts
\defaultfontfeatures{Ligatures=TeX,Scale=MatchLowercase}
\setCJKmainfont{IPAGothic}
\setCJKsansfont{IPAGothic}
\setCJKmonofont{IPAGothic}
\DeclareQuoteStyle{japanese}
  {「}
  {」}
  {『}
  {』}
\setquotestyle{japanese}
\setmainlanguage{japanese}
\setotherlanguage{english}
\fi

\usepackage{microtype}
\usepackage{fancyhdr}
\usepackage{listings}
%\usepackage{ulem} % used for \sout{}...
\usepackage{url}
\usepackage{graphicx}
\usepackage{makeidx}
\usepackage[cm]{fullpage}
%\usepackage{color}
\usepackage{fancyvrb}
\usepackage{xspace}
\usepackage{tabularx}
\usepackage{framed}
\usepackage{parskip}
\usepackage{epigraph}
\usepackage{ccicons}
\usepackage[nottoc]{tocbibind}
\usepackage{longtable}
\usepackage[footnote,printonlyused,withpage]{acronym}
\usepackage[]{bookmark,hyperref} % must be last
\usepackage[official]{eurosym}
\usepackage[usestackEOL]{stackengine}

% ************** myref
% http://tex.stackexchange.com/questions/228286/how-to-mix-ref-and-pageref#228292
\ifdefined\RUSSIAN
\newcommand{\myref}[1]{%
  \ref{#1}
  (стр.~\pageref{#1})%
  }
% FIXME: I wasn't able to force varioref to output russian text...
\else
\usepackage{varioref}
\newcommand{\myref}[1]{\vref{#1}}
\fi
% ************** myref

\usepackage{glossaries}
\usepackage{tikz}
%\usepackage{fixltx2e}
\usepackage{bytefield}

\usepackage{amsmath}
\usepackage{MnSymbol}
\undef\mathdollar

\usepackage{float}

\usepackage{shorttoc}
\usetikzlibrary{calc,positioning,chains,arrows}
\usepackage[margin=0.5in,headheight=15.5pt]{geometry}
% FIXME would be good without "margin="...
%\usepackage[headheight=15.5pt]{geometry}

%--------------------
% to prevent clashing of numbers and titles in TOC:
% https://tex.stackexchange.com/a/64124
\usepackage{tocloft}% http://ctan.org/pkg/tocloft
\makeatletter
\renewcommand{\numberline}[1]{%
  \@cftbsnum #1\@cftasnum~\@cftasnumb%
}
\makeatother
%--------------------

\newcommand{\footnoteref}[1]{\textsuperscript{\ref{#1}}}

%\definecolor{lstbgcolor}{rgb}{0.94,0.94,0.94}

% I don't know why this voodoo works, but without all-caps, it can't find LIGHT-GRAY color. WTF?
% see also: https://tex.stackexchange.com/questions/64298/error-with-xcolor-package
\definecolor{light-gray}{gray}{0.87}
\definecolor{LIGHT-GRAY}{gray}{0.87}
\definecolor{RED}{rgb}{1,0,0}
\makeindex

\newcommand{\GitHubURL}{https://github.com/DennisYurichev/RE-for-beginners}
\newcommand{\GitHubBlobMasterURL}{\GitHubURL/blob/master}
\newcommand{\GitHubTreeMasterURL}{\GitHubURL/tree/master}

\newcommand{\ESph}{\ES{{\color{red}Spanish text placeholder}}}
\newcommand{\PTBRph}{\PTBR{{\color{red}Brazilian Portuguese text placeholder}}}
\newcommand{\PLph}{\PL{{\color{red}Polish text placeholder}}}
\newcommand{\ITph}{\IT{{\color{red}Italian text placeholder}}}
\newcommand{\DEph}{\DE{{\color{red}German text placeholder}}}
\newcommand{\THAph}{\THA{{\color{red}Thai text placeholder}}}
\newcommand{\NLph}{\NL{{\color{red}Dutch text placeholder}}}
\newcommand{\FRph}{\FR{{\color{red}French text placeholder}}}
\newcommand{\JAph}{\JA{{\color{red}Japanese text placeholder}}}
\newcommand{\TRph}{\TR{{\color{red}Turkish text placeholder}}}

\newcommand{\myhrule}{\begin{center}\rule{0.5\textwidth}{.4pt}\end{center}}

% TODO: find better name:
\newcommand{\myincludegraphics}[1]{\includegraphics[width=\textwidth]{#1}}
\newcommand{\myincludegraphicsSmall}[1]{\includegraphics[width=0.3\textwidth]{#1}}

\newcommand{\myincludegraphicsSmallOrNormalForEbook}{\myincludegraphicsSmall}

% advertisement:
%\ifdefined\ENGLISH{}
%\newcommand{\mysection}[2][]{
%    \clearpage\input{ad_EN}\clearpage
%    \ifx&#1& \section{#2}
%    \else    \section[#1]{#2}
%    \fi
%}
%\else
\newcommand{\mysection}[2][]{
    \ifx&#1& \section{#2}
    \else    \section[#1]{#2}
    \fi
}
%\fi

%\newcommand*{\TT}[1]{\texttt{#1}}
% synonyms, so far
\newcommand*{\InSqBrackets}[1]{\lbrack{}#1\rbrack{}}

% too contrast text?
%\newcommand*{\TT}[1]{\colorbox{light-gray}{\texttt{#1}}}
%\newcommand*{\GTT}[1]{\colorbox{light-gray}{\texttt{#1}}}
\newcommand*{\TT}[1]{\texttt{#1}}
\newcommand*{\GTT}[1]{\texttt{#1}}

%\newcommand*{\EN}[1]{\iflanguage{english}{#1}{}}

\newcommand{\HeaderColor}{\cellcolor{blue!25}}

\ifdefined\ENGLISH{}
\newcommand*{\EN}[1]{#1}
\else
\newcommand*{\EN}[1]{}
\fi

\ifdefined\RUSSIAN{}
%\newcommand*{\RU}[1]{\iflanguage{russian}{#1}{}}
\newcommand*{\RU}[1]{#1}
\else
\newcommand*{\RU}[1]{}
\fi

\ifdefined\CHINESE{}
%\newcommand*{\CN}[1]{\iflanguage{chinese}{#1}{}}
\newcommand*{\CN}[1]{#1}
\else
\newcommand*{\CN}[1]{}
\fi

\ifdefined\SPANISH{}
%\newcommand*{\ES}[1]{\iflanguage{spanish}{#1}{}}
\newcommand*{\ES}[1]{#1}
\else
\newcommand*{\ES}[1]{}
\fi

\ifdefined\ITALIAN{}
%\newcommand*{\IT}[1]{\iflanguage{italian}{#1}{}}
\newcommand*{\IT}[1]{#1}
\else
\newcommand*{\IT}[1]{}
\fi

\ifdefined\BRAZILIAN{}
%\newcommand*{\PTBR}[1]{\iflanguage{portuges}{#1}{}}
\newcommand*{\PTBR}[1]{#1}
\else
\newcommand*{\PTBR}[1]{}
\fi

\ifdefined\POLISH{}
%\newcommand*{\PL}[1]{\iflanguage{polish}{#1}{}}
\newcommand*{\PL}[1]{#1}
\else
\newcommand*{\PL}[1]{}
\fi

\ifdefined\GERMAN{}
%\newcommand*{\DE}[1]{\iflanguage{german}{#1}{}}
\newcommand*{\DE}[1]{#1}
\else
\newcommand*{\DE}[1]{}
\fi

\ifdefined\THAI{}
%\newcommand*{\THA}[1]{\iflanguage{thai}{#1}{}}
\newcommand*{\THA}[1]{#1}
\else
\newcommand*{\THA}[1]{}
\fi

\ifdefined\DUTCH{}
%\newcommand*{\NL}[1]{\iflanguage{dutch}{#1}{}}
\newcommand*{\NL}[1]{#1}
\else
\newcommand*{\NL}[1]{}
\fi

\ifdefined\FRENCH{}
%\newcommand*{\FR}[1]{\iflanguage{french}{#1}{}}
\newcommand*{\FR}[1]{#1}
\else
\newcommand*{\FR}[1]{}
\fi

\ifdefined\JAPANESE{}
%\newcommand*{\JPN}[1]{\iflanguage{japanese}{#1}{}}
\newcommand*{\JA}[1]{#1}
\else
\newcommand*{\JA}[1]{}
\fi

\ifdefined\TURKISH{}
%\newcommand*{\TR}[1]{\iflanguage{turkish}{#1}{}}
\newcommand*{\TR}[1]{#1}
\else
\newcommand*{\TR}[1]{}
\fi



%\newcommand{\EMAIL}{dennis(@)yurichev.com}
\newcommand{\EMAIL}{dennis@yurichev.com}

\newcommand*{\dittoclosing}{---''---}
\newcommand*{\AsteriskOne}{${}^{*}$}
\newcommand*{\AsteriskTwo}{${}^{**}$}
\newcommand*{\AsteriskThree}{${}^{***}$}

% http://www.ctan.org/pkg/csquotes
\newcommand{\q}[1]{\enquote{#1}}

\newcommand{\var}[1]{\textit{#1}}

\newcommand{\ttf}{\GTT{f()}\xspace}
\newcommand{\ttfone}{\GTT{f1()}\xspace}

% http://tex.stackexchange.com/questions/32160/new-line-after-paragraph
\newcommand{\myparagraph}[1]{\paragraph{#1}\mbox{}\\}
\newcommand{\mysubparagraph}[1]{\subparagraph{#1}\mbox{}\\}

\newcommand{\figref}[1]{\figname{}\ref{#1}\xspace}
\newcommand{\lstref}[1]{\listingname{}\ref{#1}\xspace}
\newcommand{\MacOSX}{Mac OS X\xspace}

% FIXME TODO non-overlapping color!
% \newcommand{\headercolor}{\cellcolor{blue!25}}
\newcommand{\headercolor}{}

% FIXME: get rid of:
\newcommand{\IDA}{\ac{IDA}\xspace}

% FIXME: get rid of:
\newcommand{\tracer}{\protect\gls{tracer}\xspace}

\newcommand{\Tchar}{\emph{char}\xspace}
\newcommand{\Tint}{\emph{int}\xspace}
\newcommand{\Tbool}{\emph{bool}\xspace}
\newcommand{\Tfloat}{\emph{float}\xspace}
\newcommand{\Tdouble}{\emph{double}\xspace}
\newcommand{\Tvoid}{\emph{void}\xspace}
\newcommand{\ITthis}{\emph{this}\xspace}

\newcommand{\Ox}{\GTT{/Ox}\xspace}
\newcommand{\Obzero}{\GTT{/Ob0}\xspace}
\newcommand{\Othree}{\GTT{-O3}\xspace}

\newcommand{\oracle}{Oracle RDBMS\xspace}

\newcommand{\idevices}{iPod/iPhone/iPad\xspace}
\newcommand{\olly}{OllyDbg\xspace}

% common C functions
\newcommand{\printf}{\GTT{printf()}\xspace}
\newcommand{\puts}{\GTT{puts()}\xspace}
\newcommand{\main}{\GTT{main()}\xspace}
\newcommand{\qsort}{\GTT{qsort()}\xspace}
\newcommand{\strlen}{\GTT{strlen()}\xspace}
\newcommand{\scanf}{\GTT{scanf()}\xspace}
\newcommand{\rand}{\GTT{rand()}\xspace}


% for easier fiddling with formatting of all instructions together
\newcommand{\INS}[1]{\GTT{#1}\xspace}

% x86 instructions
\newcommand{\ADD}{\INS{ADD}}
\newcommand{\ADRP}{\INS{ADRP}}
\newcommand{\AND}{\INS{AND}}
\newcommand{\CALL}{\INS{CALL}}
\newcommand{\CPUID}{\INS{CPUID}}
\newcommand{\CMP}{\INS{CMP}}
\newcommand{\DEC}{\INS{DEC}}
\newcommand{\FADDP}{\INS{FADDP}}
\newcommand{\FCOM}{\INS{FCOM}}
\newcommand{\FCOMP}{\INS{FCOMP}}
\newcommand{\FCOMI}{\INS{FCOMI}}
\newcommand{\FCOMIP}{\INS{FCOMIP}}
\newcommand{\FUCOM}{\INS{FUCOM}}
\newcommand{\FUCOMI}{\INS{FUCOMI}}
\newcommand{\FUCOMIP}{\INS{FUCOMIP}}
\newcommand{\FUCOMPP}{\INS{FUCOMPP}}
\newcommand{\FDIVR}{\INS{FDIVR}}
\newcommand{\FDIV}{\INS{FDIV}}
\newcommand{\FLD}{\INS{FLD}}
\newcommand{\FMUL}{\INS{FMUL}}
\newcommand{\MUL}{\INS{MUL}}
\newcommand{\FSTP}{\INS{FSTP}}
\newcommand{\FDIVP}{\INS{FDIVP}}
\newcommand{\IDIV}{\INS{IDIV}}
\newcommand{\IMUL}{\INS{IMUL}}
\newcommand{\INC}{\INS{INC}}
\newcommand{\JAE}{\INS{JAE}}
%\newcommand{\JA}{\INS{JA}} % now used for Japanese text
\newcommand{\JBE}{\INS{JBE}}
\newcommand{\JB}{\INS{JB}}
\newcommand{\JE}{\INS{JE}}
\newcommand{\JGE}{\INS{JGE}}
\newcommand{\JG}{\INS{JG}}
\newcommand{\JLE}{\INS{JLE}}
\newcommand{\JL}{\INS{JL}}
\newcommand{\JMP}{\INS{JMP}}
\newcommand{\JNE}{\INS{JNE}}
\newcommand{\JNZ}{\INS{JNZ}}
\newcommand{\JNA}{\INS{JNA}}
\newcommand{\JNAE}{\INS{JNAE}}
\newcommand{\JNB}{\INS{JNB}}
\newcommand{\JNBE}{\INS{JNBE}}
\newcommand{\JZ}{\INS{JZ}}
\newcommand{\JP}{\INS{JP}}
\newcommand{\Jcc}{\INS{Jcc}}
\newcommand{\SETcc}{\INS{SETcc}}
\newcommand{\LEA}{\INS{LEA}}
\newcommand{\LOOP}{\INS{LOOP}}
\newcommand{\MOVSX}{\INS{MOVSX}}
\newcommand{\MOVZX}{\INS{MOVZX}}
\newcommand{\MOV}{\INS{MOV}}
\newcommand{\NOP}{\INS{NOP}}
\newcommand{\POP}{\INS{POP}}
\newcommand{\PUSH}{\INS{PUSH}}
\newcommand{\NOT}{\INS{NOT}}
\newcommand{\NOR}{\INS{NOR}}
\newcommand{\RET}{\INS{RET}}
\newcommand{\RETN}{\INS{RETN}}
\newcommand{\SETNZ}{\INS{SETNZ}}
\newcommand{\SETBE}{\INS{SETBE}}
\newcommand{\SETNBE}{\INS{SETNBE}}
\newcommand{\SUB}{\INS{SUB}}
\newcommand{\TEST}{\INS{TEST}}
\newcommand{\TST}{\INS{TST}}
\newcommand{\FNSTSW}{\INS{FNSTSW}}
\newcommand{\SAHF}{\INS{SAHF}}
\newcommand{\XOR}{\INS{XOR}}
\newcommand{\OR}{\INS{OR}}
\newcommand{\SHL}{\INS{SHL}}
\newcommand{\SHR}{\INS{SHR}}
\newcommand{\SAR}{\INS{SAR}}
\newcommand{\LEAVE}{\INS{LEAVE}}
\newcommand{\MOVDQA}{\INS{MOVDQA}}
\newcommand{\MOVDQU}{\INS{MOVDQU}}
\newcommand{\PADDD}{\INS{PADDD}}
\newcommand{\PCMPEQB}{\INS{PCMPEQB}}
\newcommand{\LDR}{\INS{LDR}}
\newcommand{\LSL}{\INS{LSL}}
\newcommand{\LSR}{\INS{LSR}}
\newcommand{\ASR}{\INS{ASR}}
\newcommand{\RSB}{\INS{RSB}}
\newcommand{\BTR}{\INS{BTR}}
\newcommand{\BTS}{\INS{BTS}}
\newcommand{\BTC}{\INS{BTC}}
\newcommand{\LUI}{\INS{LUI}}
\newcommand{\ORI}{\INS{ORI}}
\newcommand{\BIC}{\INS{BIC}}
\newcommand{\EOR}{\INS{EOR}}
\newcommand{\MOVS}{\INS{MOVS}}
\newcommand{\LSLS}{\INS{LSLS}}
\newcommand{\LSRS}{\INS{LSRS}}
\newcommand{\FMRS}{\INS{FMRS}}
\newcommand{\CMOVNE}{\INS{CMOVNE}}
\newcommand{\CMOVNZ}{\INS{CMOVNZ}}
\newcommand{\ROL}{\INS{ROL}}
\newcommand{\CSEL}{\INS{CSEL}}
\newcommand{\SLL}{\INS{SLL}}
\newcommand{\SLLV}{\INS{SLLV}}
\newcommand{\SW}{\INS{SW}}
\newcommand{\LW}{\INS{LW}}

% x86 flags

\newcommand{\ZF}{\GTT{ZF}\xspace}
\newcommand{\CF}{\GTT{CF}\xspace}
\newcommand{\PF}{\GTT{PF}\xspace}

% x86 registers

\newcommand{\AL}{\GTT{AL}\xspace}
\newcommand{\AH}{\GTT{AH}\xspace}
\newcommand{\AX}{\GTT{AX}\xspace}
\newcommand{\EAX}{\GTT{EAX}\xspace}
\newcommand{\EBX}{\GTT{EBX}\xspace}
\newcommand{\ECX}{\GTT{ECX}\xspace}
\newcommand{\EDX}{\GTT{EDX}\xspace}
\newcommand{\DL}{\GTT{DL}\xspace}
\newcommand{\ESI}{\GTT{ESI}\xspace}
\newcommand{\EDI}{\GTT{EDI}\xspace}
\newcommand{\EBP}{\GTT{EBP}\xspace}
\newcommand{\ESP}{\GTT{ESP}\xspace}
\newcommand{\RSP}{\GTT{RSP}\xspace}
\newcommand{\EIP}{\GTT{EIP}\xspace}
\newcommand{\RIP}{\GTT{RIP}\xspace}
\newcommand{\RAX}{\GTT{RAX}\xspace}
\newcommand{\RBX}{\GTT{RBX}\xspace}
\newcommand{\RCX}{\GTT{RCX}\xspace}
\newcommand{\RDX}{\GTT{RDX}\xspace}
\newcommand{\RBP}{\GTT{RBP}\xspace}
\newcommand{\RSI}{\GTT{RSI}\xspace}
\newcommand{\RDI}{\GTT{RDI}\xspace}
\newcommand*{\ST}[1]{\GTT{ST(#1)}\xspace}
\newcommand*{\XMM}[1]{\GTT{XMM#1}\xspace}

% ARM
\newcommand*{\Reg}[1]{\GTT{R#1}\xspace}
\newcommand*{\RegX}[1]{\GTT{X#1}\xspace}
\newcommand*{\RegW}[1]{\GTT{W#1}\xspace}
\newcommand*{\RegD}[1]{\GTT{D#1}\xspace}
\newcommand{\ADREQ}{\GTT{ADREQ}\xspace}
\newcommand{\ADRNE}{\GTT{ADRNE}\xspace}
\newcommand{\BEQ}{\GTT{BEQ}\xspace}

% FIXME tidy this pls
\newcommand{\RegTableThree}[5]{
\begin{center}
\begin{tabular}{ | l | l | l | l | l | l | l | l | l |}
\hline
\RegHeaderTop \\
\hline
\RegHeader \\
\hline
\multicolumn{8}{ | c | }{#1} \\
\hline
\multicolumn{4}{ | c | }{} & \multicolumn{4}{ c | }{#2} \\
\hline
\multicolumn{6}{ | c | }{} & \multicolumn{2}{ c | }{#3} \\
\hline
\multicolumn{6}{ | c | }{} & #4 & #5 \\
\hline
\end{tabular}
\end{center}
}

\newcommand{\RegTableOne}[5]{\RegTableThree{#1\textsuperscript{x64}}{#2}{#3}{#4}{#5}}

\newcommand{\RegTableTwo}[4]{
\begin{center}
\begin{tabular}{ | l | l | l | l | l | l | l | l | l |}
\hline
\RegHeaderTop \\
\hline
\RegHeader \\
\hline
\multicolumn{8}{ | c | }{#1\textsuperscript{x64}} \\
\hline
\multicolumn{4}{ | c | }{} & \multicolumn{4}{ c | }{#2} \\
\hline
\multicolumn{6}{ | c | }{} & \multicolumn{2}{ c | }{#3} \\
\hline
\multicolumn{7}{ | c | }{} & #4\textsuperscript{x64} \\
\hline
\end{tabular}
\end{center}
}

\newcommand{\RegTableFour}[4]{
\begin{center}
\begin{tabular}{ | l | l | l | l | l | l | l | l | l |}
\hline
\RegHeaderTop \\
\hline
\RegHeader \\
\hline
\multicolumn{8}{ | c | }{#1} \\
\hline
\multicolumn{4}{ | c | }{} & \multicolumn{4}{ c | }{#2} \\
\hline
\multicolumn{6}{ | c | }{} & \multicolumn{2}{ c | }{#3} \\
\hline
\multicolumn{7}{ | c | }{} & #4 \\
\hline
\end{tabular}
\end{center}
}

\newcommand{\NonOptimizingKeilVI}{\NonOptimizing Keil 6/2013\xspace}
\newcommand{\OptimizingKeilVI}{\Optimizing Keil 6/2013\xspace}
\newcommand{\NonOptimizingXcodeIV}{\NonOptimizing Xcode 4.6.3 (LLVM)\xspace}
\newcommand{\OptimizingXcodeIV}{\Optimizing Xcode 4.6.3 (LLVM)\xspace}

\newcommand{\OracleTablesName}{oracle tables\xspace}
\newcommand{\oracletables}{\OracleTablesName\footnote{\href{http://go.yurichev.com/17014}{yurichev.com}}\xspace}

\newcommand{\BGREPURL}{\href{http://go.yurichev.com/17017}{GitHub}}
\newcommand{\FNMSDNROTxURL}{\footnote{\href{http://go.yurichev.com/17018}{MSDN}}}

\newcommand{\YurichevIDAIDCScripts}{http://go.yurichev.com/17019}

% sources: books, etc
\newcommand{\TAOCPvolI}{Donald E. Knuth, \emph{The Art of Computer Programming}, Volume 1, 3rd ed., (1997)}
\newcommand{\TAOCPvolII}{Donald E. Knuth, \emph{The Art of Computer Programming}, Volume 2, 3rd ed., (1997)}
\newcommand{\Russinovich}{Mark Russinovich, \emph{Microsoft Windows Internals}}
\newcommand{\Schneier}{Bruce Schneier, \emph{Applied Cryptography}, (John Wiley \& Sons, 1994)}
\newcommand{\AgnerFog}{Agner Fog, \emph{The microarchitecture of Intel, AMD and VIA CPUs}, (2016)}
\newcommand{\AgnerFogCPP}{Agner Fog, \emph{Optimizing software in C++} (2015)}
\newcommand{\AgnerFogCC}{Agner Fog, \emph{Calling conventions} (2015)}
\newcommand{\JavaBook}{[Tim Lindholm, Frank Yellin, Gilad Bracha, Alex Buckley, \emph{The Java(R) Virtual Machine Specification / Java SE 7 Edition}]
\footnote{\AlsoAvailableAs \url{https://docs.oracle.com/javase/specs/jvms/se7/jvms7.pdf}; \url{http://docs.oracle.com/javase/specs/jvms/se7/html/}}}

\newcommand{\ARMPCS}{\InSqBrackets{\emph{Procedure Call Standard for the ARM 64-bit Architecture (AArch64)}, (2013)}\footnote{\AlsoAvailableAs \url{http://go.yurichev.com/17287}}}

\newcommand{\IgorSkochinsky}{[Igor Skochinsky, \emph{Compiler Internals: Exceptions and RTTI}, (2012)] \footnote{\AlsoAvailableAs \url{http://go.yurichev.com/17294}}}

\newcommand{\PietrekSEH}{[Matt Pietrek, \emph{A Crash Course on the Depths of Win32\texttrademark{} Structured Exception Handling}, (1997)]\footnote{\AlsoAvailableAs \url{http://go.yurichev.com/17293}}}

\newcommand{\PietrekPE}{Matt Pietrek, \emph{An In-Depth Look into the Win32 Portable Executable File Format}, (2002)]}

\newcommand{\PietrekPEURL}{\PietrekPE\footnote{\AlsoAvailableAs \url{http://go.yurichev.com/17318}}}

\newcommand{\RitchieDevC}{[Dennis M. Ritchie, \emph{The development of the C language}, (1993)]\footnote{\AlsoAvailableAs \url{http://go.yurichev.com/17264}}}

\newcommand{\RitchieThompsonUNIX}{[D. M. Ritchie and K. Thompson, \emph{The UNIX Time Sharing System}, (1974)]\footnote{\AlsoAvailableAs \url{http://go.yurichev.com/17270}}}

\newcommand{\DrepperTLS}{[Ulrich Drepper, \emph{ELF Handling For Thread-Local Storage}, (2013)]\footnote{\AlsoAvailableAs \url{http://go.yurichev.com/17272}}}

\newcommand{\DrepperMemory}{[Ulrich Drepper, \emph{What Every Programmer Should Know About Memory}, (2007)]\footnote{\AlsoAvailableAs \url{http://go.yurichev.com/17341}}}

\newcommand{\AlephOne}{[Aleph One, \emph{Smashing The Stack For Fun And Profit}, (1996)]\footnote{\AlsoAvailableAs \url{http://go.yurichev.com/17266}}}

\ifdefined\RUSSIAN
\newcommand{\KRBook}{Брайан Керниган, Деннис Ритчи, \emph{Язык программирования Си}, второе издание, (1988, 2009)}
\newcommand{\CppOneOneStd}{Стандарт Си++11}
\newcommand{\CNotes}{Денис Юричев, \emph{Заметки о языке программирования Си/Си++}}
\else
\newcommand{\KRBook}{Brian W. Kernighan, Dennis M. Ritchie, \emph{The C Programming Language}, 2ed, (1988)}
\newcommand{\CppOneOneStd}{C++11 standard}
\newcommand{\CNotes}{Dennis Yurichev, \emph{C/C++ programming language notes}}
\fi

\newcommand{\RobPikePractice}{Brian W. Kernighan, Rob Pike, \emph{Practice of Programming}, (1999)}

\newcommand{\ARMSixFourRef}{\emph{ARM Architecture Reference Manual, ARMv8, for ARMv8-A architecture profile}, (2013)}
\newcommand{\ARMSixFourRefURL}{\InSqBrackets{\ARMSixFourRef}\footnote{\AlsoAvailableAs \url{http://yurichev.com/mirrors/ARMv8-A_Architecture_Reference_Manual_(Issue_A.a).pdf}}}

\newcommand{\SysVABI}{[Michael Matz, Jan Hubicka, Andreas Jaeger, Mark Mitchell, \emph{System V Application Binary Interface. AMD64 Architecture Processor Supplement}, (2013)]
\footnote{\AlsoAvailableAs \url{https://software.intel.com/sites/default/files/article/402129/mpx-linux64-abi.pdf}}}

\newcommand{\IOSABI}{\InSqBrackets{\emph{iOS ABI Function Call Guide}, (2010)}\footnote{\AlsoAvailableAs \url{http://go.yurichev.com/17276}}}

\newcommand{\CNineNineStd}{\emph{ISO/IEC 9899:TC3 (C C99 standard)}, (2007)}

\newcommand{\TCPPPL}{Bjarne Stroustrup, \emph{The C++ Programming Language, 4th Edition}, (2013)}

\newcommand{\AMDOptimization}{\emph{Software Optimization Guide for AMD Family 16h Processors}, (2013)}
% note = "\AlsoAvailableAs \url{http://go.yurichev.com/17285}",

\newcommand{\IntelOptimization}{\emph{Intel® 64 and IA-32 Architectures Optimization Reference Manual}, (2014)}
% note = "\AlsoAvailableAs \url{http://go.yurichev.com/17342}",

\newcommand{\TAOUP}{Eric S. Raymond, \emph{The Art of UNIX Programming}, (2003)}
% note = "\AlsoAvailableAs \url{http://go.yurichev.com/17277}",

\newcommand{\HenryWarren}{Henry S. Warren, \emph{Hacker's Delight}, (2002)}

\newcommand{\ParashiftCPPFAQ}{Marshall Cline, \emph{C++ FAQ}}
% note = "\AlsoAvailableAs \url{http://go.yurichev.com/17291}",

\newcommand{\ARMCookBook}{Advanced RISC Machines Ltd, \emph{The ARM Cookbook}, (1994)}
% note = "\AlsoAvailableAs \url{http://go.yurichev.com/17273}",

\newcommand{\ARMSevenRef}{\emph{ARM(R) Architecture Reference Manual, ARMv7-A and ARMv7-R edition}, (2012)}

\newcommand{\MAbrash}{Michael Abrash, \emph{Graphics Programming Black Book}, 1997}

\newcommand{\MathForProg}{Mathematics for Programmers\footnote{\url{https://yurichev.com/writings/Math-for-programmers.pdf}}}

\newcommand{\radare}{rada.re}

\newcommand{\RedditHiringThread}{\href{https://www.reddit.com/r/ReverseEngineering/comments/4sbd11/rreverseengineerings_2016_triannual_hiring_thread/}{2016}}
\newcommand{\NetsecHiringThread}{\href{https://www.reddit.com/r/netsec/comments/552rz1/rnetsecs_q4_2016_information_security_hiring/}{2016}}

\newcommand{\FNURLREDDIT}{\footnote{\href{http://go.yurichev.com/17027}{reddit.com/r/ReverseEngineering/}}}

\EN{\newcommand{\AcronymsUsed}{Acronyms Used}

\newcommand{\TitleRE}{Reverse Engineering for Beginners}

\newcommand{\TitleUAL}{Understanding Assembly Language}

\newcommand{\AUTHOR}{Dennis Yurichev}

\newcommand{\figname}{fig.\xspace}
\newcommand{\listingname}{listing.\xspace}
% FIXME get rid of:
\newcommand{\bitsENRU}{bits\xspace}
\newcommand{\Sourcecode}{Source code\xspace}
\newcommand{\Seealso}{See also\xspace}
\newcommand{\tableheader}{\headercolor{} offset & \headercolor{} description}
% instructions descriptions
\newcommand{\ASRdesc}{arithmetic shift right}

% x86 registers tables
\newcommand{\RegHeaderTop}{ \multicolumn{8}{ | c | }{ Byte number: } }
% TODO: non-overlapping color!
\newcommand{\RegHeader}{ 7th & 6th & 5th & 4th & 3rd & 2nd & 1st & 0th }
\newcommand{\ReturnAddress}{Return Address}

\newcommand{\localVariable}{local variable}

\newcommand{\savedValueOf}{saved value of}

% for index
\newcommand{\GrepUsage}{grep usage}
\newcommand{\SyntacticSugar}{Syntactic Sugar}
\newcommand{\CompilerAnomaly}{Compiler's anomalies}
\newcommand{\CLanguageElements}{C language elements}
\newcommand{\CStandardLibrary}{C standard library}
\newcommand{\Instructions}{Instructions}
\newcommand{\Pseudoinstructions}{Pseudoinstructions}
\newcommand{\Prefixes}{Prefixes}

\newcommand{\Flags}{Flags}
\newcommand{\Registers}{Registers}
\newcommand{\registers}{registers}
\newcommand{\Stack}{Stack}
\newcommand{\Recursion}{Recursion}
\newcommand{\RAM}{RAM}
\newcommand{\ROM}{ROM}
\newcommand{\Pointers}{Pointers}
\newcommand{\BufferOverflow}{Buffer Overflow}

% DE: also "Zusammenfassung"
\newcommand{\Conclusion}{Conclusion}

\newcommand{\Exercise}{Exercise}
\newcommand{\Exercises}{Exercises}
\newcommand{\Arrays}{Arrays}
\newcommand{\Cpp}{C++\xspace}
\newcommand{\CCpp}{C/C++\xspace}
\newcommand{\NonOptimizing}{Non-optimizing\xspace}
\newcommand{\Optimizing}{Optimizing\xspace}
\newcommand{\ARMMode}{ARM mode\xspace}
\newcommand{\ThumbMode}{Thumb mode\xspace}
\newcommand{\ThumbTwoMode}{Thumb-2 mode\xspace}
\newcommand{\AndENRU}{and\xspace}
\newcommand{\OrENRU}{or\xspace}
\newcommand{\InENRU}{in\xspace}
\newcommand{\ForENRU}{for\xspace}
\newcommand{\LineENRU}{line\xspace}

\newcommand{\DataProcessingInstructionsFootNote}{These instructions are also called \q{data processing instructions}}
% for .bib files
\newcommand{\AlsoAvailableAs}{Also available as\xspace}

% section names
\newcommand{\ShiftsSectionName}{Shifts}
\newcommand{\SignedNumbersSectionName}{Signed number representations}
\newcommand{\HelloWorldSectionName}{Hello, world!}
\newcommand{\SwitchCaseDefaultSectionName}{switch()/case/default}
\newcommand{\PrintfSeveralArgumentsSectionName}{printf() with several arguments}
\newcommand{\BitfieldsChapter}{Manipulating specific bit(s)}
\newcommand{\ArithOptimizations}{Replacing arithmetic instructions to other ones}
\newcommand{\FPUChapterName}{Floating-point unit}
\newcommand{\MoreAboutStrings}{More about strings}
\newcommand{\DivisionByMultSectionName}{Division using multiplication}
\newcommand{\Answer}{Answer}
\newcommand{\WhatThisCodeDoes}{What does this code do}
\newcommand{\WorkingWithFloatAsWithStructSubSubSectionName}{Handling float data type as a structure}

\newcommand{\MinesweeperWinXPExampleChapterName}{Minesweeper (Windows XP)}
\newcommand{\StructurePackingSectionName}{Fields packing in structure}
\newcommand{\StructuresChapterName}{Structures}
\newcommand{\PICcode}{position-independent code}
\newcommand{\CapitalPICcode}{Position-independent code}
\newcommand{\Loops}{Loops}

% C
\newcommand{\PostIncrement}{Post-increment}
\newcommand{\PostDecrement}{Post-decrement}
\newcommand{\PreIncrement}{Pre-increment}
\newcommand{\PreDecrement}{Pre-decrement}

% MIPS
\newcommand{\GlobalPointer}{Global Pointer}

\newcommand{\garbage}{garbage}
\newcommand{\IntelSyntax}{Intel syntax}
\newcommand{\ATTSyntax}{AT\&T syntax}
\newcommand{\randomNoise}{random noise}
\newcommand{\Example}{Example}
\newcommand{\argument}{argument}
\newcommand{\MarkedInIDAAs}{marked in \IDA as}
\newcommand{\stepover}{step over}
\newcommand{\ShortHotKeyCheatsheet}{Hot-keys cheatsheet}

\newcommand{\assemblyOutput}{assembly output}

% ML prefix is for multi-lingual words and sentences:
\newcommand{\MLHeap}{Heap}
\newcommand{\MLStack}{Stack}
\newcommand{\MLStackOverflow}{Stack overflow}
\newcommand{\MLStartOfHeap}{Start of heap}
\newcommand{\MLStartOfStack}{Start of stack}
\newcommand{\MLinputA}{input A}
\newcommand{\MLinputB}{input B}
\newcommand{\MLoutput}{output}
\newcommand{\SoftwareCracking}{Software cracking}


}
\RU{\newcommand{\AcronymsUsed}{Список принятых сокращений}

\newcommand{\TitleRE}{Reverse Engineering для начинающих}

\newcommand{\TitleUAL}{Понимание языка ассемблера}

\newcommand{\AUTHOR}{Денис Юричев}

\newcommand{\figname}{илл.\xspace}
\newcommand{\listingname}{листинг.\xspace}
% FIXME get rid of:
\newcommand{\bitsENRU}{бита\xspace}
\newcommand{\Sourcecode}{Исходный код}
\newcommand{\Seealso}{См. также\xspace}
\newcommand{\tableheader}{\headercolor{} смещение & \headercolor{} описание }
% instructions descriptions
\newcommand{\ASRdesc}{арифметический сдвиг вправо}

% x86 registers tables
\newcommand{\RegHeaderTop}{ \multicolumn{8}{ | c | }{ Номер байта: } }
% TODO: non-overlapping color!
\newcommand{\RegHeader}{ 7-й & 6-й & 5-й & 4-й & 3-й & 2-й & 1-й & 0-й }
\newcommand{\ReturnAddress}{Адрес возврата}

\newcommand{\localVariable}{локальная переменная}

\newcommand{\savedValueOf}{сохраненное значение}

% for index
\newcommand{\GrepUsage}{Использование grep}
\newcommand{\SyntacticSugar}{Синтаксический сахар}
\newcommand{\CompilerAnomaly}{Аномалии компиляторов}
\newcommand{\CLanguageElements}{Элементы языка Си}
\newcommand{\CStandardLibrary}{Стандартная библиотека Си}
\newcommand{\Instructions}{Инструкции}
\newcommand{\Pseudoinstructions}{Псевдоинструкции}
\newcommand{\Prefixes}{Префиксы}

\newcommand{\Flags}{Флаги}
\newcommand{\Registers}{Регистры}
\newcommand{\registers}{регистры}
\newcommand{\Stack}{Стек}
\newcommand{\Recursion}{Рекурсия}
\newcommand{\RAM}{ОЗУ}
\newcommand{\ROM}{ПЗУ}
\newcommand{\Pointers}{Указатели}
\newcommand{\BufferOverflow}{Переполнение буфера}

\newcommand{\Conclusion}{Вывод}

\newcommand{\Exercise}{Упражнение}
\newcommand{\Exercises}{Упражнения}
\newcommand{\Arrays}{Массивы}
\newcommand{\Cpp}{Си++}
\newcommand{\CCpp}{Си/Си++}
\newcommand{\NonOptimizing}{Неоптимизирующий}
\newcommand{\Optimizing}{Оптимизирующий}
\newcommand{\ARMMode}{Режим ARM}
\newcommand{\ThumbMode}{Режим Thumb\xspace}
\newcommand{\ThumbTwoMode}{Режим Thumb-2\xspace}
\newcommand{\AndENRU}{и\xspace}
\newcommand{\OrENRU}{или\xspace}
\newcommand{\InENRU}{в\xspace}
\newcommand{\ForENRU}{для\xspace}
\newcommand{\LineENRU}{строка\xspace}

\newcommand{\DataProcessingInstructionsFootNote}{Эти инструкции также называются \q{data processing instructions}}

% for .bib files
\newcommand{\AlsoAvailableAs}{Также доступно здесь:\xspace}

% section names
\newcommand{\ShiftsSectionName}{Сдвиги}
\newcommand{\SignedNumbersSectionName}{Представление знака в числах}
\newcommand{\HelloWorldSectionName}{Hello, world!}
\newcommand{\SwitchCaseDefaultSectionName}{switch()/case/default}
\newcommand{\PrintfSeveralArgumentsSectionName}{printf() с несколькими аргументами}
\newcommand{\BitfieldsChapter}{Работа с отдельными битами}
\newcommand{\ArithOptimizations}{Замена одних арифметических инструкций на другие}

\newcommand{\FPUChapterName}{Работа с FPU}
\newcommand{\MoreAboutStrings}{Еще кое-что о строках}
\newcommand{\DivisionByMultSectionName}{Деление используя умножение}
\newcommand{\Answer}{Ответ}
\newcommand{\WhatThisCodeDoes}{Что делает этот код}
\newcommand{\WorkingWithFloatAsWithStructSubSubSectionName}{Работа с типом float как со структурой}

\newcommand{\MinesweeperWinXPExampleChapterName}{Сапёр (Windows XP)}
\newcommand{\StructurePackingSectionName}{Упаковка полей в структуре}
\newcommand{\StructuresChapterName}{Структуры}
\newcommand{\PICcode}{адресно-независимый код}
\newcommand{\CapitalPICcode}{Адресно-независимый код}
\newcommand{\Loops}{Циклы}

% C
\newcommand{\PostIncrement}{Пост-инкремент}
\newcommand{\PostDecrement}{Пост-декремент}
\newcommand{\PreIncrement}{Пре-инкремент}
\newcommand{\PreDecrement}{Пре-декремент}

% MIPS
\newcommand{\GlobalPointer}{Глобальный указатель}

\newcommand{\garbage}{мусор}
\newcommand{\IntelSyntax}{Синтаксис Intel}
\newcommand{\ATTSyntax}{Синтаксис AT\&T}
\newcommand{\randomNoise}{случайный шум}
\newcommand{\Example}{Пример}
\newcommand{\argument}{аргумент}
\newcommand{\MarkedInIDAAs}{маркируется в \IDA как}
\newcommand{\stepover}{сделать шаг, не входя в функцию}
\newcommand{\ShortHotKeyCheatsheet}{Краткий справочник горячих клавиш}

\newcommand{\assemblyOutput}{вывод на ассемблере}

% ML prefix is for multi-lingual words and sentences:
\newcommand{\MLHeap}{Куча}
\newcommand{\MLStack}{Стэк}
\newcommand{\MLStackOverflow}{Переполнение стека}
\newcommand{\MLStartOfHeap}{Начало кучи}
\newcommand{\MLStartOfStack}{Вершина стека}
\newcommand{\MLinputA}{вход А}
\newcommand{\MLinputB}{вход Б}
\newcommand{\MLoutput}{выход}
\newcommand{\SoftwareCracking}{Взлом ПО}

}
\FR{\newcommand{\AcronymsUsed}{Acronymes utilisés}

\newcommand{\TitleRE}{Rétro-ingénierie pour Débutants}

\newcommand{\TitleUAL}{Comprendre le langage d'assemblage}

\newcommand{\AUTHOR}{Dennis Yurichev}

\newcommand{\figname}{fig.\xspace}
\newcommand{\listingname}{listado.\xspace}
% FIXME get rid of:
\newcommand{\bitsENRU}{bits\xspace}
\newcommand{\Sourcecode}{Code source\xspace}
\newcommand{\Seealso}{Voir également\xspace}
\newcommand{\tableheader}{\headercolor{} offset & \headercolor{} description }
% instructions descriptions
\newcommand{\ASRdesc}{décalage arithmétique vers la droite}

% x86 registers tables
\newcommand{\RegHeaderTop}{ \multicolumn{8}{ | c | }{ Octet d'indice } }
% TODO: non-overlapping color!
\newcommand{\RegHeader}{7 & 6 & 5 & 4 & 3 & 2 & 1 & 0}

\newcommand{\ReturnAddress}{Adresse de retour}

\newcommand{\localVariable}{variable locale}

\newcommand{\savedValueOf}{valeur enregistrée de}

% for index
\newcommand{\GrepUsage}{Utilisation de grep}
\newcommand{\SyntacticSugar}{Sucre syntaxique}
\newcommand{\CompilerAnomaly}{Anomalies du compilateur}
\newcommand{\CLanguageElements}{Éléments du langage C}
\newcommand{\CStandardLibrary}{Bibliothèque standard C}
\newcommand{\Instructions}{Instructions}
\newcommand{\Pseudoinstructions}{Pseudo-instructions}
\newcommand{\Prefixes}{Préfixes}

\newcommand{\Flags}{Flags}
\newcommand{\Registers}{Registres}
\newcommand{\registers}{registres}
\newcommand{\Stack}{Pile}
\newcommand{\Recursion}{Récursivité}
\newcommand{\RAM}{RAM}
\newcommand{\ROM}{ROM}
\newcommand{\Pointers}{Pointeurs}
\newcommand{\BufferOverflow}{Débordement de tampon}

\newcommand{\Conclusion}{Conclusion}

\newcommand{\Exercise}{Exercice}
\newcommand{\Exercises}{Exercices\xspace}
\newcommand{\Arrays}{Tableaux}
\newcommand{\Cpp}{C++\xspace}
\newcommand{\CCpp}{C/C++\xspace}
\newcommand{\NonOptimizing}{sans optimisation\xspace}
\newcommand{\Optimizing}{avec optimisation\xspace}
\newcommand{\ARMMode}{Mode ARM\xspace}
\newcommand{\ThumbMode}{Mode Thumb\xspace}
\newcommand{\ThumbTwoMode}{Mode Thumb-2\xspace}
\newcommand{\AndENRU}{et\xspace}
\newcommand{\OrENRU}{ou\xspace}
\newcommand{\InENRU}{dans\xspace}
\newcommand{\ForENRU}{pour\xspace}
\newcommand{\LineENRU}{ligne\xspace}

\newcommand{\DataProcessingInstructionsFootNote}{Ces instructions sont également appelées \q{instructions de traitement de données}}

% for .bib files
\newcommand{\AlsoAvailableAs}{Aussi disponible en\xspace}

% section names
\newcommand{\ShiftsSectionName}{Décalages}
\newcommand{\SignedNumbersSectionName}{Représentations des nombres signés}
\newcommand{\HelloWorldSectionName}{Hello, world!}
\newcommand{\SwitchCaseDefaultSectionName}{switch()/case/default}
\newcommand{\PrintfSeveralArgumentsSectionName}{printf() avec plusieurs arguments}
\newcommand{\BitfieldsChapter}{Manipulation de bits spécifiques}
\newcommand{\ArithOptimizations}{Remplacement de certaines instructions arithmétiques par d'autres}

\newcommand{\FPUChapterName}{Unité à virgule flottante}
\newcommand{\MoreAboutStrings}{Plus d'information sur les chaînes}
\newcommand{\DivisionByMultSectionName}{Division par la multiplication}
\newcommand{\Answer}{Réponse}
\newcommand{\WhatThisCodeDoes}{Que fait ce code ?}
\newcommand{\WorkingWithFloatAsWithStructSubSubSectionName}{Travailler avec le type float comme une structure}

\newcommand{\MinesweeperWinXPExampleChapterName}{Démineur (Windows XP)}
\newcommand{\StructurePackingSectionName}{Organisation des champs dans la structure}
\newcommand{\StructuresChapterName}{Structures}
\newcommand{\PICcode}{code indépendant de la position}
\newcommand{\CapitalPICcode}{Code indépendant de la position}
\newcommand{\Loops}{Boucles}

% C
\newcommand{\PostIncrement}{Post-incrémentation}
\newcommand{\PostDecrement}{Post-décrémentation}
\newcommand{\PreIncrement}{Pré-incrémentation}
\newcommand{\PreDecrement}{Pré-décrémentation}

% MIPS
\newcommand{\GlobalPointer}{Pointeur Global}

\newcommand{\garbage}{déchets}
\newcommand{\IntelSyntax}{Syntaxe Intel}
\newcommand{\ATTSyntax}{Syntaxe AT\&T}
\newcommand{\randomNoise}{bruit aléatoire}
\newcommand{\Example}{Exemple}
\newcommand{\argument}{argument}
\newcommand{\MarkedInIDAAs}{marqué dans \IDA comme}
\newcommand{\stepover}{enjamber}
\newcommand{\ShortHotKeyCheatsheet}{Anti-sèche des touches de raccourci}

\newcommand{\assemblyOutput}{résultat en sortie de l'assembleur}
% was in common_URLS.tex:
\newcommand{\URLWPDA}{\href{https://fr.wikipedia.org/wiki/Alignement_en_m\%C3\%A9moire}{Wikipedia: Alignement en mémoire}}

% ML prefix is for multi-lingual words and sentences:
\newcommand{\MLHeap}{Heap}
\newcommand{\MLStack}{Pile}
\newcommand{\MLStackOverflow}{Débordement de pile}
\newcommand{\MLStartOfHeap}{Début du heap}
\newcommand{\MLStartOfStack}{Début de la pile}
\newcommand{\MLinputA}{entrée A}
\newcommand{\MLinputB}{entrée B}
\newcommand{\MLoutput}{sortie}
\newcommand{\SoftwareCracking}{cracking de logiciel}

}
\DE{\newcommand{\AcronymsUsed}{Verwendete Abkürzungen}

\newcommand{\TitleRE}{Reverse Engineering für Einsteiger}

% TBT?
\newcommand{\TitleUAL}{Understanding Assembly Language}

\newcommand{\AUTHOR}{Dennis Yurichev}

\newcommand{\figname}{Abb.\xspace}

\newcommand{\listingname}{\DE{Listing.}\xspace}

% FIXME get rid of:
\newcommand{\bitsENRU}{bits\xspace}
\newcommand{\Sourcecode}{Quellcode\xspace}
\newcommand{\Seealso}{siehe auch\xspace}
\newcommand{\tableheader}{\headercolor{}Offset & \headercolor{} Beschreibung}
% instructions descriptions
\newcommand{\ASRdesc}{\DEph{}}

% x86 registers tables
\newcommand{\RegHeaderTop}{ \multicolumn{8}{ | c | }{ Byte-Nummer: }}

% TODO: non-overlapping color!
\newcommand{\RegHeader}{\DEph{}}

\newcommand{\ReturnAddress}{Rücksprungadresse}

\newcommand{\localVariable}{lokale Variable}

\newcommand{\savedValueOf}{\DEph{}}

% for index
\newcommand{\GrepUsage}{\DEph}
\newcommand{\SyntacticSugar}{\DEph{}}
\newcommand{\CompilerAnomaly}{\DEph{}}
\newcommand{\CLanguageElements}{C Sprachelemente}
\newcommand{\CStandardLibrary}{\DEph{}}
\newcommand{\Instructions}{\DEph{}}
\newcommand{\Pseudoinstructions}{\DEph{}}
\newcommand{\Prefixes}{\DEph{}}

\newcommand{\Flags}{\DEph{}}
\newcommand{\Registers}{\DEph{}}
\newcommand{\registers}{\DEph{}}
\newcommand{\Stack}{Stack}
\newcommand{\Recursion}{\DEph{}}
\newcommand{\RAM}{RAM}
\newcommand{\ROM}{ROM}
\newcommand{\Pointers}{\DEph{}}
\newcommand{\BufferOverflow}{\DEph{}}

% DE: also "Zusammenfassung"
\newcommand{\Conclusion}{Fazit}

\newcommand{\Exercise}{\DEph{}\xspace}
\newcommand{\Exercises}{Übungen\xspace}
\newcommand{\Arrays}{Arrays}
\newcommand{\Cpp}{C++\xspace}
\newcommand{\CCpp}{C/C++\xspace}
\newcommand{\NonOptimizing}{\DEph{}\xspace}
\newcommand{\Optimizing}{\DEph{}\xspace}
\newcommand{\ARMMode}{\DEph{}\xspace}
\newcommand{\ThumbMode}{\DEph{}\xspace}
\newcommand{\ThumbTwoMode}{\DEph{}\xspace}
\newcommand{\AndENRU}{\DEph{}\xspace}
\newcommand{\OrENRU}{\DEph{}\xspace}
\newcommand{\InENRU}{\DEph{}\xspace}
\newcommand{\ForENRU}{\DEph{}\xspace}
\newcommand{\LineENRU}{\DEph{}\xspace}

\newcommand{\DataProcessingInstructionsFootNote}{\DEph{}}
% for .bib files
\newcommand{\AlsoAvailableAs}{\DEph{}\xspace}

% section names
\newcommand{\ShiftsSectionName}{\DEph{}}
\newcommand{\SignedNumbersSectionName}{\DEph{}}
\newcommand{\HelloWorldSectionName}{Hallo, Welt!}
\newcommand{\SwitchCaseDefaultSectionName}{switch()/case/default}
\newcommand{\PrintfSeveralArgumentsSectionName}{\DEph{}}
\newcommand{\BitfieldsChapter}{Manipulieren einzelner Bits}
\newcommand{\ArithOptimizations}{Ersetzen von arithmetischen Operationen}
\newcommand{\FPUChapterName}{Gleitkommaeinheit}
\newcommand{\MoreAboutStrings}{Mehr über Zeichenketten}
\newcommand{\DivisionByMultSectionName}{\DEph{}}
\newcommand{\Answer}{\DEph{}}
\newcommand{\WhatThisCodeDoes}{\DEph{}}
\newcommand{\WorkingWithFloatAsWithStructSubSubSectionName}{\DEph{}}

\newcommand{\MinesweeperWinXPExampleChapterName}{\DEph{} (Windows XP)}
\newcommand{\StructurePackingSectionName}{\DEph{}}
\newcommand{\StructuresChapterName}{\DEph{}}
\newcommand{\PICcode}{positionsabhängiger Code}
\newcommand{\CapitalPICcode}{Positionsabhängiger Code}
\newcommand{\Loops}{Schleifen}

% C
\newcommand{\PostIncrement}{\DEph{}}
\newcommand{\PostDecrement}{\DEph{}}
\newcommand{\PreIncrement}{\DEph{}}
\newcommand{\PreDecrement}{\DEph{}}

% MIPS
\newcommand{\GlobalPointer}{\DEph{}}

\newcommand{\garbage}{\DEph{}}
\newcommand{\IntelSyntax}{\DEph{}}
\newcommand{\ATTSyntax}{\DEph{}}
\newcommand{\randomNoise}{\DEph{}}
\newcommand{\Example}{\DEph{}}
\newcommand{\argument}{\DEph{}}
\newcommand{\MarkedInIDAAs}{\DEph{}}
\newcommand{\stepover}{\DEph{}}
\newcommand{\ShortHotKeyCheatsheet}{\DEph{}}

\newcommand{\assemblyOutput}{\DEph{}}

% ML prefix is for multi-lingual words and sentences:
\newcommand{\MLHeap}{\DEph{}}
\newcommand{\MLStack}{\DEph{}}
\newcommand{\MLStackOverflow}{\DEph{}}
\newcommand{\MLStartOfHeap}{\DEph{}}
\newcommand{\MLStartOfStack}{\DEph{}}
\newcommand{\MLinputA}{\DEph{}}
\newcommand{\MLinputB}{\DEph{}}
\newcommand{\MLoutput}{\DEph{}}
\newcommand{\SoftwareCracking}{\DEph{}}

}
\JA{\newcommand{\AcronymsUsed}{頭字語}

\newcommand{\TitleRE}{リバースエンジニアリング入門}

%TBT
\newcommand{\TitleUAL}{Understanding Assembly Language}

\newcommand{\AUTHOR}{Dennis Yurichev}

\newcommand{\figname}{fig.\xspace}
\newcommand{\listingname}{リスト}

% FIXME get rid of:
\newcommand{\bitsENRU}{bits\xspace}
\newcommand{\Sourcecode}{ソースコード\xspace}
\newcommand{\Seealso}{参照\xspace}
\newcommand{\tableheader}{\headercolor{} オフセット & \headercolor{} 記述 }
% instructions descriptions
\newcommand{\ASRdesc}{\JAph{}}

% x86 registers tables
\newcommand{\RegHeaderTop}{ \multicolumn{8}{ | c | }{ バイトの並び順 } }
% TODO: non-overlapping color!
\newcommand{\RegHeader}{ 第7 & 第6 & 第5 & 第4 & 第3 & 第2 & 第1 & 第0 }

\newcommand{\ReturnAddress}{リターンアドレス}

\newcommand{\localVariable}{ローカル変数}

\newcommand{\savedValueOf}{saved value of}

% for index
\newcommand{\GrepUsage}{\JAph{}}
\newcommand{\SyntacticSugar}{糖衣構文}
\newcommand{\CompilerAnomaly}{コンパイラアノマリ}
\newcommand{\CLanguageElements}{C言語の要素}
\newcommand{\CStandardLibrary}{C標準ライブラリ}
\newcommand{\Instructions}{命令}
\newcommand{\Pseudoinstructions}{疑似命令}
\newcommand{\Prefixes}{プリフィックス}

\newcommand{\Flags}{フラグ}
\newcommand{\Registers}{レジスタ}
\newcommand{\registers}{レジスタ}
\newcommand{\Stack}{スタック}
\newcommand{\Recursion}{再帰}
\newcommand{\RAM}{RAM}
\newcommand{\ROM}{ROM}
\newcommand{\Pointers}{ポインタ}
\newcommand{\BufferOverflow}{バッファオーバーフロー}

% DE: also "Zusammenfassung"
\newcommand{\Conclusion}{結論}

\newcommand{\Exercise}{練習問題\xspace}
\newcommand{\Exercises}{練習問題\xspace}
\newcommand{\Arrays}{配列}
\newcommand{\Cpp}{C++\xspace}
\newcommand{\CCpp}{C/C++\xspace}
\newcommand{\NonOptimizing}{非最適化\xspace}
\newcommand{\Optimizing}{最適化\xspace}
\newcommand{\ARMMode}{ARMモード\xspace}
\newcommand{\ThumbMode}{Thumbモード\xspace}
\newcommand{\ThumbTwoMode}{Thumb-2モード\xspace}
\newcommand{\AndENRU}{\JAph{}}
\newcommand{\OrENRU}{\JAph{}}
\newcommand{\InENRU}{\JAph{}}
\newcommand{\ForENRU}{\JAph{}}
\newcommand{\LineENRU}{\JAph{}}

\newcommand{\DataProcessingInstructionsFootNote}{この命令は \q{データプロセス命令} とも呼ばれます}

% for .bib files
\newcommand{\AlsoAvailableAs}{以下で利用可能\xspace}

% section names
\newcommand{\ShiftsSectionName}{シフト}
\newcommand{\SignedNumbersSectionName}{符号付き整数表現}
\newcommand{\HelloWorldSectionName}{ハローワールド!}
\newcommand{\SwitchCaseDefaultSectionName}{switch()/case/default}
\newcommand{\PrintfSeveralArgumentsSectionName}{printf() 引数を取って}
\newcommand{\BitfieldsChapter}{特定のビットを操作する}
\newcommand{\ArithOptimizations}{算術命令を他の命令に置換する}

\newcommand{\FPUChapterName}{フローティングポイントユニット}
\newcommand{\MoreAboutStrings}{文字列に関する加筆}
\newcommand{\DivisionByMultSectionName}{乗法を使用した除算}
\newcommand{\Answer}{解答}
\newcommand{\WhatThisCodeDoes}{このコードは何をしている}
\newcommand{\WorkingWithFloatAsWithStructSubSubSectionName}{フロート型のデータを構造体として扱う}

\newcommand{\MinesweeperWinXPExampleChapterName}{マインスイーパー (Windows XP)}
\newcommand{\StructurePackingSectionName}{フィールドを構造体にパッキングする}
\newcommand{\StructuresChapterName}{構造体}
\newcommand{\PICcode}{位置独立コード}
\newcommand{\CapitalPICcode}{位置独立コード}
\newcommand{\Loops}{ループ}

% C
\newcommand{\PostIncrement}{後置インクリメント}
\newcommand{\PostDecrement}{後置デクリメント}
\newcommand{\PreIncrement}{前置インクリメント}
\newcommand{\PreDecrement}{前置デクリメント}

% MIPS
\newcommand{\GlobalPointer}{グローバルポインタ}

\newcommand{\garbage}{ガーベッジ}
\newcommand{\IntelSyntax}{インテル構文}
\newcommand{\ATTSyntax}{AT\&T構文}
\newcommand{\randomNoise}{ランダムノイズ}
\newcommand{\Example}{例}
\newcommand{\argument}{引数}
\newcommand{\MarkedInIDAAs}{\IDA にマークする}
\newcommand{\stepover}{ステップオーバー}
\newcommand{\ShortHotKeyCheatsheet}{ホットキーチートシート}

\newcommand{\assemblyOutput}{アセンブリ出力}

% ML prefix is for multi-lingual words and sentences:
\newcommand{\MLHeap}{ヒープ}
\newcommand{\MLStack}{スタック}
\newcommand{\MLStackOverflow}{スタックオーバーフロー}
\newcommand{\MLStartOfHeap}{ヒープの開始}
\newcommand{\MLStartOfStack}{スタックの開始}
\newcommand{\MLinputA}{入力A}
\newcommand{\MLinputB}{入力B}
\newcommand{\MLoutput}{出力}
\newcommand{\SoftwareCracking}{\JAph{}}

}
\IT{\newcommand{\AcronymsUsed}{Acronimi utilizzati}

\newcommand{\TitleRE}{Reverse Engineering per Principianti}

\newcommand{\TitleUAL}{Capire il linguaggio Assembly}

\newcommand{\AUTHOR}{Dennis Yurichev}

\newcommand{\figname}{fig.\xspace}
\newcommand{\listingname}{listato.\xspace}
% FIXME get rid of:
\newcommand{\bitsENRU}{bit\xspace}
\newcommand{\Sourcecode}{Codice sorgente\xspace}
\newcommand{\Seealso}{Vedi anche\xspace}
\newcommand{\tableheader}{\headercolor{} offset & \headercolor{} descrizione}

% instructions descriptions
\newcommand{\ASRdesc}{scorrimento aritmetico a destra}

% x86 registers tables
\newcommand{\RegHeaderTop}{ \multicolumn{8}{ | c | }{ Numero byte } }
% TODO: non-overlapping color!
\newcommand{\RegHeader}{ 7°  & 6°  & 5°  & 4°  & 3°  & 2°  & 1°  & 0 }
\newcommand{\ReturnAddress}{Indirizzo di Ritorno}

\newcommand{\localVariable}{variabile locale}

\newcommand{\savedValueOf}{valore memorizzato di}
% for index
\newcommand{\GrepUsage}{uso di grep}
\newcommand{\SyntacticSugar}{Syntactic Sugar}
\newcommand{\CompilerAnomaly}{Anomalie del compilatore}
\newcommand{\CLanguageElements}{Elementi del linguaggio C}
\newcommand{\CStandardLibrary}{Libreria C standard}
\newcommand{\Instructions}{Istruzioni}
\newcommand{\Pseudoinstructions}{Pseudo-istruzioni}
\newcommand{\Prefixes}{Prefissi}

\newcommand{\Flags}{Flag}
\newcommand{\Registers}{Registri}
\newcommand{\registers}{registri}
\newcommand{\Stack}{Stack}
\newcommand{\Recursion}{Recursione}
\newcommand{\RAM}{RAM}
\newcommand{\ROM}{ROM}
\newcommand{\Pointers}{Puntatori}
\newcommand{\BufferOverflow}{Buffer Overflow}

\newcommand{\Conclusion}{Conclusione}

\newcommand{\Exercise}{Esercizio}
\newcommand{\Exercises}{Esercizi\xspace}
\newcommand{\Arrays}{Array}
\newcommand{\Cpp}{C++\xspace}
\newcommand{\CCpp}{C/C++\xspace}
\newcommand{\NonOptimizing}{Senza ottimizzazione\xspace}
\newcommand{\Optimizing}{Con ottimizzazione\xspace}
\newcommand{\ARMMode}{Modalità ARM\xspace}
\newcommand{\ThumbMode}{Modalità Thumb\xspace}
\newcommand{\ThumbTwoMode}{Modalità Thumb-2\xspace}
\newcommand{\AndENRU}{\ITph{}}
\newcommand{\OrENRU}{\ITph{}}
\newcommand{\InENRU}{\ITph{}}
\newcommand{\ForENRU}{\ITph{}}
\newcommand{\LineENRU}{\ITph{}}

\newcommand{\DataProcessingInstructionsFootNote}{\ITph{}}

% for .bib files
\newcommand{\AlsoAvailableAs}{\ITph{}}

% section names
\newcommand{\ShiftsSectionName}{\ITph{}}
\newcommand{\SignedNumbersSectionName}{Rappresentazione di numeri con segno}
\newcommand{\HelloWorldSectionName}{Hello, world!}
\newcommand{\SwitchCaseDefaultSectionName}{switch()/case/default}
\newcommand{\PrintfSeveralArgumentsSectionName}{printf() con più argomenti}
\newcommand{\BitfieldsChapter}{Manipolando dei bit speifici}
\newcommand{\ArithOptimizations}{Sostituzione di istruzioni aritmetiche con altre}
\newcommand{\FPUChapterName}{Floating-point unit}
\newcommand{\MoreAboutStrings}{Maggiori informazioni sulle stringhe}
\newcommand{\DivisionByMultSectionName}{Divisione utilizzando la moltiplicazione}
\newcommand{\Answer}{Risposta}
\newcommand{\WhatThisCodeDoes}{Cosa fa qusto codice}
\newcommand{\WorkingWithFloatAsWithStructSubSubSectionName}{Utilizzare i tipi di dato float come una struttura}

\newcommand{\MinesweeperWinXPExampleChapterName}{\ITph{} (Windows XP)}
\newcommand{\StructurePackingSectionName}{Organizzazione dei campi in una struttura}
\newcommand{\StructuresChapterName}{Strutture}
\newcommand{\PICcode}{codice indipendente dalla posizione}
\newcommand{\CapitalPICcode}{Codice indipendente dalla posizione}
\newcommand{\Loops}{Cicli}

% C
\newcommand{\PostIncrement}{Post-incremento}
\newcommand{\PostDecrement}{Post-decremento}
\newcommand{\PreIncrement}{Pre-incremento}
\newcommand{\PreDecrement}{Pre-decremento}

% MIPS
\newcommand{\GlobalPointer}{Puntatore Globale}

\newcommand{\garbage}{garbage}
\newcommand{\IntelSyntax}{Sintassi Intel}
\newcommand{\ATTSyntax}{Sintassi AT\&T}
\newcommand{\randomNoise}{rumore casuale}
\newcommand{\Example}{Esempio}
\newcommand{\argument}{argomento}
\newcommand{\MarkedInIDAAs}{marcato in IDA come}
\newcommand{\stepover}{step over}
\newcommand{\ShortHotKeyCheatsheet}{Elenco delle scorciatoie da tastiera}

\newcommand{\assemblyOutput}{risultato dell'assembly}

% ML prefix is for multi-lingual words and sentences:
\newcommand{\MLHeap}{Heap}
\newcommand{\MLStack}{Stack}
\newcommand{\MLStackOverflow}{Stack overflow}
\newcommand{\MLStartOfHeap}{Inizio dell'heap}
\newcommand{\MLStartOfStack}{Inizio dello stack}
\newcommand{\MLinputA}{input A}
\newcommand{\MLinputB}{input A}
\newcommand{\MLoutput}{output}
\newcommand{\SoftwareCracking}{\ITph{}}

}
\PL{\input{macro_lang_PL}}
\ES{\newcommand{\AcronymsUsed}{Acr\'onimos utilizados}

\newcommand{\TitleRE}{Ingenier\'ia Inversa para Principiantes}

% TBT?
\newcommand{\TitleUAL}{Understanding Assembly Language}

\newcommand{\AUTHOR}{Dennis Yurichev}

\newcommand{\figname}{fig.\xspace}
\newcommand{\listingname}{listado.\xspace}
% FIXME get rid of:
\newcommand{\bitsENRU}{bits}\xspace}
\newcommand{\Sourcecode}{C\'odigo fuente\xspace}
\newcommand{\Seealso}{V\'ease tambi\'en\xspace}
\newcommand{\tableheader}{\headercolor{} offset & \headercolor{} descripci\'on }
% instructions descriptions
\newcommand{\ASRdesc}{desplazamiento aritm\'etico a la derecha}

% x86 registers tables
\newcommand{\RegHeaderTop}{ \multicolumn{8}{ | c | }{ \ESph{} } }
% TODO: non-overlapping color!
\newcommand{\RegHeader}{ 7mo & 6to & 5to & 4to & 3ro & 2do & 1ro & 0 }
\newcommand{\ReturnAddress}{ Direcci\'on de Retorno }

\newcommand{\localVariable}{\ESph{}}

\newcommand{\savedValueOf}{\ESph{}}

% for index
\newcommand{\GrepUsage}{Uso de grep}
\newcommand{\SyntacticSugar}{Azúcar sintáctica}
\newcommand{\CompilerAnomaly}{Anomalías del compilador}
\newcommand{\CLanguageElements}{Elementos del lenguaje C}
\newcommand{\CStandardLibrary}{Librería estándar C}
\newcommand{\Instructions}{Instrucciones}
\newcommand{\Pseudoinstructions}{Pseudo-instrucciones}
\newcommand{\Prefixes}{Prefijos}

\newcommand{\Flags}{Flags}
\newcommand{\Registers}{Registros}
\newcommand{\registers}{registros}
\newcommand{\Stack}{Pila}
\newcommand{\Recursion}{Recursión}
\newcommand{\RAM}{RAM}
\newcommand{\ROM}{ROM}
\newcommand{\Pointers}{Apuntadores}
\newcommand{\BufferOverflow}{Desbordamiento de buffer}

% DE: also "Zusammenfassung"
\newcommand{\Conclusion}{Conclusión}

\newcommand{\Exercise}{Ejercicio\xspace}
\newcommand{\Exercises}{Ejercicios\xspace}
\newcommand{\Arrays}{Matriz}
\newcommand{\Cpp}{C++}
\newcommand{\CCpp}{C/C++}
\newcommand{\NonOptimizing}{Sin optimización}
\newcommand{\Optimizing}{Con optimización}
\newcommand{\ARMMode}{Modo ARM\xspace}
\newcommand{\ThumbMode}{Modo Thumb\xspace}
\newcommand{\ThumbTwoMode}{Modo Thumb-2\xspace}
\newcommand{\AndENRU}{y\xspace}
\newcommand{\OrENRU}{o\xspace}
\newcommand{\InENRU}{en\xspace}
\newcommand{\ForENRU}{para\xspace}
\newcommand{\LineENRU}{línea\xspace}

\newcommand{\DataProcessingInstructionsFootNote}{Estas instrucciones también son llamadas \q{instrucciones de procesamiento de datos}}

% for .bib files
\newcommand{\AlsoAvailableAs}{También disponible como\xspace}

% section names
\newcommand{\ShiftsSectionName}{Desplazamientos}
\newcommand{\SignedNumbersSectionName}{Representaci\'on de n\'umeros con signo}
\newcommand{\HelloWorldSectionName}{!`Hola, mundo!}
\newcommand{\SwitchCaseDefaultSectionName}{switch()/case/default}
\newcommand{\PrintfSeveralArgumentsSectionName}{printf() con varios argumentos}
\newcommand{\BitfieldsChapter}{Manipulando bit(s) específicos}
\newcommand{\ArithOptimizations}{Substituición de instrucciones aritméticas por otras}
	
\newcommand{\FPUChapterName}{Unidad de punto flotante}
\newcommand{\MoreAboutStrings}{\ESph{}}
\newcommand{\DivisionByMultSectionName}{División entre 9}
\newcommand{\Answer}{Respuesta}
\newcommand{\WhatThisCodeDoes}{?`Qu\'e hace este código?}
\newcommand{\WorkingWithFloatAsWithStructSubSubSectionName}{Trabajando con el tipo float como una estructura}

\newcommand{\MinesweeperWinXPExampleChapterName}{Buscaminas (Windows XP)}
\newcommand{\StructurePackingSectionName}{Organización de campos en la estructura}
\newcommand{\StructuresChapterName}{Estructuras}
\newcommand{\PICcode}{código independiente de la posición}
\newcommand{\CapitalPICcode}{Código independiente de lá posición}
\newcommand{\Loops}{Bucles}

% C
\newcommand{\PostIncrement}{Post-incremento}
\newcommand{\PostDecrement}{Post-decremento}
\newcommand{\PreIncrement}{Pre-incremento}
\newcommand{\PreDecrement}{Pre-decremento}

% MIPS
\newcommand{\GlobalPointer}{Apuntador Global}

\newcommand{\garbage}{Basura}
\newcommand{\IntelSyntax}{Sintaxis Intel}
\newcommand{\ATTSyntax}{Sintaxis AT\&T}
\newcommand{\randomNoise}{Ruido aleatorio}
\newcommand{\Example}{Ejemplo}
\newcommand{\argument}{argumento}
\newcommand{\MarkedInIDAAs}{Marcado en \IDA como}
\newcommand{\stepover}{pasar por encima}
\newcommand{\ShortHotKeyCheatsheet}{Cheatsheet de teclas de acceso rápido}

\newcommand{\assemblyOutput}{salida del ensamblador}

% ML prefix is for multi-lingual words and sentences:
\newcommand{\MLHeap}{\ESph{}}
\newcommand{\MLStack}{\ESph{}}
\newcommand{\MLStackOverflow}{\ESph{}}
\newcommand{\MLStartOfHeap}{\ESph{}}
\newcommand{\MLStartOfStack}{\ESph{}}
\newcommand{\MLinputA}{\ESph{}}
\newcommand{\MLinputB}{\ESph{}}
\newcommand{\MLoutput}{\ESph{}}
\newcommand{\SoftwareCracking}{\ESph{}}

}
\NL{\input{macro_lang_NL}}
\TR{\input{macro_lang_TR}}
\PTBR{\newcommand{\ReturnAddress}{Endereço de retorno}

\newcommand{\localVariable}{Variável local}

\newcommand{\savedValueOf}{Valor salvo de}

% for index
\newcommand{\GrepUsage}{Uso do grep}
\newcommand{\SyntacticSugar}{Açúcar sintático}
\newcommand{\CompilerAnomaly}{Anomalias do compilador}
\newcommand{\CLanguageElements}{Elementos da linguagem C}
\newcommand{\CStandardLibrary}{Biblioteca padrão C}
\newcommand{\Instructions}{Instruções}
\newcommand{\Pseudoinstructions}{Pseudo-instruções}
\newcommand{\Prefixes}{Prefixos}

\newcommand{\Flags}{Flags}
\newcommand{\Registers}{Registradores}
\newcommand{\registers}{registradores}
\newcommand{\Stack}{Pilha}
\newcommand{\Recursion}{Recursividade}
\newcommand{\RAM}{RAM}
\newcommand{\ROM}{ROM}
\newcommand{\Pointers}{Ponteiros}
\newcommand{\BufferOverflow}{Buffer Overflow}

% DE: also "Zusammenfassung"
\newcommand{\Conclusion}{Conclusão}

\newcommand{\Exercise}{Exercício}
\newcommand{\Exercises}{Exercícios}
\newcommand{\Arrays}{Matriz}
\newcommand{\Cpp}{C++\xspace}
\newcommand{\CCpp}{C/C++\xspace}
\newcommand{\NonOptimizing}{Sem otimização\xspace}
\newcommand{\Optimizing}{Com otimização\xspace}
\newcommand{\ARMMode}{Modo ARM\xspace}
\newcommand{\ThumbMode}{Modo Thumb\xspace}
\newcommand{\ThumbTwoMode}{Modo Thumb-2\xspace}
\newcommand{\AndENRU}{e\xspace}
\newcommand{\OrENRU}{ou\xspace}
\newcommand{\InENRU}{em\xspace}
\newcommand{\ForENRU}{para\xspace}
\newcommand{\LineENRU}{linha\xspace}

\newcommand{\DataProcessingInstructionsFootNote}{Estas intruções também são chamadas \q{data processing instructions}}

% for .bib files
\newcommand{\AlsoAvailableAs}{Também disponível como\xspace}

% section names
\newcommand{\ShiftsSectionName}{Shifts}
\newcommand{\SwitchCaseDefaultSectionName}{switch()/case/default}
\newcommand{\BitfieldsChapter}{Manipulando bit(s) específicos}
\newcommand{\ArithOptimizations}{Substituição de instruções aritiméticas por outras}
\newcommand{\FPUChapterName}{Unidade de Ponto flutuante}
\newcommand{\DivisionByMultSectionName}{Divisão por 9}
\newcommand{\Answer}{Responda}
\newcommand{\WhatThisCodeDoes}{O que este código faz}
\newcommand{\WorkingWithFloatAsWithStructSubSubSectionName}{Trabalhando com o tipo float como uma estrutura}

\newcommand{\MinesweeperWinXPExampleChapterName}{Campo minado (Windows XP)}
\newcommand{\StructurePackingSectionName}{Organização de campos na estrutura}
\newcommand{\StructuresChapterName}{Estruturas}
\newcommand{\PICcode}{código independente de posição}
\newcommand{\CapitalPICcode}{Código independente de posição}
\newcommand{\Loops}{Laços}

% C
\newcommand{\PostIncrement}{Pós-incremento}
\newcommand{\PostDecrement}{Pós-decremento}
\newcommand{\PreIncrement}{Pré-incremento}
\newcommand{\PreDecrement}{Pré-decremento}

% MIPS
\newcommand{\GlobalPointer}{Ponteiro Global}

\newcommand{\garbage}{Lixo}
\newcommand{\IntelSyntax}{Sintaxe Intel}
\newcommand{\ATTSyntax}{Sintaxe AT\&T}
\newcommand{\randomNoise}{Ruído aleatório}
\newcommand{\Example}{Exemplo}
\newcommand{\argument}{argumento}
\newcommand{\MarkedInIDAAs}{Marcado no \IDA como}
\newcommand{\stepover}{passar por cima}
\newcommand{\ShortHotKeyCheatsheet}{Cheatsheet de teclas de atalho}

\newcommand{\assemblyOutput}{saída do assembly}

% ML prefix is for multi-lingual words and sentences:
\newcommand{\MLHeap}{Heap}
\newcommand{\MLStack}{Pilha}
\newcommand{\MLStackOverflow}{\PTBRph{}}
\newcommand{\MLStartOfHeap}{começo da heap}
\newcommand{\MLStartOfStack}{começo da pilha}
\newcommand{\SoftwareCracking}{\PTBRph{}}

}

\ifdefined\UAL
\newcommand{\TitleMain}{\TitleUAL}
\newcommand{\TitleAux}{\TitleRE}
\else
\newcommand{\TitleMain}{\TitleRE}
\newcommand{\TitleAux}{\TitleUAL}
\fi

\EN{\input{glossary_EN}}
\RU{\input{glossary_RU}}
\FR{\input{glossary_FR}}
\DE{\input{glossary_DE}}
\IT{\newglossaryentry{caller}
{
  name=caller,
  description={\ITph{}}
}

\newglossaryentry{callee}
{
  name=callee,
  description={\ITph{}}
}

\newglossaryentry{stack pointer}
{
  name={\ITph{}},
  description={\ITph{}}
}

\newglossaryentry{prodotto}
{
  name={\ITph{}},
  description={\ITph{}}
}

\newglossaryentry{GiB}
{
  name={GiB},
  description={\ITph{}}
}

\newglossaryentry{increment}
{
  name={\ITph},
  description={\ITph{}}
}

\newglossaryentry{decrement}
{
  name={\ITph},
  description={\ITph{}}
}

\newglossaryentry{stdout}
{
  name={\ITph},
  description={\ITph{}}
}

\newglossaryentry{endianness}
{
  name={\ITph},
  description={\ITph{}}
}

\newglossaryentry{thunk function}
{
  name={\ITph},
  description={\ITph{}}
}

\newglossaryentry{leaf function}
{
  name={\ITph},
  description={\ITph{}}
}

\newglossaryentry{heap}
{
  name={\ITph},
  description={\ITph{}}
}

\newglossaryentry{link register}
{
  name={\ITph},
  description={\ITph{}}
}

\newglossaryentry{anti-pattern}
{
  name={\ITph},
  description={\ITph{}}
}

\newglossaryentry{stack frame}
{
  name={\ITph},
  description={\ITph{}}
}

\newglossaryentry{jump offset}
{
  name={\ITph},
  description={\ITph{}}
}

\newglossaryentry{loop unwinding}
{
  name={\ITph},
  description={\ITph{}}
}

\newglossaryentry{tracer}
{
  name={\ITph},
  description={\ITph{}}
}

\newglossaryentry{register allocator}
{
  name={\ITph},
  description={\ITph{}}
}

\newglossaryentry{quotient}
{
  name={\ITph},
  description={\ITph{}}
}

\newglossaryentry{real number}
{
  name={\ITph},
  description={\ITph{}}
}

\newglossaryentry{NaN}
{
  name={\ITph},
  description={\ITph{}}
}

\newglossaryentry{Windows NT}
{
  name={\ITph},
  description={\ITph{}}
}

\newglossaryentry{word}
{
  name={\ITph},
  description={\ITph{}}
}

\newglossaryentry{PDB}
{
  name={\ITph},
  description={\ITph{}}
}

\newglossaryentry{name mangling}
{
  name={\ITph},
  description={\ITph{}}
}

\newglossaryentry{padding}
{
  name={\ITph},
  description={\ITph{}}
}

\newglossaryentry{NOP}
{
  name={\ITph},
  description={\ITph{}}
}

\newglossaryentry{POKE}
{
  name={\ITph},
  description={\ITph{}}
}

\newglossaryentry{xoring}
{
  name={\ITph},
  description={\ITph{}}
}

\newglossaryentry{atomic operation}
{
  name={\ITph},
  description={\ITph{}}
}

\newglossaryentry{basic block}
{
  name={\ITph},
  description={\ITph{}}
}

\newglossaryentry{reverse engineering}
{
  name={\ITph},
  description={\ITph{}}
}

\newglossaryentry{compiler intrinsic}
{
  name={\ITph},
  description={\ITph{}}
}


}
\JA{\input{glossary_JA}}

\makeglossaries

\hypersetup{
    colorlinks=true,
    allcolors=blue,
    pdfauthor={\AUTHOR},
    pdftitle={\TitleMain}
    }

%\ifdefined\RUSSIAN
\newcommand{\LstStyle}{\ttfamily\small}
%\else
%\newcommand{\LstStyle}{\ttfamily}
%\fi

% inspired by http://prismjs.com/
\definecolor{digits}{RGB}{0,0,0}
\definecolor{bg}{RGB}{255,255,255}
%\definecolor{bg}{RGB}{255,252,250}
\definecolor{col1}{RGB}{154,20,150}
\definecolor{col2}{RGB}{112,128,144}
\definecolor{col3}{RGB}{10,120,180}
\definecolor{col4}{RGB}{106,164,108}

\lstset{
    %backgroundcolor=\color{lstbgcolor},
    %backgroundcolor=\color{light-gray},
    backgroundcolor=\color{bg},
    basicstyle=\LstStyle,
    breaklines=true,
    %prebreak=\raisebox{0ex}[0ex][0ex]{->},
    %postbreak=\raisebox{0ex}[0ex][0ex]{->},
    prebreak=\raisebox{0ex}[0ex][0ex]{\ensuremath{\rhookswarrow}},
    postbreak=\raisebox{0ex}[0ex][0ex]{\ensuremath{\rcurvearrowse\space}},
    frame=single,
    columns=fullflexible,keepspaces,
    escapeinside=§§,
    inputencoding=utf8
}

\input{syntax_color}

\ifdefined\RUSSIAN
\renewcommand\lstlistingname{Листинг}
\renewcommand\lstlistlistingname{Листинг}
\fi

\DeclareMathSizes{12}{30}{16}{12}%

% see also:
% http://tex.stackexchange.com/questions/129225/how-can-i-get-get-makeindex-to-ignore-capital-letters
% http://tex.stackexchange.com/questions/18336/correct-sorting-of-index-entries-containing-macros
\def\myindex#1{\expandafter\index\expandafter{#1}}

\begin{document}

\iffalse
% fancyhdr =============================================================================================================================
\pagestyle{fancy}
\setlength{\headheight}{13pt}
% https://tex.stackexchange.com/questions/10043/page-number-position
\fancyhf{}
\fancyhead[R]{\thepage} % suppress chapter name, add page number (upper right corner)

\ifdefined\ENGLISH
\cfoot{\small (For statistics collecting purposes) if you've read this far, please click \href{https://beginners.re/stat/EN \today p.\thepage}{here}. Thanks! \normalsize}
% https://tex.stackexchange.com/questions/13406/how-to-add-a-horizontal-line-above-the-footer-with-fancyhdr
\renewcommand{\footrulewidth}{0.4pt}
\fi

\ifdefined\RUSSIAN
\cfoot{\small (В целях сбора статистики) если вы дочитали до этого места, пожалуйста, нажмите \href{https://beginners.re/stat/RU \today p.\thepage}{здесь}. Спасибо! \normalsize}
% https://tex.stackexchange.com/questions/13406/how-to-add-a-horizontal-line-above-the-footer-with-fancyhdr
\renewcommand{\footrulewidth}{0.4pt}
\fi
% ======================================================================================================================================
\fi

\VerbatimFootnotes

\frontmatter

\RU{% To translators: don't bother to translate this... english-only version.

\begin{center}
\LARGE{} Это моя собственная доска объявления \normalsize{}
\end{center}

\textbf{Эта книга наверняка уже устарела}.
(Если только не была скачана прямо сейчас с \url{https://beginners.re/}.)

Книга \href{\GitHubURL/commits/master}{меняется очень часто},
контент добавляется, ошибки (будем надеяться) исправляются.
Также, в первую очередь книга пишется на английском, а перевод на русский немного запаздывает.
Последняя версия всегда на \url{https://beginners.re/}.

А PDF-файл, который вы сейчас читаете, был скомпилирован \today{}.

\myhrule{}

Если вы распечатали эту книгу на бумаге, не могли бы вы прислать мне её фотографию, для коллекции?\\
\EMAIL{}, Telegram: @yurichev.

\myhrule{}

Мои дорогие читатели! Время от времени, у меня появляются вопросы, и я не знаю, кого (или где) спросить.
Или я просто ленив...
Поможете мне?

\myhrule{}

Есть вопросы по математической статистике. Кто-нибудь может помочь?

\myhrule{}

Где можно накачать баз и телефонных справочников, которые используются на сайте \url{http://nomerorg.website/}?

\myhrule{}

Блютузовые наушники ERGO BT-590 имеют сенсорные кнопки, слишком чувствительные, и их легко задеть одеждой.
Как в Андроиде сделать так, чтобы Андроид игнорировал сообщения от наушников о нажатии кнопок?

\myhrule{}

Есть комп с дешевыми кулерами, в старом, дешевом и раздолбанном кузове.
Linux раз в 1-2 секунды меняет скорость вращения кулера процессора, но немного.
Например, между 833 RPM и 834 RPM.
И кулера и кузов дешевые, они друг с другом, видимо, немного резонируют,
поэтому от таких изменений меняется тон гула/шума, раз в 1-2 секунды, что дико раздражает.
Как заставить Linux менять скорость реже, например один раз в 10-20 секунд?
\verb|INTERVAL=10| в \verb|/etc/fancontrol| не помогает.

Тут конфиг и вывод sensors: \url{https://pastebin.com/yk7EuiBL}.

Мать: Intel Desktop Board DZ77SL-50K.

\myhrule{}

Кто-нибудь может мне помочь с CBMC? Есть вопросы...

\myhrule{}

В самом начале 1990-х вышла книга на русском, где были законы Мерфи, Паркинсона, итд...
Как называлась?
Забыл.

\myhrule{}

Какой меломанский HiFi mp3-плеер за \$200-300 выбрать?
Hifiman HM-601 меня устраивал целиком...

\myhrule{}

Нужно проиндексировать пачку текстов. Потом сделать поиск по ним. Желателен простенький query-язык.
Какую поискать легковесную библиотеку для этого?
Желательно Питон или С++.

\myhrule{}

Как инсталлировать и запускать Cyc?

\myhrule{}

Win32-процесс А запущен.
Процесс Б аттачится к нему как отладчик, либо открывает его используя OpenProcess().
ReadProcessMemory() работает, но не работает, если пытается читать незакоммиченные страницы процесса А.

Проблема: как заставить менеджер памяти Windows закоммитить страницу в процессе А из userland-а процесса Б?
Я могу всунуть в процесс А инструкцию чтения, запустить её, страница закоммитится, но это не решение.

\myhrule{}

Если знаете что-то, пожалуйста помогите мне: \EMAIL{}, Telegram: @yurichev
}
\EN{% To translators: don't bother to translate this... english-only version.

\begin{center}
\LARGE{} This is my own bulletin board \normalsize{}
\end{center}

\textbf{This book is probably outdated already}.
(Unless it was just downloaded from \url{https://beginners.re/}.)

The book is \href{https://github.com/DennisYurichev/RE-for-beginners/commits/master}{changing too often},
content being added, bugs are (hopefully) being fixed.
The latest version is always at \url{https://beginners.re/}.

This PDF you currently reading was compiled at \today{}.

\myhrule{}

If you have printed this book on paper, can you please send me a picture of it, for collection?\\
\EMAIL{}, Telegram: @yurichev.

\myhrule{}

My dear readers! From time to time, I have questions, I don't know who (or where) to ask.
Or I'm just lazy...
Can you please help me?

\myhrule{}

Can anyone help me with CBMC? I have a question.

\myhrule{}

What HiFi mp3-player for \$200-300 is good for its money?
I was happy with Hifiman HM-601...

\myhrule{}

A pack of texts are to be indexed. Then a search is required. A simple query-language is desirable.
What lightweight library would you recommend?
Preferably Python or C++.

\myhrule{}

How to install and run Cyc?

\myhrule{}

How do you install VMware Remote Console 10.0.4 on Ubuntu 19? It just suddenly exits during installation.
Is it known symptom?

Or what do you use to run VMware Workstation VMs on remote Ubuntu box?

\myhrule{}

What does tilde means in versions number in .dsc files, which describing Ubuntu packages?
Some kind of wildcard?
And what does \verb|>>| means?
\emph{Pipe} is just \emph{OR}?
For example:

\begin{lstlisting}
Build-Depends: cython-dbg | python-pyrex, ca-certificates, debhelper (>> 8.1.0~), python (>= 2.6.6-3), python-all-dev (>= 2.6.6-3), python-all-dbg (>= 2.6.6-3), python-configobj (>= 4.7.2+ds-2), python-docutils, python-paramiko, python-pycurl-dbg, python-subunit, python-testtools (>= 0.9.5~)
\end{lstlisting}

\myhrule{}

A win32 process A is running.
Process B is attaching to it as a debugger, or opens it using OpenProcess().
ReadProcessMemory() works OK, but fails if it tries to read uncommited memory pages of process A.

The problem: how to force the Windows Memory Manager to commit a page in process A from userland of process B?
I can inject a read instruction into process A, run it, and the page would be committed, but this is not the solution.

\myhrule{}

If you know something, please help me: \EMAIL{}, Telegram: @yurichev, Skype: dennis.yurichev

}
%\DE{\vspace*{\fill}

\iffalse
\huge
	Bitte nehmen Sie an der kurzen Umfrage teil, unter
\normalsize

\bigskip
\bigskip
\bigskip

\dots \url{https://beginners.re/survey.html}.
Dies kann sehr hilfreich für den Autor sein!

\bigskip
\bigskip
\bigskip
\fi

\huge
	\DE{Meine Leistungen}
\normalsize

\bigskip
\bigskip
\bigskip

Das vorliegende Buch ist \href{http://beginners.re/}{kostenlos} und
\href{\RepoURL}{als OpenSource erhältlich}.
Manchmal muss ich jedoch auch Geld verdienen, aus diesem Grund entschuldige ich mich im Voraus
für das Platzieren der Werbung an dieser Stelle.

\iffalse
\Large Benötigen Sie Dokumentationen? \normalsize

Ich kann versuchen Dokumentationen, Referenzen und Handbücher für einige APIs,
Sprachen, Frameworks und so weiter zu schreiben.

Manchmal bin ich gut im Finden von präzisen und klaren Beispielen für jedes API- oder Sprachfeature.
Dieses Buch ist ein Beispiel dafür.
Ich kann versuchen dies in einer ausführlichen und zuverlässigen Art zu tun.

Auf der anderen Seite ist mein Englisch weit entfernt davon fließend zu sein und
ich könnte lange brauchen um mich tief in Produkte einzuarbeiten die ich nicht kenne.

Ich wäre aber erfreut existierende Dokumentationsprojekte zu überarbeiten.
Eine Beispielreferenz die ich bewundere ist Wolfram Mathematica: \url{http://reference.wolfram.com/language/}.
\fi

\Large Reverse engineering \normalsize

Ich kann keine Vollzeit-Jobs annehmen. Meistens arbeite ich von zuhause aus an kleinen Aufgaben wie:

\large Entschlüsseln von Datenbanken, welche unbekannte Datentypen verwalten \normalsize

Aufgrund einer Geheimhaltungsvereinbarung kann ich nicht viel über den letzten Auftrag
sagen, aber der Inhalt des Abschnitts \myref{encrypted_DB1} entstammt realen Arbeiten von mir.

\large Nachprogrammieren ausführbarer Dateien wie alte EXE- oder DLL-Dateien in C/C++ \normalsize

\large Dongle \normalsize

Gelegentlich realisiere ich Ersatz für
\href{https://en.wikipedia.org/wiki/Software_protection_dongle}{Kopierschutzstecker} oder Dongle-Emulatoren.
In der Regel ist dies nicht erlaubt, deswegen bestehen die folgenden Bedingungen:

\begin{itemize}
\item die Herstellerfirma der Software existiert nach meinem besten Wissen nicht mehr;
\item die Software ist älter als 10 Jahre;
\item Sie haben einen Dongle um die Informationen auszulesen. Mit anderen Worten, kann ich Ihnen
nur helfen wenn Sie noch sehr alte Software benutzen mit der Sie komplett zufrieden sind, jedoch
einen Defekt des Dongles fürchten und keine Firma Ersatz liefern kann.
\end{itemize}

Dies schließt alte MS-DOS- und UNIX-Software mit ein. Produkte für exotischere Computer-Architekturen
(wie MIPS, DEC Alpha, PowerPC) akzeptiere ich ebenfalls.

Beispiele meiner Arbeit finden Sie hier:

\begin{itemize}
\item Mein Buch über Reverse Engineering beinhaltet einen Teil über Kopierschutzstecker: \myref{dongles}.
\item \href{http://yurichev.com/writings/z3_rockey.pdf}{Finden von unbekannten Algorithmen durch Eingangs-/-Ausgangs-Paare
und Z3 SMT-Solver-Artikel}
\item \href{http://yurichev.com/blog/56/}{über MicroPhar (93c46-basierte Dongle) Emulation in DosBox}.
\item \href{http://conus.info/dongle/src/microph.asm}{Quellcode vom DOS MicroPhar-Emulator mit der EMM386 I/O API}
\end{itemize}

\large Kontaktieren Sie mich \normalsize

E-Mail: \GTT{\EMAIL}.

\large Möchten Sie immer noch einen Reverse Engineer / Security-Forscher in Vollzeit engagieren? \normalsize

Sie könnten es hier versuchen: \href{https://www.reddit.com/r/ReverseEngineering/comments/49cza0/rreverseengineerings_2015_triannual_hiring_thread/}{Reddit RE Thread}.
Außerdem gibt es ein russischsprachiges Forum mit einem \href{https://forum.reverse4you.org/forumdisplay.php?f=252}{Abschnitt für RE-Jobs}.

\vspace*{\fill}
\vfill
}
%\ES{\input{1st_page_ES}}
%\CN{\vspace*{\fill}

\iffalse
\huge
    请参加一个小调查
\normalsize

\bigskip
\bigskip
\bigskip

\dots 地址在此: \url{https://beginners.re/survey.html}。
您的参与对作者至关重要!

\bigskip
\bigskip
\bigskip
\fi

\huge
    我能提供的服务
\normalsize

\bigskip
\bigskip
\bigskip


您正阅读的本书是\href{http://beginners.re/}{免费的}并已经\href{\GitHubURL}{以开放源代码的方式发布}。
但我有时需要做一些工作以获取收入。所以,很抱歉将我个人的广告信息发布在这里。

\iffalse
\Large 需要文档? \normalsize

我可以为一部分API、程序语言和框架等来编写文档/参考资料/使用手册。

通常,我擅于为各种API/程序语言的功能找出简明扼要的样例。
而本书便是样例之一。
我可以长期稳定地进行文档编写和维护工作。

另一方面,我的英语水平远未达到能称之为『熟练』的水平。
并且,我往往需要较长的时间以深入了解一个未知领域的产品。

然而,我将乐于重新修订现有的文档项目。

我所钦佩的,是Wolfram Mathematica所提供的文档: \url{http://reference.wolfram.com/language/}。
\fi

\Large 逆向工程 \normalsize

我并不能接受全职的工作请求。通常我可以远程完成一些小任务,例如:

\large 解密一个数据库,或对未知类型的文件进行管理 \normalsize

根据NDA协议,我不能在此披露更多关于案例的细节,但是在\myref{encrypted_DB1}一节阐述的内容很大一部分基于一个真实案例。

\large 将旧版本的EXE或DLL文件重写为C/C++代码 \normalsize

\large 加密狗\footnote{接在计算机上的外接设备,通常用于防止软件被拷贝或修改} \normalsize

我偶尔会制作\href{https://en.wikipedia.org/wiki/Software_protection_dongle}{加密狗}的替代品或模拟器。通常,破坏软件的保护措施是违反法律的,所以我仅在满足以下条件的时候才进行此类工作:

\begin{itemize}
\item 据我所知,开发该软件的软件公司已不复存在;
\item 该软件已经有10年以上的寿命;
\item 你持有一个可以从中读取信息的加密狗设备。换句话说,我只为那些仍在使用很老的软件、并完全满足于其功能,却担心因加密狗损坏而没有途径购买并替换的人服务。
\end{itemize}

上文所述的工作也涵盖了古老的MS-DOS和UNIX程序的情况。另外,我也接受基于其他计算机架构(如MIPS, DEC Alpha, PowerPC)的工作。

关于我的工作,您可以在这里看到一些样例:

\begin{itemize}
\item 我这本关于逆向工程的书中,有一节专门介绍了用于防止拷贝的加密狗设备: \myref{dongles}。
\item \href{http://yurichev.com/writings/z3_rockey.pdf}{仅通过输入、输出对和Z3 SMT求解器探索未知算法的文章}
\item \href{http://yurichev.com/blog/56/}{关于在DosBox中模拟MicroPhar(一种基于93c46的加密狗)的细节}
\item \href{http://conus.info/dongle/src/microph.asm}{基于EMM386的I/O中断API的DOS MicroPhar模拟器源码}
\end{itemize}

\large 联系我 \normalsize

E-Mail: \GTT{\EMAIL}

\large 依旧想要雇用全职的逆向工程师/安全研究员? \normalsize

您可以尝试\href{https://www.reddit.com/r/ReverseEngineering/comments/49cza0/rreverseengineerings_2015_triannual_hiring_thread/}{Reddit的逆向工程话题的招聘版面}。
另外,这里还有一个俄语论坛,其中包含了一个\href{https://forum.reverse4you.org/forumdisplay.php?f=252}{关于逆向工程工作的版面}。

\vspace*{\fill}
\vfill
}

\iffalse
\RU{\input{dedication_RU}}
\EN{\input{dedication_EN}}
\FR{\input{dedication_FR}}
\JA{\input{dedication_JA}}
\DE{\input{dedication_DE}}
\IT{\input{dedication_IT}}
\fi

\input{page_after_cover}
\input{call_for_translators}

\shorttoc{%
    \RU{Краткое оглавление}%
    \EN{Abridged contents}%
    \ES{Contenidos abreviados}%
    \PTBRph{}%
    \DE{Inhaltsverzeichnis (gekürzt)}%
    \PLph{}%
    \IT{Sommario}%
    \THAph{}\NLph{}%
    \FR{Contenus abrégés}%
    \JA{簡略版}
    \TR{İçindekiler}
}{0}

\tableofcontents
\cleardoublepage

\cleardoublepage
\input{preface}

\mainmatter

% only chapters here!
\EN{% TODO translate
\mysection{Breaking simple executable cryptor}

I've got an executable file which is encrypted by relatively simple encryption.
\href{\GitHubBlobMasterURL/examples/simple_exec_crypto/files/cipher.bin}{Here is it} (only executable section is left here).

First, all encryption function does is just adds number of position in buffer to the byte.
Here is how this can be encoded in Python:

\begin{lstlisting}[caption=Python script,style=custompy]
#!/usr/bin/env python
def e(i, k):
    return chr ((ord(i)+k) % 256)

def encrypt(buf):
    return e(buf[0], 0)+ e(buf[1], 1)+ e(buf[2], 2) + e(buf[3], 3)+ e(buf[4], 4)+ e(buf[5], 5)+ e(buf[6], 6)+ e(buf[7], 7)+
           e(buf[8], 8)+ e(buf[9], 9)+ e(buf[10], 10)+ e(buf[11], 11)+ e(buf[12], 12)+ e(buf[13], 13)+ e(buf[14], 14)+ e(buf[15], 15)
\end{lstlisting}

Hence, if you encrypt buffer with 16 zeros, you'll get \emph{0, 1, 2, 3 ... 12, 13, 14, 15}.

\myindex{Propagating Cipher Block Chaining}
Propagating Cipher Block Chaining (PCBC) is also used, here is how it works:

\begin{figure}[H]
\centering
\myincludegraphics{examples/simple_exec_crypto/601px-PCBC_encryption.png}
\caption{Propagating Cipher Block Chaining encryption (image is taken from Wikipedia article)}
\end{figure}

The problem is that it's too boring to recover IV (Initialization Vector) each time.
Brute-force is also not an option, because IV is too long (16 bytes).
Let's see, if it's possible to recover IV for arbitrary encrypted executable file?

Let's try simple frequency analysis.
This is 32-bit x86 executable code, so let's gather statistics about most frequent bytes and opcodes.
I tried huge oracle.exe file from Oracle RDBMS version 11.2 for windows x86 and I've found that the most frequent byte (no surprise) is zero (~10\%).
The next most frequent byte is (again, no surprise) 0xFF (~5\%).
The next is 0x8B (~5\%).

\myindex{x86!\Instructions!MOV}
0x8B is opcode for \INS{MOV}, this is indeed one of the most busy x86 instructions.
Now what about popularity of zero byte?
If compiler needs to encode value bigger than 127, it has to use 32-bit displacement instead of 8-bit one, but large values are very rare,
so it is padded by zeros.
\myindex{x86!\Instructions!LEA}
\myindex{x86!\Instructions!PUSH}
\myindex{x86!\Instructions!CALL}
This is at least in \INS{LEA}, \INS{MOV}, \INS{PUSH}, \INS{CALL}.

For example:

\begin{lstlisting}[style=customasmx86]
8D B0 28 01 00 00                 lea     esi, [eax+128h]
8D BF 40 38 00 00                 lea     edi, [edi+3840h]
\end{lstlisting}

Displacements bigger than 127 are very popular, but they are rarely exceeds 0x10000
(indeed, such large memory buffers/structures are also rare).

Same story with \INS{MOV}, large constants are rare, the most heavily used are 0, 1, 10, 100, $2^n$, and so on.
Compiler has to pad small constants by zeros to represent them as 32-bit values:

\begin{lstlisting}[style=customasmx86]
BF 02 00 00 00                    mov     edi, 2
BF 01 00 00 00                    mov     edi, 1
\end{lstlisting}

Now about 00 and FF bytes combined: jumps (including conditional) and calls can pass execution flow forward or backwards, but very often,
within the limits of the current executable module.
If forward, displacement is not very big and also padded with zeros.
If backwards, displacement is represented as negative value, so padded with FF bytes.
For example, transfer execution flow forward:

\begin{lstlisting}[style=customasmx86]
E8 43 0C 00 00                    call    _function1
E8 5C 00 00 00                    call    _function2
0F 84 F0 0A 00 00                 jz      loc_4F09A0
0F 84 EB 00 00 00                 jz      loc_4EFBB8
\end{lstlisting}

Backwards:

\begin{lstlisting}[style=customasmx86]
E8 79 0C FE FF                    call    _function1
E8 F4 16 FF FF                    call    _function2
0F 84 F8 FB FF FF                 jz      loc_8212BC
0F 84 06 FD FF FF                 jz      loc_FF1E7D
\end{lstlisting}

FF byte is also very often occurred in negative displacements like these:

\begin{lstlisting}[style=customasmx86]
8D 85 1E FF FF FF                 lea     eax, [ebp-0E2h]
8D 95 F8 5C FF FF                 lea     edx, [ebp-0A308h]
\end{lstlisting}

So far so good. Now we have to try various 16-byte keys, decrypt executable section and measure how often 00, FF and 8B bytes are occurred.
Let's also keep in sight how PCBC decryption works:

\begin{figure}[H]
\centering
\myincludegraphics{examples/simple_exec_crypto/640px-PCBC_decryption.png}
\caption{Propagating Cipher Block Chaining decryption (image is taken from Wikipedia article)}
\end{figure}

The good news is that we don't really have to decrypt whole piece of data, but only slice by slice, this is exactly how I did in my previous example: \myref{XOR_mask_2}.

Now I'm trying all possible bytes (0..255) for each byte in key and just pick the byte producing maximal amount of 00/FF/8B bytes in a decrypted slice:

\begin{lstlisting}[style=custompy]
#!/usr/bin/env python
import sys, hexdump, array, string, operator

KEY_LEN=16

def chunks(l, n):
    # split n by l-byte chunks
    # https://stackoverflow.com/q/312443
    n = max(1, n)
    return [l[i:i + n] for i in range(0, len(l), n)]

def read_file(fname):
    file=open(fname, mode='rb')
    content=file.read()
    file.close()
    return content

def decrypt_byte (c, key):
    return chr((ord(c)-key) % 256)

def XOR_PCBC_step (IV, buf, k):
    prev=IV
    rt=""
    for c in buf:
	new_c=decrypt_byte(c, k)
        plain=chr(ord(new_c)^ord(prev))
	prev=chr(ord(c)^ord(plain))
	rt=rt+plain
    return rt

each_Nth_byte=[""]*KEY_LEN

content=read_file(sys.argv[1])
# split input by 16-byte chunks:
all_chunks=chunks(content, KEY_LEN)
for c in all_chunks:
    for i in range(KEY_LEN):
        each_Nth_byte[i]=each_Nth_byte[i] + c[i]

# try each byte of key
for N in range(KEY_LEN):
    print "N=", N
    stat={}
    for i in range(256):
        tmp_key=chr(i)
	tmp=XOR_PCBC_step(tmp_key,each_Nth_byte[N], N)
        # count 0, FFs and 8Bs in decrypted buffer:
	important_bytes=tmp.count('\x00')+tmp.count('\xFF')+tmp.count('\x8B')
	stat[i]=important_bytes
    sorted_stat = sorted(stat.iteritems(), key=operator.itemgetter(1), reverse=True)
    print sorted_stat[0]
\end{lstlisting}

(Source code can be downloaded \href{\GitHubBlobMasterURL/examples/simple_exec_crypto/files/decrypt.py}{here}.)

I run it and here is a key for which 00/FF/8B bytes presence in decrypted buffer is maximal:

\begin{lstlisting}
N= 0
(147, 1224)
N= 1
(94, 1327)
N= 2
(252, 1223)
N= 3
(218, 1266)
N= 4
(38, 1209)
N= 5
(192, 1378)
N= 6
(199, 1204)
N= 7
(213, 1332)
N= 8
(225, 1251)
N= 9
(112, 1223)
N= 10
(143, 1177)
N= 11
(108, 1286)
N= 12
(10, 1164)
N= 13
(3, 1271)
N= 14
(128, 1253)
N= 15
(232, 1330)
\end{lstlisting}

Let's write decryption utility with the key we got:

\begin{lstlisting}[style=custompy]
#!/usr/bin/env python
import sys, hexdump, array

def xor_strings(s,t):
    # \verb|https://en.wikipedia.org/wiki/XOR_cipher#Example_implementation|
    """xor two strings together"""
    return "".join(chr(ord(a)^ord(b)) for a,b in zip(s,t))

IV=array.array('B', [147, 94, 252, 218, 38, 192, 199, 213, 225, 112, 143, 108, 10, 3, 128, 232]).tostring()

def chunks(l, n):
    n = max(1, n)
    return [l[i:i + n] for i in range(0, len(l), n)]

def read_file(fname):
    file=open(fname, mode='rb')
    content=file.read()
    file.close()
    return content

def decrypt_byte(i, k):
    return chr ((ord(i)-k) % 256)

def decrypt(buf):
    return "".join(decrypt_byte(buf[i], i) for i in range(16))

fout=open(sys.argv[2], mode='wb')

prev=IV
content=read_file(sys.argv[1])
tmp=chunks(content, 16)
for c in tmp:
    new_c=decrypt(c)
    p=xor_strings (new_c, prev)
    prev=xor_strings(c, p)
    fout.write(p)
fout.close()
\end{lstlisting}

(Source code can be downloaded \href{\GitHubBlobMasterURL/examples/simple_exec_crypto/files/decrypt2.py}{here}.)

Let's check resulting file:

\lstinputlisting{examples/simple_exec_crypto/objdump_result.txt}

Yes, this is seems correctly disassembled piece of x86 code.
The whole decryped file can be downloaded \href{\GitHubBlobMasterURL/examples/simple_exec_crypto/files/decrypted.bin}{here}.

In fact, this is text section from regedit.exe from Windows 7.
But this example is based on a real case I encountered, so just executable is different (and key), algorithm is the same.

\subsection{Other ideas to consider}

What if I would fail with such simple frequency analysis?
There are other ideas on how to measure correctness of decrypted/decompressed x86 code:

\begin{itemize}

\item Many modern compilers aligns functions on 0x10 border.
So the space left before is filled with NOPs (0x90) or other NOP instructions with known opcodes: \myref{sec:npad}.

\item Perhaps, the most frequent pattern in any assembly language is function call:\\
\TT{PUSH chain / CALL / ADD ESP, X}.
This sequence can easily detected and found.
I've even gathered statistics about average number of function arguments: \myref{args_stat}.
(Hence, this is average length of PUSH chain.)

\end{itemize}

Read more about incorrectly/correctly disassembled code: \myref{ISA_detect}.
}%
\FR{\mysection{Une fonction vide: redux}

Revenons sur l'exemple de la fonction vide \myref{empty_func}.
Maintenant que nous connaissons le prologue et l'épilogue de fonction, ceci est
une fonction vide \myref{lst:empty_func} compilée par GCC sans optimisation:

\lstinputlisting[caption=GCC 8.2 x64 \NonOptimizing (\assemblyOutput),style=customasmx86]{patterns/016_empty_redux/1.s}

C'est \INS{RET}, mais le prologue et l'épilogue de la fonction, probablement, n'ont
pas été optimisés et laissés tels quels.
\INS{NOP} semble être un autre artefact du compilateur.
De toutes façons, la seule instruction effective ici est \INS{RET}.
Toutes les autres instructions peuvent être supprimées (ou optimisées).

}

% FIXME: why chapter?! it should be section
\chapter{%
	\RU{Важные фундаментальные вещи}%
	\EN{Important fundamentals}%
	\ES{Fundamentos importantes}%
	\PTBRph{}%
	\DE{Wichtige Grundlagen}%
	\FR{Fondamentaux importants}%
	\PLph{}%
	\ITph{}%
	\JAph{}%
}

% sections
\EN{% TODO rework structure and hierarchy
\mysection{Integral datatypes}

Integral datatype is a type for a value which can be converted to number.
These are numbers, enumerations, booleans.

\subsection{Bit}

Obvious usage for bits are boolean values: 0 for \emph{false} and 1 for \emph{true}.

Set of booleans can be packed into \gls{word}: there will be 32 booleans in 32-bit word, etc.
This way is called \emph{bitmap} or \emph{bitfield}.

But it has obvious overhead: a bit jiggling, isolating, etc.
While using \gls{word} (or \emph{int} type) for boolean variable is not economic, but highly efficient.

In C/C++ environment, 0 is for \emph{false} and any non-zero value is for \emph{true}.
For example:

\lstinputlisting[style=customc]{fundamentals/data_types_and_numbers_EN_lst1.c}

This is popular way of enumerating characters in a C-string:

\lstinputlisting[style=customc]{fundamentals/data_types_and_numbers_EN_lst2.c}

\subsection{Nibble AKA nybble}

\ac{AKA} half-byte, tetrade.
Equals to 4 bits.

All these terms are still in use today.

\subsubsection{Binary-coded decimal (\ac{BCD})}
\label{BCD}

\myindex{Intel 4004}

4-bit nibbles were used in 4-bit CPUs like legendary Intel 4004 (used in calculators).

It's interesting to know that there was \emph{binary-coded decimal} (\ac{BCD}) way of representing decimal digit using 4 bits.
Decimal 0 is represented as 0b0000, decimal 9 as 0b1001 and higher values are not used.
Decimal 1234 is represented as 0x1234.
Of course, this way is not economical.

Nevertheless, it has one advantage: decimal to \ac{BCD}-packed number conversion and back is extremely easy.
BCD-numbers can be added, subtracted, etc., but an additional correction is needed.
x86 CPUs has rare instructions for that:
\INS{AAA}/\INS{DAA} (adjust after addition),
\INS{AAS}/\INS{DAS} (adjust after subtraction),
\INS{AAM} (after multiplication),
\INS{AAD} (after division).

\myindex{x86!\Registers!AF}
The need for CPUs to support \ac{BCD} numbers is a reason why \emph{half-carry flag} (on 8080/Z80) and
\emph{auxiliary flag} (\TT{AF} on x86)
are exist: this is carry-flag generated after proceeding of lower 4 bits. The flag is then used for adjustment instructions.

The fact of easy conversion had led to popularity of
\InSqBrackets{Peter Abel, \emph{IBM PC assembly language and programming} (1987)} book.
But aside of this book, the author of these notes never seen \ac{BCD} numbers in practice, except for
\emph{magic numbers} (\myref{magic_numbers}),
like when someone's birthday is encoded like 0x19791011---this is indeed packed \ac{BCD} number.

Surprisingly, the author found a use of \ac{BCD}-encoded numbers in SAP software: \url{https://yurichev.com/blog/SAP/}.
Some numbers, including prices, are encoded in \ac{BCD} form in database.
Perhaps, they used it to make it compatible with some ancient software/hardware?

\ac{BCD} instructions in x86 were often used for other purposes, especially in undocumented ways, for example:

\begin{lstlisting}[style=customasmx86]
	cmp al,10
	sbb al,69h
	das
\end{lstlisting}

This obscure code converts number in 0..15 range into \ac{ASCII} character '0'..'9', 'A'..'F'.

\myparagraph{Z80}
\myindex{Z80}

Z80 was clone of 8-bit Intel 8080 CPU, and because of space constraints, it has 4-bit \ac{ALU}, i.e., each operation
over two 8-bit numbers had to be proceeded in two steps.
One side-effect of this was easy and natural generation of \emph{half-carry flag}.

\subsection{Byte}

Byte is primarily used for character storage.
8-bit bytes were not common as today.
Punched tapes for teletypes had 5 and 6 possible holes, this is 5 or 6 bits for byte.

\myindex{octet}
\myindex{fetchmail}
To emphasize the fact the byte has 8 bits, byte is sometimes called \emph{octet}:
at least \emph{fetchmail} uses this terminology.

9-bit bytes used to exist in 36-bit architectures: 4 9-bit bytes would fit in a single \gls{word}.
Probably because of this fact, C/C++ standard tells that \emph{char} has to have a room for \emph{at least} 8 bits, but more
bits are allowable.

For example, in the early C language manual\footnote{\url{https://yurichev.com/mirrors/C/bwk-tutor.html}}, we can find this:

\begin{lstlisting}
char  one byte character (PDP-11, IBM360: 8 bits; H6070: 9 bits)
\end{lstlisting}

\myindex{Honeywell 6070}
By H6070 they probably meant Honeywell 6070, with 36-bit words.

\subsubsection{Standard ASCII table}

7-bit ASCII table is standard, which has only 128 possible characters.
Early E-Mail transport software were operating only on 7-bit ASCII codes, so a \ac{MIME} standard needed to encode messages
in non-Latin writing systems.
7-bit ASCII code was augmented by parity bit, resulting in 8 bits.

\emph{Data Encryption Standard} (\ac{DES}) has a 56 bits key, this is 8 7-bit bytes,
leaving a space to parity bit for each character.

There is no need to memorize whole \ac{ASCII} table, but rather ranges.
\InSqBrackets{0..0x1F} are control characters (non-printable).
\InSqBrackets{0x20..0x7E} are printable ones.
Codes starting at 0x80 are usually used for non-Latin writing systems and/or pseudographics.

Significant codes which will be easily memorized are:
0 (end of C-string, \TT{'\textbackslash{}0'} in C/C++);
0xA or 10 (\emph{line feed}, \TT{'\textbackslash{}n'} in C/C++);
0xD or 13 (\emph{carriage return}, \TT{'\textbackslash{}r'} in C/C++).

0x20 (space) is also often memorized.

\subsubsection{8-bit CPUs}

x86 has capability to work with byte(s) on register level (because they are descendants of 8-bit 8080 CPU),
RISC CPUs like ARM and MIPS---not.

\subsection{Wide char}
\myindex{UTF-16}
\myindex{UCS-2}

This is an attempt to support multi-lingual environment by extending byte to 16-bit.
Most well-known example is Windows NT kernel and win32 functions with \emph{W} suffix.
This is why each Latin character in plain English text string is interleaved with zero byte.
This encoding is called UCS-2 or UTF-16

Usually, \emph{wchar\_t} is synonym to 16-bit \emph{short} data type.

\subsection{Signed integer vs unsigned}

Some may argue, why unsigned data types exist at first place, since any unsigned number can be represented as signed.
Yes, but absence of sign bit in a value extends its range twice.
Hence, signed byte has range of -128..127, and unsigned one: 0..255.
Another benefit of using unsigned data types is self-documenting:
you define a variable which can't be assigned to negative values.

\myindex{Java}
Unsigned data types are absent in Java, for which it's criticized.
It's hard to implement cryptographical algorithms using boolean operations over signed data types.

Values like 0xFFFFFFFF (-1) are used often, mostly as error codes.

\subsection{Word}

\Gls{word} word is somewhat ambiguous term and usually denotes a data type fitting in \ac{GPR}.
Bytes are practical for characters, but impractical for other arithmetical calculations.

Hence, many \ac{CPU}s have \ac{GPR}s with width of 16, 32 or 64 bits.
Even 8-bit CPUs like 8080 and Z80 offer to work with 8-bit register pairs, each pair forming a 16-bit \emph{pseudoregister}
(\emph{BC}, \emph{DE}, \emph{HL}, etc.).
Z80 has some capability to work with register pairs, and this is, in a sense, some kind of 16-bit CPU emulation.

In general, if a CPU marketed as ``n-bit CPU'', this usually means it has n-bit \ac{GPR}s.

There was a time when hard disks and \ac{RAM} modules were marketed as having \emph{n} kilo-words instead of
\emph{b} kilobytes/megabytes.

For example, \emph{Apollo Guidance Computer}
has 2048 words of \ac{RAM}.
This was a 16-bit computer, so there was 4096 bytes of \ac{RAM}.

\emph{TX-0} had 64K of 18-bit words of magnetic core memory,
i.e., 64 kilo-words.

\emph{DECSYSTEM-2060}
could have up to 4096 kilowords of \emph{solid state memory}
(i.e., hard disks, tapes, etc).
This was 36-bit computer, so this is 18432 kilobytes or ~18 megabytes.

Essentially, why do you need bytes if you have words?
Mostly for text strings processing.
Words can be used in almost any other situations.

\myhrule{}

\emph{int} in C/C++ is almost always mapped to \gls{word}.
(Except of AMD64 architecture where \emph{int} is still 32-bit one, perhaps, for the reason of better portability.)

\emph{int} is 16-bit on PDP-11 and old MS-DOS compilers.
\emph{int} is 32-bit on VAX, on x86 starting at 80386, etc.

Even more than that, if type declaration for a variable is omitted in C/C++ program, \emph{int} is used silently by default.
Perhaps, this is inheritance of B programming language\footnote{\url{http://yurichev.com/blog/typeless/}}.

\myhrule{}

\ac{GPR} is usually fastest container for variable, faster than packed bit,
and sometimes even faster than byte (because there is no need to isolate a single bit/byte from \ac{GPR}).
Even if you use it as a container for loop counter in 0..99 range.

\myhrule{}

\Gls{word} in assembly language is still 16-bit for x86, because it was so for 16-bit 8086.
\emph{Double word} is 32-bit, \emph{quad word} is 64-bit.
That's why 16-bit words are declared using \TT{DW} in x86 assembly, 32-bit ones using \TT{DD} and 64-bit ones using \TT{DQ}.

\Gls{word} is 32-bit for ARM, MIPS, etc., 16-bit data types are called \emph{half-word} there.
Hence, \emph{double word} on 32-bit RISC is 64-bit data type.

\emph{GDB} has the following terminology: \emph{halfword} for 16-bit, \gls{word} for 32-bit and \emph{giant word} for 64-bit.

16-bit C/C++ environment on PDP-11 and MS-DOS has \emph{long} data type with width of 32 bits, perhaps,
they meant \emph{long word} or \emph{long int}?

32-bit C/C++ environment has \emph{long long} data type with width of 64 bits.

Now you see why the \emph{word} word is ambiguous.

\subsubsection{Should I use \emph{int}?}

Some people argue that \emph{int} shouldn't be used at all, because it ambiguity can lead to bugs.
For example, well-known \emph{lzhuf} library uses \emph{int} at one point and everything works fine on 16-bit architecture.
But if ported to architecture with 32-bit \emph{int}, it can crash: \url{http://yurichev.com/blog/lzhuf/}.

Less ambiguous types are defined in \emph{stdint.h} file:
\emph{uint8\_t}, \emph{uint16\_t}, \emph{uint32\_t}, \emph{uint64\_t}, etc.

\myindex{Donald E. Knuth}
Some people like Donald E. Knuth proposed\footnote{\url{http://www-cs-faculty.stanford.edu/~uno/news98.html}}
more sonorous words
for these types: \emph{byte/wyde/tetrabyte/octabyte}.
But these names are less popular than clear terms with inclusion of \emph{u} (\emph{unsigned}) character 
and number right into the type name.

\subsubsection{Word-oriented computers}

Despite the ambiguity of the \gls{word} term, modern computers are still word-oriented: \ac{RAM} and all levels of cache
are still organized by words, not by bytes.
However, size in bytes is used in marketing.
% <!-- TODO word length on intel, etc... -->

Access to RAM/cache by address aligned by word boundary is often cheaper than non-aligned.

During data structures development, which are supposed to be fast and efficient,
one should always take into consideration length of the \gls{word} on the CPU to be executed on.
Sometimes the compiler will do this for programmer, sometimes not.

\subsection{Address register}

For those who fostered on 32-bit and/or 64-bit x86, and/or RISC of 90s like ARM, MIPS, PowerPC, it's natural that
address bus has the same width as \ac{GPR} or \gls{word}.
Nevertheless, width of address bus can be different on other architectures.

8-bit Z80 can address $2^{16}$ bytes, using 8-bit registers pairs or dedicated registers (\emph{IX}, \emph{IY}).
\emph{SP} and \emph{PC} registers are also 16-bit ones.

\myindex{Cray}
Cray-1 supercomputer has 64-bit GPRs, but 24-bit address registers, so it can address $2^{24}$ 
(16 megawords or 128 megabytes).
RAM was very expensive in 1970s, and a typical Cray had 1048576 (0x100000) words of RAM or ~8MB.
So why to allocate 64-bit register for address or pointer?

8086/8088 CPUs had a really weird addressing scheme:
values of two 16-bit registers were summed in a weird manner resulting in a 20-bit address.
Perhaps, this was some kind of toy-level virtualization (\myref{8086_memory_model})?
8086 could run several programs (not simultaneously, though).

\myindex{ARM!ARM1}
Early ARM1 has an interesting artifact:

\begin{framed}
\begin{quotation}
Another interesting thing about the register file is the PC register is missing a few bits. Since the ARM1 uses 26-bit addresses, the top 6 bits are not used. Because all instructions are aligned on a 32-bit boundary, the bottom two address bits in the PC are always zero. These 8 bits are not only unused, they are omitted from the chip entirely.
\end{quotation}
\end{framed}

( \url{http://www.righto.com/2015/12/reverse-engineering-arm1-ancestor-of.html} )

Hence, it's physically not possible to push a value with one of two last bits set into PC register.
Nor it's possible to set any bits in high 6 bits of PC.

x86-64 architecture has virtual 64-bit pointers/addresses, but internally, width of address bus is 48 bits
(seems enough to address 256TB of \ac{RAM}).

\subsection{Numbers}

What are numbers used for?

When you see some number(s) altering in a CPU register, you may be interested in what this number means.
It's an important skill for a reverse engineer to determine possible data type from a set of changing numbers.

\subsubsection{Boolean}

If the number is switching from 0 to 1 and back, most chances that this value has boolean data type.

\subsubsection{Loop counter, array index}

Variable increasing from 0, like: 0, 1, 2, 3\dots---a good chance this is a loop counter and/or array index.

\subsubsection{Signed numbers}

If you see a variable which holds very low numbers and sometimes very high numbers,
like 0, 1, 2, 3, and 0xFFFFFFFF, 0xFFFFFFFE, 0xFFFFFFFD,
there's a good chance it is a signed variable in \emph{two's complement} form (\myref{sec:signednumbers}),
and last 3 numbers are -1, -2, -3.

\subsubsection{32-bit numbers}

There are numbers so large,
that there is even a special notation which exists to represent them (Knuth's up-arrow notation).
These numbers are so large so these are not practical for engineering, science and mathematics.

Almost all engineers and scientists are happy with IEEE 754 double precision floating point, which has maximal
value around $1.8 \cdot 10^{308}$.
(As a comparison, the number of atoms in the observable universe, is estimated to be between
$4 \cdot 10^{79}$ and $4 \cdot 10^{81}$.)

In fact, upper bound in practical computing is much, much lower.
In MS-DOS era 16-bit \emph{int} was used almost for everything (array indices, loop counters),
while 32-bit \emph{long} was used rarely.

During advent of x86-64, it was decided for \emph{int} to stay as 32 bit size integer, because, probably,
usage of 64-bit \emph{int} is even rarer.

I would say, 16-bit numbers in range 0..65535 are probably most used numbers in computing.

Given that, if you see unusually large 32-bit value like 0x87654321, this is a good chance this can be:

\begin{itemize}

\item this can still be a 16-bit number, but signed, between 0xFFFF8000 (-32768) and 0xFFFFFFFF (-1).
% TODO: [Example](https://github.com/DennisYurichev/random_notes/blob/master/timedate.md).
\item address of memory cell (can be checked using memory map feature of debugger).
\item packed bytes (can be checked visually).
\item bit flags.
\item something related to (amateur) cryptography.
\item magic number (\myref{magic_numbers}).
\item IEEE 754 floating point number (can also be checked).

\end{itemize}

Almost same story for 64-bit values.

\myparagraph{\dots so 16-bit \emph{int} is enough for almost everything?}

It's interesting to note: in \InSqBrackets{\MAbrash{} chapter 13}
we can find that there are plenty cases in which 16-bit variables are just enough.
In a meantime, Michael Abrash has a pity that 80386 and 80486 CPUs has so little available registers, so he offers to put
two 16-bit values into one 32-bit register and then to rotate it using
\INS{ROR reg, 16} (on 80386 and later) (\INS{ROL reg, 16} will also work) or 
\INS{BSWAP} (on 80486 and later) instruction.

That reminds us Z80 with alternate pack of registers (suffixed with apostrophe), to which CPU can switch
(and then switch back) using \INS{EXX} instruction.

\subsubsection{Size of buffer}

When a programmer needs to declare the size of some buffer, values in form of $2^x$ are usually used (512 bytes, 1024, etc.).
Values in $2^x$ form are easily recognizable (\myref{2n_numbers_table}) in decimal, hexadecimal and binary base.

But needless to say, programmers are still humans with their decimal culture.
And somehow, in \ac{DBMS} area, size of textual database fields is often chosen as $10^x$ number, like 100, 200.
They just think \q{Okay, 100 is enough, wait, 200 will be better}.
And they are right, of course.

Maximum width of \emph{VARCHAR2} data type in \oracle is 4000 characters, not 4096.

There is nothing wrong with this, this is just a place where numbers like $10^x$ can be encountered.

\subsubsection{Address}

It's always a good idea to keep in mind an approximate memory map of the process you currently debug.
For example, many win32 executables started at 0x00401000, so an address like 0x00451230 is probably located inside
executable section. You'll see addresses like these in the \TT{EIP} register.

Stack is usually located somewhere below. % TODO

Many debuggers are able to show the memory map of the debuggee, for example: \myref{olly_memory_map_example}.

If a value is increasing by step 4 on 32-bit architecture or by step 8 on 64-bit one,
this probably sliding address of some elements of array.

It's important to know that win32 doesn't use addresses below 0x10000, so if you see some number below this constant,
this cannot be an address (see also: \url{https://msdn.microsoft.com/en-us/library/ms810627.aspx}).

Anyway, many debuggers can show you if the value in a register can be an address to something.
OllyDbg can also show an ASCII string if the value is an address of it.

\subsubsection{Bit field}

If you see a value where one (or more) bit(s) are flipping from time to time like 0xABCD1234 $\rightarrow$ 0xABCD1434 and back,
this is probably a bit field (or bitmap).

\subsubsection{Packed bytes}

\myindex{\CStandardLibrary!strcmp()}
\myindex{\CStandardLibrary!memcmp()}
When \emph{strcmp()} or \emph{memcmp()} copies a buffer, it loads/stores 4 (or 8) bytes simultaneously,
so if a string containing \q{4321}, and it would be copied to another place,
at one point you'll see 0x31323334 value in some register.
This is 4 packed bytes into a 32-bit value.

}
\RU{% TODO rework structure and hierarchy
\mysection{Целочисленные типы данных}

Целочисленный тип данных (\emph{integral}) это тип для значения, которое может быть сконвертировано в число.
Это числа, перечисления (\emph{enumerations}), булевые типы.

\subsection{Бит}

Очевидное использования бит это булевые значения: 0 для \emph{ложно/false} и 1 для \emph{true/истинно}.

Набор булевых значений можно упаковать в \glslink{word}{слово}: в 32-битном слове будет 32 булевых значения, итд.
Этот метод также называется \emph{bitmap} или \emph{bitfield}.

Но есть очевидные накладки: тасовка бит, изоляция оных, итд.
В то время как использование \glslink{word}{слова} (или типа \emph{int}) для булевого значения это не экономично,
но очень эффективно.

В среде \CCpp, 0 это \emph{false/ложно} и любое ненулевое значение это \emph{true/истинно}.
Например:

\lstinputlisting[style=customc]{fundamentals/data_types_and_numbers_RU_lst1.c}

Это популярный способ перечислить все символы в Си-строке:

\lstinputlisting[style=customc]{fundamentals/data_types_and_numbers_RU_lst2.c}

\subsection{Ниббл AKA nibble AKA nybble}

\ac{AKA} полубайт, тетрада.
Равняется 4-м битам.

Все эти термины в ходу и сегодня.

\subsubsection{Двоично-десятичный код (\ac{BCD})}
\label{BCD}

\myindex{Intel 4004}

4-битные нибблы использовались в 4-битных процессорах, например, в легендарном Intel 4004 (который использовался в
калькуляторах).

Интересно знать, что числа там представлялись в виде \emph{binary-coded decimal} (\ac{BCD}).
Десятичный 0 кодировался как 0b0000, десятичная 9 как 0b1001, а остальные значения не использовались.
Десятичное 1234 представлялось как 0x1234.
Конечно, этот способ не очень экономичный.

Тем не менее, он имеет одно преимущество: очень легко конвертировать значения из десятичного в \ac{BCD}-запакованное и назад.
BCD-числа можно складывать, вычитать, итд, но нужна дополнительная корректировка.
В x86 CPUs для этого есть редкие инструкции:
\INS{AAA}/\INS{DAA} (adjust after addition: корректировка после сложения),
\INS{AAS}/\INS{DAS} (adjust after subtraction: корректировка после вычитания),
\INS{AAM} (after multiplication: после умножения),
\INS{AAD} (after division: после деления).

\myindex{x86!\Registers!AF}
Необходимость поддерживать \ac{BCD}-числа в CPU это причина, почему существуют флаги \emph{half-carry flag} (флаг полупереноса)
(в 8080/Z80) и
\emph{auxiliary flag} (вспомогательный флаг) (\TT{AF} в x86):
это флаг переноса, генерируемый после обработки младших 4-х бит. Флаг затем используется корректирующими инструкциями.

Тот факт, что числа легко конвертировать, привел к популярности этой книги:
\InSqBrackets{Peter Abel, \emph{IBM PC assembly language and programming} (1987)}.
Но кроме этой книги, автор этих заметок, никогда не видел \ac{BCD}-числа на практике, исключая
\emph{magic numbers} (\myref{magic_numbers}),
как, например, дата чьего-то дня рождения, закодированная как 0x19791011 --- это действительно запакованное
\ac{BCD}-число.

На удивление, автор нашел использование чисел закодированных в \ac{BCD} в ПО SAP: \url{https://yurichev.com/blog/SAP/}.
Некоторые числа, включая цены, кодируются в виде \ac{BCD} в базе.
Вероятно, они использовали это для совместимости с каким-то древним ПО или железом?

Инструкции для \ac{BCD} в x86 часто использовались для других целей, использовались их недокументированные особенности,
например:

\begin{lstlisting}[style=customasmx86]
	cmp al,10
	sbb al,69h
	das
\end{lstlisting}

Этот запутанный код конвертирует число в пределах 0..15 в \ac{ASCII}-символ '0'..'9', 'A'..'F'.

\myparagraph{Z80}
\myindex{Z80}

Z80 был клоном 8-битного Intel 8080 CPU, и из-за экономии места, он имеет 4-битный \ac{ALU}, т.е., каждая
операция над двумя 8-битными числами происходит за два шага.
Один из побочных эффектов в том, что легко генерировать \emph{half-carry flag} (флаг полупереноса).

\subsection{Байт}

Байт, в первую очередь, применяется для хранения символов.
8-битные байты не всегда были популярны, как сейчас.
Перфоленты для телетайпов имели 5 и 6 возможных дырок, это 5 или 6 бит на байт.

\myindex{octet}
\myindex{fetchmail}
Чтобы подчеркнуть тот факт, что в байте 8 бит, байт иногда называется \emph{октетом} (\emph{octet}):
по крайней мере \emph{fetchmail} использует эту терминологию.

9-битные байты существовали на 36-битных архитектурах: 4 9-битных байта помещались в одно \glslink{word}{слово}.
Вероятно из-за этого, стандарты \CCpp говорят что в \emph{char} должно быть \emph{как минимум} 8 бит, но может быть и больше.

Например, в ранней документации к языку Си\footnote{\url{https://yurichev.com/mirrors/C/bwk-tutor.html}}, можно найти такое:

\begin{lstlisting}
char  one byte character (PDP-11, IBM360: 8 bits; H6070: 9 bits) 
\end{lstlisting}

\myindex{Honeywell 6070}
Под H6070, вероятно, подразумевается Honeywell 6070, с 36-битными словами.

\subsubsection{Стандартная ASCII-таблица}

7-битная ASCII-таблица стандартная, которая содержит только 128 возможных символов.
Раннее ПО для передачи е-мейлов работало только с 7-битными ASCII-символами, так что понадобился стандарт \ac{MIME}
для кодирования сообщений в нелатинских системах письменности.
7-битные ASCII коды дополнялись битом чётности, давая в итоге 8 бит.

\emph{Data Encryption Standard} (\ac{DES}) имеет 56-битный ключ, это 8 7-битных байт,
оставляя место для бита чётности для каждого символа.

Заучивать на память всю таблицу \ac{ASCII} незачем, но можно запомнить интервалы.
\InSqBrackets{0..0x1F} это управляющие символы (непечатные).
\InSqBrackets{0x20..0x7E} это печатные.
Коды начиная с 0x80 обычно используются для нелатинских систем письменности и/или псевдографики.

Некоторые важные коды, которые легко запомнить:
0 (конец Си-строки, \TT{'\textbackslash{}0'} в C/C++);
0xA или 10 (\emph{line feed} (перевод строки), \TT{'\textbackslash{}n'} в C/C++);
0xD или 13 (\emph{carriage return} (возврат каретки), \TT{'\textbackslash{}r'} в C/C++).

0x20 (пробел) также часто запоминается.

\subsubsection{8-битные процессоры}

x86 имеют возможность работать с байтами на уровне регистров (потому что они наследники 8-битного процессора 8080),
а RISC как ARM и MIPS --- нет.

\subsection{Wide char}
\myindex{UTF-16}
\myindex{UCS-2}

Это попытка поддерживать многоязычную среду расширяя байт до 16-и бит.
Самый известный пример это ядро Windows NT и win32-функции с суффиксом \emph{W}.
Вот почему если закодировать обычный текст на английском,
то каждый латинский символ в текстовой строке будет перемежаться с нулевым байтом.
Эта кодировка также называется UCS-2 или UTF-16

Обычно, \emph{wchar\_t} это синоним 16-битного типа данных \emph{short}.

\subsection{Знаковые целочисленные и беззнаковые}

Некоторые люди могут удивляться, почему беззнаковые типы данных вообще существуют, т.к., любое беззнаковое число
можно представить как знаковое.
Да, но отсутствие бита знака в значении расширяет интервал в два раза.
Следовательно, знаковый байт имеет интервал -128..127, а беззнаковый: 0..255.
Еще одно преимущество беззнаковых типов данных это самодокументация:
вы определяете переменную, которая не может принимать отрицательные значения.

\myindex{Java}
Беззнаковые типы данных отсутствуют в Java, за что её критикуют.
Трудно реализовать криптографические алгоритмы используя булевы операции над знаковыми типами.

Значения вроде 0xFFFFFFFF (-1) часто используются, в основном, как коды ошибок.

\subsection{Слово (word)}

Слово \glslink{word}{слово} это неоднозначный термин, и обычно означает тип данных, помещающийся в \ac{GPR}.
Байты практичны для символов, но непрактичны для арифметических расчетов.

Так что, многие процессоры имеют \ac{GPR} шириной 16, 32 или 64 бит.
Даже 8-битные \ac{CPU} как 8080 и Z80 предлагают работать с парами 8-битными регистров, каждая пара формирует 16-битный
псевдорегистр
(\emph{BC}, \emph{DE}, \emph{HL}, итд.).
Z80 имеет некоторые возможности для работы с парами регистров, и это, в каком-то смысле, эмуляция 16-битного CPU.

В общем, если в рекламе CPU говорят о нем как о ``n-битном процессоре'', это обычно означает, что он имеет n-битные \ac{GPR}.

Было время, когда в рекламе жестких дисков и модулей \ac{RAM} писали, что они имеют \emph{n} килослов вместо
\emph{b} килобайт/мегабайт.

Например, \emph{Apollo Guidance Computer}
имел 2048 слов \ac{RAM}.
Это был 16-битный компьютер, так что там было 4096 байт \ac{RAM}.

\emph{TX-0} имел 64K 18-битных слов памяти на магнитных сердечниках,
т.е., 64 килослов.

\emph{DECSYSTEM-2060}
мог иметь вплоть до 4096 килослов \emph{твердотельной памяти}
(т.е., жесткие диски, ленты, итд).
Это был 36-битный компьютер, так что это 18432 килобайта или ~18 мегабайт.

В сущности, зачем нужны байты, если есть слова?
Если только для работы с текстовыми строками.
Почти во всех остальных случаях можно использовать слова.

\myhrule{}

\emph{int} в \CCpp почти всегда связан со \glslink{word}{словом}.
(Кроме архитектуры AMD64, где \emph{int} остался 32-битным, вероятно, ради лучшей обратной совместимости.)

\emph{int} 16-битный на PDP-11 и старых компьютерах с MS-DOS.
\emph{int} 32-битный на VAX, и на x86 начиная с 80386, итд.

И даже более того, если в программе на \CCpp{} определение типа для переменной отсутствует,
то по умолчанию подразумевается \emph{int}.
Вероятно, это наследие языка программирования B\footnote{\url{http://yurichev.com/blog/typeless/}}.

\myhrule{}

\ac{GPR} это обычно самый быстрый контейнер для переменной, быстрее чем запакованный бит, и иногда даже быстрее запакованного
байта (потому что нет нужны изоировать единственный бит/байт из \ac{GPR}).
Даже если вы используете его как контейнер для счетчика в цикле, в интервале 0..99.

\myhrule{}

В языке ассемблера, \gls{word} всё еще 16-битный для x86, потому что так было во времена 16-битного 8086.
\emph{Double word} 32-битный, \emph{quad word} 64-битный.
Вот почему 16-битные слова определяются при помощи \TT{DW} в ассемблере на x86, для 32-битных используется \TT{DD}
и для 64-битных --- \TT{DQ}.

\Gls{word} 32-битный для ARM, MIPS, итд, 16-битные типы данных называются здесь \emph{half-word} (полуслово).
Следовательно, \emph{double word} на 32-битном RISC это 64-битный тип данных.

В \emph{GDB} такая терминология: \emph{halfword} для 16-битных, \gls{word} для 32-битных и \emph{giant word} для 64-битных.

В 16-битной среде \CCpp{} на PDP-11 и MS-DOS был тип \emph{long} шириной в 32 бита, вероятно, они имели ввиду
\emph{long word} или \emph{long int}?

В 32-битных средах \CCpp{} имеется тип \emph{long long} для типов данных шириной 64 бита.

Теперь вы видите, почему термин \emph{слово} такой неоднозначный.

\subsubsection{Нужно ли использовать \emph{int}?}

Некоторые люди говорят о том, что тип \emph{int} лучше не использовать вообще, потому что его неоднозначность приводит
к ошибкам.
Например, хорошо известная библиотека \emph{lzhuf} использует тип \emph{int} в одном месте, и всё работает нормально на 16-битной
архитектуре.
Но если она портируется на архитектуру с 32-битным \emph{int}, она может падать: \url{http://yurichev.com/blog/lzhuf/}.

Более однозначные типы определены в файле \emph{stdint.h}:
\emph{uint8\_t}, \emph{uint16\_t}, \emph{uint32\_t}, \emph{uint64\_t}, итд.

\myindex{Дональд Э. Кнут}
Некоторые люди, как Дональд Э. Кнут, предлагают\footnote{\url{http://www-cs-faculty.stanford.edu/~uno/news98.html}}
более звучные слова для этих типов:\\
\emph{byte/wyde/tetrabyte/tetra/octabyte/octa}.
Но эти имена менее популярны чем ясные термины с включением символа \emph{u} (\emph{unsigned})
и числом прямо в названии типа.

\subsubsection{Компьютеры ориентированные на слово}

Не смотря на неоднозначность термина \glslink{word}{слово}, современные компьютеры всё еще ориентированы на слово:
\ac{RAM} и все уровни кэш-памяти организованы по словам а не байтам.
Впрочем, в рекламе пишут о размере именно в байтах.
% <!-- TODO word length on intel, etc... -->

Доступ по адресу в памяти и кэш-памяти выровненный по границе слова зачастую быстрее, чем невыровненный.

При разработке структур данных, от которых ждут скорости и эффективности, всегда нужно учитывать длину \glslink{word}{слова}
CPU, на котором это будет исполняться.
Иногда компилятор делает это за программиста, иногда нет.

\subsection{Регистр адреса}

Для тех, кто был воспитан на 32-битных и/или 64-битных x86, и/или RISC 90-х годов, как ARM, MIPS, PowerPC, считается
обычным, что шина адреса имеет такую же ширину как \ac{GPR} или \glslink{word}{слово}.
Тем не менее, на других архитектурах, ширина шины адреса может быть другой.

8-битный Z80 может адресовать $2^{16}$ байт, используя пары 8-битных регистров, или специальные регистры (\emph{IX}, \emph{IY}).
Регистры \emph{SP} и \emph{PC} также 16-битные.

\myindex{Cray}
Суперкомпьютер Cray-1 имел 64-битные GPR, но 24-битные регистры для адресов, так что он мог адресовать
$2^{24}$ (16 мегаслов или 128 мегабайт).
Память в 1970-ые была очень дорогой, и типичный Cray-1 имел 1048576 (0x100000) слов ОЗУ или ~8MB.
Тогда зачем выделять целый 64-битный регистр для адреса или указателя?

Процессоры 8086/8088 имели крайне странную схему адресации:
значения двух 16-битных регистров суммировались в очень странной манере, производя 20-битный адрес.
Вероятно, то было что-то вроде игрушечной виртуализации (\myref{8086_memory_model})?
8086 мог исполнять несколько программ (хотя и не одновременно).

\myindex{ARM!ARM1}
Ранний ARM1 имеет интересный артефакт:

\begin{framed}
\begin{quotation}
Another interesting thing about the register file is the PC register is missing a few bits. Since the ARM1 uses 26-bit addresses, the top 6 bits are not used. Because all instructions are aligned on a 32-bit boundary, the bottom two address bits in the PC are always zero. These 8 bits are not only unused, they are omitted from the chip entirely.
\end{quotation}
\end{framed}

( \url{http://www.righto.com/2015/12/reverse-engineering-arm1-ancestor-of.html} )

Так что, значение где в двух младших битах что-то есть, невозможно записать в регистр PC просто физически.
Также невозможно установить любой бит в старших 6 битах PC.

Архитектура x86-64 имеет 64-битные виртуальные указателя/адреса, но внутри адресная шина 48-битная
(этого достаточно для адресации 256TB памяти).

\subsection{Числа}

Для чего используются числа?

Когда вы видите как некое число/числа меняются в регистре процесса, вы можете заинтересоваться, что это число значит.
Это довольно важное качество реверс-инжинира, определять возможный тип данных по набору изменяемых чисел.

\subsubsection{Булевы значения}

Если число меняется от 0 до 1 и назад, скорее всего, это значение имеет булевый тип данных.

\subsubsection{Счетчик циклов, индекс массива}

Переменная увеличивающаяся с 0, как: 0, 1, 2, 3\dots --- большая вероятность что это счетчик цикла и/или индекс массива.

\subsubsection{Знаковые числа}

Если вы видите переменную, которая содержит очень маленькие числа, и иногда очень большие,
как 0, 1, 2, 3, и 0xFFFFFFFF, 0xFFFFFFFE, 0xFFFFFFFD,
есть шанс что это знаковая переменная в виде \emph{дополнительного кода} (\myref{sec:signednumbers}),
и последние три числа это -1, -2, -3.

\subsubsection{32-битные числа}

Существуют настолько большие числа,
что для них даже существует специальная нотация (Knuth's up-arrow notation).
Эти числа настолько большие, что им нет практического применения в инженерии, науке и математике.

Почти всем инженерам и ученым зачастую достаточно чисел в формате IEEE 754 в двойной точности,
где максимальное значение близко к $1.8 \cdot 10^{308}$.
(Для сравнения, количество атомов в наблюдаемой Вселенной оценивается от $4 \cdot 10^{79}$ до $4 \cdot 10^{81}$.)

А в практическом программировании, верхний предел значительно ниже.
Так было в эпоху MS-DOS: 16-битные \emph{int} использовались почти везде (индексы массивов, счетчики циклов),
в то время как 32-битные \emph{long} использовались редко.

Во время появления x86-64, было решено оставить тип \emph{int} 32-битным, вероятно, потому что 
необходимость использования 64-битного \emph{int} еще меньше.

Я бы сказал, что 16-битные числа в интервале 0..65535, вероятно, наиболее используемые числа в программировании вообще.

Учитывая всё это, если вы видите необычно большое 32-битное значение вроде 0x87654321, большая вероятность,
что это может быть:

\begin{itemize}

\item это всё еще может быть 16-битное число, но знаковое, между 0xFFFF8000 (-32768) и 0xFFFFFFFF (-1).
% TODO: [Example](https://github.com/DennisYurichev/random_notes/blob/master/timedate.md).
\item адрес ячейки памяти (можно проверить используя в карте памяти в отладчике);
\item запакованные байты (можно проверить визуально);
\item битовые флаги;
\item что-то связанное с (любительской) криптографией;
\item \emph{магическое число} (\myref{magic_numbers});
\item число с плавающей точкой в формате IEEE 754 (тоже легко проверить).

\end{itemize}

Та же история и для 64-битных значений.

\myparagraph{\dots так что, 16-битного \emph{int} достаточно почти для всего?}

Интересно заметить: в \InSqBrackets{\MAbrash{} глава 13}
мы можем найти множество случаев, когда 16-битных переменных просто достаточно.
В то же время, Майкл Абраш жалеет о том что в процессорах 80386 и 80486 маловато доступных регистров,
так что он предлагает хранить два 16-битных значения в одном 32-битном регистре и затем прокручивать его 
используя инструкцию \INS{ROR reg, 16} (на 80386 и позже) (\INS{ROL reg, 16} тоже будет работать) или
\INS{BSWAP} (на 80486 и позже).

Это нам напоминает как в Z80 был набор альтернативных регистров (с апострофом в конце), на которые CPU мог переключаться
(и затем переключаться назад) используя инструкцию \INS{EXX}.

\subsubsection{Размер буфера}

Когда программисту нужно обознать размер некоторого буфера, обычно используются значения вида $2^x$ (512 байт, 1024, итд.).
Значения вида $2^x$ легко опознать (\myref{2n_numbers_table}) в десятичной, шестнадцатеричной и двоичной системе.

Но надо сказать что программисты также и люди со своей десятичной культурой.
И иногда, в среде \ac{DBMS}, размер текстовых полей в БД часто выбирается в виде числа $10^x$, как 100, 200.
Они думают что-то вроде \q{Окей, 100 достаточно, погодите, лучше пусть будет 200}.
И они правы, конечно.

Максимальный размер типа данных \emph{VARCHAR2} в \oracle это 4000 символов, а не 4096.

В этом нет ничего плохого, это просто еще одно место, где можно встретить числа вида $10^x$.

\subsubsection{Адрес}

Всегда хорошая идея это держать в памяти примерную карту памяти процесса, который вы отлаживаете.
Например, многие исполняемые файлы в win32 начинаются с 0x00401000, так что адрес вроде 0x00451230 скорее всего находится в секции с исполняемым кодом.
Вы увидите адреса вроде этих в регистре \TT{EIP}.

Стек обычно расположен где-то ниже. % TODO

Многие отладчики могут показывать карту памяти отлаживаемого процесса, например: \myref{olly_memory_map_example}.

Если значение увеличивается с шагом 4 на 32-битной архитектуре или с шагом 8 на 64-битной,
это вероятно сдвигающийся адрес некоторых элементов массива.

Важно знать что win32 не использует адреса ниже 0x10000, так что если вы видите какое-то число ниже этой константы,
это не может быть адресом (см.также: \url{https://msdn.microsoft.com/en-us/library/ms810627.aspx}).

Так или иначе, многие отладчики могут показывать, является ли значение в регистре адресом чего-либо.
OllyDbg также может показывать ASCII-строку, если значение является её адресом.

\subsubsection{Битовые поля}

Если вы видите как в значении один (или больше) бит меняются от времени к времени, как 0xABCD1234 $\rightarrow$ 0xABCD1434 и назад,
это вероятно битовое поле (или \emph{bitmap}).

\subsubsection{Запакованные байты}

\myindex{\CStandardLibrary!strcmp()}
\myindex{\CStandardLibrary!memcmp()}
Когда \emph{strcmp()} или \emph{memcmp()} копирует буфер, они загружают/записывают 4 (или 8) байт одновременно,
так что если строка содержит \q{4321} и будет скопирована в другое место,
в какой-то момент вы увидите значение 0x31323334 в каком-либо регистре.
Это 4 запакованных байта в одном 32-битном значении.

}
\FR{% TODO rework structure and hierarchy
\mysection{Types intégraux}

Un type intégral est un type de données dont les valeurs peuvent être converties en nombres.
Les types intégraux comportent les nombres, les énumérations et les booléens.

\subsection{Bit}

Les valeurs booléennes sont une utilisation évidente des bits: 0 pour \emph{faux} et 1 pour \emph{vrai}.

Plusieurs valeurs booléennes peuvent être regroupées en un \glslink{word}{mot}: Un mot de 32 bits contiendra 32 valeur booléennes, etc.
On appelle \emph{bitmap} ou \emph{bitfield} un tel assemblage.

Cette approche engendre un surcoût de traitement: décalages, extraction, etc.
A l'inverse l'utilisation d'un \glslink{word}{mot} (ou d'un type \emph{int}) pour chaque booléen gaspille de l'espace, au profit des performances.

Dans les environnements C/C++, la valeur 0 représente \emph{faux} et toutes les autres valeurs \emph{vrai}.
Par exemple:

\lstinputlisting[style=customc]{fundamentals/data_types_and_numbers_FR_lst1.c}

Une manière courante d'énumérer les caractères d'une chaîne en langage C:

\lstinputlisting[style=customc]{fundamentals/data_types_and_numbers_FR_lst2.c}

\subsection{Nibble}

\ac{AKA} demi-octet, tétrade.
Représente 4 bits.

Toutes ces expressions sont toujours en usage.

\subsubsection{Binary-Coded decimal (\ac{BCD})}
\label{BCD}

\myindex{Intel 4004}

Les demi-octets ont été utilisés par des CPU 4-bits tel que le Intel 4004 (utilisé dans les calculatrices).

On notera que la représentation \emph{binary-coded decimal} (\ac{BCD}) a été utilisée pour représenter les nombres sur 4 bits.
L'entier 0 est représenté par la valeur 0b0000, l'entier 9 par 0b1001 tandis que les valeurs supérieures ne sont pas utilisées.
La valeur décimale 1234 est ainsi représentée par 0x1234.
Il est évident que cette représentation n'est pas la plus efficace en matière d'espace.

Elle possède en revanche un avantage: la conversion des nombres depuis et vers le format \ac{BCD} est extrêmement simple.
Les nombres au format BCD peuvent être additionnés, soustraits, etc., au prix d'une opération supplémentaire de gestion des demi-retenues.
Les CPUs x86 proposent pour cela quelques instructions assez rares:
\INS{AAA}/\INS{DAA} (gestion de la demi-retenue après addition),
\INS{AAS}/\INS{DAS} (gestion de la demi-retenue après soustraction),
\INS{AAM} (après multiplication),
\INS{AAD} (après division).

\myindex{x86!\Registers!AF}
Le support par les CPUs des nombres au format \ac{BCD} est la raison d'être des \emph{half-carry flag} (sur 8080/Z80) et
\emph{auxiliary flag} (\TT{AF} sur x86).
Ils représentent la retenue générée après traitement des 4 bits de poids faible (d'un octet).
Le drapeau est utilisé par les instructions de gestion de retenue ci-dessus.

Le livre \InSqBrackets{Peter Abel, \emph{IBM PC assembly language and programming} (1987)} doit sa popularité à la facilité de ces conversions.
Hormis ce livre, l'auteur de ces notes n'a jamais rencontré en pratique de nombres au format \ac{BCD}, sauf dans certains
\emph{nombres magiques} (\myref{magic_numbers}),
tels que lorsque la date de naissance d'un individu est encodé sous la forme 0x19791011---qui n'est autre qu'un nombre au format \ac{BCD}.

Étonnement, j'ai trouvé que des nombres encodés \ac{BCD} sont utilisés dans le logiciel
SAP: \url{https://yurichev.com/blog/SAP/}.
Certains nombres, prix inclus, sont encodés en format \ac{BCD} dans la base de données.
Peut-être ont-ils utilisé ce format pour être compatible avec d'anciens logiciels
ou matériel?

Les instructions x86 destinées au traitement des nombres \ac{BCD} ont parfois été utilisées à d'autres fins, le plus souvent non documentées, par exemple:

\begin{lstlisting}[style=customasmx86]
	cmp al,10
	sbb al,69h
	das
\end{lstlisting}

Ce fragment de code abscons converties les nombres de 0 à 15 en caratères \ac{ASCII} '0'..'9', 'A'..'F'.

\myparagraph{Z80}
\myindex{Z80}

Le processeur Z80 était un clone de la CPU 8 bits 8080 d'Intel. Par manque de place, il utilisait une \ac{ALU} de 4 bits.
Chaque opération impliquant deux nombres de 8 bits devait être traitée en deux étapes.
Il en a découlé une utilisation naturelle des \emph{half-carry flag}.

\subsection{Caractère}

A l'heure actuelle, l'utilisation de 8 bits par caractère est pratique courante.
Il n'en a pas toujours été ainsi.
Les cartes perforées utilisées pour les télétypes ne pouvaient comporter que 5 ou 6 perforations par caractères, et donc autant de bits.

\myindex{octet}
\myindex{fetchmail}
Le terme \emph{octet} met l'accent sur l'utilisation de 8 bits.:
\emph{fetchmail} est un de ceux qui utilise cette terminologie.

Sur les architectures à 36 bits, l'utilisation de 9 bits par caractère a été utilisée: un \glslink{word}{mot} pouvait contenir 4 caractères.
Ceci explique peut-être que le standard C/C++ indique que le type \emph{char} doit supporter \emph{au moins} 8 bits, mais que l'utilisation
d'un nombre plus importants de bits est autorisé.

Par exemple, dans l'un des premiers ouvrage sur le langage C \footnote{\url{https://yurichev.com/mirrors/C/bwk-tutor.html}}, nous trouvons :

\begin{lstlisting}
char  one byte character (PDP-11, IBM360: 8 bits; H6070: 9 bits)
\end{lstlisting}

\myindex{Honeywell 6070}
H6070 signifie probablement Honeywell 6070, qui comprenait des mots de 36 bits.

\subsubsection{table ASCII standard}

La représentation ASCII des caractères sur 7 bits constitue un standard, qui supporte donc 128 caractères différents.
Les premiers logiciels de transport de mails fonctionnaient avec des codes ASCII sur 7 bits.
Le standard \ac{MIME} nécessitait donc l'encodage des messages rédigés avec des alphabets non latins.
Le code ASCII sur 7 bits a ensuite été augmenté d'un bit de parité qui a aboutit à la représentation sur 8 bits.

Les clefs de chiffrage utilisées par \emph{Data Encryption Standard} (\ac{DES}) comportent 56 bits, soit 8 groupes de 7 bits
ce qui laisse un espace pour un bit de parité dans chaque groupe.

La mémorisation de la table \ac{ASCII} est inutile. Il suffit de se souvenir de certains intervalles.
\InSqBrackets{0..0x1F} sont les caractères de contrôle (non imprimables).
\InSqBrackets{0x20..0x7E} sont les caractères imprimables.
Les codes à partir de la valeur 0x80 sont généralement utilisés pour les caractères non latins et pour certains caractères pseudo graphiques.

Quelques valeurs typiques à mémoriser sont :
0 (terminateur d'une chaîne de caractères en C, \TT{'\textbackslash{}0'} et C/C++);
0xA ou 10 (\emph{fin de ligne}, \TT{'\textbackslash{}n'} en C/C++);
0xD ou 13 (\emph{retour chariot}, \TT{'\textbackslash{}r'} en C/C++).

0x20 (espace).

\subsubsection{CPUs 8 bits}

Les processeurs x86 - descendants des CPUs 8080 8 bits - supportent la manipulation d'octet(s) au sein des registres.
Les CPUs d'architecture RISC telles que les processeurs ARM et MIPS n'offrent pas cette possibilité.

\subsection{Alphabet élargi}
\myindex{UTF-16}
\myindex{UCS-2}

Il s'agit d'une tentative de supporter des langues non européennes en étendant le stockage d'un caractère à 16 bits.
L'exemple le plus connu en est le noyau Windows NT et les fonctions win32 suffixées d'un \emph{W}.
Cet encodage est nommé UCS-2 ou UTF-16.
Son utilisation explique la présence d'octets à zéro entre chaque caractère d'un texte en anglais ne comportant que des caractères latins.

En règle général, la notation \emph{wchar\_t} est un synonyme du type \emph{short} qui utilise 16 bits.

\subsection{Entier signé ou non signé}

Certains s'étonneront qu'il existe un type de données entier non signé (positif ou nul) puisque chaque entier de ce type peut être représenté par un entier signé (positif ou négatif).
Certes, mais le fait de ne pas avoir à utiliser un bit pour représenter le signe permet de doubler la taille de l'intervalle des valeurs qu'il est possible de représenter.
Ainsi un octet signé permet de représenter les valeurs de -128 à +127, et l'octet non signé les valeurs de 0 à 255.
Un autre avantage d'utiliser un type de données non signé est l'auto-documentation:
vous définissez une variable qui ne peut pas recevoir de valeurs négatives.

\myindex{Java}
L'absence de type de données non signées dans le langage Java a été critiqué.
L'implémentation d'algorithmes cryptographiques à base d'opérations booléennes avec les seuls types de données signées est compliquée.

Une valeur telle que 0xFFFFFFFF (-1) est souvent utilisée, en particulier pour représenter un code d'erreur.

\subsection{Mot}

\glslink{word}{mot} Le terme de 'mot' est quelque peu ambigu et dénote en général un type de données dont la taille correspond à celle d'un \ac{GPR}.
L'utilisation d'octets est pratique pour le stockage des caractères, mais souvent inadapté aux calculs arithmétiques.

C'est pourquoi, nombre de \ac{CPU}s possèdent des \ac{GPR}s dont la taille est de 16, 32 ou 64 bits.
Les CPUs 8 bits tels que le 8080 et le Z80 proposent quant à eux de travailler sur des paires de registres 8 bits, dont chacune constitue un \emph{pseudo-registre} de 16 bits.
(\emph{BC}, \emph{DE}, \emph{HL}, etc.).
Les capacités des paires de registres du Z80 en font, en quelque sorte, un émulateur d'une CPU 16 bits.

En règle générale, un CPU présenté comme ''CPU n-bits'' possède des \ac{GPR}s dont la taille est de n bits.

À une certaine époque, les disques durs et les barettes de \ac{RAM} étaient caractérisés comme ayant \emph{n} kilo-mots
et non pas \emph{b} kilooctets/megaoctets.

Par exemple, \emph{Apollo Guidance Computer}\footnote{\url{https://en.wikipedia.org/wiki/Apollo_Guidance_Computer}}
possède 2048 mots de \ac{RAM}.
S'agissant d'un ordinateur 16 bits, il y avait donc 4096 octets de \ac{RAM}.

La mémoire magnétique du \emph{TX-0}\footnote{\url{https://en.wikipedia.org/wiki/TX-0}} était de 64K mots de 18 bits,
i.e., 64 kilo-mots.

\emph{DECSYSTEM-2060}\footnote{\url{https://en.wikipedia.org/wiki/DECSYSTEM-20}}
pouvait supporter jusqu'à 4096 kilo mots de \emph{solid state memory}
(i.e., hard disks, tapes, etc).
S'agissant d'un ordinateur 36 bits, cela représentait 18432 kilo octets ou ~18 mega octets.

En fait, pourquoi auriez-vous besoin d'octets si vous avez des mots?
Surtout pour le traitement des chaînes de texte.
Les mots peuvent être utilisés dans presque toutes les autres situations.

\myhrule{}

\emph{int} en C/C++ est presque systématiquement représenté par un \glslink{word}{mot}.
(L'architecture AMD64 fait exception car le type \emph{int} possède une taille de 32 bits, peut-être pour une meilleure portabilité.)

Le type \emph{int} est représenté sur 16 bits par le PDP-11 et les anciens compilateurs MS-DOS.
Le type \emph{int} est représenté sur 32 bits sur VAX, ainsi que sur l'architecture x86 à partir du 80386, etc.

De plus, dans les programmes C/C++, le type \emph{int} est utilisé par défaut lorsque le type d'une variable n'est pas explicitement déclaré.
Cette pratique peut apparaître comme un héritage du langage de programmation B\footnote{\url{http://yurichev.com/blog/typeless/}}.

\myhrule{}

L'accès le plus rapide à une variable s'effectue lorsqu'elle est contenue dans un \ac{GPR}, plus même qu'un ensemble de bits,
et parfois même plus rapide qu'un octet (puisqu'il n'est pas besoin d'isoler un bit ou un octet au sein d'un \ac{GPR}).
Ceci reste vrai même lorsque le registre est utilisé comme compteur d'itération d'une boucle de 0 à 99.

\myhrule{}

En langage assembleur x86, un \glslink{word}{mot} représente 16 bits, car il en était ainsi sur les processeurs 8086 16 bits.
Un \emph{Double word} représente 32 bits, et un \emph{quad word} 64 bits.
C'est pourquoi, les mots de 16 bits sont déclarés par \TT{DW} en assembleur x86, ceux de 32 bits par \TT{DD} et ceux de 64 bits par \TT{DQ}.

Dans les architectures ARM, MIPS, etc... un \glslink{word}{mot} représente 32 bits, on parlera alors de \emph{demi-mot} pour les types sur 16 bits.
En conséquence, un \emph{double word} sur une architecture RISC 32 bits est un type de données qui représente 64 bits.

\emph{GDB} utilise la terminologie suivante : \emph{demi-mot} pour 16 bits, \glslink{word}{mot} pour 32 bits et \emph{mot géant} pour 64 bits.

Les environnements C/C++ 16 bits sur PDP-11 et MS-DOS définissent le type \emph{long} comme ayant une taille de 32 bits,
ce qui serait sans doute une abréviation de \emph{long word} ou de \emph{long int}.

Les environnements C/C++ 32 bits définissent le type \emph{long long} dont la taille est de 64 bits.

L'ambiguïté du terme \emph{mot} est donc désormais évidente.

\subsubsection{Dois-je utiliser le type \emph{int}?}

Certains affirment que le type \emph{int} ne doit jamais être utilisé, l'ambiguïté de sa définition pouvant être génératrice de bugs.
A une certaine époque, la bibliothèque bien connue \emph{lzhuf} utilisais le type \emph{int} et fonctionnait parfaitement sur les architectures 16 bits.
Portée sur une architecture pour laquelle le type \emph{int} représentait 32 bits, elle pouvait alors crasher: \url{http://yurichev.com/blog/lzhuf/}.

Des types de données moins ambigus sont définis dans le fichier \emph{stdint.h}:
\emph{uint8\_t}, \emph{uint16\_t}, \emph{uint32\_t}, \emph{uint64\_t}, etc.

\myindex{Donald E. Knuth}
Donald E. Knuth fut l'un de ceux qui proposa\footnote{\url{http://www-cs-faculty.stanford.edu/~uno/news98.html}}
d'utiliser pour ces différents types des dénominations aux consonances distinctes: \emph{octet/wyde/tetrabyte/octabyte}.
Cette pratique est cependant moins courante que celle consistant à inclure directement dans le nom du type
les termes \emph{u} (\emph{unsigned}) ainsi que le nombre de bits.

\subsubsection{Ordinateurs à base de mots}

En dépit de l'ambiguïté du terme \glslink{word}{mot}, les ordinateurs modernes restent conçus sur ce concept: la \ac{RAM} ainsi que tous
les niveaux de mémoire cache demeurent organisés en mots et non pas en octets.
La notion d'octet reste prépondérante en marketing.
% <!-- TODO word length on intel, etc... -->

Les accès aux adresses mémoire et cache alignées sur des frontières de mots est souvent plus performante que lorsque l'adresse n'est pas alignée.

Afin de rendre performante l'utilisation des structures de données, il convient toujours de de prendre en compte la longueur du
\glslink{word}{mot} du CPU sur lequel sera exécuté le programme lors de la définition des structures de données.
Certains compilateurs - mais pas tous - prennent en charge cet alignement.

\subsection{Registre d'adresse}

Ceux qui ont fait leur premières armes sur les processeurs x86 32 et 64 bits, ou les processeurs RISC des années 90
tels que ARM, MIPS ou PowerPC prennent pour acquis que la taille du bus d'adresse est la même que celle d'un \ac{GPR}
ou d'un \glslink{word}{mot}.
Cependant, cette règle n'est pas toujours respectée sur d'autres architectures.

Le processeur 8 bits Z80 peut adresser $2^{16}$ octets, en utilisant une paire de registres 8 bits ou certains registres
spécialisés (\emph{IX}, \emph{IY}). En outre sur ce processeur les registres \emph{SP} et \emph{PC} contiennent 16 bits.

\myindex{Cray}
Le super calculateur Cray-1 possèdent des registres généraux de 64-bit, et des registres d'adressage de 24 bits.
Il peut donc adresser $2^{24}$ octets, soit (16 mega mots ou 128 mega octets).
La RAM coûtait très cher, et un Cray typique avait 1048576 (0x100000) mots de RAM, soit ~8MB.
Dans les années 70, la RAM était très coûteuse. Il paraissait alors inconcevable qu'un tel calculateur atteigne
les 128 Mo. Dès lors pourquoi aurait-on utilisé des registres 64 bits pour l'adressage?

Les processeurs 8086/8088 utilisent un schéma d'adressage particulièrement bizarre:
Les valeurs de deux registres de 16 bits sont additionnées de manière étrange afin d'obtenir une adresse sur 20 bits.
S'agirait-il d'une sorte de virtualisation gadget (\myref{8086_memory_model})?
Les processeurs 8086 pouvaient en effet faire fonctionner plusieurs programmes côte à côte (mais pas simultanément bien sûr).

\myindex{ARM!ARM1}
Les premiers processeurs ARM1 implémentent un artefact intéressant:

\begin{framed}
\begin{quotation}
Un autre point intéressant est l'absence de quelques bits dans le registre PC. Le processeur ARM1 utilisant des adresses
sur 26 bits, les 6 bits de poids fort ne sont pas utilisés. Comme toutes les adresses sont alignées sur une frontière de
32 bits, les deux bits les moins significatifs du registre PC sont toujours égaux à 0. Ces 8 bits sont non seulement
inutilisés mais purement et simplement absents du processeur.
\end{quotation}
\end{framed}

( \url{http://www.righto.com/2015/12/reverse-engineering-arm1-ancestor-of.html} )

En conséquence, il n'est pas possible d'affecter au registre PC une valeur dont l'un des deux bits de poids faible est
différent de 0, pas plus qu'il n'est possible de positionner à 1 l'un des 6 bits de poids fort.

L'architecture x86-64 utilise des pointeurs et des adresses sur 64 bits, cependant en interne la largeur du bus
d'adresse est de 48 bits, (ce qui est suffisant pour adresser 256 Tera octets de \ac{RAM}).

\subsection{Nombres}

A quoi sont utilisés les nombres ?

Lorsque vous constatez que la valeur d'un registre de la CPU est modifié selon un certain motif, vous pouvez
chercher à comprendre à quoi correspond ce motif.
La capacité à déterminer le type de données qui découle de ce motif est une compétence précieuse pour le reverse engineer .

\subsubsection{Booléen}

Si le nombre alterne entre les valeurs 0 et 1, il y a des chances importantes pour qu'il s'agisse d'une valeur booléenne.

\subsubsection{Compteur de boucle, index dans un tableau}

Une variable dont la valeur augmente régulièrement en partant de 0, tel que 0, 1, 2, 3\dots--- est probablement un
compteur de boucle et/ou un index dans un tableau.

\subsubsection{Nombres signés}

Si vous constatez qu'une variable contient parfois des nombres très petits et d'autre fois des nombres très grands,
tels que 0, 1, 2, 3, et 0xFFFFFFFF, 0xFFFFFFFE, 0xFFFFFFFD, il est probable qu'il s'agisse d'un entier signé sous
forme de \emph{two's complement} (\myref{sec:signednumbers}) auquel cas les 3 dernières valeurs représentent en réalité
-1, -2, -3.

\subsubsection{Nombres sur 32 bits}

Il existe des nombres tellement grands \footnote{\url{https://en.wikipedia.org/wiki/Large_numbers}},
qu'il existe une notation spéciale pour les représenter (Notation exponentielle de Knuth's
\footnote{\url{https://en.wikipedia.org/wiki/Knuth\%27s_up-arrow_notation}}).
De tels nombres sont tellement grands qu'ils s'avèrent peu pratiques pour l'ingénierie, les sciences
ou les mathématiques.

La plupart des ingénieurs et des scientifiques sont donc ravis d'utiliser la notation IEEE 754 pour les
nombres flottants à double précision, laquelle peut représenter des valeurs allant jusqu'à $1.8 \cdot 10^{308}$.
(En comparaison, le nombre d'atomes dans l'univers observable est estimé être entre $4 \cdot 10^{79}$ et $4 \cdot 10^{81}$.)

De fait, la limite supérieure des nombres utilisés dans les opérations concrètes est très très inférieure.

Pareil à l'époque de MS-DOS: les \emph{int} 16 bits étaient utilisés pratiquement pour tout (indice de tableau, compteur de boucle),
tandis que le type \emph{long} sur 32 bits ne l'était que rarement.

Durant l'avènement de l'architecture x86-64, il fut décidé que le type \emph{int} conserverait une taille de 32 bits,
probablement parce que l'utilisation d'un type \emph{int} de 64 bits est encore plus rare.

Je dirais que les nombre sur 16 bits qui couvrent l'intervalle 0..65535 sont probablement les nombres les plus
utilisés en informatique.

Ceci étant, si vous rencontrez des nombres sur 32 bits particulièrement élevé tels que 0x87654321, il existe une
bonne chance qu'il s'agisse :

\begin{itemize}

\item Il peut toujours s'agir d'un entier sur 16 bits, mais signé lorsque la valeur est entre 0xFFFF8000 (-32768) et 0xFFFFFFFF (-1).
% TODO: [Example](https://github.com/DennisYurichev/random_notes/blob/master/timedate.md).
\item une adresse mémoire (ce qui peut être vérifié en utilisant les fonctionnalités de gestion mémoire du débogueur).
\item des octets compactés (ce qui peut être vérifié visuellement).
\item un ensemble de drapeaux binaires.
\item de la cryptographie (amateur).
\item un nombre magique (\myref{magic_numbers}).
\item un nombre flottant utilisant la représentation IEEE 754 (également vérifiable).

\end{itemize}

Il en va à peu près de même pour les valeurs sur 64 bits.

\myparagraph{\dots donc un \emph{int} sur 16 bits est suffisant pour à peu près n'importe quoi?}

Il est intéressant de constater que dans \InSqBrackets{\MAbrash{} chapitre 13}
nous pouvons lire qu'il existe pléthore de cas pour lesquels des variables sur 16 bits sont largement suffisantes.
Dans le même temps, Michael Abrash se plaint que les CPUs 80386 et 80486 disposent de si peu de registres et propose donc de
placer deux registres de 16 bits dans un registre de 32 bits et d'en effectuer des rotations en utilisant les instructions
\INS{ROR reg, 16} (sur 80386 et suivant) (\INS{ROL reg, 16} fonctionne également) ou
\INS{BSWAP} (sur 80486 et suivant).

Cette approche rappelle celle du Z80 et de ses groupes de registres alternatifs (suffixés d'une apostrophe)
vers lesquels le CPU pouvait être basculé (et inversement) au moyen de l'instruction \INS{EXX}.

\subsubsection{Taille des buffers}

Lorssqu'un programmeur doit déclarer la taille d'un buffer, il utilise généralement une valeur de la
forme $2^x$ (512 octets, 1024, etc.).
Les valeurs de la forme $2^x$ sont faciles à reconnaître (\myref{2n_numbers_table}) en décimal, en hexadécimal
et en binaire.

Les programmeurs restent cependant des humains et conservent leur culture décimale.
C'est pourquoi, dans le domaine des \ac{DBMS}, la taille des champs textuels est souvent choisie sous la forme $10^x$,
100, 200 par exemple.
Ils pensent simplement \q{Okay, 100 suffira, attendez, 200 ira mieux}.
Et bien sûr, ils ont raison.

La taille maximale du type \emph{VARCHAR2} dans \oracle est de 4000 caractères et non de 4096.

Il n'y a rien à redire à ceci, ce n'est qu'un exemple d'utilisation des nombres sous la forme d'un
multiple d'une puissance de dix.

\subsubsection{Addresse}

Garder à l'esprit une cartographie approximative de l'occupation mémoire du processus que vous
déboguez est toujours une bonne idée.
Ainsi, beaucoup d'exécutables win32 démarrent à l'adresse 0x00401000, donc une adresse telle que
0x00451230 se situe probablement dans sa section exécutable. Vous trouverez des adresses de cette
sorte dans le registre \TT{EIP}.

La pile est généralement située à une adresse inférieure à % TODO

Beaucoup de débogueurs sont capables d'afficher la cartographie d'occupation mémoire du processus
débogué, par exemple: \myref{olly_memory_map_example}.

Une adresse qui augmente par pas de 4 sur une architecture 32-bit, ou par pas de 8 sur une
architecture 64-bit constitue probablement l'énumération des adresses des éléments d'un tableau.

Il convient de savoir que win32 n'utilise pas les adresses inférieures à 0x10000, donc si vous
observez un nombre inférieur à cette valeur, ce ne peut être une adresse (voir aussi
\url{https://msdn.microsoft.com/en-us/library/ms810627.aspx}).

De toute manière, beaucoup de débogueurs savent vous indiquer si la valeur contenue dans un
registre peut représenter l'adresse d'un élément. OllyDbg peut également vous afficher le
contenu d'une chaîne de caractères ASCII si la valeur du registre est l'adresse d'une telle
chaîne.

\subsubsection{Drapeaux}

Si vous observez une valeur pour laquelle un ou plusieurs bits changent de valeur de temps en
temps tel que 0xABCD1234 $\rightarrow$ 0xABCD1434 et retour, il s'agit probablement d'un ensemble
de drapeaux ou bitmap.

\subsubsection{Compactage de caractères}

\myindex{\CStandardLibrary!strcmp()}
\myindex{\CStandardLibrary!memcmp()}
Quand \emph{strcmp()} ou \emph{memcmp()} copient un buffer, ils traitent 4 (ou 8) octets à la fois.
Donc, si une chaîne de caractères \q{4321} est recopié à une autre adresse, il adviendra un
moment où vous observerez la valeur 0x31323334 dans un registre.
Il s'agit du bloc de 4 caractères traité comme un entier sur 32 bits.

}

\EN{\mysection{\SignedNumbersSectionName}
\label{sec:signednumbers}
\myindex{Signed numbers}

There are several methods for representing signed numbers,
but \q{two's complement} is the most popular one in computers.

Here is a table for some byte values:

\begin{center}
\begin{tabular}{ | l | l | l | l | }
\hline
\HeaderColor binary & \HeaderColor hexadecimal & \HeaderColor unsigned & \HeaderColor signed \\
\hline
01111111 & 0x7f & 127 & 127 \\
\hline
01111110 & 0x7e & 126 & 126 \\
\hline
\multicolumn{4}{ |c| }{...} \\
\hline
00000110 & 0x6 & 6 & 6 \\
\hline
00000101 & 0x5 & 5 & 5 \\
\hline
00000100 & 0x4 & 4 & 4 \\
\hline
00000011 & 0x3 & 3 & 3 \\
\hline
00000010 & 0x2 & 2 & 2 \\
\hline
00000001 & 0x1 & 1 & 1 \\
\hline
00000000 & 0x0 & 0 & 0 \\
\hline
11111111 & 0xff & 255 & -1 \\
\hline
11111110 & 0xfe & 254 & -2 \\
\hline
11111101 & 0xfd & 253 & -3 \\
\hline
11111100 & 0xfc & 252 & -4 \\
\hline
11111011 & 0xfb & 251 & -5 \\
\hline
11111010 & 0xfa & 250 & -6 \\
\hline
\multicolumn{4}{ |c| }{...} \\
\hline
10000010 & 0x82 & 130 & -126 \\
\hline
10000001 & 0x81 & 129 & -127 \\
\hline
10000000 & 0x80 & 128 & -128 \\
\hline
\end{tabular}
\end{center}

\myindex{x86!\Instructions!JA}
\myindex{x86!\Instructions!JB}
\myindex{x86!\Instructions!JL}
\myindex{x86!\Instructions!JG}
The difference between signed and unsigned numbers is that if we represent \TT{0xFFFFFFFE} and \TT{0x00000002} 
as unsigned, then the first number (4294967294) is bigger than the second one (2). 
If we represent them both as signed, the first one becomes $-2$, and it is smaller than the second (2). 
That is the reason why conditional jumps~(\myref{sec:Jcc}) are present both for signed (e.g. \JG, \JL) 
and unsigned (\INS{JA}, \JB) operations.

For the sake of simplicity, this is what one needs to know:

\begin{itemize}
\item Numbers can be signed or unsigned.

\item \CCpp signed types:

  \begin{itemize}
    \item \TT{int64\_t} (-9,223,372,036,854,775,808 .. 9,223,372,036,854,775,807)
	  (-~9.2..~9.2 quintillions) or \\
                \TT{0x8000000000000000..0x7FFFFFFFFFFFFFFF}),
    \item \Tint (-2,147,483,648..2,147,483,647 (-~2.15..~2.15Gb) or \\
	    \TT{0x80000000..0x7FFFFFFF}),
    \item \Tchar (-128..127 or \TT{0x80..0x7F}),
    \item \TT{ssize\_t}.
   \end{itemize}

	Unsigned:
  \begin{itemize}
	  \item \TT{uint64\_t} (0..18,446,744,073,709,551,615 
		  (~18 quintillions) or \TT{0..0xFFFFFFFFFFFFFFFF}),
   \item \TT{unsigned int} (0..4,294,967,295 (~4.3Gb) or \TT{0..0xFFFFFFFF}),
   \item \TT{unsigned char} (0..255 or \TT{0..0xFF}), 
   \item \TT{size\_t}.
  \end{itemize}

\item Signed types have the sign in the \ac{MSB}: 1 means \q{minus}, 0 means \q{plus}.

\item Promoting to a larger data types is simple:
\myref{subsec:sign_extending_32_to_64}.

\label{sec:signednumbers:negation}
\item Negation is simple: just invert all bits and add 1.

We can keep in mind that a number of inverse sign is located on the opposite side at the same proximity from zero.
The addition of one is needed because zero is present in the middle.

\myindex{x86!\Instructions!IDIV}
\myindex{x86!\Instructions!DIV}
\myindex{x86!\Instructions!IMUL}
\myindex{x86!\Instructions!MUL}
\myindex{x86!\Instructions!CBW}
\myindex{x86!\Instructions!CWD}
\myindex{x86!\Instructions!CWDE}
\myindex{x86!\Instructions!CDQ}
\myindex{x86!\Instructions!CDQE}
\myindex{x86!\Instructions!MOVSX}
\myindex{x86!\Instructions!SAR}
\item 
	The addition and subtraction operations work well for both signed and unsigned values.
	But for multiplication and division operations, x86 has different instructions:
	\TT{IDIV}/\TT{IMUL} for signed and \TT{DIV}/\TT{MUL} for unsigned.
\item
	Here are some more instructions that work with signed numbers:\\
	\TT{CBW/CWD/CWDE/CDQ/CDQE} (\myref{ins:CBW_CWD_etc}), \TT{MOVSX} (\myref{MOVSX}), \TT{SAR} (\myref{ins:SAR}).
\end{itemize}

A table of some negative and positive values (\myref{signed_tbl}) looks like thermometer with Celsius scale.
This is why addition and subtraction works equally well for both signed and unsigned numbers:
if the first addend is represented as mark on thermometer, and one need to add a second addend,
and it's positive, we just shift mark up on thermometer by the value of second addend.
If the second addend is negative, then we shift mark down to absolute value of the second addend.

Addition of two negative numbers works as follows.
For example, we need to add -2 and -3 using 16-bit registers.
-2 and -3 is 0xfffe and 0xfffd respectively.
If we add these numbers as unsigned, we will get 0xfffe+0xfffd=0x1fffb.
But we work on 16-bit registers, so the result is \emph{cut off}, the first 1 is dropped,
0xfffb is left, and this is -5.
This works because -2 (or 0xfffe) can be represented using plain English like this:
``2 lacks in this value up to maximal value in 16-bit register + 1''.
-3 can be represented as ``\dots 3 lacks in this value up to \dots''.
Maximal value of 16-bit register + 1 is 0x10000.
During addition of two numbers and \emph{cutting off} by $2^{16}$ modulo, $2+3=5$ \emph{will be lacking}.

% subsections:
\input{fundamentals/MUL_IMUL_EN}
\input{fundamentals/one_more_EN}

\subsection{-1}

Now you see that $-1$ is when all bits are set.
Often, you can find the $-1$ constant in all sorts of code, where a constant with all bits set are needed, for example, a mask.

For example: \myref{using_OR_instead_of_MOV}.

}
\RU{\mysection{\SignedNumbersSectionName}
\label{sec:signednumbers}
\myindex{Signed numbers}

Методов представления чисел со знаком \q{плюс} или \q{минус} несколько, 
но в компьютерах обычно применяется метод \q{дополнительный код} или \q{two's complement}.

Вот таблица некоторых значений байтов:

\label{signed_tbl}
\begin{center}
\begin{tabular}{ | l | l | l | l | }
\hline
\HeaderColor двоичное & \HeaderColor шестнадцатеричное & \HeaderColor беззнаковое & \HeaderColor знаковое \\
\hline
01111111 & 0x7f & 127 & 127 \\
\hline
01111110 & 0x7e & 126 & 126 \\
\hline
\multicolumn{4}{ |c| }{...} \\
\hline
00000110 & 0x6 & 6 & 6 \\
\hline
00000101 & 0x5 & 5 & 5 \\
\hline
00000100 & 0x4 & 4 & 4 \\
\hline
00000011 & 0x3 & 3 & 3 \\
\hline
00000010 & 0x2 & 2 & 2 \\
\hline
00000001 & 0x1 & 1 & 1 \\
\hline
00000000 & 0x0 & 0 & 0 \\
\hline
11111111 & 0xff & 255 & -1 \\
\hline
11111110 & 0xfe & 254 & -2 \\
\hline
11111101 & 0xfd & 253 & -3 \\
\hline
11111100 & 0xfc & 252 & -4 \\
\hline
11111011 & 0xfb & 251 & -5 \\
\hline
11111010 & 0xfa & 250 & -6 \\
\hline
\multicolumn{4}{ |c| }{...} \\
\hline
10000010 & 0x82 & 130 & -126 \\
\hline
10000001 & 0x81 & 129 & -127 \\
\hline
10000000 & 0x80 & 128 & -128 \\
\hline
\end{tabular}
\end{center}

\myindex{x86!\Instructions!JA}
\myindex{x86!\Instructions!JB}
\myindex{x86!\Instructions!JL}
\myindex{x86!\Instructions!JG}
Разница в подходе к знаковым/беззнаковым числам, собственно, нужна потому что, например, 
если представить \TT{0xFFFFFFFE} и \TT{0x00000002} как беззнаковое, то первое число (4294967294) больше второго (2). 
Если их оба представить как знаковые, то первое будет $-2$, которое, разумеется, меньше чем второе (2).
Вот почему инструкции для условных переходов~(\myref{sec:Jcc}) представлены в обоих версиях ~--- 
и для знаковых сравнений (например, \JG, \JL) и для беззнаковых (\INS{JA}, \JB).

Для простоты, вот что нужно знать:

\begin{itemize}
\item Числа бывают знаковые и беззнаковые.

\item Знаковые типы в \CCpp:

  \begin{itemize}
    \item \TT{int64\_t} (-9,223,372,036,854,775,808 .. 9,223,372,036,854,775,807) 
	  (-~9.2..~9.2 квинтиллионов) или \\
                \TT{0x8000000000000000..0x7FFFFFFFFFFFFFFF}),
    \item \Tint (-2,147,483,648..2,147,483,647 (-~2.15..~2.15Gb) или \\
	    \TT{0x80000000..0x7FFFFFFF}),
    \item \Tchar (-128..127 или \TT{0x80..0x7F}),
    \item \TT{ssize\_t}.
   \end{itemize}

	Беззнаковые:
  \begin{itemize}
	  \item \TT{uint64\_t} (0..18,446,744,073,709,551,615 
		  (~18 квинтиллионов) или \TT{0..0xFFFFFFFFFFFFFFFF}),
   \item \TT{unsigned int} (0..4,294,967,295 (~4.3Gb) или \TT{0..0xFFFFFFFF}),
   \item \TT{unsigned char} (0..255 или \TT{0..0xFF}), 
   \item \TT{size\_t}.
  \end{itemize}

\item У знаковых чисел знак определяется \ac{MSB}: 1 означает \q{минус}, 0 означает \q{плюс}.

\item Преобразование в б\'{о}льшие типы данных обходится легко:

	\myref{subsec:sign_extending_32_to_64}.

\label{sec:signednumbers:negation}
\item Изменить знак легко: просто инвертируйте все биты и прибавьте 1.
Мы можем заметить, что число другого знака находится на другой стороне на том же расстоянии от нуля.
Прибавление единицы необходимо из-за присутствия нуля посредине.

\myindex{x86!\Instructions!IDIV}
\myindex{x86!\Instructions!DIV}
\myindex{x86!\Instructions!IMUL}
\myindex{x86!\Instructions!MUL}
\myindex{x86!\Instructions!CBW}
\myindex{x86!\Instructions!CWD}
\myindex{x86!\Instructions!CWDE}
\myindex{x86!\Instructions!CDQ}
\myindex{x86!\Instructions!CDQE}
\myindex{x86!\Instructions!MOVSX}
\myindex{x86!\Instructions!SAR}
\item Инструкции сложения и вычитания работают одинаково хорошо и для знаковых и для беззнаковых значений.
	Но для операций умножения и деления, в x86 имеются разные инструкции:
	\TT{IDIV}/\TT{IMUL} для знаковых и \TT{DIV}/\TT{MUL} для беззнаковых.

\item Еще инструкции работающие со знаковыми числами:\\
	\TT{CBW/CWD/CWDE/CDQ/CDQE} (\myref{ins:CBW_CWD_etc}), \TT{MOVSX} (\myref{MOVSX}), \TT{SAR} (\myref{ins:SAR}).
\end{itemize}

Таблица некоторых отрицательных и положительных значений (\myref{signed_tbl}) напоминает термометр со шкалой по Цельсию.
Вот почему сложение и вычитание работает одинаково хорошо и для знаковых и беззнаковых чисел:
если первое слагаемое представить как отметку на термометре, и нужно прибавить второе слагаемое,
и оно положительне, то мы просто поднимаем отметку вверх на значение второго слагаемого.
Если второе слагаемое отрицательное, то мы опускаем отметку вниз на абсолютное значение от второго слагаемого.

Сложение двух отрицательных чисел работает так.
Например, нужно сложить -2 и -3 используя 16-битные регистры.
-2 и -3 это 0xfffe и 0xfffd соответственно.
Если сложить эти два числа как беззнаковые, то получится 0xfffe+0xfffd=0x1fffb.
Но мы работаем с 16-битными регистрами, так что результат \emph{обрезается}, первая единица выкидывается,
остается 0xfffb, а это -5.
Это работает потому что -2 (или 0xfffe) можно описать простым русским языком так:
``в этом значении не достает двух до максимального значения в регистре + 1''.
-3 можно описать ``\dots не достает трех до \dots''.
Максимальное значение 16-битного регистра + 1 это 0x10000.
При складывании двух чисел, и \emph{обрезании} по модулю $2^{16}$, \emph{не хватать} будет $2+3=5$.

% subsections:
\input{fundamentals/MUL_IMUL_RU}
\input{fundamentals/one_more_RU}

\subsection{-1}

Теперь видно, что $-1$ это когда все биты выставлены.
Часто, вы можете встретить константу $-1$ в каком угодно коде, где просто нужна константа со всеми битами, например, некая маска.

Например: \myref{using_OR_instead_of_MOV}.

}
\ES{\mysection{\SignedNumbersSectionName}
\label{sec:signednumbers}
\myindex{Signed numbers}

Hay distintos m\'etodos para representar n\'umeros con signo,
pero el \q{complemento a dos} es el m\'as popular en las computadoras.

Aqu\'i hay una tabla con los valores de algunos bytes:

\begin{center}
\begin{tabular}{ | l | l | l | l | }
\hline
\HeaderColor binario & \HeaderColor hexadecimal & \HeaderColor sin signo & \HeaderColor con signo \\
\hline
01111111 & 0x7f & 127 & 127 \\
\hline
01111110 & 0x7e & 126 & 126 \\
\hline
\multicolumn{4}{ |c| }{...} \\
\hline
00000110 & 0x6 & 6 & 6 \\
\hline
00000101 & 0x5 & 5 & 5 \\
\hline
00000100 & 0x4 & 4 & 4 \\
\hline
00000011 & 0x3 & 3 & 3 \\
\hline
00000010 & 0x2 & 2 & 2 \\
\hline
00000001 & 0x1 & 1 & 1 \\
\hline
00000000 & 0x0 & 0 & 0 \\
\hline
11111111 & 0xff & 255 & -1 \\
\hline
11111110 & 0xfe & 254 & -2 \\
\hline
11111101 & 0xfd & 253 & -3 \\
\hline
11111100 & 0xfc & 252 & -4 \\
\hline
11111011 & 0xfb & 251 & -5 \\
\hline
11111010 & 0xfa & 250 & -6 \\
\hline
\multicolumn{4}{ |c| }{...} \\
\hline
10000010 & 0x82 & 130 & -126 \\
\hline
10000001 & 0x81 & 129 & -127 \\
\hline
10000000 & 0x80 & 128 & -128 \\
\hline
\end{tabular}
\end{center}

\myindex{x86!\Instructions!JA}
\myindex{x86!\Instructions!JB}
\myindex{x86!\Instructions!JL}
\myindex{x86!\Instructions!JG}
La diferencia entre n\'umeros con signo y sin signo est\'a en que si representamos \TT{0xFFFFFFFE} y \TT{0x00000002}
sin signo, el primero (4294967294) es mayor que el segundo (2).
Pero si representamos ambos con signo, el primero se vuelve $-2$, y es menor que el segundo (2).
Esa es la raz\'on por la que se tienen saltos condicionales~(\myref{sec:Jcc}) tanto para operaciones con signo (e.g. \JG, \JL)
como sin signo (\INS{JA}, \JB).

Con el fin de la simplicidad, esto es lo que uno debe de saber:

\myindex{x86!\Instructions!IDIV}
\myindex{x86!\Instructions!DIV}
\myindex{x86!\Instructions!IMUL}
\myindex{x86!\Instructions!MUL}
\myindex{x86!\Instructions!CBW}
\myindex{x86!\Instructions!CWD}
\myindex{x86!\Instructions!CWDE}
\myindex{x86!\Instructions!CDQ}
\myindex{x86!\Instructions!CDQE}
\myindex{x86!\Instructions!MOVSX}
\myindex{x86!\Instructions!SAR}
\label{sec:signednumbers:negation}

\begin{itemize}
\item Los n\'umeros pueden ser con signo y sin signo.

\item Tipos con signo en \CCpp:

  \begin{itemize}
    \item \TT{int64\_t} (-9,223,372,036,854,775,808 .. 9,223,372,036,854,775,807) (-~9.2..~9.2 billones) \ESph{} \\
		  \TT{0x8000000000000000..0x7FFFFFFFFFFFFFFF}),
    \item \Tint (-2,147,483,648..2,147,483,647 (-~2.15..~2.15Gb) \ESph{} \\
	    \TT{0x80000000..0x7FFFFFFF}),
    \item \Tchar (-128..127 \ESph{} \TT{0x80..0x7F}),
    \item \TT{ssize\_t}.
   \end{itemize}

	Sin signo:
  \begin{itemize}
	  \item \TT{uint64\_t} (0..18,446,744,073,709,551,615 
		  (~18 billones) \ESph{}
		  \TT{0..0xFFFFFFFFFFFFFFFF}),
  \item \TT{unsigned int} (0..4,294,967,295 (~4.3Gb) \ESph{} \TT{0..0xFFFFFFFF}),
  \item \TT{unsigned char} (0..255 \ESph{} \TT{0..0xFF}), 
   \item \TT{size\_t}.
  \end{itemize}

\item En los tipos con signo, el signo se encuentra en el bit m\'as significativo: 1 significa \q{menos}, 0 significa \q{m\'as}.

\item Promover a un tipo de dato m\'as grande es sencillo: \myref{subsec:sign_extending_32_to_64}.

\item La negaci\'on es simple: s\'olo invierte todos los bits y suma 1.
Podemos recordar que un n\'umero con signo contrario se localiza en el lado opuesto a la misma distancia de cero.
La suma de 1 es necesaria porque el cero se localiza en medio.

\item Las operaciones de suma y resta funcionan bien para valores tanto con signo como sin signo.
	Sin embargo, para las operaciones de multiplicaci\'on y divisi\'on x86 tiene instrucciones distintas:
	\TT{IDIV}/\TT{IMUL} para n\'umeros con signo \ESph{} \TT{DIV}/\TT{MUL} para n\'umeros sin signo.
\item \'Estas son algunas otras instruccciones que funcionan con n\'umeros con signo:\\
	\TT{CBW/CWD/CWDE/CDQ/CDQE} (\myref{ins:CBW_CWD_etc}), \TT{MOVSX} (\myref{MOVSX}), \TT{SAR} (\myref{ins:SAR}).

\end{itemize}

% TBT

}
\DE{\mysection{\SignedNumbersSectionName}
\label{sec:signednumbers}
\myindex{Signed numbers}

Es gibt verschiedene Arten vorzeichenbehaftete Zahlen darzustellen,
jedoch ist das \q{Zweierkomplement} die am weitesten verbreitete in Computern.

Hier ist eine Tabelle für einige Byte-Werte:

\begin{center}
\begin{tabular}{ | l | l | l | l | }
\hline
\HeaderColor binär & \HeaderColor hexadezimal & \HeaderColor vorzeichenlos & \HeaderColor vorzeichenbehaftet \\
\hline
01111111 & 0x7f & 127 & 127 \\
\hline
01111110 & 0x7e & 126 & 126 \\
\hline
\multicolumn{4}{ |c| }{...} \\
\hline
00000110 & 0x6 & 6 & 6 \\
\hline
00000101 & 0x5 & 5 & 5 \\
\hline
00000100 & 0x4 & 4 & 4 \\
\hline
00000011 & 0x3 & 3 & 3 \\
\hline
00000010 & 0x2 & 2 & 2 \\
\hline
00000001 & 0x1 & 1 & 1 \\
\hline
00000000 & 0x0 & 0 & 0 \\
\hline
11111111 & 0xff & 255 & -1 \\
\hline
11111110 & 0xfe & 254 & -2 \\
\hline
11111101 & 0xfd & 253 & -3 \\
\hline
11111100 & 0xfc & 252 & -4 \\
\hline
11111011 & 0xfb & 251 & -5 \\
\hline
11111010 & 0xfa & 250 & -6 \\
\hline
\multicolumn{4}{ |c| }{...} \\
\hline
10000010 & 0x82 & 130 & -126 \\
\hline
10000001 & 0x81 & 129 & -127 \\
\hline
10000000 & 0x80 & 128 & -128 \\
\hline
\end{tabular}
\end{center}

\myindex{x86!\Instructions!JA}
\myindex{x86!\Instructions!JB}
\myindex{x86!\Instructions!JL}
\myindex{x86!\Instructions!JG}
Der Unterschied zwischen vorzeichenbehafteten und vorzeichenlosen Zahlen ist, dass wenn \TT{0xFFFFFFFE}
und \TT{0x00000002} ohne Vorzeichen repräsentiert werden, die erste Zahl (4294967294) größer ist als
die zweite Zahl (2).
Wenn beide Zahlen als vorzeichenbehaftet repräsentiert werden, wird die erste Zahl $-2$ und ist kleiner
als die zweite Zahl (2).
Das ist der Grund, warum bedingte Sprünge ~(\myref{sec:Jcc}) sowohl für vorzeichenbehaftete (e.g. \JG, \JL)
als auch vorzeichenlose (\INS{JA}, \JB) Operationen vorhanden sind.

Aus Gründen der Einfachheit ist hier was man wissen muss:

\begin{itemize}
\item Zahlen können mit oder ohne Vorzeichen sein.

\item vorzeichenbehaftete Datentypen in \CCpp:

  \begin{itemize}
    \item \TT{int64\_t} (-9,223,372,036,854,775,808 .. 9,223,372,036,854,775,807)
	  (-~9.2..~9.2 Quintillionen) oder \\
                \TT{0x8000000000000000..0x7FFFFFFFFFFFFFFF}),
    \item \Tint (-2,147,483,648..2,147,483,647 (-~2.15..~2.15Gb) oder \\
	    \TT{0x80000000..0x7FFFFFFF}),
    \item \Tchar (-128..127 oder \TT{0x80..0x7F}),
    \item \TT{ssize\_t}.
   \end{itemize}

	Vorzeichenlos:
  \begin{itemize}
	  \item \TT{uint64\_t} (0..18,446,744,073,709,551,615 
		  (~18 Quintillionen) oder \TT{0..0xFFFFFFFFFFFFFFFF}),
   \item \TT{unsigned int} (0..4,294,967,295 (~4.3Gb) oder \TT{0..0xFFFFFFFF}),
   \item \TT{unsigned char} (0..255 oder \TT{0..0xFF}), 
   \item \TT{size\_t}.
  \end{itemize}

\item Vorzeichenbehaftete Typen haben das Vorzeichen am \ac{MSB}: 1 bedeutet \q{Minus}, 0 bedeutet \q{Plus}.

\item Erweitern auf größere Datentypen ist einfach:
\myref{subsec:sign_extending_32_to_64}.

\label{sec:signednumbers:negation}
\item Negieren ist einfach: es müssen lediglich alle Bits invertiert und anschließend 1 addiert werden.

%TODO Not sure how to translate this into an understandable in german.
%We can keep in mind that a number of inverse sign is located on the opposite side at the same proximity from zero.
%The addition of one is needed because zero is present in the middle.

\myindex{x86!\Instructions!IDIV}
\myindex{x86!\Instructions!DIV}
\myindex{x86!\Instructions!IMUL}
\myindex{x86!\Instructions!MUL}
\myindex{x86!\Instructions!CBW}
\myindex{x86!\Instructions!CWD}
\myindex{x86!\Instructions!CWDE}
\myindex{x86!\Instructions!CDQ}
\myindex{x86!\Instructions!CDQE}
\myindex{x86!\Instructions!MOVSX}
\myindex{x86!\Instructions!SAR}
\item 
	Die Addition und Subtraktion funktioniert sowohl für Zahlen mit als auch ohne Vorzeichen.
	Für Multiplikation und Division gibt es bei x86 unterschiedliche Anweisungen:
	\TT{IDIV}/\TT{IMUL} für vorzeichenbehaftete und \TT{DIV}/\TT{MUL} für vorzeichenlose Zahlen.
\item
	Hier sind einige weitere Anweisungen die mit vorzeichenbehafteten Zahlen arbeiten:\\
	\TT{CBW/CWD/CWDE/CDQ/CDQE} (\myref{ins:CBW_CWD_etc}), \TT{MOVSX} (\myref{MOVSX}), \TT{SAR} (\myref{ins:SAR}).
\end{itemize}

Eine Tabelle mit negativen und positiven Werten (\myref{signed_tbl}) sieht aus wie ein Thermometer mit Celsius-Skala.
Das ist der Grund warum Addition und Subtraktion für Zahlen mit und ohne Vorzeichen gleich funktioniert:
wenn der erste Summand eine Markierung auf dem Thermometer ist und ein weiterer Summand addiert werden soll,
der positiv ist, muss lediglich die Markierung auf dem Thermometer um den Wert des zweiten Summanden nach
oben verschoben werden.
Ist der zweite Summand negativ, wird die Markierung um den entsprechenden, absoluten Wert nach unten verschoben.

Die Addition zweier negativer Zahlen funktioniert wie folgt:
Wenn beispielsweise -2 und -3 unter Verwendung eines 16-Bit-Registers addiert werden sollen,
ist die Darstellung 0xfffe beziehungsweise 0xfffd.
Wenn diese Werte ohne Vorzeichen addiert werden, ist das Ergebnis 0xfffe+0xfffd=0x1fffb.
Allerdings sollen 16-Bit-Register verwendet werden, also wird beim Ergebis die erste 1 abgeschnitten.
Es bleibt 0xfffb was -5 entspricht.
Dies funktioniert, weil -2 (oder 0xfffe) in natürlicher Sprache wie folgt repräsentiert werden kann:
``2 fehlt bei diesem Werte bis zum maximalen Wert in einem 16-Bit-Register + 1''.
-3 kann repräsentiert werden als ``\dots 3 fehlt in diesem Wert bis zu \dots''.
Der maximale Wert eines 16-Bit-Registers + 1 ist 0x10000.
Bei der Addition der beiden Zahlen und \emph{Abschneiden}  durch $2^{16}$ modulo,
wird $2+3=5$ \emph{fehlen}.

% subsections:
\input{fundamentals/MUL_IMUL_DE}
\input{fundamentals/one_more_DE}

% TBT
%\subsection{-1}

%Now you see that $-1$ is when all bits are set.
%Often, you can find the $-1$ constant in all sorts of code, where a constant with all bits set are needed, for example, a mask.

%For example: \myref{using_OR_instead_of_MOV}.

}
\FR{\mysection{\SignedNumbersSectionName}
\label{sec:signednumbers}
\myindex{Signed numbers}

Il existe plusieurs méthodes pour représenter les nombres signées,
mais le \q{complément à deux} est la plus populaire sur les ordinateurs.

Voici une table pour quelques valeurs d'octet:

\begin{center}
\begin{tabular}{ | l | l | l | l | }
\hline
\HeaderColor binaire & \HeaderColor hexadécimal & \HeaderColor non-signé & \HeaderColor signé \\
\hline
01111111 & 0x7f & 127 & 127 \\
\hline
01111110 & 0x7e & 126 & 126 \\
\hline
\multicolumn{4}{ |c| }{...} \\
\hline
00000110 & 0x6 & 6 & 6 \\
\hline
00000101 & 0x5 & 5 & 5 \\
\hline
00000100 & 0x4 & 4 & 4 \\
\hline
00000011 & 0x3 & 3 & 3 \\
\hline
00000010 & 0x2 & 2 & 2 \\
\hline
00000001 & 0x1 & 1 & 1 \\
\hline
00000000 & 0x0 & 0 & 0 \\
\hline
11111111 & 0xff & 255 & -1 \\
\hline
11111110 & 0xfe & 254 & -2 \\
\hline
11111101 & 0xfd & 253 & -3 \\
\hline
11111100 & 0xfc & 252 & -4 \\
\hline
11111011 & 0xfb & 251 & -5 \\
\hline
11111010 & 0xfa & 250 & -6 \\
\hline
\multicolumn{4}{ |c| }{...} \\
\hline
10000010 & 0x82 & 130 & -126 \\
\hline
10000001 & 0x81 & 129 & -127 \\
\hline
10000000 & 0x80 & 128 & -128 \\
\hline
\end{tabular}
\end{center}

\myindex{x86!\Instructions!JA}
\myindex{x86!\Instructions!JB}
\myindex{x86!\Instructions!JL}
\myindex{x86!\Instructions!JG}
La différence entre nombres signé et non-signé est que si l'on représente \TT{0xFFFFFFFE}
et \TT{0x00000002} comme non signées, alors le premier nombre (4294967294) est plus
grand que le second (2).
Si nous les représentons comme signés, le premier devient $-2$, et il est plus petit
que le second.
C'est la raison pour laquelle les sauts conditionnels~(\myref{sec:Jcc}) existent
à la fois pour des opérations signées (p. ex. \JG, \JL) et non-signées (\INS{JA}, \JB).

Par souci de simplicité, voici ce qu'il faut retenir:

\begin{itemize}
\item Les nombres peuvent être signés ou non-signés.

\item Types \CCpp signés:

  \begin{itemize}
    \item \TT{int64\_t} (-9,223,372,036,854,775,808 .. 9,223,372,036,854,775,807)
	  (-~9.2..~9.2 quintillions) ou \\
                \TT{0x8000000000000000..0x7FFFFFFFFFFFFFFF}),
    \item \Tint (-2,147,483,648..2,147,483,647 (-~2.15..~2.15Gb) ou \TT{0x80000000..0x7FFFFFFF}),
    \item \Tchar (-128..127 ou \TT{0x80..0x7F}),
    \item \TT{ssize\_t}.
   \end{itemize}

	Non-signés:
  \begin{itemize}
	  \item \TT{uint64\_t} (0..18,446,744,073,709,551,615 
		  (~18 quintillions) ou \TT{0..0xFFFFFFFFFFFFFFFF}),
   \item \TT{unsigned int} (0..4,294,967,295 (~4.3Gb) ou \TT{0..0xFFFFFFFF}),
   \item \TT{unsigned char} (0..255 ou \TT{0..0xFF}),
   \item \TT{size\_t}.
  \end{itemize}

\item Les types signés ont le signe dans le \ac{MSB}: 1 signifie \q{moins}, 0 signifie \q{plus}.

\item Étendre à un type de données plus large est facile:
\myref{subsec:sign_extending_32_to_64}.

\label{sec:signednumbers:negation}
\item La négation est simple: il suffit d'inverser tous les bits et d'ajouter 1.

Nous pouvons garder à l'esprit qu'un nombre de signe opposé se trouve de l'autre côté,
à la même distance de zéro.
L'addition d'un est nécessaire car zéro se trouve au milieu.

\myindex{x86!\Instructions!IDIV}
\myindex{x86!\Instructions!DIV}
\myindex{x86!\Instructions!IMUL}
\myindex{x86!\Instructions!MUL}
\myindex{x86!\Instructions!CBW}
\myindex{x86!\Instructions!CWD}
\myindex{x86!\Instructions!CWDE}
\myindex{x86!\Instructions!CDQ}
\myindex{x86!\Instructions!CDQE}
\myindex{x86!\Instructions!MOVSX}
\myindex{x86!\Instructions!SAR}
\item
	Les opérations d'addition et de soustraction fonctionnent bien pour les valeurs signées et non-signées.
	Mais pour la multiplication et la division, le x86 possède des instructions différentes:
	\TT{IDIV}/\TT{IMUL} pour les signés et \TT{DIV}/\TT{MUL} pour les non-signés.
\item
	Voici d'autres instructions qui fonctionnent avec des nombres signés:\\
	\TT{CBW/CWD/CWDE/CDQ/CDQE} (\myref{ins:CBW_CWD_etc}), \TT{MOVSX} (\myref{MOVSX}), \TT{SAR} (\myref{ins:SAR}).
\end{itemize}

Une table avec quelques valeurs négatives et positives (\myref{signed_tbl}) ressemble
à un thermomètre avec une échelle Celsius.
C'est pourquoi l'addition et la soustraction fonctionnent bien pour les nombres signés
et non-signés:
si le premier opérande est représenté par une marque sur un thermomètre, et que l'on
doit ajouter un second opérande, et qu'il est positif, nous devons juste augmenter
la marque sur le thermomètre de la valeur du second opérande.
Si le second opérande est négatif, alors nous baissons la marque de la valeur absolue
du second opérande.

L'addition de deux nombres négatifs fonctionne comme suit.
Par exemple, nous devons ajouter -2 et -3 en utilisant des registres 16-bit.
-2 et -3 sont respectivement 0xfffe et 0xfffd.
si nous les ajoutons comme nombres non-signés, nous obtenons 0xfffe+0xfffd=0x1fffb.
Mais nous travaillons avec des registres 16-bit, le résultat est \emph{tronqué},
le premier 1 est perdu, et il reste 0xfffb et c'est -5.
Ceci fonctionne car -2 (ou 0xfffe) peut être représenté en utilisant des mots simples
comme suit:
``il manque 2 à la valeur maximale d'un registre 16-bit + 1''.
-3 peut être représenté comme ``\dots il manque 3 à la valeur maximale jusqu'à \dots''.
La valeur maximale d'un registre 16-bit + 1 est 0x10000.
Pendant l'addition de deux nombres et en \emph{tronquant} modulo $2^{16}$, il manquera
$2+3=5$.

% subsections:
\input{fundamentals/MUL_IMUL_FR}
\input{fundamentals/one_more_FR}

\subsection{-1}

Vous savez maintenant que $-1$ est lorsque tous les bits sont mis à 1.
Souvent, vous pouvez trouver la constante $-1$ dans toute sorte de code qui nécessite
une constante avec tous les bits à 1, par exemple, un masque.

Par exemple: \myref{using_OR_instead_of_MOV}.

}

\EN{\mysection{Integer overflow}

I intentionally put this section after the section about signed number representation.

First, take a look at this implementation of \emph{itoa()} function from \InSqBrackets{\KRBook}:

\begin{lstlisting}[style=customc]
void itoa(int n, char s[])
{
        int i, sign;
        if ((sign = n) < 0) /* record sign */
                n = -n; /* make n positive */
        i = 0;
        do { /* generate digits in reverse order */
                s[i++] = n % 10 + '0'; /* get next digit */
        } while ((n /= 10) > 0); /* delete it */
        if (sign < 0)
                s[i++] = '-';
        s[i] = '\0';
        strrev(s);
}
\end{lstlisting}

( The full source code: \url{\RepoURL/fundamentals/itoa_KR.c} )

It has a subtle bug. Try to find it. You can download source code, compile it, etc.
The answer on the next page.

\clearpage

From \InSqBrackets{\KRBook}:

\begin{framed}
\begin{quotation}
Exercise 3-4. In a two's complement number representation, our version of \emph{itoa}
does not handle the largest negative number, that is, the value
of \emph{n} equal to $-(2^{wordsize-1})$. Explain why not. Modify it to
print that value correctly, regardless of the machine on which
it runs.
\end{quotation}
\end{framed}

The answer is: the function cannot process largest negative number (INT\_MIN or 0x80000000 or -2147483648) correctly.

How to change sign? Invert all bits and add 1.
If to invert all bits in INT\_MIN value (0x80000000), this is 0x7fffffff. Add 1 and this is 0x80000000 again.
So changing sign has no effect.
This is an important artifact of two's complement system.

Further reading:

\begin{itemize}
\item blexim -- Basic Integer Overflows\footnote{\url{http://phrack.org/issues/60/10.html}}

\item Yannick Moy, Nikolaj Bjørner, and David Sielaff -- Modular Bug-finding for Integer Overflows in the Large: Sound, Efficient, Bit-precise Static Analysis\footnote{\url{https://yurichev.com/mirrors/SMT/z3prefix.pdf}}
\end{itemize}

}
\RU{\mysection{Целочисленное переполнение (integer overflow)}

Я сознательно расположил эту секцию после секции о представлении знаковых чисел.

В начале, взгляние на эту реализацию ф-ции \emph{itoa()} из \InSqBrackets{\KRBook}:

\begin{lstlisting}[style=customc]
void itoa(int n, char s[])
{
        int i, sign;
        if ((sign = n) < 0) /* record sign */
                n = -n; /* make n positive */
        i = 0;
        do { /* generate digits in reverse order */
                s[i++] = n % 10 + '0'; /* get next digit */
        } while ((n /= 10) > 0); /* delete it */
        if (sign < 0)
                s[i++] = '-';
        s[i] = '\0';
        strrev(s);
}
\end{lstlisting}

( Полный текст: \url{\GitHubBlobMasterURL/fundamentals/itoa_KR.c} )

Здесь есть малозаметная ошибка. Попробуйте её найти. Можете скачать исходный код, скомпилировать его, итд.
Ответ на следующей странице.

\clearpage

Из \InSqBrackets{\KRBook}:

\begin{framed}
\begin{quotation}
Упражнение 3-4. В представлении чисел с помощью дополнения до двойки наша версия функции \emph{itoa}
не умеет обрабатывать самое большое по модулю отрицательное число, т.е., значение 
\emph{n}, равное $-(2^{wordsize-1})$. Объясните, почему это так. Доработайте функцию так, чтобы она
выводила это число правильно независимо от системы, в которой она работает.
\end{quotation}
\end{framed}

Ответ: ф-ция не может корректно обработать самое большое отрицательное число (INT\_MIN или 0x80000000 или -2147483648).

Как изменить знак? Инвертируйте все биты и прибавьте 1.
Если инвертировать все биты в значении INT\_MIN (0x80000000), это 0x7fffffff. Прибавьте 1 и это снова 0x80000000.
Так что смена знака не дает никакого эффекта.
Это важный артефакт two's complement-системы.

Еще об этом:

\begin{itemize}
\item blexim -- Basic Integer Overflows\footnote{\url{http://phrack.org/issues/60/10.html}}

\item Yannick Moy, Nikolaj Bjørner, and David Sielaff -- Modular Bug-finding for Integer Overflows in the Large: Sound, Efficient, Bit-precise Static Analysis\footnote{\url{https://yurichev.com/mirrors/SMT/z3prefix.pdf}}
\end{itemize}

}
\FR{\mysection{Dépassement d'entier}

J'ai intentionnellement mis cette section après celle sur la représentation des nombres
signés.

Tout d'abord, regardons l'implémentation de la fonction  \emph{itoa()} dans \InSqBrackets{\KRBook}:

\begin{lstlisting}[style=customc]
void itoa(int n, char s[])
{
        int i, sign;
        if ((sign = n) < 0) /* record sign */
                n = -n; /* make n positive */
        i = 0;
        do { /* generate digits in reverse order */
                s[i++] = n % 10 + '0'; /* get next digit */
        } while ((n /= 10) > 0); /* delete it */
        if (sign < 0)
                s[i++] = '-';
        s[i] = '\0';
        strrev(s);
}
\end{lstlisting}

( Le code source complet: \url{\RepoURL/fundamentals/itoa_KR.c} )

Elle a un bogue subtil. Essayez de le trouver. Vous pouvez téléchargez le code source,
le compiler, etc.
La réponse se trouve à la page suivante.

\clearpage

De \InSqBrackets{\KRBook}:

\begin{framed}
\begin{quotation}
Exercice 3-4. Dans un système de représentation des nombres par complément à deux
notre version de \emph{itoa} ne peut pas traiter le plus grand nombre négatif. c'est-à-
dire la valeur de \emph{n} égale à $-(2^{wordsize-1})$. Pourquoi ? Modifiez \emph{itoa}
de façon à ce qu'elle traite ce cas correctement, quelle que soit la machine utilisée.
\end{quotation}
\end{framed}

La réponse est: la fonction ne peut pas traiter le plus grand nombre négatif (INT\_MIN
ou 0x80000000 ou -2147483648) correctement.

Comment changer le signe? Inverser tous les bits et ajouter 1.
Si vous inversez tous les bits de la valeur INT\_MIN (0x80000000), ça donne 0x7fffffff.
Ajouter 1 et vous obtenez à nouveau 0x80000000.
C'est un artefact important du système de complément à deux.

Lectures  complémentaires:

\begin{itemize}
\item blexim -- Basic Integer Overflows\footnote{\url{http://phrack.org/issues/60/10.html}}

\item Yannick Moy, Nikolaj Bjørner, et David Sielaff -- Modular Bug-finding for Integer Overflows in the Large: Sound, Efficient, Bit-precise Static Analysis\footnote{\url{https://yurichev.com/mirrors/SMT/z3prefix.pdf}}
\end{itemize}

}

\EN{\input{fundamentals/AND_EN}}\RU{\input{fundamentals/AND_RU}}%
\FR{\input{fundamentals/AND_FR}}

\EN{\input{fundamentals/AND_OR_as_SUB_ADD_EN}}\RU{\mysection{\emph{И} и \emph{ИЛИ} как вычитание и сложение}
\label{AND_OR_as_SUB_ADD}

\subsection{Текстовые строки в \ac{ROM} ZX Spectrum}
\myindex{ZX Spectrum}

Те, кто пытался исследовать внутренности \ac{ROM} ZX Spectrum-а, вероятно, замечали,
что последний символ каждой текстовой строки как будто бы отсутствует.

\begin{figure}[H]
\centering
\includegraphics[width=0.3\textwidth]{fundamentals/zx_spectrum_ROM.png}
\caption{Часть \ac{ROM} ZX Spectrum}
\end{figure}

На самом деле, они присутствуют.

Вот фрагмент из дизассемблированного \ac{ROM} ZX Spectrum 128K:

\lstinputlisting{fundamentals/ZX_Spectrum_ROM.lst}
( \url{http://www.matthew-wilson.net/spectrum/rom/128_ROM0.html} )

Последний символ имеет выставленный старший бит, который означает конец строки.
Вероятно, так было сделано, чтобы сэкономить место?
В старых 8-битных компьютерах был сильный дефицит памяти.

Символы всех сообщений всегда находятся в стандартной 7-битной \ac{ASCII}-таблице, так что это гарантия,
что 7-й бит никогда не используется для символов.

Чтобы вывести такую строку, мы должны проверять \ac{MSB} каждого байта, и если он выставлен, мы должны его сбросить,
затем вывести символ, затем остановиться.
Вот пример на Си:

\begin{lstlisting}[style=customc]
unsigned char hw[]=
{
	'H',
	'e',
	'l',
	'l',
	'o'|0x80
};

void print_string()
{
	for (int i=0; ;i++)
	{
		if (hw[i]&0x80) // проверить MSB
		{
			// сбросить MSB
			// (иными словами, сбросить всё, но оставить нетронутыми младшие 7 бит)
			printf ("%c", hw[i] & 0x7F);
			// остановиться
			break;
		};
		printf ("%c", hw[i]);
	};
};
\end{lstlisting}

И вот что интересно, так как 7-й бит это самый старший бит (в байте), мы можем проверить его, выставить и сбросить
используя арифметические операции вместо логических.

Я могу переписать свой пример на Си:

\begin{lstlisting}[style=customc]
unsigned char hw[]=
{
	'H',
	'e',
	'l',
	'l',
	'o'+0x80
};

void print()
{
	for (int i=0; ;i++)
	{
		// hw[] должен иметь тип 'unsigned char'
		if (hw[i] >= 0x80) // проверить MSB
		{
			printf ("%c", hw[i]-0x80); // сбросить MSB
			// останов
			break;
		};
		printf ("%c", hw[i]);
	};
};
\end{lstlisting}

\emph{char} по умолчанию это знаковый тип в \CCpp, так что, чтобы сравнивать его с переменной вроде 0x80 (которая отрицательная
($-128$),
если считается за знаковую),
мы должны считать каждый символ сообщения как беззнаковый.

Теперь, если 7-й бит выставлен, число всегда больше или равно 0x80.
Если 7-й бит сброшен, число всегда меньше 0x80.

И даже более того: если 7-й бит выставлен, его можно сбросить вычитанием 0x80, и ничего больше.
Если он уже сброшен, впрочем, операция вычитания уничтожит другие биты.

Точно также, если 7-й бит сброшен, можно его выставить прибавлением 0x80.
Но если он уже выставлен, операция сложения уничтожит остальные биты.

На самом деле, это справедливо для любого бита.
Если 4-й бит сброшен, вы можете выставить его просто прибавлением 0x10: 0x100+0x10 = 0x110.
Если 4-й бит выставлен, вы можете его сбросить вычитанием 0x10: 0x1234-0x10 = 0x1224.

Это работает, потому что перенос не случается во время сложения/вычитания.
Хотя, он случится если бит уже выставлен перед сложением или сброшен перед вычитанием.

Точно также, сложение/вычитание можно заменить на операции \emph{ИЛИ/И} если справедливы два условия:
1) вы хотите прибавить/вычесть число вида $2^n$;
2) бит в исходном значение сброшен/выставлен.

Например, прибавление 0x20 это то же что и применение \emph{ИЛИ} со значением 0x20 с условием что этот бит был сброшен перед
этим:
0x1204|0x20 = 0x1204+0x20 = 0x1224.

Вычитание 0x20 это то же что и применение \emph{И} со значением ~0x20 (0x....FFDF), но если этот бит был выставлен до этого:
0x1234\&(\~{}0x20) = 0x1234\&0xFFDF = 0x1234-0x20 = 0x1214.

Опять же, это работает потому что перенос не случается если вы прибавляете число вида $2^n$ и этот бит до этого сброшен.

Это важное свойство булевой алгебры, его стоит понимать и помнить о нем.

Еще один пример в этой книге: \myref{toupper_bit}.

}%
\FR{\input{fundamentals/AND_OR_as_SUB_ADD_FR}}

\EN{\input{fundamentals/XOR_EN}}\RU{\input{fundamentals/XOR_RU}}%
\FR{\mysection{XOR (OU exclusif)}
\label{XOR_property}

\input{fundamentals/XOR_property_FR}

\subsection{Différence logique}

\myindex{Cray}
Dans le manuel des super-ordinateurs (1976-1977)
\footnote{\url{http://www.bitsavers.org/pdf/cray/CRAY-1/HR-0004-CRAY_1_Hardware_Reference_Manual-PRELIMINARY-1975.OCR.pdf}},
on peut trouver que l'instruction XOR était appelée \emph{différence logique}.

En effet, XOR(a,b)=1 si a!=b.

\subsection{Langage courant}

L'opération XOR est présente dans le langage courant.
Lorsque quelqu'un demande ``s'il te plaît, achète des pommes ou des bananes'', ceci
signifie généralement ``achète le premier item ou le second, mais pas les deux''---ceci
est exactement un OU exclusif, car le OU logique signifierait ``les deux objets sont bien aussi''.

Certaines personnes suggèrent que ``et/ou'' devraient être utilisés dans le langage
courant pour mettre l'accent sur le fait que le OU logique est utilisé à la place
du OU exclusif: \url{https://en.wikipedia.org/wiki/And/or}.

\subsection{Chiffrement}

XOR est beaucoup utilisé à la fois par le chiffrement amateur (\myref{simple_XOR_encryption})
et \emph{réel} (au moins dans le \emph{réseau de Feistel}).

XOR est trés pratique ici car:
$cipher\_text = plain\_text \oplus key$ et alors:
$(plain\_text \oplus key) \oplus key = plain\_text$.

\subsection{\ac{RAID}4}
\myindex{RAID4}

\ac{RAID}4 offre une méthode très simple pour protéger les disques dur.
Par exemple, il y a quelques disques ($D_1$, $D_2$, $D_3$, etc.) et un disque de
parité ($P$).
Chaque bit/octet écrit sur le disque de parité est calculé et écrit au vol:

\begin{equation} \label{eq:RAID4}
P = D_1 \oplus D_2 \oplus D_3
\end{equation}

Si n'importe lequel des disques est défaillant, par exemple, $D_2$, il est restauré
en utilisant la même méthode:

\begin{equation}
D_2 = D_1 \oplus P \oplus D_3
\end{equation}

Si le disque de parité est défaillant, il est restauré en utilisant la méthode \myref{eq:RAID4}.
Si deux disques sont défaillants, alors il n'est pas possible de les restaurer les
deux.

\ac{RAID}5 est plus avancé, mais cet propriété de XOR y est encore utilisé.

C'est pourquoi les contrôleurs \ac{RAID} ont des ``accélérateurs XOR'' matériel
pour aider les opérations XOR sur de larges morceaux de données écrites au vol.
Depuis que les ordinateurs deviennent de plus en plus rapide, cela peut maintenant
être effectué au niveau logiciel, en utilisant \ac{SIMD}.

\subsection{Algorithme d'échange XOR}

C'est difficile à croire, mais ce code échangent les valeurs dans \EAX et \EBX sans
l'aide d'aucun autre registre ni d'espace mémoire.

\begin{lstlisting}[style=customasmx86]
xor eax, ebx
xor ebx, eax
xor eax, ebx
\end{lstlisting}

Cherchons comment ça fonctionne.
D'abord, récrivons le afin de retirer le langage d'assemblage x86:

\begin{lstlisting}
X = X XOR Y
Y = Y XOR X
X = X XOR Y
\end{lstlisting}

Qu'est ce que X et Y valent à chaque étape?
Gardez à l'esprit cette règle simple: $(X \oplus Y) \oplus Y = X$ pour toutes valeurs
de X et Y.

Regardons,
après la 1ère étape $X$ vaut $X \oplus Y$;
après la 2ème étape $Y$ vaut $Y \oplus (X \oplus Y) = X$;
après la 3ème étape $X$ vaut $(X \oplus Y) \oplus X = Y$.

Difficile de dire si on doit utiliser cette astuce, mais elle est un bon exemple de
démonstration des propriétés de XOR.

L'article de Wikipédia (\url{https://en.wikipedia.org/wiki/XOR_swap_algorithm}) donne
d'autres explication: l'addition et la soustraction peuvent être utilisées à la place
de XOR:

\begin{lstlisting}
X = X + Y
Y = X - Y
X = X - Y
\end{lstlisting}

Regardons:
après la 1ère étape $X$ vaut $X+Y$;
après la 2ème étape $Y$ vaut $X+Y-Y=X$;
après la 3ème étape $X$ vaut $X+Y-X=Y$.

\subsection{liste chaînée XOR}
\myindex{Doubly linked list}

Une liste doublement chaînée est une liste dans laquelle chaque élément a un lien
sur l'élément précédent et sur le suivant.
Ainsi, il est très facile de traverser la liste dans un sens ou dans l'autre.
\TT{std::list}, qui implémente les listes doublement chaînées en C++, est également
examiné dans ce livre: \myref{std_list}.

Donc chaque élément possède deux pointeurs.
Est-il possible, peut-être dans un environnement avec peu de mémoire, de garder toutes
ces fonctionnalités, avec un seul pointeur au lieu de deux?
Oui, si la valeur de $prev \oplus next$ est stockée dans cette cellule mémoire, qui
est habituellement appelé ``lien''.

Peut-être que nous pouvons dire que l'adresse de l'élément précédent est ``chiffrée''
en utilisant l'adresse de l'élément suivant et réciproquement:
l'adresse de l'élément suivant est ``chiffrée'' en utilisant l'adresse de l'élément
précédent.

Lorsque nous traversons cette liste en avant, nous connaissons toujours l'adresse
de l'élément précédent, donc nous pouvons ``déchiffrer'' ce champ et obtenir l'adresse
de l'élément suivant.
De même, il est possible de traverser cette liste en arrière, ``déchiffrer'' ce champ
en utilisant l'adresse de l'élément suivant.

Mais il n'est pas possible de trouver l'adresse de l'élément précédent ou suivant
d'un élément spécifique sans connaître l'adresse du premier.

Deux éléments pour compléter cette solution: le premier élément aura toujours l'adresse
de l'élément suivant sans aucun XOR, le dernier élément aura l'adresse du premier
élément sans aucun XOR.

Maintenant, résumons. Ceci est un exemple d'une liste doublement chaînée de 5 éléments.
$A_x$ est l'adresse de l'élément.

\begin{center}
\begin{tabular}{ | l | l | }
	\hline
	\HeaderColor adresse & \HeaderColor contenu du champ \emph{link} \\
	\hline
	$A_0$ & $A_1$ \\
	\hline
	$A_1$ & $A_0 \oplus A_2$ \\
	\hline
	$A_2$ & $A_1 \oplus A_3$ \\
	\hline
	$A_3$ & $A_2 \oplus A_4$ \\
	\hline
	$A_4$ & $A_3$ \\
	\hline
\end{tabular}
\end{center}

À nouveau, il est difficile de dire si quelqu'un doit utiliser ce truc rusé, mais
c'est une bonne démonstration des propriétés de XOR. Avec l'algorithme d'échange
avec XOR, l'article de Wikipédia montre des méthodes pour utiliser l'addition et
la soustraction au lieu de XOR:
\url{https://en.wikipedia.org/wiki/XOR_linked_list}.

% subsection:
\input{fundamentals/XOR_switch_FR}
\input{fundamentals/zobrist_FR}

\subsection{À propos}

Le \emph{OR} usuel est parfois appelé \emph{OU inclusif} (ou même \emph{IOR}), par opposition
au \emph{OU exclusif}.
C'est ainsi dans la bibliothèque Python \emph{operator}: il y est appelé \emph{operator.ior}.

}

\EN{\input{fundamentals/AND_OR_XOR_as_MOV_EN}}
\ES{\input{fundamentals/AND_OR_XOR_as_MOV_ES}}
\RU{\input{fundamentals/AND_OR_XOR_as_MOV_RU}}
\FR{\input{fundamentals/AND_OR_XOR_as_MOV_FR}}

\EN{\input{fundamentals/POPCNT_EN}}\RU{\input{fundamentals/POPCNT_RU}}%
\FR{\input{fundamentals/POPCNT_FR}}

\EN{\input{fundamentals/endianness_EN}}
\ES{\input{fundamentals/endianness_ES}}
\RU{\input{fundamentals/endianness_RU}}
\FR{\input{fundamentals/endianness_FR}}
\IT{\input{fundamentals/endianness_IT}}

\EN{\input{fundamentals/memory_EN}}
\ES{\input{fundamentals/memory_ES}}
\RU{\input{fundamentals/memory_RU}}
\FR{\input{fundamentals/memory_FR}}
\IT{\input{fundamentals/memory_IT}}

\EN{\input{fundamentals/CPU_EN}}
\ES{\input{fundamentals/CPU_ES}}
\RU{\input{fundamentals/CPU_RU}}
\FR{\input{fundamentals/CPU_FR}}
\IT{\input{fundamentals/CPU_IT}}

\EN{\input{fundamentals/hash_EN}}
\ES{\input{fundamentals/hash_ES}}
\RU{\input{fundamentals/hash_RU}}
\FR{\input{fundamentals/hash_FR}}


\chapter{\EN{Slightly more advanced examples}\RU{Более сложные примеры}\DE{Fortgeschrittenere Beispiele}\FR{Exemples un peu plus avancés}}

% sections here:
\renewcommand{\CURPATH}{advanced/020_zero_register}
\EN{% TODO translate
\mysection{Breaking simple executable cryptor}

I've got an executable file which is encrypted by relatively simple encryption.
\href{\GitHubBlobMasterURL/examples/simple_exec_crypto/files/cipher.bin}{Here is it} (only executable section is left here).

First, all encryption function does is just adds number of position in buffer to the byte.
Here is how this can be encoded in Python:

\begin{lstlisting}[caption=Python script,style=custompy]
#!/usr/bin/env python
def e(i, k):
    return chr ((ord(i)+k) % 256)

def encrypt(buf):
    return e(buf[0], 0)+ e(buf[1], 1)+ e(buf[2], 2) + e(buf[3], 3)+ e(buf[4], 4)+ e(buf[5], 5)+ e(buf[6], 6)+ e(buf[7], 7)+
           e(buf[8], 8)+ e(buf[9], 9)+ e(buf[10], 10)+ e(buf[11], 11)+ e(buf[12], 12)+ e(buf[13], 13)+ e(buf[14], 14)+ e(buf[15], 15)
\end{lstlisting}

Hence, if you encrypt buffer with 16 zeros, you'll get \emph{0, 1, 2, 3 ... 12, 13, 14, 15}.

\myindex{Propagating Cipher Block Chaining}
Propagating Cipher Block Chaining (PCBC) is also used, here is how it works:

\begin{figure}[H]
\centering
\myincludegraphics{examples/simple_exec_crypto/601px-PCBC_encryption.png}
\caption{Propagating Cipher Block Chaining encryption (image is taken from Wikipedia article)}
\end{figure}

The problem is that it's too boring to recover IV (Initialization Vector) each time.
Brute-force is also not an option, because IV is too long (16 bytes).
Let's see, if it's possible to recover IV for arbitrary encrypted executable file?

Let's try simple frequency analysis.
This is 32-bit x86 executable code, so let's gather statistics about most frequent bytes and opcodes.
I tried huge oracle.exe file from Oracle RDBMS version 11.2 for windows x86 and I've found that the most frequent byte (no surprise) is zero (~10\%).
The next most frequent byte is (again, no surprise) 0xFF (~5\%).
The next is 0x8B (~5\%).

\myindex{x86!\Instructions!MOV}
0x8B is opcode for \INS{MOV}, this is indeed one of the most busy x86 instructions.
Now what about popularity of zero byte?
If compiler needs to encode value bigger than 127, it has to use 32-bit displacement instead of 8-bit one, but large values are very rare,
so it is padded by zeros.
\myindex{x86!\Instructions!LEA}
\myindex{x86!\Instructions!PUSH}
\myindex{x86!\Instructions!CALL}
This is at least in \INS{LEA}, \INS{MOV}, \INS{PUSH}, \INS{CALL}.

For example:

\begin{lstlisting}[style=customasmx86]
8D B0 28 01 00 00                 lea     esi, [eax+128h]
8D BF 40 38 00 00                 lea     edi, [edi+3840h]
\end{lstlisting}

Displacements bigger than 127 are very popular, but they are rarely exceeds 0x10000
(indeed, such large memory buffers/structures are also rare).

Same story with \INS{MOV}, large constants are rare, the most heavily used are 0, 1, 10, 100, $2^n$, and so on.
Compiler has to pad small constants by zeros to represent them as 32-bit values:

\begin{lstlisting}[style=customasmx86]
BF 02 00 00 00                    mov     edi, 2
BF 01 00 00 00                    mov     edi, 1
\end{lstlisting}

Now about 00 and FF bytes combined: jumps (including conditional) and calls can pass execution flow forward or backwards, but very often,
within the limits of the current executable module.
If forward, displacement is not very big and also padded with zeros.
If backwards, displacement is represented as negative value, so padded with FF bytes.
For example, transfer execution flow forward:

\begin{lstlisting}[style=customasmx86]
E8 43 0C 00 00                    call    _function1
E8 5C 00 00 00                    call    _function2
0F 84 F0 0A 00 00                 jz      loc_4F09A0
0F 84 EB 00 00 00                 jz      loc_4EFBB8
\end{lstlisting}

Backwards:

\begin{lstlisting}[style=customasmx86]
E8 79 0C FE FF                    call    _function1
E8 F4 16 FF FF                    call    _function2
0F 84 F8 FB FF FF                 jz      loc_8212BC
0F 84 06 FD FF FF                 jz      loc_FF1E7D
\end{lstlisting}

FF byte is also very often occurred in negative displacements like these:

\begin{lstlisting}[style=customasmx86]
8D 85 1E FF FF FF                 lea     eax, [ebp-0E2h]
8D 95 F8 5C FF FF                 lea     edx, [ebp-0A308h]
\end{lstlisting}

So far so good. Now we have to try various 16-byte keys, decrypt executable section and measure how often 00, FF and 8B bytes are occurred.
Let's also keep in sight how PCBC decryption works:

\begin{figure}[H]
\centering
\myincludegraphics{examples/simple_exec_crypto/640px-PCBC_decryption.png}
\caption{Propagating Cipher Block Chaining decryption (image is taken from Wikipedia article)}
\end{figure}

The good news is that we don't really have to decrypt whole piece of data, but only slice by slice, this is exactly how I did in my previous example: \myref{XOR_mask_2}.

Now I'm trying all possible bytes (0..255) for each byte in key and just pick the byte producing maximal amount of 00/FF/8B bytes in a decrypted slice:

\begin{lstlisting}[style=custompy]
#!/usr/bin/env python
import sys, hexdump, array, string, operator

KEY_LEN=16

def chunks(l, n):
    # split n by l-byte chunks
    # https://stackoverflow.com/q/312443
    n = max(1, n)
    return [l[i:i + n] for i in range(0, len(l), n)]

def read_file(fname):
    file=open(fname, mode='rb')
    content=file.read()
    file.close()
    return content

def decrypt_byte (c, key):
    return chr((ord(c)-key) % 256)

def XOR_PCBC_step (IV, buf, k):
    prev=IV
    rt=""
    for c in buf:
	new_c=decrypt_byte(c, k)
        plain=chr(ord(new_c)^ord(prev))
	prev=chr(ord(c)^ord(plain))
	rt=rt+plain
    return rt

each_Nth_byte=[""]*KEY_LEN

content=read_file(sys.argv[1])
# split input by 16-byte chunks:
all_chunks=chunks(content, KEY_LEN)
for c in all_chunks:
    for i in range(KEY_LEN):
        each_Nth_byte[i]=each_Nth_byte[i] + c[i]

# try each byte of key
for N in range(KEY_LEN):
    print "N=", N
    stat={}
    for i in range(256):
        tmp_key=chr(i)
	tmp=XOR_PCBC_step(tmp_key,each_Nth_byte[N], N)
        # count 0, FFs and 8Bs in decrypted buffer:
	important_bytes=tmp.count('\x00')+tmp.count('\xFF')+tmp.count('\x8B')
	stat[i]=important_bytes
    sorted_stat = sorted(stat.iteritems(), key=operator.itemgetter(1), reverse=True)
    print sorted_stat[0]
\end{lstlisting}

(Source code can be downloaded \href{\GitHubBlobMasterURL/examples/simple_exec_crypto/files/decrypt.py}{here}.)

I run it and here is a key for which 00/FF/8B bytes presence in decrypted buffer is maximal:

\begin{lstlisting}
N= 0
(147, 1224)
N= 1
(94, 1327)
N= 2
(252, 1223)
N= 3
(218, 1266)
N= 4
(38, 1209)
N= 5
(192, 1378)
N= 6
(199, 1204)
N= 7
(213, 1332)
N= 8
(225, 1251)
N= 9
(112, 1223)
N= 10
(143, 1177)
N= 11
(108, 1286)
N= 12
(10, 1164)
N= 13
(3, 1271)
N= 14
(128, 1253)
N= 15
(232, 1330)
\end{lstlisting}

Let's write decryption utility with the key we got:

\begin{lstlisting}[style=custompy]
#!/usr/bin/env python
import sys, hexdump, array

def xor_strings(s,t):
    # \verb|https://en.wikipedia.org/wiki/XOR_cipher#Example_implementation|
    """xor two strings together"""
    return "".join(chr(ord(a)^ord(b)) for a,b in zip(s,t))

IV=array.array('B', [147, 94, 252, 218, 38, 192, 199, 213, 225, 112, 143, 108, 10, 3, 128, 232]).tostring()

def chunks(l, n):
    n = max(1, n)
    return [l[i:i + n] for i in range(0, len(l), n)]

def read_file(fname):
    file=open(fname, mode='rb')
    content=file.read()
    file.close()
    return content

def decrypt_byte(i, k):
    return chr ((ord(i)-k) % 256)

def decrypt(buf):
    return "".join(decrypt_byte(buf[i], i) for i in range(16))

fout=open(sys.argv[2], mode='wb')

prev=IV
content=read_file(sys.argv[1])
tmp=chunks(content, 16)
for c in tmp:
    new_c=decrypt(c)
    p=xor_strings (new_c, prev)
    prev=xor_strings(c, p)
    fout.write(p)
fout.close()
\end{lstlisting}

(Source code can be downloaded \href{\GitHubBlobMasterURL/examples/simple_exec_crypto/files/decrypt2.py}{here}.)

Let's check resulting file:

\lstinputlisting{examples/simple_exec_crypto/objdump_result.txt}

Yes, this is seems correctly disassembled piece of x86 code.
The whole decryped file can be downloaded \href{\GitHubBlobMasterURL/examples/simple_exec_crypto/files/decrypted.bin}{here}.

In fact, this is text section from regedit.exe from Windows 7.
But this example is based on a real case I encountered, so just executable is different (and key), algorithm is the same.

\subsection{Other ideas to consider}

What if I would fail with such simple frequency analysis?
There are other ideas on how to measure correctness of decrypted/decompressed x86 code:

\begin{itemize}

\item Many modern compilers aligns functions on 0x10 border.
So the space left before is filled with NOPs (0x90) or other NOP instructions with known opcodes: \myref{sec:npad}.

\item Perhaps, the most frequent pattern in any assembly language is function call:\\
\TT{PUSH chain / CALL / ADD ESP, X}.
This sequence can easily detected and found.
I've even gathered statistics about average number of function arguments: \myref{args_stat}.
(Hence, this is average length of PUSH chain.)

\end{itemize}

Read more about incorrectly/correctly disassembled code: \myref{ISA_detect}.
}
\FR{\mysection{Fonction presque vide}
\label{Boolector}
\myindex{Boolector}
\myindex{x86!\Instructions!JMP}

Ceci est un morceau de code réel que j'ai trouvé dans Boolector\footnote{\url{https://boolector.github.io/}}:

\lstinputlisting[style=customc]{patterns/025_almost_empty/boolectormain.c}

Pourquoi quelqu'un ferait-il comme ça?
Je ne sais pas mais mon hypothèse est que \verb|boolector_main()| peut être compilée
dans une sorte de DLL ou bibliothèque dynamique, et appelée depuis une suite de test.
Certainement qu'une suite de test peut préparer les variables argc/argv comme
le ferait \ac{CRT}.

Il est intéressant de voir comment c'est compilé:

\lstinputlisting[caption=GCC 8.2 x64 \NonOptimizing (\assemblyOutput),style=customasmx86]{patterns/025_almost_empty/boolectormain_O0.s}

Ceci est OK, le prologue (non optimisé) déplace inutilement deux arguments,
\INS{CALL}, épilogue, \INS{RET}.
Mais regardons la version optimisée:

\lstinputlisting[caption=GCC 8.2 x64 \Optimizing (\assemblyOutput),style=customasmx86]{patterns/025_almost_empty/boolectormain_O3.s}

Aussi simple que ça: la pile et les registres ne sont pas touchés et \verb|boolector_main()|
a le même ensemble d'arguments.
Donc, tout ce que nous avons à faire est de passer l'exécution à une autre adresse.

Ceci est proche d'une \glslink{thunk function}{fonction thunk}.

Nous verons queelque chose de plus avancé plus tard: \myref{ARM_B_to_printf}, \myref{JMP_instead_of_RET}.
}

\renewcommand{\CURPATH}{advanced/030_dbl_neg}
\EN{% TODO translate
\mysection{Breaking simple executable cryptor}

I've got an executable file which is encrypted by relatively simple encryption.
\href{\GitHubBlobMasterURL/examples/simple_exec_crypto/files/cipher.bin}{Here is it} (only executable section is left here).

First, all encryption function does is just adds number of position in buffer to the byte.
Here is how this can be encoded in Python:

\begin{lstlisting}[caption=Python script,style=custompy]
#!/usr/bin/env python
def e(i, k):
    return chr ((ord(i)+k) % 256)

def encrypt(buf):
    return e(buf[0], 0)+ e(buf[1], 1)+ e(buf[2], 2) + e(buf[3], 3)+ e(buf[4], 4)+ e(buf[5], 5)+ e(buf[6], 6)+ e(buf[7], 7)+
           e(buf[8], 8)+ e(buf[9], 9)+ e(buf[10], 10)+ e(buf[11], 11)+ e(buf[12], 12)+ e(buf[13], 13)+ e(buf[14], 14)+ e(buf[15], 15)
\end{lstlisting}

Hence, if you encrypt buffer with 16 zeros, you'll get \emph{0, 1, 2, 3 ... 12, 13, 14, 15}.

\myindex{Propagating Cipher Block Chaining}
Propagating Cipher Block Chaining (PCBC) is also used, here is how it works:

\begin{figure}[H]
\centering
\myincludegraphics{examples/simple_exec_crypto/601px-PCBC_encryption.png}
\caption{Propagating Cipher Block Chaining encryption (image is taken from Wikipedia article)}
\end{figure}

The problem is that it's too boring to recover IV (Initialization Vector) each time.
Brute-force is also not an option, because IV is too long (16 bytes).
Let's see, if it's possible to recover IV for arbitrary encrypted executable file?

Let's try simple frequency analysis.
This is 32-bit x86 executable code, so let's gather statistics about most frequent bytes and opcodes.
I tried huge oracle.exe file from Oracle RDBMS version 11.2 for windows x86 and I've found that the most frequent byte (no surprise) is zero (~10\%).
The next most frequent byte is (again, no surprise) 0xFF (~5\%).
The next is 0x8B (~5\%).

\myindex{x86!\Instructions!MOV}
0x8B is opcode for \INS{MOV}, this is indeed one of the most busy x86 instructions.
Now what about popularity of zero byte?
If compiler needs to encode value bigger than 127, it has to use 32-bit displacement instead of 8-bit one, but large values are very rare,
so it is padded by zeros.
\myindex{x86!\Instructions!LEA}
\myindex{x86!\Instructions!PUSH}
\myindex{x86!\Instructions!CALL}
This is at least in \INS{LEA}, \INS{MOV}, \INS{PUSH}, \INS{CALL}.

For example:

\begin{lstlisting}[style=customasmx86]
8D B0 28 01 00 00                 lea     esi, [eax+128h]
8D BF 40 38 00 00                 lea     edi, [edi+3840h]
\end{lstlisting}

Displacements bigger than 127 are very popular, but they are rarely exceeds 0x10000
(indeed, such large memory buffers/structures are also rare).

Same story with \INS{MOV}, large constants are rare, the most heavily used are 0, 1, 10, 100, $2^n$, and so on.
Compiler has to pad small constants by zeros to represent them as 32-bit values:

\begin{lstlisting}[style=customasmx86]
BF 02 00 00 00                    mov     edi, 2
BF 01 00 00 00                    mov     edi, 1
\end{lstlisting}

Now about 00 and FF bytes combined: jumps (including conditional) and calls can pass execution flow forward or backwards, but very often,
within the limits of the current executable module.
If forward, displacement is not very big and also padded with zeros.
If backwards, displacement is represented as negative value, so padded with FF bytes.
For example, transfer execution flow forward:

\begin{lstlisting}[style=customasmx86]
E8 43 0C 00 00                    call    _function1
E8 5C 00 00 00                    call    _function2
0F 84 F0 0A 00 00                 jz      loc_4F09A0
0F 84 EB 00 00 00                 jz      loc_4EFBB8
\end{lstlisting}

Backwards:

\begin{lstlisting}[style=customasmx86]
E8 79 0C FE FF                    call    _function1
E8 F4 16 FF FF                    call    _function2
0F 84 F8 FB FF FF                 jz      loc_8212BC
0F 84 06 FD FF FF                 jz      loc_FF1E7D
\end{lstlisting}

FF byte is also very often occurred in negative displacements like these:

\begin{lstlisting}[style=customasmx86]
8D 85 1E FF FF FF                 lea     eax, [ebp-0E2h]
8D 95 F8 5C FF FF                 lea     edx, [ebp-0A308h]
\end{lstlisting}

So far so good. Now we have to try various 16-byte keys, decrypt executable section and measure how often 00, FF and 8B bytes are occurred.
Let's also keep in sight how PCBC decryption works:

\begin{figure}[H]
\centering
\myincludegraphics{examples/simple_exec_crypto/640px-PCBC_decryption.png}
\caption{Propagating Cipher Block Chaining decryption (image is taken from Wikipedia article)}
\end{figure}

The good news is that we don't really have to decrypt whole piece of data, but only slice by slice, this is exactly how I did in my previous example: \myref{XOR_mask_2}.

Now I'm trying all possible bytes (0..255) for each byte in key and just pick the byte producing maximal amount of 00/FF/8B bytes in a decrypted slice:

\begin{lstlisting}[style=custompy]
#!/usr/bin/env python
import sys, hexdump, array, string, operator

KEY_LEN=16

def chunks(l, n):
    # split n by l-byte chunks
    # https://stackoverflow.com/q/312443
    n = max(1, n)
    return [l[i:i + n] for i in range(0, len(l), n)]

def read_file(fname):
    file=open(fname, mode='rb')
    content=file.read()
    file.close()
    return content

def decrypt_byte (c, key):
    return chr((ord(c)-key) % 256)

def XOR_PCBC_step (IV, buf, k):
    prev=IV
    rt=""
    for c in buf:
	new_c=decrypt_byte(c, k)
        plain=chr(ord(new_c)^ord(prev))
	prev=chr(ord(c)^ord(plain))
	rt=rt+plain
    return rt

each_Nth_byte=[""]*KEY_LEN

content=read_file(sys.argv[1])
# split input by 16-byte chunks:
all_chunks=chunks(content, KEY_LEN)
for c in all_chunks:
    for i in range(KEY_LEN):
        each_Nth_byte[i]=each_Nth_byte[i] + c[i]

# try each byte of key
for N in range(KEY_LEN):
    print "N=", N
    stat={}
    for i in range(256):
        tmp_key=chr(i)
	tmp=XOR_PCBC_step(tmp_key,each_Nth_byte[N], N)
        # count 0, FFs and 8Bs in decrypted buffer:
	important_bytes=tmp.count('\x00')+tmp.count('\xFF')+tmp.count('\x8B')
	stat[i]=important_bytes
    sorted_stat = sorted(stat.iteritems(), key=operator.itemgetter(1), reverse=True)
    print sorted_stat[0]
\end{lstlisting}

(Source code can be downloaded \href{\GitHubBlobMasterURL/examples/simple_exec_crypto/files/decrypt.py}{here}.)

I run it and here is a key for which 00/FF/8B bytes presence in decrypted buffer is maximal:

\begin{lstlisting}
N= 0
(147, 1224)
N= 1
(94, 1327)
N= 2
(252, 1223)
N= 3
(218, 1266)
N= 4
(38, 1209)
N= 5
(192, 1378)
N= 6
(199, 1204)
N= 7
(213, 1332)
N= 8
(225, 1251)
N= 9
(112, 1223)
N= 10
(143, 1177)
N= 11
(108, 1286)
N= 12
(10, 1164)
N= 13
(3, 1271)
N= 14
(128, 1253)
N= 15
(232, 1330)
\end{lstlisting}

Let's write decryption utility with the key we got:

\begin{lstlisting}[style=custompy]
#!/usr/bin/env python
import sys, hexdump, array

def xor_strings(s,t):
    # \verb|https://en.wikipedia.org/wiki/XOR_cipher#Example_implementation|
    """xor two strings together"""
    return "".join(chr(ord(a)^ord(b)) for a,b in zip(s,t))

IV=array.array('B', [147, 94, 252, 218, 38, 192, 199, 213, 225, 112, 143, 108, 10, 3, 128, 232]).tostring()

def chunks(l, n):
    n = max(1, n)
    return [l[i:i + n] for i in range(0, len(l), n)]

def read_file(fname):
    file=open(fname, mode='rb')
    content=file.read()
    file.close()
    return content

def decrypt_byte(i, k):
    return chr ((ord(i)-k) % 256)

def decrypt(buf):
    return "".join(decrypt_byte(buf[i], i) for i in range(16))

fout=open(sys.argv[2], mode='wb')

prev=IV
content=read_file(sys.argv[1])
tmp=chunks(content, 16)
for c in tmp:
    new_c=decrypt(c)
    p=xor_strings (new_c, prev)
    prev=xor_strings(c, p)
    fout.write(p)
fout.close()
\end{lstlisting}

(Source code can be downloaded \href{\GitHubBlobMasterURL/examples/simple_exec_crypto/files/decrypt2.py}{here}.)

Let's check resulting file:

\lstinputlisting{examples/simple_exec_crypto/objdump_result.txt}

Yes, this is seems correctly disassembled piece of x86 code.
The whole decryped file can be downloaded \href{\GitHubBlobMasterURL/examples/simple_exec_crypto/files/decrypted.bin}{here}.

In fact, this is text section from regedit.exe from Windows 7.
But this example is based on a real case I encountered, so just executable is different (and key), algorithm is the same.

\subsection{Other ideas to consider}

What if I would fail with such simple frequency analysis?
There are other ideas on how to measure correctness of decrypted/decompressed x86 code:

\begin{itemize}

\item Many modern compilers aligns functions on 0x10 border.
So the space left before is filled with NOPs (0x90) or other NOP instructions with known opcodes: \myref{sec:npad}.

\item Perhaps, the most frequent pattern in any assembly language is function call:\\
\TT{PUSH chain / CALL / ADD ESP, X}.
This sequence can easily detected and found.
I've even gathered statistics about average number of function arguments: \myref{args_stat}.
(Hence, this is average length of PUSH chain.)

\end{itemize}

Read more about incorrectly/correctly disassembled code: \myref{ISA_detect}.
}
\RU{\subsection{Простое шифрование используя XOR-маску}
\label{XOR_mask_1}

Я нашел одну старую игру в стиле interactive fiction в архиве \emph{if-archive}\footnote{\url{http://www.ifarchive.org/}}:

\begin{lstlisting}
The New Castle v3.5 - Text/Adventure Game
in the style of the original Infocom (tm)
type games, Zork, Collosal Cave (Adventure),
etc.  Can you solve the mystery of the
abandoned castle?
Shareware from Software Customization.
Software Customization [ASP] Version 3.5 Feb. 2000
\end{lstlisting}

Можно скачать здесь: \url{\GitHubBlobMasterURL/ff/XOR/mask_1/files/newcastle.tgz}.

Там внутри есть файл (с названием \emph{castle.dbf}), который явно зашифрован, но не настоящим криптоалгоритмом,
и оне сжат, это что-то куда проще.
Я бы даже не стал измерять уровень энтропии (\myref{entropy}) этого файла, потому что я итак уверен, что он низкий.
Вот как он выглядит в Midnight Commander:

\begin{figure}[H]
\centering
\myincludegraphics{ff/XOR/mask_1/mc_encrypted.png}
\caption{Зашифрованный файл в Midnight Commander}
\end{figure}

Зашифрованный файл можно скачать здесь:
\url{\GitHubBlobMasterURL/ff/XOR/mask_1/files/castle.dbf.bz2}.

Можно ли расшифровать его без доступа к программе, используя просто этот файл?

Тут явно просматривается повторяющаяся строка. 
Если использовалось простое шифрование с XOR-маской, такие повторяющиеся строки это явное свидетельство,
потому что, вероятно, тут были длинные лакуны с нулевыми байтами, которые, в свою очередь, присутствуют
во мноигих исполняемых файлах, и в остальных бинарных файлах.

\myindex{UNIX!xxd}
Вот дам начала этого файла используя утилиту \emph{xxd} из UNIX:

\lstinputlisting{ff/XOR/mask_1/xxd_result.txt}

Давайте держаться за повторяющуюся строку \TT{iubgv}.
Глядя на этот дамп, мы можем легко увидеть, что период повторений этой строки это 0x51 или 81.
Вероятно, 81 это длина блока?
Длина файла 1658961, и она может быть поделена на 81 без остатка (и тогда там 20481 блоков).

Теперь я буду использовать Mathematica для анализа, есть ли тут повторяющиеся 81-байтные блоки в файле?
Я разделю входной файл на 81-байтные блоки и затем использую ф-цию
\emph{Tally[]}\footnote{\url{https://reference.wolfram.com/language/ref/Tally.html}}
которая просто считает, сколько раз каждый элемент встретился во входном списке.
Вывод Tally не отсортирован, так что я также добавлю ф-цию \emph{Sort[]} для сортировки его по кол-ву вхождений
в нисходящем порядке.

\begin{lstlisting}[style=custommath]
input = BinaryReadList["/home/dennis/.../castle.dbf"];

blocks = Partition[input, 81];

stat = Sort[Tally[blocks], #1[[2]] > #2[[2]] &]
\end{lstlisting}

И вот вывод:

\begin{lstlisting}[style=custommath]
{{{80, 103, 2, 116, 113, 102, 118, 25, 99, 8, 19, 23, 116, 125, 107, 
   25, 99, 109, 114, 102, 14, 121, 115, 31, 9, 117, 113, 111, 5, 4, 
   127, 28, 122, 101, 8, 110, 14, 18, 124, 106, 16, 20, 104, 119, 8, 
   109, 26, 106, 9, 97, 13, 99, 15, 119, 20, 105, 117, 98, 103, 118, 
   1, 126, 29, 97, 122, 17, 15, 114, 110, 3, 5, 125, 125, 99, 126, 
   119, 102, 30, 122, 2, 117}, 1739}, 
{{80, 100, 2, 116, 113, 102, 118, 25, 99, 8, 19, 23, 116, 
   125, 107, 25, 99, 109, 114, 102, 14, 121, 115, 31, 9, 117, 113, 
   111, 5, 4, 127, 28, 122, 101, 8, 110, 14, 18, 124, 106, 16, 20, 
   104, 119, 8, 109, 26, 106, 9, 97, 13, 99, 15, 119, 20, 105, 117, 
   98, 103, 118, 1, 126, 29, 97, 122, 17, 15, 114, 110, 3, 5, 125, 
   125, 99, 126, 119, 102, 30, 122, 2, 117}, 1422}, 
{{80, 101, 2, 116, 113, 102, 118, 25, 99, 8, 19, 23, 116, 
   125, 107, 25, 99, 109, 114, 102, 14, 121, 115, 31, 9, 117, 113, 
   111, 5, 4, 127, 28, 122, 101, 8, 110, 14, 18, 124, 106, 16, 20, 
   104, 119, 8, 109, 26, 106, 9, 97, 13, 99, 15, 119, 20, 105, 117, 
   98, 103, 118, 1, 126, 29, 97, 122, 17, 15, 114, 110, 3, 5, 125, 
   125, 99, 126, 119, 102, 30, 122, 2, 117}, 1012},
{{80, 120, 2, 116, 113, 102, 118, 25, 99, 8, 19, 23, 116, 
   125, 107, 25, 99, 109, 114, 102, 14, 121, 115, 31, 9, 117, 113, 
   111, 5, 4, 127, 28, 122, 101, 8, 110, 14, 18, 124, 106, 16, 20, 
   104, 119, 8, 109, 26, 106, 9, 97, 13, 99, 15, 119, 20, 105, 117, 
   98, 103, 118, 1, 126, 29, 97, 122, 17, 15, 114, 110, 3, 5, 125, 
   125, 99, 126, 119, 102, 30, 122, 2, 117}, 377},

...

{{80, 2, 74, 49, 113, 21, 62, 88, 39, 71, 68, 23, 63, 51, 36, 78, 48, 
   108, 114, 102, 14, 121, 115, 31, 9, 117, 113, 111, 5, 4, 127, 28, 
   122, 101, 8, 110, 14, 18, 124, 106, 16, 20, 104, 119, 8, 109, 26, 
   106, 9, 97, 13, 99, 15, 119, 20, 105, 117, 98, 103, 118, 1, 126, 
   29, 97, 122, 17, 15, 114, 110, 3, 5, 125, 125, 99, 126, 119, 102, 
   30, 122, 2, 117}, 1},
{{80, 1, 74, 59, 113, 45, 56, 86, 52, 91, 19, 64, 60, 60, 63, 
   25, 38, 59, 59, 42, 14, 53, 38, 77, 66, 38, 113, 38, 75, 4, 43, 84,
    63, 101, 64, 43, 79, 64, 40, 57, 16, 91, 46, 119, 69, 40, 84, 117,
    9, 97, 13, 99, 15, 119, 20, 105, 117, 98, 103, 118, 1, 126, 29, 
   97, 122, 17, 15, 114, 110, 3, 5, 125, 125, 99, 126, 119, 102, 30, 
   122, 2, 117}, 1},
{{80, 2, 74, 49, 113, 49, 51, 92, 39, 8, 92, 81, 116, 62, 57, 
   80, 46, 40, 114, 36, 75, 56, 33, 76, 9, 55, 56, 59, 81, 65, 45, 28,
    60, 55, 93, 39, 90, 28, 124, 106, 16, 20, 104, 119, 8, 109, 26, 
   106, 9, 97, 13, 99, 15, 119, 20, 105, 117, 98, 103, 118, 1, 126, 
   29, 97, 122, 17, 15, 114, 110, 3, 5, 125, 125, 99, 126, 119, 102, 
   30, 122, 2, 117}, 1}}
\end{lstlisting}

Вывод Tally это список пар, каждая пара это 81-байтный блок и количество раз, сколько он встретился в файле.
Мы видим, что наиболее частно встречающийся блок это первый, он встретился 1739 раз.
Второй встретился 1422 раза. Есть и другие: 1012 раза, 377 раз, итд.
81-байтные блоки, встреченные лишь один раз, находятся в конце вывода.

Попробуем сравнить эти блоки. Первый и второй.
Есть ли в Mathematica ф-ция для сравнения списков/массивов?
Наверняка есть, но в педагогических целях, я буду использоват операцию XOR для сравнения.
Действительно: если байты во входных массивах равны друг другу, результат операции XOR это 0.
Если не равны, результат будет ненулевой.

Сравним первый блок (встречается 1739 раз) и второй (встречается 1422 раз):

\begin{lstlisting}[style=custommath]
In[]:= BitXor[stat[[1]][[1]], stat[[2]][[1]]]
Out[]= {0, 3, 0, 0, 0, 0, 0, 0, 0, 0, 0, 0, 0, 0, 0, 0, 0, 0, 0, \
0, 0, 0, 0, 0, 0, 0, 0, 0, 0, 0, 0, 0, 0, 0, 0, 0, 0, 0, 0, 0, 0, 0, \
0, 0, 0, 0, 0, 0, 0, 0, 0, 0, 0, 0, 0, 0, 0, 0, 0, 0, 0, 0, 0, 0, 0, \
0, 0, 0, 0, 0, 0, 0, 0, 0, 0, 0, 0, 0, 0, 0, 0}
\end{lstlisting}

Они отличаются только вторым байтом.

Сравним второй блок (встречается 1422 раза) и третий (встречается 1012 раз):

\begin{lstlisting}[style=custommath]
In[]:= BitXor[stat[[2]][[1]], stat[[3]][[1]]]
Out[]= {0, 1, 0, 0, 0, 0, 0, 0, 0, 0, 0, 0, 0, 0, 0, 0, 0, 0, 0, \
0, 0, 0, 0, 0, 0, 0, 0, 0, 0, 0, 0, 0, 0, 0, 0, 0, 0, 0, 0, 0, 0, 0, \
0, 0, 0, 0, 0, 0, 0, 0, 0, 0, 0, 0, 0, 0, 0, 0, 0, 0, 0, 0, 0, 0, 0, \
0, 0, 0, 0, 0, 0, 0, 0, 0, 0, 0, 0, 0, 0, 0, 0}
\end{lstlisting}

Они тоже отличаются только вторым байтом.

Так или иначе, попробуем использовать самый встречающийся блок как XOR-ключ и попробуем расшифровать первые 4 81-байтных
блока в файле:

\begin{lstlisting}[style=custommath]
In[]:= key = stat[[1]][[1]]
Out[]= {80, 103, 2, 116, 113, 102, 118, 25, 99, 8, 19, 23, 116, \
125, 107, 25, 99, 109, 114, 102, 14, 121, 115, 31, 9, 117, 113, 111, \
5, 4, 127, 28, 122, 101, 8, 110, 14, 18, 124, 106, 16, 20, 104, 119, \
8, 109, 26, 106, 9, 97, 13, 99, 15, 119, 20, 105, 117, 98, 103, 118, \
1, 126, 29, 97, 122, 17, 15, 114, 110, 3, 5, 125, 125, 99, 126, 119, \
102, 30, 122, 2, 117}

In[]:= ToASCII[val_] := If[val == 0, " ", FromCharacterCode[val, "PrintableASCII"]]

In[]:= DecryptBlockASCII[blk_] := Map[ToASCII[#] &, BitXor[key, blk]]

In[]:= DecryptBlockASCII[blocks[[1]]]
Out[]= {" ", " ", " ", " ", " ", " ", " ", " ", " ", " ", " ", " \
", " ", " ", " ", " ", " ", " ", " ", " ", " ", " ", " ", " ", " ", " \
", " ", " ", " ", " ", " ", " ", " ", " ", " ", " ", " ", " ", " ", " \
", " ", " ", " ", " ", " ", " ", " ", " ", " ", " ", " ", " ", " ", " \
", " ", " ", " ", " ", " ", " ", " ", " ", " ", " ", " ", " ", " ", " \
", " ", " ", " ", " ", " ", " ", " ", " ", " ", " ", " ", " ", " "}

In[]:= DecryptBlockASCII[blocks[[2]]]
Out[]= {" ", "e", "H", "E", " ", "W", "E", "E", "D", " ", "O", \
"F", " ", "C", "R", "I", "M", "E", " ", "B", "E", "A", "R", "S", " ", \
"B", "I", "T", "T", "E", "R", " ", "F", "R", "U", "I", "T", "?", \
" ", " ", " ", " ", " ", " ", " ", " ", " ", " ", " ", " ", " ", " ", \
" ", " ", " ", " ", " ", " ", " ", " ", " ", " ", " ", " ", " ", " ", \
" ", " ", " ", " ", " ", " ", " ", " ", " ", " ", " ", " ", " ", " ", \
" "}

In[]:= DecryptBlockASCII[blocks[[3]]]
Out[]= {" ", "?", " ", " ", " ", " ", " ", " ", " ", " ", " \
", " ", " ", " ", " ", " ", " ", " ", " ", " ", " ", " ", " ", " ", " \
", " ", " ", " ", " ", " ", " ", " ", " ", " ", " ", " ", " ", " ", " \
", " ", " ", " ", " ", " ", " ", " ", " ", " ", " ", " ", " ", " ", " \
", " ", " ", " ", " ", " ", " ", " ", " ", " ", " ", " ", " ", " ", " \
", " ", " ", " ", " ", " ", " ", " ", " ", " ", " ", " ", " ", " ", " \
"}

In[]:= DecryptBlockASCII[blocks[[4]]]
Out[]= {" ", "f", "H", "O", " ", "K", "N", "O", "W", "S", " ", \
"W", "H", "A", "T", " ", "E", "V", "I", "L", " ", "L", "U", "R", "K", \
"S", " ", "I", "N", " ", "T", "H", "E", " ", "H", "E", "A", "R", "T", \
"S", " ", "O", "F", " ", "M", "E", "N", "?", " ", " ", " ", " ", \
" ", " ", " ", " ", " ", " ", " ", " ", " ", " ", " ", " ", " ", " ", \
" ", " ", " ", " ", " ", " ", " ", " ", " ", " ", " ", " ", " ", " ", \
" "}
\end{lstlisting}

(Я заменил непечатаемые символы на \q{?}.)

Мы видим что первый и третий блоки пустые (или почти пустые),
но второй и четвертый имеют ясно различимые английские слова/фразы.
Похоже что наше предположение насчет ключа верно (как минимум частично).
Это означает, что самый встречающийся 81-байтный блок в файле находится в местах лакун с нулевыми байтами
или что-то в этом роде.

Попробуем расшифровать весь файл:

\begin{lstlisting}[style=custommath]
DecryptBlock[blk_] := BitXor[key, blk]

decrypted = Map[DecryptBlock[#] &, blocks];

BinaryWrite["/home/dennis/.../tmp", Flatten[decrypted]]

Close["/home/dennis/.../tmp"]
\end{lstlisting}

\begin{figure}[H]
\centering
\myincludegraphics{ff/XOR/mask_1/mc_decrypted1.png}
\caption{Расшифрованный файл в Midnight Commander, первая попытка}
\end{figure}

Выглядит как английские фразы для какой-то игры, но что-то не так.
Прежде всего, регистр инвертирован: фразы и некоторые слова начинаются со строчных букв,
в то время как остальные буквы заглавные.
Также, некоторые фразы начинаются с не тех букв.
Посмотрите на самую первую фразу: \q{eHE WEED OF CRIME BEARS BITTER FRUIT}.
Что такое \q{eHE}? Разве не \q{tHE} тут должно быть?
Возможно ли что наш ключ для дешифрования имеет неверный байт в этом месте?

Посмотрим снова на второй блок в файле, на ключ и на результат дешифрования:

\begin{lstlisting}[style=custommath]
In[]:= blocks[[2]]
Out[]= {80, 2, 74, 49, 113, 49, 51, 92, 39, 8, 92, 81, 116, 62, \
57, 80, 46, 40, 114, 36, 75, 56, 33, 76, 9, 55, 56, 59, 81, 65, 45, \
28, 60, 55, 93, 39, 90, 28, 124, 106, 16, 20, 104, 119, 8, 109, 26, \
106, 9, 97, 13, 99, 15, 119, 20, 105, 117, 98, 103, 118, 1, 126, 29, \
97, 122, 17, 15, 114, 110, 3, 5, 125, 125, 99, 126, 119, 102, 30, \
122, 2, 117}

In[]:= key
Out[]= {80, 103, 2, 116, 113, 102, 118, 25, 99, 8, 19, 23, 116, \
125, 107, 25, 99, 109, 114, 102, 14, 121, 115, 31, 9, 117, 113, 111, \
5, 4, 127, 28, 122, 101, 8, 110, 14, 18, 124, 106, 16, 20, 104, 119, \
8, 109, 26, 106, 9, 97, 13, 99, 15, 119, 20, 105, 117, 98, 103, 118, \
1, 126, 29, 97, 122, 17, 15, 114, 110, 3, 5, 125, 125, 99, 126, 119, \
102, 30, 122, 2, 117}

In[]:= BitXor[key, blocks[[2]]]
Out[]= {0, 101, 72, 69, 0, 87, 69, 69, 68, 0, 79, 70, 0, 67, 82, \
73, 77, 69, 0, 66, 69, 65, 82, 83, 0, 66, 73, 84, 84, 69, 82, 0, 70, \
82, 85, 73, 84, 14, 0, 0, 0, 0, 0, 0, 0, 0, 0, 0, 0, 0, 0, 0, 0, 0, \
0, 0, 0, 0, 0, 0, 0, 0, 0, 0, 0, 0, 0, 0, 0, 0, 0, 0, 0, 0, 0, 0, 0, \
0, 0, 0, 0}
\end{lstlisting}

Зашифрованный байт это 2, байт из ключа это 103, $2 \oplus 103=101$ и 101 это ASCII-код символа \q{e}.
Чему должен равнятся этот байт ключа, чтобы ASCII-код был 116 (для символа  \q{t})?
$2 \oplus 116=118$, присвоим 118 второму байту в ключе \dots

\begin{lstlisting}[style=custommath]
key = {80, 118, 2, 116, 113, 102, 118, 25, 99, 8, 19, 23, 116, 125, 
  107, 25, 99, 109, 114, 102, 14, 121, 115, 31, 9, 117, 113, 111, 5, 
  4, 127, 28, 122, 101, 8, 110, 14, 18, 124, 106, 16, 20, 104, 119, 8,
   109, 26, 106, 9, 97, 13, 99, 15, 119, 20, 105, 117, 98, 103, 118, 
  1, 126, 29, 97, 122, 17, 15, 114, 110, 3, 5, 125, 125, 99, 126, 119,
   102, 30, 122, 2, 117}
\end{lstlisting}

\dots и снова дешифруем весь файл.

\begin{figure}[H]
\centering
\myincludegraphics{ff/XOR/mask_1/mc_decrypted2.png}
\caption{Дешифрованный файл в Midnight Commander, вторая попытка}
\end{figure}

Ух ты, теперь грамматика корректна, и все фразы начинаются с корректных букв.
Но все таки, регистр подозрителен.
С чего бы разработчику игры записывать их в такой манере?
Может быть наш ключ все еще неправилен?

% TODO ASCII table somewhere in the book
Изучая таблицу ASCII мы можем заметить что ASCII-коды для букв в верхнем и нижнем регистре отличаются только на один бит
(6-й бит, если считать с первого, 0b100000):

\begin{figure}[H]
\centering
\includegraphics[width=0.7\textwidth]{ascii.png}
\caption{7-битная таблица \ac{ASCII} в Emacs}
\end{figure}

6-й бит, выставленный в нулевом байте, В десятичном виде это будет 32.
Но 32 это ASCII-код пробела!

Действительно, можно менять регистр просто применяя XOR к ASCII-коду, с 32 (больше об этом: \myref{toupper_bit}).

Возможно ли, что пустые лакуны в файле это не нулевые байты, а скорее содержащие пробелы?
Еще раз модифицируем наш XOR-ключ (я про-XOR-ю каждый байт ключа с 32):

\begin{lstlisting}[style=custommath]
(* "32" это скаляр, и "key" это вектор, но это OK *)

In[]:= key3 = BitXor[32, key]
Out[]= {112, 86, 34, 84, 81, 70, 86, 57, 67, 40, 51, 55, 84, 93, 75, \
57, 67, 77, 82, 70, 46, 89, 83, 63, 41, 85, 81, 79, 37, 36, 95, 60, \
90, 69, 40, 78, 46, 50, 92, 74, 48, 52, 72, 87, 40, 77, 58, 74, 41, \
65, 45, 67, 47, 87, 52, 73, 85, 66, 71, 86, 33, 94, 61, 65, 90, 49, \
47, 82, 78, 35, 37, 93, 93, 67, 94, 87, 70, 62, 90, 34, 85}

In[]:= DecryptBlock[blk_] := BitXor[key3, blk]
\end{lstlisting}

И снова дешифруем входной файл:

\begin{figure}[H]
\centering
\myincludegraphics{ff/XOR/mask_1/mc_decrypted.png}
\caption{Дешифрованный файл в Midnight Commander, последняя попытка}
\end{figure}

(Расшифрованный файл доступен здесь:
\url{\GitHubBlobMasterURL/ff/XOR/mask_1/files/decrypted.dat.bz2}.)

Несомненно, это корректный исходный файл.
Да, и мы видим числа в начале каждого блока. Должно быть это и есть источник некорректного XOR-ключа.
Как выходит, самый встречающийся 81-байтный блок в файле это блок заполненный пробелами и содержащий символ \q{1} на месте
второго байта.
Действительно, как-то так получилось что многие блоки здесь перемежаются с этим блоком.
Может быть это что-то вроде выравнивания (padding) для коротких фраз/сообщений?
Другой часто встречающийся 81-байтный блок также заполнен пробелами, но с другой цифрой, следовательно,
они отличаются только вторым байтом.

Вот и всё! Теперь мы можем написать утилиту для зашифрования файла назад, и, может быть, модифицировать его перед этим

Файл для Mathematica можно скачать здесь:\\
\url{\GitHubBlobMasterURL/ff/XOR/mask_1/files/XOR_mask_1.nb}.

Итог: XOR-шифрование не надежно вообще. Вероятно, разработчик игры хотел просто скрыть внутренности игры от игрока,
ничего более серьезного.
Все же, шифрование вроде этого крайне популярно вследствии его простоты, так что многие реверс инженеры обычно хорошо
с этим знакомы.

}
\FR{\mysection{Fonction presque vide}
\label{Boolector}
\myindex{Boolector}
\myindex{x86!\Instructions!JMP}

Ceci est un morceau de code réel que j'ai trouvé dans Boolector\footnote{\url{https://boolector.github.io/}}:

\lstinputlisting[style=customc]{patterns/025_almost_empty/boolectormain.c}

Pourquoi quelqu'un ferait-il comme ça?
Je ne sais pas mais mon hypothèse est que \verb|boolector_main()| peut être compilée
dans une sorte de DLL ou bibliothèque dynamique, et appelée depuis une suite de test.
Certainement qu'une suite de test peut préparer les variables argc/argv comme
le ferait \ac{CRT}.

Il est intéressant de voir comment c'est compilé:

\lstinputlisting[caption=GCC 8.2 x64 \NonOptimizing (\assemblyOutput),style=customasmx86]{patterns/025_almost_empty/boolectormain_O0.s}

Ceci est OK, le prologue (non optimisé) déplace inutilement deux arguments,
\INS{CALL}, épilogue, \INS{RET}.
Mais regardons la version optimisée:

\lstinputlisting[caption=GCC 8.2 x64 \Optimizing (\assemblyOutput),style=customasmx86]{patterns/025_almost_empty/boolectormain_O3.s}

Aussi simple que ça: la pile et les registres ne sont pas touchés et \verb|boolector_main()|
a le même ensemble d'arguments.
Donc, tout ce que nous avons à faire est de passer l'exécution à une autre adresse.

Ceci est proche d'une \glslink{thunk function}{fonction thunk}.

Nous verons queelque chose de plus avancé plus tard: \myref{ARM_B_to_printf}, \myref{JMP_instead_of_RET}.
}

\renewcommand{\CURPATH}{advanced/040_const}
\EN{% TODO translate
\mysection{Breaking simple executable cryptor}

I've got an executable file which is encrypted by relatively simple encryption.
\href{\GitHubBlobMasterURL/examples/simple_exec_crypto/files/cipher.bin}{Here is it} (only executable section is left here).

First, all encryption function does is just adds number of position in buffer to the byte.
Here is how this can be encoded in Python:

\begin{lstlisting}[caption=Python script,style=custompy]
#!/usr/bin/env python
def e(i, k):
    return chr ((ord(i)+k) % 256)

def encrypt(buf):
    return e(buf[0], 0)+ e(buf[1], 1)+ e(buf[2], 2) + e(buf[3], 3)+ e(buf[4], 4)+ e(buf[5], 5)+ e(buf[6], 6)+ e(buf[7], 7)+
           e(buf[8], 8)+ e(buf[9], 9)+ e(buf[10], 10)+ e(buf[11], 11)+ e(buf[12], 12)+ e(buf[13], 13)+ e(buf[14], 14)+ e(buf[15], 15)
\end{lstlisting}

Hence, if you encrypt buffer with 16 zeros, you'll get \emph{0, 1, 2, 3 ... 12, 13, 14, 15}.

\myindex{Propagating Cipher Block Chaining}
Propagating Cipher Block Chaining (PCBC) is also used, here is how it works:

\begin{figure}[H]
\centering
\myincludegraphics{examples/simple_exec_crypto/601px-PCBC_encryption.png}
\caption{Propagating Cipher Block Chaining encryption (image is taken from Wikipedia article)}
\end{figure}

The problem is that it's too boring to recover IV (Initialization Vector) each time.
Brute-force is also not an option, because IV is too long (16 bytes).
Let's see, if it's possible to recover IV for arbitrary encrypted executable file?

Let's try simple frequency analysis.
This is 32-bit x86 executable code, so let's gather statistics about most frequent bytes and opcodes.
I tried huge oracle.exe file from Oracle RDBMS version 11.2 for windows x86 and I've found that the most frequent byte (no surprise) is zero (~10\%).
The next most frequent byte is (again, no surprise) 0xFF (~5\%).
The next is 0x8B (~5\%).

\myindex{x86!\Instructions!MOV}
0x8B is opcode for \INS{MOV}, this is indeed one of the most busy x86 instructions.
Now what about popularity of zero byte?
If compiler needs to encode value bigger than 127, it has to use 32-bit displacement instead of 8-bit one, but large values are very rare,
so it is padded by zeros.
\myindex{x86!\Instructions!LEA}
\myindex{x86!\Instructions!PUSH}
\myindex{x86!\Instructions!CALL}
This is at least in \INS{LEA}, \INS{MOV}, \INS{PUSH}, \INS{CALL}.

For example:

\begin{lstlisting}[style=customasmx86]
8D B0 28 01 00 00                 lea     esi, [eax+128h]
8D BF 40 38 00 00                 lea     edi, [edi+3840h]
\end{lstlisting}

Displacements bigger than 127 are very popular, but they are rarely exceeds 0x10000
(indeed, such large memory buffers/structures are also rare).

Same story with \INS{MOV}, large constants are rare, the most heavily used are 0, 1, 10, 100, $2^n$, and so on.
Compiler has to pad small constants by zeros to represent them as 32-bit values:

\begin{lstlisting}[style=customasmx86]
BF 02 00 00 00                    mov     edi, 2
BF 01 00 00 00                    mov     edi, 1
\end{lstlisting}

Now about 00 and FF bytes combined: jumps (including conditional) and calls can pass execution flow forward or backwards, but very often,
within the limits of the current executable module.
If forward, displacement is not very big and also padded with zeros.
If backwards, displacement is represented as negative value, so padded with FF bytes.
For example, transfer execution flow forward:

\begin{lstlisting}[style=customasmx86]
E8 43 0C 00 00                    call    _function1
E8 5C 00 00 00                    call    _function2
0F 84 F0 0A 00 00                 jz      loc_4F09A0
0F 84 EB 00 00 00                 jz      loc_4EFBB8
\end{lstlisting}

Backwards:

\begin{lstlisting}[style=customasmx86]
E8 79 0C FE FF                    call    _function1
E8 F4 16 FF FF                    call    _function2
0F 84 F8 FB FF FF                 jz      loc_8212BC
0F 84 06 FD FF FF                 jz      loc_FF1E7D
\end{lstlisting}

FF byte is also very often occurred in negative displacements like these:

\begin{lstlisting}[style=customasmx86]
8D 85 1E FF FF FF                 lea     eax, [ebp-0E2h]
8D 95 F8 5C FF FF                 lea     edx, [ebp-0A308h]
\end{lstlisting}

So far so good. Now we have to try various 16-byte keys, decrypt executable section and measure how often 00, FF and 8B bytes are occurred.
Let's also keep in sight how PCBC decryption works:

\begin{figure}[H]
\centering
\myincludegraphics{examples/simple_exec_crypto/640px-PCBC_decryption.png}
\caption{Propagating Cipher Block Chaining decryption (image is taken from Wikipedia article)}
\end{figure}

The good news is that we don't really have to decrypt whole piece of data, but only slice by slice, this is exactly how I did in my previous example: \myref{XOR_mask_2}.

Now I'm trying all possible bytes (0..255) for each byte in key and just pick the byte producing maximal amount of 00/FF/8B bytes in a decrypted slice:

\begin{lstlisting}[style=custompy]
#!/usr/bin/env python
import sys, hexdump, array, string, operator

KEY_LEN=16

def chunks(l, n):
    # split n by l-byte chunks
    # https://stackoverflow.com/q/312443
    n = max(1, n)
    return [l[i:i + n] for i in range(0, len(l), n)]

def read_file(fname):
    file=open(fname, mode='rb')
    content=file.read()
    file.close()
    return content

def decrypt_byte (c, key):
    return chr((ord(c)-key) % 256)

def XOR_PCBC_step (IV, buf, k):
    prev=IV
    rt=""
    for c in buf:
	new_c=decrypt_byte(c, k)
        plain=chr(ord(new_c)^ord(prev))
	prev=chr(ord(c)^ord(plain))
	rt=rt+plain
    return rt

each_Nth_byte=[""]*KEY_LEN

content=read_file(sys.argv[1])
# split input by 16-byte chunks:
all_chunks=chunks(content, KEY_LEN)
for c in all_chunks:
    for i in range(KEY_LEN):
        each_Nth_byte[i]=each_Nth_byte[i] + c[i]

# try each byte of key
for N in range(KEY_LEN):
    print "N=", N
    stat={}
    for i in range(256):
        tmp_key=chr(i)
	tmp=XOR_PCBC_step(tmp_key,each_Nth_byte[N], N)
        # count 0, FFs and 8Bs in decrypted buffer:
	important_bytes=tmp.count('\x00')+tmp.count('\xFF')+tmp.count('\x8B')
	stat[i]=important_bytes
    sorted_stat = sorted(stat.iteritems(), key=operator.itemgetter(1), reverse=True)
    print sorted_stat[0]
\end{lstlisting}

(Source code can be downloaded \href{\GitHubBlobMasterURL/examples/simple_exec_crypto/files/decrypt.py}{here}.)

I run it and here is a key for which 00/FF/8B bytes presence in decrypted buffer is maximal:

\begin{lstlisting}
N= 0
(147, 1224)
N= 1
(94, 1327)
N= 2
(252, 1223)
N= 3
(218, 1266)
N= 4
(38, 1209)
N= 5
(192, 1378)
N= 6
(199, 1204)
N= 7
(213, 1332)
N= 8
(225, 1251)
N= 9
(112, 1223)
N= 10
(143, 1177)
N= 11
(108, 1286)
N= 12
(10, 1164)
N= 13
(3, 1271)
N= 14
(128, 1253)
N= 15
(232, 1330)
\end{lstlisting}

Let's write decryption utility with the key we got:

\begin{lstlisting}[style=custompy]
#!/usr/bin/env python
import sys, hexdump, array

def xor_strings(s,t):
    # \verb|https://en.wikipedia.org/wiki/XOR_cipher#Example_implementation|
    """xor two strings together"""
    return "".join(chr(ord(a)^ord(b)) for a,b in zip(s,t))

IV=array.array('B', [147, 94, 252, 218, 38, 192, 199, 213, 225, 112, 143, 108, 10, 3, 128, 232]).tostring()

def chunks(l, n):
    n = max(1, n)
    return [l[i:i + n] for i in range(0, len(l), n)]

def read_file(fname):
    file=open(fname, mode='rb')
    content=file.read()
    file.close()
    return content

def decrypt_byte(i, k):
    return chr ((ord(i)-k) % 256)

def decrypt(buf):
    return "".join(decrypt_byte(buf[i], i) for i in range(16))

fout=open(sys.argv[2], mode='wb')

prev=IV
content=read_file(sys.argv[1])
tmp=chunks(content, 16)
for c in tmp:
    new_c=decrypt(c)
    p=xor_strings (new_c, prev)
    prev=xor_strings(c, p)
    fout.write(p)
fout.close()
\end{lstlisting}

(Source code can be downloaded \href{\GitHubBlobMasterURL/examples/simple_exec_crypto/files/decrypt2.py}{here}.)

Let's check resulting file:

\lstinputlisting{examples/simple_exec_crypto/objdump_result.txt}

Yes, this is seems correctly disassembled piece of x86 code.
The whole decryped file can be downloaded \href{\GitHubBlobMasterURL/examples/simple_exec_crypto/files/decrypted.bin}{here}.

In fact, this is text section from regedit.exe from Windows 7.
But this example is based on a real case I encountered, so just executable is different (and key), algorithm is the same.

\subsection{Other ideas to consider}

What if I would fail with such simple frequency analysis?
There are other ideas on how to measure correctness of decrypted/decompressed x86 code:

\begin{itemize}

\item Many modern compilers aligns functions on 0x10 border.
So the space left before is filled with NOPs (0x90) or other NOP instructions with known opcodes: \myref{sec:npad}.

\item Perhaps, the most frequent pattern in any assembly language is function call:\\
\TT{PUSH chain / CALL / ADD ESP, X}.
This sequence can easily detected and found.
I've even gathered statistics about average number of function arguments: \myref{args_stat}.
(Hence, this is average length of PUSH chain.)

\end{itemize}

Read more about incorrectly/correctly disassembled code: \myref{ISA_detect}.
}
\RU{\subsection{Простое шифрование используя XOR-маску}
\label{XOR_mask_1}

Я нашел одну старую игру в стиле interactive fiction в архиве \emph{if-archive}\footnote{\url{http://www.ifarchive.org/}}:

\begin{lstlisting}
The New Castle v3.5 - Text/Adventure Game
in the style of the original Infocom (tm)
type games, Zork, Collosal Cave (Adventure),
etc.  Can you solve the mystery of the
abandoned castle?
Shareware from Software Customization.
Software Customization [ASP] Version 3.5 Feb. 2000
\end{lstlisting}

Можно скачать здесь: \url{\GitHubBlobMasterURL/ff/XOR/mask_1/files/newcastle.tgz}.

Там внутри есть файл (с названием \emph{castle.dbf}), который явно зашифрован, но не настоящим криптоалгоритмом,
и оне сжат, это что-то куда проще.
Я бы даже не стал измерять уровень энтропии (\myref{entropy}) этого файла, потому что я итак уверен, что он низкий.
Вот как он выглядит в Midnight Commander:

\begin{figure}[H]
\centering
\myincludegraphics{ff/XOR/mask_1/mc_encrypted.png}
\caption{Зашифрованный файл в Midnight Commander}
\end{figure}

Зашифрованный файл можно скачать здесь:
\url{\GitHubBlobMasterURL/ff/XOR/mask_1/files/castle.dbf.bz2}.

Можно ли расшифровать его без доступа к программе, используя просто этот файл?

Тут явно просматривается повторяющаяся строка. 
Если использовалось простое шифрование с XOR-маской, такие повторяющиеся строки это явное свидетельство,
потому что, вероятно, тут были длинные лакуны с нулевыми байтами, которые, в свою очередь, присутствуют
во мноигих исполняемых файлах, и в остальных бинарных файлах.

\myindex{UNIX!xxd}
Вот дам начала этого файла используя утилиту \emph{xxd} из UNIX:

\lstinputlisting{ff/XOR/mask_1/xxd_result.txt}

Давайте держаться за повторяющуюся строку \TT{iubgv}.
Глядя на этот дамп, мы можем легко увидеть, что период повторений этой строки это 0x51 или 81.
Вероятно, 81 это длина блока?
Длина файла 1658961, и она может быть поделена на 81 без остатка (и тогда там 20481 блоков).

Теперь я буду использовать Mathematica для анализа, есть ли тут повторяющиеся 81-байтные блоки в файле?
Я разделю входной файл на 81-байтные блоки и затем использую ф-цию
\emph{Tally[]}\footnote{\url{https://reference.wolfram.com/language/ref/Tally.html}}
которая просто считает, сколько раз каждый элемент встретился во входном списке.
Вывод Tally не отсортирован, так что я также добавлю ф-цию \emph{Sort[]} для сортировки его по кол-ву вхождений
в нисходящем порядке.

\begin{lstlisting}[style=custommath]
input = BinaryReadList["/home/dennis/.../castle.dbf"];

blocks = Partition[input, 81];

stat = Sort[Tally[blocks], #1[[2]] > #2[[2]] &]
\end{lstlisting}

И вот вывод:

\begin{lstlisting}[style=custommath]
{{{80, 103, 2, 116, 113, 102, 118, 25, 99, 8, 19, 23, 116, 125, 107, 
   25, 99, 109, 114, 102, 14, 121, 115, 31, 9, 117, 113, 111, 5, 4, 
   127, 28, 122, 101, 8, 110, 14, 18, 124, 106, 16, 20, 104, 119, 8, 
   109, 26, 106, 9, 97, 13, 99, 15, 119, 20, 105, 117, 98, 103, 118, 
   1, 126, 29, 97, 122, 17, 15, 114, 110, 3, 5, 125, 125, 99, 126, 
   119, 102, 30, 122, 2, 117}, 1739}, 
{{80, 100, 2, 116, 113, 102, 118, 25, 99, 8, 19, 23, 116, 
   125, 107, 25, 99, 109, 114, 102, 14, 121, 115, 31, 9, 117, 113, 
   111, 5, 4, 127, 28, 122, 101, 8, 110, 14, 18, 124, 106, 16, 20, 
   104, 119, 8, 109, 26, 106, 9, 97, 13, 99, 15, 119, 20, 105, 117, 
   98, 103, 118, 1, 126, 29, 97, 122, 17, 15, 114, 110, 3, 5, 125, 
   125, 99, 126, 119, 102, 30, 122, 2, 117}, 1422}, 
{{80, 101, 2, 116, 113, 102, 118, 25, 99, 8, 19, 23, 116, 
   125, 107, 25, 99, 109, 114, 102, 14, 121, 115, 31, 9, 117, 113, 
   111, 5, 4, 127, 28, 122, 101, 8, 110, 14, 18, 124, 106, 16, 20, 
   104, 119, 8, 109, 26, 106, 9, 97, 13, 99, 15, 119, 20, 105, 117, 
   98, 103, 118, 1, 126, 29, 97, 122, 17, 15, 114, 110, 3, 5, 125, 
   125, 99, 126, 119, 102, 30, 122, 2, 117}, 1012},
{{80, 120, 2, 116, 113, 102, 118, 25, 99, 8, 19, 23, 116, 
   125, 107, 25, 99, 109, 114, 102, 14, 121, 115, 31, 9, 117, 113, 
   111, 5, 4, 127, 28, 122, 101, 8, 110, 14, 18, 124, 106, 16, 20, 
   104, 119, 8, 109, 26, 106, 9, 97, 13, 99, 15, 119, 20, 105, 117, 
   98, 103, 118, 1, 126, 29, 97, 122, 17, 15, 114, 110, 3, 5, 125, 
   125, 99, 126, 119, 102, 30, 122, 2, 117}, 377},

...

{{80, 2, 74, 49, 113, 21, 62, 88, 39, 71, 68, 23, 63, 51, 36, 78, 48, 
   108, 114, 102, 14, 121, 115, 31, 9, 117, 113, 111, 5, 4, 127, 28, 
   122, 101, 8, 110, 14, 18, 124, 106, 16, 20, 104, 119, 8, 109, 26, 
   106, 9, 97, 13, 99, 15, 119, 20, 105, 117, 98, 103, 118, 1, 126, 
   29, 97, 122, 17, 15, 114, 110, 3, 5, 125, 125, 99, 126, 119, 102, 
   30, 122, 2, 117}, 1},
{{80, 1, 74, 59, 113, 45, 56, 86, 52, 91, 19, 64, 60, 60, 63, 
   25, 38, 59, 59, 42, 14, 53, 38, 77, 66, 38, 113, 38, 75, 4, 43, 84,
    63, 101, 64, 43, 79, 64, 40, 57, 16, 91, 46, 119, 69, 40, 84, 117,
    9, 97, 13, 99, 15, 119, 20, 105, 117, 98, 103, 118, 1, 126, 29, 
   97, 122, 17, 15, 114, 110, 3, 5, 125, 125, 99, 126, 119, 102, 30, 
   122, 2, 117}, 1},
{{80, 2, 74, 49, 113, 49, 51, 92, 39, 8, 92, 81, 116, 62, 57, 
   80, 46, 40, 114, 36, 75, 56, 33, 76, 9, 55, 56, 59, 81, 65, 45, 28,
    60, 55, 93, 39, 90, 28, 124, 106, 16, 20, 104, 119, 8, 109, 26, 
   106, 9, 97, 13, 99, 15, 119, 20, 105, 117, 98, 103, 118, 1, 126, 
   29, 97, 122, 17, 15, 114, 110, 3, 5, 125, 125, 99, 126, 119, 102, 
   30, 122, 2, 117}, 1}}
\end{lstlisting}

Вывод Tally это список пар, каждая пара это 81-байтный блок и количество раз, сколько он встретился в файле.
Мы видим, что наиболее частно встречающийся блок это первый, он встретился 1739 раз.
Второй встретился 1422 раза. Есть и другие: 1012 раза, 377 раз, итд.
81-байтные блоки, встреченные лишь один раз, находятся в конце вывода.

Попробуем сравнить эти блоки. Первый и второй.
Есть ли в Mathematica ф-ция для сравнения списков/массивов?
Наверняка есть, но в педагогических целях, я буду использоват операцию XOR для сравнения.
Действительно: если байты во входных массивах равны друг другу, результат операции XOR это 0.
Если не равны, результат будет ненулевой.

Сравним первый блок (встречается 1739 раз) и второй (встречается 1422 раз):

\begin{lstlisting}[style=custommath]
In[]:= BitXor[stat[[1]][[1]], stat[[2]][[1]]]
Out[]= {0, 3, 0, 0, 0, 0, 0, 0, 0, 0, 0, 0, 0, 0, 0, 0, 0, 0, 0, \
0, 0, 0, 0, 0, 0, 0, 0, 0, 0, 0, 0, 0, 0, 0, 0, 0, 0, 0, 0, 0, 0, 0, \
0, 0, 0, 0, 0, 0, 0, 0, 0, 0, 0, 0, 0, 0, 0, 0, 0, 0, 0, 0, 0, 0, 0, \
0, 0, 0, 0, 0, 0, 0, 0, 0, 0, 0, 0, 0, 0, 0, 0}
\end{lstlisting}

Они отличаются только вторым байтом.

Сравним второй блок (встречается 1422 раза) и третий (встречается 1012 раз):

\begin{lstlisting}[style=custommath]
In[]:= BitXor[stat[[2]][[1]], stat[[3]][[1]]]
Out[]= {0, 1, 0, 0, 0, 0, 0, 0, 0, 0, 0, 0, 0, 0, 0, 0, 0, 0, 0, \
0, 0, 0, 0, 0, 0, 0, 0, 0, 0, 0, 0, 0, 0, 0, 0, 0, 0, 0, 0, 0, 0, 0, \
0, 0, 0, 0, 0, 0, 0, 0, 0, 0, 0, 0, 0, 0, 0, 0, 0, 0, 0, 0, 0, 0, 0, \
0, 0, 0, 0, 0, 0, 0, 0, 0, 0, 0, 0, 0, 0, 0, 0}
\end{lstlisting}

Они тоже отличаются только вторым байтом.

Так или иначе, попробуем использовать самый встречающийся блок как XOR-ключ и попробуем расшифровать первые 4 81-байтных
блока в файле:

\begin{lstlisting}[style=custommath]
In[]:= key = stat[[1]][[1]]
Out[]= {80, 103, 2, 116, 113, 102, 118, 25, 99, 8, 19, 23, 116, \
125, 107, 25, 99, 109, 114, 102, 14, 121, 115, 31, 9, 117, 113, 111, \
5, 4, 127, 28, 122, 101, 8, 110, 14, 18, 124, 106, 16, 20, 104, 119, \
8, 109, 26, 106, 9, 97, 13, 99, 15, 119, 20, 105, 117, 98, 103, 118, \
1, 126, 29, 97, 122, 17, 15, 114, 110, 3, 5, 125, 125, 99, 126, 119, \
102, 30, 122, 2, 117}

In[]:= ToASCII[val_] := If[val == 0, " ", FromCharacterCode[val, "PrintableASCII"]]

In[]:= DecryptBlockASCII[blk_] := Map[ToASCII[#] &, BitXor[key, blk]]

In[]:= DecryptBlockASCII[blocks[[1]]]
Out[]= {" ", " ", " ", " ", " ", " ", " ", " ", " ", " ", " ", " \
", " ", " ", " ", " ", " ", " ", " ", " ", " ", " ", " ", " ", " ", " \
", " ", " ", " ", " ", " ", " ", " ", " ", " ", " ", " ", " ", " ", " \
", " ", " ", " ", " ", " ", " ", " ", " ", " ", " ", " ", " ", " ", " \
", " ", " ", " ", " ", " ", " ", " ", " ", " ", " ", " ", " ", " ", " \
", " ", " ", " ", " ", " ", " ", " ", " ", " ", " ", " ", " ", " "}

In[]:= DecryptBlockASCII[blocks[[2]]]
Out[]= {" ", "e", "H", "E", " ", "W", "E", "E", "D", " ", "O", \
"F", " ", "C", "R", "I", "M", "E", " ", "B", "E", "A", "R", "S", " ", \
"B", "I", "T", "T", "E", "R", " ", "F", "R", "U", "I", "T", "?", \
" ", " ", " ", " ", " ", " ", " ", " ", " ", " ", " ", " ", " ", " ", \
" ", " ", " ", " ", " ", " ", " ", " ", " ", " ", " ", " ", " ", " ", \
" ", " ", " ", " ", " ", " ", " ", " ", " ", " ", " ", " ", " ", " ", \
" "}

In[]:= DecryptBlockASCII[blocks[[3]]]
Out[]= {" ", "?", " ", " ", " ", " ", " ", " ", " ", " ", " \
", " ", " ", " ", " ", " ", " ", " ", " ", " ", " ", " ", " ", " ", " \
", " ", " ", " ", " ", " ", " ", " ", " ", " ", " ", " ", " ", " ", " \
", " ", " ", " ", " ", " ", " ", " ", " ", " ", " ", " ", " ", " ", " \
", " ", " ", " ", " ", " ", " ", " ", " ", " ", " ", " ", " ", " ", " \
", " ", " ", " ", " ", " ", " ", " ", " ", " ", " ", " ", " ", " ", " \
"}

In[]:= DecryptBlockASCII[blocks[[4]]]
Out[]= {" ", "f", "H", "O", " ", "K", "N", "O", "W", "S", " ", \
"W", "H", "A", "T", " ", "E", "V", "I", "L", " ", "L", "U", "R", "K", \
"S", " ", "I", "N", " ", "T", "H", "E", " ", "H", "E", "A", "R", "T", \
"S", " ", "O", "F", " ", "M", "E", "N", "?", " ", " ", " ", " ", \
" ", " ", " ", " ", " ", " ", " ", " ", " ", " ", " ", " ", " ", " ", \
" ", " ", " ", " ", " ", " ", " ", " ", " ", " ", " ", " ", " ", " ", \
" "}
\end{lstlisting}

(Я заменил непечатаемые символы на \q{?}.)

Мы видим что первый и третий блоки пустые (или почти пустые),
но второй и четвертый имеют ясно различимые английские слова/фразы.
Похоже что наше предположение насчет ключа верно (как минимум частично).
Это означает, что самый встречающийся 81-байтный блок в файле находится в местах лакун с нулевыми байтами
или что-то в этом роде.

Попробуем расшифровать весь файл:

\begin{lstlisting}[style=custommath]
DecryptBlock[blk_] := BitXor[key, blk]

decrypted = Map[DecryptBlock[#] &, blocks];

BinaryWrite["/home/dennis/.../tmp", Flatten[decrypted]]

Close["/home/dennis/.../tmp"]
\end{lstlisting}

\begin{figure}[H]
\centering
\myincludegraphics{ff/XOR/mask_1/mc_decrypted1.png}
\caption{Расшифрованный файл в Midnight Commander, первая попытка}
\end{figure}

Выглядит как английские фразы для какой-то игры, но что-то не так.
Прежде всего, регистр инвертирован: фразы и некоторые слова начинаются со строчных букв,
в то время как остальные буквы заглавные.
Также, некоторые фразы начинаются с не тех букв.
Посмотрите на самую первую фразу: \q{eHE WEED OF CRIME BEARS BITTER FRUIT}.
Что такое \q{eHE}? Разве не \q{tHE} тут должно быть?
Возможно ли что наш ключ для дешифрования имеет неверный байт в этом месте?

Посмотрим снова на второй блок в файле, на ключ и на результат дешифрования:

\begin{lstlisting}[style=custommath]
In[]:= blocks[[2]]
Out[]= {80, 2, 74, 49, 113, 49, 51, 92, 39, 8, 92, 81, 116, 62, \
57, 80, 46, 40, 114, 36, 75, 56, 33, 76, 9, 55, 56, 59, 81, 65, 45, \
28, 60, 55, 93, 39, 90, 28, 124, 106, 16, 20, 104, 119, 8, 109, 26, \
106, 9, 97, 13, 99, 15, 119, 20, 105, 117, 98, 103, 118, 1, 126, 29, \
97, 122, 17, 15, 114, 110, 3, 5, 125, 125, 99, 126, 119, 102, 30, \
122, 2, 117}

In[]:= key
Out[]= {80, 103, 2, 116, 113, 102, 118, 25, 99, 8, 19, 23, 116, \
125, 107, 25, 99, 109, 114, 102, 14, 121, 115, 31, 9, 117, 113, 111, \
5, 4, 127, 28, 122, 101, 8, 110, 14, 18, 124, 106, 16, 20, 104, 119, \
8, 109, 26, 106, 9, 97, 13, 99, 15, 119, 20, 105, 117, 98, 103, 118, \
1, 126, 29, 97, 122, 17, 15, 114, 110, 3, 5, 125, 125, 99, 126, 119, \
102, 30, 122, 2, 117}

In[]:= BitXor[key, blocks[[2]]]
Out[]= {0, 101, 72, 69, 0, 87, 69, 69, 68, 0, 79, 70, 0, 67, 82, \
73, 77, 69, 0, 66, 69, 65, 82, 83, 0, 66, 73, 84, 84, 69, 82, 0, 70, \
82, 85, 73, 84, 14, 0, 0, 0, 0, 0, 0, 0, 0, 0, 0, 0, 0, 0, 0, 0, 0, \
0, 0, 0, 0, 0, 0, 0, 0, 0, 0, 0, 0, 0, 0, 0, 0, 0, 0, 0, 0, 0, 0, 0, \
0, 0, 0, 0}
\end{lstlisting}

Зашифрованный байт это 2, байт из ключа это 103, $2 \oplus 103=101$ и 101 это ASCII-код символа \q{e}.
Чему должен равнятся этот байт ключа, чтобы ASCII-код был 116 (для символа  \q{t})?
$2 \oplus 116=118$, присвоим 118 второму байту в ключе \dots

\begin{lstlisting}[style=custommath]
key = {80, 118, 2, 116, 113, 102, 118, 25, 99, 8, 19, 23, 116, 125, 
  107, 25, 99, 109, 114, 102, 14, 121, 115, 31, 9, 117, 113, 111, 5, 
  4, 127, 28, 122, 101, 8, 110, 14, 18, 124, 106, 16, 20, 104, 119, 8,
   109, 26, 106, 9, 97, 13, 99, 15, 119, 20, 105, 117, 98, 103, 118, 
  1, 126, 29, 97, 122, 17, 15, 114, 110, 3, 5, 125, 125, 99, 126, 119,
   102, 30, 122, 2, 117}
\end{lstlisting}

\dots и снова дешифруем весь файл.

\begin{figure}[H]
\centering
\myincludegraphics{ff/XOR/mask_1/mc_decrypted2.png}
\caption{Дешифрованный файл в Midnight Commander, вторая попытка}
\end{figure}

Ух ты, теперь грамматика корректна, и все фразы начинаются с корректных букв.
Но все таки, регистр подозрителен.
С чего бы разработчику игры записывать их в такой манере?
Может быть наш ключ все еще неправилен?

% TODO ASCII table somewhere in the book
Изучая таблицу ASCII мы можем заметить что ASCII-коды для букв в верхнем и нижнем регистре отличаются только на один бит
(6-й бит, если считать с первого, 0b100000):

\begin{figure}[H]
\centering
\includegraphics[width=0.7\textwidth]{ascii.png}
\caption{7-битная таблица \ac{ASCII} в Emacs}
\end{figure}

6-й бит, выставленный в нулевом байте, В десятичном виде это будет 32.
Но 32 это ASCII-код пробела!

Действительно, можно менять регистр просто применяя XOR к ASCII-коду, с 32 (больше об этом: \myref{toupper_bit}).

Возможно ли, что пустые лакуны в файле это не нулевые байты, а скорее содержащие пробелы?
Еще раз модифицируем наш XOR-ключ (я про-XOR-ю каждый байт ключа с 32):

\begin{lstlisting}[style=custommath]
(* "32" это скаляр, и "key" это вектор, но это OK *)

In[]:= key3 = BitXor[32, key]
Out[]= {112, 86, 34, 84, 81, 70, 86, 57, 67, 40, 51, 55, 84, 93, 75, \
57, 67, 77, 82, 70, 46, 89, 83, 63, 41, 85, 81, 79, 37, 36, 95, 60, \
90, 69, 40, 78, 46, 50, 92, 74, 48, 52, 72, 87, 40, 77, 58, 74, 41, \
65, 45, 67, 47, 87, 52, 73, 85, 66, 71, 86, 33, 94, 61, 65, 90, 49, \
47, 82, 78, 35, 37, 93, 93, 67, 94, 87, 70, 62, 90, 34, 85}

In[]:= DecryptBlock[blk_] := BitXor[key3, blk]
\end{lstlisting}

И снова дешифруем входной файл:

\begin{figure}[H]
\centering
\myincludegraphics{ff/XOR/mask_1/mc_decrypted.png}
\caption{Дешифрованный файл в Midnight Commander, последняя попытка}
\end{figure}

(Расшифрованный файл доступен здесь:
\url{\GitHubBlobMasterURL/ff/XOR/mask_1/files/decrypted.dat.bz2}.)

Несомненно, это корректный исходный файл.
Да, и мы видим числа в начале каждого блока. Должно быть это и есть источник некорректного XOR-ключа.
Как выходит, самый встречающийся 81-байтный блок в файле это блок заполненный пробелами и содержащий символ \q{1} на месте
второго байта.
Действительно, как-то так получилось что многие блоки здесь перемежаются с этим блоком.
Может быть это что-то вроде выравнивания (padding) для коротких фраз/сообщений?
Другой часто встречающийся 81-байтный блок также заполнен пробелами, но с другой цифрой, следовательно,
они отличаются только вторым байтом.

Вот и всё! Теперь мы можем написать утилиту для зашифрования файла назад, и, может быть, модифицировать его перед этим

Файл для Mathematica можно скачать здесь:\\
\url{\GitHubBlobMasterURL/ff/XOR/mask_1/files/XOR_mask_1.nb}.

Итог: XOR-шифрование не надежно вообще. Вероятно, разработчик игры хотел просто скрыть внутренности игры от игрока,
ничего более серьезного.
Все же, шифрование вроде этого крайне популярно вследствии его простоты, так что многие реверс инженеры обычно хорошо
с этим знакомы.

}
%\DE{\myparagraph{\NonOptimizing MSVC}

MSVC 2010 erzeugt den folgenden Code:

\lstinputlisting[caption=\NonOptimizing MSVC
2010,style=customasmx86]{patterns/12_FPU/3_comparison/x86/MSVC/MSVC_DE.asm}

\myindex{x86!\Instructions!FLD}

Der Befehl \FLD lädt \GTT{\_b} nach \ST{0}.

\label{Czero_etc}
\newcommand{\Czero}{\GTT{C0}\xspace}
\newcommand{\Ctwo}{\GTT{C2}\xspace}
\newcommand{\Cthree}{\GTT{C3}\xspace}
\newcommand{\CThreeBits}{\Cthree/\Ctwo/\Czero}

\myindex{x86!\Instructions!FCOMP}
\FCOMP verlgeicht den Wert in \ST{0} mit dem Wert, der sich in \GTT{\_a}
befindet und setzt die \CThreeBits im FPU Status Register entsprechend.
Das Statusregister ist ein 16-Bit-Register, das den aktueller Zustand der FPU
abbildet.

Nachdem die Bits gesetzt worden sind, nimmer der \FCOMP Befehl auch eine
Variable vom Stack. Dieses Verhalten unterscheidet ihn von \FCOM, der einfach
zwei Werte vergleicht und den Stack unangetastet lässt.

Leider verfügen CPUs vor Intel P6\footnote{Intel P6 ist Pentium Pro, Pentium II,
etc.}über keinerlei bedingte Sprungbefehle, die die \CThreeBits prüfen.

After the bits are set, the \FCOMP instruction also pops one variable from the stack. 
This is what distinguishes it from \FCOM, which is just compares values, leaving the stack in the same state.
Vielleicht ist diese Tatsache historisch begründet (man erinnere sich: die FPU
war früher ein eigener Chip).\\
Moderne CPUs, beginnend mit Intel P6 haben \FCOMI/\FCOMIP/\FUCOMI/\FUCOMIP
Befehle~--welche im Prinzip das gleiche tun, aber die \ZF/\PF/\CF Flags der CPU
verändern können.

\myindex{x86!\Instructions!FNSTSW}
Der \FNSTSW Befehl kopiert das FPU Statusregister nach \AX.
\CThreeBits werden an den Stellen 14/10/8 abgelegt, sie befinden sich im \AX
Register an den gleichen Stellen und sie werden alle in höherwertigen Teil von
\AX{}~---\AH{} abgelegt.

\begin{itemize}
\item Falls in unserem Beispiel $b>a$, dann werden die \CThreeBits Bits wie
folgt gesetzt: 0, 0, 0.
\item Falls $a>b$, dann ist das Bitmuster: 0, 0, 1.
\item Falls $a=b$, dann ist das Bitmuster: 1, 0, 0.
\item

Wenn das Ergebnis (z.B. im Fehlerfall) ungeordnet ist, dann werden die Bits wie
folgt gesetzt: 1,1,1.
\end{itemize}
% TODO: table here?
So werden die \CThreeBits Bits im \AX Register angeordnet:

\input{C3_in_AX}

So werden die \CThreeBits Bits im \AH Register angeordnet:

\input{C3_in_AH}
Nach der Ausführung von \INS{test ah, 5}\footnote{5=101b} werden nur die \Czero
und \Ctwo Bits (an den Stellen 0 und 2) betrachtet, alle übrigen Bits werden
einfach überlesen.

\label{parity_flag}
\myindex{x86!\Registers!\Flags!Parity flag}
Werfen wir nun einen Blick auf ein anderes bemerkenswertes historisches
Überbleibsel: das \emph{parity flag}.

Dieses Flag wird auf 1 gesetzt, falls die Anzahl der Einsen im Ergebnis der
letzten Berechnung gerade ist und auf 1, falls dies nicht der Fall ist.

Schlagen wir in der Wikipedia nach\footnote{\WikipediaParityFlag}:

%TODO Quotation has been translated from English wiki article, since the
% correspondig German article doesn't offer such information.
\begin{framed}
\begin{quotation}
Ein guter Grund das Parity Flag abzufragen, hat tatsächlich gar nichts mit
Parität zu tun. Die FPU hat vier Bedingungsflags (C0 bis C3), aber diese können
nicht direkt abgefragt werden, sondern müssen zunächst in das Flags Register
kopiert werden. Wenn dies geschieht, wird C0 im Carry Flag abgelegt, C2 im
Parity Flag und C3 im Zero Flag.
Das C2 Flag ist gesetzt, wenn z.B. unvergleichbare Fließkommawerte (NaN oder
nicht unterstütztes Format) über der \FUCOM Befehl miteinander verglichen
werden.\textit{(Übersetzung aus der englischen Wikipedia.)}
\end{quotation}
\end{framed}

Wie in der Wikipedia dargestellt wird das Parity Flag manchmal im FPU Code
verwendet; schauen wir uns genauer an wie das funktioniert.

\myindex{x86!\Instructions!JP}
Das \PF Flag wird auf 1 gesetzt, wenn sowohl \Czero als auch \Ctwo beide 0 oder
beide 1 sind. In diesem Fall wird der nachfolgende Sprung \JP(\emph{jump if
PF==1}) ausgeführt.
Wenn wir die Werte der \CThreeBits in den unterschiedlichen Fällen betrachten,
dann sehen wir, dass der bedingte Sprung \JP in zwei Fällen ausgeführt wird:
wenn $b>a$ oder wenn $a=b$ (das \Cthree Bit wird hier nicht betrachtet, da es
durch den Befehl \INS{test ah,5}) gelöscht wurde).

Der Rest ist leicht nachvollziehbar.
Denn der bedingte Sprung ausgeführt wurde, lädt \FLD den Wert von \GTT{\_b} nach
\ST{0} und wenn nicht, wird der Wert von \GTT{\_a} dorthin geladen.

\mysubparagraph{Was ist mit der Abfrage von \Ctwo?}
Das \Ctwo Flag wird im Fehlerfall (\gls{NaN}, etc.) gesetzt, aber unser Code
prüft dies nicht. 
Wenn sich der Programmierer für FPU Fehler interessiert, muss er zusätzliche
Abfragen hinzufügen.

\input{patterns/12_FPU/3_comparison/x86/MSVC/olly_DE.tex}
}
\FR{\mysection{Fonction presque vide}
\label{Boolector}
\myindex{Boolector}
\myindex{x86!\Instructions!JMP}

Ceci est un morceau de code réel que j'ai trouvé dans Boolector\footnote{\url{https://boolector.github.io/}}:

\lstinputlisting[style=customc]{patterns/025_almost_empty/boolectormain.c}

Pourquoi quelqu'un ferait-il comme ça?
Je ne sais pas mais mon hypothèse est que \verb|boolector_main()| peut être compilée
dans une sorte de DLL ou bibliothèque dynamique, et appelée depuis une suite de test.
Certainement qu'une suite de test peut préparer les variables argc/argv comme
le ferait \ac{CRT}.

Il est intéressant de voir comment c'est compilé:

\lstinputlisting[caption=GCC 8.2 x64 \NonOptimizing (\assemblyOutput),style=customasmx86]{patterns/025_almost_empty/boolectormain_O0.s}

Ceci est OK, le prologue (non optimisé) déplace inutilement deux arguments,
\INS{CALL}, épilogue, \INS{RET}.
Mais regardons la version optimisée:

\lstinputlisting[caption=GCC 8.2 x64 \Optimizing (\assemblyOutput),style=customasmx86]{patterns/025_almost_empty/boolectormain_O3.s}

Aussi simple que ça: la pile et les registres ne sont pas touchés et \verb|boolector_main()|
a le même ensemble d'arguments.
Donc, tout ce que nous avons à faire est de passer l'exécution à une autre adresse.

Ceci est proche d'une \glslink{thunk function}{fonction thunk}.

Nous verons queelque chose de plus avancé plus tard: \myref{ARM_B_to_printf}, \myref{JMP_instead_of_RET}.
}

\renewcommand{\CURPATH}{advanced/050_strstr}
\EN{% TODO translate
\mysection{Breaking simple executable cryptor}

I've got an executable file which is encrypted by relatively simple encryption.
\href{\GitHubBlobMasterURL/examples/simple_exec_crypto/files/cipher.bin}{Here is it} (only executable section is left here).

First, all encryption function does is just adds number of position in buffer to the byte.
Here is how this can be encoded in Python:

\begin{lstlisting}[caption=Python script,style=custompy]
#!/usr/bin/env python
def e(i, k):
    return chr ((ord(i)+k) % 256)

def encrypt(buf):
    return e(buf[0], 0)+ e(buf[1], 1)+ e(buf[2], 2) + e(buf[3], 3)+ e(buf[4], 4)+ e(buf[5], 5)+ e(buf[6], 6)+ e(buf[7], 7)+
           e(buf[8], 8)+ e(buf[9], 9)+ e(buf[10], 10)+ e(buf[11], 11)+ e(buf[12], 12)+ e(buf[13], 13)+ e(buf[14], 14)+ e(buf[15], 15)
\end{lstlisting}

Hence, if you encrypt buffer with 16 zeros, you'll get \emph{0, 1, 2, 3 ... 12, 13, 14, 15}.

\myindex{Propagating Cipher Block Chaining}
Propagating Cipher Block Chaining (PCBC) is also used, here is how it works:

\begin{figure}[H]
\centering
\myincludegraphics{examples/simple_exec_crypto/601px-PCBC_encryption.png}
\caption{Propagating Cipher Block Chaining encryption (image is taken from Wikipedia article)}
\end{figure}

The problem is that it's too boring to recover IV (Initialization Vector) each time.
Brute-force is also not an option, because IV is too long (16 bytes).
Let's see, if it's possible to recover IV for arbitrary encrypted executable file?

Let's try simple frequency analysis.
This is 32-bit x86 executable code, so let's gather statistics about most frequent bytes and opcodes.
I tried huge oracle.exe file from Oracle RDBMS version 11.2 for windows x86 and I've found that the most frequent byte (no surprise) is zero (~10\%).
The next most frequent byte is (again, no surprise) 0xFF (~5\%).
The next is 0x8B (~5\%).

\myindex{x86!\Instructions!MOV}
0x8B is opcode for \INS{MOV}, this is indeed one of the most busy x86 instructions.
Now what about popularity of zero byte?
If compiler needs to encode value bigger than 127, it has to use 32-bit displacement instead of 8-bit one, but large values are very rare,
so it is padded by zeros.
\myindex{x86!\Instructions!LEA}
\myindex{x86!\Instructions!PUSH}
\myindex{x86!\Instructions!CALL}
This is at least in \INS{LEA}, \INS{MOV}, \INS{PUSH}, \INS{CALL}.

For example:

\begin{lstlisting}[style=customasmx86]
8D B0 28 01 00 00                 lea     esi, [eax+128h]
8D BF 40 38 00 00                 lea     edi, [edi+3840h]
\end{lstlisting}

Displacements bigger than 127 are very popular, but they are rarely exceeds 0x10000
(indeed, such large memory buffers/structures are also rare).

Same story with \INS{MOV}, large constants are rare, the most heavily used are 0, 1, 10, 100, $2^n$, and so on.
Compiler has to pad small constants by zeros to represent them as 32-bit values:

\begin{lstlisting}[style=customasmx86]
BF 02 00 00 00                    mov     edi, 2
BF 01 00 00 00                    mov     edi, 1
\end{lstlisting}

Now about 00 and FF bytes combined: jumps (including conditional) and calls can pass execution flow forward or backwards, but very often,
within the limits of the current executable module.
If forward, displacement is not very big and also padded with zeros.
If backwards, displacement is represented as negative value, so padded with FF bytes.
For example, transfer execution flow forward:

\begin{lstlisting}[style=customasmx86]
E8 43 0C 00 00                    call    _function1
E8 5C 00 00 00                    call    _function2
0F 84 F0 0A 00 00                 jz      loc_4F09A0
0F 84 EB 00 00 00                 jz      loc_4EFBB8
\end{lstlisting}

Backwards:

\begin{lstlisting}[style=customasmx86]
E8 79 0C FE FF                    call    _function1
E8 F4 16 FF FF                    call    _function2
0F 84 F8 FB FF FF                 jz      loc_8212BC
0F 84 06 FD FF FF                 jz      loc_FF1E7D
\end{lstlisting}

FF byte is also very often occurred in negative displacements like these:

\begin{lstlisting}[style=customasmx86]
8D 85 1E FF FF FF                 lea     eax, [ebp-0E2h]
8D 95 F8 5C FF FF                 lea     edx, [ebp-0A308h]
\end{lstlisting}

So far so good. Now we have to try various 16-byte keys, decrypt executable section and measure how often 00, FF and 8B bytes are occurred.
Let's also keep in sight how PCBC decryption works:

\begin{figure}[H]
\centering
\myincludegraphics{examples/simple_exec_crypto/640px-PCBC_decryption.png}
\caption{Propagating Cipher Block Chaining decryption (image is taken from Wikipedia article)}
\end{figure}

The good news is that we don't really have to decrypt whole piece of data, but only slice by slice, this is exactly how I did in my previous example: \myref{XOR_mask_2}.

Now I'm trying all possible bytes (0..255) for each byte in key and just pick the byte producing maximal amount of 00/FF/8B bytes in a decrypted slice:

\begin{lstlisting}[style=custompy]
#!/usr/bin/env python
import sys, hexdump, array, string, operator

KEY_LEN=16

def chunks(l, n):
    # split n by l-byte chunks
    # https://stackoverflow.com/q/312443
    n = max(1, n)
    return [l[i:i + n] for i in range(0, len(l), n)]

def read_file(fname):
    file=open(fname, mode='rb')
    content=file.read()
    file.close()
    return content

def decrypt_byte (c, key):
    return chr((ord(c)-key) % 256)

def XOR_PCBC_step (IV, buf, k):
    prev=IV
    rt=""
    for c in buf:
	new_c=decrypt_byte(c, k)
        plain=chr(ord(new_c)^ord(prev))
	prev=chr(ord(c)^ord(plain))
	rt=rt+plain
    return rt

each_Nth_byte=[""]*KEY_LEN

content=read_file(sys.argv[1])
# split input by 16-byte chunks:
all_chunks=chunks(content, KEY_LEN)
for c in all_chunks:
    for i in range(KEY_LEN):
        each_Nth_byte[i]=each_Nth_byte[i] + c[i]

# try each byte of key
for N in range(KEY_LEN):
    print "N=", N
    stat={}
    for i in range(256):
        tmp_key=chr(i)
	tmp=XOR_PCBC_step(tmp_key,each_Nth_byte[N], N)
        # count 0, FFs and 8Bs in decrypted buffer:
	important_bytes=tmp.count('\x00')+tmp.count('\xFF')+tmp.count('\x8B')
	stat[i]=important_bytes
    sorted_stat = sorted(stat.iteritems(), key=operator.itemgetter(1), reverse=True)
    print sorted_stat[0]
\end{lstlisting}

(Source code can be downloaded \href{\GitHubBlobMasterURL/examples/simple_exec_crypto/files/decrypt.py}{here}.)

I run it and here is a key for which 00/FF/8B bytes presence in decrypted buffer is maximal:

\begin{lstlisting}
N= 0
(147, 1224)
N= 1
(94, 1327)
N= 2
(252, 1223)
N= 3
(218, 1266)
N= 4
(38, 1209)
N= 5
(192, 1378)
N= 6
(199, 1204)
N= 7
(213, 1332)
N= 8
(225, 1251)
N= 9
(112, 1223)
N= 10
(143, 1177)
N= 11
(108, 1286)
N= 12
(10, 1164)
N= 13
(3, 1271)
N= 14
(128, 1253)
N= 15
(232, 1330)
\end{lstlisting}

Let's write decryption utility with the key we got:

\begin{lstlisting}[style=custompy]
#!/usr/bin/env python
import sys, hexdump, array

def xor_strings(s,t):
    # \verb|https://en.wikipedia.org/wiki/XOR_cipher#Example_implementation|
    """xor two strings together"""
    return "".join(chr(ord(a)^ord(b)) for a,b in zip(s,t))

IV=array.array('B', [147, 94, 252, 218, 38, 192, 199, 213, 225, 112, 143, 108, 10, 3, 128, 232]).tostring()

def chunks(l, n):
    n = max(1, n)
    return [l[i:i + n] for i in range(0, len(l), n)]

def read_file(fname):
    file=open(fname, mode='rb')
    content=file.read()
    file.close()
    return content

def decrypt_byte(i, k):
    return chr ((ord(i)-k) % 256)

def decrypt(buf):
    return "".join(decrypt_byte(buf[i], i) for i in range(16))

fout=open(sys.argv[2], mode='wb')

prev=IV
content=read_file(sys.argv[1])
tmp=chunks(content, 16)
for c in tmp:
    new_c=decrypt(c)
    p=xor_strings (new_c, prev)
    prev=xor_strings(c, p)
    fout.write(p)
fout.close()
\end{lstlisting}

(Source code can be downloaded \href{\GitHubBlobMasterURL/examples/simple_exec_crypto/files/decrypt2.py}{here}.)

Let's check resulting file:

\lstinputlisting{examples/simple_exec_crypto/objdump_result.txt}

Yes, this is seems correctly disassembled piece of x86 code.
The whole decryped file can be downloaded \href{\GitHubBlobMasterURL/examples/simple_exec_crypto/files/decrypted.bin}{here}.

In fact, this is text section from regedit.exe from Windows 7.
But this example is based on a real case I encountered, so just executable is different (and key), algorithm is the same.

\subsection{Other ideas to consider}

What if I would fail with such simple frequency analysis?
There are other ideas on how to measure correctness of decrypted/decompressed x86 code:

\begin{itemize}

\item Many modern compilers aligns functions on 0x10 border.
So the space left before is filled with NOPs (0x90) or other NOP instructions with known opcodes: \myref{sec:npad}.

\item Perhaps, the most frequent pattern in any assembly language is function call:\\
\TT{PUSH chain / CALL / ADD ESP, X}.
This sequence can easily detected and found.
I've even gathered statistics about average number of function arguments: \myref{args_stat}.
(Hence, this is average length of PUSH chain.)

\end{itemize}

Read more about incorrectly/correctly disassembled code: \myref{ISA_detect}.
}\DE{\myparagraph{\NonOptimizing MSVC}

MSVC 2010 erzeugt den folgenden Code:

\lstinputlisting[caption=\NonOptimizing MSVC
2010,style=customasmx86]{patterns/12_FPU/3_comparison/x86/MSVC/MSVC_DE.asm}

\myindex{x86!\Instructions!FLD}

Der Befehl \FLD lädt \GTT{\_b} nach \ST{0}.

\label{Czero_etc}
\newcommand{\Czero}{\GTT{C0}\xspace}
\newcommand{\Ctwo}{\GTT{C2}\xspace}
\newcommand{\Cthree}{\GTT{C3}\xspace}
\newcommand{\CThreeBits}{\Cthree/\Ctwo/\Czero}

\myindex{x86!\Instructions!FCOMP}
\FCOMP verlgeicht den Wert in \ST{0} mit dem Wert, der sich in \GTT{\_a}
befindet und setzt die \CThreeBits im FPU Status Register entsprechend.
Das Statusregister ist ein 16-Bit-Register, das den aktueller Zustand der FPU
abbildet.

Nachdem die Bits gesetzt worden sind, nimmer der \FCOMP Befehl auch eine
Variable vom Stack. Dieses Verhalten unterscheidet ihn von \FCOM, der einfach
zwei Werte vergleicht und den Stack unangetastet lässt.

Leider verfügen CPUs vor Intel P6\footnote{Intel P6 ist Pentium Pro, Pentium II,
etc.}über keinerlei bedingte Sprungbefehle, die die \CThreeBits prüfen.

After the bits are set, the \FCOMP instruction also pops one variable from the stack. 
This is what distinguishes it from \FCOM, which is just compares values, leaving the stack in the same state.
Vielleicht ist diese Tatsache historisch begründet (man erinnere sich: die FPU
war früher ein eigener Chip).\\
Moderne CPUs, beginnend mit Intel P6 haben \FCOMI/\FCOMIP/\FUCOMI/\FUCOMIP
Befehle~--welche im Prinzip das gleiche tun, aber die \ZF/\PF/\CF Flags der CPU
verändern können.

\myindex{x86!\Instructions!FNSTSW}
Der \FNSTSW Befehl kopiert das FPU Statusregister nach \AX.
\CThreeBits werden an den Stellen 14/10/8 abgelegt, sie befinden sich im \AX
Register an den gleichen Stellen und sie werden alle in höherwertigen Teil von
\AX{}~---\AH{} abgelegt.

\begin{itemize}
\item Falls in unserem Beispiel $b>a$, dann werden die \CThreeBits Bits wie
folgt gesetzt: 0, 0, 0.
\item Falls $a>b$, dann ist das Bitmuster: 0, 0, 1.
\item Falls $a=b$, dann ist das Bitmuster: 1, 0, 0.
\item

Wenn das Ergebnis (z.B. im Fehlerfall) ungeordnet ist, dann werden die Bits wie
folgt gesetzt: 1,1,1.
\end{itemize}
% TODO: table here?
So werden die \CThreeBits Bits im \AX Register angeordnet:

\input{C3_in_AX}

So werden die \CThreeBits Bits im \AH Register angeordnet:

\input{C3_in_AH}
Nach der Ausführung von \INS{test ah, 5}\footnote{5=101b} werden nur die \Czero
und \Ctwo Bits (an den Stellen 0 und 2) betrachtet, alle übrigen Bits werden
einfach überlesen.

\label{parity_flag}
\myindex{x86!\Registers!\Flags!Parity flag}
Werfen wir nun einen Blick auf ein anderes bemerkenswertes historisches
Überbleibsel: das \emph{parity flag}.

Dieses Flag wird auf 1 gesetzt, falls die Anzahl der Einsen im Ergebnis der
letzten Berechnung gerade ist und auf 1, falls dies nicht der Fall ist.

Schlagen wir in der Wikipedia nach\footnote{\WikipediaParityFlag}:

%TODO Quotation has been translated from English wiki article, since the
% correspondig German article doesn't offer such information.
\begin{framed}
\begin{quotation}
Ein guter Grund das Parity Flag abzufragen, hat tatsächlich gar nichts mit
Parität zu tun. Die FPU hat vier Bedingungsflags (C0 bis C3), aber diese können
nicht direkt abgefragt werden, sondern müssen zunächst in das Flags Register
kopiert werden. Wenn dies geschieht, wird C0 im Carry Flag abgelegt, C2 im
Parity Flag und C3 im Zero Flag.
Das C2 Flag ist gesetzt, wenn z.B. unvergleichbare Fließkommawerte (NaN oder
nicht unterstütztes Format) über der \FUCOM Befehl miteinander verglichen
werden.\textit{(Übersetzung aus der englischen Wikipedia.)}
\end{quotation}
\end{framed}

Wie in der Wikipedia dargestellt wird das Parity Flag manchmal im FPU Code
verwendet; schauen wir uns genauer an wie das funktioniert.

\myindex{x86!\Instructions!JP}
Das \PF Flag wird auf 1 gesetzt, wenn sowohl \Czero als auch \Ctwo beide 0 oder
beide 1 sind. In diesem Fall wird der nachfolgende Sprung \JP(\emph{jump if
PF==1}) ausgeführt.
Wenn wir die Werte der \CThreeBits in den unterschiedlichen Fällen betrachten,
dann sehen wir, dass der bedingte Sprung \JP in zwei Fällen ausgeführt wird:
wenn $b>a$ oder wenn $a=b$ (das \Cthree Bit wird hier nicht betrachtet, da es
durch den Befehl \INS{test ah,5}) gelöscht wurde).

Der Rest ist leicht nachvollziehbar.
Denn der bedingte Sprung ausgeführt wurde, lädt \FLD den Wert von \GTT{\_b} nach
\ST{0} und wenn nicht, wird der Wert von \GTT{\_a} dorthin geladen.

\mysubparagraph{Was ist mit der Abfrage von \Ctwo?}
Das \Ctwo Flag wird im Fehlerfall (\gls{NaN}, etc.) gesetzt, aber unser Code
prüft dies nicht. 
Wenn sich der Programmierer für FPU Fehler interessiert, muss er zusätzliche
Abfragen hinzufügen.

\input{patterns/12_FPU/3_comparison/x86/MSVC/olly_DE.tex}
}\RU{\subsection{Простое шифрование используя XOR-маску}
\label{XOR_mask_1}

Я нашел одну старую игру в стиле interactive fiction в архиве \emph{if-archive}\footnote{\url{http://www.ifarchive.org/}}:

\begin{lstlisting}
The New Castle v3.5 - Text/Adventure Game
in the style of the original Infocom (tm)
type games, Zork, Collosal Cave (Adventure),
etc.  Can you solve the mystery of the
abandoned castle?
Shareware from Software Customization.
Software Customization [ASP] Version 3.5 Feb. 2000
\end{lstlisting}

Можно скачать здесь: \url{\GitHubBlobMasterURL/ff/XOR/mask_1/files/newcastle.tgz}.

Там внутри есть файл (с названием \emph{castle.dbf}), который явно зашифрован, но не настоящим криптоалгоритмом,
и оне сжат, это что-то куда проще.
Я бы даже не стал измерять уровень энтропии (\myref{entropy}) этого файла, потому что я итак уверен, что он низкий.
Вот как он выглядит в Midnight Commander:

\begin{figure}[H]
\centering
\myincludegraphics{ff/XOR/mask_1/mc_encrypted.png}
\caption{Зашифрованный файл в Midnight Commander}
\end{figure}

Зашифрованный файл можно скачать здесь:
\url{\GitHubBlobMasterURL/ff/XOR/mask_1/files/castle.dbf.bz2}.

Можно ли расшифровать его без доступа к программе, используя просто этот файл?

Тут явно просматривается повторяющаяся строка. 
Если использовалось простое шифрование с XOR-маской, такие повторяющиеся строки это явное свидетельство,
потому что, вероятно, тут были длинные лакуны с нулевыми байтами, которые, в свою очередь, присутствуют
во мноигих исполняемых файлах, и в остальных бинарных файлах.

\myindex{UNIX!xxd}
Вот дам начала этого файла используя утилиту \emph{xxd} из UNIX:

\lstinputlisting{ff/XOR/mask_1/xxd_result.txt}

Давайте держаться за повторяющуюся строку \TT{iubgv}.
Глядя на этот дамп, мы можем легко увидеть, что период повторений этой строки это 0x51 или 81.
Вероятно, 81 это длина блока?
Длина файла 1658961, и она может быть поделена на 81 без остатка (и тогда там 20481 блоков).

Теперь я буду использовать Mathematica для анализа, есть ли тут повторяющиеся 81-байтные блоки в файле?
Я разделю входной файл на 81-байтные блоки и затем использую ф-цию
\emph{Tally[]}\footnote{\url{https://reference.wolfram.com/language/ref/Tally.html}}
которая просто считает, сколько раз каждый элемент встретился во входном списке.
Вывод Tally не отсортирован, так что я также добавлю ф-цию \emph{Sort[]} для сортировки его по кол-ву вхождений
в нисходящем порядке.

\begin{lstlisting}[style=custommath]
input = BinaryReadList["/home/dennis/.../castle.dbf"];

blocks = Partition[input, 81];

stat = Sort[Tally[blocks], #1[[2]] > #2[[2]] &]
\end{lstlisting}

И вот вывод:

\begin{lstlisting}[style=custommath]
{{{80, 103, 2, 116, 113, 102, 118, 25, 99, 8, 19, 23, 116, 125, 107, 
   25, 99, 109, 114, 102, 14, 121, 115, 31, 9, 117, 113, 111, 5, 4, 
   127, 28, 122, 101, 8, 110, 14, 18, 124, 106, 16, 20, 104, 119, 8, 
   109, 26, 106, 9, 97, 13, 99, 15, 119, 20, 105, 117, 98, 103, 118, 
   1, 126, 29, 97, 122, 17, 15, 114, 110, 3, 5, 125, 125, 99, 126, 
   119, 102, 30, 122, 2, 117}, 1739}, 
{{80, 100, 2, 116, 113, 102, 118, 25, 99, 8, 19, 23, 116, 
   125, 107, 25, 99, 109, 114, 102, 14, 121, 115, 31, 9, 117, 113, 
   111, 5, 4, 127, 28, 122, 101, 8, 110, 14, 18, 124, 106, 16, 20, 
   104, 119, 8, 109, 26, 106, 9, 97, 13, 99, 15, 119, 20, 105, 117, 
   98, 103, 118, 1, 126, 29, 97, 122, 17, 15, 114, 110, 3, 5, 125, 
   125, 99, 126, 119, 102, 30, 122, 2, 117}, 1422}, 
{{80, 101, 2, 116, 113, 102, 118, 25, 99, 8, 19, 23, 116, 
   125, 107, 25, 99, 109, 114, 102, 14, 121, 115, 31, 9, 117, 113, 
   111, 5, 4, 127, 28, 122, 101, 8, 110, 14, 18, 124, 106, 16, 20, 
   104, 119, 8, 109, 26, 106, 9, 97, 13, 99, 15, 119, 20, 105, 117, 
   98, 103, 118, 1, 126, 29, 97, 122, 17, 15, 114, 110, 3, 5, 125, 
   125, 99, 126, 119, 102, 30, 122, 2, 117}, 1012},
{{80, 120, 2, 116, 113, 102, 118, 25, 99, 8, 19, 23, 116, 
   125, 107, 25, 99, 109, 114, 102, 14, 121, 115, 31, 9, 117, 113, 
   111, 5, 4, 127, 28, 122, 101, 8, 110, 14, 18, 124, 106, 16, 20, 
   104, 119, 8, 109, 26, 106, 9, 97, 13, 99, 15, 119, 20, 105, 117, 
   98, 103, 118, 1, 126, 29, 97, 122, 17, 15, 114, 110, 3, 5, 125, 
   125, 99, 126, 119, 102, 30, 122, 2, 117}, 377},

...

{{80, 2, 74, 49, 113, 21, 62, 88, 39, 71, 68, 23, 63, 51, 36, 78, 48, 
   108, 114, 102, 14, 121, 115, 31, 9, 117, 113, 111, 5, 4, 127, 28, 
   122, 101, 8, 110, 14, 18, 124, 106, 16, 20, 104, 119, 8, 109, 26, 
   106, 9, 97, 13, 99, 15, 119, 20, 105, 117, 98, 103, 118, 1, 126, 
   29, 97, 122, 17, 15, 114, 110, 3, 5, 125, 125, 99, 126, 119, 102, 
   30, 122, 2, 117}, 1},
{{80, 1, 74, 59, 113, 45, 56, 86, 52, 91, 19, 64, 60, 60, 63, 
   25, 38, 59, 59, 42, 14, 53, 38, 77, 66, 38, 113, 38, 75, 4, 43, 84,
    63, 101, 64, 43, 79, 64, 40, 57, 16, 91, 46, 119, 69, 40, 84, 117,
    9, 97, 13, 99, 15, 119, 20, 105, 117, 98, 103, 118, 1, 126, 29, 
   97, 122, 17, 15, 114, 110, 3, 5, 125, 125, 99, 126, 119, 102, 30, 
   122, 2, 117}, 1},
{{80, 2, 74, 49, 113, 49, 51, 92, 39, 8, 92, 81, 116, 62, 57, 
   80, 46, 40, 114, 36, 75, 56, 33, 76, 9, 55, 56, 59, 81, 65, 45, 28,
    60, 55, 93, 39, 90, 28, 124, 106, 16, 20, 104, 119, 8, 109, 26, 
   106, 9, 97, 13, 99, 15, 119, 20, 105, 117, 98, 103, 118, 1, 126, 
   29, 97, 122, 17, 15, 114, 110, 3, 5, 125, 125, 99, 126, 119, 102, 
   30, 122, 2, 117}, 1}}
\end{lstlisting}

Вывод Tally это список пар, каждая пара это 81-байтный блок и количество раз, сколько он встретился в файле.
Мы видим, что наиболее частно встречающийся блок это первый, он встретился 1739 раз.
Второй встретился 1422 раза. Есть и другие: 1012 раза, 377 раз, итд.
81-байтные блоки, встреченные лишь один раз, находятся в конце вывода.

Попробуем сравнить эти блоки. Первый и второй.
Есть ли в Mathematica ф-ция для сравнения списков/массивов?
Наверняка есть, но в педагогических целях, я буду использоват операцию XOR для сравнения.
Действительно: если байты во входных массивах равны друг другу, результат операции XOR это 0.
Если не равны, результат будет ненулевой.

Сравним первый блок (встречается 1739 раз) и второй (встречается 1422 раз):

\begin{lstlisting}[style=custommath]
In[]:= BitXor[stat[[1]][[1]], stat[[2]][[1]]]
Out[]= {0, 3, 0, 0, 0, 0, 0, 0, 0, 0, 0, 0, 0, 0, 0, 0, 0, 0, 0, \
0, 0, 0, 0, 0, 0, 0, 0, 0, 0, 0, 0, 0, 0, 0, 0, 0, 0, 0, 0, 0, 0, 0, \
0, 0, 0, 0, 0, 0, 0, 0, 0, 0, 0, 0, 0, 0, 0, 0, 0, 0, 0, 0, 0, 0, 0, \
0, 0, 0, 0, 0, 0, 0, 0, 0, 0, 0, 0, 0, 0, 0, 0}
\end{lstlisting}

Они отличаются только вторым байтом.

Сравним второй блок (встречается 1422 раза) и третий (встречается 1012 раз):

\begin{lstlisting}[style=custommath]
In[]:= BitXor[stat[[2]][[1]], stat[[3]][[1]]]
Out[]= {0, 1, 0, 0, 0, 0, 0, 0, 0, 0, 0, 0, 0, 0, 0, 0, 0, 0, 0, \
0, 0, 0, 0, 0, 0, 0, 0, 0, 0, 0, 0, 0, 0, 0, 0, 0, 0, 0, 0, 0, 0, 0, \
0, 0, 0, 0, 0, 0, 0, 0, 0, 0, 0, 0, 0, 0, 0, 0, 0, 0, 0, 0, 0, 0, 0, \
0, 0, 0, 0, 0, 0, 0, 0, 0, 0, 0, 0, 0, 0, 0, 0}
\end{lstlisting}

Они тоже отличаются только вторым байтом.

Так или иначе, попробуем использовать самый встречающийся блок как XOR-ключ и попробуем расшифровать первые 4 81-байтных
блока в файле:

\begin{lstlisting}[style=custommath]
In[]:= key = stat[[1]][[1]]
Out[]= {80, 103, 2, 116, 113, 102, 118, 25, 99, 8, 19, 23, 116, \
125, 107, 25, 99, 109, 114, 102, 14, 121, 115, 31, 9, 117, 113, 111, \
5, 4, 127, 28, 122, 101, 8, 110, 14, 18, 124, 106, 16, 20, 104, 119, \
8, 109, 26, 106, 9, 97, 13, 99, 15, 119, 20, 105, 117, 98, 103, 118, \
1, 126, 29, 97, 122, 17, 15, 114, 110, 3, 5, 125, 125, 99, 126, 119, \
102, 30, 122, 2, 117}

In[]:= ToASCII[val_] := If[val == 0, " ", FromCharacterCode[val, "PrintableASCII"]]

In[]:= DecryptBlockASCII[blk_] := Map[ToASCII[#] &, BitXor[key, blk]]

In[]:= DecryptBlockASCII[blocks[[1]]]
Out[]= {" ", " ", " ", " ", " ", " ", " ", " ", " ", " ", " ", " \
", " ", " ", " ", " ", " ", " ", " ", " ", " ", " ", " ", " ", " ", " \
", " ", " ", " ", " ", " ", " ", " ", " ", " ", " ", " ", " ", " ", " \
", " ", " ", " ", " ", " ", " ", " ", " ", " ", " ", " ", " ", " ", " \
", " ", " ", " ", " ", " ", " ", " ", " ", " ", " ", " ", " ", " ", " \
", " ", " ", " ", " ", " ", " ", " ", " ", " ", " ", " ", " ", " "}

In[]:= DecryptBlockASCII[blocks[[2]]]
Out[]= {" ", "e", "H", "E", " ", "W", "E", "E", "D", " ", "O", \
"F", " ", "C", "R", "I", "M", "E", " ", "B", "E", "A", "R", "S", " ", \
"B", "I", "T", "T", "E", "R", " ", "F", "R", "U", "I", "T", "?", \
" ", " ", " ", " ", " ", " ", " ", " ", " ", " ", " ", " ", " ", " ", \
" ", " ", " ", " ", " ", " ", " ", " ", " ", " ", " ", " ", " ", " ", \
" ", " ", " ", " ", " ", " ", " ", " ", " ", " ", " ", " ", " ", " ", \
" "}

In[]:= DecryptBlockASCII[blocks[[3]]]
Out[]= {" ", "?", " ", " ", " ", " ", " ", " ", " ", " ", " \
", " ", " ", " ", " ", " ", " ", " ", " ", " ", " ", " ", " ", " ", " \
", " ", " ", " ", " ", " ", " ", " ", " ", " ", " ", " ", " ", " ", " \
", " ", " ", " ", " ", " ", " ", " ", " ", " ", " ", " ", " ", " ", " \
", " ", " ", " ", " ", " ", " ", " ", " ", " ", " ", " ", " ", " ", " \
", " ", " ", " ", " ", " ", " ", " ", " ", " ", " ", " ", " ", " ", " \
"}

In[]:= DecryptBlockASCII[blocks[[4]]]
Out[]= {" ", "f", "H", "O", " ", "K", "N", "O", "W", "S", " ", \
"W", "H", "A", "T", " ", "E", "V", "I", "L", " ", "L", "U", "R", "K", \
"S", " ", "I", "N", " ", "T", "H", "E", " ", "H", "E", "A", "R", "T", \
"S", " ", "O", "F", " ", "M", "E", "N", "?", " ", " ", " ", " ", \
" ", " ", " ", " ", " ", " ", " ", " ", " ", " ", " ", " ", " ", " ", \
" ", " ", " ", " ", " ", " ", " ", " ", " ", " ", " ", " ", " ", " ", \
" "}
\end{lstlisting}

(Я заменил непечатаемые символы на \q{?}.)

Мы видим что первый и третий блоки пустые (или почти пустые),
но второй и четвертый имеют ясно различимые английские слова/фразы.
Похоже что наше предположение насчет ключа верно (как минимум частично).
Это означает, что самый встречающийся 81-байтный блок в файле находится в местах лакун с нулевыми байтами
или что-то в этом роде.

Попробуем расшифровать весь файл:

\begin{lstlisting}[style=custommath]
DecryptBlock[blk_] := BitXor[key, blk]

decrypted = Map[DecryptBlock[#] &, blocks];

BinaryWrite["/home/dennis/.../tmp", Flatten[decrypted]]

Close["/home/dennis/.../tmp"]
\end{lstlisting}

\begin{figure}[H]
\centering
\myincludegraphics{ff/XOR/mask_1/mc_decrypted1.png}
\caption{Расшифрованный файл в Midnight Commander, первая попытка}
\end{figure}

Выглядит как английские фразы для какой-то игры, но что-то не так.
Прежде всего, регистр инвертирован: фразы и некоторые слова начинаются со строчных букв,
в то время как остальные буквы заглавные.
Также, некоторые фразы начинаются с не тех букв.
Посмотрите на самую первую фразу: \q{eHE WEED OF CRIME BEARS BITTER FRUIT}.
Что такое \q{eHE}? Разве не \q{tHE} тут должно быть?
Возможно ли что наш ключ для дешифрования имеет неверный байт в этом месте?

Посмотрим снова на второй блок в файле, на ключ и на результат дешифрования:

\begin{lstlisting}[style=custommath]
In[]:= blocks[[2]]
Out[]= {80, 2, 74, 49, 113, 49, 51, 92, 39, 8, 92, 81, 116, 62, \
57, 80, 46, 40, 114, 36, 75, 56, 33, 76, 9, 55, 56, 59, 81, 65, 45, \
28, 60, 55, 93, 39, 90, 28, 124, 106, 16, 20, 104, 119, 8, 109, 26, \
106, 9, 97, 13, 99, 15, 119, 20, 105, 117, 98, 103, 118, 1, 126, 29, \
97, 122, 17, 15, 114, 110, 3, 5, 125, 125, 99, 126, 119, 102, 30, \
122, 2, 117}

In[]:= key
Out[]= {80, 103, 2, 116, 113, 102, 118, 25, 99, 8, 19, 23, 116, \
125, 107, 25, 99, 109, 114, 102, 14, 121, 115, 31, 9, 117, 113, 111, \
5, 4, 127, 28, 122, 101, 8, 110, 14, 18, 124, 106, 16, 20, 104, 119, \
8, 109, 26, 106, 9, 97, 13, 99, 15, 119, 20, 105, 117, 98, 103, 118, \
1, 126, 29, 97, 122, 17, 15, 114, 110, 3, 5, 125, 125, 99, 126, 119, \
102, 30, 122, 2, 117}

In[]:= BitXor[key, blocks[[2]]]
Out[]= {0, 101, 72, 69, 0, 87, 69, 69, 68, 0, 79, 70, 0, 67, 82, \
73, 77, 69, 0, 66, 69, 65, 82, 83, 0, 66, 73, 84, 84, 69, 82, 0, 70, \
82, 85, 73, 84, 14, 0, 0, 0, 0, 0, 0, 0, 0, 0, 0, 0, 0, 0, 0, 0, 0, \
0, 0, 0, 0, 0, 0, 0, 0, 0, 0, 0, 0, 0, 0, 0, 0, 0, 0, 0, 0, 0, 0, 0, \
0, 0, 0, 0}
\end{lstlisting}

Зашифрованный байт это 2, байт из ключа это 103, $2 \oplus 103=101$ и 101 это ASCII-код символа \q{e}.
Чему должен равнятся этот байт ключа, чтобы ASCII-код был 116 (для символа  \q{t})?
$2 \oplus 116=118$, присвоим 118 второму байту в ключе \dots

\begin{lstlisting}[style=custommath]
key = {80, 118, 2, 116, 113, 102, 118, 25, 99, 8, 19, 23, 116, 125, 
  107, 25, 99, 109, 114, 102, 14, 121, 115, 31, 9, 117, 113, 111, 5, 
  4, 127, 28, 122, 101, 8, 110, 14, 18, 124, 106, 16, 20, 104, 119, 8,
   109, 26, 106, 9, 97, 13, 99, 15, 119, 20, 105, 117, 98, 103, 118, 
  1, 126, 29, 97, 122, 17, 15, 114, 110, 3, 5, 125, 125, 99, 126, 119,
   102, 30, 122, 2, 117}
\end{lstlisting}

\dots и снова дешифруем весь файл.

\begin{figure}[H]
\centering
\myincludegraphics{ff/XOR/mask_1/mc_decrypted2.png}
\caption{Дешифрованный файл в Midnight Commander, вторая попытка}
\end{figure}

Ух ты, теперь грамматика корректна, и все фразы начинаются с корректных букв.
Но все таки, регистр подозрителен.
С чего бы разработчику игры записывать их в такой манере?
Может быть наш ключ все еще неправилен?

% TODO ASCII table somewhere in the book
Изучая таблицу ASCII мы можем заметить что ASCII-коды для букв в верхнем и нижнем регистре отличаются только на один бит
(6-й бит, если считать с первого, 0b100000):

\begin{figure}[H]
\centering
\includegraphics[width=0.7\textwidth]{ascii.png}
\caption{7-битная таблица \ac{ASCII} в Emacs}
\end{figure}

6-й бит, выставленный в нулевом байте, В десятичном виде это будет 32.
Но 32 это ASCII-код пробела!

Действительно, можно менять регистр просто применяя XOR к ASCII-коду, с 32 (больше об этом: \myref{toupper_bit}).

Возможно ли, что пустые лакуны в файле это не нулевые байты, а скорее содержащие пробелы?
Еще раз модифицируем наш XOR-ключ (я про-XOR-ю каждый байт ключа с 32):

\begin{lstlisting}[style=custommath]
(* "32" это скаляр, и "key" это вектор, но это OK *)

In[]:= key3 = BitXor[32, key]
Out[]= {112, 86, 34, 84, 81, 70, 86, 57, 67, 40, 51, 55, 84, 93, 75, \
57, 67, 77, 82, 70, 46, 89, 83, 63, 41, 85, 81, 79, 37, 36, 95, 60, \
90, 69, 40, 78, 46, 50, 92, 74, 48, 52, 72, 87, 40, 77, 58, 74, 41, \
65, 45, 67, 47, 87, 52, 73, 85, 66, 71, 86, 33, 94, 61, 65, 90, 49, \
47, 82, 78, 35, 37, 93, 93, 67, 94, 87, 70, 62, 90, 34, 85}

In[]:= DecryptBlock[blk_] := BitXor[key3, blk]
\end{lstlisting}

И снова дешифруем входной файл:

\begin{figure}[H]
\centering
\myincludegraphics{ff/XOR/mask_1/mc_decrypted.png}
\caption{Дешифрованный файл в Midnight Commander, последняя попытка}
\end{figure}

(Расшифрованный файл доступен здесь:
\url{\GitHubBlobMasterURL/ff/XOR/mask_1/files/decrypted.dat.bz2}.)

Несомненно, это корректный исходный файл.
Да, и мы видим числа в начале каждого блока. Должно быть это и есть источник некорректного XOR-ключа.
Как выходит, самый встречающийся 81-байтный блок в файле это блок заполненный пробелами и содержащий символ \q{1} на месте
второго байта.
Действительно, как-то так получилось что многие блоки здесь перемежаются с этим блоком.
Может быть это что-то вроде выравнивания (padding) для коротких фраз/сообщений?
Другой часто встречающийся 81-байтный блок также заполнен пробелами, но с другой цифрой, следовательно,
они отличаются только вторым байтом.

Вот и всё! Теперь мы можем написать утилиту для зашифрования файла назад, и, может быть, модифицировать его перед этим

Файл для Mathematica можно скачать здесь:\\
\url{\GitHubBlobMasterURL/ff/XOR/mask_1/files/XOR_mask_1.nb}.

Итог: XOR-шифрование не надежно вообще. Вероятно, разработчик игры хотел просто скрыть внутренности игры от игрока,
ничего более серьезного.
Все же, шифрование вроде этого крайне популярно вследствии его простоты, так что многие реверс инженеры обычно хорошо
с этим знакомы.

}\FR{\mysection{Fonction presque vide}
\label{Boolector}
\myindex{Boolector}
\myindex{x86!\Instructions!JMP}

Ceci est un morceau de code réel que j'ai trouvé dans Boolector\footnote{\url{https://boolector.github.io/}}:

\lstinputlisting[style=customc]{patterns/025_almost_empty/boolectormain.c}

Pourquoi quelqu'un ferait-il comme ça?
Je ne sais pas mais mon hypothèse est que \verb|boolector_main()| peut être compilée
dans une sorte de DLL ou bibliothèque dynamique, et appelée depuis une suite de test.
Certainement qu'une suite de test peut préparer les variables argc/argv comme
le ferait \ac{CRT}.

Il est intéressant de voir comment c'est compilé:

\lstinputlisting[caption=GCC 8.2 x64 \NonOptimizing (\assemblyOutput),style=customasmx86]{patterns/025_almost_empty/boolectormain_O0.s}

Ceci est OK, le prologue (non optimisé) déplace inutilement deux arguments,
\INS{CALL}, épilogue, \INS{RET}.
Mais regardons la version optimisée:

\lstinputlisting[caption=GCC 8.2 x64 \Optimizing (\assemblyOutput),style=customasmx86]{patterns/025_almost_empty/boolectormain_O3.s}

Aussi simple que ça: la pile et les registres ne sont pas touchés et \verb|boolector_main()|
a le même ensemble d'arguments.
Donc, tout ce que nous avons à faire est de passer l'exécution à une autre adresse.

Ceci est proche d'une \glslink{thunk function}{fonction thunk}.

Nous verons queelque chose de plus avancé plus tard: \myref{ARM_B_to_printf}, \myref{JMP_instead_of_RET}.
}

\renewcommand{\CURPATH}{advanced/060_qsort_redux}
\EN{% TODO translate
\mysection{Breaking simple executable cryptor}

I've got an executable file which is encrypted by relatively simple encryption.
\href{\GitHubBlobMasterURL/examples/simple_exec_crypto/files/cipher.bin}{Here is it} (only executable section is left here).

First, all encryption function does is just adds number of position in buffer to the byte.
Here is how this can be encoded in Python:

\begin{lstlisting}[caption=Python script,style=custompy]
#!/usr/bin/env python
def e(i, k):
    return chr ((ord(i)+k) % 256)

def encrypt(buf):
    return e(buf[0], 0)+ e(buf[1], 1)+ e(buf[2], 2) + e(buf[3], 3)+ e(buf[4], 4)+ e(buf[5], 5)+ e(buf[6], 6)+ e(buf[7], 7)+
           e(buf[8], 8)+ e(buf[9], 9)+ e(buf[10], 10)+ e(buf[11], 11)+ e(buf[12], 12)+ e(buf[13], 13)+ e(buf[14], 14)+ e(buf[15], 15)
\end{lstlisting}

Hence, if you encrypt buffer with 16 zeros, you'll get \emph{0, 1, 2, 3 ... 12, 13, 14, 15}.

\myindex{Propagating Cipher Block Chaining}
Propagating Cipher Block Chaining (PCBC) is also used, here is how it works:

\begin{figure}[H]
\centering
\myincludegraphics{examples/simple_exec_crypto/601px-PCBC_encryption.png}
\caption{Propagating Cipher Block Chaining encryption (image is taken from Wikipedia article)}
\end{figure}

The problem is that it's too boring to recover IV (Initialization Vector) each time.
Brute-force is also not an option, because IV is too long (16 bytes).
Let's see, if it's possible to recover IV for arbitrary encrypted executable file?

Let's try simple frequency analysis.
This is 32-bit x86 executable code, so let's gather statistics about most frequent bytes and opcodes.
I tried huge oracle.exe file from Oracle RDBMS version 11.2 for windows x86 and I've found that the most frequent byte (no surprise) is zero (~10\%).
The next most frequent byte is (again, no surprise) 0xFF (~5\%).
The next is 0x8B (~5\%).

\myindex{x86!\Instructions!MOV}
0x8B is opcode for \INS{MOV}, this is indeed one of the most busy x86 instructions.
Now what about popularity of zero byte?
If compiler needs to encode value bigger than 127, it has to use 32-bit displacement instead of 8-bit one, but large values are very rare,
so it is padded by zeros.
\myindex{x86!\Instructions!LEA}
\myindex{x86!\Instructions!PUSH}
\myindex{x86!\Instructions!CALL}
This is at least in \INS{LEA}, \INS{MOV}, \INS{PUSH}, \INS{CALL}.

For example:

\begin{lstlisting}[style=customasmx86]
8D B0 28 01 00 00                 lea     esi, [eax+128h]
8D BF 40 38 00 00                 lea     edi, [edi+3840h]
\end{lstlisting}

Displacements bigger than 127 are very popular, but they are rarely exceeds 0x10000
(indeed, such large memory buffers/structures are also rare).

Same story with \INS{MOV}, large constants are rare, the most heavily used are 0, 1, 10, 100, $2^n$, and so on.
Compiler has to pad small constants by zeros to represent them as 32-bit values:

\begin{lstlisting}[style=customasmx86]
BF 02 00 00 00                    mov     edi, 2
BF 01 00 00 00                    mov     edi, 1
\end{lstlisting}

Now about 00 and FF bytes combined: jumps (including conditional) and calls can pass execution flow forward or backwards, but very often,
within the limits of the current executable module.
If forward, displacement is not very big and also padded with zeros.
If backwards, displacement is represented as negative value, so padded with FF bytes.
For example, transfer execution flow forward:

\begin{lstlisting}[style=customasmx86]
E8 43 0C 00 00                    call    _function1
E8 5C 00 00 00                    call    _function2
0F 84 F0 0A 00 00                 jz      loc_4F09A0
0F 84 EB 00 00 00                 jz      loc_4EFBB8
\end{lstlisting}

Backwards:

\begin{lstlisting}[style=customasmx86]
E8 79 0C FE FF                    call    _function1
E8 F4 16 FF FF                    call    _function2
0F 84 F8 FB FF FF                 jz      loc_8212BC
0F 84 06 FD FF FF                 jz      loc_FF1E7D
\end{lstlisting}

FF byte is also very often occurred in negative displacements like these:

\begin{lstlisting}[style=customasmx86]
8D 85 1E FF FF FF                 lea     eax, [ebp-0E2h]
8D 95 F8 5C FF FF                 lea     edx, [ebp-0A308h]
\end{lstlisting}

So far so good. Now we have to try various 16-byte keys, decrypt executable section and measure how often 00, FF and 8B bytes are occurred.
Let's also keep in sight how PCBC decryption works:

\begin{figure}[H]
\centering
\myincludegraphics{examples/simple_exec_crypto/640px-PCBC_decryption.png}
\caption{Propagating Cipher Block Chaining decryption (image is taken from Wikipedia article)}
\end{figure}

The good news is that we don't really have to decrypt whole piece of data, but only slice by slice, this is exactly how I did in my previous example: \myref{XOR_mask_2}.

Now I'm trying all possible bytes (0..255) for each byte in key and just pick the byte producing maximal amount of 00/FF/8B bytes in a decrypted slice:

\begin{lstlisting}[style=custompy]
#!/usr/bin/env python
import sys, hexdump, array, string, operator

KEY_LEN=16

def chunks(l, n):
    # split n by l-byte chunks
    # https://stackoverflow.com/q/312443
    n = max(1, n)
    return [l[i:i + n] for i in range(0, len(l), n)]

def read_file(fname):
    file=open(fname, mode='rb')
    content=file.read()
    file.close()
    return content

def decrypt_byte (c, key):
    return chr((ord(c)-key) % 256)

def XOR_PCBC_step (IV, buf, k):
    prev=IV
    rt=""
    for c in buf:
	new_c=decrypt_byte(c, k)
        plain=chr(ord(new_c)^ord(prev))
	prev=chr(ord(c)^ord(plain))
	rt=rt+plain
    return rt

each_Nth_byte=[""]*KEY_LEN

content=read_file(sys.argv[1])
# split input by 16-byte chunks:
all_chunks=chunks(content, KEY_LEN)
for c in all_chunks:
    for i in range(KEY_LEN):
        each_Nth_byte[i]=each_Nth_byte[i] + c[i]

# try each byte of key
for N in range(KEY_LEN):
    print "N=", N
    stat={}
    for i in range(256):
        tmp_key=chr(i)
	tmp=XOR_PCBC_step(tmp_key,each_Nth_byte[N], N)
        # count 0, FFs and 8Bs in decrypted buffer:
	important_bytes=tmp.count('\x00')+tmp.count('\xFF')+tmp.count('\x8B')
	stat[i]=important_bytes
    sorted_stat = sorted(stat.iteritems(), key=operator.itemgetter(1), reverse=True)
    print sorted_stat[0]
\end{lstlisting}

(Source code can be downloaded \href{\GitHubBlobMasterURL/examples/simple_exec_crypto/files/decrypt.py}{here}.)

I run it and here is a key for which 00/FF/8B bytes presence in decrypted buffer is maximal:

\begin{lstlisting}
N= 0
(147, 1224)
N= 1
(94, 1327)
N= 2
(252, 1223)
N= 3
(218, 1266)
N= 4
(38, 1209)
N= 5
(192, 1378)
N= 6
(199, 1204)
N= 7
(213, 1332)
N= 8
(225, 1251)
N= 9
(112, 1223)
N= 10
(143, 1177)
N= 11
(108, 1286)
N= 12
(10, 1164)
N= 13
(3, 1271)
N= 14
(128, 1253)
N= 15
(232, 1330)
\end{lstlisting}

Let's write decryption utility with the key we got:

\begin{lstlisting}[style=custompy]
#!/usr/bin/env python
import sys, hexdump, array

def xor_strings(s,t):
    # \verb|https://en.wikipedia.org/wiki/XOR_cipher#Example_implementation|
    """xor two strings together"""
    return "".join(chr(ord(a)^ord(b)) for a,b in zip(s,t))

IV=array.array('B', [147, 94, 252, 218, 38, 192, 199, 213, 225, 112, 143, 108, 10, 3, 128, 232]).tostring()

def chunks(l, n):
    n = max(1, n)
    return [l[i:i + n] for i in range(0, len(l), n)]

def read_file(fname):
    file=open(fname, mode='rb')
    content=file.read()
    file.close()
    return content

def decrypt_byte(i, k):
    return chr ((ord(i)-k) % 256)

def decrypt(buf):
    return "".join(decrypt_byte(buf[i], i) for i in range(16))

fout=open(sys.argv[2], mode='wb')

prev=IV
content=read_file(sys.argv[1])
tmp=chunks(content, 16)
for c in tmp:
    new_c=decrypt(c)
    p=xor_strings (new_c, prev)
    prev=xor_strings(c, p)
    fout.write(p)
fout.close()
\end{lstlisting}

(Source code can be downloaded \href{\GitHubBlobMasterURL/examples/simple_exec_crypto/files/decrypt2.py}{here}.)

Let's check resulting file:

\lstinputlisting{examples/simple_exec_crypto/objdump_result.txt}

Yes, this is seems correctly disassembled piece of x86 code.
The whole decryped file can be downloaded \href{\GitHubBlobMasterURL/examples/simple_exec_crypto/files/decrypted.bin}{here}.

In fact, this is text section from regedit.exe from Windows 7.
But this example is based on a real case I encountered, so just executable is different (and key), algorithm is the same.

\subsection{Other ideas to consider}

What if I would fail with such simple frequency analysis?
There are other ideas on how to measure correctness of decrypted/decompressed x86 code:

\begin{itemize}

\item Many modern compilers aligns functions on 0x10 border.
So the space left before is filled with NOPs (0x90) or other NOP instructions with known opcodes: \myref{sec:npad}.

\item Perhaps, the most frequent pattern in any assembly language is function call:\\
\TT{PUSH chain / CALL / ADD ESP, X}.
This sequence can easily detected and found.
I've even gathered statistics about average number of function arguments: \myref{args_stat}.
(Hence, this is average length of PUSH chain.)

\end{itemize}

Read more about incorrectly/correctly disassembled code: \myref{ISA_detect}.
}

\renewcommand{\CURPATH}{advanced/100_fahrenheit}
\EN{% TODO translate
\mysection{Breaking simple executable cryptor}

I've got an executable file which is encrypted by relatively simple encryption.
\href{\GitHubBlobMasterURL/examples/simple_exec_crypto/files/cipher.bin}{Here is it} (only executable section is left here).

First, all encryption function does is just adds number of position in buffer to the byte.
Here is how this can be encoded in Python:

\begin{lstlisting}[caption=Python script,style=custompy]
#!/usr/bin/env python
def e(i, k):
    return chr ((ord(i)+k) % 256)

def encrypt(buf):
    return e(buf[0], 0)+ e(buf[1], 1)+ e(buf[2], 2) + e(buf[3], 3)+ e(buf[4], 4)+ e(buf[5], 5)+ e(buf[6], 6)+ e(buf[7], 7)+
           e(buf[8], 8)+ e(buf[9], 9)+ e(buf[10], 10)+ e(buf[11], 11)+ e(buf[12], 12)+ e(buf[13], 13)+ e(buf[14], 14)+ e(buf[15], 15)
\end{lstlisting}

Hence, if you encrypt buffer with 16 zeros, you'll get \emph{0, 1, 2, 3 ... 12, 13, 14, 15}.

\myindex{Propagating Cipher Block Chaining}
Propagating Cipher Block Chaining (PCBC) is also used, here is how it works:

\begin{figure}[H]
\centering
\myincludegraphics{examples/simple_exec_crypto/601px-PCBC_encryption.png}
\caption{Propagating Cipher Block Chaining encryption (image is taken from Wikipedia article)}
\end{figure}

The problem is that it's too boring to recover IV (Initialization Vector) each time.
Brute-force is also not an option, because IV is too long (16 bytes).
Let's see, if it's possible to recover IV for arbitrary encrypted executable file?

Let's try simple frequency analysis.
This is 32-bit x86 executable code, so let's gather statistics about most frequent bytes and opcodes.
I tried huge oracle.exe file from Oracle RDBMS version 11.2 for windows x86 and I've found that the most frequent byte (no surprise) is zero (~10\%).
The next most frequent byte is (again, no surprise) 0xFF (~5\%).
The next is 0x8B (~5\%).

\myindex{x86!\Instructions!MOV}
0x8B is opcode for \INS{MOV}, this is indeed one of the most busy x86 instructions.
Now what about popularity of zero byte?
If compiler needs to encode value bigger than 127, it has to use 32-bit displacement instead of 8-bit one, but large values are very rare,
so it is padded by zeros.
\myindex{x86!\Instructions!LEA}
\myindex{x86!\Instructions!PUSH}
\myindex{x86!\Instructions!CALL}
This is at least in \INS{LEA}, \INS{MOV}, \INS{PUSH}, \INS{CALL}.

For example:

\begin{lstlisting}[style=customasmx86]
8D B0 28 01 00 00                 lea     esi, [eax+128h]
8D BF 40 38 00 00                 lea     edi, [edi+3840h]
\end{lstlisting}

Displacements bigger than 127 are very popular, but they are rarely exceeds 0x10000
(indeed, such large memory buffers/structures are also rare).

Same story with \INS{MOV}, large constants are rare, the most heavily used are 0, 1, 10, 100, $2^n$, and so on.
Compiler has to pad small constants by zeros to represent them as 32-bit values:

\begin{lstlisting}[style=customasmx86]
BF 02 00 00 00                    mov     edi, 2
BF 01 00 00 00                    mov     edi, 1
\end{lstlisting}

Now about 00 and FF bytes combined: jumps (including conditional) and calls can pass execution flow forward or backwards, but very often,
within the limits of the current executable module.
If forward, displacement is not very big and also padded with zeros.
If backwards, displacement is represented as negative value, so padded with FF bytes.
For example, transfer execution flow forward:

\begin{lstlisting}[style=customasmx86]
E8 43 0C 00 00                    call    _function1
E8 5C 00 00 00                    call    _function2
0F 84 F0 0A 00 00                 jz      loc_4F09A0
0F 84 EB 00 00 00                 jz      loc_4EFBB8
\end{lstlisting}

Backwards:

\begin{lstlisting}[style=customasmx86]
E8 79 0C FE FF                    call    _function1
E8 F4 16 FF FF                    call    _function2
0F 84 F8 FB FF FF                 jz      loc_8212BC
0F 84 06 FD FF FF                 jz      loc_FF1E7D
\end{lstlisting}

FF byte is also very often occurred in negative displacements like these:

\begin{lstlisting}[style=customasmx86]
8D 85 1E FF FF FF                 lea     eax, [ebp-0E2h]
8D 95 F8 5C FF FF                 lea     edx, [ebp-0A308h]
\end{lstlisting}

So far so good. Now we have to try various 16-byte keys, decrypt executable section and measure how often 00, FF and 8B bytes are occurred.
Let's also keep in sight how PCBC decryption works:

\begin{figure}[H]
\centering
\myincludegraphics{examples/simple_exec_crypto/640px-PCBC_decryption.png}
\caption{Propagating Cipher Block Chaining decryption (image is taken from Wikipedia article)}
\end{figure}

The good news is that we don't really have to decrypt whole piece of data, but only slice by slice, this is exactly how I did in my previous example: \myref{XOR_mask_2}.

Now I'm trying all possible bytes (0..255) for each byte in key and just pick the byte producing maximal amount of 00/FF/8B bytes in a decrypted slice:

\begin{lstlisting}[style=custompy]
#!/usr/bin/env python
import sys, hexdump, array, string, operator

KEY_LEN=16

def chunks(l, n):
    # split n by l-byte chunks
    # https://stackoverflow.com/q/312443
    n = max(1, n)
    return [l[i:i + n] for i in range(0, len(l), n)]

def read_file(fname):
    file=open(fname, mode='rb')
    content=file.read()
    file.close()
    return content

def decrypt_byte (c, key):
    return chr((ord(c)-key) % 256)

def XOR_PCBC_step (IV, buf, k):
    prev=IV
    rt=""
    for c in buf:
	new_c=decrypt_byte(c, k)
        plain=chr(ord(new_c)^ord(prev))
	prev=chr(ord(c)^ord(plain))
	rt=rt+plain
    return rt

each_Nth_byte=[""]*KEY_LEN

content=read_file(sys.argv[1])
# split input by 16-byte chunks:
all_chunks=chunks(content, KEY_LEN)
for c in all_chunks:
    for i in range(KEY_LEN):
        each_Nth_byte[i]=each_Nth_byte[i] + c[i]

# try each byte of key
for N in range(KEY_LEN):
    print "N=", N
    stat={}
    for i in range(256):
        tmp_key=chr(i)
	tmp=XOR_PCBC_step(tmp_key,each_Nth_byte[N], N)
        # count 0, FFs and 8Bs in decrypted buffer:
	important_bytes=tmp.count('\x00')+tmp.count('\xFF')+tmp.count('\x8B')
	stat[i]=important_bytes
    sorted_stat = sorted(stat.iteritems(), key=operator.itemgetter(1), reverse=True)
    print sorted_stat[0]
\end{lstlisting}

(Source code can be downloaded \href{\GitHubBlobMasterURL/examples/simple_exec_crypto/files/decrypt.py}{here}.)

I run it and here is a key for which 00/FF/8B bytes presence in decrypted buffer is maximal:

\begin{lstlisting}
N= 0
(147, 1224)
N= 1
(94, 1327)
N= 2
(252, 1223)
N= 3
(218, 1266)
N= 4
(38, 1209)
N= 5
(192, 1378)
N= 6
(199, 1204)
N= 7
(213, 1332)
N= 8
(225, 1251)
N= 9
(112, 1223)
N= 10
(143, 1177)
N= 11
(108, 1286)
N= 12
(10, 1164)
N= 13
(3, 1271)
N= 14
(128, 1253)
N= 15
(232, 1330)
\end{lstlisting}

Let's write decryption utility with the key we got:

\begin{lstlisting}[style=custompy]
#!/usr/bin/env python
import sys, hexdump, array

def xor_strings(s,t):
    # \verb|https://en.wikipedia.org/wiki/XOR_cipher#Example_implementation|
    """xor two strings together"""
    return "".join(chr(ord(a)^ord(b)) for a,b in zip(s,t))

IV=array.array('B', [147, 94, 252, 218, 38, 192, 199, 213, 225, 112, 143, 108, 10, 3, 128, 232]).tostring()

def chunks(l, n):
    n = max(1, n)
    return [l[i:i + n] for i in range(0, len(l), n)]

def read_file(fname):
    file=open(fname, mode='rb')
    content=file.read()
    file.close()
    return content

def decrypt_byte(i, k):
    return chr ((ord(i)-k) % 256)

def decrypt(buf):
    return "".join(decrypt_byte(buf[i], i) for i in range(16))

fout=open(sys.argv[2], mode='wb')

prev=IV
content=read_file(sys.argv[1])
tmp=chunks(content, 16)
for c in tmp:
    new_c=decrypt(c)
    p=xor_strings (new_c, prev)
    prev=xor_strings(c, p)
    fout.write(p)
fout.close()
\end{lstlisting}

(Source code can be downloaded \href{\GitHubBlobMasterURL/examples/simple_exec_crypto/files/decrypt2.py}{here}.)

Let's check resulting file:

\lstinputlisting{examples/simple_exec_crypto/objdump_result.txt}

Yes, this is seems correctly disassembled piece of x86 code.
The whole decryped file can be downloaded \href{\GitHubBlobMasterURL/examples/simple_exec_crypto/files/decrypted.bin}{here}.

In fact, this is text section from regedit.exe from Windows 7.
But this example is based on a real case I encountered, so just executable is different (and key), algorithm is the same.

\subsection{Other ideas to consider}

What if I would fail with such simple frequency analysis?
There are other ideas on how to measure correctness of decrypted/decompressed x86 code:

\begin{itemize}

\item Many modern compilers aligns functions on 0x10 border.
So the space left before is filled with NOPs (0x90) or other NOP instructions with known opcodes: \myref{sec:npad}.

\item Perhaps, the most frequent pattern in any assembly language is function call:\\
\TT{PUSH chain / CALL / ADD ESP, X}.
This sequence can easily detected and found.
I've even gathered statistics about average number of function arguments: \myref{args_stat}.
(Hence, this is average length of PUSH chain.)

\end{itemize}

Read more about incorrectly/correctly disassembled code: \myref{ISA_detect}.
}\RU{\subsection{Простое шифрование используя XOR-маску}
\label{XOR_mask_1}

Я нашел одну старую игру в стиле interactive fiction в архиве \emph{if-archive}\footnote{\url{http://www.ifarchive.org/}}:

\begin{lstlisting}
The New Castle v3.5 - Text/Adventure Game
in the style of the original Infocom (tm)
type games, Zork, Collosal Cave (Adventure),
etc.  Can you solve the mystery of the
abandoned castle?
Shareware from Software Customization.
Software Customization [ASP] Version 3.5 Feb. 2000
\end{lstlisting}

Можно скачать здесь: \url{\GitHubBlobMasterURL/ff/XOR/mask_1/files/newcastle.tgz}.

Там внутри есть файл (с названием \emph{castle.dbf}), который явно зашифрован, но не настоящим криптоалгоритмом,
и оне сжат, это что-то куда проще.
Я бы даже не стал измерять уровень энтропии (\myref{entropy}) этого файла, потому что я итак уверен, что он низкий.
Вот как он выглядит в Midnight Commander:

\begin{figure}[H]
\centering
\myincludegraphics{ff/XOR/mask_1/mc_encrypted.png}
\caption{Зашифрованный файл в Midnight Commander}
\end{figure}

Зашифрованный файл можно скачать здесь:
\url{\GitHubBlobMasterURL/ff/XOR/mask_1/files/castle.dbf.bz2}.

Можно ли расшифровать его без доступа к программе, используя просто этот файл?

Тут явно просматривается повторяющаяся строка. 
Если использовалось простое шифрование с XOR-маской, такие повторяющиеся строки это явное свидетельство,
потому что, вероятно, тут были длинные лакуны с нулевыми байтами, которые, в свою очередь, присутствуют
во мноигих исполняемых файлах, и в остальных бинарных файлах.

\myindex{UNIX!xxd}
Вот дам начала этого файла используя утилиту \emph{xxd} из UNIX:

\lstinputlisting{ff/XOR/mask_1/xxd_result.txt}

Давайте держаться за повторяющуюся строку \TT{iubgv}.
Глядя на этот дамп, мы можем легко увидеть, что период повторений этой строки это 0x51 или 81.
Вероятно, 81 это длина блока?
Длина файла 1658961, и она может быть поделена на 81 без остатка (и тогда там 20481 блоков).

Теперь я буду использовать Mathematica для анализа, есть ли тут повторяющиеся 81-байтные блоки в файле?
Я разделю входной файл на 81-байтные блоки и затем использую ф-цию
\emph{Tally[]}\footnote{\url{https://reference.wolfram.com/language/ref/Tally.html}}
которая просто считает, сколько раз каждый элемент встретился во входном списке.
Вывод Tally не отсортирован, так что я также добавлю ф-цию \emph{Sort[]} для сортировки его по кол-ву вхождений
в нисходящем порядке.

\begin{lstlisting}[style=custommath]
input = BinaryReadList["/home/dennis/.../castle.dbf"];

blocks = Partition[input, 81];

stat = Sort[Tally[blocks], #1[[2]] > #2[[2]] &]
\end{lstlisting}

И вот вывод:

\begin{lstlisting}[style=custommath]
{{{80, 103, 2, 116, 113, 102, 118, 25, 99, 8, 19, 23, 116, 125, 107, 
   25, 99, 109, 114, 102, 14, 121, 115, 31, 9, 117, 113, 111, 5, 4, 
   127, 28, 122, 101, 8, 110, 14, 18, 124, 106, 16, 20, 104, 119, 8, 
   109, 26, 106, 9, 97, 13, 99, 15, 119, 20, 105, 117, 98, 103, 118, 
   1, 126, 29, 97, 122, 17, 15, 114, 110, 3, 5, 125, 125, 99, 126, 
   119, 102, 30, 122, 2, 117}, 1739}, 
{{80, 100, 2, 116, 113, 102, 118, 25, 99, 8, 19, 23, 116, 
   125, 107, 25, 99, 109, 114, 102, 14, 121, 115, 31, 9, 117, 113, 
   111, 5, 4, 127, 28, 122, 101, 8, 110, 14, 18, 124, 106, 16, 20, 
   104, 119, 8, 109, 26, 106, 9, 97, 13, 99, 15, 119, 20, 105, 117, 
   98, 103, 118, 1, 126, 29, 97, 122, 17, 15, 114, 110, 3, 5, 125, 
   125, 99, 126, 119, 102, 30, 122, 2, 117}, 1422}, 
{{80, 101, 2, 116, 113, 102, 118, 25, 99, 8, 19, 23, 116, 
   125, 107, 25, 99, 109, 114, 102, 14, 121, 115, 31, 9, 117, 113, 
   111, 5, 4, 127, 28, 122, 101, 8, 110, 14, 18, 124, 106, 16, 20, 
   104, 119, 8, 109, 26, 106, 9, 97, 13, 99, 15, 119, 20, 105, 117, 
   98, 103, 118, 1, 126, 29, 97, 122, 17, 15, 114, 110, 3, 5, 125, 
   125, 99, 126, 119, 102, 30, 122, 2, 117}, 1012},
{{80, 120, 2, 116, 113, 102, 118, 25, 99, 8, 19, 23, 116, 
   125, 107, 25, 99, 109, 114, 102, 14, 121, 115, 31, 9, 117, 113, 
   111, 5, 4, 127, 28, 122, 101, 8, 110, 14, 18, 124, 106, 16, 20, 
   104, 119, 8, 109, 26, 106, 9, 97, 13, 99, 15, 119, 20, 105, 117, 
   98, 103, 118, 1, 126, 29, 97, 122, 17, 15, 114, 110, 3, 5, 125, 
   125, 99, 126, 119, 102, 30, 122, 2, 117}, 377},

...

{{80, 2, 74, 49, 113, 21, 62, 88, 39, 71, 68, 23, 63, 51, 36, 78, 48, 
   108, 114, 102, 14, 121, 115, 31, 9, 117, 113, 111, 5, 4, 127, 28, 
   122, 101, 8, 110, 14, 18, 124, 106, 16, 20, 104, 119, 8, 109, 26, 
   106, 9, 97, 13, 99, 15, 119, 20, 105, 117, 98, 103, 118, 1, 126, 
   29, 97, 122, 17, 15, 114, 110, 3, 5, 125, 125, 99, 126, 119, 102, 
   30, 122, 2, 117}, 1},
{{80, 1, 74, 59, 113, 45, 56, 86, 52, 91, 19, 64, 60, 60, 63, 
   25, 38, 59, 59, 42, 14, 53, 38, 77, 66, 38, 113, 38, 75, 4, 43, 84,
    63, 101, 64, 43, 79, 64, 40, 57, 16, 91, 46, 119, 69, 40, 84, 117,
    9, 97, 13, 99, 15, 119, 20, 105, 117, 98, 103, 118, 1, 126, 29, 
   97, 122, 17, 15, 114, 110, 3, 5, 125, 125, 99, 126, 119, 102, 30, 
   122, 2, 117}, 1},
{{80, 2, 74, 49, 113, 49, 51, 92, 39, 8, 92, 81, 116, 62, 57, 
   80, 46, 40, 114, 36, 75, 56, 33, 76, 9, 55, 56, 59, 81, 65, 45, 28,
    60, 55, 93, 39, 90, 28, 124, 106, 16, 20, 104, 119, 8, 109, 26, 
   106, 9, 97, 13, 99, 15, 119, 20, 105, 117, 98, 103, 118, 1, 126, 
   29, 97, 122, 17, 15, 114, 110, 3, 5, 125, 125, 99, 126, 119, 102, 
   30, 122, 2, 117}, 1}}
\end{lstlisting}

Вывод Tally это список пар, каждая пара это 81-байтный блок и количество раз, сколько он встретился в файле.
Мы видим, что наиболее частно встречающийся блок это первый, он встретился 1739 раз.
Второй встретился 1422 раза. Есть и другие: 1012 раза, 377 раз, итд.
81-байтные блоки, встреченные лишь один раз, находятся в конце вывода.

Попробуем сравнить эти блоки. Первый и второй.
Есть ли в Mathematica ф-ция для сравнения списков/массивов?
Наверняка есть, но в педагогических целях, я буду использоват операцию XOR для сравнения.
Действительно: если байты во входных массивах равны друг другу, результат операции XOR это 0.
Если не равны, результат будет ненулевой.

Сравним первый блок (встречается 1739 раз) и второй (встречается 1422 раз):

\begin{lstlisting}[style=custommath]
In[]:= BitXor[stat[[1]][[1]], stat[[2]][[1]]]
Out[]= {0, 3, 0, 0, 0, 0, 0, 0, 0, 0, 0, 0, 0, 0, 0, 0, 0, 0, 0, \
0, 0, 0, 0, 0, 0, 0, 0, 0, 0, 0, 0, 0, 0, 0, 0, 0, 0, 0, 0, 0, 0, 0, \
0, 0, 0, 0, 0, 0, 0, 0, 0, 0, 0, 0, 0, 0, 0, 0, 0, 0, 0, 0, 0, 0, 0, \
0, 0, 0, 0, 0, 0, 0, 0, 0, 0, 0, 0, 0, 0, 0, 0}
\end{lstlisting}

Они отличаются только вторым байтом.

Сравним второй блок (встречается 1422 раза) и третий (встречается 1012 раз):

\begin{lstlisting}[style=custommath]
In[]:= BitXor[stat[[2]][[1]], stat[[3]][[1]]]
Out[]= {0, 1, 0, 0, 0, 0, 0, 0, 0, 0, 0, 0, 0, 0, 0, 0, 0, 0, 0, \
0, 0, 0, 0, 0, 0, 0, 0, 0, 0, 0, 0, 0, 0, 0, 0, 0, 0, 0, 0, 0, 0, 0, \
0, 0, 0, 0, 0, 0, 0, 0, 0, 0, 0, 0, 0, 0, 0, 0, 0, 0, 0, 0, 0, 0, 0, \
0, 0, 0, 0, 0, 0, 0, 0, 0, 0, 0, 0, 0, 0, 0, 0}
\end{lstlisting}

Они тоже отличаются только вторым байтом.

Так или иначе, попробуем использовать самый встречающийся блок как XOR-ключ и попробуем расшифровать первые 4 81-байтных
блока в файле:

\begin{lstlisting}[style=custommath]
In[]:= key = stat[[1]][[1]]
Out[]= {80, 103, 2, 116, 113, 102, 118, 25, 99, 8, 19, 23, 116, \
125, 107, 25, 99, 109, 114, 102, 14, 121, 115, 31, 9, 117, 113, 111, \
5, 4, 127, 28, 122, 101, 8, 110, 14, 18, 124, 106, 16, 20, 104, 119, \
8, 109, 26, 106, 9, 97, 13, 99, 15, 119, 20, 105, 117, 98, 103, 118, \
1, 126, 29, 97, 122, 17, 15, 114, 110, 3, 5, 125, 125, 99, 126, 119, \
102, 30, 122, 2, 117}

In[]:= ToASCII[val_] := If[val == 0, " ", FromCharacterCode[val, "PrintableASCII"]]

In[]:= DecryptBlockASCII[blk_] := Map[ToASCII[#] &, BitXor[key, blk]]

In[]:= DecryptBlockASCII[blocks[[1]]]
Out[]= {" ", " ", " ", " ", " ", " ", " ", " ", " ", " ", " ", " \
", " ", " ", " ", " ", " ", " ", " ", " ", " ", " ", " ", " ", " ", " \
", " ", " ", " ", " ", " ", " ", " ", " ", " ", " ", " ", " ", " ", " \
", " ", " ", " ", " ", " ", " ", " ", " ", " ", " ", " ", " ", " ", " \
", " ", " ", " ", " ", " ", " ", " ", " ", " ", " ", " ", " ", " ", " \
", " ", " ", " ", " ", " ", " ", " ", " ", " ", " ", " ", " ", " "}

In[]:= DecryptBlockASCII[blocks[[2]]]
Out[]= {" ", "e", "H", "E", " ", "W", "E", "E", "D", " ", "O", \
"F", " ", "C", "R", "I", "M", "E", " ", "B", "E", "A", "R", "S", " ", \
"B", "I", "T", "T", "E", "R", " ", "F", "R", "U", "I", "T", "?", \
" ", " ", " ", " ", " ", " ", " ", " ", " ", " ", " ", " ", " ", " ", \
" ", " ", " ", " ", " ", " ", " ", " ", " ", " ", " ", " ", " ", " ", \
" ", " ", " ", " ", " ", " ", " ", " ", " ", " ", " ", " ", " ", " ", \
" "}

In[]:= DecryptBlockASCII[blocks[[3]]]
Out[]= {" ", "?", " ", " ", " ", " ", " ", " ", " ", " ", " \
", " ", " ", " ", " ", " ", " ", " ", " ", " ", " ", " ", " ", " ", " \
", " ", " ", " ", " ", " ", " ", " ", " ", " ", " ", " ", " ", " ", " \
", " ", " ", " ", " ", " ", " ", " ", " ", " ", " ", " ", " ", " ", " \
", " ", " ", " ", " ", " ", " ", " ", " ", " ", " ", " ", " ", " ", " \
", " ", " ", " ", " ", " ", " ", " ", " ", " ", " ", " ", " ", " ", " \
"}

In[]:= DecryptBlockASCII[blocks[[4]]]
Out[]= {" ", "f", "H", "O", " ", "K", "N", "O", "W", "S", " ", \
"W", "H", "A", "T", " ", "E", "V", "I", "L", " ", "L", "U", "R", "K", \
"S", " ", "I", "N", " ", "T", "H", "E", " ", "H", "E", "A", "R", "T", \
"S", " ", "O", "F", " ", "M", "E", "N", "?", " ", " ", " ", " ", \
" ", " ", " ", " ", " ", " ", " ", " ", " ", " ", " ", " ", " ", " ", \
" ", " ", " ", " ", " ", " ", " ", " ", " ", " ", " ", " ", " ", " ", \
" "}
\end{lstlisting}

(Я заменил непечатаемые символы на \q{?}.)

Мы видим что первый и третий блоки пустые (или почти пустые),
но второй и четвертый имеют ясно различимые английские слова/фразы.
Похоже что наше предположение насчет ключа верно (как минимум частично).
Это означает, что самый встречающийся 81-байтный блок в файле находится в местах лакун с нулевыми байтами
или что-то в этом роде.

Попробуем расшифровать весь файл:

\begin{lstlisting}[style=custommath]
DecryptBlock[blk_] := BitXor[key, blk]

decrypted = Map[DecryptBlock[#] &, blocks];

BinaryWrite["/home/dennis/.../tmp", Flatten[decrypted]]

Close["/home/dennis/.../tmp"]
\end{lstlisting}

\begin{figure}[H]
\centering
\myincludegraphics{ff/XOR/mask_1/mc_decrypted1.png}
\caption{Расшифрованный файл в Midnight Commander, первая попытка}
\end{figure}

Выглядит как английские фразы для какой-то игры, но что-то не так.
Прежде всего, регистр инвертирован: фразы и некоторые слова начинаются со строчных букв,
в то время как остальные буквы заглавные.
Также, некоторые фразы начинаются с не тех букв.
Посмотрите на самую первую фразу: \q{eHE WEED OF CRIME BEARS BITTER FRUIT}.
Что такое \q{eHE}? Разве не \q{tHE} тут должно быть?
Возможно ли что наш ключ для дешифрования имеет неверный байт в этом месте?

Посмотрим снова на второй блок в файле, на ключ и на результат дешифрования:

\begin{lstlisting}[style=custommath]
In[]:= blocks[[2]]
Out[]= {80, 2, 74, 49, 113, 49, 51, 92, 39, 8, 92, 81, 116, 62, \
57, 80, 46, 40, 114, 36, 75, 56, 33, 76, 9, 55, 56, 59, 81, 65, 45, \
28, 60, 55, 93, 39, 90, 28, 124, 106, 16, 20, 104, 119, 8, 109, 26, \
106, 9, 97, 13, 99, 15, 119, 20, 105, 117, 98, 103, 118, 1, 126, 29, \
97, 122, 17, 15, 114, 110, 3, 5, 125, 125, 99, 126, 119, 102, 30, \
122, 2, 117}

In[]:= key
Out[]= {80, 103, 2, 116, 113, 102, 118, 25, 99, 8, 19, 23, 116, \
125, 107, 25, 99, 109, 114, 102, 14, 121, 115, 31, 9, 117, 113, 111, \
5, 4, 127, 28, 122, 101, 8, 110, 14, 18, 124, 106, 16, 20, 104, 119, \
8, 109, 26, 106, 9, 97, 13, 99, 15, 119, 20, 105, 117, 98, 103, 118, \
1, 126, 29, 97, 122, 17, 15, 114, 110, 3, 5, 125, 125, 99, 126, 119, \
102, 30, 122, 2, 117}

In[]:= BitXor[key, blocks[[2]]]
Out[]= {0, 101, 72, 69, 0, 87, 69, 69, 68, 0, 79, 70, 0, 67, 82, \
73, 77, 69, 0, 66, 69, 65, 82, 83, 0, 66, 73, 84, 84, 69, 82, 0, 70, \
82, 85, 73, 84, 14, 0, 0, 0, 0, 0, 0, 0, 0, 0, 0, 0, 0, 0, 0, 0, 0, \
0, 0, 0, 0, 0, 0, 0, 0, 0, 0, 0, 0, 0, 0, 0, 0, 0, 0, 0, 0, 0, 0, 0, \
0, 0, 0, 0}
\end{lstlisting}

Зашифрованный байт это 2, байт из ключа это 103, $2 \oplus 103=101$ и 101 это ASCII-код символа \q{e}.
Чему должен равнятся этот байт ключа, чтобы ASCII-код был 116 (для символа  \q{t})?
$2 \oplus 116=118$, присвоим 118 второму байту в ключе \dots

\begin{lstlisting}[style=custommath]
key = {80, 118, 2, 116, 113, 102, 118, 25, 99, 8, 19, 23, 116, 125, 
  107, 25, 99, 109, 114, 102, 14, 121, 115, 31, 9, 117, 113, 111, 5, 
  4, 127, 28, 122, 101, 8, 110, 14, 18, 124, 106, 16, 20, 104, 119, 8,
   109, 26, 106, 9, 97, 13, 99, 15, 119, 20, 105, 117, 98, 103, 118, 
  1, 126, 29, 97, 122, 17, 15, 114, 110, 3, 5, 125, 125, 99, 126, 119,
   102, 30, 122, 2, 117}
\end{lstlisting}

\dots и снова дешифруем весь файл.

\begin{figure}[H]
\centering
\myincludegraphics{ff/XOR/mask_1/mc_decrypted2.png}
\caption{Дешифрованный файл в Midnight Commander, вторая попытка}
\end{figure}

Ух ты, теперь грамматика корректна, и все фразы начинаются с корректных букв.
Но все таки, регистр подозрителен.
С чего бы разработчику игры записывать их в такой манере?
Может быть наш ключ все еще неправилен?

% TODO ASCII table somewhere in the book
Изучая таблицу ASCII мы можем заметить что ASCII-коды для букв в верхнем и нижнем регистре отличаются только на один бит
(6-й бит, если считать с первого, 0b100000):

\begin{figure}[H]
\centering
\includegraphics[width=0.7\textwidth]{ascii.png}
\caption{7-битная таблица \ac{ASCII} в Emacs}
\end{figure}

6-й бит, выставленный в нулевом байте, В десятичном виде это будет 32.
Но 32 это ASCII-код пробела!

Действительно, можно менять регистр просто применяя XOR к ASCII-коду, с 32 (больше об этом: \myref{toupper_bit}).

Возможно ли, что пустые лакуны в файле это не нулевые байты, а скорее содержащие пробелы?
Еще раз модифицируем наш XOR-ключ (я про-XOR-ю каждый байт ключа с 32):

\begin{lstlisting}[style=custommath]
(* "32" это скаляр, и "key" это вектор, но это OK *)

In[]:= key3 = BitXor[32, key]
Out[]= {112, 86, 34, 84, 81, 70, 86, 57, 67, 40, 51, 55, 84, 93, 75, \
57, 67, 77, 82, 70, 46, 89, 83, 63, 41, 85, 81, 79, 37, 36, 95, 60, \
90, 69, 40, 78, 46, 50, 92, 74, 48, 52, 72, 87, 40, 77, 58, 74, 41, \
65, 45, 67, 47, 87, 52, 73, 85, 66, 71, 86, 33, 94, 61, 65, 90, 49, \
47, 82, 78, 35, 37, 93, 93, 67, 94, 87, 70, 62, 90, 34, 85}

In[]:= DecryptBlock[blk_] := BitXor[key3, blk]
\end{lstlisting}

И снова дешифруем входной файл:

\begin{figure}[H]
\centering
\myincludegraphics{ff/XOR/mask_1/mc_decrypted.png}
\caption{Дешифрованный файл в Midnight Commander, последняя попытка}
\end{figure}

(Расшифрованный файл доступен здесь:
\url{\GitHubBlobMasterURL/ff/XOR/mask_1/files/decrypted.dat.bz2}.)

Несомненно, это корректный исходный файл.
Да, и мы видим числа в начале каждого блока. Должно быть это и есть источник некорректного XOR-ключа.
Как выходит, самый встречающийся 81-байтный блок в файле это блок заполненный пробелами и содержащий символ \q{1} на месте
второго байта.
Действительно, как-то так получилось что многие блоки здесь перемежаются с этим блоком.
Может быть это что-то вроде выравнивания (padding) для коротких фраз/сообщений?
Другой часто встречающийся 81-байтный блок также заполнен пробелами, но с другой цифрой, следовательно,
они отличаются только вторым байтом.

Вот и всё! Теперь мы можем написать утилиту для зашифрования файла назад, и, может быть, модифицировать его перед этим

Файл для Mathematica можно скачать здесь:\\
\url{\GitHubBlobMasterURL/ff/XOR/mask_1/files/XOR_mask_1.nb}.

Итог: XOR-шифрование не надежно вообще. Вероятно, разработчик игры хотел просто скрыть внутренности игры от игрока,
ничего более серьезного.
Все же, шифрование вроде этого крайне популярно вследствии его простоты, так что многие реверс инженеры обычно хорошо
с этим знакомы.

}\FR{\mysection{Fonction presque vide}
\label{Boolector}
\myindex{Boolector}
\myindex{x86!\Instructions!JMP}

Ceci est un morceau de code réel que j'ai trouvé dans Boolector\footnote{\url{https://boolector.github.io/}}:

\lstinputlisting[style=customc]{patterns/025_almost_empty/boolectormain.c}

Pourquoi quelqu'un ferait-il comme ça?
Je ne sais pas mais mon hypothèse est que \verb|boolector_main()| peut être compilée
dans une sorte de DLL ou bibliothèque dynamique, et appelée depuis une suite de test.
Certainement qu'une suite de test peut préparer les variables argc/argv comme
le ferait \ac{CRT}.

Il est intéressant de voir comment c'est compilé:

\lstinputlisting[caption=GCC 8.2 x64 \NonOptimizing (\assemblyOutput),style=customasmx86]{patterns/025_almost_empty/boolectormain_O0.s}

Ceci est OK, le prologue (non optimisé) déplace inutilement deux arguments,
\INS{CALL}, épilogue, \INS{RET}.
Mais regardons la version optimisée:

\lstinputlisting[caption=GCC 8.2 x64 \Optimizing (\assemblyOutput),style=customasmx86]{patterns/025_almost_empty/boolectormain_O3.s}

Aussi simple que ça: la pile et les registres ne sont pas touchés et \verb|boolector_main()|
a le même ensemble d'arguments.
Donc, tout ce que nous avons à faire est de passer l'exécution à une autre adresse.

Ceci est proche d'une \glslink{thunk function}{fonction thunk}.

Nous verons queelque chose de plus avancé plus tard: \myref{ARM_B_to_printf}, \myref{JMP_instead_of_RET}.
}

\renewcommand{\CURPATH}{advanced/102_fib}
\EN{\EN{\input{patterns/016_empty_redux/main_EN}}%
\FR{\input{patterns/016_empty_redux/main_FR}}
}\RU{\EN{\input{patterns/016_empty_redux/main_EN}}%
\FR{\input{patterns/016_empty_redux/main_FR}}
}\FR{\EN{\input{patterns/016_empty_redux/main_EN}}%
\FR{\input{patterns/016_empty_redux/main_FR}}
}

\renewcommand{\CURPATH}{advanced/110_CRC32}
\EN{% TODO translate
\mysection{Breaking simple executable cryptor}

I've got an executable file which is encrypted by relatively simple encryption.
\href{\GitHubBlobMasterURL/examples/simple_exec_crypto/files/cipher.bin}{Here is it} (only executable section is left here).

First, all encryption function does is just adds number of position in buffer to the byte.
Here is how this can be encoded in Python:

\begin{lstlisting}[caption=Python script,style=custompy]
#!/usr/bin/env python
def e(i, k):
    return chr ((ord(i)+k) % 256)

def encrypt(buf):
    return e(buf[0], 0)+ e(buf[1], 1)+ e(buf[2], 2) + e(buf[3], 3)+ e(buf[4], 4)+ e(buf[5], 5)+ e(buf[6], 6)+ e(buf[7], 7)+
           e(buf[8], 8)+ e(buf[9], 9)+ e(buf[10], 10)+ e(buf[11], 11)+ e(buf[12], 12)+ e(buf[13], 13)+ e(buf[14], 14)+ e(buf[15], 15)
\end{lstlisting}

Hence, if you encrypt buffer with 16 zeros, you'll get \emph{0, 1, 2, 3 ... 12, 13, 14, 15}.

\myindex{Propagating Cipher Block Chaining}
Propagating Cipher Block Chaining (PCBC) is also used, here is how it works:

\begin{figure}[H]
\centering
\myincludegraphics{examples/simple_exec_crypto/601px-PCBC_encryption.png}
\caption{Propagating Cipher Block Chaining encryption (image is taken from Wikipedia article)}
\end{figure}

The problem is that it's too boring to recover IV (Initialization Vector) each time.
Brute-force is also not an option, because IV is too long (16 bytes).
Let's see, if it's possible to recover IV for arbitrary encrypted executable file?

Let's try simple frequency analysis.
This is 32-bit x86 executable code, so let's gather statistics about most frequent bytes and opcodes.
I tried huge oracle.exe file from Oracle RDBMS version 11.2 for windows x86 and I've found that the most frequent byte (no surprise) is zero (~10\%).
The next most frequent byte is (again, no surprise) 0xFF (~5\%).
The next is 0x8B (~5\%).

\myindex{x86!\Instructions!MOV}
0x8B is opcode for \INS{MOV}, this is indeed one of the most busy x86 instructions.
Now what about popularity of zero byte?
If compiler needs to encode value bigger than 127, it has to use 32-bit displacement instead of 8-bit one, but large values are very rare,
so it is padded by zeros.
\myindex{x86!\Instructions!LEA}
\myindex{x86!\Instructions!PUSH}
\myindex{x86!\Instructions!CALL}
This is at least in \INS{LEA}, \INS{MOV}, \INS{PUSH}, \INS{CALL}.

For example:

\begin{lstlisting}[style=customasmx86]
8D B0 28 01 00 00                 lea     esi, [eax+128h]
8D BF 40 38 00 00                 lea     edi, [edi+3840h]
\end{lstlisting}

Displacements bigger than 127 are very popular, but they are rarely exceeds 0x10000
(indeed, such large memory buffers/structures are also rare).

Same story with \INS{MOV}, large constants are rare, the most heavily used are 0, 1, 10, 100, $2^n$, and so on.
Compiler has to pad small constants by zeros to represent them as 32-bit values:

\begin{lstlisting}[style=customasmx86]
BF 02 00 00 00                    mov     edi, 2
BF 01 00 00 00                    mov     edi, 1
\end{lstlisting}

Now about 00 and FF bytes combined: jumps (including conditional) and calls can pass execution flow forward or backwards, but very often,
within the limits of the current executable module.
If forward, displacement is not very big and also padded with zeros.
If backwards, displacement is represented as negative value, so padded with FF bytes.
For example, transfer execution flow forward:

\begin{lstlisting}[style=customasmx86]
E8 43 0C 00 00                    call    _function1
E8 5C 00 00 00                    call    _function2
0F 84 F0 0A 00 00                 jz      loc_4F09A0
0F 84 EB 00 00 00                 jz      loc_4EFBB8
\end{lstlisting}

Backwards:

\begin{lstlisting}[style=customasmx86]
E8 79 0C FE FF                    call    _function1
E8 F4 16 FF FF                    call    _function2
0F 84 F8 FB FF FF                 jz      loc_8212BC
0F 84 06 FD FF FF                 jz      loc_FF1E7D
\end{lstlisting}

FF byte is also very often occurred in negative displacements like these:

\begin{lstlisting}[style=customasmx86]
8D 85 1E FF FF FF                 lea     eax, [ebp-0E2h]
8D 95 F8 5C FF FF                 lea     edx, [ebp-0A308h]
\end{lstlisting}

So far so good. Now we have to try various 16-byte keys, decrypt executable section and measure how often 00, FF and 8B bytes are occurred.
Let's also keep in sight how PCBC decryption works:

\begin{figure}[H]
\centering
\myincludegraphics{examples/simple_exec_crypto/640px-PCBC_decryption.png}
\caption{Propagating Cipher Block Chaining decryption (image is taken from Wikipedia article)}
\end{figure}

The good news is that we don't really have to decrypt whole piece of data, but only slice by slice, this is exactly how I did in my previous example: \myref{XOR_mask_2}.

Now I'm trying all possible bytes (0..255) for each byte in key and just pick the byte producing maximal amount of 00/FF/8B bytes in a decrypted slice:

\begin{lstlisting}[style=custompy]
#!/usr/bin/env python
import sys, hexdump, array, string, operator

KEY_LEN=16

def chunks(l, n):
    # split n by l-byte chunks
    # https://stackoverflow.com/q/312443
    n = max(1, n)
    return [l[i:i + n] for i in range(0, len(l), n)]

def read_file(fname):
    file=open(fname, mode='rb')
    content=file.read()
    file.close()
    return content

def decrypt_byte (c, key):
    return chr((ord(c)-key) % 256)

def XOR_PCBC_step (IV, buf, k):
    prev=IV
    rt=""
    for c in buf:
	new_c=decrypt_byte(c, k)
        plain=chr(ord(new_c)^ord(prev))
	prev=chr(ord(c)^ord(plain))
	rt=rt+plain
    return rt

each_Nth_byte=[""]*KEY_LEN

content=read_file(sys.argv[1])
# split input by 16-byte chunks:
all_chunks=chunks(content, KEY_LEN)
for c in all_chunks:
    for i in range(KEY_LEN):
        each_Nth_byte[i]=each_Nth_byte[i] + c[i]

# try each byte of key
for N in range(KEY_LEN):
    print "N=", N
    stat={}
    for i in range(256):
        tmp_key=chr(i)
	tmp=XOR_PCBC_step(tmp_key,each_Nth_byte[N], N)
        # count 0, FFs and 8Bs in decrypted buffer:
	important_bytes=tmp.count('\x00')+tmp.count('\xFF')+tmp.count('\x8B')
	stat[i]=important_bytes
    sorted_stat = sorted(stat.iteritems(), key=operator.itemgetter(1), reverse=True)
    print sorted_stat[0]
\end{lstlisting}

(Source code can be downloaded \href{\GitHubBlobMasterURL/examples/simple_exec_crypto/files/decrypt.py}{here}.)

I run it and here is a key for which 00/FF/8B bytes presence in decrypted buffer is maximal:

\begin{lstlisting}
N= 0
(147, 1224)
N= 1
(94, 1327)
N= 2
(252, 1223)
N= 3
(218, 1266)
N= 4
(38, 1209)
N= 5
(192, 1378)
N= 6
(199, 1204)
N= 7
(213, 1332)
N= 8
(225, 1251)
N= 9
(112, 1223)
N= 10
(143, 1177)
N= 11
(108, 1286)
N= 12
(10, 1164)
N= 13
(3, 1271)
N= 14
(128, 1253)
N= 15
(232, 1330)
\end{lstlisting}

Let's write decryption utility with the key we got:

\begin{lstlisting}[style=custompy]
#!/usr/bin/env python
import sys, hexdump, array

def xor_strings(s,t):
    # \verb|https://en.wikipedia.org/wiki/XOR_cipher#Example_implementation|
    """xor two strings together"""
    return "".join(chr(ord(a)^ord(b)) for a,b in zip(s,t))

IV=array.array('B', [147, 94, 252, 218, 38, 192, 199, 213, 225, 112, 143, 108, 10, 3, 128, 232]).tostring()

def chunks(l, n):
    n = max(1, n)
    return [l[i:i + n] for i in range(0, len(l), n)]

def read_file(fname):
    file=open(fname, mode='rb')
    content=file.read()
    file.close()
    return content

def decrypt_byte(i, k):
    return chr ((ord(i)-k) % 256)

def decrypt(buf):
    return "".join(decrypt_byte(buf[i], i) for i in range(16))

fout=open(sys.argv[2], mode='wb')

prev=IV
content=read_file(sys.argv[1])
tmp=chunks(content, 16)
for c in tmp:
    new_c=decrypt(c)
    p=xor_strings (new_c, prev)
    prev=xor_strings(c, p)
    fout.write(p)
fout.close()
\end{lstlisting}

(Source code can be downloaded \href{\GitHubBlobMasterURL/examples/simple_exec_crypto/files/decrypt2.py}{here}.)

Let's check resulting file:

\lstinputlisting{examples/simple_exec_crypto/objdump_result.txt}

Yes, this is seems correctly disassembled piece of x86 code.
The whole decryped file can be downloaded \href{\GitHubBlobMasterURL/examples/simple_exec_crypto/files/decrypted.bin}{here}.

In fact, this is text section from regedit.exe from Windows 7.
But this example is based on a real case I encountered, so just executable is different (and key), algorithm is the same.

\subsection{Other ideas to consider}

What if I would fail with such simple frequency analysis?
There are other ideas on how to measure correctness of decrypted/decompressed x86 code:

\begin{itemize}

\item Many modern compilers aligns functions on 0x10 border.
So the space left before is filled with NOPs (0x90) or other NOP instructions with known opcodes: \myref{sec:npad}.

\item Perhaps, the most frequent pattern in any assembly language is function call:\\
\TT{PUSH chain / CALL / ADD ESP, X}.
This sequence can easily detected and found.
I've even gathered statistics about average number of function arguments: \myref{args_stat}.
(Hence, this is average length of PUSH chain.)

\end{itemize}

Read more about incorrectly/correctly disassembled code: \myref{ISA_detect}.
}\RU{\subsection{Простое шифрование используя XOR-маску}
\label{XOR_mask_1}

Я нашел одну старую игру в стиле interactive fiction в архиве \emph{if-archive}\footnote{\url{http://www.ifarchive.org/}}:

\begin{lstlisting}
The New Castle v3.5 - Text/Adventure Game
in the style of the original Infocom (tm)
type games, Zork, Collosal Cave (Adventure),
etc.  Can you solve the mystery of the
abandoned castle?
Shareware from Software Customization.
Software Customization [ASP] Version 3.5 Feb. 2000
\end{lstlisting}

Можно скачать здесь: \url{\GitHubBlobMasterURL/ff/XOR/mask_1/files/newcastle.tgz}.

Там внутри есть файл (с названием \emph{castle.dbf}), который явно зашифрован, но не настоящим криптоалгоритмом,
и оне сжат, это что-то куда проще.
Я бы даже не стал измерять уровень энтропии (\myref{entropy}) этого файла, потому что я итак уверен, что он низкий.
Вот как он выглядит в Midnight Commander:

\begin{figure}[H]
\centering
\myincludegraphics{ff/XOR/mask_1/mc_encrypted.png}
\caption{Зашифрованный файл в Midnight Commander}
\end{figure}

Зашифрованный файл можно скачать здесь:
\url{\GitHubBlobMasterURL/ff/XOR/mask_1/files/castle.dbf.bz2}.

Можно ли расшифровать его без доступа к программе, используя просто этот файл?

Тут явно просматривается повторяющаяся строка. 
Если использовалось простое шифрование с XOR-маской, такие повторяющиеся строки это явное свидетельство,
потому что, вероятно, тут были длинные лакуны с нулевыми байтами, которые, в свою очередь, присутствуют
во мноигих исполняемых файлах, и в остальных бинарных файлах.

\myindex{UNIX!xxd}
Вот дам начала этого файла используя утилиту \emph{xxd} из UNIX:

\lstinputlisting{ff/XOR/mask_1/xxd_result.txt}

Давайте держаться за повторяющуюся строку \TT{iubgv}.
Глядя на этот дамп, мы можем легко увидеть, что период повторений этой строки это 0x51 или 81.
Вероятно, 81 это длина блока?
Длина файла 1658961, и она может быть поделена на 81 без остатка (и тогда там 20481 блоков).

Теперь я буду использовать Mathematica для анализа, есть ли тут повторяющиеся 81-байтные блоки в файле?
Я разделю входной файл на 81-байтные блоки и затем использую ф-цию
\emph{Tally[]}\footnote{\url{https://reference.wolfram.com/language/ref/Tally.html}}
которая просто считает, сколько раз каждый элемент встретился во входном списке.
Вывод Tally не отсортирован, так что я также добавлю ф-цию \emph{Sort[]} для сортировки его по кол-ву вхождений
в нисходящем порядке.

\begin{lstlisting}[style=custommath]
input = BinaryReadList["/home/dennis/.../castle.dbf"];

blocks = Partition[input, 81];

stat = Sort[Tally[blocks], #1[[2]] > #2[[2]] &]
\end{lstlisting}

И вот вывод:

\begin{lstlisting}[style=custommath]
{{{80, 103, 2, 116, 113, 102, 118, 25, 99, 8, 19, 23, 116, 125, 107, 
   25, 99, 109, 114, 102, 14, 121, 115, 31, 9, 117, 113, 111, 5, 4, 
   127, 28, 122, 101, 8, 110, 14, 18, 124, 106, 16, 20, 104, 119, 8, 
   109, 26, 106, 9, 97, 13, 99, 15, 119, 20, 105, 117, 98, 103, 118, 
   1, 126, 29, 97, 122, 17, 15, 114, 110, 3, 5, 125, 125, 99, 126, 
   119, 102, 30, 122, 2, 117}, 1739}, 
{{80, 100, 2, 116, 113, 102, 118, 25, 99, 8, 19, 23, 116, 
   125, 107, 25, 99, 109, 114, 102, 14, 121, 115, 31, 9, 117, 113, 
   111, 5, 4, 127, 28, 122, 101, 8, 110, 14, 18, 124, 106, 16, 20, 
   104, 119, 8, 109, 26, 106, 9, 97, 13, 99, 15, 119, 20, 105, 117, 
   98, 103, 118, 1, 126, 29, 97, 122, 17, 15, 114, 110, 3, 5, 125, 
   125, 99, 126, 119, 102, 30, 122, 2, 117}, 1422}, 
{{80, 101, 2, 116, 113, 102, 118, 25, 99, 8, 19, 23, 116, 
   125, 107, 25, 99, 109, 114, 102, 14, 121, 115, 31, 9, 117, 113, 
   111, 5, 4, 127, 28, 122, 101, 8, 110, 14, 18, 124, 106, 16, 20, 
   104, 119, 8, 109, 26, 106, 9, 97, 13, 99, 15, 119, 20, 105, 117, 
   98, 103, 118, 1, 126, 29, 97, 122, 17, 15, 114, 110, 3, 5, 125, 
   125, 99, 126, 119, 102, 30, 122, 2, 117}, 1012},
{{80, 120, 2, 116, 113, 102, 118, 25, 99, 8, 19, 23, 116, 
   125, 107, 25, 99, 109, 114, 102, 14, 121, 115, 31, 9, 117, 113, 
   111, 5, 4, 127, 28, 122, 101, 8, 110, 14, 18, 124, 106, 16, 20, 
   104, 119, 8, 109, 26, 106, 9, 97, 13, 99, 15, 119, 20, 105, 117, 
   98, 103, 118, 1, 126, 29, 97, 122, 17, 15, 114, 110, 3, 5, 125, 
   125, 99, 126, 119, 102, 30, 122, 2, 117}, 377},

...

{{80, 2, 74, 49, 113, 21, 62, 88, 39, 71, 68, 23, 63, 51, 36, 78, 48, 
   108, 114, 102, 14, 121, 115, 31, 9, 117, 113, 111, 5, 4, 127, 28, 
   122, 101, 8, 110, 14, 18, 124, 106, 16, 20, 104, 119, 8, 109, 26, 
   106, 9, 97, 13, 99, 15, 119, 20, 105, 117, 98, 103, 118, 1, 126, 
   29, 97, 122, 17, 15, 114, 110, 3, 5, 125, 125, 99, 126, 119, 102, 
   30, 122, 2, 117}, 1},
{{80, 1, 74, 59, 113, 45, 56, 86, 52, 91, 19, 64, 60, 60, 63, 
   25, 38, 59, 59, 42, 14, 53, 38, 77, 66, 38, 113, 38, 75, 4, 43, 84,
    63, 101, 64, 43, 79, 64, 40, 57, 16, 91, 46, 119, 69, 40, 84, 117,
    9, 97, 13, 99, 15, 119, 20, 105, 117, 98, 103, 118, 1, 126, 29, 
   97, 122, 17, 15, 114, 110, 3, 5, 125, 125, 99, 126, 119, 102, 30, 
   122, 2, 117}, 1},
{{80, 2, 74, 49, 113, 49, 51, 92, 39, 8, 92, 81, 116, 62, 57, 
   80, 46, 40, 114, 36, 75, 56, 33, 76, 9, 55, 56, 59, 81, 65, 45, 28,
    60, 55, 93, 39, 90, 28, 124, 106, 16, 20, 104, 119, 8, 109, 26, 
   106, 9, 97, 13, 99, 15, 119, 20, 105, 117, 98, 103, 118, 1, 126, 
   29, 97, 122, 17, 15, 114, 110, 3, 5, 125, 125, 99, 126, 119, 102, 
   30, 122, 2, 117}, 1}}
\end{lstlisting}

Вывод Tally это список пар, каждая пара это 81-байтный блок и количество раз, сколько он встретился в файле.
Мы видим, что наиболее частно встречающийся блок это первый, он встретился 1739 раз.
Второй встретился 1422 раза. Есть и другие: 1012 раза, 377 раз, итд.
81-байтные блоки, встреченные лишь один раз, находятся в конце вывода.

Попробуем сравнить эти блоки. Первый и второй.
Есть ли в Mathematica ф-ция для сравнения списков/массивов?
Наверняка есть, но в педагогических целях, я буду использоват операцию XOR для сравнения.
Действительно: если байты во входных массивах равны друг другу, результат операции XOR это 0.
Если не равны, результат будет ненулевой.

Сравним первый блок (встречается 1739 раз) и второй (встречается 1422 раз):

\begin{lstlisting}[style=custommath]
In[]:= BitXor[stat[[1]][[1]], stat[[2]][[1]]]
Out[]= {0, 3, 0, 0, 0, 0, 0, 0, 0, 0, 0, 0, 0, 0, 0, 0, 0, 0, 0, \
0, 0, 0, 0, 0, 0, 0, 0, 0, 0, 0, 0, 0, 0, 0, 0, 0, 0, 0, 0, 0, 0, 0, \
0, 0, 0, 0, 0, 0, 0, 0, 0, 0, 0, 0, 0, 0, 0, 0, 0, 0, 0, 0, 0, 0, 0, \
0, 0, 0, 0, 0, 0, 0, 0, 0, 0, 0, 0, 0, 0, 0, 0}
\end{lstlisting}

Они отличаются только вторым байтом.

Сравним второй блок (встречается 1422 раза) и третий (встречается 1012 раз):

\begin{lstlisting}[style=custommath]
In[]:= BitXor[stat[[2]][[1]], stat[[3]][[1]]]
Out[]= {0, 1, 0, 0, 0, 0, 0, 0, 0, 0, 0, 0, 0, 0, 0, 0, 0, 0, 0, \
0, 0, 0, 0, 0, 0, 0, 0, 0, 0, 0, 0, 0, 0, 0, 0, 0, 0, 0, 0, 0, 0, 0, \
0, 0, 0, 0, 0, 0, 0, 0, 0, 0, 0, 0, 0, 0, 0, 0, 0, 0, 0, 0, 0, 0, 0, \
0, 0, 0, 0, 0, 0, 0, 0, 0, 0, 0, 0, 0, 0, 0, 0}
\end{lstlisting}

Они тоже отличаются только вторым байтом.

Так или иначе, попробуем использовать самый встречающийся блок как XOR-ключ и попробуем расшифровать первые 4 81-байтных
блока в файле:

\begin{lstlisting}[style=custommath]
In[]:= key = stat[[1]][[1]]
Out[]= {80, 103, 2, 116, 113, 102, 118, 25, 99, 8, 19, 23, 116, \
125, 107, 25, 99, 109, 114, 102, 14, 121, 115, 31, 9, 117, 113, 111, \
5, 4, 127, 28, 122, 101, 8, 110, 14, 18, 124, 106, 16, 20, 104, 119, \
8, 109, 26, 106, 9, 97, 13, 99, 15, 119, 20, 105, 117, 98, 103, 118, \
1, 126, 29, 97, 122, 17, 15, 114, 110, 3, 5, 125, 125, 99, 126, 119, \
102, 30, 122, 2, 117}

In[]:= ToASCII[val_] := If[val == 0, " ", FromCharacterCode[val, "PrintableASCII"]]

In[]:= DecryptBlockASCII[blk_] := Map[ToASCII[#] &, BitXor[key, blk]]

In[]:= DecryptBlockASCII[blocks[[1]]]
Out[]= {" ", " ", " ", " ", " ", " ", " ", " ", " ", " ", " ", " \
", " ", " ", " ", " ", " ", " ", " ", " ", " ", " ", " ", " ", " ", " \
", " ", " ", " ", " ", " ", " ", " ", " ", " ", " ", " ", " ", " ", " \
", " ", " ", " ", " ", " ", " ", " ", " ", " ", " ", " ", " ", " ", " \
", " ", " ", " ", " ", " ", " ", " ", " ", " ", " ", " ", " ", " ", " \
", " ", " ", " ", " ", " ", " ", " ", " ", " ", " ", " ", " ", " "}

In[]:= DecryptBlockASCII[blocks[[2]]]
Out[]= {" ", "e", "H", "E", " ", "W", "E", "E", "D", " ", "O", \
"F", " ", "C", "R", "I", "M", "E", " ", "B", "E", "A", "R", "S", " ", \
"B", "I", "T", "T", "E", "R", " ", "F", "R", "U", "I", "T", "?", \
" ", " ", " ", " ", " ", " ", " ", " ", " ", " ", " ", " ", " ", " ", \
" ", " ", " ", " ", " ", " ", " ", " ", " ", " ", " ", " ", " ", " ", \
" ", " ", " ", " ", " ", " ", " ", " ", " ", " ", " ", " ", " ", " ", \
" "}

In[]:= DecryptBlockASCII[blocks[[3]]]
Out[]= {" ", "?", " ", " ", " ", " ", " ", " ", " ", " ", " \
", " ", " ", " ", " ", " ", " ", " ", " ", " ", " ", " ", " ", " ", " \
", " ", " ", " ", " ", " ", " ", " ", " ", " ", " ", " ", " ", " ", " \
", " ", " ", " ", " ", " ", " ", " ", " ", " ", " ", " ", " ", " ", " \
", " ", " ", " ", " ", " ", " ", " ", " ", " ", " ", " ", " ", " ", " \
", " ", " ", " ", " ", " ", " ", " ", " ", " ", " ", " ", " ", " ", " \
"}

In[]:= DecryptBlockASCII[blocks[[4]]]
Out[]= {" ", "f", "H", "O", " ", "K", "N", "O", "W", "S", " ", \
"W", "H", "A", "T", " ", "E", "V", "I", "L", " ", "L", "U", "R", "K", \
"S", " ", "I", "N", " ", "T", "H", "E", " ", "H", "E", "A", "R", "T", \
"S", " ", "O", "F", " ", "M", "E", "N", "?", " ", " ", " ", " ", \
" ", " ", " ", " ", " ", " ", " ", " ", " ", " ", " ", " ", " ", " ", \
" ", " ", " ", " ", " ", " ", " ", " ", " ", " ", " ", " ", " ", " ", \
" "}
\end{lstlisting}

(Я заменил непечатаемые символы на \q{?}.)

Мы видим что первый и третий блоки пустые (или почти пустые),
но второй и четвертый имеют ясно различимые английские слова/фразы.
Похоже что наше предположение насчет ключа верно (как минимум частично).
Это означает, что самый встречающийся 81-байтный блок в файле находится в местах лакун с нулевыми байтами
или что-то в этом роде.

Попробуем расшифровать весь файл:

\begin{lstlisting}[style=custommath]
DecryptBlock[blk_] := BitXor[key, blk]

decrypted = Map[DecryptBlock[#] &, blocks];

BinaryWrite["/home/dennis/.../tmp", Flatten[decrypted]]

Close["/home/dennis/.../tmp"]
\end{lstlisting}

\begin{figure}[H]
\centering
\myincludegraphics{ff/XOR/mask_1/mc_decrypted1.png}
\caption{Расшифрованный файл в Midnight Commander, первая попытка}
\end{figure}

Выглядит как английские фразы для какой-то игры, но что-то не так.
Прежде всего, регистр инвертирован: фразы и некоторые слова начинаются со строчных букв,
в то время как остальные буквы заглавные.
Также, некоторые фразы начинаются с не тех букв.
Посмотрите на самую первую фразу: \q{eHE WEED OF CRIME BEARS BITTER FRUIT}.
Что такое \q{eHE}? Разве не \q{tHE} тут должно быть?
Возможно ли что наш ключ для дешифрования имеет неверный байт в этом месте?

Посмотрим снова на второй блок в файле, на ключ и на результат дешифрования:

\begin{lstlisting}[style=custommath]
In[]:= blocks[[2]]
Out[]= {80, 2, 74, 49, 113, 49, 51, 92, 39, 8, 92, 81, 116, 62, \
57, 80, 46, 40, 114, 36, 75, 56, 33, 76, 9, 55, 56, 59, 81, 65, 45, \
28, 60, 55, 93, 39, 90, 28, 124, 106, 16, 20, 104, 119, 8, 109, 26, \
106, 9, 97, 13, 99, 15, 119, 20, 105, 117, 98, 103, 118, 1, 126, 29, \
97, 122, 17, 15, 114, 110, 3, 5, 125, 125, 99, 126, 119, 102, 30, \
122, 2, 117}

In[]:= key
Out[]= {80, 103, 2, 116, 113, 102, 118, 25, 99, 8, 19, 23, 116, \
125, 107, 25, 99, 109, 114, 102, 14, 121, 115, 31, 9, 117, 113, 111, \
5, 4, 127, 28, 122, 101, 8, 110, 14, 18, 124, 106, 16, 20, 104, 119, \
8, 109, 26, 106, 9, 97, 13, 99, 15, 119, 20, 105, 117, 98, 103, 118, \
1, 126, 29, 97, 122, 17, 15, 114, 110, 3, 5, 125, 125, 99, 126, 119, \
102, 30, 122, 2, 117}

In[]:= BitXor[key, blocks[[2]]]
Out[]= {0, 101, 72, 69, 0, 87, 69, 69, 68, 0, 79, 70, 0, 67, 82, \
73, 77, 69, 0, 66, 69, 65, 82, 83, 0, 66, 73, 84, 84, 69, 82, 0, 70, \
82, 85, 73, 84, 14, 0, 0, 0, 0, 0, 0, 0, 0, 0, 0, 0, 0, 0, 0, 0, 0, \
0, 0, 0, 0, 0, 0, 0, 0, 0, 0, 0, 0, 0, 0, 0, 0, 0, 0, 0, 0, 0, 0, 0, \
0, 0, 0, 0}
\end{lstlisting}

Зашифрованный байт это 2, байт из ключа это 103, $2 \oplus 103=101$ и 101 это ASCII-код символа \q{e}.
Чему должен равнятся этот байт ключа, чтобы ASCII-код был 116 (для символа  \q{t})?
$2 \oplus 116=118$, присвоим 118 второму байту в ключе \dots

\begin{lstlisting}[style=custommath]
key = {80, 118, 2, 116, 113, 102, 118, 25, 99, 8, 19, 23, 116, 125, 
  107, 25, 99, 109, 114, 102, 14, 121, 115, 31, 9, 117, 113, 111, 5, 
  4, 127, 28, 122, 101, 8, 110, 14, 18, 124, 106, 16, 20, 104, 119, 8,
   109, 26, 106, 9, 97, 13, 99, 15, 119, 20, 105, 117, 98, 103, 118, 
  1, 126, 29, 97, 122, 17, 15, 114, 110, 3, 5, 125, 125, 99, 126, 119,
   102, 30, 122, 2, 117}
\end{lstlisting}

\dots и снова дешифруем весь файл.

\begin{figure}[H]
\centering
\myincludegraphics{ff/XOR/mask_1/mc_decrypted2.png}
\caption{Дешифрованный файл в Midnight Commander, вторая попытка}
\end{figure}

Ух ты, теперь грамматика корректна, и все фразы начинаются с корректных букв.
Но все таки, регистр подозрителен.
С чего бы разработчику игры записывать их в такой манере?
Может быть наш ключ все еще неправилен?

% TODO ASCII table somewhere in the book
Изучая таблицу ASCII мы можем заметить что ASCII-коды для букв в верхнем и нижнем регистре отличаются только на один бит
(6-й бит, если считать с первого, 0b100000):

\begin{figure}[H]
\centering
\includegraphics[width=0.7\textwidth]{ascii.png}
\caption{7-битная таблица \ac{ASCII} в Emacs}
\end{figure}

6-й бит, выставленный в нулевом байте, В десятичном виде это будет 32.
Но 32 это ASCII-код пробела!

Действительно, можно менять регистр просто применяя XOR к ASCII-коду, с 32 (больше об этом: \myref{toupper_bit}).

Возможно ли, что пустые лакуны в файле это не нулевые байты, а скорее содержащие пробелы?
Еще раз модифицируем наш XOR-ключ (я про-XOR-ю каждый байт ключа с 32):

\begin{lstlisting}[style=custommath]
(* "32" это скаляр, и "key" это вектор, но это OK *)

In[]:= key3 = BitXor[32, key]
Out[]= {112, 86, 34, 84, 81, 70, 86, 57, 67, 40, 51, 55, 84, 93, 75, \
57, 67, 77, 82, 70, 46, 89, 83, 63, 41, 85, 81, 79, 37, 36, 95, 60, \
90, 69, 40, 78, 46, 50, 92, 74, 48, 52, 72, 87, 40, 77, 58, 74, 41, \
65, 45, 67, 47, 87, 52, 73, 85, 66, 71, 86, 33, 94, 61, 65, 90, 49, \
47, 82, 78, 35, 37, 93, 93, 67, 94, 87, 70, 62, 90, 34, 85}

In[]:= DecryptBlock[blk_] := BitXor[key3, blk]
\end{lstlisting}

И снова дешифруем входной файл:

\begin{figure}[H]
\centering
\myincludegraphics{ff/XOR/mask_1/mc_decrypted.png}
\caption{Дешифрованный файл в Midnight Commander, последняя попытка}
\end{figure}

(Расшифрованный файл доступен здесь:
\url{\GitHubBlobMasterURL/ff/XOR/mask_1/files/decrypted.dat.bz2}.)

Несомненно, это корректный исходный файл.
Да, и мы видим числа в начале каждого блока. Должно быть это и есть источник некорректного XOR-ключа.
Как выходит, самый встречающийся 81-байтный блок в файле это блок заполненный пробелами и содержащий символ \q{1} на месте
второго байта.
Действительно, как-то так получилось что многие блоки здесь перемежаются с этим блоком.
Может быть это что-то вроде выравнивания (padding) для коротких фраз/сообщений?
Другой часто встречающийся 81-байтный блок также заполнен пробелами, но с другой цифрой, следовательно,
они отличаются только вторым байтом.

Вот и всё! Теперь мы можем написать утилиту для зашифрования файла назад, и, может быть, модифицировать его перед этим

Файл для Mathematica можно скачать здесь:\\
\url{\GitHubBlobMasterURL/ff/XOR/mask_1/files/XOR_mask_1.nb}.

Итог: XOR-шифрование не надежно вообще. Вероятно, разработчик игры хотел просто скрыть внутренности игры от игрока,
ничего более серьезного.
Все же, шифрование вроде этого крайне популярно вследствии его простоты, так что многие реверс инженеры обычно хорошо
с этим знакомы.

}\FR{\mysection{Fonction presque vide}
\label{Boolector}
\myindex{Boolector}
\myindex{x86!\Instructions!JMP}

Ceci est un morceau de code réel que j'ai trouvé dans Boolector\footnote{\url{https://boolector.github.io/}}:

\lstinputlisting[style=customc]{patterns/025_almost_empty/boolectormain.c}

Pourquoi quelqu'un ferait-il comme ça?
Je ne sais pas mais mon hypothèse est que \verb|boolector_main()| peut être compilée
dans une sorte de DLL ou bibliothèque dynamique, et appelée depuis une suite de test.
Certainement qu'une suite de test peut préparer les variables argc/argv comme
le ferait \ac{CRT}.

Il est intéressant de voir comment c'est compilé:

\lstinputlisting[caption=GCC 8.2 x64 \NonOptimizing (\assemblyOutput),style=customasmx86]{patterns/025_almost_empty/boolectormain_O0.s}

Ceci est OK, le prologue (non optimisé) déplace inutilement deux arguments,
\INS{CALL}, épilogue, \INS{RET}.
Mais regardons la version optimisée:

\lstinputlisting[caption=GCC 8.2 x64 \Optimizing (\assemblyOutput),style=customasmx86]{patterns/025_almost_empty/boolectormain_O3.s}

Aussi simple que ça: la pile et les registres ne sont pas touchés et \verb|boolector_main()|
a le même ensemble d'arguments.
Donc, tout ce que nous avons à faire est de passer l'exécution à une autre adresse.

Ceci est proche d'une \glslink{thunk function}{fonction thunk}.

Nous verons queelque chose de plus avancé plus tard: \myref{ARM_B_to_printf}, \myref{JMP_instead_of_RET}.
}

\renewcommand{\CURPATH}{advanced/111_netmask}
\EN{% TODO translate
\mysection{Breaking simple executable cryptor}

I've got an executable file which is encrypted by relatively simple encryption.
\href{\GitHubBlobMasterURL/examples/simple_exec_crypto/files/cipher.bin}{Here is it} (only executable section is left here).

First, all encryption function does is just adds number of position in buffer to the byte.
Here is how this can be encoded in Python:

\begin{lstlisting}[caption=Python script,style=custompy]
#!/usr/bin/env python
def e(i, k):
    return chr ((ord(i)+k) % 256)

def encrypt(buf):
    return e(buf[0], 0)+ e(buf[1], 1)+ e(buf[2], 2) + e(buf[3], 3)+ e(buf[4], 4)+ e(buf[5], 5)+ e(buf[6], 6)+ e(buf[7], 7)+
           e(buf[8], 8)+ e(buf[9], 9)+ e(buf[10], 10)+ e(buf[11], 11)+ e(buf[12], 12)+ e(buf[13], 13)+ e(buf[14], 14)+ e(buf[15], 15)
\end{lstlisting}

Hence, if you encrypt buffer with 16 zeros, you'll get \emph{0, 1, 2, 3 ... 12, 13, 14, 15}.

\myindex{Propagating Cipher Block Chaining}
Propagating Cipher Block Chaining (PCBC) is also used, here is how it works:

\begin{figure}[H]
\centering
\myincludegraphics{examples/simple_exec_crypto/601px-PCBC_encryption.png}
\caption{Propagating Cipher Block Chaining encryption (image is taken from Wikipedia article)}
\end{figure}

The problem is that it's too boring to recover IV (Initialization Vector) each time.
Brute-force is also not an option, because IV is too long (16 bytes).
Let's see, if it's possible to recover IV for arbitrary encrypted executable file?

Let's try simple frequency analysis.
This is 32-bit x86 executable code, so let's gather statistics about most frequent bytes and opcodes.
I tried huge oracle.exe file from Oracle RDBMS version 11.2 for windows x86 and I've found that the most frequent byte (no surprise) is zero (~10\%).
The next most frequent byte is (again, no surprise) 0xFF (~5\%).
The next is 0x8B (~5\%).

\myindex{x86!\Instructions!MOV}
0x8B is opcode for \INS{MOV}, this is indeed one of the most busy x86 instructions.
Now what about popularity of zero byte?
If compiler needs to encode value bigger than 127, it has to use 32-bit displacement instead of 8-bit one, but large values are very rare,
so it is padded by zeros.
\myindex{x86!\Instructions!LEA}
\myindex{x86!\Instructions!PUSH}
\myindex{x86!\Instructions!CALL}
This is at least in \INS{LEA}, \INS{MOV}, \INS{PUSH}, \INS{CALL}.

For example:

\begin{lstlisting}[style=customasmx86]
8D B0 28 01 00 00                 lea     esi, [eax+128h]
8D BF 40 38 00 00                 lea     edi, [edi+3840h]
\end{lstlisting}

Displacements bigger than 127 are very popular, but they are rarely exceeds 0x10000
(indeed, such large memory buffers/structures are also rare).

Same story with \INS{MOV}, large constants are rare, the most heavily used are 0, 1, 10, 100, $2^n$, and so on.
Compiler has to pad small constants by zeros to represent them as 32-bit values:

\begin{lstlisting}[style=customasmx86]
BF 02 00 00 00                    mov     edi, 2
BF 01 00 00 00                    mov     edi, 1
\end{lstlisting}

Now about 00 and FF bytes combined: jumps (including conditional) and calls can pass execution flow forward or backwards, but very often,
within the limits of the current executable module.
If forward, displacement is not very big and also padded with zeros.
If backwards, displacement is represented as negative value, so padded with FF bytes.
For example, transfer execution flow forward:

\begin{lstlisting}[style=customasmx86]
E8 43 0C 00 00                    call    _function1
E8 5C 00 00 00                    call    _function2
0F 84 F0 0A 00 00                 jz      loc_4F09A0
0F 84 EB 00 00 00                 jz      loc_4EFBB8
\end{lstlisting}

Backwards:

\begin{lstlisting}[style=customasmx86]
E8 79 0C FE FF                    call    _function1
E8 F4 16 FF FF                    call    _function2
0F 84 F8 FB FF FF                 jz      loc_8212BC
0F 84 06 FD FF FF                 jz      loc_FF1E7D
\end{lstlisting}

FF byte is also very often occurred in negative displacements like these:

\begin{lstlisting}[style=customasmx86]
8D 85 1E FF FF FF                 lea     eax, [ebp-0E2h]
8D 95 F8 5C FF FF                 lea     edx, [ebp-0A308h]
\end{lstlisting}

So far so good. Now we have to try various 16-byte keys, decrypt executable section and measure how often 00, FF and 8B bytes are occurred.
Let's also keep in sight how PCBC decryption works:

\begin{figure}[H]
\centering
\myincludegraphics{examples/simple_exec_crypto/640px-PCBC_decryption.png}
\caption{Propagating Cipher Block Chaining decryption (image is taken from Wikipedia article)}
\end{figure}

The good news is that we don't really have to decrypt whole piece of data, but only slice by slice, this is exactly how I did in my previous example: \myref{XOR_mask_2}.

Now I'm trying all possible bytes (0..255) for each byte in key and just pick the byte producing maximal amount of 00/FF/8B bytes in a decrypted slice:

\begin{lstlisting}[style=custompy]
#!/usr/bin/env python
import sys, hexdump, array, string, operator

KEY_LEN=16

def chunks(l, n):
    # split n by l-byte chunks
    # https://stackoverflow.com/q/312443
    n = max(1, n)
    return [l[i:i + n] for i in range(0, len(l), n)]

def read_file(fname):
    file=open(fname, mode='rb')
    content=file.read()
    file.close()
    return content

def decrypt_byte (c, key):
    return chr((ord(c)-key) % 256)

def XOR_PCBC_step (IV, buf, k):
    prev=IV
    rt=""
    for c in buf:
	new_c=decrypt_byte(c, k)
        plain=chr(ord(new_c)^ord(prev))
	prev=chr(ord(c)^ord(plain))
	rt=rt+plain
    return rt

each_Nth_byte=[""]*KEY_LEN

content=read_file(sys.argv[1])
# split input by 16-byte chunks:
all_chunks=chunks(content, KEY_LEN)
for c in all_chunks:
    for i in range(KEY_LEN):
        each_Nth_byte[i]=each_Nth_byte[i] + c[i]

# try each byte of key
for N in range(KEY_LEN):
    print "N=", N
    stat={}
    for i in range(256):
        tmp_key=chr(i)
	tmp=XOR_PCBC_step(tmp_key,each_Nth_byte[N], N)
        # count 0, FFs and 8Bs in decrypted buffer:
	important_bytes=tmp.count('\x00')+tmp.count('\xFF')+tmp.count('\x8B')
	stat[i]=important_bytes
    sorted_stat = sorted(stat.iteritems(), key=operator.itemgetter(1), reverse=True)
    print sorted_stat[0]
\end{lstlisting}

(Source code can be downloaded \href{\GitHubBlobMasterURL/examples/simple_exec_crypto/files/decrypt.py}{here}.)

I run it and here is a key for which 00/FF/8B bytes presence in decrypted buffer is maximal:

\begin{lstlisting}
N= 0
(147, 1224)
N= 1
(94, 1327)
N= 2
(252, 1223)
N= 3
(218, 1266)
N= 4
(38, 1209)
N= 5
(192, 1378)
N= 6
(199, 1204)
N= 7
(213, 1332)
N= 8
(225, 1251)
N= 9
(112, 1223)
N= 10
(143, 1177)
N= 11
(108, 1286)
N= 12
(10, 1164)
N= 13
(3, 1271)
N= 14
(128, 1253)
N= 15
(232, 1330)
\end{lstlisting}

Let's write decryption utility with the key we got:

\begin{lstlisting}[style=custompy]
#!/usr/bin/env python
import sys, hexdump, array

def xor_strings(s,t):
    # \verb|https://en.wikipedia.org/wiki/XOR_cipher#Example_implementation|
    """xor two strings together"""
    return "".join(chr(ord(a)^ord(b)) for a,b in zip(s,t))

IV=array.array('B', [147, 94, 252, 218, 38, 192, 199, 213, 225, 112, 143, 108, 10, 3, 128, 232]).tostring()

def chunks(l, n):
    n = max(1, n)
    return [l[i:i + n] for i in range(0, len(l), n)]

def read_file(fname):
    file=open(fname, mode='rb')
    content=file.read()
    file.close()
    return content

def decrypt_byte(i, k):
    return chr ((ord(i)-k) % 256)

def decrypt(buf):
    return "".join(decrypt_byte(buf[i], i) for i in range(16))

fout=open(sys.argv[2], mode='wb')

prev=IV
content=read_file(sys.argv[1])
tmp=chunks(content, 16)
for c in tmp:
    new_c=decrypt(c)
    p=xor_strings (new_c, prev)
    prev=xor_strings(c, p)
    fout.write(p)
fout.close()
\end{lstlisting}

(Source code can be downloaded \href{\GitHubBlobMasterURL/examples/simple_exec_crypto/files/decrypt2.py}{here}.)

Let's check resulting file:

\lstinputlisting{examples/simple_exec_crypto/objdump_result.txt}

Yes, this is seems correctly disassembled piece of x86 code.
The whole decryped file can be downloaded \href{\GitHubBlobMasterURL/examples/simple_exec_crypto/files/decrypted.bin}{here}.

In fact, this is text section from regedit.exe from Windows 7.
But this example is based on a real case I encountered, so just executable is different (and key), algorithm is the same.

\subsection{Other ideas to consider}

What if I would fail with such simple frequency analysis?
There are other ideas on how to measure correctness of decrypted/decompressed x86 code:

\begin{itemize}

\item Many modern compilers aligns functions on 0x10 border.
So the space left before is filled with NOPs (0x90) or other NOP instructions with known opcodes: \myref{sec:npad}.

\item Perhaps, the most frequent pattern in any assembly language is function call:\\
\TT{PUSH chain / CALL / ADD ESP, X}.
This sequence can easily detected and found.
I've even gathered statistics about average number of function arguments: \myref{args_stat}.
(Hence, this is average length of PUSH chain.)

\end{itemize}

Read more about incorrectly/correctly disassembled code: \myref{ISA_detect}.
}\RU{\subsection{Простое шифрование используя XOR-маску}
\label{XOR_mask_1}

Я нашел одну старую игру в стиле interactive fiction в архиве \emph{if-archive}\footnote{\url{http://www.ifarchive.org/}}:

\begin{lstlisting}
The New Castle v3.5 - Text/Adventure Game
in the style of the original Infocom (tm)
type games, Zork, Collosal Cave (Adventure),
etc.  Can you solve the mystery of the
abandoned castle?
Shareware from Software Customization.
Software Customization [ASP] Version 3.5 Feb. 2000
\end{lstlisting}

Можно скачать здесь: \url{\GitHubBlobMasterURL/ff/XOR/mask_1/files/newcastle.tgz}.

Там внутри есть файл (с названием \emph{castle.dbf}), который явно зашифрован, но не настоящим криптоалгоритмом,
и оне сжат, это что-то куда проще.
Я бы даже не стал измерять уровень энтропии (\myref{entropy}) этого файла, потому что я итак уверен, что он низкий.
Вот как он выглядит в Midnight Commander:

\begin{figure}[H]
\centering
\myincludegraphics{ff/XOR/mask_1/mc_encrypted.png}
\caption{Зашифрованный файл в Midnight Commander}
\end{figure}

Зашифрованный файл можно скачать здесь:
\url{\GitHubBlobMasterURL/ff/XOR/mask_1/files/castle.dbf.bz2}.

Можно ли расшифровать его без доступа к программе, используя просто этот файл?

Тут явно просматривается повторяющаяся строка. 
Если использовалось простое шифрование с XOR-маской, такие повторяющиеся строки это явное свидетельство,
потому что, вероятно, тут были длинные лакуны с нулевыми байтами, которые, в свою очередь, присутствуют
во мноигих исполняемых файлах, и в остальных бинарных файлах.

\myindex{UNIX!xxd}
Вот дам начала этого файла используя утилиту \emph{xxd} из UNIX:

\lstinputlisting{ff/XOR/mask_1/xxd_result.txt}

Давайте держаться за повторяющуюся строку \TT{iubgv}.
Глядя на этот дамп, мы можем легко увидеть, что период повторений этой строки это 0x51 или 81.
Вероятно, 81 это длина блока?
Длина файла 1658961, и она может быть поделена на 81 без остатка (и тогда там 20481 блоков).

Теперь я буду использовать Mathematica для анализа, есть ли тут повторяющиеся 81-байтные блоки в файле?
Я разделю входной файл на 81-байтные блоки и затем использую ф-цию
\emph{Tally[]}\footnote{\url{https://reference.wolfram.com/language/ref/Tally.html}}
которая просто считает, сколько раз каждый элемент встретился во входном списке.
Вывод Tally не отсортирован, так что я также добавлю ф-цию \emph{Sort[]} для сортировки его по кол-ву вхождений
в нисходящем порядке.

\begin{lstlisting}[style=custommath]
input = BinaryReadList["/home/dennis/.../castle.dbf"];

blocks = Partition[input, 81];

stat = Sort[Tally[blocks], #1[[2]] > #2[[2]] &]
\end{lstlisting}

И вот вывод:

\begin{lstlisting}[style=custommath]
{{{80, 103, 2, 116, 113, 102, 118, 25, 99, 8, 19, 23, 116, 125, 107, 
   25, 99, 109, 114, 102, 14, 121, 115, 31, 9, 117, 113, 111, 5, 4, 
   127, 28, 122, 101, 8, 110, 14, 18, 124, 106, 16, 20, 104, 119, 8, 
   109, 26, 106, 9, 97, 13, 99, 15, 119, 20, 105, 117, 98, 103, 118, 
   1, 126, 29, 97, 122, 17, 15, 114, 110, 3, 5, 125, 125, 99, 126, 
   119, 102, 30, 122, 2, 117}, 1739}, 
{{80, 100, 2, 116, 113, 102, 118, 25, 99, 8, 19, 23, 116, 
   125, 107, 25, 99, 109, 114, 102, 14, 121, 115, 31, 9, 117, 113, 
   111, 5, 4, 127, 28, 122, 101, 8, 110, 14, 18, 124, 106, 16, 20, 
   104, 119, 8, 109, 26, 106, 9, 97, 13, 99, 15, 119, 20, 105, 117, 
   98, 103, 118, 1, 126, 29, 97, 122, 17, 15, 114, 110, 3, 5, 125, 
   125, 99, 126, 119, 102, 30, 122, 2, 117}, 1422}, 
{{80, 101, 2, 116, 113, 102, 118, 25, 99, 8, 19, 23, 116, 
   125, 107, 25, 99, 109, 114, 102, 14, 121, 115, 31, 9, 117, 113, 
   111, 5, 4, 127, 28, 122, 101, 8, 110, 14, 18, 124, 106, 16, 20, 
   104, 119, 8, 109, 26, 106, 9, 97, 13, 99, 15, 119, 20, 105, 117, 
   98, 103, 118, 1, 126, 29, 97, 122, 17, 15, 114, 110, 3, 5, 125, 
   125, 99, 126, 119, 102, 30, 122, 2, 117}, 1012},
{{80, 120, 2, 116, 113, 102, 118, 25, 99, 8, 19, 23, 116, 
   125, 107, 25, 99, 109, 114, 102, 14, 121, 115, 31, 9, 117, 113, 
   111, 5, 4, 127, 28, 122, 101, 8, 110, 14, 18, 124, 106, 16, 20, 
   104, 119, 8, 109, 26, 106, 9, 97, 13, 99, 15, 119, 20, 105, 117, 
   98, 103, 118, 1, 126, 29, 97, 122, 17, 15, 114, 110, 3, 5, 125, 
   125, 99, 126, 119, 102, 30, 122, 2, 117}, 377},

...

{{80, 2, 74, 49, 113, 21, 62, 88, 39, 71, 68, 23, 63, 51, 36, 78, 48, 
   108, 114, 102, 14, 121, 115, 31, 9, 117, 113, 111, 5, 4, 127, 28, 
   122, 101, 8, 110, 14, 18, 124, 106, 16, 20, 104, 119, 8, 109, 26, 
   106, 9, 97, 13, 99, 15, 119, 20, 105, 117, 98, 103, 118, 1, 126, 
   29, 97, 122, 17, 15, 114, 110, 3, 5, 125, 125, 99, 126, 119, 102, 
   30, 122, 2, 117}, 1},
{{80, 1, 74, 59, 113, 45, 56, 86, 52, 91, 19, 64, 60, 60, 63, 
   25, 38, 59, 59, 42, 14, 53, 38, 77, 66, 38, 113, 38, 75, 4, 43, 84,
    63, 101, 64, 43, 79, 64, 40, 57, 16, 91, 46, 119, 69, 40, 84, 117,
    9, 97, 13, 99, 15, 119, 20, 105, 117, 98, 103, 118, 1, 126, 29, 
   97, 122, 17, 15, 114, 110, 3, 5, 125, 125, 99, 126, 119, 102, 30, 
   122, 2, 117}, 1},
{{80, 2, 74, 49, 113, 49, 51, 92, 39, 8, 92, 81, 116, 62, 57, 
   80, 46, 40, 114, 36, 75, 56, 33, 76, 9, 55, 56, 59, 81, 65, 45, 28,
    60, 55, 93, 39, 90, 28, 124, 106, 16, 20, 104, 119, 8, 109, 26, 
   106, 9, 97, 13, 99, 15, 119, 20, 105, 117, 98, 103, 118, 1, 126, 
   29, 97, 122, 17, 15, 114, 110, 3, 5, 125, 125, 99, 126, 119, 102, 
   30, 122, 2, 117}, 1}}
\end{lstlisting}

Вывод Tally это список пар, каждая пара это 81-байтный блок и количество раз, сколько он встретился в файле.
Мы видим, что наиболее частно встречающийся блок это первый, он встретился 1739 раз.
Второй встретился 1422 раза. Есть и другие: 1012 раза, 377 раз, итд.
81-байтные блоки, встреченные лишь один раз, находятся в конце вывода.

Попробуем сравнить эти блоки. Первый и второй.
Есть ли в Mathematica ф-ция для сравнения списков/массивов?
Наверняка есть, но в педагогических целях, я буду использоват операцию XOR для сравнения.
Действительно: если байты во входных массивах равны друг другу, результат операции XOR это 0.
Если не равны, результат будет ненулевой.

Сравним первый блок (встречается 1739 раз) и второй (встречается 1422 раз):

\begin{lstlisting}[style=custommath]
In[]:= BitXor[stat[[1]][[1]], stat[[2]][[1]]]
Out[]= {0, 3, 0, 0, 0, 0, 0, 0, 0, 0, 0, 0, 0, 0, 0, 0, 0, 0, 0, \
0, 0, 0, 0, 0, 0, 0, 0, 0, 0, 0, 0, 0, 0, 0, 0, 0, 0, 0, 0, 0, 0, 0, \
0, 0, 0, 0, 0, 0, 0, 0, 0, 0, 0, 0, 0, 0, 0, 0, 0, 0, 0, 0, 0, 0, 0, \
0, 0, 0, 0, 0, 0, 0, 0, 0, 0, 0, 0, 0, 0, 0, 0}
\end{lstlisting}

Они отличаются только вторым байтом.

Сравним второй блок (встречается 1422 раза) и третий (встречается 1012 раз):

\begin{lstlisting}[style=custommath]
In[]:= BitXor[stat[[2]][[1]], stat[[3]][[1]]]
Out[]= {0, 1, 0, 0, 0, 0, 0, 0, 0, 0, 0, 0, 0, 0, 0, 0, 0, 0, 0, \
0, 0, 0, 0, 0, 0, 0, 0, 0, 0, 0, 0, 0, 0, 0, 0, 0, 0, 0, 0, 0, 0, 0, \
0, 0, 0, 0, 0, 0, 0, 0, 0, 0, 0, 0, 0, 0, 0, 0, 0, 0, 0, 0, 0, 0, 0, \
0, 0, 0, 0, 0, 0, 0, 0, 0, 0, 0, 0, 0, 0, 0, 0}
\end{lstlisting}

Они тоже отличаются только вторым байтом.

Так или иначе, попробуем использовать самый встречающийся блок как XOR-ключ и попробуем расшифровать первые 4 81-байтных
блока в файле:

\begin{lstlisting}[style=custommath]
In[]:= key = stat[[1]][[1]]
Out[]= {80, 103, 2, 116, 113, 102, 118, 25, 99, 8, 19, 23, 116, \
125, 107, 25, 99, 109, 114, 102, 14, 121, 115, 31, 9, 117, 113, 111, \
5, 4, 127, 28, 122, 101, 8, 110, 14, 18, 124, 106, 16, 20, 104, 119, \
8, 109, 26, 106, 9, 97, 13, 99, 15, 119, 20, 105, 117, 98, 103, 118, \
1, 126, 29, 97, 122, 17, 15, 114, 110, 3, 5, 125, 125, 99, 126, 119, \
102, 30, 122, 2, 117}

In[]:= ToASCII[val_] := If[val == 0, " ", FromCharacterCode[val, "PrintableASCII"]]

In[]:= DecryptBlockASCII[blk_] := Map[ToASCII[#] &, BitXor[key, blk]]

In[]:= DecryptBlockASCII[blocks[[1]]]
Out[]= {" ", " ", " ", " ", " ", " ", " ", " ", " ", " ", " ", " \
", " ", " ", " ", " ", " ", " ", " ", " ", " ", " ", " ", " ", " ", " \
", " ", " ", " ", " ", " ", " ", " ", " ", " ", " ", " ", " ", " ", " \
", " ", " ", " ", " ", " ", " ", " ", " ", " ", " ", " ", " ", " ", " \
", " ", " ", " ", " ", " ", " ", " ", " ", " ", " ", " ", " ", " ", " \
", " ", " ", " ", " ", " ", " ", " ", " ", " ", " ", " ", " ", " "}

In[]:= DecryptBlockASCII[blocks[[2]]]
Out[]= {" ", "e", "H", "E", " ", "W", "E", "E", "D", " ", "O", \
"F", " ", "C", "R", "I", "M", "E", " ", "B", "E", "A", "R", "S", " ", \
"B", "I", "T", "T", "E", "R", " ", "F", "R", "U", "I", "T", "?", \
" ", " ", " ", " ", " ", " ", " ", " ", " ", " ", " ", " ", " ", " ", \
" ", " ", " ", " ", " ", " ", " ", " ", " ", " ", " ", " ", " ", " ", \
" ", " ", " ", " ", " ", " ", " ", " ", " ", " ", " ", " ", " ", " ", \
" "}

In[]:= DecryptBlockASCII[blocks[[3]]]
Out[]= {" ", "?", " ", " ", " ", " ", " ", " ", " ", " ", " \
", " ", " ", " ", " ", " ", " ", " ", " ", " ", " ", " ", " ", " ", " \
", " ", " ", " ", " ", " ", " ", " ", " ", " ", " ", " ", " ", " ", " \
", " ", " ", " ", " ", " ", " ", " ", " ", " ", " ", " ", " ", " ", " \
", " ", " ", " ", " ", " ", " ", " ", " ", " ", " ", " ", " ", " ", " \
", " ", " ", " ", " ", " ", " ", " ", " ", " ", " ", " ", " ", " ", " \
"}

In[]:= DecryptBlockASCII[blocks[[4]]]
Out[]= {" ", "f", "H", "O", " ", "K", "N", "O", "W", "S", " ", \
"W", "H", "A", "T", " ", "E", "V", "I", "L", " ", "L", "U", "R", "K", \
"S", " ", "I", "N", " ", "T", "H", "E", " ", "H", "E", "A", "R", "T", \
"S", " ", "O", "F", " ", "M", "E", "N", "?", " ", " ", " ", " ", \
" ", " ", " ", " ", " ", " ", " ", " ", " ", " ", " ", " ", " ", " ", \
" ", " ", " ", " ", " ", " ", " ", " ", " ", " ", " ", " ", " ", " ", \
" "}
\end{lstlisting}

(Я заменил непечатаемые символы на \q{?}.)

Мы видим что первый и третий блоки пустые (или почти пустые),
но второй и четвертый имеют ясно различимые английские слова/фразы.
Похоже что наше предположение насчет ключа верно (как минимум частично).
Это означает, что самый встречающийся 81-байтный блок в файле находится в местах лакун с нулевыми байтами
или что-то в этом роде.

Попробуем расшифровать весь файл:

\begin{lstlisting}[style=custommath]
DecryptBlock[blk_] := BitXor[key, blk]

decrypted = Map[DecryptBlock[#] &, blocks];

BinaryWrite["/home/dennis/.../tmp", Flatten[decrypted]]

Close["/home/dennis/.../tmp"]
\end{lstlisting}

\begin{figure}[H]
\centering
\myincludegraphics{ff/XOR/mask_1/mc_decrypted1.png}
\caption{Расшифрованный файл в Midnight Commander, первая попытка}
\end{figure}

Выглядит как английские фразы для какой-то игры, но что-то не так.
Прежде всего, регистр инвертирован: фразы и некоторые слова начинаются со строчных букв,
в то время как остальные буквы заглавные.
Также, некоторые фразы начинаются с не тех букв.
Посмотрите на самую первую фразу: \q{eHE WEED OF CRIME BEARS BITTER FRUIT}.
Что такое \q{eHE}? Разве не \q{tHE} тут должно быть?
Возможно ли что наш ключ для дешифрования имеет неверный байт в этом месте?

Посмотрим снова на второй блок в файле, на ключ и на результат дешифрования:

\begin{lstlisting}[style=custommath]
In[]:= blocks[[2]]
Out[]= {80, 2, 74, 49, 113, 49, 51, 92, 39, 8, 92, 81, 116, 62, \
57, 80, 46, 40, 114, 36, 75, 56, 33, 76, 9, 55, 56, 59, 81, 65, 45, \
28, 60, 55, 93, 39, 90, 28, 124, 106, 16, 20, 104, 119, 8, 109, 26, \
106, 9, 97, 13, 99, 15, 119, 20, 105, 117, 98, 103, 118, 1, 126, 29, \
97, 122, 17, 15, 114, 110, 3, 5, 125, 125, 99, 126, 119, 102, 30, \
122, 2, 117}

In[]:= key
Out[]= {80, 103, 2, 116, 113, 102, 118, 25, 99, 8, 19, 23, 116, \
125, 107, 25, 99, 109, 114, 102, 14, 121, 115, 31, 9, 117, 113, 111, \
5, 4, 127, 28, 122, 101, 8, 110, 14, 18, 124, 106, 16, 20, 104, 119, \
8, 109, 26, 106, 9, 97, 13, 99, 15, 119, 20, 105, 117, 98, 103, 118, \
1, 126, 29, 97, 122, 17, 15, 114, 110, 3, 5, 125, 125, 99, 126, 119, \
102, 30, 122, 2, 117}

In[]:= BitXor[key, blocks[[2]]]
Out[]= {0, 101, 72, 69, 0, 87, 69, 69, 68, 0, 79, 70, 0, 67, 82, \
73, 77, 69, 0, 66, 69, 65, 82, 83, 0, 66, 73, 84, 84, 69, 82, 0, 70, \
82, 85, 73, 84, 14, 0, 0, 0, 0, 0, 0, 0, 0, 0, 0, 0, 0, 0, 0, 0, 0, \
0, 0, 0, 0, 0, 0, 0, 0, 0, 0, 0, 0, 0, 0, 0, 0, 0, 0, 0, 0, 0, 0, 0, \
0, 0, 0, 0}
\end{lstlisting}

Зашифрованный байт это 2, байт из ключа это 103, $2 \oplus 103=101$ и 101 это ASCII-код символа \q{e}.
Чему должен равнятся этот байт ключа, чтобы ASCII-код был 116 (для символа  \q{t})?
$2 \oplus 116=118$, присвоим 118 второму байту в ключе \dots

\begin{lstlisting}[style=custommath]
key = {80, 118, 2, 116, 113, 102, 118, 25, 99, 8, 19, 23, 116, 125, 
  107, 25, 99, 109, 114, 102, 14, 121, 115, 31, 9, 117, 113, 111, 5, 
  4, 127, 28, 122, 101, 8, 110, 14, 18, 124, 106, 16, 20, 104, 119, 8,
   109, 26, 106, 9, 97, 13, 99, 15, 119, 20, 105, 117, 98, 103, 118, 
  1, 126, 29, 97, 122, 17, 15, 114, 110, 3, 5, 125, 125, 99, 126, 119,
   102, 30, 122, 2, 117}
\end{lstlisting}

\dots и снова дешифруем весь файл.

\begin{figure}[H]
\centering
\myincludegraphics{ff/XOR/mask_1/mc_decrypted2.png}
\caption{Дешифрованный файл в Midnight Commander, вторая попытка}
\end{figure}

Ух ты, теперь грамматика корректна, и все фразы начинаются с корректных букв.
Но все таки, регистр подозрителен.
С чего бы разработчику игры записывать их в такой манере?
Может быть наш ключ все еще неправилен?

% TODO ASCII table somewhere in the book
Изучая таблицу ASCII мы можем заметить что ASCII-коды для букв в верхнем и нижнем регистре отличаются только на один бит
(6-й бит, если считать с первого, 0b100000):

\begin{figure}[H]
\centering
\includegraphics[width=0.7\textwidth]{ascii.png}
\caption{7-битная таблица \ac{ASCII} в Emacs}
\end{figure}

6-й бит, выставленный в нулевом байте, В десятичном виде это будет 32.
Но 32 это ASCII-код пробела!

Действительно, можно менять регистр просто применяя XOR к ASCII-коду, с 32 (больше об этом: \myref{toupper_bit}).

Возможно ли, что пустые лакуны в файле это не нулевые байты, а скорее содержащие пробелы?
Еще раз модифицируем наш XOR-ключ (я про-XOR-ю каждый байт ключа с 32):

\begin{lstlisting}[style=custommath]
(* "32" это скаляр, и "key" это вектор, но это OK *)

In[]:= key3 = BitXor[32, key]
Out[]= {112, 86, 34, 84, 81, 70, 86, 57, 67, 40, 51, 55, 84, 93, 75, \
57, 67, 77, 82, 70, 46, 89, 83, 63, 41, 85, 81, 79, 37, 36, 95, 60, \
90, 69, 40, 78, 46, 50, 92, 74, 48, 52, 72, 87, 40, 77, 58, 74, 41, \
65, 45, 67, 47, 87, 52, 73, 85, 66, 71, 86, 33, 94, 61, 65, 90, 49, \
47, 82, 78, 35, 37, 93, 93, 67, 94, 87, 70, 62, 90, 34, 85}

In[]:= DecryptBlock[blk_] := BitXor[key3, blk]
\end{lstlisting}

И снова дешифруем входной файл:

\begin{figure}[H]
\centering
\myincludegraphics{ff/XOR/mask_1/mc_decrypted.png}
\caption{Дешифрованный файл в Midnight Commander, последняя попытка}
\end{figure}

(Расшифрованный файл доступен здесь:
\url{\GitHubBlobMasterURL/ff/XOR/mask_1/files/decrypted.dat.bz2}.)

Несомненно, это корректный исходный файл.
Да, и мы видим числа в начале каждого блока. Должно быть это и есть источник некорректного XOR-ключа.
Как выходит, самый встречающийся 81-байтный блок в файле это блок заполненный пробелами и содержащий символ \q{1} на месте
второго байта.
Действительно, как-то так получилось что многие блоки здесь перемежаются с этим блоком.
Может быть это что-то вроде выравнивания (padding) для коротких фраз/сообщений?
Другой часто встречающийся 81-байтный блок также заполнен пробелами, но с другой цифрой, следовательно,
они отличаются только вторым байтом.

Вот и всё! Теперь мы можем написать утилиту для зашифрования файла назад, и, может быть, модифицировать его перед этим

Файл для Mathematica можно скачать здесь:\\
\url{\GitHubBlobMasterURL/ff/XOR/mask_1/files/XOR_mask_1.nb}.

Итог: XOR-шифрование не надежно вообще. Вероятно, разработчик игры хотел просто скрыть внутренности игры от игрока,
ничего более серьезного.
Все же, шифрование вроде этого крайне популярно вследствии его простоты, так что многие реверс инженеры обычно хорошо
с этим знакомы.

}\FR{\mysection{Fonction presque vide}
\label{Boolector}
\myindex{Boolector}
\myindex{x86!\Instructions!JMP}

Ceci est un morceau de code réel que j'ai trouvé dans Boolector\footnote{\url{https://boolector.github.io/}}:

\lstinputlisting[style=customc]{patterns/025_almost_empty/boolectormain.c}

Pourquoi quelqu'un ferait-il comme ça?
Je ne sais pas mais mon hypothèse est que \verb|boolector_main()| peut être compilée
dans une sorte de DLL ou bibliothèque dynamique, et appelée depuis une suite de test.
Certainement qu'une suite de test peut préparer les variables argc/argv comme
le ferait \ac{CRT}.

Il est intéressant de voir comment c'est compilé:

\lstinputlisting[caption=GCC 8.2 x64 \NonOptimizing (\assemblyOutput),style=customasmx86]{patterns/025_almost_empty/boolectormain_O0.s}

Ceci est OK, le prologue (non optimisé) déplace inutilement deux arguments,
\INS{CALL}, épilogue, \INS{RET}.
Mais regardons la version optimisée:

\lstinputlisting[caption=GCC 8.2 x64 \Optimizing (\assemblyOutput),style=customasmx86]{patterns/025_almost_empty/boolectormain_O3.s}

Aussi simple que ça: la pile et les registres ne sont pas touchés et \verb|boolector_main()|
a le même ensemble d'arguments.
Donc, tout ce que nous avons à faire est de passer l'exécution à une autre adresse.

Ceci est proche d'une \glslink{thunk function}{fonction thunk}.

Nous verons queelque chose de plus avancé plus tard: \myref{ARM_B_to_printf}, \myref{JMP_instead_of_RET}.
}

\renewcommand{\CURPATH}{advanced/115_loop_iterators}
\EN{% TODO translate
\mysection{Breaking simple executable cryptor}

I've got an executable file which is encrypted by relatively simple encryption.
\href{\GitHubBlobMasterURL/examples/simple_exec_crypto/files/cipher.bin}{Here is it} (only executable section is left here).

First, all encryption function does is just adds number of position in buffer to the byte.
Here is how this can be encoded in Python:

\begin{lstlisting}[caption=Python script,style=custompy]
#!/usr/bin/env python
def e(i, k):
    return chr ((ord(i)+k) % 256)

def encrypt(buf):
    return e(buf[0], 0)+ e(buf[1], 1)+ e(buf[2], 2) + e(buf[3], 3)+ e(buf[4], 4)+ e(buf[5], 5)+ e(buf[6], 6)+ e(buf[7], 7)+
           e(buf[8], 8)+ e(buf[9], 9)+ e(buf[10], 10)+ e(buf[11], 11)+ e(buf[12], 12)+ e(buf[13], 13)+ e(buf[14], 14)+ e(buf[15], 15)
\end{lstlisting}

Hence, if you encrypt buffer with 16 zeros, you'll get \emph{0, 1, 2, 3 ... 12, 13, 14, 15}.

\myindex{Propagating Cipher Block Chaining}
Propagating Cipher Block Chaining (PCBC) is also used, here is how it works:

\begin{figure}[H]
\centering
\myincludegraphics{examples/simple_exec_crypto/601px-PCBC_encryption.png}
\caption{Propagating Cipher Block Chaining encryption (image is taken from Wikipedia article)}
\end{figure}

The problem is that it's too boring to recover IV (Initialization Vector) each time.
Brute-force is also not an option, because IV is too long (16 bytes).
Let's see, if it's possible to recover IV for arbitrary encrypted executable file?

Let's try simple frequency analysis.
This is 32-bit x86 executable code, so let's gather statistics about most frequent bytes and opcodes.
I tried huge oracle.exe file from Oracle RDBMS version 11.2 for windows x86 and I've found that the most frequent byte (no surprise) is zero (~10\%).
The next most frequent byte is (again, no surprise) 0xFF (~5\%).
The next is 0x8B (~5\%).

\myindex{x86!\Instructions!MOV}
0x8B is opcode for \INS{MOV}, this is indeed one of the most busy x86 instructions.
Now what about popularity of zero byte?
If compiler needs to encode value bigger than 127, it has to use 32-bit displacement instead of 8-bit one, but large values are very rare,
so it is padded by zeros.
\myindex{x86!\Instructions!LEA}
\myindex{x86!\Instructions!PUSH}
\myindex{x86!\Instructions!CALL}
This is at least in \INS{LEA}, \INS{MOV}, \INS{PUSH}, \INS{CALL}.

For example:

\begin{lstlisting}[style=customasmx86]
8D B0 28 01 00 00                 lea     esi, [eax+128h]
8D BF 40 38 00 00                 lea     edi, [edi+3840h]
\end{lstlisting}

Displacements bigger than 127 are very popular, but they are rarely exceeds 0x10000
(indeed, such large memory buffers/structures are also rare).

Same story with \INS{MOV}, large constants are rare, the most heavily used are 0, 1, 10, 100, $2^n$, and so on.
Compiler has to pad small constants by zeros to represent them as 32-bit values:

\begin{lstlisting}[style=customasmx86]
BF 02 00 00 00                    mov     edi, 2
BF 01 00 00 00                    mov     edi, 1
\end{lstlisting}

Now about 00 and FF bytes combined: jumps (including conditional) and calls can pass execution flow forward or backwards, but very often,
within the limits of the current executable module.
If forward, displacement is not very big and also padded with zeros.
If backwards, displacement is represented as negative value, so padded with FF bytes.
For example, transfer execution flow forward:

\begin{lstlisting}[style=customasmx86]
E8 43 0C 00 00                    call    _function1
E8 5C 00 00 00                    call    _function2
0F 84 F0 0A 00 00                 jz      loc_4F09A0
0F 84 EB 00 00 00                 jz      loc_4EFBB8
\end{lstlisting}

Backwards:

\begin{lstlisting}[style=customasmx86]
E8 79 0C FE FF                    call    _function1
E8 F4 16 FF FF                    call    _function2
0F 84 F8 FB FF FF                 jz      loc_8212BC
0F 84 06 FD FF FF                 jz      loc_FF1E7D
\end{lstlisting}

FF byte is also very often occurred in negative displacements like these:

\begin{lstlisting}[style=customasmx86]
8D 85 1E FF FF FF                 lea     eax, [ebp-0E2h]
8D 95 F8 5C FF FF                 lea     edx, [ebp-0A308h]
\end{lstlisting}

So far so good. Now we have to try various 16-byte keys, decrypt executable section and measure how often 00, FF and 8B bytes are occurred.
Let's also keep in sight how PCBC decryption works:

\begin{figure}[H]
\centering
\myincludegraphics{examples/simple_exec_crypto/640px-PCBC_decryption.png}
\caption{Propagating Cipher Block Chaining decryption (image is taken from Wikipedia article)}
\end{figure}

The good news is that we don't really have to decrypt whole piece of data, but only slice by slice, this is exactly how I did in my previous example: \myref{XOR_mask_2}.

Now I'm trying all possible bytes (0..255) for each byte in key and just pick the byte producing maximal amount of 00/FF/8B bytes in a decrypted slice:

\begin{lstlisting}[style=custompy]
#!/usr/bin/env python
import sys, hexdump, array, string, operator

KEY_LEN=16

def chunks(l, n):
    # split n by l-byte chunks
    # https://stackoverflow.com/q/312443
    n = max(1, n)
    return [l[i:i + n] for i in range(0, len(l), n)]

def read_file(fname):
    file=open(fname, mode='rb')
    content=file.read()
    file.close()
    return content

def decrypt_byte (c, key):
    return chr((ord(c)-key) % 256)

def XOR_PCBC_step (IV, buf, k):
    prev=IV
    rt=""
    for c in buf:
	new_c=decrypt_byte(c, k)
        plain=chr(ord(new_c)^ord(prev))
	prev=chr(ord(c)^ord(plain))
	rt=rt+plain
    return rt

each_Nth_byte=[""]*KEY_LEN

content=read_file(sys.argv[1])
# split input by 16-byte chunks:
all_chunks=chunks(content, KEY_LEN)
for c in all_chunks:
    for i in range(KEY_LEN):
        each_Nth_byte[i]=each_Nth_byte[i] + c[i]

# try each byte of key
for N in range(KEY_LEN):
    print "N=", N
    stat={}
    for i in range(256):
        tmp_key=chr(i)
	tmp=XOR_PCBC_step(tmp_key,each_Nth_byte[N], N)
        # count 0, FFs and 8Bs in decrypted buffer:
	important_bytes=tmp.count('\x00')+tmp.count('\xFF')+tmp.count('\x8B')
	stat[i]=important_bytes
    sorted_stat = sorted(stat.iteritems(), key=operator.itemgetter(1), reverse=True)
    print sorted_stat[0]
\end{lstlisting}

(Source code can be downloaded \href{\GitHubBlobMasterURL/examples/simple_exec_crypto/files/decrypt.py}{here}.)

I run it and here is a key for which 00/FF/8B bytes presence in decrypted buffer is maximal:

\begin{lstlisting}
N= 0
(147, 1224)
N= 1
(94, 1327)
N= 2
(252, 1223)
N= 3
(218, 1266)
N= 4
(38, 1209)
N= 5
(192, 1378)
N= 6
(199, 1204)
N= 7
(213, 1332)
N= 8
(225, 1251)
N= 9
(112, 1223)
N= 10
(143, 1177)
N= 11
(108, 1286)
N= 12
(10, 1164)
N= 13
(3, 1271)
N= 14
(128, 1253)
N= 15
(232, 1330)
\end{lstlisting}

Let's write decryption utility with the key we got:

\begin{lstlisting}[style=custompy]
#!/usr/bin/env python
import sys, hexdump, array

def xor_strings(s,t):
    # \verb|https://en.wikipedia.org/wiki/XOR_cipher#Example_implementation|
    """xor two strings together"""
    return "".join(chr(ord(a)^ord(b)) for a,b in zip(s,t))

IV=array.array('B', [147, 94, 252, 218, 38, 192, 199, 213, 225, 112, 143, 108, 10, 3, 128, 232]).tostring()

def chunks(l, n):
    n = max(1, n)
    return [l[i:i + n] for i in range(0, len(l), n)]

def read_file(fname):
    file=open(fname, mode='rb')
    content=file.read()
    file.close()
    return content

def decrypt_byte(i, k):
    return chr ((ord(i)-k) % 256)

def decrypt(buf):
    return "".join(decrypt_byte(buf[i], i) for i in range(16))

fout=open(sys.argv[2], mode='wb')

prev=IV
content=read_file(sys.argv[1])
tmp=chunks(content, 16)
for c in tmp:
    new_c=decrypt(c)
    p=xor_strings (new_c, prev)
    prev=xor_strings(c, p)
    fout.write(p)
fout.close()
\end{lstlisting}

(Source code can be downloaded \href{\GitHubBlobMasterURL/examples/simple_exec_crypto/files/decrypt2.py}{here}.)

Let's check resulting file:

\lstinputlisting{examples/simple_exec_crypto/objdump_result.txt}

Yes, this is seems correctly disassembled piece of x86 code.
The whole decryped file can be downloaded \href{\GitHubBlobMasterURL/examples/simple_exec_crypto/files/decrypted.bin}{here}.

In fact, this is text section from regedit.exe from Windows 7.
But this example is based on a real case I encountered, so just executable is different (and key), algorithm is the same.

\subsection{Other ideas to consider}

What if I would fail with such simple frequency analysis?
There are other ideas on how to measure correctness of decrypted/decompressed x86 code:

\begin{itemize}

\item Many modern compilers aligns functions on 0x10 border.
So the space left before is filled with NOPs (0x90) or other NOP instructions with known opcodes: \myref{sec:npad}.

\item Perhaps, the most frequent pattern in any assembly language is function call:\\
\TT{PUSH chain / CALL / ADD ESP, X}.
This sequence can easily detected and found.
I've even gathered statistics about average number of function arguments: \myref{args_stat}.
(Hence, this is average length of PUSH chain.)

\end{itemize}

Read more about incorrectly/correctly disassembled code: \myref{ISA_detect}.
}\RU{\subsection{Простое шифрование используя XOR-маску}
\label{XOR_mask_1}

Я нашел одну старую игру в стиле interactive fiction в архиве \emph{if-archive}\footnote{\url{http://www.ifarchive.org/}}:

\begin{lstlisting}
The New Castle v3.5 - Text/Adventure Game
in the style of the original Infocom (tm)
type games, Zork, Collosal Cave (Adventure),
etc.  Can you solve the mystery of the
abandoned castle?
Shareware from Software Customization.
Software Customization [ASP] Version 3.5 Feb. 2000
\end{lstlisting}

Можно скачать здесь: \url{\GitHubBlobMasterURL/ff/XOR/mask_1/files/newcastle.tgz}.

Там внутри есть файл (с названием \emph{castle.dbf}), который явно зашифрован, но не настоящим криптоалгоритмом,
и оне сжат, это что-то куда проще.
Я бы даже не стал измерять уровень энтропии (\myref{entropy}) этого файла, потому что я итак уверен, что он низкий.
Вот как он выглядит в Midnight Commander:

\begin{figure}[H]
\centering
\myincludegraphics{ff/XOR/mask_1/mc_encrypted.png}
\caption{Зашифрованный файл в Midnight Commander}
\end{figure}

Зашифрованный файл можно скачать здесь:
\url{\GitHubBlobMasterURL/ff/XOR/mask_1/files/castle.dbf.bz2}.

Можно ли расшифровать его без доступа к программе, используя просто этот файл?

Тут явно просматривается повторяющаяся строка. 
Если использовалось простое шифрование с XOR-маской, такие повторяющиеся строки это явное свидетельство,
потому что, вероятно, тут были длинные лакуны с нулевыми байтами, которые, в свою очередь, присутствуют
во мноигих исполняемых файлах, и в остальных бинарных файлах.

\myindex{UNIX!xxd}
Вот дам начала этого файла используя утилиту \emph{xxd} из UNIX:

\lstinputlisting{ff/XOR/mask_1/xxd_result.txt}

Давайте держаться за повторяющуюся строку \TT{iubgv}.
Глядя на этот дамп, мы можем легко увидеть, что период повторений этой строки это 0x51 или 81.
Вероятно, 81 это длина блока?
Длина файла 1658961, и она может быть поделена на 81 без остатка (и тогда там 20481 блоков).

Теперь я буду использовать Mathematica для анализа, есть ли тут повторяющиеся 81-байтные блоки в файле?
Я разделю входной файл на 81-байтные блоки и затем использую ф-цию
\emph{Tally[]}\footnote{\url{https://reference.wolfram.com/language/ref/Tally.html}}
которая просто считает, сколько раз каждый элемент встретился во входном списке.
Вывод Tally не отсортирован, так что я также добавлю ф-цию \emph{Sort[]} для сортировки его по кол-ву вхождений
в нисходящем порядке.

\begin{lstlisting}[style=custommath]
input = BinaryReadList["/home/dennis/.../castle.dbf"];

blocks = Partition[input, 81];

stat = Sort[Tally[blocks], #1[[2]] > #2[[2]] &]
\end{lstlisting}

И вот вывод:

\begin{lstlisting}[style=custommath]
{{{80, 103, 2, 116, 113, 102, 118, 25, 99, 8, 19, 23, 116, 125, 107, 
   25, 99, 109, 114, 102, 14, 121, 115, 31, 9, 117, 113, 111, 5, 4, 
   127, 28, 122, 101, 8, 110, 14, 18, 124, 106, 16, 20, 104, 119, 8, 
   109, 26, 106, 9, 97, 13, 99, 15, 119, 20, 105, 117, 98, 103, 118, 
   1, 126, 29, 97, 122, 17, 15, 114, 110, 3, 5, 125, 125, 99, 126, 
   119, 102, 30, 122, 2, 117}, 1739}, 
{{80, 100, 2, 116, 113, 102, 118, 25, 99, 8, 19, 23, 116, 
   125, 107, 25, 99, 109, 114, 102, 14, 121, 115, 31, 9, 117, 113, 
   111, 5, 4, 127, 28, 122, 101, 8, 110, 14, 18, 124, 106, 16, 20, 
   104, 119, 8, 109, 26, 106, 9, 97, 13, 99, 15, 119, 20, 105, 117, 
   98, 103, 118, 1, 126, 29, 97, 122, 17, 15, 114, 110, 3, 5, 125, 
   125, 99, 126, 119, 102, 30, 122, 2, 117}, 1422}, 
{{80, 101, 2, 116, 113, 102, 118, 25, 99, 8, 19, 23, 116, 
   125, 107, 25, 99, 109, 114, 102, 14, 121, 115, 31, 9, 117, 113, 
   111, 5, 4, 127, 28, 122, 101, 8, 110, 14, 18, 124, 106, 16, 20, 
   104, 119, 8, 109, 26, 106, 9, 97, 13, 99, 15, 119, 20, 105, 117, 
   98, 103, 118, 1, 126, 29, 97, 122, 17, 15, 114, 110, 3, 5, 125, 
   125, 99, 126, 119, 102, 30, 122, 2, 117}, 1012},
{{80, 120, 2, 116, 113, 102, 118, 25, 99, 8, 19, 23, 116, 
   125, 107, 25, 99, 109, 114, 102, 14, 121, 115, 31, 9, 117, 113, 
   111, 5, 4, 127, 28, 122, 101, 8, 110, 14, 18, 124, 106, 16, 20, 
   104, 119, 8, 109, 26, 106, 9, 97, 13, 99, 15, 119, 20, 105, 117, 
   98, 103, 118, 1, 126, 29, 97, 122, 17, 15, 114, 110, 3, 5, 125, 
   125, 99, 126, 119, 102, 30, 122, 2, 117}, 377},

...

{{80, 2, 74, 49, 113, 21, 62, 88, 39, 71, 68, 23, 63, 51, 36, 78, 48, 
   108, 114, 102, 14, 121, 115, 31, 9, 117, 113, 111, 5, 4, 127, 28, 
   122, 101, 8, 110, 14, 18, 124, 106, 16, 20, 104, 119, 8, 109, 26, 
   106, 9, 97, 13, 99, 15, 119, 20, 105, 117, 98, 103, 118, 1, 126, 
   29, 97, 122, 17, 15, 114, 110, 3, 5, 125, 125, 99, 126, 119, 102, 
   30, 122, 2, 117}, 1},
{{80, 1, 74, 59, 113, 45, 56, 86, 52, 91, 19, 64, 60, 60, 63, 
   25, 38, 59, 59, 42, 14, 53, 38, 77, 66, 38, 113, 38, 75, 4, 43, 84,
    63, 101, 64, 43, 79, 64, 40, 57, 16, 91, 46, 119, 69, 40, 84, 117,
    9, 97, 13, 99, 15, 119, 20, 105, 117, 98, 103, 118, 1, 126, 29, 
   97, 122, 17, 15, 114, 110, 3, 5, 125, 125, 99, 126, 119, 102, 30, 
   122, 2, 117}, 1},
{{80, 2, 74, 49, 113, 49, 51, 92, 39, 8, 92, 81, 116, 62, 57, 
   80, 46, 40, 114, 36, 75, 56, 33, 76, 9, 55, 56, 59, 81, 65, 45, 28,
    60, 55, 93, 39, 90, 28, 124, 106, 16, 20, 104, 119, 8, 109, 26, 
   106, 9, 97, 13, 99, 15, 119, 20, 105, 117, 98, 103, 118, 1, 126, 
   29, 97, 122, 17, 15, 114, 110, 3, 5, 125, 125, 99, 126, 119, 102, 
   30, 122, 2, 117}, 1}}
\end{lstlisting}

Вывод Tally это список пар, каждая пара это 81-байтный блок и количество раз, сколько он встретился в файле.
Мы видим, что наиболее частно встречающийся блок это первый, он встретился 1739 раз.
Второй встретился 1422 раза. Есть и другие: 1012 раза, 377 раз, итд.
81-байтные блоки, встреченные лишь один раз, находятся в конце вывода.

Попробуем сравнить эти блоки. Первый и второй.
Есть ли в Mathematica ф-ция для сравнения списков/массивов?
Наверняка есть, но в педагогических целях, я буду использоват операцию XOR для сравнения.
Действительно: если байты во входных массивах равны друг другу, результат операции XOR это 0.
Если не равны, результат будет ненулевой.

Сравним первый блок (встречается 1739 раз) и второй (встречается 1422 раз):

\begin{lstlisting}[style=custommath]
In[]:= BitXor[stat[[1]][[1]], stat[[2]][[1]]]
Out[]= {0, 3, 0, 0, 0, 0, 0, 0, 0, 0, 0, 0, 0, 0, 0, 0, 0, 0, 0, \
0, 0, 0, 0, 0, 0, 0, 0, 0, 0, 0, 0, 0, 0, 0, 0, 0, 0, 0, 0, 0, 0, 0, \
0, 0, 0, 0, 0, 0, 0, 0, 0, 0, 0, 0, 0, 0, 0, 0, 0, 0, 0, 0, 0, 0, 0, \
0, 0, 0, 0, 0, 0, 0, 0, 0, 0, 0, 0, 0, 0, 0, 0}
\end{lstlisting}

Они отличаются только вторым байтом.

Сравним второй блок (встречается 1422 раза) и третий (встречается 1012 раз):

\begin{lstlisting}[style=custommath]
In[]:= BitXor[stat[[2]][[1]], stat[[3]][[1]]]
Out[]= {0, 1, 0, 0, 0, 0, 0, 0, 0, 0, 0, 0, 0, 0, 0, 0, 0, 0, 0, \
0, 0, 0, 0, 0, 0, 0, 0, 0, 0, 0, 0, 0, 0, 0, 0, 0, 0, 0, 0, 0, 0, 0, \
0, 0, 0, 0, 0, 0, 0, 0, 0, 0, 0, 0, 0, 0, 0, 0, 0, 0, 0, 0, 0, 0, 0, \
0, 0, 0, 0, 0, 0, 0, 0, 0, 0, 0, 0, 0, 0, 0, 0}
\end{lstlisting}

Они тоже отличаются только вторым байтом.

Так или иначе, попробуем использовать самый встречающийся блок как XOR-ключ и попробуем расшифровать первые 4 81-байтных
блока в файле:

\begin{lstlisting}[style=custommath]
In[]:= key = stat[[1]][[1]]
Out[]= {80, 103, 2, 116, 113, 102, 118, 25, 99, 8, 19, 23, 116, \
125, 107, 25, 99, 109, 114, 102, 14, 121, 115, 31, 9, 117, 113, 111, \
5, 4, 127, 28, 122, 101, 8, 110, 14, 18, 124, 106, 16, 20, 104, 119, \
8, 109, 26, 106, 9, 97, 13, 99, 15, 119, 20, 105, 117, 98, 103, 118, \
1, 126, 29, 97, 122, 17, 15, 114, 110, 3, 5, 125, 125, 99, 126, 119, \
102, 30, 122, 2, 117}

In[]:= ToASCII[val_] := If[val == 0, " ", FromCharacterCode[val, "PrintableASCII"]]

In[]:= DecryptBlockASCII[blk_] := Map[ToASCII[#] &, BitXor[key, blk]]

In[]:= DecryptBlockASCII[blocks[[1]]]
Out[]= {" ", " ", " ", " ", " ", " ", " ", " ", " ", " ", " ", " \
", " ", " ", " ", " ", " ", " ", " ", " ", " ", " ", " ", " ", " ", " \
", " ", " ", " ", " ", " ", " ", " ", " ", " ", " ", " ", " ", " ", " \
", " ", " ", " ", " ", " ", " ", " ", " ", " ", " ", " ", " ", " ", " \
", " ", " ", " ", " ", " ", " ", " ", " ", " ", " ", " ", " ", " ", " \
", " ", " ", " ", " ", " ", " ", " ", " ", " ", " ", " ", " ", " "}

In[]:= DecryptBlockASCII[blocks[[2]]]
Out[]= {" ", "e", "H", "E", " ", "W", "E", "E", "D", " ", "O", \
"F", " ", "C", "R", "I", "M", "E", " ", "B", "E", "A", "R", "S", " ", \
"B", "I", "T", "T", "E", "R", " ", "F", "R", "U", "I", "T", "?", \
" ", " ", " ", " ", " ", " ", " ", " ", " ", " ", " ", " ", " ", " ", \
" ", " ", " ", " ", " ", " ", " ", " ", " ", " ", " ", " ", " ", " ", \
" ", " ", " ", " ", " ", " ", " ", " ", " ", " ", " ", " ", " ", " ", \
" "}

In[]:= DecryptBlockASCII[blocks[[3]]]
Out[]= {" ", "?", " ", " ", " ", " ", " ", " ", " ", " ", " \
", " ", " ", " ", " ", " ", " ", " ", " ", " ", " ", " ", " ", " ", " \
", " ", " ", " ", " ", " ", " ", " ", " ", " ", " ", " ", " ", " ", " \
", " ", " ", " ", " ", " ", " ", " ", " ", " ", " ", " ", " ", " ", " \
", " ", " ", " ", " ", " ", " ", " ", " ", " ", " ", " ", " ", " ", " \
", " ", " ", " ", " ", " ", " ", " ", " ", " ", " ", " ", " ", " ", " \
"}

In[]:= DecryptBlockASCII[blocks[[4]]]
Out[]= {" ", "f", "H", "O", " ", "K", "N", "O", "W", "S", " ", \
"W", "H", "A", "T", " ", "E", "V", "I", "L", " ", "L", "U", "R", "K", \
"S", " ", "I", "N", " ", "T", "H", "E", " ", "H", "E", "A", "R", "T", \
"S", " ", "O", "F", " ", "M", "E", "N", "?", " ", " ", " ", " ", \
" ", " ", " ", " ", " ", " ", " ", " ", " ", " ", " ", " ", " ", " ", \
" ", " ", " ", " ", " ", " ", " ", " ", " ", " ", " ", " ", " ", " ", \
" "}
\end{lstlisting}

(Я заменил непечатаемые символы на \q{?}.)

Мы видим что первый и третий блоки пустые (или почти пустые),
но второй и четвертый имеют ясно различимые английские слова/фразы.
Похоже что наше предположение насчет ключа верно (как минимум частично).
Это означает, что самый встречающийся 81-байтный блок в файле находится в местах лакун с нулевыми байтами
или что-то в этом роде.

Попробуем расшифровать весь файл:

\begin{lstlisting}[style=custommath]
DecryptBlock[blk_] := BitXor[key, blk]

decrypted = Map[DecryptBlock[#] &, blocks];

BinaryWrite["/home/dennis/.../tmp", Flatten[decrypted]]

Close["/home/dennis/.../tmp"]
\end{lstlisting}

\begin{figure}[H]
\centering
\myincludegraphics{ff/XOR/mask_1/mc_decrypted1.png}
\caption{Расшифрованный файл в Midnight Commander, первая попытка}
\end{figure}

Выглядит как английские фразы для какой-то игры, но что-то не так.
Прежде всего, регистр инвертирован: фразы и некоторые слова начинаются со строчных букв,
в то время как остальные буквы заглавные.
Также, некоторые фразы начинаются с не тех букв.
Посмотрите на самую первую фразу: \q{eHE WEED OF CRIME BEARS BITTER FRUIT}.
Что такое \q{eHE}? Разве не \q{tHE} тут должно быть?
Возможно ли что наш ключ для дешифрования имеет неверный байт в этом месте?

Посмотрим снова на второй блок в файле, на ключ и на результат дешифрования:

\begin{lstlisting}[style=custommath]
In[]:= blocks[[2]]
Out[]= {80, 2, 74, 49, 113, 49, 51, 92, 39, 8, 92, 81, 116, 62, \
57, 80, 46, 40, 114, 36, 75, 56, 33, 76, 9, 55, 56, 59, 81, 65, 45, \
28, 60, 55, 93, 39, 90, 28, 124, 106, 16, 20, 104, 119, 8, 109, 26, \
106, 9, 97, 13, 99, 15, 119, 20, 105, 117, 98, 103, 118, 1, 126, 29, \
97, 122, 17, 15, 114, 110, 3, 5, 125, 125, 99, 126, 119, 102, 30, \
122, 2, 117}

In[]:= key
Out[]= {80, 103, 2, 116, 113, 102, 118, 25, 99, 8, 19, 23, 116, \
125, 107, 25, 99, 109, 114, 102, 14, 121, 115, 31, 9, 117, 113, 111, \
5, 4, 127, 28, 122, 101, 8, 110, 14, 18, 124, 106, 16, 20, 104, 119, \
8, 109, 26, 106, 9, 97, 13, 99, 15, 119, 20, 105, 117, 98, 103, 118, \
1, 126, 29, 97, 122, 17, 15, 114, 110, 3, 5, 125, 125, 99, 126, 119, \
102, 30, 122, 2, 117}

In[]:= BitXor[key, blocks[[2]]]
Out[]= {0, 101, 72, 69, 0, 87, 69, 69, 68, 0, 79, 70, 0, 67, 82, \
73, 77, 69, 0, 66, 69, 65, 82, 83, 0, 66, 73, 84, 84, 69, 82, 0, 70, \
82, 85, 73, 84, 14, 0, 0, 0, 0, 0, 0, 0, 0, 0, 0, 0, 0, 0, 0, 0, 0, \
0, 0, 0, 0, 0, 0, 0, 0, 0, 0, 0, 0, 0, 0, 0, 0, 0, 0, 0, 0, 0, 0, 0, \
0, 0, 0, 0}
\end{lstlisting}

Зашифрованный байт это 2, байт из ключа это 103, $2 \oplus 103=101$ и 101 это ASCII-код символа \q{e}.
Чему должен равнятся этот байт ключа, чтобы ASCII-код был 116 (для символа  \q{t})?
$2 \oplus 116=118$, присвоим 118 второму байту в ключе \dots

\begin{lstlisting}[style=custommath]
key = {80, 118, 2, 116, 113, 102, 118, 25, 99, 8, 19, 23, 116, 125, 
  107, 25, 99, 109, 114, 102, 14, 121, 115, 31, 9, 117, 113, 111, 5, 
  4, 127, 28, 122, 101, 8, 110, 14, 18, 124, 106, 16, 20, 104, 119, 8,
   109, 26, 106, 9, 97, 13, 99, 15, 119, 20, 105, 117, 98, 103, 118, 
  1, 126, 29, 97, 122, 17, 15, 114, 110, 3, 5, 125, 125, 99, 126, 119,
   102, 30, 122, 2, 117}
\end{lstlisting}

\dots и снова дешифруем весь файл.

\begin{figure}[H]
\centering
\myincludegraphics{ff/XOR/mask_1/mc_decrypted2.png}
\caption{Дешифрованный файл в Midnight Commander, вторая попытка}
\end{figure}

Ух ты, теперь грамматика корректна, и все фразы начинаются с корректных букв.
Но все таки, регистр подозрителен.
С чего бы разработчику игры записывать их в такой манере?
Может быть наш ключ все еще неправилен?

% TODO ASCII table somewhere in the book
Изучая таблицу ASCII мы можем заметить что ASCII-коды для букв в верхнем и нижнем регистре отличаются только на один бит
(6-й бит, если считать с первого, 0b100000):

\begin{figure}[H]
\centering
\includegraphics[width=0.7\textwidth]{ascii.png}
\caption{7-битная таблица \ac{ASCII} в Emacs}
\end{figure}

6-й бит, выставленный в нулевом байте, В десятичном виде это будет 32.
Но 32 это ASCII-код пробела!

Действительно, можно менять регистр просто применяя XOR к ASCII-коду, с 32 (больше об этом: \myref{toupper_bit}).

Возможно ли, что пустые лакуны в файле это не нулевые байты, а скорее содержащие пробелы?
Еще раз модифицируем наш XOR-ключ (я про-XOR-ю каждый байт ключа с 32):

\begin{lstlisting}[style=custommath]
(* "32" это скаляр, и "key" это вектор, но это OK *)

In[]:= key3 = BitXor[32, key]
Out[]= {112, 86, 34, 84, 81, 70, 86, 57, 67, 40, 51, 55, 84, 93, 75, \
57, 67, 77, 82, 70, 46, 89, 83, 63, 41, 85, 81, 79, 37, 36, 95, 60, \
90, 69, 40, 78, 46, 50, 92, 74, 48, 52, 72, 87, 40, 77, 58, 74, 41, \
65, 45, 67, 47, 87, 52, 73, 85, 66, 71, 86, 33, 94, 61, 65, 90, 49, \
47, 82, 78, 35, 37, 93, 93, 67, 94, 87, 70, 62, 90, 34, 85}

In[]:= DecryptBlock[blk_] := BitXor[key3, blk]
\end{lstlisting}

И снова дешифруем входной файл:

\begin{figure}[H]
\centering
\myincludegraphics{ff/XOR/mask_1/mc_decrypted.png}
\caption{Дешифрованный файл в Midnight Commander, последняя попытка}
\end{figure}

(Расшифрованный файл доступен здесь:
\url{\GitHubBlobMasterURL/ff/XOR/mask_1/files/decrypted.dat.bz2}.)

Несомненно, это корректный исходный файл.
Да, и мы видим числа в начале каждого блока. Должно быть это и есть источник некорректного XOR-ключа.
Как выходит, самый встречающийся 81-байтный блок в файле это блок заполненный пробелами и содержащий символ \q{1} на месте
второго байта.
Действительно, как-то так получилось что многие блоки здесь перемежаются с этим блоком.
Может быть это что-то вроде выравнивания (padding) для коротких фраз/сообщений?
Другой часто встречающийся 81-байтный блок также заполнен пробелами, но с другой цифрой, следовательно,
они отличаются только вторым байтом.

Вот и всё! Теперь мы можем написать утилиту для зашифрования файла назад, и, может быть, модифицировать его перед этим

Файл для Mathematica можно скачать здесь:\\
\url{\GitHubBlobMasterURL/ff/XOR/mask_1/files/XOR_mask_1.nb}.

Итог: XOR-шифрование не надежно вообще. Вероятно, разработчик игры хотел просто скрыть внутренности игры от игрока,
ничего более серьезного.
Все же, шифрование вроде этого крайне популярно вследствии его простоты, так что многие реверс инженеры обычно хорошо
с этим знакомы.

}\FR{\mysection{Fonction presque vide}
\label{Boolector}
\myindex{Boolector}
\myindex{x86!\Instructions!JMP}

Ceci est un morceau de code réel que j'ai trouvé dans Boolector\footnote{\url{https://boolector.github.io/}}:

\lstinputlisting[style=customc]{patterns/025_almost_empty/boolectormain.c}

Pourquoi quelqu'un ferait-il comme ça?
Je ne sais pas mais mon hypothèse est que \verb|boolector_main()| peut être compilée
dans une sorte de DLL ou bibliothèque dynamique, et appelée depuis une suite de test.
Certainement qu'une suite de test peut préparer les variables argc/argv comme
le ferait \ac{CRT}.

Il est intéressant de voir comment c'est compilé:

\lstinputlisting[caption=GCC 8.2 x64 \NonOptimizing (\assemblyOutput),style=customasmx86]{patterns/025_almost_empty/boolectormain_O0.s}

Ceci est OK, le prologue (non optimisé) déplace inutilement deux arguments,
\INS{CALL}, épilogue, \INS{RET}.
Mais regardons la version optimisée:

\lstinputlisting[caption=GCC 8.2 x64 \Optimizing (\assemblyOutput),style=customasmx86]{patterns/025_almost_empty/boolectormain_O3.s}

Aussi simple que ça: la pile et les registres ne sont pas touchés et \verb|boolector_main()|
a le même ensemble d'arguments.
Donc, tout ce que nous avons à faire est de passer l'exécution à une autre adresse.

Ceci est proche d'une \glslink{thunk function}{fonction thunk}.

Nous verons queelque chose de plus avancé plus tard: \myref{ARM_B_to_printf}, \myref{JMP_instead_of_RET}.
}

\renewcommand{\CURPATH}{advanced/117_duff_device}
\EN{% TODO translate
\mysection{Breaking simple executable cryptor}

I've got an executable file which is encrypted by relatively simple encryption.
\href{\GitHubBlobMasterURL/examples/simple_exec_crypto/files/cipher.bin}{Here is it} (only executable section is left here).

First, all encryption function does is just adds number of position in buffer to the byte.
Here is how this can be encoded in Python:

\begin{lstlisting}[caption=Python script,style=custompy]
#!/usr/bin/env python
def e(i, k):
    return chr ((ord(i)+k) % 256)

def encrypt(buf):
    return e(buf[0], 0)+ e(buf[1], 1)+ e(buf[2], 2) + e(buf[3], 3)+ e(buf[4], 4)+ e(buf[5], 5)+ e(buf[6], 6)+ e(buf[7], 7)+
           e(buf[8], 8)+ e(buf[9], 9)+ e(buf[10], 10)+ e(buf[11], 11)+ e(buf[12], 12)+ e(buf[13], 13)+ e(buf[14], 14)+ e(buf[15], 15)
\end{lstlisting}

Hence, if you encrypt buffer with 16 zeros, you'll get \emph{0, 1, 2, 3 ... 12, 13, 14, 15}.

\myindex{Propagating Cipher Block Chaining}
Propagating Cipher Block Chaining (PCBC) is also used, here is how it works:

\begin{figure}[H]
\centering
\myincludegraphics{examples/simple_exec_crypto/601px-PCBC_encryption.png}
\caption{Propagating Cipher Block Chaining encryption (image is taken from Wikipedia article)}
\end{figure}

The problem is that it's too boring to recover IV (Initialization Vector) each time.
Brute-force is also not an option, because IV is too long (16 bytes).
Let's see, if it's possible to recover IV for arbitrary encrypted executable file?

Let's try simple frequency analysis.
This is 32-bit x86 executable code, so let's gather statistics about most frequent bytes and opcodes.
I tried huge oracle.exe file from Oracle RDBMS version 11.2 for windows x86 and I've found that the most frequent byte (no surprise) is zero (~10\%).
The next most frequent byte is (again, no surprise) 0xFF (~5\%).
The next is 0x8B (~5\%).

\myindex{x86!\Instructions!MOV}
0x8B is opcode for \INS{MOV}, this is indeed one of the most busy x86 instructions.
Now what about popularity of zero byte?
If compiler needs to encode value bigger than 127, it has to use 32-bit displacement instead of 8-bit one, but large values are very rare,
so it is padded by zeros.
\myindex{x86!\Instructions!LEA}
\myindex{x86!\Instructions!PUSH}
\myindex{x86!\Instructions!CALL}
This is at least in \INS{LEA}, \INS{MOV}, \INS{PUSH}, \INS{CALL}.

For example:

\begin{lstlisting}[style=customasmx86]
8D B0 28 01 00 00                 lea     esi, [eax+128h]
8D BF 40 38 00 00                 lea     edi, [edi+3840h]
\end{lstlisting}

Displacements bigger than 127 are very popular, but they are rarely exceeds 0x10000
(indeed, such large memory buffers/structures are also rare).

Same story with \INS{MOV}, large constants are rare, the most heavily used are 0, 1, 10, 100, $2^n$, and so on.
Compiler has to pad small constants by zeros to represent them as 32-bit values:

\begin{lstlisting}[style=customasmx86]
BF 02 00 00 00                    mov     edi, 2
BF 01 00 00 00                    mov     edi, 1
\end{lstlisting}

Now about 00 and FF bytes combined: jumps (including conditional) and calls can pass execution flow forward or backwards, but very often,
within the limits of the current executable module.
If forward, displacement is not very big and also padded with zeros.
If backwards, displacement is represented as negative value, so padded with FF bytes.
For example, transfer execution flow forward:

\begin{lstlisting}[style=customasmx86]
E8 43 0C 00 00                    call    _function1
E8 5C 00 00 00                    call    _function2
0F 84 F0 0A 00 00                 jz      loc_4F09A0
0F 84 EB 00 00 00                 jz      loc_4EFBB8
\end{lstlisting}

Backwards:

\begin{lstlisting}[style=customasmx86]
E8 79 0C FE FF                    call    _function1
E8 F4 16 FF FF                    call    _function2
0F 84 F8 FB FF FF                 jz      loc_8212BC
0F 84 06 FD FF FF                 jz      loc_FF1E7D
\end{lstlisting}

FF byte is also very often occurred in negative displacements like these:

\begin{lstlisting}[style=customasmx86]
8D 85 1E FF FF FF                 lea     eax, [ebp-0E2h]
8D 95 F8 5C FF FF                 lea     edx, [ebp-0A308h]
\end{lstlisting}

So far so good. Now we have to try various 16-byte keys, decrypt executable section and measure how often 00, FF and 8B bytes are occurred.
Let's also keep in sight how PCBC decryption works:

\begin{figure}[H]
\centering
\myincludegraphics{examples/simple_exec_crypto/640px-PCBC_decryption.png}
\caption{Propagating Cipher Block Chaining decryption (image is taken from Wikipedia article)}
\end{figure}

The good news is that we don't really have to decrypt whole piece of data, but only slice by slice, this is exactly how I did in my previous example: \myref{XOR_mask_2}.

Now I'm trying all possible bytes (0..255) for each byte in key and just pick the byte producing maximal amount of 00/FF/8B bytes in a decrypted slice:

\begin{lstlisting}[style=custompy]
#!/usr/bin/env python
import sys, hexdump, array, string, operator

KEY_LEN=16

def chunks(l, n):
    # split n by l-byte chunks
    # https://stackoverflow.com/q/312443
    n = max(1, n)
    return [l[i:i + n] for i in range(0, len(l), n)]

def read_file(fname):
    file=open(fname, mode='rb')
    content=file.read()
    file.close()
    return content

def decrypt_byte (c, key):
    return chr((ord(c)-key) % 256)

def XOR_PCBC_step (IV, buf, k):
    prev=IV
    rt=""
    for c in buf:
	new_c=decrypt_byte(c, k)
        plain=chr(ord(new_c)^ord(prev))
	prev=chr(ord(c)^ord(plain))
	rt=rt+plain
    return rt

each_Nth_byte=[""]*KEY_LEN

content=read_file(sys.argv[1])
# split input by 16-byte chunks:
all_chunks=chunks(content, KEY_LEN)
for c in all_chunks:
    for i in range(KEY_LEN):
        each_Nth_byte[i]=each_Nth_byte[i] + c[i]

# try each byte of key
for N in range(KEY_LEN):
    print "N=", N
    stat={}
    for i in range(256):
        tmp_key=chr(i)
	tmp=XOR_PCBC_step(tmp_key,each_Nth_byte[N], N)
        # count 0, FFs and 8Bs in decrypted buffer:
	important_bytes=tmp.count('\x00')+tmp.count('\xFF')+tmp.count('\x8B')
	stat[i]=important_bytes
    sorted_stat = sorted(stat.iteritems(), key=operator.itemgetter(1), reverse=True)
    print sorted_stat[0]
\end{lstlisting}

(Source code can be downloaded \href{\GitHubBlobMasterURL/examples/simple_exec_crypto/files/decrypt.py}{here}.)

I run it and here is a key for which 00/FF/8B bytes presence in decrypted buffer is maximal:

\begin{lstlisting}
N= 0
(147, 1224)
N= 1
(94, 1327)
N= 2
(252, 1223)
N= 3
(218, 1266)
N= 4
(38, 1209)
N= 5
(192, 1378)
N= 6
(199, 1204)
N= 7
(213, 1332)
N= 8
(225, 1251)
N= 9
(112, 1223)
N= 10
(143, 1177)
N= 11
(108, 1286)
N= 12
(10, 1164)
N= 13
(3, 1271)
N= 14
(128, 1253)
N= 15
(232, 1330)
\end{lstlisting}

Let's write decryption utility with the key we got:

\begin{lstlisting}[style=custompy]
#!/usr/bin/env python
import sys, hexdump, array

def xor_strings(s,t):
    # \verb|https://en.wikipedia.org/wiki/XOR_cipher#Example_implementation|
    """xor two strings together"""
    return "".join(chr(ord(a)^ord(b)) for a,b in zip(s,t))

IV=array.array('B', [147, 94, 252, 218, 38, 192, 199, 213, 225, 112, 143, 108, 10, 3, 128, 232]).tostring()

def chunks(l, n):
    n = max(1, n)
    return [l[i:i + n] for i in range(0, len(l), n)]

def read_file(fname):
    file=open(fname, mode='rb')
    content=file.read()
    file.close()
    return content

def decrypt_byte(i, k):
    return chr ((ord(i)-k) % 256)

def decrypt(buf):
    return "".join(decrypt_byte(buf[i], i) for i in range(16))

fout=open(sys.argv[2], mode='wb')

prev=IV
content=read_file(sys.argv[1])
tmp=chunks(content, 16)
for c in tmp:
    new_c=decrypt(c)
    p=xor_strings (new_c, prev)
    prev=xor_strings(c, p)
    fout.write(p)
fout.close()
\end{lstlisting}

(Source code can be downloaded \href{\GitHubBlobMasterURL/examples/simple_exec_crypto/files/decrypt2.py}{here}.)

Let's check resulting file:

\lstinputlisting{examples/simple_exec_crypto/objdump_result.txt}

Yes, this is seems correctly disassembled piece of x86 code.
The whole decryped file can be downloaded \href{\GitHubBlobMasterURL/examples/simple_exec_crypto/files/decrypted.bin}{here}.

In fact, this is text section from regedit.exe from Windows 7.
But this example is based on a real case I encountered, so just executable is different (and key), algorithm is the same.

\subsection{Other ideas to consider}

What if I would fail with such simple frequency analysis?
There are other ideas on how to measure correctness of decrypted/decompressed x86 code:

\begin{itemize}

\item Many modern compilers aligns functions on 0x10 border.
So the space left before is filled with NOPs (0x90) or other NOP instructions with known opcodes: \myref{sec:npad}.

\item Perhaps, the most frequent pattern in any assembly language is function call:\\
\TT{PUSH chain / CALL / ADD ESP, X}.
This sequence can easily detected and found.
I've even gathered statistics about average number of function arguments: \myref{args_stat}.
(Hence, this is average length of PUSH chain.)

\end{itemize}

Read more about incorrectly/correctly disassembled code: \myref{ISA_detect}.
}\RU{\subsection{Простое шифрование используя XOR-маску}
\label{XOR_mask_1}

Я нашел одну старую игру в стиле interactive fiction в архиве \emph{if-archive}\footnote{\url{http://www.ifarchive.org/}}:

\begin{lstlisting}
The New Castle v3.5 - Text/Adventure Game
in the style of the original Infocom (tm)
type games, Zork, Collosal Cave (Adventure),
etc.  Can you solve the mystery of the
abandoned castle?
Shareware from Software Customization.
Software Customization [ASP] Version 3.5 Feb. 2000
\end{lstlisting}

Можно скачать здесь: \url{\GitHubBlobMasterURL/ff/XOR/mask_1/files/newcastle.tgz}.

Там внутри есть файл (с названием \emph{castle.dbf}), который явно зашифрован, но не настоящим криптоалгоритмом,
и оне сжат, это что-то куда проще.
Я бы даже не стал измерять уровень энтропии (\myref{entropy}) этого файла, потому что я итак уверен, что он низкий.
Вот как он выглядит в Midnight Commander:

\begin{figure}[H]
\centering
\myincludegraphics{ff/XOR/mask_1/mc_encrypted.png}
\caption{Зашифрованный файл в Midnight Commander}
\end{figure}

Зашифрованный файл можно скачать здесь:
\url{\GitHubBlobMasterURL/ff/XOR/mask_1/files/castle.dbf.bz2}.

Можно ли расшифровать его без доступа к программе, используя просто этот файл?

Тут явно просматривается повторяющаяся строка. 
Если использовалось простое шифрование с XOR-маской, такие повторяющиеся строки это явное свидетельство,
потому что, вероятно, тут были длинные лакуны с нулевыми байтами, которые, в свою очередь, присутствуют
во мноигих исполняемых файлах, и в остальных бинарных файлах.

\myindex{UNIX!xxd}
Вот дам начала этого файла используя утилиту \emph{xxd} из UNIX:

\lstinputlisting{ff/XOR/mask_1/xxd_result.txt}

Давайте держаться за повторяющуюся строку \TT{iubgv}.
Глядя на этот дамп, мы можем легко увидеть, что период повторений этой строки это 0x51 или 81.
Вероятно, 81 это длина блока?
Длина файла 1658961, и она может быть поделена на 81 без остатка (и тогда там 20481 блоков).

Теперь я буду использовать Mathematica для анализа, есть ли тут повторяющиеся 81-байтные блоки в файле?
Я разделю входной файл на 81-байтные блоки и затем использую ф-цию
\emph{Tally[]}\footnote{\url{https://reference.wolfram.com/language/ref/Tally.html}}
которая просто считает, сколько раз каждый элемент встретился во входном списке.
Вывод Tally не отсортирован, так что я также добавлю ф-цию \emph{Sort[]} для сортировки его по кол-ву вхождений
в нисходящем порядке.

\begin{lstlisting}[style=custommath]
input = BinaryReadList["/home/dennis/.../castle.dbf"];

blocks = Partition[input, 81];

stat = Sort[Tally[blocks], #1[[2]] > #2[[2]] &]
\end{lstlisting}

И вот вывод:

\begin{lstlisting}[style=custommath]
{{{80, 103, 2, 116, 113, 102, 118, 25, 99, 8, 19, 23, 116, 125, 107, 
   25, 99, 109, 114, 102, 14, 121, 115, 31, 9, 117, 113, 111, 5, 4, 
   127, 28, 122, 101, 8, 110, 14, 18, 124, 106, 16, 20, 104, 119, 8, 
   109, 26, 106, 9, 97, 13, 99, 15, 119, 20, 105, 117, 98, 103, 118, 
   1, 126, 29, 97, 122, 17, 15, 114, 110, 3, 5, 125, 125, 99, 126, 
   119, 102, 30, 122, 2, 117}, 1739}, 
{{80, 100, 2, 116, 113, 102, 118, 25, 99, 8, 19, 23, 116, 
   125, 107, 25, 99, 109, 114, 102, 14, 121, 115, 31, 9, 117, 113, 
   111, 5, 4, 127, 28, 122, 101, 8, 110, 14, 18, 124, 106, 16, 20, 
   104, 119, 8, 109, 26, 106, 9, 97, 13, 99, 15, 119, 20, 105, 117, 
   98, 103, 118, 1, 126, 29, 97, 122, 17, 15, 114, 110, 3, 5, 125, 
   125, 99, 126, 119, 102, 30, 122, 2, 117}, 1422}, 
{{80, 101, 2, 116, 113, 102, 118, 25, 99, 8, 19, 23, 116, 
   125, 107, 25, 99, 109, 114, 102, 14, 121, 115, 31, 9, 117, 113, 
   111, 5, 4, 127, 28, 122, 101, 8, 110, 14, 18, 124, 106, 16, 20, 
   104, 119, 8, 109, 26, 106, 9, 97, 13, 99, 15, 119, 20, 105, 117, 
   98, 103, 118, 1, 126, 29, 97, 122, 17, 15, 114, 110, 3, 5, 125, 
   125, 99, 126, 119, 102, 30, 122, 2, 117}, 1012},
{{80, 120, 2, 116, 113, 102, 118, 25, 99, 8, 19, 23, 116, 
   125, 107, 25, 99, 109, 114, 102, 14, 121, 115, 31, 9, 117, 113, 
   111, 5, 4, 127, 28, 122, 101, 8, 110, 14, 18, 124, 106, 16, 20, 
   104, 119, 8, 109, 26, 106, 9, 97, 13, 99, 15, 119, 20, 105, 117, 
   98, 103, 118, 1, 126, 29, 97, 122, 17, 15, 114, 110, 3, 5, 125, 
   125, 99, 126, 119, 102, 30, 122, 2, 117}, 377},

...

{{80, 2, 74, 49, 113, 21, 62, 88, 39, 71, 68, 23, 63, 51, 36, 78, 48, 
   108, 114, 102, 14, 121, 115, 31, 9, 117, 113, 111, 5, 4, 127, 28, 
   122, 101, 8, 110, 14, 18, 124, 106, 16, 20, 104, 119, 8, 109, 26, 
   106, 9, 97, 13, 99, 15, 119, 20, 105, 117, 98, 103, 118, 1, 126, 
   29, 97, 122, 17, 15, 114, 110, 3, 5, 125, 125, 99, 126, 119, 102, 
   30, 122, 2, 117}, 1},
{{80, 1, 74, 59, 113, 45, 56, 86, 52, 91, 19, 64, 60, 60, 63, 
   25, 38, 59, 59, 42, 14, 53, 38, 77, 66, 38, 113, 38, 75, 4, 43, 84,
    63, 101, 64, 43, 79, 64, 40, 57, 16, 91, 46, 119, 69, 40, 84, 117,
    9, 97, 13, 99, 15, 119, 20, 105, 117, 98, 103, 118, 1, 126, 29, 
   97, 122, 17, 15, 114, 110, 3, 5, 125, 125, 99, 126, 119, 102, 30, 
   122, 2, 117}, 1},
{{80, 2, 74, 49, 113, 49, 51, 92, 39, 8, 92, 81, 116, 62, 57, 
   80, 46, 40, 114, 36, 75, 56, 33, 76, 9, 55, 56, 59, 81, 65, 45, 28,
    60, 55, 93, 39, 90, 28, 124, 106, 16, 20, 104, 119, 8, 109, 26, 
   106, 9, 97, 13, 99, 15, 119, 20, 105, 117, 98, 103, 118, 1, 126, 
   29, 97, 122, 17, 15, 114, 110, 3, 5, 125, 125, 99, 126, 119, 102, 
   30, 122, 2, 117}, 1}}
\end{lstlisting}

Вывод Tally это список пар, каждая пара это 81-байтный блок и количество раз, сколько он встретился в файле.
Мы видим, что наиболее частно встречающийся блок это первый, он встретился 1739 раз.
Второй встретился 1422 раза. Есть и другие: 1012 раза, 377 раз, итд.
81-байтные блоки, встреченные лишь один раз, находятся в конце вывода.

Попробуем сравнить эти блоки. Первый и второй.
Есть ли в Mathematica ф-ция для сравнения списков/массивов?
Наверняка есть, но в педагогических целях, я буду использоват операцию XOR для сравнения.
Действительно: если байты во входных массивах равны друг другу, результат операции XOR это 0.
Если не равны, результат будет ненулевой.

Сравним первый блок (встречается 1739 раз) и второй (встречается 1422 раз):

\begin{lstlisting}[style=custommath]
In[]:= BitXor[stat[[1]][[1]], stat[[2]][[1]]]
Out[]= {0, 3, 0, 0, 0, 0, 0, 0, 0, 0, 0, 0, 0, 0, 0, 0, 0, 0, 0, \
0, 0, 0, 0, 0, 0, 0, 0, 0, 0, 0, 0, 0, 0, 0, 0, 0, 0, 0, 0, 0, 0, 0, \
0, 0, 0, 0, 0, 0, 0, 0, 0, 0, 0, 0, 0, 0, 0, 0, 0, 0, 0, 0, 0, 0, 0, \
0, 0, 0, 0, 0, 0, 0, 0, 0, 0, 0, 0, 0, 0, 0, 0}
\end{lstlisting}

Они отличаются только вторым байтом.

Сравним второй блок (встречается 1422 раза) и третий (встречается 1012 раз):

\begin{lstlisting}[style=custommath]
In[]:= BitXor[stat[[2]][[1]], stat[[3]][[1]]]
Out[]= {0, 1, 0, 0, 0, 0, 0, 0, 0, 0, 0, 0, 0, 0, 0, 0, 0, 0, 0, \
0, 0, 0, 0, 0, 0, 0, 0, 0, 0, 0, 0, 0, 0, 0, 0, 0, 0, 0, 0, 0, 0, 0, \
0, 0, 0, 0, 0, 0, 0, 0, 0, 0, 0, 0, 0, 0, 0, 0, 0, 0, 0, 0, 0, 0, 0, \
0, 0, 0, 0, 0, 0, 0, 0, 0, 0, 0, 0, 0, 0, 0, 0}
\end{lstlisting}

Они тоже отличаются только вторым байтом.

Так или иначе, попробуем использовать самый встречающийся блок как XOR-ключ и попробуем расшифровать первые 4 81-байтных
блока в файле:

\begin{lstlisting}[style=custommath]
In[]:= key = stat[[1]][[1]]
Out[]= {80, 103, 2, 116, 113, 102, 118, 25, 99, 8, 19, 23, 116, \
125, 107, 25, 99, 109, 114, 102, 14, 121, 115, 31, 9, 117, 113, 111, \
5, 4, 127, 28, 122, 101, 8, 110, 14, 18, 124, 106, 16, 20, 104, 119, \
8, 109, 26, 106, 9, 97, 13, 99, 15, 119, 20, 105, 117, 98, 103, 118, \
1, 126, 29, 97, 122, 17, 15, 114, 110, 3, 5, 125, 125, 99, 126, 119, \
102, 30, 122, 2, 117}

In[]:= ToASCII[val_] := If[val == 0, " ", FromCharacterCode[val, "PrintableASCII"]]

In[]:= DecryptBlockASCII[blk_] := Map[ToASCII[#] &, BitXor[key, blk]]

In[]:= DecryptBlockASCII[blocks[[1]]]
Out[]= {" ", " ", " ", " ", " ", " ", " ", " ", " ", " ", " ", " \
", " ", " ", " ", " ", " ", " ", " ", " ", " ", " ", " ", " ", " ", " \
", " ", " ", " ", " ", " ", " ", " ", " ", " ", " ", " ", " ", " ", " \
", " ", " ", " ", " ", " ", " ", " ", " ", " ", " ", " ", " ", " ", " \
", " ", " ", " ", " ", " ", " ", " ", " ", " ", " ", " ", " ", " ", " \
", " ", " ", " ", " ", " ", " ", " ", " ", " ", " ", " ", " ", " "}

In[]:= DecryptBlockASCII[blocks[[2]]]
Out[]= {" ", "e", "H", "E", " ", "W", "E", "E", "D", " ", "O", \
"F", " ", "C", "R", "I", "M", "E", " ", "B", "E", "A", "R", "S", " ", \
"B", "I", "T", "T", "E", "R", " ", "F", "R", "U", "I", "T", "?", \
" ", " ", " ", " ", " ", " ", " ", " ", " ", " ", " ", " ", " ", " ", \
" ", " ", " ", " ", " ", " ", " ", " ", " ", " ", " ", " ", " ", " ", \
" ", " ", " ", " ", " ", " ", " ", " ", " ", " ", " ", " ", " ", " ", \
" "}

In[]:= DecryptBlockASCII[blocks[[3]]]
Out[]= {" ", "?", " ", " ", " ", " ", " ", " ", " ", " ", " \
", " ", " ", " ", " ", " ", " ", " ", " ", " ", " ", " ", " ", " ", " \
", " ", " ", " ", " ", " ", " ", " ", " ", " ", " ", " ", " ", " ", " \
", " ", " ", " ", " ", " ", " ", " ", " ", " ", " ", " ", " ", " ", " \
", " ", " ", " ", " ", " ", " ", " ", " ", " ", " ", " ", " ", " ", " \
", " ", " ", " ", " ", " ", " ", " ", " ", " ", " ", " ", " ", " ", " \
"}

In[]:= DecryptBlockASCII[blocks[[4]]]
Out[]= {" ", "f", "H", "O", " ", "K", "N", "O", "W", "S", " ", \
"W", "H", "A", "T", " ", "E", "V", "I", "L", " ", "L", "U", "R", "K", \
"S", " ", "I", "N", " ", "T", "H", "E", " ", "H", "E", "A", "R", "T", \
"S", " ", "O", "F", " ", "M", "E", "N", "?", " ", " ", " ", " ", \
" ", " ", " ", " ", " ", " ", " ", " ", " ", " ", " ", " ", " ", " ", \
" ", " ", " ", " ", " ", " ", " ", " ", " ", " ", " ", " ", " ", " ", \
" "}
\end{lstlisting}

(Я заменил непечатаемые символы на \q{?}.)

Мы видим что первый и третий блоки пустые (или почти пустые),
но второй и четвертый имеют ясно различимые английские слова/фразы.
Похоже что наше предположение насчет ключа верно (как минимум частично).
Это означает, что самый встречающийся 81-байтный блок в файле находится в местах лакун с нулевыми байтами
или что-то в этом роде.

Попробуем расшифровать весь файл:

\begin{lstlisting}[style=custommath]
DecryptBlock[blk_] := BitXor[key, blk]

decrypted = Map[DecryptBlock[#] &, blocks];

BinaryWrite["/home/dennis/.../tmp", Flatten[decrypted]]

Close["/home/dennis/.../tmp"]
\end{lstlisting}

\begin{figure}[H]
\centering
\myincludegraphics{ff/XOR/mask_1/mc_decrypted1.png}
\caption{Расшифрованный файл в Midnight Commander, первая попытка}
\end{figure}

Выглядит как английские фразы для какой-то игры, но что-то не так.
Прежде всего, регистр инвертирован: фразы и некоторые слова начинаются со строчных букв,
в то время как остальные буквы заглавные.
Также, некоторые фразы начинаются с не тех букв.
Посмотрите на самую первую фразу: \q{eHE WEED OF CRIME BEARS BITTER FRUIT}.
Что такое \q{eHE}? Разве не \q{tHE} тут должно быть?
Возможно ли что наш ключ для дешифрования имеет неверный байт в этом месте?

Посмотрим снова на второй блок в файле, на ключ и на результат дешифрования:

\begin{lstlisting}[style=custommath]
In[]:= blocks[[2]]
Out[]= {80, 2, 74, 49, 113, 49, 51, 92, 39, 8, 92, 81, 116, 62, \
57, 80, 46, 40, 114, 36, 75, 56, 33, 76, 9, 55, 56, 59, 81, 65, 45, \
28, 60, 55, 93, 39, 90, 28, 124, 106, 16, 20, 104, 119, 8, 109, 26, \
106, 9, 97, 13, 99, 15, 119, 20, 105, 117, 98, 103, 118, 1, 126, 29, \
97, 122, 17, 15, 114, 110, 3, 5, 125, 125, 99, 126, 119, 102, 30, \
122, 2, 117}

In[]:= key
Out[]= {80, 103, 2, 116, 113, 102, 118, 25, 99, 8, 19, 23, 116, \
125, 107, 25, 99, 109, 114, 102, 14, 121, 115, 31, 9, 117, 113, 111, \
5, 4, 127, 28, 122, 101, 8, 110, 14, 18, 124, 106, 16, 20, 104, 119, \
8, 109, 26, 106, 9, 97, 13, 99, 15, 119, 20, 105, 117, 98, 103, 118, \
1, 126, 29, 97, 122, 17, 15, 114, 110, 3, 5, 125, 125, 99, 126, 119, \
102, 30, 122, 2, 117}

In[]:= BitXor[key, blocks[[2]]]
Out[]= {0, 101, 72, 69, 0, 87, 69, 69, 68, 0, 79, 70, 0, 67, 82, \
73, 77, 69, 0, 66, 69, 65, 82, 83, 0, 66, 73, 84, 84, 69, 82, 0, 70, \
82, 85, 73, 84, 14, 0, 0, 0, 0, 0, 0, 0, 0, 0, 0, 0, 0, 0, 0, 0, 0, \
0, 0, 0, 0, 0, 0, 0, 0, 0, 0, 0, 0, 0, 0, 0, 0, 0, 0, 0, 0, 0, 0, 0, \
0, 0, 0, 0}
\end{lstlisting}

Зашифрованный байт это 2, байт из ключа это 103, $2 \oplus 103=101$ и 101 это ASCII-код символа \q{e}.
Чему должен равнятся этот байт ключа, чтобы ASCII-код был 116 (для символа  \q{t})?
$2 \oplus 116=118$, присвоим 118 второму байту в ключе \dots

\begin{lstlisting}[style=custommath]
key = {80, 118, 2, 116, 113, 102, 118, 25, 99, 8, 19, 23, 116, 125, 
  107, 25, 99, 109, 114, 102, 14, 121, 115, 31, 9, 117, 113, 111, 5, 
  4, 127, 28, 122, 101, 8, 110, 14, 18, 124, 106, 16, 20, 104, 119, 8,
   109, 26, 106, 9, 97, 13, 99, 15, 119, 20, 105, 117, 98, 103, 118, 
  1, 126, 29, 97, 122, 17, 15, 114, 110, 3, 5, 125, 125, 99, 126, 119,
   102, 30, 122, 2, 117}
\end{lstlisting}

\dots и снова дешифруем весь файл.

\begin{figure}[H]
\centering
\myincludegraphics{ff/XOR/mask_1/mc_decrypted2.png}
\caption{Дешифрованный файл в Midnight Commander, вторая попытка}
\end{figure}

Ух ты, теперь грамматика корректна, и все фразы начинаются с корректных букв.
Но все таки, регистр подозрителен.
С чего бы разработчику игры записывать их в такой манере?
Может быть наш ключ все еще неправилен?

% TODO ASCII table somewhere in the book
Изучая таблицу ASCII мы можем заметить что ASCII-коды для букв в верхнем и нижнем регистре отличаются только на один бит
(6-й бит, если считать с первого, 0b100000):

\begin{figure}[H]
\centering
\includegraphics[width=0.7\textwidth]{ascii.png}
\caption{7-битная таблица \ac{ASCII} в Emacs}
\end{figure}

6-й бит, выставленный в нулевом байте, В десятичном виде это будет 32.
Но 32 это ASCII-код пробела!

Действительно, можно менять регистр просто применяя XOR к ASCII-коду, с 32 (больше об этом: \myref{toupper_bit}).

Возможно ли, что пустые лакуны в файле это не нулевые байты, а скорее содержащие пробелы?
Еще раз модифицируем наш XOR-ключ (я про-XOR-ю каждый байт ключа с 32):

\begin{lstlisting}[style=custommath]
(* "32" это скаляр, и "key" это вектор, но это OK *)

In[]:= key3 = BitXor[32, key]
Out[]= {112, 86, 34, 84, 81, 70, 86, 57, 67, 40, 51, 55, 84, 93, 75, \
57, 67, 77, 82, 70, 46, 89, 83, 63, 41, 85, 81, 79, 37, 36, 95, 60, \
90, 69, 40, 78, 46, 50, 92, 74, 48, 52, 72, 87, 40, 77, 58, 74, 41, \
65, 45, 67, 47, 87, 52, 73, 85, 66, 71, 86, 33, 94, 61, 65, 90, 49, \
47, 82, 78, 35, 37, 93, 93, 67, 94, 87, 70, 62, 90, 34, 85}

In[]:= DecryptBlock[blk_] := BitXor[key3, blk]
\end{lstlisting}

И снова дешифруем входной файл:

\begin{figure}[H]
\centering
\myincludegraphics{ff/XOR/mask_1/mc_decrypted.png}
\caption{Дешифрованный файл в Midnight Commander, последняя попытка}
\end{figure}

(Расшифрованный файл доступен здесь:
\url{\GitHubBlobMasterURL/ff/XOR/mask_1/files/decrypted.dat.bz2}.)

Несомненно, это корректный исходный файл.
Да, и мы видим числа в начале каждого блока. Должно быть это и есть источник некорректного XOR-ключа.
Как выходит, самый встречающийся 81-байтный блок в файле это блок заполненный пробелами и содержащий символ \q{1} на месте
второго байта.
Действительно, как-то так получилось что многие блоки здесь перемежаются с этим блоком.
Может быть это что-то вроде выравнивания (padding) для коротких фраз/сообщений?
Другой часто встречающийся 81-байтный блок также заполнен пробелами, но с другой цифрой, следовательно,
они отличаются только вторым байтом.

Вот и всё! Теперь мы можем написать утилиту для зашифрования файла назад, и, может быть, модифицировать его перед этим

Файл для Mathematica можно скачать здесь:\\
\url{\GitHubBlobMasterURL/ff/XOR/mask_1/files/XOR_mask_1.nb}.

Итог: XOR-шифрование не надежно вообще. Вероятно, разработчик игры хотел просто скрыть внутренности игры от игрока,
ничего более серьезного.
Все же, шифрование вроде этого крайне популярно вследствии его простоты, так что многие реверс инженеры обычно хорошо
с этим знакомы.

}\FR{\mysection{Fonction presque vide}
\label{Boolector}
\myindex{Boolector}
\myindex{x86!\Instructions!JMP}

Ceci est un morceau de code réel que j'ai trouvé dans Boolector\footnote{\url{https://boolector.github.io/}}:

\lstinputlisting[style=customc]{patterns/025_almost_empty/boolectormain.c}

Pourquoi quelqu'un ferait-il comme ça?
Je ne sais pas mais mon hypothèse est que \verb|boolector_main()| peut être compilée
dans une sorte de DLL ou bibliothèque dynamique, et appelée depuis une suite de test.
Certainement qu'une suite de test peut préparer les variables argc/argv comme
le ferait \ac{CRT}.

Il est intéressant de voir comment c'est compilé:

\lstinputlisting[caption=GCC 8.2 x64 \NonOptimizing (\assemblyOutput),style=customasmx86]{patterns/025_almost_empty/boolectormain_O0.s}

Ceci est OK, le prologue (non optimisé) déplace inutilement deux arguments,
\INS{CALL}, épilogue, \INS{RET}.
Mais regardons la version optimisée:

\lstinputlisting[caption=GCC 8.2 x64 \Optimizing (\assemblyOutput),style=customasmx86]{patterns/025_almost_empty/boolectormain_O3.s}

Aussi simple que ça: la pile et les registres ne sont pas touchés et \verb|boolector_main()|
a le même ensemble d'arguments.
Donc, tout ce que nous avons à faire est de passer l'exécution à une autre adresse.

Ceci est proche d'une \glslink{thunk function}{fonction thunk}.

Nous verons queelque chose de plus avancé plus tard: \myref{ARM_B_to_printf}, \myref{JMP_instead_of_RET}.
}

\renewcommand{\CURPATH}{advanced/120_division_by_mult}
\EN{\EN{\input{patterns/016_empty_redux/main_EN}}%
\FR{\input{patterns/016_empty_redux/main_FR}}
}\RU{\EN{\input{patterns/016_empty_redux/main_EN}}%
\FR{\input{patterns/016_empty_redux/main_FR}}
}\FR{\EN{\input{patterns/016_empty_redux/main_EN}}%
\FR{\input{patterns/016_empty_redux/main_FR}}
}

\renewcommand{\CURPATH}{advanced/125_atoi}
\EN{% TODO translate
\mysection{Breaking simple executable cryptor}

I've got an executable file which is encrypted by relatively simple encryption.
\href{\GitHubBlobMasterURL/examples/simple_exec_crypto/files/cipher.bin}{Here is it} (only executable section is left here).

First, all encryption function does is just adds number of position in buffer to the byte.
Here is how this can be encoded in Python:

\begin{lstlisting}[caption=Python script,style=custompy]
#!/usr/bin/env python
def e(i, k):
    return chr ((ord(i)+k) % 256)

def encrypt(buf):
    return e(buf[0], 0)+ e(buf[1], 1)+ e(buf[2], 2) + e(buf[3], 3)+ e(buf[4], 4)+ e(buf[5], 5)+ e(buf[6], 6)+ e(buf[7], 7)+
           e(buf[8], 8)+ e(buf[9], 9)+ e(buf[10], 10)+ e(buf[11], 11)+ e(buf[12], 12)+ e(buf[13], 13)+ e(buf[14], 14)+ e(buf[15], 15)
\end{lstlisting}

Hence, if you encrypt buffer with 16 zeros, you'll get \emph{0, 1, 2, 3 ... 12, 13, 14, 15}.

\myindex{Propagating Cipher Block Chaining}
Propagating Cipher Block Chaining (PCBC) is also used, here is how it works:

\begin{figure}[H]
\centering
\myincludegraphics{examples/simple_exec_crypto/601px-PCBC_encryption.png}
\caption{Propagating Cipher Block Chaining encryption (image is taken from Wikipedia article)}
\end{figure}

The problem is that it's too boring to recover IV (Initialization Vector) each time.
Brute-force is also not an option, because IV is too long (16 bytes).
Let's see, if it's possible to recover IV for arbitrary encrypted executable file?

Let's try simple frequency analysis.
This is 32-bit x86 executable code, so let's gather statistics about most frequent bytes and opcodes.
I tried huge oracle.exe file from Oracle RDBMS version 11.2 for windows x86 and I've found that the most frequent byte (no surprise) is zero (~10\%).
The next most frequent byte is (again, no surprise) 0xFF (~5\%).
The next is 0x8B (~5\%).

\myindex{x86!\Instructions!MOV}
0x8B is opcode for \INS{MOV}, this is indeed one of the most busy x86 instructions.
Now what about popularity of zero byte?
If compiler needs to encode value bigger than 127, it has to use 32-bit displacement instead of 8-bit one, but large values are very rare,
so it is padded by zeros.
\myindex{x86!\Instructions!LEA}
\myindex{x86!\Instructions!PUSH}
\myindex{x86!\Instructions!CALL}
This is at least in \INS{LEA}, \INS{MOV}, \INS{PUSH}, \INS{CALL}.

For example:

\begin{lstlisting}[style=customasmx86]
8D B0 28 01 00 00                 lea     esi, [eax+128h]
8D BF 40 38 00 00                 lea     edi, [edi+3840h]
\end{lstlisting}

Displacements bigger than 127 are very popular, but they are rarely exceeds 0x10000
(indeed, such large memory buffers/structures are also rare).

Same story with \INS{MOV}, large constants are rare, the most heavily used are 0, 1, 10, 100, $2^n$, and so on.
Compiler has to pad small constants by zeros to represent them as 32-bit values:

\begin{lstlisting}[style=customasmx86]
BF 02 00 00 00                    mov     edi, 2
BF 01 00 00 00                    mov     edi, 1
\end{lstlisting}

Now about 00 and FF bytes combined: jumps (including conditional) and calls can pass execution flow forward or backwards, but very often,
within the limits of the current executable module.
If forward, displacement is not very big and also padded with zeros.
If backwards, displacement is represented as negative value, so padded with FF bytes.
For example, transfer execution flow forward:

\begin{lstlisting}[style=customasmx86]
E8 43 0C 00 00                    call    _function1
E8 5C 00 00 00                    call    _function2
0F 84 F0 0A 00 00                 jz      loc_4F09A0
0F 84 EB 00 00 00                 jz      loc_4EFBB8
\end{lstlisting}

Backwards:

\begin{lstlisting}[style=customasmx86]
E8 79 0C FE FF                    call    _function1
E8 F4 16 FF FF                    call    _function2
0F 84 F8 FB FF FF                 jz      loc_8212BC
0F 84 06 FD FF FF                 jz      loc_FF1E7D
\end{lstlisting}

FF byte is also very often occurred in negative displacements like these:

\begin{lstlisting}[style=customasmx86]
8D 85 1E FF FF FF                 lea     eax, [ebp-0E2h]
8D 95 F8 5C FF FF                 lea     edx, [ebp-0A308h]
\end{lstlisting}

So far so good. Now we have to try various 16-byte keys, decrypt executable section and measure how often 00, FF and 8B bytes are occurred.
Let's also keep in sight how PCBC decryption works:

\begin{figure}[H]
\centering
\myincludegraphics{examples/simple_exec_crypto/640px-PCBC_decryption.png}
\caption{Propagating Cipher Block Chaining decryption (image is taken from Wikipedia article)}
\end{figure}

The good news is that we don't really have to decrypt whole piece of data, but only slice by slice, this is exactly how I did in my previous example: \myref{XOR_mask_2}.

Now I'm trying all possible bytes (0..255) for each byte in key and just pick the byte producing maximal amount of 00/FF/8B bytes in a decrypted slice:

\begin{lstlisting}[style=custompy]
#!/usr/bin/env python
import sys, hexdump, array, string, operator

KEY_LEN=16

def chunks(l, n):
    # split n by l-byte chunks
    # https://stackoverflow.com/q/312443
    n = max(1, n)
    return [l[i:i + n] for i in range(0, len(l), n)]

def read_file(fname):
    file=open(fname, mode='rb')
    content=file.read()
    file.close()
    return content

def decrypt_byte (c, key):
    return chr((ord(c)-key) % 256)

def XOR_PCBC_step (IV, buf, k):
    prev=IV
    rt=""
    for c in buf:
	new_c=decrypt_byte(c, k)
        plain=chr(ord(new_c)^ord(prev))
	prev=chr(ord(c)^ord(plain))
	rt=rt+plain
    return rt

each_Nth_byte=[""]*KEY_LEN

content=read_file(sys.argv[1])
# split input by 16-byte chunks:
all_chunks=chunks(content, KEY_LEN)
for c in all_chunks:
    for i in range(KEY_LEN):
        each_Nth_byte[i]=each_Nth_byte[i] + c[i]

# try each byte of key
for N in range(KEY_LEN):
    print "N=", N
    stat={}
    for i in range(256):
        tmp_key=chr(i)
	tmp=XOR_PCBC_step(tmp_key,each_Nth_byte[N], N)
        # count 0, FFs and 8Bs in decrypted buffer:
	important_bytes=tmp.count('\x00')+tmp.count('\xFF')+tmp.count('\x8B')
	stat[i]=important_bytes
    sorted_stat = sorted(stat.iteritems(), key=operator.itemgetter(1), reverse=True)
    print sorted_stat[0]
\end{lstlisting}

(Source code can be downloaded \href{\GitHubBlobMasterURL/examples/simple_exec_crypto/files/decrypt.py}{here}.)

I run it and here is a key for which 00/FF/8B bytes presence in decrypted buffer is maximal:

\begin{lstlisting}
N= 0
(147, 1224)
N= 1
(94, 1327)
N= 2
(252, 1223)
N= 3
(218, 1266)
N= 4
(38, 1209)
N= 5
(192, 1378)
N= 6
(199, 1204)
N= 7
(213, 1332)
N= 8
(225, 1251)
N= 9
(112, 1223)
N= 10
(143, 1177)
N= 11
(108, 1286)
N= 12
(10, 1164)
N= 13
(3, 1271)
N= 14
(128, 1253)
N= 15
(232, 1330)
\end{lstlisting}

Let's write decryption utility with the key we got:

\begin{lstlisting}[style=custompy]
#!/usr/bin/env python
import sys, hexdump, array

def xor_strings(s,t):
    # \verb|https://en.wikipedia.org/wiki/XOR_cipher#Example_implementation|
    """xor two strings together"""
    return "".join(chr(ord(a)^ord(b)) for a,b in zip(s,t))

IV=array.array('B', [147, 94, 252, 218, 38, 192, 199, 213, 225, 112, 143, 108, 10, 3, 128, 232]).tostring()

def chunks(l, n):
    n = max(1, n)
    return [l[i:i + n] for i in range(0, len(l), n)]

def read_file(fname):
    file=open(fname, mode='rb')
    content=file.read()
    file.close()
    return content

def decrypt_byte(i, k):
    return chr ((ord(i)-k) % 256)

def decrypt(buf):
    return "".join(decrypt_byte(buf[i], i) for i in range(16))

fout=open(sys.argv[2], mode='wb')

prev=IV
content=read_file(sys.argv[1])
tmp=chunks(content, 16)
for c in tmp:
    new_c=decrypt(c)
    p=xor_strings (new_c, prev)
    prev=xor_strings(c, p)
    fout.write(p)
fout.close()
\end{lstlisting}

(Source code can be downloaded \href{\GitHubBlobMasterURL/examples/simple_exec_crypto/files/decrypt2.py}{here}.)

Let's check resulting file:

\lstinputlisting{examples/simple_exec_crypto/objdump_result.txt}

Yes, this is seems correctly disassembled piece of x86 code.
The whole decryped file can be downloaded \href{\GitHubBlobMasterURL/examples/simple_exec_crypto/files/decrypted.bin}{here}.

In fact, this is text section from regedit.exe from Windows 7.
But this example is based on a real case I encountered, so just executable is different (and key), algorithm is the same.

\subsection{Other ideas to consider}

What if I would fail with such simple frequency analysis?
There are other ideas on how to measure correctness of decrypted/decompressed x86 code:

\begin{itemize}

\item Many modern compilers aligns functions on 0x10 border.
So the space left before is filled with NOPs (0x90) or other NOP instructions with known opcodes: \myref{sec:npad}.

\item Perhaps, the most frequent pattern in any assembly language is function call:\\
\TT{PUSH chain / CALL / ADD ESP, X}.
This sequence can easily detected and found.
I've even gathered statistics about average number of function arguments: \myref{args_stat}.
(Hence, this is average length of PUSH chain.)

\end{itemize}

Read more about incorrectly/correctly disassembled code: \myref{ISA_detect}.
}\RU{\subsection{Простое шифрование используя XOR-маску}
\label{XOR_mask_1}

Я нашел одну старую игру в стиле interactive fiction в архиве \emph{if-archive}\footnote{\url{http://www.ifarchive.org/}}:

\begin{lstlisting}
The New Castle v3.5 - Text/Adventure Game
in the style of the original Infocom (tm)
type games, Zork, Collosal Cave (Adventure),
etc.  Can you solve the mystery of the
abandoned castle?
Shareware from Software Customization.
Software Customization [ASP] Version 3.5 Feb. 2000
\end{lstlisting}

Можно скачать здесь: \url{\GitHubBlobMasterURL/ff/XOR/mask_1/files/newcastle.tgz}.

Там внутри есть файл (с названием \emph{castle.dbf}), который явно зашифрован, но не настоящим криптоалгоритмом,
и оне сжат, это что-то куда проще.
Я бы даже не стал измерять уровень энтропии (\myref{entropy}) этого файла, потому что я итак уверен, что он низкий.
Вот как он выглядит в Midnight Commander:

\begin{figure}[H]
\centering
\myincludegraphics{ff/XOR/mask_1/mc_encrypted.png}
\caption{Зашифрованный файл в Midnight Commander}
\end{figure}

Зашифрованный файл можно скачать здесь:
\url{\GitHubBlobMasterURL/ff/XOR/mask_1/files/castle.dbf.bz2}.

Можно ли расшифровать его без доступа к программе, используя просто этот файл?

Тут явно просматривается повторяющаяся строка. 
Если использовалось простое шифрование с XOR-маской, такие повторяющиеся строки это явное свидетельство,
потому что, вероятно, тут были длинные лакуны с нулевыми байтами, которые, в свою очередь, присутствуют
во мноигих исполняемых файлах, и в остальных бинарных файлах.

\myindex{UNIX!xxd}
Вот дам начала этого файла используя утилиту \emph{xxd} из UNIX:

\lstinputlisting{ff/XOR/mask_1/xxd_result.txt}

Давайте держаться за повторяющуюся строку \TT{iubgv}.
Глядя на этот дамп, мы можем легко увидеть, что период повторений этой строки это 0x51 или 81.
Вероятно, 81 это длина блока?
Длина файла 1658961, и она может быть поделена на 81 без остатка (и тогда там 20481 блоков).

Теперь я буду использовать Mathematica для анализа, есть ли тут повторяющиеся 81-байтные блоки в файле?
Я разделю входной файл на 81-байтные блоки и затем использую ф-цию
\emph{Tally[]}\footnote{\url{https://reference.wolfram.com/language/ref/Tally.html}}
которая просто считает, сколько раз каждый элемент встретился во входном списке.
Вывод Tally не отсортирован, так что я также добавлю ф-цию \emph{Sort[]} для сортировки его по кол-ву вхождений
в нисходящем порядке.

\begin{lstlisting}[style=custommath]
input = BinaryReadList["/home/dennis/.../castle.dbf"];

blocks = Partition[input, 81];

stat = Sort[Tally[blocks], #1[[2]] > #2[[2]] &]
\end{lstlisting}

И вот вывод:

\begin{lstlisting}[style=custommath]
{{{80, 103, 2, 116, 113, 102, 118, 25, 99, 8, 19, 23, 116, 125, 107, 
   25, 99, 109, 114, 102, 14, 121, 115, 31, 9, 117, 113, 111, 5, 4, 
   127, 28, 122, 101, 8, 110, 14, 18, 124, 106, 16, 20, 104, 119, 8, 
   109, 26, 106, 9, 97, 13, 99, 15, 119, 20, 105, 117, 98, 103, 118, 
   1, 126, 29, 97, 122, 17, 15, 114, 110, 3, 5, 125, 125, 99, 126, 
   119, 102, 30, 122, 2, 117}, 1739}, 
{{80, 100, 2, 116, 113, 102, 118, 25, 99, 8, 19, 23, 116, 
   125, 107, 25, 99, 109, 114, 102, 14, 121, 115, 31, 9, 117, 113, 
   111, 5, 4, 127, 28, 122, 101, 8, 110, 14, 18, 124, 106, 16, 20, 
   104, 119, 8, 109, 26, 106, 9, 97, 13, 99, 15, 119, 20, 105, 117, 
   98, 103, 118, 1, 126, 29, 97, 122, 17, 15, 114, 110, 3, 5, 125, 
   125, 99, 126, 119, 102, 30, 122, 2, 117}, 1422}, 
{{80, 101, 2, 116, 113, 102, 118, 25, 99, 8, 19, 23, 116, 
   125, 107, 25, 99, 109, 114, 102, 14, 121, 115, 31, 9, 117, 113, 
   111, 5, 4, 127, 28, 122, 101, 8, 110, 14, 18, 124, 106, 16, 20, 
   104, 119, 8, 109, 26, 106, 9, 97, 13, 99, 15, 119, 20, 105, 117, 
   98, 103, 118, 1, 126, 29, 97, 122, 17, 15, 114, 110, 3, 5, 125, 
   125, 99, 126, 119, 102, 30, 122, 2, 117}, 1012},
{{80, 120, 2, 116, 113, 102, 118, 25, 99, 8, 19, 23, 116, 
   125, 107, 25, 99, 109, 114, 102, 14, 121, 115, 31, 9, 117, 113, 
   111, 5, 4, 127, 28, 122, 101, 8, 110, 14, 18, 124, 106, 16, 20, 
   104, 119, 8, 109, 26, 106, 9, 97, 13, 99, 15, 119, 20, 105, 117, 
   98, 103, 118, 1, 126, 29, 97, 122, 17, 15, 114, 110, 3, 5, 125, 
   125, 99, 126, 119, 102, 30, 122, 2, 117}, 377},

...

{{80, 2, 74, 49, 113, 21, 62, 88, 39, 71, 68, 23, 63, 51, 36, 78, 48, 
   108, 114, 102, 14, 121, 115, 31, 9, 117, 113, 111, 5, 4, 127, 28, 
   122, 101, 8, 110, 14, 18, 124, 106, 16, 20, 104, 119, 8, 109, 26, 
   106, 9, 97, 13, 99, 15, 119, 20, 105, 117, 98, 103, 118, 1, 126, 
   29, 97, 122, 17, 15, 114, 110, 3, 5, 125, 125, 99, 126, 119, 102, 
   30, 122, 2, 117}, 1},
{{80, 1, 74, 59, 113, 45, 56, 86, 52, 91, 19, 64, 60, 60, 63, 
   25, 38, 59, 59, 42, 14, 53, 38, 77, 66, 38, 113, 38, 75, 4, 43, 84,
    63, 101, 64, 43, 79, 64, 40, 57, 16, 91, 46, 119, 69, 40, 84, 117,
    9, 97, 13, 99, 15, 119, 20, 105, 117, 98, 103, 118, 1, 126, 29, 
   97, 122, 17, 15, 114, 110, 3, 5, 125, 125, 99, 126, 119, 102, 30, 
   122, 2, 117}, 1},
{{80, 2, 74, 49, 113, 49, 51, 92, 39, 8, 92, 81, 116, 62, 57, 
   80, 46, 40, 114, 36, 75, 56, 33, 76, 9, 55, 56, 59, 81, 65, 45, 28,
    60, 55, 93, 39, 90, 28, 124, 106, 16, 20, 104, 119, 8, 109, 26, 
   106, 9, 97, 13, 99, 15, 119, 20, 105, 117, 98, 103, 118, 1, 126, 
   29, 97, 122, 17, 15, 114, 110, 3, 5, 125, 125, 99, 126, 119, 102, 
   30, 122, 2, 117}, 1}}
\end{lstlisting}

Вывод Tally это список пар, каждая пара это 81-байтный блок и количество раз, сколько он встретился в файле.
Мы видим, что наиболее частно встречающийся блок это первый, он встретился 1739 раз.
Второй встретился 1422 раза. Есть и другие: 1012 раза, 377 раз, итд.
81-байтные блоки, встреченные лишь один раз, находятся в конце вывода.

Попробуем сравнить эти блоки. Первый и второй.
Есть ли в Mathematica ф-ция для сравнения списков/массивов?
Наверняка есть, но в педагогических целях, я буду использоват операцию XOR для сравнения.
Действительно: если байты во входных массивах равны друг другу, результат операции XOR это 0.
Если не равны, результат будет ненулевой.

Сравним первый блок (встречается 1739 раз) и второй (встречается 1422 раз):

\begin{lstlisting}[style=custommath]
In[]:= BitXor[stat[[1]][[1]], stat[[2]][[1]]]
Out[]= {0, 3, 0, 0, 0, 0, 0, 0, 0, 0, 0, 0, 0, 0, 0, 0, 0, 0, 0, \
0, 0, 0, 0, 0, 0, 0, 0, 0, 0, 0, 0, 0, 0, 0, 0, 0, 0, 0, 0, 0, 0, 0, \
0, 0, 0, 0, 0, 0, 0, 0, 0, 0, 0, 0, 0, 0, 0, 0, 0, 0, 0, 0, 0, 0, 0, \
0, 0, 0, 0, 0, 0, 0, 0, 0, 0, 0, 0, 0, 0, 0, 0}
\end{lstlisting}

Они отличаются только вторым байтом.

Сравним второй блок (встречается 1422 раза) и третий (встречается 1012 раз):

\begin{lstlisting}[style=custommath]
In[]:= BitXor[stat[[2]][[1]], stat[[3]][[1]]]
Out[]= {0, 1, 0, 0, 0, 0, 0, 0, 0, 0, 0, 0, 0, 0, 0, 0, 0, 0, 0, \
0, 0, 0, 0, 0, 0, 0, 0, 0, 0, 0, 0, 0, 0, 0, 0, 0, 0, 0, 0, 0, 0, 0, \
0, 0, 0, 0, 0, 0, 0, 0, 0, 0, 0, 0, 0, 0, 0, 0, 0, 0, 0, 0, 0, 0, 0, \
0, 0, 0, 0, 0, 0, 0, 0, 0, 0, 0, 0, 0, 0, 0, 0}
\end{lstlisting}

Они тоже отличаются только вторым байтом.

Так или иначе, попробуем использовать самый встречающийся блок как XOR-ключ и попробуем расшифровать первые 4 81-байтных
блока в файле:

\begin{lstlisting}[style=custommath]
In[]:= key = stat[[1]][[1]]
Out[]= {80, 103, 2, 116, 113, 102, 118, 25, 99, 8, 19, 23, 116, \
125, 107, 25, 99, 109, 114, 102, 14, 121, 115, 31, 9, 117, 113, 111, \
5, 4, 127, 28, 122, 101, 8, 110, 14, 18, 124, 106, 16, 20, 104, 119, \
8, 109, 26, 106, 9, 97, 13, 99, 15, 119, 20, 105, 117, 98, 103, 118, \
1, 126, 29, 97, 122, 17, 15, 114, 110, 3, 5, 125, 125, 99, 126, 119, \
102, 30, 122, 2, 117}

In[]:= ToASCII[val_] := If[val == 0, " ", FromCharacterCode[val, "PrintableASCII"]]

In[]:= DecryptBlockASCII[blk_] := Map[ToASCII[#] &, BitXor[key, blk]]

In[]:= DecryptBlockASCII[blocks[[1]]]
Out[]= {" ", " ", " ", " ", " ", " ", " ", " ", " ", " ", " ", " \
", " ", " ", " ", " ", " ", " ", " ", " ", " ", " ", " ", " ", " ", " \
", " ", " ", " ", " ", " ", " ", " ", " ", " ", " ", " ", " ", " ", " \
", " ", " ", " ", " ", " ", " ", " ", " ", " ", " ", " ", " ", " ", " \
", " ", " ", " ", " ", " ", " ", " ", " ", " ", " ", " ", " ", " ", " \
", " ", " ", " ", " ", " ", " ", " ", " ", " ", " ", " ", " ", " "}

In[]:= DecryptBlockASCII[blocks[[2]]]
Out[]= {" ", "e", "H", "E", " ", "W", "E", "E", "D", " ", "O", \
"F", " ", "C", "R", "I", "M", "E", " ", "B", "E", "A", "R", "S", " ", \
"B", "I", "T", "T", "E", "R", " ", "F", "R", "U", "I", "T", "?", \
" ", " ", " ", " ", " ", " ", " ", " ", " ", " ", " ", " ", " ", " ", \
" ", " ", " ", " ", " ", " ", " ", " ", " ", " ", " ", " ", " ", " ", \
" ", " ", " ", " ", " ", " ", " ", " ", " ", " ", " ", " ", " ", " ", \
" "}

In[]:= DecryptBlockASCII[blocks[[3]]]
Out[]= {" ", "?", " ", " ", " ", " ", " ", " ", " ", " ", " \
", " ", " ", " ", " ", " ", " ", " ", " ", " ", " ", " ", " ", " ", " \
", " ", " ", " ", " ", " ", " ", " ", " ", " ", " ", " ", " ", " ", " \
", " ", " ", " ", " ", " ", " ", " ", " ", " ", " ", " ", " ", " ", " \
", " ", " ", " ", " ", " ", " ", " ", " ", " ", " ", " ", " ", " ", " \
", " ", " ", " ", " ", " ", " ", " ", " ", " ", " ", " ", " ", " ", " \
"}

In[]:= DecryptBlockASCII[blocks[[4]]]
Out[]= {" ", "f", "H", "O", " ", "K", "N", "O", "W", "S", " ", \
"W", "H", "A", "T", " ", "E", "V", "I", "L", " ", "L", "U", "R", "K", \
"S", " ", "I", "N", " ", "T", "H", "E", " ", "H", "E", "A", "R", "T", \
"S", " ", "O", "F", " ", "M", "E", "N", "?", " ", " ", " ", " ", \
" ", " ", " ", " ", " ", " ", " ", " ", " ", " ", " ", " ", " ", " ", \
" ", " ", " ", " ", " ", " ", " ", " ", " ", " ", " ", " ", " ", " ", \
" "}
\end{lstlisting}

(Я заменил непечатаемые символы на \q{?}.)

Мы видим что первый и третий блоки пустые (или почти пустые),
но второй и четвертый имеют ясно различимые английские слова/фразы.
Похоже что наше предположение насчет ключа верно (как минимум частично).
Это означает, что самый встречающийся 81-байтный блок в файле находится в местах лакун с нулевыми байтами
или что-то в этом роде.

Попробуем расшифровать весь файл:

\begin{lstlisting}[style=custommath]
DecryptBlock[blk_] := BitXor[key, blk]

decrypted = Map[DecryptBlock[#] &, blocks];

BinaryWrite["/home/dennis/.../tmp", Flatten[decrypted]]

Close["/home/dennis/.../tmp"]
\end{lstlisting}

\begin{figure}[H]
\centering
\myincludegraphics{ff/XOR/mask_1/mc_decrypted1.png}
\caption{Расшифрованный файл в Midnight Commander, первая попытка}
\end{figure}

Выглядит как английские фразы для какой-то игры, но что-то не так.
Прежде всего, регистр инвертирован: фразы и некоторые слова начинаются со строчных букв,
в то время как остальные буквы заглавные.
Также, некоторые фразы начинаются с не тех букв.
Посмотрите на самую первую фразу: \q{eHE WEED OF CRIME BEARS BITTER FRUIT}.
Что такое \q{eHE}? Разве не \q{tHE} тут должно быть?
Возможно ли что наш ключ для дешифрования имеет неверный байт в этом месте?

Посмотрим снова на второй блок в файле, на ключ и на результат дешифрования:

\begin{lstlisting}[style=custommath]
In[]:= blocks[[2]]
Out[]= {80, 2, 74, 49, 113, 49, 51, 92, 39, 8, 92, 81, 116, 62, \
57, 80, 46, 40, 114, 36, 75, 56, 33, 76, 9, 55, 56, 59, 81, 65, 45, \
28, 60, 55, 93, 39, 90, 28, 124, 106, 16, 20, 104, 119, 8, 109, 26, \
106, 9, 97, 13, 99, 15, 119, 20, 105, 117, 98, 103, 118, 1, 126, 29, \
97, 122, 17, 15, 114, 110, 3, 5, 125, 125, 99, 126, 119, 102, 30, \
122, 2, 117}

In[]:= key
Out[]= {80, 103, 2, 116, 113, 102, 118, 25, 99, 8, 19, 23, 116, \
125, 107, 25, 99, 109, 114, 102, 14, 121, 115, 31, 9, 117, 113, 111, \
5, 4, 127, 28, 122, 101, 8, 110, 14, 18, 124, 106, 16, 20, 104, 119, \
8, 109, 26, 106, 9, 97, 13, 99, 15, 119, 20, 105, 117, 98, 103, 118, \
1, 126, 29, 97, 122, 17, 15, 114, 110, 3, 5, 125, 125, 99, 126, 119, \
102, 30, 122, 2, 117}

In[]:= BitXor[key, blocks[[2]]]
Out[]= {0, 101, 72, 69, 0, 87, 69, 69, 68, 0, 79, 70, 0, 67, 82, \
73, 77, 69, 0, 66, 69, 65, 82, 83, 0, 66, 73, 84, 84, 69, 82, 0, 70, \
82, 85, 73, 84, 14, 0, 0, 0, 0, 0, 0, 0, 0, 0, 0, 0, 0, 0, 0, 0, 0, \
0, 0, 0, 0, 0, 0, 0, 0, 0, 0, 0, 0, 0, 0, 0, 0, 0, 0, 0, 0, 0, 0, 0, \
0, 0, 0, 0}
\end{lstlisting}

Зашифрованный байт это 2, байт из ключа это 103, $2 \oplus 103=101$ и 101 это ASCII-код символа \q{e}.
Чему должен равнятся этот байт ключа, чтобы ASCII-код был 116 (для символа  \q{t})?
$2 \oplus 116=118$, присвоим 118 второму байту в ключе \dots

\begin{lstlisting}[style=custommath]
key = {80, 118, 2, 116, 113, 102, 118, 25, 99, 8, 19, 23, 116, 125, 
  107, 25, 99, 109, 114, 102, 14, 121, 115, 31, 9, 117, 113, 111, 5, 
  4, 127, 28, 122, 101, 8, 110, 14, 18, 124, 106, 16, 20, 104, 119, 8,
   109, 26, 106, 9, 97, 13, 99, 15, 119, 20, 105, 117, 98, 103, 118, 
  1, 126, 29, 97, 122, 17, 15, 114, 110, 3, 5, 125, 125, 99, 126, 119,
   102, 30, 122, 2, 117}
\end{lstlisting}

\dots и снова дешифруем весь файл.

\begin{figure}[H]
\centering
\myincludegraphics{ff/XOR/mask_1/mc_decrypted2.png}
\caption{Дешифрованный файл в Midnight Commander, вторая попытка}
\end{figure}

Ух ты, теперь грамматика корректна, и все фразы начинаются с корректных букв.
Но все таки, регистр подозрителен.
С чего бы разработчику игры записывать их в такой манере?
Может быть наш ключ все еще неправилен?

% TODO ASCII table somewhere in the book
Изучая таблицу ASCII мы можем заметить что ASCII-коды для букв в верхнем и нижнем регистре отличаются только на один бит
(6-й бит, если считать с первого, 0b100000):

\begin{figure}[H]
\centering
\includegraphics[width=0.7\textwidth]{ascii.png}
\caption{7-битная таблица \ac{ASCII} в Emacs}
\end{figure}

6-й бит, выставленный в нулевом байте, В десятичном виде это будет 32.
Но 32 это ASCII-код пробела!

Действительно, можно менять регистр просто применяя XOR к ASCII-коду, с 32 (больше об этом: \myref{toupper_bit}).

Возможно ли, что пустые лакуны в файле это не нулевые байты, а скорее содержащие пробелы?
Еще раз модифицируем наш XOR-ключ (я про-XOR-ю каждый байт ключа с 32):

\begin{lstlisting}[style=custommath]
(* "32" это скаляр, и "key" это вектор, но это OK *)

In[]:= key3 = BitXor[32, key]
Out[]= {112, 86, 34, 84, 81, 70, 86, 57, 67, 40, 51, 55, 84, 93, 75, \
57, 67, 77, 82, 70, 46, 89, 83, 63, 41, 85, 81, 79, 37, 36, 95, 60, \
90, 69, 40, 78, 46, 50, 92, 74, 48, 52, 72, 87, 40, 77, 58, 74, 41, \
65, 45, 67, 47, 87, 52, 73, 85, 66, 71, 86, 33, 94, 61, 65, 90, 49, \
47, 82, 78, 35, 37, 93, 93, 67, 94, 87, 70, 62, 90, 34, 85}

In[]:= DecryptBlock[blk_] := BitXor[key3, blk]
\end{lstlisting}

И снова дешифруем входной файл:

\begin{figure}[H]
\centering
\myincludegraphics{ff/XOR/mask_1/mc_decrypted.png}
\caption{Дешифрованный файл в Midnight Commander, последняя попытка}
\end{figure}

(Расшифрованный файл доступен здесь:
\url{\GitHubBlobMasterURL/ff/XOR/mask_1/files/decrypted.dat.bz2}.)

Несомненно, это корректный исходный файл.
Да, и мы видим числа в начале каждого блока. Должно быть это и есть источник некорректного XOR-ключа.
Как выходит, самый встречающийся 81-байтный блок в файле это блок заполненный пробелами и содержащий символ \q{1} на месте
второго байта.
Действительно, как-то так получилось что многие блоки здесь перемежаются с этим блоком.
Может быть это что-то вроде выравнивания (padding) для коротких фраз/сообщений?
Другой часто встречающийся 81-байтный блок также заполнен пробелами, но с другой цифрой, следовательно,
они отличаются только вторым байтом.

Вот и всё! Теперь мы можем написать утилиту для зашифрования файла назад, и, может быть, модифицировать его перед этим

Файл для Mathematica можно скачать здесь:\\
\url{\GitHubBlobMasterURL/ff/XOR/mask_1/files/XOR_mask_1.nb}.

Итог: XOR-шифрование не надежно вообще. Вероятно, разработчик игры хотел просто скрыть внутренности игры от игрока,
ничего более серьезного.
Все же, шифрование вроде этого крайне популярно вследствии его простоты, так что многие реверс инженеры обычно хорошо
с этим знакомы.

}\FR{\mysection{Fonction presque vide}
\label{Boolector}
\myindex{Boolector}
\myindex{x86!\Instructions!JMP}

Ceci est un morceau de code réel que j'ai trouvé dans Boolector\footnote{\url{https://boolector.github.io/}}:

\lstinputlisting[style=customc]{patterns/025_almost_empty/boolectormain.c}

Pourquoi quelqu'un ferait-il comme ça?
Je ne sais pas mais mon hypothèse est que \verb|boolector_main()| peut être compilée
dans une sorte de DLL ou bibliothèque dynamique, et appelée depuis une suite de test.
Certainement qu'une suite de test peut préparer les variables argc/argv comme
le ferait \ac{CRT}.

Il est intéressant de voir comment c'est compilé:

\lstinputlisting[caption=GCC 8.2 x64 \NonOptimizing (\assemblyOutput),style=customasmx86]{patterns/025_almost_empty/boolectormain_O0.s}

Ceci est OK, le prologue (non optimisé) déplace inutilement deux arguments,
\INS{CALL}, épilogue, \INS{RET}.
Mais regardons la version optimisée:

\lstinputlisting[caption=GCC 8.2 x64 \Optimizing (\assemblyOutput),style=customasmx86]{patterns/025_almost_empty/boolectormain_O3.s}

Aussi simple que ça: la pile et les registres ne sont pas touchés et \verb|boolector_main()|
a le même ensemble d'arguments.
Donc, tout ce que nous avons à faire est de passer l'exécution à une autre adresse.

Ceci est proche d'une \glslink{thunk function}{fonction thunk}.

Nous verons queelque chose de plus avancé plus tard: \myref{ARM_B_to_printf}, \myref{JMP_instead_of_RET}.
}

\renewcommand{\CURPATH}{advanced/127_inline_function}
\EN{% TODO translate
\mysection{Breaking simple executable cryptor}

I've got an executable file which is encrypted by relatively simple encryption.
\href{\GitHubBlobMasterURL/examples/simple_exec_crypto/files/cipher.bin}{Here is it} (only executable section is left here).

First, all encryption function does is just adds number of position in buffer to the byte.
Here is how this can be encoded in Python:

\begin{lstlisting}[caption=Python script,style=custompy]
#!/usr/bin/env python
def e(i, k):
    return chr ((ord(i)+k) % 256)

def encrypt(buf):
    return e(buf[0], 0)+ e(buf[1], 1)+ e(buf[2], 2) + e(buf[3], 3)+ e(buf[4], 4)+ e(buf[5], 5)+ e(buf[6], 6)+ e(buf[7], 7)+
           e(buf[8], 8)+ e(buf[9], 9)+ e(buf[10], 10)+ e(buf[11], 11)+ e(buf[12], 12)+ e(buf[13], 13)+ e(buf[14], 14)+ e(buf[15], 15)
\end{lstlisting}

Hence, if you encrypt buffer with 16 zeros, you'll get \emph{0, 1, 2, 3 ... 12, 13, 14, 15}.

\myindex{Propagating Cipher Block Chaining}
Propagating Cipher Block Chaining (PCBC) is also used, here is how it works:

\begin{figure}[H]
\centering
\myincludegraphics{examples/simple_exec_crypto/601px-PCBC_encryption.png}
\caption{Propagating Cipher Block Chaining encryption (image is taken from Wikipedia article)}
\end{figure}

The problem is that it's too boring to recover IV (Initialization Vector) each time.
Brute-force is also not an option, because IV is too long (16 bytes).
Let's see, if it's possible to recover IV for arbitrary encrypted executable file?

Let's try simple frequency analysis.
This is 32-bit x86 executable code, so let's gather statistics about most frequent bytes and opcodes.
I tried huge oracle.exe file from Oracle RDBMS version 11.2 for windows x86 and I've found that the most frequent byte (no surprise) is zero (~10\%).
The next most frequent byte is (again, no surprise) 0xFF (~5\%).
The next is 0x8B (~5\%).

\myindex{x86!\Instructions!MOV}
0x8B is opcode for \INS{MOV}, this is indeed one of the most busy x86 instructions.
Now what about popularity of zero byte?
If compiler needs to encode value bigger than 127, it has to use 32-bit displacement instead of 8-bit one, but large values are very rare,
so it is padded by zeros.
\myindex{x86!\Instructions!LEA}
\myindex{x86!\Instructions!PUSH}
\myindex{x86!\Instructions!CALL}
This is at least in \INS{LEA}, \INS{MOV}, \INS{PUSH}, \INS{CALL}.

For example:

\begin{lstlisting}[style=customasmx86]
8D B0 28 01 00 00                 lea     esi, [eax+128h]
8D BF 40 38 00 00                 lea     edi, [edi+3840h]
\end{lstlisting}

Displacements bigger than 127 are very popular, but they are rarely exceeds 0x10000
(indeed, such large memory buffers/structures are also rare).

Same story with \INS{MOV}, large constants are rare, the most heavily used are 0, 1, 10, 100, $2^n$, and so on.
Compiler has to pad small constants by zeros to represent them as 32-bit values:

\begin{lstlisting}[style=customasmx86]
BF 02 00 00 00                    mov     edi, 2
BF 01 00 00 00                    mov     edi, 1
\end{lstlisting}

Now about 00 and FF bytes combined: jumps (including conditional) and calls can pass execution flow forward or backwards, but very often,
within the limits of the current executable module.
If forward, displacement is not very big and also padded with zeros.
If backwards, displacement is represented as negative value, so padded with FF bytes.
For example, transfer execution flow forward:

\begin{lstlisting}[style=customasmx86]
E8 43 0C 00 00                    call    _function1
E8 5C 00 00 00                    call    _function2
0F 84 F0 0A 00 00                 jz      loc_4F09A0
0F 84 EB 00 00 00                 jz      loc_4EFBB8
\end{lstlisting}

Backwards:

\begin{lstlisting}[style=customasmx86]
E8 79 0C FE FF                    call    _function1
E8 F4 16 FF FF                    call    _function2
0F 84 F8 FB FF FF                 jz      loc_8212BC
0F 84 06 FD FF FF                 jz      loc_FF1E7D
\end{lstlisting}

FF byte is also very often occurred in negative displacements like these:

\begin{lstlisting}[style=customasmx86]
8D 85 1E FF FF FF                 lea     eax, [ebp-0E2h]
8D 95 F8 5C FF FF                 lea     edx, [ebp-0A308h]
\end{lstlisting}

So far so good. Now we have to try various 16-byte keys, decrypt executable section and measure how often 00, FF and 8B bytes are occurred.
Let's also keep in sight how PCBC decryption works:

\begin{figure}[H]
\centering
\myincludegraphics{examples/simple_exec_crypto/640px-PCBC_decryption.png}
\caption{Propagating Cipher Block Chaining decryption (image is taken from Wikipedia article)}
\end{figure}

The good news is that we don't really have to decrypt whole piece of data, but only slice by slice, this is exactly how I did in my previous example: \myref{XOR_mask_2}.

Now I'm trying all possible bytes (0..255) for each byte in key and just pick the byte producing maximal amount of 00/FF/8B bytes in a decrypted slice:

\begin{lstlisting}[style=custompy]
#!/usr/bin/env python
import sys, hexdump, array, string, operator

KEY_LEN=16

def chunks(l, n):
    # split n by l-byte chunks
    # https://stackoverflow.com/q/312443
    n = max(1, n)
    return [l[i:i + n] for i in range(0, len(l), n)]

def read_file(fname):
    file=open(fname, mode='rb')
    content=file.read()
    file.close()
    return content

def decrypt_byte (c, key):
    return chr((ord(c)-key) % 256)

def XOR_PCBC_step (IV, buf, k):
    prev=IV
    rt=""
    for c in buf:
	new_c=decrypt_byte(c, k)
        plain=chr(ord(new_c)^ord(prev))
	prev=chr(ord(c)^ord(plain))
	rt=rt+plain
    return rt

each_Nth_byte=[""]*KEY_LEN

content=read_file(sys.argv[1])
# split input by 16-byte chunks:
all_chunks=chunks(content, KEY_LEN)
for c in all_chunks:
    for i in range(KEY_LEN):
        each_Nth_byte[i]=each_Nth_byte[i] + c[i]

# try each byte of key
for N in range(KEY_LEN):
    print "N=", N
    stat={}
    for i in range(256):
        tmp_key=chr(i)
	tmp=XOR_PCBC_step(tmp_key,each_Nth_byte[N], N)
        # count 0, FFs and 8Bs in decrypted buffer:
	important_bytes=tmp.count('\x00')+tmp.count('\xFF')+tmp.count('\x8B')
	stat[i]=important_bytes
    sorted_stat = sorted(stat.iteritems(), key=operator.itemgetter(1), reverse=True)
    print sorted_stat[0]
\end{lstlisting}

(Source code can be downloaded \href{\GitHubBlobMasterURL/examples/simple_exec_crypto/files/decrypt.py}{here}.)

I run it and here is a key for which 00/FF/8B bytes presence in decrypted buffer is maximal:

\begin{lstlisting}
N= 0
(147, 1224)
N= 1
(94, 1327)
N= 2
(252, 1223)
N= 3
(218, 1266)
N= 4
(38, 1209)
N= 5
(192, 1378)
N= 6
(199, 1204)
N= 7
(213, 1332)
N= 8
(225, 1251)
N= 9
(112, 1223)
N= 10
(143, 1177)
N= 11
(108, 1286)
N= 12
(10, 1164)
N= 13
(3, 1271)
N= 14
(128, 1253)
N= 15
(232, 1330)
\end{lstlisting}

Let's write decryption utility with the key we got:

\begin{lstlisting}[style=custompy]
#!/usr/bin/env python
import sys, hexdump, array

def xor_strings(s,t):
    # \verb|https://en.wikipedia.org/wiki/XOR_cipher#Example_implementation|
    """xor two strings together"""
    return "".join(chr(ord(a)^ord(b)) for a,b in zip(s,t))

IV=array.array('B', [147, 94, 252, 218, 38, 192, 199, 213, 225, 112, 143, 108, 10, 3, 128, 232]).tostring()

def chunks(l, n):
    n = max(1, n)
    return [l[i:i + n] for i in range(0, len(l), n)]

def read_file(fname):
    file=open(fname, mode='rb')
    content=file.read()
    file.close()
    return content

def decrypt_byte(i, k):
    return chr ((ord(i)-k) % 256)

def decrypt(buf):
    return "".join(decrypt_byte(buf[i], i) for i in range(16))

fout=open(sys.argv[2], mode='wb')

prev=IV
content=read_file(sys.argv[1])
tmp=chunks(content, 16)
for c in tmp:
    new_c=decrypt(c)
    p=xor_strings (new_c, prev)
    prev=xor_strings(c, p)
    fout.write(p)
fout.close()
\end{lstlisting}

(Source code can be downloaded \href{\GitHubBlobMasterURL/examples/simple_exec_crypto/files/decrypt2.py}{here}.)

Let's check resulting file:

\lstinputlisting{examples/simple_exec_crypto/objdump_result.txt}

Yes, this is seems correctly disassembled piece of x86 code.
The whole decryped file can be downloaded \href{\GitHubBlobMasterURL/examples/simple_exec_crypto/files/decrypted.bin}{here}.

In fact, this is text section from regedit.exe from Windows 7.
But this example is based on a real case I encountered, so just executable is different (and key), algorithm is the same.

\subsection{Other ideas to consider}

What if I would fail with such simple frequency analysis?
There are other ideas on how to measure correctness of decrypted/decompressed x86 code:

\begin{itemize}

\item Many modern compilers aligns functions on 0x10 border.
So the space left before is filled with NOPs (0x90) or other NOP instructions with known opcodes: \myref{sec:npad}.

\item Perhaps, the most frequent pattern in any assembly language is function call:\\
\TT{PUSH chain / CALL / ADD ESP, X}.
This sequence can easily detected and found.
I've even gathered statistics about average number of function arguments: \myref{args_stat}.
(Hence, this is average length of PUSH chain.)

\end{itemize}

Read more about incorrectly/correctly disassembled code: \myref{ISA_detect}.
}\RU{\subsection{Простое шифрование используя XOR-маску}
\label{XOR_mask_1}

Я нашел одну старую игру в стиле interactive fiction в архиве \emph{if-archive}\footnote{\url{http://www.ifarchive.org/}}:

\begin{lstlisting}
The New Castle v3.5 - Text/Adventure Game
in the style of the original Infocom (tm)
type games, Zork, Collosal Cave (Adventure),
etc.  Can you solve the mystery of the
abandoned castle?
Shareware from Software Customization.
Software Customization [ASP] Version 3.5 Feb. 2000
\end{lstlisting}

Можно скачать здесь: \url{\GitHubBlobMasterURL/ff/XOR/mask_1/files/newcastle.tgz}.

Там внутри есть файл (с названием \emph{castle.dbf}), который явно зашифрован, но не настоящим криптоалгоритмом,
и оне сжат, это что-то куда проще.
Я бы даже не стал измерять уровень энтропии (\myref{entropy}) этого файла, потому что я итак уверен, что он низкий.
Вот как он выглядит в Midnight Commander:

\begin{figure}[H]
\centering
\myincludegraphics{ff/XOR/mask_1/mc_encrypted.png}
\caption{Зашифрованный файл в Midnight Commander}
\end{figure}

Зашифрованный файл можно скачать здесь:
\url{\GitHubBlobMasterURL/ff/XOR/mask_1/files/castle.dbf.bz2}.

Можно ли расшифровать его без доступа к программе, используя просто этот файл?

Тут явно просматривается повторяющаяся строка. 
Если использовалось простое шифрование с XOR-маской, такие повторяющиеся строки это явное свидетельство,
потому что, вероятно, тут были длинные лакуны с нулевыми байтами, которые, в свою очередь, присутствуют
во мноигих исполняемых файлах, и в остальных бинарных файлах.

\myindex{UNIX!xxd}
Вот дам начала этого файла используя утилиту \emph{xxd} из UNIX:

\lstinputlisting{ff/XOR/mask_1/xxd_result.txt}

Давайте держаться за повторяющуюся строку \TT{iubgv}.
Глядя на этот дамп, мы можем легко увидеть, что период повторений этой строки это 0x51 или 81.
Вероятно, 81 это длина блока?
Длина файла 1658961, и она может быть поделена на 81 без остатка (и тогда там 20481 блоков).

Теперь я буду использовать Mathematica для анализа, есть ли тут повторяющиеся 81-байтные блоки в файле?
Я разделю входной файл на 81-байтные блоки и затем использую ф-цию
\emph{Tally[]}\footnote{\url{https://reference.wolfram.com/language/ref/Tally.html}}
которая просто считает, сколько раз каждый элемент встретился во входном списке.
Вывод Tally не отсортирован, так что я также добавлю ф-цию \emph{Sort[]} для сортировки его по кол-ву вхождений
в нисходящем порядке.

\begin{lstlisting}[style=custommath]
input = BinaryReadList["/home/dennis/.../castle.dbf"];

blocks = Partition[input, 81];

stat = Sort[Tally[blocks], #1[[2]] > #2[[2]] &]
\end{lstlisting}

И вот вывод:

\begin{lstlisting}[style=custommath]
{{{80, 103, 2, 116, 113, 102, 118, 25, 99, 8, 19, 23, 116, 125, 107, 
   25, 99, 109, 114, 102, 14, 121, 115, 31, 9, 117, 113, 111, 5, 4, 
   127, 28, 122, 101, 8, 110, 14, 18, 124, 106, 16, 20, 104, 119, 8, 
   109, 26, 106, 9, 97, 13, 99, 15, 119, 20, 105, 117, 98, 103, 118, 
   1, 126, 29, 97, 122, 17, 15, 114, 110, 3, 5, 125, 125, 99, 126, 
   119, 102, 30, 122, 2, 117}, 1739}, 
{{80, 100, 2, 116, 113, 102, 118, 25, 99, 8, 19, 23, 116, 
   125, 107, 25, 99, 109, 114, 102, 14, 121, 115, 31, 9, 117, 113, 
   111, 5, 4, 127, 28, 122, 101, 8, 110, 14, 18, 124, 106, 16, 20, 
   104, 119, 8, 109, 26, 106, 9, 97, 13, 99, 15, 119, 20, 105, 117, 
   98, 103, 118, 1, 126, 29, 97, 122, 17, 15, 114, 110, 3, 5, 125, 
   125, 99, 126, 119, 102, 30, 122, 2, 117}, 1422}, 
{{80, 101, 2, 116, 113, 102, 118, 25, 99, 8, 19, 23, 116, 
   125, 107, 25, 99, 109, 114, 102, 14, 121, 115, 31, 9, 117, 113, 
   111, 5, 4, 127, 28, 122, 101, 8, 110, 14, 18, 124, 106, 16, 20, 
   104, 119, 8, 109, 26, 106, 9, 97, 13, 99, 15, 119, 20, 105, 117, 
   98, 103, 118, 1, 126, 29, 97, 122, 17, 15, 114, 110, 3, 5, 125, 
   125, 99, 126, 119, 102, 30, 122, 2, 117}, 1012},
{{80, 120, 2, 116, 113, 102, 118, 25, 99, 8, 19, 23, 116, 
   125, 107, 25, 99, 109, 114, 102, 14, 121, 115, 31, 9, 117, 113, 
   111, 5, 4, 127, 28, 122, 101, 8, 110, 14, 18, 124, 106, 16, 20, 
   104, 119, 8, 109, 26, 106, 9, 97, 13, 99, 15, 119, 20, 105, 117, 
   98, 103, 118, 1, 126, 29, 97, 122, 17, 15, 114, 110, 3, 5, 125, 
   125, 99, 126, 119, 102, 30, 122, 2, 117}, 377},

...

{{80, 2, 74, 49, 113, 21, 62, 88, 39, 71, 68, 23, 63, 51, 36, 78, 48, 
   108, 114, 102, 14, 121, 115, 31, 9, 117, 113, 111, 5, 4, 127, 28, 
   122, 101, 8, 110, 14, 18, 124, 106, 16, 20, 104, 119, 8, 109, 26, 
   106, 9, 97, 13, 99, 15, 119, 20, 105, 117, 98, 103, 118, 1, 126, 
   29, 97, 122, 17, 15, 114, 110, 3, 5, 125, 125, 99, 126, 119, 102, 
   30, 122, 2, 117}, 1},
{{80, 1, 74, 59, 113, 45, 56, 86, 52, 91, 19, 64, 60, 60, 63, 
   25, 38, 59, 59, 42, 14, 53, 38, 77, 66, 38, 113, 38, 75, 4, 43, 84,
    63, 101, 64, 43, 79, 64, 40, 57, 16, 91, 46, 119, 69, 40, 84, 117,
    9, 97, 13, 99, 15, 119, 20, 105, 117, 98, 103, 118, 1, 126, 29, 
   97, 122, 17, 15, 114, 110, 3, 5, 125, 125, 99, 126, 119, 102, 30, 
   122, 2, 117}, 1},
{{80, 2, 74, 49, 113, 49, 51, 92, 39, 8, 92, 81, 116, 62, 57, 
   80, 46, 40, 114, 36, 75, 56, 33, 76, 9, 55, 56, 59, 81, 65, 45, 28,
    60, 55, 93, 39, 90, 28, 124, 106, 16, 20, 104, 119, 8, 109, 26, 
   106, 9, 97, 13, 99, 15, 119, 20, 105, 117, 98, 103, 118, 1, 126, 
   29, 97, 122, 17, 15, 114, 110, 3, 5, 125, 125, 99, 126, 119, 102, 
   30, 122, 2, 117}, 1}}
\end{lstlisting}

Вывод Tally это список пар, каждая пара это 81-байтный блок и количество раз, сколько он встретился в файле.
Мы видим, что наиболее частно встречающийся блок это первый, он встретился 1739 раз.
Второй встретился 1422 раза. Есть и другие: 1012 раза, 377 раз, итд.
81-байтные блоки, встреченные лишь один раз, находятся в конце вывода.

Попробуем сравнить эти блоки. Первый и второй.
Есть ли в Mathematica ф-ция для сравнения списков/массивов?
Наверняка есть, но в педагогических целях, я буду использоват операцию XOR для сравнения.
Действительно: если байты во входных массивах равны друг другу, результат операции XOR это 0.
Если не равны, результат будет ненулевой.

Сравним первый блок (встречается 1739 раз) и второй (встречается 1422 раз):

\begin{lstlisting}[style=custommath]
In[]:= BitXor[stat[[1]][[1]], stat[[2]][[1]]]
Out[]= {0, 3, 0, 0, 0, 0, 0, 0, 0, 0, 0, 0, 0, 0, 0, 0, 0, 0, 0, \
0, 0, 0, 0, 0, 0, 0, 0, 0, 0, 0, 0, 0, 0, 0, 0, 0, 0, 0, 0, 0, 0, 0, \
0, 0, 0, 0, 0, 0, 0, 0, 0, 0, 0, 0, 0, 0, 0, 0, 0, 0, 0, 0, 0, 0, 0, \
0, 0, 0, 0, 0, 0, 0, 0, 0, 0, 0, 0, 0, 0, 0, 0}
\end{lstlisting}

Они отличаются только вторым байтом.

Сравним второй блок (встречается 1422 раза) и третий (встречается 1012 раз):

\begin{lstlisting}[style=custommath]
In[]:= BitXor[stat[[2]][[1]], stat[[3]][[1]]]
Out[]= {0, 1, 0, 0, 0, 0, 0, 0, 0, 0, 0, 0, 0, 0, 0, 0, 0, 0, 0, \
0, 0, 0, 0, 0, 0, 0, 0, 0, 0, 0, 0, 0, 0, 0, 0, 0, 0, 0, 0, 0, 0, 0, \
0, 0, 0, 0, 0, 0, 0, 0, 0, 0, 0, 0, 0, 0, 0, 0, 0, 0, 0, 0, 0, 0, 0, \
0, 0, 0, 0, 0, 0, 0, 0, 0, 0, 0, 0, 0, 0, 0, 0}
\end{lstlisting}

Они тоже отличаются только вторым байтом.

Так или иначе, попробуем использовать самый встречающийся блок как XOR-ключ и попробуем расшифровать первые 4 81-байтных
блока в файле:

\begin{lstlisting}[style=custommath]
In[]:= key = stat[[1]][[1]]
Out[]= {80, 103, 2, 116, 113, 102, 118, 25, 99, 8, 19, 23, 116, \
125, 107, 25, 99, 109, 114, 102, 14, 121, 115, 31, 9, 117, 113, 111, \
5, 4, 127, 28, 122, 101, 8, 110, 14, 18, 124, 106, 16, 20, 104, 119, \
8, 109, 26, 106, 9, 97, 13, 99, 15, 119, 20, 105, 117, 98, 103, 118, \
1, 126, 29, 97, 122, 17, 15, 114, 110, 3, 5, 125, 125, 99, 126, 119, \
102, 30, 122, 2, 117}

In[]:= ToASCII[val_] := If[val == 0, " ", FromCharacterCode[val, "PrintableASCII"]]

In[]:= DecryptBlockASCII[blk_] := Map[ToASCII[#] &, BitXor[key, blk]]

In[]:= DecryptBlockASCII[blocks[[1]]]
Out[]= {" ", " ", " ", " ", " ", " ", " ", " ", " ", " ", " ", " \
", " ", " ", " ", " ", " ", " ", " ", " ", " ", " ", " ", " ", " ", " \
", " ", " ", " ", " ", " ", " ", " ", " ", " ", " ", " ", " ", " ", " \
", " ", " ", " ", " ", " ", " ", " ", " ", " ", " ", " ", " ", " ", " \
", " ", " ", " ", " ", " ", " ", " ", " ", " ", " ", " ", " ", " ", " \
", " ", " ", " ", " ", " ", " ", " ", " ", " ", " ", " ", " ", " "}

In[]:= DecryptBlockASCII[blocks[[2]]]
Out[]= {" ", "e", "H", "E", " ", "W", "E", "E", "D", " ", "O", \
"F", " ", "C", "R", "I", "M", "E", " ", "B", "E", "A", "R", "S", " ", \
"B", "I", "T", "T", "E", "R", " ", "F", "R", "U", "I", "T", "?", \
" ", " ", " ", " ", " ", " ", " ", " ", " ", " ", " ", " ", " ", " ", \
" ", " ", " ", " ", " ", " ", " ", " ", " ", " ", " ", " ", " ", " ", \
" ", " ", " ", " ", " ", " ", " ", " ", " ", " ", " ", " ", " ", " ", \
" "}

In[]:= DecryptBlockASCII[blocks[[3]]]
Out[]= {" ", "?", " ", " ", " ", " ", " ", " ", " ", " ", " \
", " ", " ", " ", " ", " ", " ", " ", " ", " ", " ", " ", " ", " ", " \
", " ", " ", " ", " ", " ", " ", " ", " ", " ", " ", " ", " ", " ", " \
", " ", " ", " ", " ", " ", " ", " ", " ", " ", " ", " ", " ", " ", " \
", " ", " ", " ", " ", " ", " ", " ", " ", " ", " ", " ", " ", " ", " \
", " ", " ", " ", " ", " ", " ", " ", " ", " ", " ", " ", " ", " ", " \
"}

In[]:= DecryptBlockASCII[blocks[[4]]]
Out[]= {" ", "f", "H", "O", " ", "K", "N", "O", "W", "S", " ", \
"W", "H", "A", "T", " ", "E", "V", "I", "L", " ", "L", "U", "R", "K", \
"S", " ", "I", "N", " ", "T", "H", "E", " ", "H", "E", "A", "R", "T", \
"S", " ", "O", "F", " ", "M", "E", "N", "?", " ", " ", " ", " ", \
" ", " ", " ", " ", " ", " ", " ", " ", " ", " ", " ", " ", " ", " ", \
" ", " ", " ", " ", " ", " ", " ", " ", " ", " ", " ", " ", " ", " ", \
" "}
\end{lstlisting}

(Я заменил непечатаемые символы на \q{?}.)

Мы видим что первый и третий блоки пустые (или почти пустые),
но второй и четвертый имеют ясно различимые английские слова/фразы.
Похоже что наше предположение насчет ключа верно (как минимум частично).
Это означает, что самый встречающийся 81-байтный блок в файле находится в местах лакун с нулевыми байтами
или что-то в этом роде.

Попробуем расшифровать весь файл:

\begin{lstlisting}[style=custommath]
DecryptBlock[blk_] := BitXor[key, blk]

decrypted = Map[DecryptBlock[#] &, blocks];

BinaryWrite["/home/dennis/.../tmp", Flatten[decrypted]]

Close["/home/dennis/.../tmp"]
\end{lstlisting}

\begin{figure}[H]
\centering
\myincludegraphics{ff/XOR/mask_1/mc_decrypted1.png}
\caption{Расшифрованный файл в Midnight Commander, первая попытка}
\end{figure}

Выглядит как английские фразы для какой-то игры, но что-то не так.
Прежде всего, регистр инвертирован: фразы и некоторые слова начинаются со строчных букв,
в то время как остальные буквы заглавные.
Также, некоторые фразы начинаются с не тех букв.
Посмотрите на самую первую фразу: \q{eHE WEED OF CRIME BEARS BITTER FRUIT}.
Что такое \q{eHE}? Разве не \q{tHE} тут должно быть?
Возможно ли что наш ключ для дешифрования имеет неверный байт в этом месте?

Посмотрим снова на второй блок в файле, на ключ и на результат дешифрования:

\begin{lstlisting}[style=custommath]
In[]:= blocks[[2]]
Out[]= {80, 2, 74, 49, 113, 49, 51, 92, 39, 8, 92, 81, 116, 62, \
57, 80, 46, 40, 114, 36, 75, 56, 33, 76, 9, 55, 56, 59, 81, 65, 45, \
28, 60, 55, 93, 39, 90, 28, 124, 106, 16, 20, 104, 119, 8, 109, 26, \
106, 9, 97, 13, 99, 15, 119, 20, 105, 117, 98, 103, 118, 1, 126, 29, \
97, 122, 17, 15, 114, 110, 3, 5, 125, 125, 99, 126, 119, 102, 30, \
122, 2, 117}

In[]:= key
Out[]= {80, 103, 2, 116, 113, 102, 118, 25, 99, 8, 19, 23, 116, \
125, 107, 25, 99, 109, 114, 102, 14, 121, 115, 31, 9, 117, 113, 111, \
5, 4, 127, 28, 122, 101, 8, 110, 14, 18, 124, 106, 16, 20, 104, 119, \
8, 109, 26, 106, 9, 97, 13, 99, 15, 119, 20, 105, 117, 98, 103, 118, \
1, 126, 29, 97, 122, 17, 15, 114, 110, 3, 5, 125, 125, 99, 126, 119, \
102, 30, 122, 2, 117}

In[]:= BitXor[key, blocks[[2]]]
Out[]= {0, 101, 72, 69, 0, 87, 69, 69, 68, 0, 79, 70, 0, 67, 82, \
73, 77, 69, 0, 66, 69, 65, 82, 83, 0, 66, 73, 84, 84, 69, 82, 0, 70, \
82, 85, 73, 84, 14, 0, 0, 0, 0, 0, 0, 0, 0, 0, 0, 0, 0, 0, 0, 0, 0, \
0, 0, 0, 0, 0, 0, 0, 0, 0, 0, 0, 0, 0, 0, 0, 0, 0, 0, 0, 0, 0, 0, 0, \
0, 0, 0, 0}
\end{lstlisting}

Зашифрованный байт это 2, байт из ключа это 103, $2 \oplus 103=101$ и 101 это ASCII-код символа \q{e}.
Чему должен равнятся этот байт ключа, чтобы ASCII-код был 116 (для символа  \q{t})?
$2 \oplus 116=118$, присвоим 118 второму байту в ключе \dots

\begin{lstlisting}[style=custommath]
key = {80, 118, 2, 116, 113, 102, 118, 25, 99, 8, 19, 23, 116, 125, 
  107, 25, 99, 109, 114, 102, 14, 121, 115, 31, 9, 117, 113, 111, 5, 
  4, 127, 28, 122, 101, 8, 110, 14, 18, 124, 106, 16, 20, 104, 119, 8,
   109, 26, 106, 9, 97, 13, 99, 15, 119, 20, 105, 117, 98, 103, 118, 
  1, 126, 29, 97, 122, 17, 15, 114, 110, 3, 5, 125, 125, 99, 126, 119,
   102, 30, 122, 2, 117}
\end{lstlisting}

\dots и снова дешифруем весь файл.

\begin{figure}[H]
\centering
\myincludegraphics{ff/XOR/mask_1/mc_decrypted2.png}
\caption{Дешифрованный файл в Midnight Commander, вторая попытка}
\end{figure}

Ух ты, теперь грамматика корректна, и все фразы начинаются с корректных букв.
Но все таки, регистр подозрителен.
С чего бы разработчику игры записывать их в такой манере?
Может быть наш ключ все еще неправилен?

% TODO ASCII table somewhere in the book
Изучая таблицу ASCII мы можем заметить что ASCII-коды для букв в верхнем и нижнем регистре отличаются только на один бит
(6-й бит, если считать с первого, 0b100000):

\begin{figure}[H]
\centering
\includegraphics[width=0.7\textwidth]{ascii.png}
\caption{7-битная таблица \ac{ASCII} в Emacs}
\end{figure}

6-й бит, выставленный в нулевом байте, В десятичном виде это будет 32.
Но 32 это ASCII-код пробела!

Действительно, можно менять регистр просто применяя XOR к ASCII-коду, с 32 (больше об этом: \myref{toupper_bit}).

Возможно ли, что пустые лакуны в файле это не нулевые байты, а скорее содержащие пробелы?
Еще раз модифицируем наш XOR-ключ (я про-XOR-ю каждый байт ключа с 32):

\begin{lstlisting}[style=custommath]
(* "32" это скаляр, и "key" это вектор, но это OK *)

In[]:= key3 = BitXor[32, key]
Out[]= {112, 86, 34, 84, 81, 70, 86, 57, 67, 40, 51, 55, 84, 93, 75, \
57, 67, 77, 82, 70, 46, 89, 83, 63, 41, 85, 81, 79, 37, 36, 95, 60, \
90, 69, 40, 78, 46, 50, 92, 74, 48, 52, 72, 87, 40, 77, 58, 74, 41, \
65, 45, 67, 47, 87, 52, 73, 85, 66, 71, 86, 33, 94, 61, 65, 90, 49, \
47, 82, 78, 35, 37, 93, 93, 67, 94, 87, 70, 62, 90, 34, 85}

In[]:= DecryptBlock[blk_] := BitXor[key3, blk]
\end{lstlisting}

И снова дешифруем входной файл:

\begin{figure}[H]
\centering
\myincludegraphics{ff/XOR/mask_1/mc_decrypted.png}
\caption{Дешифрованный файл в Midnight Commander, последняя попытка}
\end{figure}

(Расшифрованный файл доступен здесь:
\url{\GitHubBlobMasterURL/ff/XOR/mask_1/files/decrypted.dat.bz2}.)

Несомненно, это корректный исходный файл.
Да, и мы видим числа в начале каждого блока. Должно быть это и есть источник некорректного XOR-ключа.
Как выходит, самый встречающийся 81-байтный блок в файле это блок заполненный пробелами и содержащий символ \q{1} на месте
второго байта.
Действительно, как-то так получилось что многие блоки здесь перемежаются с этим блоком.
Может быть это что-то вроде выравнивания (padding) для коротких фраз/сообщений?
Другой часто встречающийся 81-байтный блок также заполнен пробелами, но с другой цифрой, следовательно,
они отличаются только вторым байтом.

Вот и всё! Теперь мы можем написать утилиту для зашифрования файла назад, и, может быть, модифицировать его перед этим

Файл для Mathematica можно скачать здесь:\\
\url{\GitHubBlobMasterURL/ff/XOR/mask_1/files/XOR_mask_1.nb}.

Итог: XOR-шифрование не надежно вообще. Вероятно, разработчик игры хотел просто скрыть внутренности игры от игрока,
ничего более серьезного.
Все же, шифрование вроде этого крайне популярно вследствии его простоты, так что многие реверс инженеры обычно хорошо
с этим знакомы.

}\FR{\mysection{Fonction presque vide}
\label{Boolector}
\myindex{Boolector}
\myindex{x86!\Instructions!JMP}

Ceci est un morceau de code réel que j'ai trouvé dans Boolector\footnote{\url{https://boolector.github.io/}}:

\lstinputlisting[style=customc]{patterns/025_almost_empty/boolectormain.c}

Pourquoi quelqu'un ferait-il comme ça?
Je ne sais pas mais mon hypothèse est que \verb|boolector_main()| peut être compilée
dans une sorte de DLL ou bibliothèque dynamique, et appelée depuis une suite de test.
Certainement qu'une suite de test peut préparer les variables argc/argv comme
le ferait \ac{CRT}.

Il est intéressant de voir comment c'est compilé:

\lstinputlisting[caption=GCC 8.2 x64 \NonOptimizing (\assemblyOutput),style=customasmx86]{patterns/025_almost_empty/boolectormain_O0.s}

Ceci est OK, le prologue (non optimisé) déplace inutilement deux arguments,
\INS{CALL}, épilogue, \INS{RET}.
Mais regardons la version optimisée:

\lstinputlisting[caption=GCC 8.2 x64 \Optimizing (\assemblyOutput),style=customasmx86]{patterns/025_almost_empty/boolectormain_O3.s}

Aussi simple que ça: la pile et les registres ne sont pas touchés et \verb|boolector_main()|
a le même ensemble d'arguments.
Donc, tout ce que nous avons à faire est de passer l'exécution à une autre adresse.

Ceci est proche d'une \glslink{thunk function}{fonction thunk}.

Nous verons queelque chose de plus avancé plus tard: \myref{ARM_B_to_printf}, \myref{JMP_instead_of_RET}.
}

\renewcommand{\CURPATH}{advanced/130_C99_restrict}
\EN{% TODO translate
\mysection{Breaking simple executable cryptor}

I've got an executable file which is encrypted by relatively simple encryption.
\href{\GitHubBlobMasterURL/examples/simple_exec_crypto/files/cipher.bin}{Here is it} (only executable section is left here).

First, all encryption function does is just adds number of position in buffer to the byte.
Here is how this can be encoded in Python:

\begin{lstlisting}[caption=Python script,style=custompy]
#!/usr/bin/env python
def e(i, k):
    return chr ((ord(i)+k) % 256)

def encrypt(buf):
    return e(buf[0], 0)+ e(buf[1], 1)+ e(buf[2], 2) + e(buf[3], 3)+ e(buf[4], 4)+ e(buf[5], 5)+ e(buf[6], 6)+ e(buf[7], 7)+
           e(buf[8], 8)+ e(buf[9], 9)+ e(buf[10], 10)+ e(buf[11], 11)+ e(buf[12], 12)+ e(buf[13], 13)+ e(buf[14], 14)+ e(buf[15], 15)
\end{lstlisting}

Hence, if you encrypt buffer with 16 zeros, you'll get \emph{0, 1, 2, 3 ... 12, 13, 14, 15}.

\myindex{Propagating Cipher Block Chaining}
Propagating Cipher Block Chaining (PCBC) is also used, here is how it works:

\begin{figure}[H]
\centering
\myincludegraphics{examples/simple_exec_crypto/601px-PCBC_encryption.png}
\caption{Propagating Cipher Block Chaining encryption (image is taken from Wikipedia article)}
\end{figure}

The problem is that it's too boring to recover IV (Initialization Vector) each time.
Brute-force is also not an option, because IV is too long (16 bytes).
Let's see, if it's possible to recover IV for arbitrary encrypted executable file?

Let's try simple frequency analysis.
This is 32-bit x86 executable code, so let's gather statistics about most frequent bytes and opcodes.
I tried huge oracle.exe file from Oracle RDBMS version 11.2 for windows x86 and I've found that the most frequent byte (no surprise) is zero (~10\%).
The next most frequent byte is (again, no surprise) 0xFF (~5\%).
The next is 0x8B (~5\%).

\myindex{x86!\Instructions!MOV}
0x8B is opcode for \INS{MOV}, this is indeed one of the most busy x86 instructions.
Now what about popularity of zero byte?
If compiler needs to encode value bigger than 127, it has to use 32-bit displacement instead of 8-bit one, but large values are very rare,
so it is padded by zeros.
\myindex{x86!\Instructions!LEA}
\myindex{x86!\Instructions!PUSH}
\myindex{x86!\Instructions!CALL}
This is at least in \INS{LEA}, \INS{MOV}, \INS{PUSH}, \INS{CALL}.

For example:

\begin{lstlisting}[style=customasmx86]
8D B0 28 01 00 00                 lea     esi, [eax+128h]
8D BF 40 38 00 00                 lea     edi, [edi+3840h]
\end{lstlisting}

Displacements bigger than 127 are very popular, but they are rarely exceeds 0x10000
(indeed, such large memory buffers/structures are also rare).

Same story with \INS{MOV}, large constants are rare, the most heavily used are 0, 1, 10, 100, $2^n$, and so on.
Compiler has to pad small constants by zeros to represent them as 32-bit values:

\begin{lstlisting}[style=customasmx86]
BF 02 00 00 00                    mov     edi, 2
BF 01 00 00 00                    mov     edi, 1
\end{lstlisting}

Now about 00 and FF bytes combined: jumps (including conditional) and calls can pass execution flow forward or backwards, but very often,
within the limits of the current executable module.
If forward, displacement is not very big and also padded with zeros.
If backwards, displacement is represented as negative value, so padded with FF bytes.
For example, transfer execution flow forward:

\begin{lstlisting}[style=customasmx86]
E8 43 0C 00 00                    call    _function1
E8 5C 00 00 00                    call    _function2
0F 84 F0 0A 00 00                 jz      loc_4F09A0
0F 84 EB 00 00 00                 jz      loc_4EFBB8
\end{lstlisting}

Backwards:

\begin{lstlisting}[style=customasmx86]
E8 79 0C FE FF                    call    _function1
E8 F4 16 FF FF                    call    _function2
0F 84 F8 FB FF FF                 jz      loc_8212BC
0F 84 06 FD FF FF                 jz      loc_FF1E7D
\end{lstlisting}

FF byte is also very often occurred in negative displacements like these:

\begin{lstlisting}[style=customasmx86]
8D 85 1E FF FF FF                 lea     eax, [ebp-0E2h]
8D 95 F8 5C FF FF                 lea     edx, [ebp-0A308h]
\end{lstlisting}

So far so good. Now we have to try various 16-byte keys, decrypt executable section and measure how often 00, FF and 8B bytes are occurred.
Let's also keep in sight how PCBC decryption works:

\begin{figure}[H]
\centering
\myincludegraphics{examples/simple_exec_crypto/640px-PCBC_decryption.png}
\caption{Propagating Cipher Block Chaining decryption (image is taken from Wikipedia article)}
\end{figure}

The good news is that we don't really have to decrypt whole piece of data, but only slice by slice, this is exactly how I did in my previous example: \myref{XOR_mask_2}.

Now I'm trying all possible bytes (0..255) for each byte in key and just pick the byte producing maximal amount of 00/FF/8B bytes in a decrypted slice:

\begin{lstlisting}[style=custompy]
#!/usr/bin/env python
import sys, hexdump, array, string, operator

KEY_LEN=16

def chunks(l, n):
    # split n by l-byte chunks
    # https://stackoverflow.com/q/312443
    n = max(1, n)
    return [l[i:i + n] for i in range(0, len(l), n)]

def read_file(fname):
    file=open(fname, mode='rb')
    content=file.read()
    file.close()
    return content

def decrypt_byte (c, key):
    return chr((ord(c)-key) % 256)

def XOR_PCBC_step (IV, buf, k):
    prev=IV
    rt=""
    for c in buf:
	new_c=decrypt_byte(c, k)
        plain=chr(ord(new_c)^ord(prev))
	prev=chr(ord(c)^ord(plain))
	rt=rt+plain
    return rt

each_Nth_byte=[""]*KEY_LEN

content=read_file(sys.argv[1])
# split input by 16-byte chunks:
all_chunks=chunks(content, KEY_LEN)
for c in all_chunks:
    for i in range(KEY_LEN):
        each_Nth_byte[i]=each_Nth_byte[i] + c[i]

# try each byte of key
for N in range(KEY_LEN):
    print "N=", N
    stat={}
    for i in range(256):
        tmp_key=chr(i)
	tmp=XOR_PCBC_step(tmp_key,each_Nth_byte[N], N)
        # count 0, FFs and 8Bs in decrypted buffer:
	important_bytes=tmp.count('\x00')+tmp.count('\xFF')+tmp.count('\x8B')
	stat[i]=important_bytes
    sorted_stat = sorted(stat.iteritems(), key=operator.itemgetter(1), reverse=True)
    print sorted_stat[0]
\end{lstlisting}

(Source code can be downloaded \href{\GitHubBlobMasterURL/examples/simple_exec_crypto/files/decrypt.py}{here}.)

I run it and here is a key for which 00/FF/8B bytes presence in decrypted buffer is maximal:

\begin{lstlisting}
N= 0
(147, 1224)
N= 1
(94, 1327)
N= 2
(252, 1223)
N= 3
(218, 1266)
N= 4
(38, 1209)
N= 5
(192, 1378)
N= 6
(199, 1204)
N= 7
(213, 1332)
N= 8
(225, 1251)
N= 9
(112, 1223)
N= 10
(143, 1177)
N= 11
(108, 1286)
N= 12
(10, 1164)
N= 13
(3, 1271)
N= 14
(128, 1253)
N= 15
(232, 1330)
\end{lstlisting}

Let's write decryption utility with the key we got:

\begin{lstlisting}[style=custompy]
#!/usr/bin/env python
import sys, hexdump, array

def xor_strings(s,t):
    # \verb|https://en.wikipedia.org/wiki/XOR_cipher#Example_implementation|
    """xor two strings together"""
    return "".join(chr(ord(a)^ord(b)) for a,b in zip(s,t))

IV=array.array('B', [147, 94, 252, 218, 38, 192, 199, 213, 225, 112, 143, 108, 10, 3, 128, 232]).tostring()

def chunks(l, n):
    n = max(1, n)
    return [l[i:i + n] for i in range(0, len(l), n)]

def read_file(fname):
    file=open(fname, mode='rb')
    content=file.read()
    file.close()
    return content

def decrypt_byte(i, k):
    return chr ((ord(i)-k) % 256)

def decrypt(buf):
    return "".join(decrypt_byte(buf[i], i) for i in range(16))

fout=open(sys.argv[2], mode='wb')

prev=IV
content=read_file(sys.argv[1])
tmp=chunks(content, 16)
for c in tmp:
    new_c=decrypt(c)
    p=xor_strings (new_c, prev)
    prev=xor_strings(c, p)
    fout.write(p)
fout.close()
\end{lstlisting}

(Source code can be downloaded \href{\GitHubBlobMasterURL/examples/simple_exec_crypto/files/decrypt2.py}{here}.)

Let's check resulting file:

\lstinputlisting{examples/simple_exec_crypto/objdump_result.txt}

Yes, this is seems correctly disassembled piece of x86 code.
The whole decryped file can be downloaded \href{\GitHubBlobMasterURL/examples/simple_exec_crypto/files/decrypted.bin}{here}.

In fact, this is text section from regedit.exe from Windows 7.
But this example is based on a real case I encountered, so just executable is different (and key), algorithm is the same.

\subsection{Other ideas to consider}

What if I would fail with such simple frequency analysis?
There are other ideas on how to measure correctness of decrypted/decompressed x86 code:

\begin{itemize}

\item Many modern compilers aligns functions on 0x10 border.
So the space left before is filled with NOPs (0x90) or other NOP instructions with known opcodes: \myref{sec:npad}.

\item Perhaps, the most frequent pattern in any assembly language is function call:\\
\TT{PUSH chain / CALL / ADD ESP, X}.
This sequence can easily detected and found.
I've even gathered statistics about average number of function arguments: \myref{args_stat}.
(Hence, this is average length of PUSH chain.)

\end{itemize}

Read more about incorrectly/correctly disassembled code: \myref{ISA_detect}.
}\RU{\subsection{Простое шифрование используя XOR-маску}
\label{XOR_mask_1}

Я нашел одну старую игру в стиле interactive fiction в архиве \emph{if-archive}\footnote{\url{http://www.ifarchive.org/}}:

\begin{lstlisting}
The New Castle v3.5 - Text/Adventure Game
in the style of the original Infocom (tm)
type games, Zork, Collosal Cave (Adventure),
etc.  Can you solve the mystery of the
abandoned castle?
Shareware from Software Customization.
Software Customization [ASP] Version 3.5 Feb. 2000
\end{lstlisting}

Можно скачать здесь: \url{\GitHubBlobMasterURL/ff/XOR/mask_1/files/newcastle.tgz}.

Там внутри есть файл (с названием \emph{castle.dbf}), который явно зашифрован, но не настоящим криптоалгоритмом,
и оне сжат, это что-то куда проще.
Я бы даже не стал измерять уровень энтропии (\myref{entropy}) этого файла, потому что я итак уверен, что он низкий.
Вот как он выглядит в Midnight Commander:

\begin{figure}[H]
\centering
\myincludegraphics{ff/XOR/mask_1/mc_encrypted.png}
\caption{Зашифрованный файл в Midnight Commander}
\end{figure}

Зашифрованный файл можно скачать здесь:
\url{\GitHubBlobMasterURL/ff/XOR/mask_1/files/castle.dbf.bz2}.

Можно ли расшифровать его без доступа к программе, используя просто этот файл?

Тут явно просматривается повторяющаяся строка. 
Если использовалось простое шифрование с XOR-маской, такие повторяющиеся строки это явное свидетельство,
потому что, вероятно, тут были длинные лакуны с нулевыми байтами, которые, в свою очередь, присутствуют
во мноигих исполняемых файлах, и в остальных бинарных файлах.

\myindex{UNIX!xxd}
Вот дам начала этого файла используя утилиту \emph{xxd} из UNIX:

\lstinputlisting{ff/XOR/mask_1/xxd_result.txt}

Давайте держаться за повторяющуюся строку \TT{iubgv}.
Глядя на этот дамп, мы можем легко увидеть, что период повторений этой строки это 0x51 или 81.
Вероятно, 81 это длина блока?
Длина файла 1658961, и она может быть поделена на 81 без остатка (и тогда там 20481 блоков).

Теперь я буду использовать Mathematica для анализа, есть ли тут повторяющиеся 81-байтные блоки в файле?
Я разделю входной файл на 81-байтные блоки и затем использую ф-цию
\emph{Tally[]}\footnote{\url{https://reference.wolfram.com/language/ref/Tally.html}}
которая просто считает, сколько раз каждый элемент встретился во входном списке.
Вывод Tally не отсортирован, так что я также добавлю ф-цию \emph{Sort[]} для сортировки его по кол-ву вхождений
в нисходящем порядке.

\begin{lstlisting}[style=custommath]
input = BinaryReadList["/home/dennis/.../castle.dbf"];

blocks = Partition[input, 81];

stat = Sort[Tally[blocks], #1[[2]] > #2[[2]] &]
\end{lstlisting}

И вот вывод:

\begin{lstlisting}[style=custommath]
{{{80, 103, 2, 116, 113, 102, 118, 25, 99, 8, 19, 23, 116, 125, 107, 
   25, 99, 109, 114, 102, 14, 121, 115, 31, 9, 117, 113, 111, 5, 4, 
   127, 28, 122, 101, 8, 110, 14, 18, 124, 106, 16, 20, 104, 119, 8, 
   109, 26, 106, 9, 97, 13, 99, 15, 119, 20, 105, 117, 98, 103, 118, 
   1, 126, 29, 97, 122, 17, 15, 114, 110, 3, 5, 125, 125, 99, 126, 
   119, 102, 30, 122, 2, 117}, 1739}, 
{{80, 100, 2, 116, 113, 102, 118, 25, 99, 8, 19, 23, 116, 
   125, 107, 25, 99, 109, 114, 102, 14, 121, 115, 31, 9, 117, 113, 
   111, 5, 4, 127, 28, 122, 101, 8, 110, 14, 18, 124, 106, 16, 20, 
   104, 119, 8, 109, 26, 106, 9, 97, 13, 99, 15, 119, 20, 105, 117, 
   98, 103, 118, 1, 126, 29, 97, 122, 17, 15, 114, 110, 3, 5, 125, 
   125, 99, 126, 119, 102, 30, 122, 2, 117}, 1422}, 
{{80, 101, 2, 116, 113, 102, 118, 25, 99, 8, 19, 23, 116, 
   125, 107, 25, 99, 109, 114, 102, 14, 121, 115, 31, 9, 117, 113, 
   111, 5, 4, 127, 28, 122, 101, 8, 110, 14, 18, 124, 106, 16, 20, 
   104, 119, 8, 109, 26, 106, 9, 97, 13, 99, 15, 119, 20, 105, 117, 
   98, 103, 118, 1, 126, 29, 97, 122, 17, 15, 114, 110, 3, 5, 125, 
   125, 99, 126, 119, 102, 30, 122, 2, 117}, 1012},
{{80, 120, 2, 116, 113, 102, 118, 25, 99, 8, 19, 23, 116, 
   125, 107, 25, 99, 109, 114, 102, 14, 121, 115, 31, 9, 117, 113, 
   111, 5, 4, 127, 28, 122, 101, 8, 110, 14, 18, 124, 106, 16, 20, 
   104, 119, 8, 109, 26, 106, 9, 97, 13, 99, 15, 119, 20, 105, 117, 
   98, 103, 118, 1, 126, 29, 97, 122, 17, 15, 114, 110, 3, 5, 125, 
   125, 99, 126, 119, 102, 30, 122, 2, 117}, 377},

...

{{80, 2, 74, 49, 113, 21, 62, 88, 39, 71, 68, 23, 63, 51, 36, 78, 48, 
   108, 114, 102, 14, 121, 115, 31, 9, 117, 113, 111, 5, 4, 127, 28, 
   122, 101, 8, 110, 14, 18, 124, 106, 16, 20, 104, 119, 8, 109, 26, 
   106, 9, 97, 13, 99, 15, 119, 20, 105, 117, 98, 103, 118, 1, 126, 
   29, 97, 122, 17, 15, 114, 110, 3, 5, 125, 125, 99, 126, 119, 102, 
   30, 122, 2, 117}, 1},
{{80, 1, 74, 59, 113, 45, 56, 86, 52, 91, 19, 64, 60, 60, 63, 
   25, 38, 59, 59, 42, 14, 53, 38, 77, 66, 38, 113, 38, 75, 4, 43, 84,
    63, 101, 64, 43, 79, 64, 40, 57, 16, 91, 46, 119, 69, 40, 84, 117,
    9, 97, 13, 99, 15, 119, 20, 105, 117, 98, 103, 118, 1, 126, 29, 
   97, 122, 17, 15, 114, 110, 3, 5, 125, 125, 99, 126, 119, 102, 30, 
   122, 2, 117}, 1},
{{80, 2, 74, 49, 113, 49, 51, 92, 39, 8, 92, 81, 116, 62, 57, 
   80, 46, 40, 114, 36, 75, 56, 33, 76, 9, 55, 56, 59, 81, 65, 45, 28,
    60, 55, 93, 39, 90, 28, 124, 106, 16, 20, 104, 119, 8, 109, 26, 
   106, 9, 97, 13, 99, 15, 119, 20, 105, 117, 98, 103, 118, 1, 126, 
   29, 97, 122, 17, 15, 114, 110, 3, 5, 125, 125, 99, 126, 119, 102, 
   30, 122, 2, 117}, 1}}
\end{lstlisting}

Вывод Tally это список пар, каждая пара это 81-байтный блок и количество раз, сколько он встретился в файле.
Мы видим, что наиболее частно встречающийся блок это первый, он встретился 1739 раз.
Второй встретился 1422 раза. Есть и другие: 1012 раза, 377 раз, итд.
81-байтные блоки, встреченные лишь один раз, находятся в конце вывода.

Попробуем сравнить эти блоки. Первый и второй.
Есть ли в Mathematica ф-ция для сравнения списков/массивов?
Наверняка есть, но в педагогических целях, я буду использоват операцию XOR для сравнения.
Действительно: если байты во входных массивах равны друг другу, результат операции XOR это 0.
Если не равны, результат будет ненулевой.

Сравним первый блок (встречается 1739 раз) и второй (встречается 1422 раз):

\begin{lstlisting}[style=custommath]
In[]:= BitXor[stat[[1]][[1]], stat[[2]][[1]]]
Out[]= {0, 3, 0, 0, 0, 0, 0, 0, 0, 0, 0, 0, 0, 0, 0, 0, 0, 0, 0, \
0, 0, 0, 0, 0, 0, 0, 0, 0, 0, 0, 0, 0, 0, 0, 0, 0, 0, 0, 0, 0, 0, 0, \
0, 0, 0, 0, 0, 0, 0, 0, 0, 0, 0, 0, 0, 0, 0, 0, 0, 0, 0, 0, 0, 0, 0, \
0, 0, 0, 0, 0, 0, 0, 0, 0, 0, 0, 0, 0, 0, 0, 0}
\end{lstlisting}

Они отличаются только вторым байтом.

Сравним второй блок (встречается 1422 раза) и третий (встречается 1012 раз):

\begin{lstlisting}[style=custommath]
In[]:= BitXor[stat[[2]][[1]], stat[[3]][[1]]]
Out[]= {0, 1, 0, 0, 0, 0, 0, 0, 0, 0, 0, 0, 0, 0, 0, 0, 0, 0, 0, \
0, 0, 0, 0, 0, 0, 0, 0, 0, 0, 0, 0, 0, 0, 0, 0, 0, 0, 0, 0, 0, 0, 0, \
0, 0, 0, 0, 0, 0, 0, 0, 0, 0, 0, 0, 0, 0, 0, 0, 0, 0, 0, 0, 0, 0, 0, \
0, 0, 0, 0, 0, 0, 0, 0, 0, 0, 0, 0, 0, 0, 0, 0}
\end{lstlisting}

Они тоже отличаются только вторым байтом.

Так или иначе, попробуем использовать самый встречающийся блок как XOR-ключ и попробуем расшифровать первые 4 81-байтных
блока в файле:

\begin{lstlisting}[style=custommath]
In[]:= key = stat[[1]][[1]]
Out[]= {80, 103, 2, 116, 113, 102, 118, 25, 99, 8, 19, 23, 116, \
125, 107, 25, 99, 109, 114, 102, 14, 121, 115, 31, 9, 117, 113, 111, \
5, 4, 127, 28, 122, 101, 8, 110, 14, 18, 124, 106, 16, 20, 104, 119, \
8, 109, 26, 106, 9, 97, 13, 99, 15, 119, 20, 105, 117, 98, 103, 118, \
1, 126, 29, 97, 122, 17, 15, 114, 110, 3, 5, 125, 125, 99, 126, 119, \
102, 30, 122, 2, 117}

In[]:= ToASCII[val_] := If[val == 0, " ", FromCharacterCode[val, "PrintableASCII"]]

In[]:= DecryptBlockASCII[blk_] := Map[ToASCII[#] &, BitXor[key, blk]]

In[]:= DecryptBlockASCII[blocks[[1]]]
Out[]= {" ", " ", " ", " ", " ", " ", " ", " ", " ", " ", " ", " \
", " ", " ", " ", " ", " ", " ", " ", " ", " ", " ", " ", " ", " ", " \
", " ", " ", " ", " ", " ", " ", " ", " ", " ", " ", " ", " ", " ", " \
", " ", " ", " ", " ", " ", " ", " ", " ", " ", " ", " ", " ", " ", " \
", " ", " ", " ", " ", " ", " ", " ", " ", " ", " ", " ", " ", " ", " \
", " ", " ", " ", " ", " ", " ", " ", " ", " ", " ", " ", " ", " "}

In[]:= DecryptBlockASCII[blocks[[2]]]
Out[]= {" ", "e", "H", "E", " ", "W", "E", "E", "D", " ", "O", \
"F", " ", "C", "R", "I", "M", "E", " ", "B", "E", "A", "R", "S", " ", \
"B", "I", "T", "T", "E", "R", " ", "F", "R", "U", "I", "T", "?", \
" ", " ", " ", " ", " ", " ", " ", " ", " ", " ", " ", " ", " ", " ", \
" ", " ", " ", " ", " ", " ", " ", " ", " ", " ", " ", " ", " ", " ", \
" ", " ", " ", " ", " ", " ", " ", " ", " ", " ", " ", " ", " ", " ", \
" "}

In[]:= DecryptBlockASCII[blocks[[3]]]
Out[]= {" ", "?", " ", " ", " ", " ", " ", " ", " ", " ", " \
", " ", " ", " ", " ", " ", " ", " ", " ", " ", " ", " ", " ", " ", " \
", " ", " ", " ", " ", " ", " ", " ", " ", " ", " ", " ", " ", " ", " \
", " ", " ", " ", " ", " ", " ", " ", " ", " ", " ", " ", " ", " ", " \
", " ", " ", " ", " ", " ", " ", " ", " ", " ", " ", " ", " ", " ", " \
", " ", " ", " ", " ", " ", " ", " ", " ", " ", " ", " ", " ", " ", " \
"}

In[]:= DecryptBlockASCII[blocks[[4]]]
Out[]= {" ", "f", "H", "O", " ", "K", "N", "O", "W", "S", " ", \
"W", "H", "A", "T", " ", "E", "V", "I", "L", " ", "L", "U", "R", "K", \
"S", " ", "I", "N", " ", "T", "H", "E", " ", "H", "E", "A", "R", "T", \
"S", " ", "O", "F", " ", "M", "E", "N", "?", " ", " ", " ", " ", \
" ", " ", " ", " ", " ", " ", " ", " ", " ", " ", " ", " ", " ", " ", \
" ", " ", " ", " ", " ", " ", " ", " ", " ", " ", " ", " ", " ", " ", \
" "}
\end{lstlisting}

(Я заменил непечатаемые символы на \q{?}.)

Мы видим что первый и третий блоки пустые (или почти пустые),
но второй и четвертый имеют ясно различимые английские слова/фразы.
Похоже что наше предположение насчет ключа верно (как минимум частично).
Это означает, что самый встречающийся 81-байтный блок в файле находится в местах лакун с нулевыми байтами
или что-то в этом роде.

Попробуем расшифровать весь файл:

\begin{lstlisting}[style=custommath]
DecryptBlock[blk_] := BitXor[key, blk]

decrypted = Map[DecryptBlock[#] &, blocks];

BinaryWrite["/home/dennis/.../tmp", Flatten[decrypted]]

Close["/home/dennis/.../tmp"]
\end{lstlisting}

\begin{figure}[H]
\centering
\myincludegraphics{ff/XOR/mask_1/mc_decrypted1.png}
\caption{Расшифрованный файл в Midnight Commander, первая попытка}
\end{figure}

Выглядит как английские фразы для какой-то игры, но что-то не так.
Прежде всего, регистр инвертирован: фразы и некоторые слова начинаются со строчных букв,
в то время как остальные буквы заглавные.
Также, некоторые фразы начинаются с не тех букв.
Посмотрите на самую первую фразу: \q{eHE WEED OF CRIME BEARS BITTER FRUIT}.
Что такое \q{eHE}? Разве не \q{tHE} тут должно быть?
Возможно ли что наш ключ для дешифрования имеет неверный байт в этом месте?

Посмотрим снова на второй блок в файле, на ключ и на результат дешифрования:

\begin{lstlisting}[style=custommath]
In[]:= blocks[[2]]
Out[]= {80, 2, 74, 49, 113, 49, 51, 92, 39, 8, 92, 81, 116, 62, \
57, 80, 46, 40, 114, 36, 75, 56, 33, 76, 9, 55, 56, 59, 81, 65, 45, \
28, 60, 55, 93, 39, 90, 28, 124, 106, 16, 20, 104, 119, 8, 109, 26, \
106, 9, 97, 13, 99, 15, 119, 20, 105, 117, 98, 103, 118, 1, 126, 29, \
97, 122, 17, 15, 114, 110, 3, 5, 125, 125, 99, 126, 119, 102, 30, \
122, 2, 117}

In[]:= key
Out[]= {80, 103, 2, 116, 113, 102, 118, 25, 99, 8, 19, 23, 116, \
125, 107, 25, 99, 109, 114, 102, 14, 121, 115, 31, 9, 117, 113, 111, \
5, 4, 127, 28, 122, 101, 8, 110, 14, 18, 124, 106, 16, 20, 104, 119, \
8, 109, 26, 106, 9, 97, 13, 99, 15, 119, 20, 105, 117, 98, 103, 118, \
1, 126, 29, 97, 122, 17, 15, 114, 110, 3, 5, 125, 125, 99, 126, 119, \
102, 30, 122, 2, 117}

In[]:= BitXor[key, blocks[[2]]]
Out[]= {0, 101, 72, 69, 0, 87, 69, 69, 68, 0, 79, 70, 0, 67, 82, \
73, 77, 69, 0, 66, 69, 65, 82, 83, 0, 66, 73, 84, 84, 69, 82, 0, 70, \
82, 85, 73, 84, 14, 0, 0, 0, 0, 0, 0, 0, 0, 0, 0, 0, 0, 0, 0, 0, 0, \
0, 0, 0, 0, 0, 0, 0, 0, 0, 0, 0, 0, 0, 0, 0, 0, 0, 0, 0, 0, 0, 0, 0, \
0, 0, 0, 0}
\end{lstlisting}

Зашифрованный байт это 2, байт из ключа это 103, $2 \oplus 103=101$ и 101 это ASCII-код символа \q{e}.
Чему должен равнятся этот байт ключа, чтобы ASCII-код был 116 (для символа  \q{t})?
$2 \oplus 116=118$, присвоим 118 второму байту в ключе \dots

\begin{lstlisting}[style=custommath]
key = {80, 118, 2, 116, 113, 102, 118, 25, 99, 8, 19, 23, 116, 125, 
  107, 25, 99, 109, 114, 102, 14, 121, 115, 31, 9, 117, 113, 111, 5, 
  4, 127, 28, 122, 101, 8, 110, 14, 18, 124, 106, 16, 20, 104, 119, 8,
   109, 26, 106, 9, 97, 13, 99, 15, 119, 20, 105, 117, 98, 103, 118, 
  1, 126, 29, 97, 122, 17, 15, 114, 110, 3, 5, 125, 125, 99, 126, 119,
   102, 30, 122, 2, 117}
\end{lstlisting}

\dots и снова дешифруем весь файл.

\begin{figure}[H]
\centering
\myincludegraphics{ff/XOR/mask_1/mc_decrypted2.png}
\caption{Дешифрованный файл в Midnight Commander, вторая попытка}
\end{figure}

Ух ты, теперь грамматика корректна, и все фразы начинаются с корректных букв.
Но все таки, регистр подозрителен.
С чего бы разработчику игры записывать их в такой манере?
Может быть наш ключ все еще неправилен?

% TODO ASCII table somewhere in the book
Изучая таблицу ASCII мы можем заметить что ASCII-коды для букв в верхнем и нижнем регистре отличаются только на один бит
(6-й бит, если считать с первого, 0b100000):

\begin{figure}[H]
\centering
\includegraphics[width=0.7\textwidth]{ascii.png}
\caption{7-битная таблица \ac{ASCII} в Emacs}
\end{figure}

6-й бит, выставленный в нулевом байте, В десятичном виде это будет 32.
Но 32 это ASCII-код пробела!

Действительно, можно менять регистр просто применяя XOR к ASCII-коду, с 32 (больше об этом: \myref{toupper_bit}).

Возможно ли, что пустые лакуны в файле это не нулевые байты, а скорее содержащие пробелы?
Еще раз модифицируем наш XOR-ключ (я про-XOR-ю каждый байт ключа с 32):

\begin{lstlisting}[style=custommath]
(* "32" это скаляр, и "key" это вектор, но это OK *)

In[]:= key3 = BitXor[32, key]
Out[]= {112, 86, 34, 84, 81, 70, 86, 57, 67, 40, 51, 55, 84, 93, 75, \
57, 67, 77, 82, 70, 46, 89, 83, 63, 41, 85, 81, 79, 37, 36, 95, 60, \
90, 69, 40, 78, 46, 50, 92, 74, 48, 52, 72, 87, 40, 77, 58, 74, 41, \
65, 45, 67, 47, 87, 52, 73, 85, 66, 71, 86, 33, 94, 61, 65, 90, 49, \
47, 82, 78, 35, 37, 93, 93, 67, 94, 87, 70, 62, 90, 34, 85}

In[]:= DecryptBlock[blk_] := BitXor[key3, blk]
\end{lstlisting}

И снова дешифруем входной файл:

\begin{figure}[H]
\centering
\myincludegraphics{ff/XOR/mask_1/mc_decrypted.png}
\caption{Дешифрованный файл в Midnight Commander, последняя попытка}
\end{figure}

(Расшифрованный файл доступен здесь:
\url{\GitHubBlobMasterURL/ff/XOR/mask_1/files/decrypted.dat.bz2}.)

Несомненно, это корректный исходный файл.
Да, и мы видим числа в начале каждого блока. Должно быть это и есть источник некорректного XOR-ключа.
Как выходит, самый встречающийся 81-байтный блок в файле это блок заполненный пробелами и содержащий символ \q{1} на месте
второго байта.
Действительно, как-то так получилось что многие блоки здесь перемежаются с этим блоком.
Может быть это что-то вроде выравнивания (padding) для коротких фраз/сообщений?
Другой часто встречающийся 81-байтный блок также заполнен пробелами, но с другой цифрой, следовательно,
они отличаются только вторым байтом.

Вот и всё! Теперь мы можем написать утилиту для зашифрования файла назад, и, может быть, модифицировать его перед этим

Файл для Mathematica можно скачать здесь:\\
\url{\GitHubBlobMasterURL/ff/XOR/mask_1/files/XOR_mask_1.nb}.

Итог: XOR-шифрование не надежно вообще. Вероятно, разработчик игры хотел просто скрыть внутренности игры от игрока,
ничего более серьезного.
Все же, шифрование вроде этого крайне популярно вследствии его простоты, так что многие реверс инженеры обычно хорошо
с этим знакомы.

}\FR{\mysection{Fonction presque vide}
\label{Boolector}
\myindex{Boolector}
\myindex{x86!\Instructions!JMP}

Ceci est un morceau de code réel que j'ai trouvé dans Boolector\footnote{\url{https://boolector.github.io/}}:

\lstinputlisting[style=customc]{patterns/025_almost_empty/boolectormain.c}

Pourquoi quelqu'un ferait-il comme ça?
Je ne sais pas mais mon hypothèse est que \verb|boolector_main()| peut être compilée
dans une sorte de DLL ou bibliothèque dynamique, et appelée depuis une suite de test.
Certainement qu'une suite de test peut préparer les variables argc/argv comme
le ferait \ac{CRT}.

Il est intéressant de voir comment c'est compilé:

\lstinputlisting[caption=GCC 8.2 x64 \NonOptimizing (\assemblyOutput),style=customasmx86]{patterns/025_almost_empty/boolectormain_O0.s}

Ceci est OK, le prologue (non optimisé) déplace inutilement deux arguments,
\INS{CALL}, épilogue, \INS{RET}.
Mais regardons la version optimisée:

\lstinputlisting[caption=GCC 8.2 x64 \Optimizing (\assemblyOutput),style=customasmx86]{patterns/025_almost_empty/boolectormain_O3.s}

Aussi simple que ça: la pile et les registres ne sont pas touchés et \verb|boolector_main()|
a le même ensemble d'arguments.
Donc, tout ce que nous avons à faire est de passer l'exécution à une autre adresse.

Ceci est proche d'une \glslink{thunk function}{fonction thunk}.

Nous verons queelque chose de plus avancé plus tard: \myref{ARM_B_to_printf}, \myref{JMP_instead_of_RET}.
}

\renewcommand{\CURPATH}{advanced/135_abs_branchless}
\EN{% TODO translate
\mysection{Breaking simple executable cryptor}

I've got an executable file which is encrypted by relatively simple encryption.
\href{\GitHubBlobMasterURL/examples/simple_exec_crypto/files/cipher.bin}{Here is it} (only executable section is left here).

First, all encryption function does is just adds number of position in buffer to the byte.
Here is how this can be encoded in Python:

\begin{lstlisting}[caption=Python script,style=custompy]
#!/usr/bin/env python
def e(i, k):
    return chr ((ord(i)+k) % 256)

def encrypt(buf):
    return e(buf[0], 0)+ e(buf[1], 1)+ e(buf[2], 2) + e(buf[3], 3)+ e(buf[4], 4)+ e(buf[5], 5)+ e(buf[6], 6)+ e(buf[7], 7)+
           e(buf[8], 8)+ e(buf[9], 9)+ e(buf[10], 10)+ e(buf[11], 11)+ e(buf[12], 12)+ e(buf[13], 13)+ e(buf[14], 14)+ e(buf[15], 15)
\end{lstlisting}

Hence, if you encrypt buffer with 16 zeros, you'll get \emph{0, 1, 2, 3 ... 12, 13, 14, 15}.

\myindex{Propagating Cipher Block Chaining}
Propagating Cipher Block Chaining (PCBC) is also used, here is how it works:

\begin{figure}[H]
\centering
\myincludegraphics{examples/simple_exec_crypto/601px-PCBC_encryption.png}
\caption{Propagating Cipher Block Chaining encryption (image is taken from Wikipedia article)}
\end{figure}

The problem is that it's too boring to recover IV (Initialization Vector) each time.
Brute-force is also not an option, because IV is too long (16 bytes).
Let's see, if it's possible to recover IV for arbitrary encrypted executable file?

Let's try simple frequency analysis.
This is 32-bit x86 executable code, so let's gather statistics about most frequent bytes and opcodes.
I tried huge oracle.exe file from Oracle RDBMS version 11.2 for windows x86 and I've found that the most frequent byte (no surprise) is zero (~10\%).
The next most frequent byte is (again, no surprise) 0xFF (~5\%).
The next is 0x8B (~5\%).

\myindex{x86!\Instructions!MOV}
0x8B is opcode for \INS{MOV}, this is indeed one of the most busy x86 instructions.
Now what about popularity of zero byte?
If compiler needs to encode value bigger than 127, it has to use 32-bit displacement instead of 8-bit one, but large values are very rare,
so it is padded by zeros.
\myindex{x86!\Instructions!LEA}
\myindex{x86!\Instructions!PUSH}
\myindex{x86!\Instructions!CALL}
This is at least in \INS{LEA}, \INS{MOV}, \INS{PUSH}, \INS{CALL}.

For example:

\begin{lstlisting}[style=customasmx86]
8D B0 28 01 00 00                 lea     esi, [eax+128h]
8D BF 40 38 00 00                 lea     edi, [edi+3840h]
\end{lstlisting}

Displacements bigger than 127 are very popular, but they are rarely exceeds 0x10000
(indeed, such large memory buffers/structures are also rare).

Same story with \INS{MOV}, large constants are rare, the most heavily used are 0, 1, 10, 100, $2^n$, and so on.
Compiler has to pad small constants by zeros to represent them as 32-bit values:

\begin{lstlisting}[style=customasmx86]
BF 02 00 00 00                    mov     edi, 2
BF 01 00 00 00                    mov     edi, 1
\end{lstlisting}

Now about 00 and FF bytes combined: jumps (including conditional) and calls can pass execution flow forward or backwards, but very often,
within the limits of the current executable module.
If forward, displacement is not very big and also padded with zeros.
If backwards, displacement is represented as negative value, so padded with FF bytes.
For example, transfer execution flow forward:

\begin{lstlisting}[style=customasmx86]
E8 43 0C 00 00                    call    _function1
E8 5C 00 00 00                    call    _function2
0F 84 F0 0A 00 00                 jz      loc_4F09A0
0F 84 EB 00 00 00                 jz      loc_4EFBB8
\end{lstlisting}

Backwards:

\begin{lstlisting}[style=customasmx86]
E8 79 0C FE FF                    call    _function1
E8 F4 16 FF FF                    call    _function2
0F 84 F8 FB FF FF                 jz      loc_8212BC
0F 84 06 FD FF FF                 jz      loc_FF1E7D
\end{lstlisting}

FF byte is also very often occurred in negative displacements like these:

\begin{lstlisting}[style=customasmx86]
8D 85 1E FF FF FF                 lea     eax, [ebp-0E2h]
8D 95 F8 5C FF FF                 lea     edx, [ebp-0A308h]
\end{lstlisting}

So far so good. Now we have to try various 16-byte keys, decrypt executable section and measure how often 00, FF and 8B bytes are occurred.
Let's also keep in sight how PCBC decryption works:

\begin{figure}[H]
\centering
\myincludegraphics{examples/simple_exec_crypto/640px-PCBC_decryption.png}
\caption{Propagating Cipher Block Chaining decryption (image is taken from Wikipedia article)}
\end{figure}

The good news is that we don't really have to decrypt whole piece of data, but only slice by slice, this is exactly how I did in my previous example: \myref{XOR_mask_2}.

Now I'm trying all possible bytes (0..255) for each byte in key and just pick the byte producing maximal amount of 00/FF/8B bytes in a decrypted slice:

\begin{lstlisting}[style=custompy]
#!/usr/bin/env python
import sys, hexdump, array, string, operator

KEY_LEN=16

def chunks(l, n):
    # split n by l-byte chunks
    # https://stackoverflow.com/q/312443
    n = max(1, n)
    return [l[i:i + n] for i in range(0, len(l), n)]

def read_file(fname):
    file=open(fname, mode='rb')
    content=file.read()
    file.close()
    return content

def decrypt_byte (c, key):
    return chr((ord(c)-key) % 256)

def XOR_PCBC_step (IV, buf, k):
    prev=IV
    rt=""
    for c in buf:
	new_c=decrypt_byte(c, k)
        plain=chr(ord(new_c)^ord(prev))
	prev=chr(ord(c)^ord(plain))
	rt=rt+plain
    return rt

each_Nth_byte=[""]*KEY_LEN

content=read_file(sys.argv[1])
# split input by 16-byte chunks:
all_chunks=chunks(content, KEY_LEN)
for c in all_chunks:
    for i in range(KEY_LEN):
        each_Nth_byte[i]=each_Nth_byte[i] + c[i]

# try each byte of key
for N in range(KEY_LEN):
    print "N=", N
    stat={}
    for i in range(256):
        tmp_key=chr(i)
	tmp=XOR_PCBC_step(tmp_key,each_Nth_byte[N], N)
        # count 0, FFs and 8Bs in decrypted buffer:
	important_bytes=tmp.count('\x00')+tmp.count('\xFF')+tmp.count('\x8B')
	stat[i]=important_bytes
    sorted_stat = sorted(stat.iteritems(), key=operator.itemgetter(1), reverse=True)
    print sorted_stat[0]
\end{lstlisting}

(Source code can be downloaded \href{\GitHubBlobMasterURL/examples/simple_exec_crypto/files/decrypt.py}{here}.)

I run it and here is a key for which 00/FF/8B bytes presence in decrypted buffer is maximal:

\begin{lstlisting}
N= 0
(147, 1224)
N= 1
(94, 1327)
N= 2
(252, 1223)
N= 3
(218, 1266)
N= 4
(38, 1209)
N= 5
(192, 1378)
N= 6
(199, 1204)
N= 7
(213, 1332)
N= 8
(225, 1251)
N= 9
(112, 1223)
N= 10
(143, 1177)
N= 11
(108, 1286)
N= 12
(10, 1164)
N= 13
(3, 1271)
N= 14
(128, 1253)
N= 15
(232, 1330)
\end{lstlisting}

Let's write decryption utility with the key we got:

\begin{lstlisting}[style=custompy]
#!/usr/bin/env python
import sys, hexdump, array

def xor_strings(s,t):
    # \verb|https://en.wikipedia.org/wiki/XOR_cipher#Example_implementation|
    """xor two strings together"""
    return "".join(chr(ord(a)^ord(b)) for a,b in zip(s,t))

IV=array.array('B', [147, 94, 252, 218, 38, 192, 199, 213, 225, 112, 143, 108, 10, 3, 128, 232]).tostring()

def chunks(l, n):
    n = max(1, n)
    return [l[i:i + n] for i in range(0, len(l), n)]

def read_file(fname):
    file=open(fname, mode='rb')
    content=file.read()
    file.close()
    return content

def decrypt_byte(i, k):
    return chr ((ord(i)-k) % 256)

def decrypt(buf):
    return "".join(decrypt_byte(buf[i], i) for i in range(16))

fout=open(sys.argv[2], mode='wb')

prev=IV
content=read_file(sys.argv[1])
tmp=chunks(content, 16)
for c in tmp:
    new_c=decrypt(c)
    p=xor_strings (new_c, prev)
    prev=xor_strings(c, p)
    fout.write(p)
fout.close()
\end{lstlisting}

(Source code can be downloaded \href{\GitHubBlobMasterURL/examples/simple_exec_crypto/files/decrypt2.py}{here}.)

Let's check resulting file:

\lstinputlisting{examples/simple_exec_crypto/objdump_result.txt}

Yes, this is seems correctly disassembled piece of x86 code.
The whole decryped file can be downloaded \href{\GitHubBlobMasterURL/examples/simple_exec_crypto/files/decrypted.bin}{here}.

In fact, this is text section from regedit.exe from Windows 7.
But this example is based on a real case I encountered, so just executable is different (and key), algorithm is the same.

\subsection{Other ideas to consider}

What if I would fail with such simple frequency analysis?
There are other ideas on how to measure correctness of decrypted/decompressed x86 code:

\begin{itemize}

\item Many modern compilers aligns functions on 0x10 border.
So the space left before is filled with NOPs (0x90) or other NOP instructions with known opcodes: \myref{sec:npad}.

\item Perhaps, the most frequent pattern in any assembly language is function call:\\
\TT{PUSH chain / CALL / ADD ESP, X}.
This sequence can easily detected and found.
I've even gathered statistics about average number of function arguments: \myref{args_stat}.
(Hence, this is average length of PUSH chain.)

\end{itemize}

Read more about incorrectly/correctly disassembled code: \myref{ISA_detect}.
}\RU{\subsection{Простое шифрование используя XOR-маску}
\label{XOR_mask_1}

Я нашел одну старую игру в стиле interactive fiction в архиве \emph{if-archive}\footnote{\url{http://www.ifarchive.org/}}:

\begin{lstlisting}
The New Castle v3.5 - Text/Adventure Game
in the style of the original Infocom (tm)
type games, Zork, Collosal Cave (Adventure),
etc.  Can you solve the mystery of the
abandoned castle?
Shareware from Software Customization.
Software Customization [ASP] Version 3.5 Feb. 2000
\end{lstlisting}

Можно скачать здесь: \url{\GitHubBlobMasterURL/ff/XOR/mask_1/files/newcastle.tgz}.

Там внутри есть файл (с названием \emph{castle.dbf}), который явно зашифрован, но не настоящим криптоалгоритмом,
и оне сжат, это что-то куда проще.
Я бы даже не стал измерять уровень энтропии (\myref{entropy}) этого файла, потому что я итак уверен, что он низкий.
Вот как он выглядит в Midnight Commander:

\begin{figure}[H]
\centering
\myincludegraphics{ff/XOR/mask_1/mc_encrypted.png}
\caption{Зашифрованный файл в Midnight Commander}
\end{figure}

Зашифрованный файл можно скачать здесь:
\url{\GitHubBlobMasterURL/ff/XOR/mask_1/files/castle.dbf.bz2}.

Можно ли расшифровать его без доступа к программе, используя просто этот файл?

Тут явно просматривается повторяющаяся строка. 
Если использовалось простое шифрование с XOR-маской, такие повторяющиеся строки это явное свидетельство,
потому что, вероятно, тут были длинные лакуны с нулевыми байтами, которые, в свою очередь, присутствуют
во мноигих исполняемых файлах, и в остальных бинарных файлах.

\myindex{UNIX!xxd}
Вот дам начала этого файла используя утилиту \emph{xxd} из UNIX:

\lstinputlisting{ff/XOR/mask_1/xxd_result.txt}

Давайте держаться за повторяющуюся строку \TT{iubgv}.
Глядя на этот дамп, мы можем легко увидеть, что период повторений этой строки это 0x51 или 81.
Вероятно, 81 это длина блока?
Длина файла 1658961, и она может быть поделена на 81 без остатка (и тогда там 20481 блоков).

Теперь я буду использовать Mathematica для анализа, есть ли тут повторяющиеся 81-байтные блоки в файле?
Я разделю входной файл на 81-байтные блоки и затем использую ф-цию
\emph{Tally[]}\footnote{\url{https://reference.wolfram.com/language/ref/Tally.html}}
которая просто считает, сколько раз каждый элемент встретился во входном списке.
Вывод Tally не отсортирован, так что я также добавлю ф-цию \emph{Sort[]} для сортировки его по кол-ву вхождений
в нисходящем порядке.

\begin{lstlisting}[style=custommath]
input = BinaryReadList["/home/dennis/.../castle.dbf"];

blocks = Partition[input, 81];

stat = Sort[Tally[blocks], #1[[2]] > #2[[2]] &]
\end{lstlisting}

И вот вывод:

\begin{lstlisting}[style=custommath]
{{{80, 103, 2, 116, 113, 102, 118, 25, 99, 8, 19, 23, 116, 125, 107, 
   25, 99, 109, 114, 102, 14, 121, 115, 31, 9, 117, 113, 111, 5, 4, 
   127, 28, 122, 101, 8, 110, 14, 18, 124, 106, 16, 20, 104, 119, 8, 
   109, 26, 106, 9, 97, 13, 99, 15, 119, 20, 105, 117, 98, 103, 118, 
   1, 126, 29, 97, 122, 17, 15, 114, 110, 3, 5, 125, 125, 99, 126, 
   119, 102, 30, 122, 2, 117}, 1739}, 
{{80, 100, 2, 116, 113, 102, 118, 25, 99, 8, 19, 23, 116, 
   125, 107, 25, 99, 109, 114, 102, 14, 121, 115, 31, 9, 117, 113, 
   111, 5, 4, 127, 28, 122, 101, 8, 110, 14, 18, 124, 106, 16, 20, 
   104, 119, 8, 109, 26, 106, 9, 97, 13, 99, 15, 119, 20, 105, 117, 
   98, 103, 118, 1, 126, 29, 97, 122, 17, 15, 114, 110, 3, 5, 125, 
   125, 99, 126, 119, 102, 30, 122, 2, 117}, 1422}, 
{{80, 101, 2, 116, 113, 102, 118, 25, 99, 8, 19, 23, 116, 
   125, 107, 25, 99, 109, 114, 102, 14, 121, 115, 31, 9, 117, 113, 
   111, 5, 4, 127, 28, 122, 101, 8, 110, 14, 18, 124, 106, 16, 20, 
   104, 119, 8, 109, 26, 106, 9, 97, 13, 99, 15, 119, 20, 105, 117, 
   98, 103, 118, 1, 126, 29, 97, 122, 17, 15, 114, 110, 3, 5, 125, 
   125, 99, 126, 119, 102, 30, 122, 2, 117}, 1012},
{{80, 120, 2, 116, 113, 102, 118, 25, 99, 8, 19, 23, 116, 
   125, 107, 25, 99, 109, 114, 102, 14, 121, 115, 31, 9, 117, 113, 
   111, 5, 4, 127, 28, 122, 101, 8, 110, 14, 18, 124, 106, 16, 20, 
   104, 119, 8, 109, 26, 106, 9, 97, 13, 99, 15, 119, 20, 105, 117, 
   98, 103, 118, 1, 126, 29, 97, 122, 17, 15, 114, 110, 3, 5, 125, 
   125, 99, 126, 119, 102, 30, 122, 2, 117}, 377},

...

{{80, 2, 74, 49, 113, 21, 62, 88, 39, 71, 68, 23, 63, 51, 36, 78, 48, 
   108, 114, 102, 14, 121, 115, 31, 9, 117, 113, 111, 5, 4, 127, 28, 
   122, 101, 8, 110, 14, 18, 124, 106, 16, 20, 104, 119, 8, 109, 26, 
   106, 9, 97, 13, 99, 15, 119, 20, 105, 117, 98, 103, 118, 1, 126, 
   29, 97, 122, 17, 15, 114, 110, 3, 5, 125, 125, 99, 126, 119, 102, 
   30, 122, 2, 117}, 1},
{{80, 1, 74, 59, 113, 45, 56, 86, 52, 91, 19, 64, 60, 60, 63, 
   25, 38, 59, 59, 42, 14, 53, 38, 77, 66, 38, 113, 38, 75, 4, 43, 84,
    63, 101, 64, 43, 79, 64, 40, 57, 16, 91, 46, 119, 69, 40, 84, 117,
    9, 97, 13, 99, 15, 119, 20, 105, 117, 98, 103, 118, 1, 126, 29, 
   97, 122, 17, 15, 114, 110, 3, 5, 125, 125, 99, 126, 119, 102, 30, 
   122, 2, 117}, 1},
{{80, 2, 74, 49, 113, 49, 51, 92, 39, 8, 92, 81, 116, 62, 57, 
   80, 46, 40, 114, 36, 75, 56, 33, 76, 9, 55, 56, 59, 81, 65, 45, 28,
    60, 55, 93, 39, 90, 28, 124, 106, 16, 20, 104, 119, 8, 109, 26, 
   106, 9, 97, 13, 99, 15, 119, 20, 105, 117, 98, 103, 118, 1, 126, 
   29, 97, 122, 17, 15, 114, 110, 3, 5, 125, 125, 99, 126, 119, 102, 
   30, 122, 2, 117}, 1}}
\end{lstlisting}

Вывод Tally это список пар, каждая пара это 81-байтный блок и количество раз, сколько он встретился в файле.
Мы видим, что наиболее частно встречающийся блок это первый, он встретился 1739 раз.
Второй встретился 1422 раза. Есть и другие: 1012 раза, 377 раз, итд.
81-байтные блоки, встреченные лишь один раз, находятся в конце вывода.

Попробуем сравнить эти блоки. Первый и второй.
Есть ли в Mathematica ф-ция для сравнения списков/массивов?
Наверняка есть, но в педагогических целях, я буду использоват операцию XOR для сравнения.
Действительно: если байты во входных массивах равны друг другу, результат операции XOR это 0.
Если не равны, результат будет ненулевой.

Сравним первый блок (встречается 1739 раз) и второй (встречается 1422 раз):

\begin{lstlisting}[style=custommath]
In[]:= BitXor[stat[[1]][[1]], stat[[2]][[1]]]
Out[]= {0, 3, 0, 0, 0, 0, 0, 0, 0, 0, 0, 0, 0, 0, 0, 0, 0, 0, 0, \
0, 0, 0, 0, 0, 0, 0, 0, 0, 0, 0, 0, 0, 0, 0, 0, 0, 0, 0, 0, 0, 0, 0, \
0, 0, 0, 0, 0, 0, 0, 0, 0, 0, 0, 0, 0, 0, 0, 0, 0, 0, 0, 0, 0, 0, 0, \
0, 0, 0, 0, 0, 0, 0, 0, 0, 0, 0, 0, 0, 0, 0, 0}
\end{lstlisting}

Они отличаются только вторым байтом.

Сравним второй блок (встречается 1422 раза) и третий (встречается 1012 раз):

\begin{lstlisting}[style=custommath]
In[]:= BitXor[stat[[2]][[1]], stat[[3]][[1]]]
Out[]= {0, 1, 0, 0, 0, 0, 0, 0, 0, 0, 0, 0, 0, 0, 0, 0, 0, 0, 0, \
0, 0, 0, 0, 0, 0, 0, 0, 0, 0, 0, 0, 0, 0, 0, 0, 0, 0, 0, 0, 0, 0, 0, \
0, 0, 0, 0, 0, 0, 0, 0, 0, 0, 0, 0, 0, 0, 0, 0, 0, 0, 0, 0, 0, 0, 0, \
0, 0, 0, 0, 0, 0, 0, 0, 0, 0, 0, 0, 0, 0, 0, 0}
\end{lstlisting}

Они тоже отличаются только вторым байтом.

Так или иначе, попробуем использовать самый встречающийся блок как XOR-ключ и попробуем расшифровать первые 4 81-байтных
блока в файле:

\begin{lstlisting}[style=custommath]
In[]:= key = stat[[1]][[1]]
Out[]= {80, 103, 2, 116, 113, 102, 118, 25, 99, 8, 19, 23, 116, \
125, 107, 25, 99, 109, 114, 102, 14, 121, 115, 31, 9, 117, 113, 111, \
5, 4, 127, 28, 122, 101, 8, 110, 14, 18, 124, 106, 16, 20, 104, 119, \
8, 109, 26, 106, 9, 97, 13, 99, 15, 119, 20, 105, 117, 98, 103, 118, \
1, 126, 29, 97, 122, 17, 15, 114, 110, 3, 5, 125, 125, 99, 126, 119, \
102, 30, 122, 2, 117}

In[]:= ToASCII[val_] := If[val == 0, " ", FromCharacterCode[val, "PrintableASCII"]]

In[]:= DecryptBlockASCII[blk_] := Map[ToASCII[#] &, BitXor[key, blk]]

In[]:= DecryptBlockASCII[blocks[[1]]]
Out[]= {" ", " ", " ", " ", " ", " ", " ", " ", " ", " ", " ", " \
", " ", " ", " ", " ", " ", " ", " ", " ", " ", " ", " ", " ", " ", " \
", " ", " ", " ", " ", " ", " ", " ", " ", " ", " ", " ", " ", " ", " \
", " ", " ", " ", " ", " ", " ", " ", " ", " ", " ", " ", " ", " ", " \
", " ", " ", " ", " ", " ", " ", " ", " ", " ", " ", " ", " ", " ", " \
", " ", " ", " ", " ", " ", " ", " ", " ", " ", " ", " ", " ", " "}

In[]:= DecryptBlockASCII[blocks[[2]]]
Out[]= {" ", "e", "H", "E", " ", "W", "E", "E", "D", " ", "O", \
"F", " ", "C", "R", "I", "M", "E", " ", "B", "E", "A", "R", "S", " ", \
"B", "I", "T", "T", "E", "R", " ", "F", "R", "U", "I", "T", "?", \
" ", " ", " ", " ", " ", " ", " ", " ", " ", " ", " ", " ", " ", " ", \
" ", " ", " ", " ", " ", " ", " ", " ", " ", " ", " ", " ", " ", " ", \
" ", " ", " ", " ", " ", " ", " ", " ", " ", " ", " ", " ", " ", " ", \
" "}

In[]:= DecryptBlockASCII[blocks[[3]]]
Out[]= {" ", "?", " ", " ", " ", " ", " ", " ", " ", " ", " \
", " ", " ", " ", " ", " ", " ", " ", " ", " ", " ", " ", " ", " ", " \
", " ", " ", " ", " ", " ", " ", " ", " ", " ", " ", " ", " ", " ", " \
", " ", " ", " ", " ", " ", " ", " ", " ", " ", " ", " ", " ", " ", " \
", " ", " ", " ", " ", " ", " ", " ", " ", " ", " ", " ", " ", " ", " \
", " ", " ", " ", " ", " ", " ", " ", " ", " ", " ", " ", " ", " ", " \
"}

In[]:= DecryptBlockASCII[blocks[[4]]]
Out[]= {" ", "f", "H", "O", " ", "K", "N", "O", "W", "S", " ", \
"W", "H", "A", "T", " ", "E", "V", "I", "L", " ", "L", "U", "R", "K", \
"S", " ", "I", "N", " ", "T", "H", "E", " ", "H", "E", "A", "R", "T", \
"S", " ", "O", "F", " ", "M", "E", "N", "?", " ", " ", " ", " ", \
" ", " ", " ", " ", " ", " ", " ", " ", " ", " ", " ", " ", " ", " ", \
" ", " ", " ", " ", " ", " ", " ", " ", " ", " ", " ", " ", " ", " ", \
" "}
\end{lstlisting}

(Я заменил непечатаемые символы на \q{?}.)

Мы видим что первый и третий блоки пустые (или почти пустые),
но второй и четвертый имеют ясно различимые английские слова/фразы.
Похоже что наше предположение насчет ключа верно (как минимум частично).
Это означает, что самый встречающийся 81-байтный блок в файле находится в местах лакун с нулевыми байтами
или что-то в этом роде.

Попробуем расшифровать весь файл:

\begin{lstlisting}[style=custommath]
DecryptBlock[blk_] := BitXor[key, blk]

decrypted = Map[DecryptBlock[#] &, blocks];

BinaryWrite["/home/dennis/.../tmp", Flatten[decrypted]]

Close["/home/dennis/.../tmp"]
\end{lstlisting}

\begin{figure}[H]
\centering
\myincludegraphics{ff/XOR/mask_1/mc_decrypted1.png}
\caption{Расшифрованный файл в Midnight Commander, первая попытка}
\end{figure}

Выглядит как английские фразы для какой-то игры, но что-то не так.
Прежде всего, регистр инвертирован: фразы и некоторые слова начинаются со строчных букв,
в то время как остальные буквы заглавные.
Также, некоторые фразы начинаются с не тех букв.
Посмотрите на самую первую фразу: \q{eHE WEED OF CRIME BEARS BITTER FRUIT}.
Что такое \q{eHE}? Разве не \q{tHE} тут должно быть?
Возможно ли что наш ключ для дешифрования имеет неверный байт в этом месте?

Посмотрим снова на второй блок в файле, на ключ и на результат дешифрования:

\begin{lstlisting}[style=custommath]
In[]:= blocks[[2]]
Out[]= {80, 2, 74, 49, 113, 49, 51, 92, 39, 8, 92, 81, 116, 62, \
57, 80, 46, 40, 114, 36, 75, 56, 33, 76, 9, 55, 56, 59, 81, 65, 45, \
28, 60, 55, 93, 39, 90, 28, 124, 106, 16, 20, 104, 119, 8, 109, 26, \
106, 9, 97, 13, 99, 15, 119, 20, 105, 117, 98, 103, 118, 1, 126, 29, \
97, 122, 17, 15, 114, 110, 3, 5, 125, 125, 99, 126, 119, 102, 30, \
122, 2, 117}

In[]:= key
Out[]= {80, 103, 2, 116, 113, 102, 118, 25, 99, 8, 19, 23, 116, \
125, 107, 25, 99, 109, 114, 102, 14, 121, 115, 31, 9, 117, 113, 111, \
5, 4, 127, 28, 122, 101, 8, 110, 14, 18, 124, 106, 16, 20, 104, 119, \
8, 109, 26, 106, 9, 97, 13, 99, 15, 119, 20, 105, 117, 98, 103, 118, \
1, 126, 29, 97, 122, 17, 15, 114, 110, 3, 5, 125, 125, 99, 126, 119, \
102, 30, 122, 2, 117}

In[]:= BitXor[key, blocks[[2]]]
Out[]= {0, 101, 72, 69, 0, 87, 69, 69, 68, 0, 79, 70, 0, 67, 82, \
73, 77, 69, 0, 66, 69, 65, 82, 83, 0, 66, 73, 84, 84, 69, 82, 0, 70, \
82, 85, 73, 84, 14, 0, 0, 0, 0, 0, 0, 0, 0, 0, 0, 0, 0, 0, 0, 0, 0, \
0, 0, 0, 0, 0, 0, 0, 0, 0, 0, 0, 0, 0, 0, 0, 0, 0, 0, 0, 0, 0, 0, 0, \
0, 0, 0, 0}
\end{lstlisting}

Зашифрованный байт это 2, байт из ключа это 103, $2 \oplus 103=101$ и 101 это ASCII-код символа \q{e}.
Чему должен равнятся этот байт ключа, чтобы ASCII-код был 116 (для символа  \q{t})?
$2 \oplus 116=118$, присвоим 118 второму байту в ключе \dots

\begin{lstlisting}[style=custommath]
key = {80, 118, 2, 116, 113, 102, 118, 25, 99, 8, 19, 23, 116, 125, 
  107, 25, 99, 109, 114, 102, 14, 121, 115, 31, 9, 117, 113, 111, 5, 
  4, 127, 28, 122, 101, 8, 110, 14, 18, 124, 106, 16, 20, 104, 119, 8,
   109, 26, 106, 9, 97, 13, 99, 15, 119, 20, 105, 117, 98, 103, 118, 
  1, 126, 29, 97, 122, 17, 15, 114, 110, 3, 5, 125, 125, 99, 126, 119,
   102, 30, 122, 2, 117}
\end{lstlisting}

\dots и снова дешифруем весь файл.

\begin{figure}[H]
\centering
\myincludegraphics{ff/XOR/mask_1/mc_decrypted2.png}
\caption{Дешифрованный файл в Midnight Commander, вторая попытка}
\end{figure}

Ух ты, теперь грамматика корректна, и все фразы начинаются с корректных букв.
Но все таки, регистр подозрителен.
С чего бы разработчику игры записывать их в такой манере?
Может быть наш ключ все еще неправилен?

% TODO ASCII table somewhere in the book
Изучая таблицу ASCII мы можем заметить что ASCII-коды для букв в верхнем и нижнем регистре отличаются только на один бит
(6-й бит, если считать с первого, 0b100000):

\begin{figure}[H]
\centering
\includegraphics[width=0.7\textwidth]{ascii.png}
\caption{7-битная таблица \ac{ASCII} в Emacs}
\end{figure}

6-й бит, выставленный в нулевом байте, В десятичном виде это будет 32.
Но 32 это ASCII-код пробела!

Действительно, можно менять регистр просто применяя XOR к ASCII-коду, с 32 (больше об этом: \myref{toupper_bit}).

Возможно ли, что пустые лакуны в файле это не нулевые байты, а скорее содержащие пробелы?
Еще раз модифицируем наш XOR-ключ (я про-XOR-ю каждый байт ключа с 32):

\begin{lstlisting}[style=custommath]
(* "32" это скаляр, и "key" это вектор, но это OK *)

In[]:= key3 = BitXor[32, key]
Out[]= {112, 86, 34, 84, 81, 70, 86, 57, 67, 40, 51, 55, 84, 93, 75, \
57, 67, 77, 82, 70, 46, 89, 83, 63, 41, 85, 81, 79, 37, 36, 95, 60, \
90, 69, 40, 78, 46, 50, 92, 74, 48, 52, 72, 87, 40, 77, 58, 74, 41, \
65, 45, 67, 47, 87, 52, 73, 85, 66, 71, 86, 33, 94, 61, 65, 90, 49, \
47, 82, 78, 35, 37, 93, 93, 67, 94, 87, 70, 62, 90, 34, 85}

In[]:= DecryptBlock[blk_] := BitXor[key3, blk]
\end{lstlisting}

И снова дешифруем входной файл:

\begin{figure}[H]
\centering
\myincludegraphics{ff/XOR/mask_1/mc_decrypted.png}
\caption{Дешифрованный файл в Midnight Commander, последняя попытка}
\end{figure}

(Расшифрованный файл доступен здесь:
\url{\GitHubBlobMasterURL/ff/XOR/mask_1/files/decrypted.dat.bz2}.)

Несомненно, это корректный исходный файл.
Да, и мы видим числа в начале каждого блока. Должно быть это и есть источник некорректного XOR-ключа.
Как выходит, самый встречающийся 81-байтный блок в файле это блок заполненный пробелами и содержащий символ \q{1} на месте
второго байта.
Действительно, как-то так получилось что многие блоки здесь перемежаются с этим блоком.
Может быть это что-то вроде выравнивания (padding) для коротких фраз/сообщений?
Другой часто встречающийся 81-байтный блок также заполнен пробелами, но с другой цифрой, следовательно,
они отличаются только вторым байтом.

Вот и всё! Теперь мы можем написать утилиту для зашифрования файла назад, и, может быть, модифицировать его перед этим

Файл для Mathematica можно скачать здесь:\\
\url{\GitHubBlobMasterURL/ff/XOR/mask_1/files/XOR_mask_1.nb}.

Итог: XOR-шифрование не надежно вообще. Вероятно, разработчик игры хотел просто скрыть внутренности игры от игрока,
ничего более серьезного.
Все же, шифрование вроде этого крайне популярно вследствии его простоты, так что многие реверс инженеры обычно хорошо
с этим знакомы.

}\FR{\mysection{Fonction presque vide}
\label{Boolector}
\myindex{Boolector}
\myindex{x86!\Instructions!JMP}

Ceci est un morceau de code réel que j'ai trouvé dans Boolector\footnote{\url{https://boolector.github.io/}}:

\lstinputlisting[style=customc]{patterns/025_almost_empty/boolectormain.c}

Pourquoi quelqu'un ferait-il comme ça?
Je ne sais pas mais mon hypothèse est que \verb|boolector_main()| peut être compilée
dans une sorte de DLL ou bibliothèque dynamique, et appelée depuis une suite de test.
Certainement qu'une suite de test peut préparer les variables argc/argv comme
le ferait \ac{CRT}.

Il est intéressant de voir comment c'est compilé:

\lstinputlisting[caption=GCC 8.2 x64 \NonOptimizing (\assemblyOutput),style=customasmx86]{patterns/025_almost_empty/boolectormain_O0.s}

Ceci est OK, le prologue (non optimisé) déplace inutilement deux arguments,
\INS{CALL}, épilogue, \INS{RET}.
Mais regardons la version optimisée:

\lstinputlisting[caption=GCC 8.2 x64 \Optimizing (\assemblyOutput),style=customasmx86]{patterns/025_almost_empty/boolectormain_O3.s}

Aussi simple que ça: la pile et les registres ne sont pas touchés et \verb|boolector_main()|
a le même ensemble d'arguments.
Donc, tout ce que nous avons à faire est de passer l'exécution à une autre adresse.

Ceci est proche d'une \glslink{thunk function}{fonction thunk}.

Nous verons queelque chose de plus avancé plus tard: \myref{ARM_B_to_printf}, \myref{JMP_instead_of_RET}.
}

\renewcommand{\CURPATH}{advanced/170_variadic_functions}
\EN{% TODO translate
\mysection{Breaking simple executable cryptor}

I've got an executable file which is encrypted by relatively simple encryption.
\href{\GitHubBlobMasterURL/examples/simple_exec_crypto/files/cipher.bin}{Here is it} (only executable section is left here).

First, all encryption function does is just adds number of position in buffer to the byte.
Here is how this can be encoded in Python:

\begin{lstlisting}[caption=Python script,style=custompy]
#!/usr/bin/env python
def e(i, k):
    return chr ((ord(i)+k) % 256)

def encrypt(buf):
    return e(buf[0], 0)+ e(buf[1], 1)+ e(buf[2], 2) + e(buf[3], 3)+ e(buf[4], 4)+ e(buf[5], 5)+ e(buf[6], 6)+ e(buf[7], 7)+
           e(buf[8], 8)+ e(buf[9], 9)+ e(buf[10], 10)+ e(buf[11], 11)+ e(buf[12], 12)+ e(buf[13], 13)+ e(buf[14], 14)+ e(buf[15], 15)
\end{lstlisting}

Hence, if you encrypt buffer with 16 zeros, you'll get \emph{0, 1, 2, 3 ... 12, 13, 14, 15}.

\myindex{Propagating Cipher Block Chaining}
Propagating Cipher Block Chaining (PCBC) is also used, here is how it works:

\begin{figure}[H]
\centering
\myincludegraphics{examples/simple_exec_crypto/601px-PCBC_encryption.png}
\caption{Propagating Cipher Block Chaining encryption (image is taken from Wikipedia article)}
\end{figure}

The problem is that it's too boring to recover IV (Initialization Vector) each time.
Brute-force is also not an option, because IV is too long (16 bytes).
Let's see, if it's possible to recover IV for arbitrary encrypted executable file?

Let's try simple frequency analysis.
This is 32-bit x86 executable code, so let's gather statistics about most frequent bytes and opcodes.
I tried huge oracle.exe file from Oracle RDBMS version 11.2 for windows x86 and I've found that the most frequent byte (no surprise) is zero (~10\%).
The next most frequent byte is (again, no surprise) 0xFF (~5\%).
The next is 0x8B (~5\%).

\myindex{x86!\Instructions!MOV}
0x8B is opcode for \INS{MOV}, this is indeed one of the most busy x86 instructions.
Now what about popularity of zero byte?
If compiler needs to encode value bigger than 127, it has to use 32-bit displacement instead of 8-bit one, but large values are very rare,
so it is padded by zeros.
\myindex{x86!\Instructions!LEA}
\myindex{x86!\Instructions!PUSH}
\myindex{x86!\Instructions!CALL}
This is at least in \INS{LEA}, \INS{MOV}, \INS{PUSH}, \INS{CALL}.

For example:

\begin{lstlisting}[style=customasmx86]
8D B0 28 01 00 00                 lea     esi, [eax+128h]
8D BF 40 38 00 00                 lea     edi, [edi+3840h]
\end{lstlisting}

Displacements bigger than 127 are very popular, but they are rarely exceeds 0x10000
(indeed, such large memory buffers/structures are also rare).

Same story with \INS{MOV}, large constants are rare, the most heavily used are 0, 1, 10, 100, $2^n$, and so on.
Compiler has to pad small constants by zeros to represent them as 32-bit values:

\begin{lstlisting}[style=customasmx86]
BF 02 00 00 00                    mov     edi, 2
BF 01 00 00 00                    mov     edi, 1
\end{lstlisting}

Now about 00 and FF bytes combined: jumps (including conditional) and calls can pass execution flow forward or backwards, but very often,
within the limits of the current executable module.
If forward, displacement is not very big and also padded with zeros.
If backwards, displacement is represented as negative value, so padded with FF bytes.
For example, transfer execution flow forward:

\begin{lstlisting}[style=customasmx86]
E8 43 0C 00 00                    call    _function1
E8 5C 00 00 00                    call    _function2
0F 84 F0 0A 00 00                 jz      loc_4F09A0
0F 84 EB 00 00 00                 jz      loc_4EFBB8
\end{lstlisting}

Backwards:

\begin{lstlisting}[style=customasmx86]
E8 79 0C FE FF                    call    _function1
E8 F4 16 FF FF                    call    _function2
0F 84 F8 FB FF FF                 jz      loc_8212BC
0F 84 06 FD FF FF                 jz      loc_FF1E7D
\end{lstlisting}

FF byte is also very often occurred in negative displacements like these:

\begin{lstlisting}[style=customasmx86]
8D 85 1E FF FF FF                 lea     eax, [ebp-0E2h]
8D 95 F8 5C FF FF                 lea     edx, [ebp-0A308h]
\end{lstlisting}

So far so good. Now we have to try various 16-byte keys, decrypt executable section and measure how often 00, FF and 8B bytes are occurred.
Let's also keep in sight how PCBC decryption works:

\begin{figure}[H]
\centering
\myincludegraphics{examples/simple_exec_crypto/640px-PCBC_decryption.png}
\caption{Propagating Cipher Block Chaining decryption (image is taken from Wikipedia article)}
\end{figure}

The good news is that we don't really have to decrypt whole piece of data, but only slice by slice, this is exactly how I did in my previous example: \myref{XOR_mask_2}.

Now I'm trying all possible bytes (0..255) for each byte in key and just pick the byte producing maximal amount of 00/FF/8B bytes in a decrypted slice:

\begin{lstlisting}[style=custompy]
#!/usr/bin/env python
import sys, hexdump, array, string, operator

KEY_LEN=16

def chunks(l, n):
    # split n by l-byte chunks
    # https://stackoverflow.com/q/312443
    n = max(1, n)
    return [l[i:i + n] for i in range(0, len(l), n)]

def read_file(fname):
    file=open(fname, mode='rb')
    content=file.read()
    file.close()
    return content

def decrypt_byte (c, key):
    return chr((ord(c)-key) % 256)

def XOR_PCBC_step (IV, buf, k):
    prev=IV
    rt=""
    for c in buf:
	new_c=decrypt_byte(c, k)
        plain=chr(ord(new_c)^ord(prev))
	prev=chr(ord(c)^ord(plain))
	rt=rt+plain
    return rt

each_Nth_byte=[""]*KEY_LEN

content=read_file(sys.argv[1])
# split input by 16-byte chunks:
all_chunks=chunks(content, KEY_LEN)
for c in all_chunks:
    for i in range(KEY_LEN):
        each_Nth_byte[i]=each_Nth_byte[i] + c[i]

# try each byte of key
for N in range(KEY_LEN):
    print "N=", N
    stat={}
    for i in range(256):
        tmp_key=chr(i)
	tmp=XOR_PCBC_step(tmp_key,each_Nth_byte[N], N)
        # count 0, FFs and 8Bs in decrypted buffer:
	important_bytes=tmp.count('\x00')+tmp.count('\xFF')+tmp.count('\x8B')
	stat[i]=important_bytes
    sorted_stat = sorted(stat.iteritems(), key=operator.itemgetter(1), reverse=True)
    print sorted_stat[0]
\end{lstlisting}

(Source code can be downloaded \href{\GitHubBlobMasterURL/examples/simple_exec_crypto/files/decrypt.py}{here}.)

I run it and here is a key for which 00/FF/8B bytes presence in decrypted buffer is maximal:

\begin{lstlisting}
N= 0
(147, 1224)
N= 1
(94, 1327)
N= 2
(252, 1223)
N= 3
(218, 1266)
N= 4
(38, 1209)
N= 5
(192, 1378)
N= 6
(199, 1204)
N= 7
(213, 1332)
N= 8
(225, 1251)
N= 9
(112, 1223)
N= 10
(143, 1177)
N= 11
(108, 1286)
N= 12
(10, 1164)
N= 13
(3, 1271)
N= 14
(128, 1253)
N= 15
(232, 1330)
\end{lstlisting}

Let's write decryption utility with the key we got:

\begin{lstlisting}[style=custompy]
#!/usr/bin/env python
import sys, hexdump, array

def xor_strings(s,t):
    # \verb|https://en.wikipedia.org/wiki/XOR_cipher#Example_implementation|
    """xor two strings together"""
    return "".join(chr(ord(a)^ord(b)) for a,b in zip(s,t))

IV=array.array('B', [147, 94, 252, 218, 38, 192, 199, 213, 225, 112, 143, 108, 10, 3, 128, 232]).tostring()

def chunks(l, n):
    n = max(1, n)
    return [l[i:i + n] for i in range(0, len(l), n)]

def read_file(fname):
    file=open(fname, mode='rb')
    content=file.read()
    file.close()
    return content

def decrypt_byte(i, k):
    return chr ((ord(i)-k) % 256)

def decrypt(buf):
    return "".join(decrypt_byte(buf[i], i) for i in range(16))

fout=open(sys.argv[2], mode='wb')

prev=IV
content=read_file(sys.argv[1])
tmp=chunks(content, 16)
for c in tmp:
    new_c=decrypt(c)
    p=xor_strings (new_c, prev)
    prev=xor_strings(c, p)
    fout.write(p)
fout.close()
\end{lstlisting}

(Source code can be downloaded \href{\GitHubBlobMasterURL/examples/simple_exec_crypto/files/decrypt2.py}{here}.)

Let's check resulting file:

\lstinputlisting{examples/simple_exec_crypto/objdump_result.txt}

Yes, this is seems correctly disassembled piece of x86 code.
The whole decryped file can be downloaded \href{\GitHubBlobMasterURL/examples/simple_exec_crypto/files/decrypted.bin}{here}.

In fact, this is text section from regedit.exe from Windows 7.
But this example is based on a real case I encountered, so just executable is different (and key), algorithm is the same.

\subsection{Other ideas to consider}

What if I would fail with such simple frequency analysis?
There are other ideas on how to measure correctness of decrypted/decompressed x86 code:

\begin{itemize}

\item Many modern compilers aligns functions on 0x10 border.
So the space left before is filled with NOPs (0x90) or other NOP instructions with known opcodes: \myref{sec:npad}.

\item Perhaps, the most frequent pattern in any assembly language is function call:\\
\TT{PUSH chain / CALL / ADD ESP, X}.
This sequence can easily detected and found.
I've even gathered statistics about average number of function arguments: \myref{args_stat}.
(Hence, this is average length of PUSH chain.)

\end{itemize}

Read more about incorrectly/correctly disassembled code: \myref{ISA_detect}.
}\RU{\subsection{Простое шифрование используя XOR-маску}
\label{XOR_mask_1}

Я нашел одну старую игру в стиле interactive fiction в архиве \emph{if-archive}\footnote{\url{http://www.ifarchive.org/}}:

\begin{lstlisting}
The New Castle v3.5 - Text/Adventure Game
in the style of the original Infocom (tm)
type games, Zork, Collosal Cave (Adventure),
etc.  Can you solve the mystery of the
abandoned castle?
Shareware from Software Customization.
Software Customization [ASP] Version 3.5 Feb. 2000
\end{lstlisting}

Можно скачать здесь: \url{\GitHubBlobMasterURL/ff/XOR/mask_1/files/newcastle.tgz}.

Там внутри есть файл (с названием \emph{castle.dbf}), который явно зашифрован, но не настоящим криптоалгоритмом,
и оне сжат, это что-то куда проще.
Я бы даже не стал измерять уровень энтропии (\myref{entropy}) этого файла, потому что я итак уверен, что он низкий.
Вот как он выглядит в Midnight Commander:

\begin{figure}[H]
\centering
\myincludegraphics{ff/XOR/mask_1/mc_encrypted.png}
\caption{Зашифрованный файл в Midnight Commander}
\end{figure}

Зашифрованный файл можно скачать здесь:
\url{\GitHubBlobMasterURL/ff/XOR/mask_1/files/castle.dbf.bz2}.

Можно ли расшифровать его без доступа к программе, используя просто этот файл?

Тут явно просматривается повторяющаяся строка. 
Если использовалось простое шифрование с XOR-маской, такие повторяющиеся строки это явное свидетельство,
потому что, вероятно, тут были длинные лакуны с нулевыми байтами, которые, в свою очередь, присутствуют
во мноигих исполняемых файлах, и в остальных бинарных файлах.

\myindex{UNIX!xxd}
Вот дам начала этого файла используя утилиту \emph{xxd} из UNIX:

\lstinputlisting{ff/XOR/mask_1/xxd_result.txt}

Давайте держаться за повторяющуюся строку \TT{iubgv}.
Глядя на этот дамп, мы можем легко увидеть, что период повторений этой строки это 0x51 или 81.
Вероятно, 81 это длина блока?
Длина файла 1658961, и она может быть поделена на 81 без остатка (и тогда там 20481 блоков).

Теперь я буду использовать Mathematica для анализа, есть ли тут повторяющиеся 81-байтные блоки в файле?
Я разделю входной файл на 81-байтные блоки и затем использую ф-цию
\emph{Tally[]}\footnote{\url{https://reference.wolfram.com/language/ref/Tally.html}}
которая просто считает, сколько раз каждый элемент встретился во входном списке.
Вывод Tally не отсортирован, так что я также добавлю ф-цию \emph{Sort[]} для сортировки его по кол-ву вхождений
в нисходящем порядке.

\begin{lstlisting}[style=custommath]
input = BinaryReadList["/home/dennis/.../castle.dbf"];

blocks = Partition[input, 81];

stat = Sort[Tally[blocks], #1[[2]] > #2[[2]] &]
\end{lstlisting}

И вот вывод:

\begin{lstlisting}[style=custommath]
{{{80, 103, 2, 116, 113, 102, 118, 25, 99, 8, 19, 23, 116, 125, 107, 
   25, 99, 109, 114, 102, 14, 121, 115, 31, 9, 117, 113, 111, 5, 4, 
   127, 28, 122, 101, 8, 110, 14, 18, 124, 106, 16, 20, 104, 119, 8, 
   109, 26, 106, 9, 97, 13, 99, 15, 119, 20, 105, 117, 98, 103, 118, 
   1, 126, 29, 97, 122, 17, 15, 114, 110, 3, 5, 125, 125, 99, 126, 
   119, 102, 30, 122, 2, 117}, 1739}, 
{{80, 100, 2, 116, 113, 102, 118, 25, 99, 8, 19, 23, 116, 
   125, 107, 25, 99, 109, 114, 102, 14, 121, 115, 31, 9, 117, 113, 
   111, 5, 4, 127, 28, 122, 101, 8, 110, 14, 18, 124, 106, 16, 20, 
   104, 119, 8, 109, 26, 106, 9, 97, 13, 99, 15, 119, 20, 105, 117, 
   98, 103, 118, 1, 126, 29, 97, 122, 17, 15, 114, 110, 3, 5, 125, 
   125, 99, 126, 119, 102, 30, 122, 2, 117}, 1422}, 
{{80, 101, 2, 116, 113, 102, 118, 25, 99, 8, 19, 23, 116, 
   125, 107, 25, 99, 109, 114, 102, 14, 121, 115, 31, 9, 117, 113, 
   111, 5, 4, 127, 28, 122, 101, 8, 110, 14, 18, 124, 106, 16, 20, 
   104, 119, 8, 109, 26, 106, 9, 97, 13, 99, 15, 119, 20, 105, 117, 
   98, 103, 118, 1, 126, 29, 97, 122, 17, 15, 114, 110, 3, 5, 125, 
   125, 99, 126, 119, 102, 30, 122, 2, 117}, 1012},
{{80, 120, 2, 116, 113, 102, 118, 25, 99, 8, 19, 23, 116, 
   125, 107, 25, 99, 109, 114, 102, 14, 121, 115, 31, 9, 117, 113, 
   111, 5, 4, 127, 28, 122, 101, 8, 110, 14, 18, 124, 106, 16, 20, 
   104, 119, 8, 109, 26, 106, 9, 97, 13, 99, 15, 119, 20, 105, 117, 
   98, 103, 118, 1, 126, 29, 97, 122, 17, 15, 114, 110, 3, 5, 125, 
   125, 99, 126, 119, 102, 30, 122, 2, 117}, 377},

...

{{80, 2, 74, 49, 113, 21, 62, 88, 39, 71, 68, 23, 63, 51, 36, 78, 48, 
   108, 114, 102, 14, 121, 115, 31, 9, 117, 113, 111, 5, 4, 127, 28, 
   122, 101, 8, 110, 14, 18, 124, 106, 16, 20, 104, 119, 8, 109, 26, 
   106, 9, 97, 13, 99, 15, 119, 20, 105, 117, 98, 103, 118, 1, 126, 
   29, 97, 122, 17, 15, 114, 110, 3, 5, 125, 125, 99, 126, 119, 102, 
   30, 122, 2, 117}, 1},
{{80, 1, 74, 59, 113, 45, 56, 86, 52, 91, 19, 64, 60, 60, 63, 
   25, 38, 59, 59, 42, 14, 53, 38, 77, 66, 38, 113, 38, 75, 4, 43, 84,
    63, 101, 64, 43, 79, 64, 40, 57, 16, 91, 46, 119, 69, 40, 84, 117,
    9, 97, 13, 99, 15, 119, 20, 105, 117, 98, 103, 118, 1, 126, 29, 
   97, 122, 17, 15, 114, 110, 3, 5, 125, 125, 99, 126, 119, 102, 30, 
   122, 2, 117}, 1},
{{80, 2, 74, 49, 113, 49, 51, 92, 39, 8, 92, 81, 116, 62, 57, 
   80, 46, 40, 114, 36, 75, 56, 33, 76, 9, 55, 56, 59, 81, 65, 45, 28,
    60, 55, 93, 39, 90, 28, 124, 106, 16, 20, 104, 119, 8, 109, 26, 
   106, 9, 97, 13, 99, 15, 119, 20, 105, 117, 98, 103, 118, 1, 126, 
   29, 97, 122, 17, 15, 114, 110, 3, 5, 125, 125, 99, 126, 119, 102, 
   30, 122, 2, 117}, 1}}
\end{lstlisting}

Вывод Tally это список пар, каждая пара это 81-байтный блок и количество раз, сколько он встретился в файле.
Мы видим, что наиболее частно встречающийся блок это первый, он встретился 1739 раз.
Второй встретился 1422 раза. Есть и другие: 1012 раза, 377 раз, итд.
81-байтные блоки, встреченные лишь один раз, находятся в конце вывода.

Попробуем сравнить эти блоки. Первый и второй.
Есть ли в Mathematica ф-ция для сравнения списков/массивов?
Наверняка есть, но в педагогических целях, я буду использоват операцию XOR для сравнения.
Действительно: если байты во входных массивах равны друг другу, результат операции XOR это 0.
Если не равны, результат будет ненулевой.

Сравним первый блок (встречается 1739 раз) и второй (встречается 1422 раз):

\begin{lstlisting}[style=custommath]
In[]:= BitXor[stat[[1]][[1]], stat[[2]][[1]]]
Out[]= {0, 3, 0, 0, 0, 0, 0, 0, 0, 0, 0, 0, 0, 0, 0, 0, 0, 0, 0, \
0, 0, 0, 0, 0, 0, 0, 0, 0, 0, 0, 0, 0, 0, 0, 0, 0, 0, 0, 0, 0, 0, 0, \
0, 0, 0, 0, 0, 0, 0, 0, 0, 0, 0, 0, 0, 0, 0, 0, 0, 0, 0, 0, 0, 0, 0, \
0, 0, 0, 0, 0, 0, 0, 0, 0, 0, 0, 0, 0, 0, 0, 0}
\end{lstlisting}

Они отличаются только вторым байтом.

Сравним второй блок (встречается 1422 раза) и третий (встречается 1012 раз):

\begin{lstlisting}[style=custommath]
In[]:= BitXor[stat[[2]][[1]], stat[[3]][[1]]]
Out[]= {0, 1, 0, 0, 0, 0, 0, 0, 0, 0, 0, 0, 0, 0, 0, 0, 0, 0, 0, \
0, 0, 0, 0, 0, 0, 0, 0, 0, 0, 0, 0, 0, 0, 0, 0, 0, 0, 0, 0, 0, 0, 0, \
0, 0, 0, 0, 0, 0, 0, 0, 0, 0, 0, 0, 0, 0, 0, 0, 0, 0, 0, 0, 0, 0, 0, \
0, 0, 0, 0, 0, 0, 0, 0, 0, 0, 0, 0, 0, 0, 0, 0}
\end{lstlisting}

Они тоже отличаются только вторым байтом.

Так или иначе, попробуем использовать самый встречающийся блок как XOR-ключ и попробуем расшифровать первые 4 81-байтных
блока в файле:

\begin{lstlisting}[style=custommath]
In[]:= key = stat[[1]][[1]]
Out[]= {80, 103, 2, 116, 113, 102, 118, 25, 99, 8, 19, 23, 116, \
125, 107, 25, 99, 109, 114, 102, 14, 121, 115, 31, 9, 117, 113, 111, \
5, 4, 127, 28, 122, 101, 8, 110, 14, 18, 124, 106, 16, 20, 104, 119, \
8, 109, 26, 106, 9, 97, 13, 99, 15, 119, 20, 105, 117, 98, 103, 118, \
1, 126, 29, 97, 122, 17, 15, 114, 110, 3, 5, 125, 125, 99, 126, 119, \
102, 30, 122, 2, 117}

In[]:= ToASCII[val_] := If[val == 0, " ", FromCharacterCode[val, "PrintableASCII"]]

In[]:= DecryptBlockASCII[blk_] := Map[ToASCII[#] &, BitXor[key, blk]]

In[]:= DecryptBlockASCII[blocks[[1]]]
Out[]= {" ", " ", " ", " ", " ", " ", " ", " ", " ", " ", " ", " \
", " ", " ", " ", " ", " ", " ", " ", " ", " ", " ", " ", " ", " ", " \
", " ", " ", " ", " ", " ", " ", " ", " ", " ", " ", " ", " ", " ", " \
", " ", " ", " ", " ", " ", " ", " ", " ", " ", " ", " ", " ", " ", " \
", " ", " ", " ", " ", " ", " ", " ", " ", " ", " ", " ", " ", " ", " \
", " ", " ", " ", " ", " ", " ", " ", " ", " ", " ", " ", " ", " "}

In[]:= DecryptBlockASCII[blocks[[2]]]
Out[]= {" ", "e", "H", "E", " ", "W", "E", "E", "D", " ", "O", \
"F", " ", "C", "R", "I", "M", "E", " ", "B", "E", "A", "R", "S", " ", \
"B", "I", "T", "T", "E", "R", " ", "F", "R", "U", "I", "T", "?", \
" ", " ", " ", " ", " ", " ", " ", " ", " ", " ", " ", " ", " ", " ", \
" ", " ", " ", " ", " ", " ", " ", " ", " ", " ", " ", " ", " ", " ", \
" ", " ", " ", " ", " ", " ", " ", " ", " ", " ", " ", " ", " ", " ", \
" "}

In[]:= DecryptBlockASCII[blocks[[3]]]
Out[]= {" ", "?", " ", " ", " ", " ", " ", " ", " ", " ", " \
", " ", " ", " ", " ", " ", " ", " ", " ", " ", " ", " ", " ", " ", " \
", " ", " ", " ", " ", " ", " ", " ", " ", " ", " ", " ", " ", " ", " \
", " ", " ", " ", " ", " ", " ", " ", " ", " ", " ", " ", " ", " ", " \
", " ", " ", " ", " ", " ", " ", " ", " ", " ", " ", " ", " ", " ", " \
", " ", " ", " ", " ", " ", " ", " ", " ", " ", " ", " ", " ", " ", " \
"}

In[]:= DecryptBlockASCII[blocks[[4]]]
Out[]= {" ", "f", "H", "O", " ", "K", "N", "O", "W", "S", " ", \
"W", "H", "A", "T", " ", "E", "V", "I", "L", " ", "L", "U", "R", "K", \
"S", " ", "I", "N", " ", "T", "H", "E", " ", "H", "E", "A", "R", "T", \
"S", " ", "O", "F", " ", "M", "E", "N", "?", " ", " ", " ", " ", \
" ", " ", " ", " ", " ", " ", " ", " ", " ", " ", " ", " ", " ", " ", \
" ", " ", " ", " ", " ", " ", " ", " ", " ", " ", " ", " ", " ", " ", \
" "}
\end{lstlisting}

(Я заменил непечатаемые символы на \q{?}.)

Мы видим что первый и третий блоки пустые (или почти пустые),
но второй и четвертый имеют ясно различимые английские слова/фразы.
Похоже что наше предположение насчет ключа верно (как минимум частично).
Это означает, что самый встречающийся 81-байтный блок в файле находится в местах лакун с нулевыми байтами
или что-то в этом роде.

Попробуем расшифровать весь файл:

\begin{lstlisting}[style=custommath]
DecryptBlock[blk_] := BitXor[key, blk]

decrypted = Map[DecryptBlock[#] &, blocks];

BinaryWrite["/home/dennis/.../tmp", Flatten[decrypted]]

Close["/home/dennis/.../tmp"]
\end{lstlisting}

\begin{figure}[H]
\centering
\myincludegraphics{ff/XOR/mask_1/mc_decrypted1.png}
\caption{Расшифрованный файл в Midnight Commander, первая попытка}
\end{figure}

Выглядит как английские фразы для какой-то игры, но что-то не так.
Прежде всего, регистр инвертирован: фразы и некоторые слова начинаются со строчных букв,
в то время как остальные буквы заглавные.
Также, некоторые фразы начинаются с не тех букв.
Посмотрите на самую первую фразу: \q{eHE WEED OF CRIME BEARS BITTER FRUIT}.
Что такое \q{eHE}? Разве не \q{tHE} тут должно быть?
Возможно ли что наш ключ для дешифрования имеет неверный байт в этом месте?

Посмотрим снова на второй блок в файле, на ключ и на результат дешифрования:

\begin{lstlisting}[style=custommath]
In[]:= blocks[[2]]
Out[]= {80, 2, 74, 49, 113, 49, 51, 92, 39, 8, 92, 81, 116, 62, \
57, 80, 46, 40, 114, 36, 75, 56, 33, 76, 9, 55, 56, 59, 81, 65, 45, \
28, 60, 55, 93, 39, 90, 28, 124, 106, 16, 20, 104, 119, 8, 109, 26, \
106, 9, 97, 13, 99, 15, 119, 20, 105, 117, 98, 103, 118, 1, 126, 29, \
97, 122, 17, 15, 114, 110, 3, 5, 125, 125, 99, 126, 119, 102, 30, \
122, 2, 117}

In[]:= key
Out[]= {80, 103, 2, 116, 113, 102, 118, 25, 99, 8, 19, 23, 116, \
125, 107, 25, 99, 109, 114, 102, 14, 121, 115, 31, 9, 117, 113, 111, \
5, 4, 127, 28, 122, 101, 8, 110, 14, 18, 124, 106, 16, 20, 104, 119, \
8, 109, 26, 106, 9, 97, 13, 99, 15, 119, 20, 105, 117, 98, 103, 118, \
1, 126, 29, 97, 122, 17, 15, 114, 110, 3, 5, 125, 125, 99, 126, 119, \
102, 30, 122, 2, 117}

In[]:= BitXor[key, blocks[[2]]]
Out[]= {0, 101, 72, 69, 0, 87, 69, 69, 68, 0, 79, 70, 0, 67, 82, \
73, 77, 69, 0, 66, 69, 65, 82, 83, 0, 66, 73, 84, 84, 69, 82, 0, 70, \
82, 85, 73, 84, 14, 0, 0, 0, 0, 0, 0, 0, 0, 0, 0, 0, 0, 0, 0, 0, 0, \
0, 0, 0, 0, 0, 0, 0, 0, 0, 0, 0, 0, 0, 0, 0, 0, 0, 0, 0, 0, 0, 0, 0, \
0, 0, 0, 0}
\end{lstlisting}

Зашифрованный байт это 2, байт из ключа это 103, $2 \oplus 103=101$ и 101 это ASCII-код символа \q{e}.
Чему должен равнятся этот байт ключа, чтобы ASCII-код был 116 (для символа  \q{t})?
$2 \oplus 116=118$, присвоим 118 второму байту в ключе \dots

\begin{lstlisting}[style=custommath]
key = {80, 118, 2, 116, 113, 102, 118, 25, 99, 8, 19, 23, 116, 125, 
  107, 25, 99, 109, 114, 102, 14, 121, 115, 31, 9, 117, 113, 111, 5, 
  4, 127, 28, 122, 101, 8, 110, 14, 18, 124, 106, 16, 20, 104, 119, 8,
   109, 26, 106, 9, 97, 13, 99, 15, 119, 20, 105, 117, 98, 103, 118, 
  1, 126, 29, 97, 122, 17, 15, 114, 110, 3, 5, 125, 125, 99, 126, 119,
   102, 30, 122, 2, 117}
\end{lstlisting}

\dots и снова дешифруем весь файл.

\begin{figure}[H]
\centering
\myincludegraphics{ff/XOR/mask_1/mc_decrypted2.png}
\caption{Дешифрованный файл в Midnight Commander, вторая попытка}
\end{figure}

Ух ты, теперь грамматика корректна, и все фразы начинаются с корректных букв.
Но все таки, регистр подозрителен.
С чего бы разработчику игры записывать их в такой манере?
Может быть наш ключ все еще неправилен?

% TODO ASCII table somewhere in the book
Изучая таблицу ASCII мы можем заметить что ASCII-коды для букв в верхнем и нижнем регистре отличаются только на один бит
(6-й бит, если считать с первого, 0b100000):

\begin{figure}[H]
\centering
\includegraphics[width=0.7\textwidth]{ascii.png}
\caption{7-битная таблица \ac{ASCII} в Emacs}
\end{figure}

6-й бит, выставленный в нулевом байте, В десятичном виде это будет 32.
Но 32 это ASCII-код пробела!

Действительно, можно менять регистр просто применяя XOR к ASCII-коду, с 32 (больше об этом: \myref{toupper_bit}).

Возможно ли, что пустые лакуны в файле это не нулевые байты, а скорее содержащие пробелы?
Еще раз модифицируем наш XOR-ключ (я про-XOR-ю каждый байт ключа с 32):

\begin{lstlisting}[style=custommath]
(* "32" это скаляр, и "key" это вектор, но это OK *)

In[]:= key3 = BitXor[32, key]
Out[]= {112, 86, 34, 84, 81, 70, 86, 57, 67, 40, 51, 55, 84, 93, 75, \
57, 67, 77, 82, 70, 46, 89, 83, 63, 41, 85, 81, 79, 37, 36, 95, 60, \
90, 69, 40, 78, 46, 50, 92, 74, 48, 52, 72, 87, 40, 77, 58, 74, 41, \
65, 45, 67, 47, 87, 52, 73, 85, 66, 71, 86, 33, 94, 61, 65, 90, 49, \
47, 82, 78, 35, 37, 93, 93, 67, 94, 87, 70, 62, 90, 34, 85}

In[]:= DecryptBlock[blk_] := BitXor[key3, blk]
\end{lstlisting}

И снова дешифруем входной файл:

\begin{figure}[H]
\centering
\myincludegraphics{ff/XOR/mask_1/mc_decrypted.png}
\caption{Дешифрованный файл в Midnight Commander, последняя попытка}
\end{figure}

(Расшифрованный файл доступен здесь:
\url{\GitHubBlobMasterURL/ff/XOR/mask_1/files/decrypted.dat.bz2}.)

Несомненно, это корректный исходный файл.
Да, и мы видим числа в начале каждого блока. Должно быть это и есть источник некорректного XOR-ключа.
Как выходит, самый встречающийся 81-байтный блок в файле это блок заполненный пробелами и содержащий символ \q{1} на месте
второго байта.
Действительно, как-то так получилось что многие блоки здесь перемежаются с этим блоком.
Может быть это что-то вроде выравнивания (padding) для коротких фраз/сообщений?
Другой часто встречающийся 81-байтный блок также заполнен пробелами, но с другой цифрой, следовательно,
они отличаются только вторым байтом.

Вот и всё! Теперь мы можем написать утилиту для зашифрования файла назад, и, может быть, модифицировать его перед этим

Файл для Mathematica можно скачать здесь:\\
\url{\GitHubBlobMasterURL/ff/XOR/mask_1/files/XOR_mask_1.nb}.

Итог: XOR-шифрование не надежно вообще. Вероятно, разработчик игры хотел просто скрыть внутренности игры от игрока,
ничего более серьезного.
Все же, шифрование вроде этого крайне популярно вследствии его простоты, так что многие реверс инженеры обычно хорошо
с этим знакомы.

}\FR{\mysection{Fonction presque vide}
\label{Boolector}
\myindex{Boolector}
\myindex{x86!\Instructions!JMP}

Ceci est un morceau de code réel que j'ai trouvé dans Boolector\footnote{\url{https://boolector.github.io/}}:

\lstinputlisting[style=customc]{patterns/025_almost_empty/boolectormain.c}

Pourquoi quelqu'un ferait-il comme ça?
Je ne sais pas mais mon hypothèse est que \verb|boolector_main()| peut être compilée
dans une sorte de DLL ou bibliothèque dynamique, et appelée depuis une suite de test.
Certainement qu'une suite de test peut préparer les variables argc/argv comme
le ferait \ac{CRT}.

Il est intéressant de voir comment c'est compilé:

\lstinputlisting[caption=GCC 8.2 x64 \NonOptimizing (\assemblyOutput),style=customasmx86]{patterns/025_almost_empty/boolectormain_O0.s}

Ceci est OK, le prologue (non optimisé) déplace inutilement deux arguments,
\INS{CALL}, épilogue, \INS{RET}.
Mais regardons la version optimisée:

\lstinputlisting[caption=GCC 8.2 x64 \Optimizing (\assemblyOutput),style=customasmx86]{patterns/025_almost_empty/boolectormain_O3.s}

Aussi simple que ça: la pile et les registres ne sont pas touchés et \verb|boolector_main()|
a le même ensemble d'arguments.
Donc, tout ce que nous avons à faire est de passer l'exécution à une autre adresse.

Ceci est proche d'une \glslink{thunk function}{fonction thunk}.

Nous verons queelque chose de plus avancé plus tard: \myref{ARM_B_to_printf}, \myref{JMP_instead_of_RET}.
}

\renewcommand{\CURPATH}{advanced/200_string_trim}
\EN{% TODO translate
\mysection{Breaking simple executable cryptor}

I've got an executable file which is encrypted by relatively simple encryption.
\href{\GitHubBlobMasterURL/examples/simple_exec_crypto/files/cipher.bin}{Here is it} (only executable section is left here).

First, all encryption function does is just adds number of position in buffer to the byte.
Here is how this can be encoded in Python:

\begin{lstlisting}[caption=Python script,style=custompy]
#!/usr/bin/env python
def e(i, k):
    return chr ((ord(i)+k) % 256)

def encrypt(buf):
    return e(buf[0], 0)+ e(buf[1], 1)+ e(buf[2], 2) + e(buf[3], 3)+ e(buf[4], 4)+ e(buf[5], 5)+ e(buf[6], 6)+ e(buf[7], 7)+
           e(buf[8], 8)+ e(buf[9], 9)+ e(buf[10], 10)+ e(buf[11], 11)+ e(buf[12], 12)+ e(buf[13], 13)+ e(buf[14], 14)+ e(buf[15], 15)
\end{lstlisting}

Hence, if you encrypt buffer with 16 zeros, you'll get \emph{0, 1, 2, 3 ... 12, 13, 14, 15}.

\myindex{Propagating Cipher Block Chaining}
Propagating Cipher Block Chaining (PCBC) is also used, here is how it works:

\begin{figure}[H]
\centering
\myincludegraphics{examples/simple_exec_crypto/601px-PCBC_encryption.png}
\caption{Propagating Cipher Block Chaining encryption (image is taken from Wikipedia article)}
\end{figure}

The problem is that it's too boring to recover IV (Initialization Vector) each time.
Brute-force is also not an option, because IV is too long (16 bytes).
Let's see, if it's possible to recover IV for arbitrary encrypted executable file?

Let's try simple frequency analysis.
This is 32-bit x86 executable code, so let's gather statistics about most frequent bytes and opcodes.
I tried huge oracle.exe file from Oracle RDBMS version 11.2 for windows x86 and I've found that the most frequent byte (no surprise) is zero (~10\%).
The next most frequent byte is (again, no surprise) 0xFF (~5\%).
The next is 0x8B (~5\%).

\myindex{x86!\Instructions!MOV}
0x8B is opcode for \INS{MOV}, this is indeed one of the most busy x86 instructions.
Now what about popularity of zero byte?
If compiler needs to encode value bigger than 127, it has to use 32-bit displacement instead of 8-bit one, but large values are very rare,
so it is padded by zeros.
\myindex{x86!\Instructions!LEA}
\myindex{x86!\Instructions!PUSH}
\myindex{x86!\Instructions!CALL}
This is at least in \INS{LEA}, \INS{MOV}, \INS{PUSH}, \INS{CALL}.

For example:

\begin{lstlisting}[style=customasmx86]
8D B0 28 01 00 00                 lea     esi, [eax+128h]
8D BF 40 38 00 00                 lea     edi, [edi+3840h]
\end{lstlisting}

Displacements bigger than 127 are very popular, but they are rarely exceeds 0x10000
(indeed, such large memory buffers/structures are also rare).

Same story with \INS{MOV}, large constants are rare, the most heavily used are 0, 1, 10, 100, $2^n$, and so on.
Compiler has to pad small constants by zeros to represent them as 32-bit values:

\begin{lstlisting}[style=customasmx86]
BF 02 00 00 00                    mov     edi, 2
BF 01 00 00 00                    mov     edi, 1
\end{lstlisting}

Now about 00 and FF bytes combined: jumps (including conditional) and calls can pass execution flow forward or backwards, but very often,
within the limits of the current executable module.
If forward, displacement is not very big and also padded with zeros.
If backwards, displacement is represented as negative value, so padded with FF bytes.
For example, transfer execution flow forward:

\begin{lstlisting}[style=customasmx86]
E8 43 0C 00 00                    call    _function1
E8 5C 00 00 00                    call    _function2
0F 84 F0 0A 00 00                 jz      loc_4F09A0
0F 84 EB 00 00 00                 jz      loc_4EFBB8
\end{lstlisting}

Backwards:

\begin{lstlisting}[style=customasmx86]
E8 79 0C FE FF                    call    _function1
E8 F4 16 FF FF                    call    _function2
0F 84 F8 FB FF FF                 jz      loc_8212BC
0F 84 06 FD FF FF                 jz      loc_FF1E7D
\end{lstlisting}

FF byte is also very often occurred in negative displacements like these:

\begin{lstlisting}[style=customasmx86]
8D 85 1E FF FF FF                 lea     eax, [ebp-0E2h]
8D 95 F8 5C FF FF                 lea     edx, [ebp-0A308h]
\end{lstlisting}

So far so good. Now we have to try various 16-byte keys, decrypt executable section and measure how often 00, FF and 8B bytes are occurred.
Let's also keep in sight how PCBC decryption works:

\begin{figure}[H]
\centering
\myincludegraphics{examples/simple_exec_crypto/640px-PCBC_decryption.png}
\caption{Propagating Cipher Block Chaining decryption (image is taken from Wikipedia article)}
\end{figure}

The good news is that we don't really have to decrypt whole piece of data, but only slice by slice, this is exactly how I did in my previous example: \myref{XOR_mask_2}.

Now I'm trying all possible bytes (0..255) for each byte in key and just pick the byte producing maximal amount of 00/FF/8B bytes in a decrypted slice:

\begin{lstlisting}[style=custompy]
#!/usr/bin/env python
import sys, hexdump, array, string, operator

KEY_LEN=16

def chunks(l, n):
    # split n by l-byte chunks
    # https://stackoverflow.com/q/312443
    n = max(1, n)
    return [l[i:i + n] for i in range(0, len(l), n)]

def read_file(fname):
    file=open(fname, mode='rb')
    content=file.read()
    file.close()
    return content

def decrypt_byte (c, key):
    return chr((ord(c)-key) % 256)

def XOR_PCBC_step (IV, buf, k):
    prev=IV
    rt=""
    for c in buf:
	new_c=decrypt_byte(c, k)
        plain=chr(ord(new_c)^ord(prev))
	prev=chr(ord(c)^ord(plain))
	rt=rt+plain
    return rt

each_Nth_byte=[""]*KEY_LEN

content=read_file(sys.argv[1])
# split input by 16-byte chunks:
all_chunks=chunks(content, KEY_LEN)
for c in all_chunks:
    for i in range(KEY_LEN):
        each_Nth_byte[i]=each_Nth_byte[i] + c[i]

# try each byte of key
for N in range(KEY_LEN):
    print "N=", N
    stat={}
    for i in range(256):
        tmp_key=chr(i)
	tmp=XOR_PCBC_step(tmp_key,each_Nth_byte[N], N)
        # count 0, FFs and 8Bs in decrypted buffer:
	important_bytes=tmp.count('\x00')+tmp.count('\xFF')+tmp.count('\x8B')
	stat[i]=important_bytes
    sorted_stat = sorted(stat.iteritems(), key=operator.itemgetter(1), reverse=True)
    print sorted_stat[0]
\end{lstlisting}

(Source code can be downloaded \href{\GitHubBlobMasterURL/examples/simple_exec_crypto/files/decrypt.py}{here}.)

I run it and here is a key for which 00/FF/8B bytes presence in decrypted buffer is maximal:

\begin{lstlisting}
N= 0
(147, 1224)
N= 1
(94, 1327)
N= 2
(252, 1223)
N= 3
(218, 1266)
N= 4
(38, 1209)
N= 5
(192, 1378)
N= 6
(199, 1204)
N= 7
(213, 1332)
N= 8
(225, 1251)
N= 9
(112, 1223)
N= 10
(143, 1177)
N= 11
(108, 1286)
N= 12
(10, 1164)
N= 13
(3, 1271)
N= 14
(128, 1253)
N= 15
(232, 1330)
\end{lstlisting}

Let's write decryption utility with the key we got:

\begin{lstlisting}[style=custompy]
#!/usr/bin/env python
import sys, hexdump, array

def xor_strings(s,t):
    # \verb|https://en.wikipedia.org/wiki/XOR_cipher#Example_implementation|
    """xor two strings together"""
    return "".join(chr(ord(a)^ord(b)) for a,b in zip(s,t))

IV=array.array('B', [147, 94, 252, 218, 38, 192, 199, 213, 225, 112, 143, 108, 10, 3, 128, 232]).tostring()

def chunks(l, n):
    n = max(1, n)
    return [l[i:i + n] for i in range(0, len(l), n)]

def read_file(fname):
    file=open(fname, mode='rb')
    content=file.read()
    file.close()
    return content

def decrypt_byte(i, k):
    return chr ((ord(i)-k) % 256)

def decrypt(buf):
    return "".join(decrypt_byte(buf[i], i) for i in range(16))

fout=open(sys.argv[2], mode='wb')

prev=IV
content=read_file(sys.argv[1])
tmp=chunks(content, 16)
for c in tmp:
    new_c=decrypt(c)
    p=xor_strings (new_c, prev)
    prev=xor_strings(c, p)
    fout.write(p)
fout.close()
\end{lstlisting}

(Source code can be downloaded \href{\GitHubBlobMasterURL/examples/simple_exec_crypto/files/decrypt2.py}{here}.)

Let's check resulting file:

\lstinputlisting{examples/simple_exec_crypto/objdump_result.txt}

Yes, this is seems correctly disassembled piece of x86 code.
The whole decryped file can be downloaded \href{\GitHubBlobMasterURL/examples/simple_exec_crypto/files/decrypted.bin}{here}.

In fact, this is text section from regedit.exe from Windows 7.
But this example is based on a real case I encountered, so just executable is different (and key), algorithm is the same.

\subsection{Other ideas to consider}

What if I would fail with such simple frequency analysis?
There are other ideas on how to measure correctness of decrypted/decompressed x86 code:

\begin{itemize}

\item Many modern compilers aligns functions on 0x10 border.
So the space left before is filled with NOPs (0x90) or other NOP instructions with known opcodes: \myref{sec:npad}.

\item Perhaps, the most frequent pattern in any assembly language is function call:\\
\TT{PUSH chain / CALL / ADD ESP, X}.
This sequence can easily detected and found.
I've even gathered statistics about average number of function arguments: \myref{args_stat}.
(Hence, this is average length of PUSH chain.)

\end{itemize}

Read more about incorrectly/correctly disassembled code: \myref{ISA_detect}.
}\RU{\subsection{Простое шифрование используя XOR-маску}
\label{XOR_mask_1}

Я нашел одну старую игру в стиле interactive fiction в архиве \emph{if-archive}\footnote{\url{http://www.ifarchive.org/}}:

\begin{lstlisting}
The New Castle v3.5 - Text/Adventure Game
in the style of the original Infocom (tm)
type games, Zork, Collosal Cave (Adventure),
etc.  Can you solve the mystery of the
abandoned castle?
Shareware from Software Customization.
Software Customization [ASP] Version 3.5 Feb. 2000
\end{lstlisting}

Можно скачать здесь: \url{\GitHubBlobMasterURL/ff/XOR/mask_1/files/newcastle.tgz}.

Там внутри есть файл (с названием \emph{castle.dbf}), который явно зашифрован, но не настоящим криптоалгоритмом,
и оне сжат, это что-то куда проще.
Я бы даже не стал измерять уровень энтропии (\myref{entropy}) этого файла, потому что я итак уверен, что он низкий.
Вот как он выглядит в Midnight Commander:

\begin{figure}[H]
\centering
\myincludegraphics{ff/XOR/mask_1/mc_encrypted.png}
\caption{Зашифрованный файл в Midnight Commander}
\end{figure}

Зашифрованный файл можно скачать здесь:
\url{\GitHubBlobMasterURL/ff/XOR/mask_1/files/castle.dbf.bz2}.

Можно ли расшифровать его без доступа к программе, используя просто этот файл?

Тут явно просматривается повторяющаяся строка. 
Если использовалось простое шифрование с XOR-маской, такие повторяющиеся строки это явное свидетельство,
потому что, вероятно, тут были длинные лакуны с нулевыми байтами, которые, в свою очередь, присутствуют
во мноигих исполняемых файлах, и в остальных бинарных файлах.

\myindex{UNIX!xxd}
Вот дам начала этого файла используя утилиту \emph{xxd} из UNIX:

\lstinputlisting{ff/XOR/mask_1/xxd_result.txt}

Давайте держаться за повторяющуюся строку \TT{iubgv}.
Глядя на этот дамп, мы можем легко увидеть, что период повторений этой строки это 0x51 или 81.
Вероятно, 81 это длина блока?
Длина файла 1658961, и она может быть поделена на 81 без остатка (и тогда там 20481 блоков).

Теперь я буду использовать Mathematica для анализа, есть ли тут повторяющиеся 81-байтные блоки в файле?
Я разделю входной файл на 81-байтные блоки и затем использую ф-цию
\emph{Tally[]}\footnote{\url{https://reference.wolfram.com/language/ref/Tally.html}}
которая просто считает, сколько раз каждый элемент встретился во входном списке.
Вывод Tally не отсортирован, так что я также добавлю ф-цию \emph{Sort[]} для сортировки его по кол-ву вхождений
в нисходящем порядке.

\begin{lstlisting}[style=custommath]
input = BinaryReadList["/home/dennis/.../castle.dbf"];

blocks = Partition[input, 81];

stat = Sort[Tally[blocks], #1[[2]] > #2[[2]] &]
\end{lstlisting}

И вот вывод:

\begin{lstlisting}[style=custommath]
{{{80, 103, 2, 116, 113, 102, 118, 25, 99, 8, 19, 23, 116, 125, 107, 
   25, 99, 109, 114, 102, 14, 121, 115, 31, 9, 117, 113, 111, 5, 4, 
   127, 28, 122, 101, 8, 110, 14, 18, 124, 106, 16, 20, 104, 119, 8, 
   109, 26, 106, 9, 97, 13, 99, 15, 119, 20, 105, 117, 98, 103, 118, 
   1, 126, 29, 97, 122, 17, 15, 114, 110, 3, 5, 125, 125, 99, 126, 
   119, 102, 30, 122, 2, 117}, 1739}, 
{{80, 100, 2, 116, 113, 102, 118, 25, 99, 8, 19, 23, 116, 
   125, 107, 25, 99, 109, 114, 102, 14, 121, 115, 31, 9, 117, 113, 
   111, 5, 4, 127, 28, 122, 101, 8, 110, 14, 18, 124, 106, 16, 20, 
   104, 119, 8, 109, 26, 106, 9, 97, 13, 99, 15, 119, 20, 105, 117, 
   98, 103, 118, 1, 126, 29, 97, 122, 17, 15, 114, 110, 3, 5, 125, 
   125, 99, 126, 119, 102, 30, 122, 2, 117}, 1422}, 
{{80, 101, 2, 116, 113, 102, 118, 25, 99, 8, 19, 23, 116, 
   125, 107, 25, 99, 109, 114, 102, 14, 121, 115, 31, 9, 117, 113, 
   111, 5, 4, 127, 28, 122, 101, 8, 110, 14, 18, 124, 106, 16, 20, 
   104, 119, 8, 109, 26, 106, 9, 97, 13, 99, 15, 119, 20, 105, 117, 
   98, 103, 118, 1, 126, 29, 97, 122, 17, 15, 114, 110, 3, 5, 125, 
   125, 99, 126, 119, 102, 30, 122, 2, 117}, 1012},
{{80, 120, 2, 116, 113, 102, 118, 25, 99, 8, 19, 23, 116, 
   125, 107, 25, 99, 109, 114, 102, 14, 121, 115, 31, 9, 117, 113, 
   111, 5, 4, 127, 28, 122, 101, 8, 110, 14, 18, 124, 106, 16, 20, 
   104, 119, 8, 109, 26, 106, 9, 97, 13, 99, 15, 119, 20, 105, 117, 
   98, 103, 118, 1, 126, 29, 97, 122, 17, 15, 114, 110, 3, 5, 125, 
   125, 99, 126, 119, 102, 30, 122, 2, 117}, 377},

...

{{80, 2, 74, 49, 113, 21, 62, 88, 39, 71, 68, 23, 63, 51, 36, 78, 48, 
   108, 114, 102, 14, 121, 115, 31, 9, 117, 113, 111, 5, 4, 127, 28, 
   122, 101, 8, 110, 14, 18, 124, 106, 16, 20, 104, 119, 8, 109, 26, 
   106, 9, 97, 13, 99, 15, 119, 20, 105, 117, 98, 103, 118, 1, 126, 
   29, 97, 122, 17, 15, 114, 110, 3, 5, 125, 125, 99, 126, 119, 102, 
   30, 122, 2, 117}, 1},
{{80, 1, 74, 59, 113, 45, 56, 86, 52, 91, 19, 64, 60, 60, 63, 
   25, 38, 59, 59, 42, 14, 53, 38, 77, 66, 38, 113, 38, 75, 4, 43, 84,
    63, 101, 64, 43, 79, 64, 40, 57, 16, 91, 46, 119, 69, 40, 84, 117,
    9, 97, 13, 99, 15, 119, 20, 105, 117, 98, 103, 118, 1, 126, 29, 
   97, 122, 17, 15, 114, 110, 3, 5, 125, 125, 99, 126, 119, 102, 30, 
   122, 2, 117}, 1},
{{80, 2, 74, 49, 113, 49, 51, 92, 39, 8, 92, 81, 116, 62, 57, 
   80, 46, 40, 114, 36, 75, 56, 33, 76, 9, 55, 56, 59, 81, 65, 45, 28,
    60, 55, 93, 39, 90, 28, 124, 106, 16, 20, 104, 119, 8, 109, 26, 
   106, 9, 97, 13, 99, 15, 119, 20, 105, 117, 98, 103, 118, 1, 126, 
   29, 97, 122, 17, 15, 114, 110, 3, 5, 125, 125, 99, 126, 119, 102, 
   30, 122, 2, 117}, 1}}
\end{lstlisting}

Вывод Tally это список пар, каждая пара это 81-байтный блок и количество раз, сколько он встретился в файле.
Мы видим, что наиболее частно встречающийся блок это первый, он встретился 1739 раз.
Второй встретился 1422 раза. Есть и другие: 1012 раза, 377 раз, итд.
81-байтные блоки, встреченные лишь один раз, находятся в конце вывода.

Попробуем сравнить эти блоки. Первый и второй.
Есть ли в Mathematica ф-ция для сравнения списков/массивов?
Наверняка есть, но в педагогических целях, я буду использоват операцию XOR для сравнения.
Действительно: если байты во входных массивах равны друг другу, результат операции XOR это 0.
Если не равны, результат будет ненулевой.

Сравним первый блок (встречается 1739 раз) и второй (встречается 1422 раз):

\begin{lstlisting}[style=custommath]
In[]:= BitXor[stat[[1]][[1]], stat[[2]][[1]]]
Out[]= {0, 3, 0, 0, 0, 0, 0, 0, 0, 0, 0, 0, 0, 0, 0, 0, 0, 0, 0, \
0, 0, 0, 0, 0, 0, 0, 0, 0, 0, 0, 0, 0, 0, 0, 0, 0, 0, 0, 0, 0, 0, 0, \
0, 0, 0, 0, 0, 0, 0, 0, 0, 0, 0, 0, 0, 0, 0, 0, 0, 0, 0, 0, 0, 0, 0, \
0, 0, 0, 0, 0, 0, 0, 0, 0, 0, 0, 0, 0, 0, 0, 0}
\end{lstlisting}

Они отличаются только вторым байтом.

Сравним второй блок (встречается 1422 раза) и третий (встречается 1012 раз):

\begin{lstlisting}[style=custommath]
In[]:= BitXor[stat[[2]][[1]], stat[[3]][[1]]]
Out[]= {0, 1, 0, 0, 0, 0, 0, 0, 0, 0, 0, 0, 0, 0, 0, 0, 0, 0, 0, \
0, 0, 0, 0, 0, 0, 0, 0, 0, 0, 0, 0, 0, 0, 0, 0, 0, 0, 0, 0, 0, 0, 0, \
0, 0, 0, 0, 0, 0, 0, 0, 0, 0, 0, 0, 0, 0, 0, 0, 0, 0, 0, 0, 0, 0, 0, \
0, 0, 0, 0, 0, 0, 0, 0, 0, 0, 0, 0, 0, 0, 0, 0}
\end{lstlisting}

Они тоже отличаются только вторым байтом.

Так или иначе, попробуем использовать самый встречающийся блок как XOR-ключ и попробуем расшифровать первые 4 81-байтных
блока в файле:

\begin{lstlisting}[style=custommath]
In[]:= key = stat[[1]][[1]]
Out[]= {80, 103, 2, 116, 113, 102, 118, 25, 99, 8, 19, 23, 116, \
125, 107, 25, 99, 109, 114, 102, 14, 121, 115, 31, 9, 117, 113, 111, \
5, 4, 127, 28, 122, 101, 8, 110, 14, 18, 124, 106, 16, 20, 104, 119, \
8, 109, 26, 106, 9, 97, 13, 99, 15, 119, 20, 105, 117, 98, 103, 118, \
1, 126, 29, 97, 122, 17, 15, 114, 110, 3, 5, 125, 125, 99, 126, 119, \
102, 30, 122, 2, 117}

In[]:= ToASCII[val_] := If[val == 0, " ", FromCharacterCode[val, "PrintableASCII"]]

In[]:= DecryptBlockASCII[blk_] := Map[ToASCII[#] &, BitXor[key, blk]]

In[]:= DecryptBlockASCII[blocks[[1]]]
Out[]= {" ", " ", " ", " ", " ", " ", " ", " ", " ", " ", " ", " \
", " ", " ", " ", " ", " ", " ", " ", " ", " ", " ", " ", " ", " ", " \
", " ", " ", " ", " ", " ", " ", " ", " ", " ", " ", " ", " ", " ", " \
", " ", " ", " ", " ", " ", " ", " ", " ", " ", " ", " ", " ", " ", " \
", " ", " ", " ", " ", " ", " ", " ", " ", " ", " ", " ", " ", " ", " \
", " ", " ", " ", " ", " ", " ", " ", " ", " ", " ", " ", " ", " "}

In[]:= DecryptBlockASCII[blocks[[2]]]
Out[]= {" ", "e", "H", "E", " ", "W", "E", "E", "D", " ", "O", \
"F", " ", "C", "R", "I", "M", "E", " ", "B", "E", "A", "R", "S", " ", \
"B", "I", "T", "T", "E", "R", " ", "F", "R", "U", "I", "T", "?", \
" ", " ", " ", " ", " ", " ", " ", " ", " ", " ", " ", " ", " ", " ", \
" ", " ", " ", " ", " ", " ", " ", " ", " ", " ", " ", " ", " ", " ", \
" ", " ", " ", " ", " ", " ", " ", " ", " ", " ", " ", " ", " ", " ", \
" "}

In[]:= DecryptBlockASCII[blocks[[3]]]
Out[]= {" ", "?", " ", " ", " ", " ", " ", " ", " ", " ", " \
", " ", " ", " ", " ", " ", " ", " ", " ", " ", " ", " ", " ", " ", " \
", " ", " ", " ", " ", " ", " ", " ", " ", " ", " ", " ", " ", " ", " \
", " ", " ", " ", " ", " ", " ", " ", " ", " ", " ", " ", " ", " ", " \
", " ", " ", " ", " ", " ", " ", " ", " ", " ", " ", " ", " ", " ", " \
", " ", " ", " ", " ", " ", " ", " ", " ", " ", " ", " ", " ", " ", " \
"}

In[]:= DecryptBlockASCII[blocks[[4]]]
Out[]= {" ", "f", "H", "O", " ", "K", "N", "O", "W", "S", " ", \
"W", "H", "A", "T", " ", "E", "V", "I", "L", " ", "L", "U", "R", "K", \
"S", " ", "I", "N", " ", "T", "H", "E", " ", "H", "E", "A", "R", "T", \
"S", " ", "O", "F", " ", "M", "E", "N", "?", " ", " ", " ", " ", \
" ", " ", " ", " ", " ", " ", " ", " ", " ", " ", " ", " ", " ", " ", \
" ", " ", " ", " ", " ", " ", " ", " ", " ", " ", " ", " ", " ", " ", \
" "}
\end{lstlisting}

(Я заменил непечатаемые символы на \q{?}.)

Мы видим что первый и третий блоки пустые (или почти пустые),
но второй и четвертый имеют ясно различимые английские слова/фразы.
Похоже что наше предположение насчет ключа верно (как минимум частично).
Это означает, что самый встречающийся 81-байтный блок в файле находится в местах лакун с нулевыми байтами
или что-то в этом роде.

Попробуем расшифровать весь файл:

\begin{lstlisting}[style=custommath]
DecryptBlock[blk_] := BitXor[key, blk]

decrypted = Map[DecryptBlock[#] &, blocks];

BinaryWrite["/home/dennis/.../tmp", Flatten[decrypted]]

Close["/home/dennis/.../tmp"]
\end{lstlisting}

\begin{figure}[H]
\centering
\myincludegraphics{ff/XOR/mask_1/mc_decrypted1.png}
\caption{Расшифрованный файл в Midnight Commander, первая попытка}
\end{figure}

Выглядит как английские фразы для какой-то игры, но что-то не так.
Прежде всего, регистр инвертирован: фразы и некоторые слова начинаются со строчных букв,
в то время как остальные буквы заглавные.
Также, некоторые фразы начинаются с не тех букв.
Посмотрите на самую первую фразу: \q{eHE WEED OF CRIME BEARS BITTER FRUIT}.
Что такое \q{eHE}? Разве не \q{tHE} тут должно быть?
Возможно ли что наш ключ для дешифрования имеет неверный байт в этом месте?

Посмотрим снова на второй блок в файле, на ключ и на результат дешифрования:

\begin{lstlisting}[style=custommath]
In[]:= blocks[[2]]
Out[]= {80, 2, 74, 49, 113, 49, 51, 92, 39, 8, 92, 81, 116, 62, \
57, 80, 46, 40, 114, 36, 75, 56, 33, 76, 9, 55, 56, 59, 81, 65, 45, \
28, 60, 55, 93, 39, 90, 28, 124, 106, 16, 20, 104, 119, 8, 109, 26, \
106, 9, 97, 13, 99, 15, 119, 20, 105, 117, 98, 103, 118, 1, 126, 29, \
97, 122, 17, 15, 114, 110, 3, 5, 125, 125, 99, 126, 119, 102, 30, \
122, 2, 117}

In[]:= key
Out[]= {80, 103, 2, 116, 113, 102, 118, 25, 99, 8, 19, 23, 116, \
125, 107, 25, 99, 109, 114, 102, 14, 121, 115, 31, 9, 117, 113, 111, \
5, 4, 127, 28, 122, 101, 8, 110, 14, 18, 124, 106, 16, 20, 104, 119, \
8, 109, 26, 106, 9, 97, 13, 99, 15, 119, 20, 105, 117, 98, 103, 118, \
1, 126, 29, 97, 122, 17, 15, 114, 110, 3, 5, 125, 125, 99, 126, 119, \
102, 30, 122, 2, 117}

In[]:= BitXor[key, blocks[[2]]]
Out[]= {0, 101, 72, 69, 0, 87, 69, 69, 68, 0, 79, 70, 0, 67, 82, \
73, 77, 69, 0, 66, 69, 65, 82, 83, 0, 66, 73, 84, 84, 69, 82, 0, 70, \
82, 85, 73, 84, 14, 0, 0, 0, 0, 0, 0, 0, 0, 0, 0, 0, 0, 0, 0, 0, 0, \
0, 0, 0, 0, 0, 0, 0, 0, 0, 0, 0, 0, 0, 0, 0, 0, 0, 0, 0, 0, 0, 0, 0, \
0, 0, 0, 0}
\end{lstlisting}

Зашифрованный байт это 2, байт из ключа это 103, $2 \oplus 103=101$ и 101 это ASCII-код символа \q{e}.
Чему должен равнятся этот байт ключа, чтобы ASCII-код был 116 (для символа  \q{t})?
$2 \oplus 116=118$, присвоим 118 второму байту в ключе \dots

\begin{lstlisting}[style=custommath]
key = {80, 118, 2, 116, 113, 102, 118, 25, 99, 8, 19, 23, 116, 125, 
  107, 25, 99, 109, 114, 102, 14, 121, 115, 31, 9, 117, 113, 111, 5, 
  4, 127, 28, 122, 101, 8, 110, 14, 18, 124, 106, 16, 20, 104, 119, 8,
   109, 26, 106, 9, 97, 13, 99, 15, 119, 20, 105, 117, 98, 103, 118, 
  1, 126, 29, 97, 122, 17, 15, 114, 110, 3, 5, 125, 125, 99, 126, 119,
   102, 30, 122, 2, 117}
\end{lstlisting}

\dots и снова дешифруем весь файл.

\begin{figure}[H]
\centering
\myincludegraphics{ff/XOR/mask_1/mc_decrypted2.png}
\caption{Дешифрованный файл в Midnight Commander, вторая попытка}
\end{figure}

Ух ты, теперь грамматика корректна, и все фразы начинаются с корректных букв.
Но все таки, регистр подозрителен.
С чего бы разработчику игры записывать их в такой манере?
Может быть наш ключ все еще неправилен?

% TODO ASCII table somewhere in the book
Изучая таблицу ASCII мы можем заметить что ASCII-коды для букв в верхнем и нижнем регистре отличаются только на один бит
(6-й бит, если считать с первого, 0b100000):

\begin{figure}[H]
\centering
\includegraphics[width=0.7\textwidth]{ascii.png}
\caption{7-битная таблица \ac{ASCII} в Emacs}
\end{figure}

6-й бит, выставленный в нулевом байте, В десятичном виде это будет 32.
Но 32 это ASCII-код пробела!

Действительно, можно менять регистр просто применяя XOR к ASCII-коду, с 32 (больше об этом: \myref{toupper_bit}).

Возможно ли, что пустые лакуны в файле это не нулевые байты, а скорее содержащие пробелы?
Еще раз модифицируем наш XOR-ключ (я про-XOR-ю каждый байт ключа с 32):

\begin{lstlisting}[style=custommath]
(* "32" это скаляр, и "key" это вектор, но это OK *)

In[]:= key3 = BitXor[32, key]
Out[]= {112, 86, 34, 84, 81, 70, 86, 57, 67, 40, 51, 55, 84, 93, 75, \
57, 67, 77, 82, 70, 46, 89, 83, 63, 41, 85, 81, 79, 37, 36, 95, 60, \
90, 69, 40, 78, 46, 50, 92, 74, 48, 52, 72, 87, 40, 77, 58, 74, 41, \
65, 45, 67, 47, 87, 52, 73, 85, 66, 71, 86, 33, 94, 61, 65, 90, 49, \
47, 82, 78, 35, 37, 93, 93, 67, 94, 87, 70, 62, 90, 34, 85}

In[]:= DecryptBlock[blk_] := BitXor[key3, blk]
\end{lstlisting}

И снова дешифруем входной файл:

\begin{figure}[H]
\centering
\myincludegraphics{ff/XOR/mask_1/mc_decrypted.png}
\caption{Дешифрованный файл в Midnight Commander, последняя попытка}
\end{figure}

(Расшифрованный файл доступен здесь:
\url{\GitHubBlobMasterURL/ff/XOR/mask_1/files/decrypted.dat.bz2}.)

Несомненно, это корректный исходный файл.
Да, и мы видим числа в начале каждого блока. Должно быть это и есть источник некорректного XOR-ключа.
Как выходит, самый встречающийся 81-байтный блок в файле это блок заполненный пробелами и содержащий символ \q{1} на месте
второго байта.
Действительно, как-то так получилось что многие блоки здесь перемежаются с этим блоком.
Может быть это что-то вроде выравнивания (padding) для коротких фраз/сообщений?
Другой часто встречающийся 81-байтный блок также заполнен пробелами, но с другой цифрой, следовательно,
они отличаются только вторым байтом.

Вот и всё! Теперь мы можем написать утилиту для зашифрования файла назад, и, может быть, модифицировать его перед этим

Файл для Mathematica можно скачать здесь:\\
\url{\GitHubBlobMasterURL/ff/XOR/mask_1/files/XOR_mask_1.nb}.

Итог: XOR-шифрование не надежно вообще. Вероятно, разработчик игры хотел просто скрыть внутренности игры от игрока,
ничего более серьезного.
Все же, шифрование вроде этого крайне популярно вследствии его простоты, так что многие реверс инженеры обычно хорошо
с этим знакомы.

}\FR{\mysection{Fonction presque vide}
\label{Boolector}
\myindex{Boolector}
\myindex{x86!\Instructions!JMP}

Ceci est un morceau de code réel que j'ai trouvé dans Boolector\footnote{\url{https://boolector.github.io/}}:

\lstinputlisting[style=customc]{patterns/025_almost_empty/boolectormain.c}

Pourquoi quelqu'un ferait-il comme ça?
Je ne sais pas mais mon hypothèse est que \verb|boolector_main()| peut être compilée
dans une sorte de DLL ou bibliothèque dynamique, et appelée depuis une suite de test.
Certainement qu'une suite de test peut préparer les variables argc/argv comme
le ferait \ac{CRT}.

Il est intéressant de voir comment c'est compilé:

\lstinputlisting[caption=GCC 8.2 x64 \NonOptimizing (\assemblyOutput),style=customasmx86]{patterns/025_almost_empty/boolectormain_O0.s}

Ceci est OK, le prologue (non optimisé) déplace inutilement deux arguments,
\INS{CALL}, épilogue, \INS{RET}.
Mais regardons la version optimisée:

\lstinputlisting[caption=GCC 8.2 x64 \Optimizing (\assemblyOutput),style=customasmx86]{patterns/025_almost_empty/boolectormain_O3.s}

Aussi simple que ça: la pile et les registres ne sont pas touchés et \verb|boolector_main()|
a le même ensemble d'arguments.
Donc, tout ce que nous avons à faire est de passer l'exécution à une autre adresse.

Ceci est proche d'une \glslink{thunk function}{fonction thunk}.

Nous verons queelque chose de plus avancé plus tard: \myref{ARM_B_to_printf}, \myref{JMP_instead_of_RET}.
}

\renewcommand{\CURPATH}{advanced/250_toupper}
\EN{% TODO translate
\mysection{Breaking simple executable cryptor}

I've got an executable file which is encrypted by relatively simple encryption.
\href{\GitHubBlobMasterURL/examples/simple_exec_crypto/files/cipher.bin}{Here is it} (only executable section is left here).

First, all encryption function does is just adds number of position in buffer to the byte.
Here is how this can be encoded in Python:

\begin{lstlisting}[caption=Python script,style=custompy]
#!/usr/bin/env python
def e(i, k):
    return chr ((ord(i)+k) % 256)

def encrypt(buf):
    return e(buf[0], 0)+ e(buf[1], 1)+ e(buf[2], 2) + e(buf[3], 3)+ e(buf[4], 4)+ e(buf[5], 5)+ e(buf[6], 6)+ e(buf[7], 7)+
           e(buf[8], 8)+ e(buf[9], 9)+ e(buf[10], 10)+ e(buf[11], 11)+ e(buf[12], 12)+ e(buf[13], 13)+ e(buf[14], 14)+ e(buf[15], 15)
\end{lstlisting}

Hence, if you encrypt buffer with 16 zeros, you'll get \emph{0, 1, 2, 3 ... 12, 13, 14, 15}.

\myindex{Propagating Cipher Block Chaining}
Propagating Cipher Block Chaining (PCBC) is also used, here is how it works:

\begin{figure}[H]
\centering
\myincludegraphics{examples/simple_exec_crypto/601px-PCBC_encryption.png}
\caption{Propagating Cipher Block Chaining encryption (image is taken from Wikipedia article)}
\end{figure}

The problem is that it's too boring to recover IV (Initialization Vector) each time.
Brute-force is also not an option, because IV is too long (16 bytes).
Let's see, if it's possible to recover IV for arbitrary encrypted executable file?

Let's try simple frequency analysis.
This is 32-bit x86 executable code, so let's gather statistics about most frequent bytes and opcodes.
I tried huge oracle.exe file from Oracle RDBMS version 11.2 for windows x86 and I've found that the most frequent byte (no surprise) is zero (~10\%).
The next most frequent byte is (again, no surprise) 0xFF (~5\%).
The next is 0x8B (~5\%).

\myindex{x86!\Instructions!MOV}
0x8B is opcode for \INS{MOV}, this is indeed one of the most busy x86 instructions.
Now what about popularity of zero byte?
If compiler needs to encode value bigger than 127, it has to use 32-bit displacement instead of 8-bit one, but large values are very rare,
so it is padded by zeros.
\myindex{x86!\Instructions!LEA}
\myindex{x86!\Instructions!PUSH}
\myindex{x86!\Instructions!CALL}
This is at least in \INS{LEA}, \INS{MOV}, \INS{PUSH}, \INS{CALL}.

For example:

\begin{lstlisting}[style=customasmx86]
8D B0 28 01 00 00                 lea     esi, [eax+128h]
8D BF 40 38 00 00                 lea     edi, [edi+3840h]
\end{lstlisting}

Displacements bigger than 127 are very popular, but they are rarely exceeds 0x10000
(indeed, such large memory buffers/structures are also rare).

Same story with \INS{MOV}, large constants are rare, the most heavily used are 0, 1, 10, 100, $2^n$, and so on.
Compiler has to pad small constants by zeros to represent them as 32-bit values:

\begin{lstlisting}[style=customasmx86]
BF 02 00 00 00                    mov     edi, 2
BF 01 00 00 00                    mov     edi, 1
\end{lstlisting}

Now about 00 and FF bytes combined: jumps (including conditional) and calls can pass execution flow forward or backwards, but very often,
within the limits of the current executable module.
If forward, displacement is not very big and also padded with zeros.
If backwards, displacement is represented as negative value, so padded with FF bytes.
For example, transfer execution flow forward:

\begin{lstlisting}[style=customasmx86]
E8 43 0C 00 00                    call    _function1
E8 5C 00 00 00                    call    _function2
0F 84 F0 0A 00 00                 jz      loc_4F09A0
0F 84 EB 00 00 00                 jz      loc_4EFBB8
\end{lstlisting}

Backwards:

\begin{lstlisting}[style=customasmx86]
E8 79 0C FE FF                    call    _function1
E8 F4 16 FF FF                    call    _function2
0F 84 F8 FB FF FF                 jz      loc_8212BC
0F 84 06 FD FF FF                 jz      loc_FF1E7D
\end{lstlisting}

FF byte is also very often occurred in negative displacements like these:

\begin{lstlisting}[style=customasmx86]
8D 85 1E FF FF FF                 lea     eax, [ebp-0E2h]
8D 95 F8 5C FF FF                 lea     edx, [ebp-0A308h]
\end{lstlisting}

So far so good. Now we have to try various 16-byte keys, decrypt executable section and measure how often 00, FF and 8B bytes are occurred.
Let's also keep in sight how PCBC decryption works:

\begin{figure}[H]
\centering
\myincludegraphics{examples/simple_exec_crypto/640px-PCBC_decryption.png}
\caption{Propagating Cipher Block Chaining decryption (image is taken from Wikipedia article)}
\end{figure}

The good news is that we don't really have to decrypt whole piece of data, but only slice by slice, this is exactly how I did in my previous example: \myref{XOR_mask_2}.

Now I'm trying all possible bytes (0..255) for each byte in key and just pick the byte producing maximal amount of 00/FF/8B bytes in a decrypted slice:

\begin{lstlisting}[style=custompy]
#!/usr/bin/env python
import sys, hexdump, array, string, operator

KEY_LEN=16

def chunks(l, n):
    # split n by l-byte chunks
    # https://stackoverflow.com/q/312443
    n = max(1, n)
    return [l[i:i + n] for i in range(0, len(l), n)]

def read_file(fname):
    file=open(fname, mode='rb')
    content=file.read()
    file.close()
    return content

def decrypt_byte (c, key):
    return chr((ord(c)-key) % 256)

def XOR_PCBC_step (IV, buf, k):
    prev=IV
    rt=""
    for c in buf:
	new_c=decrypt_byte(c, k)
        plain=chr(ord(new_c)^ord(prev))
	prev=chr(ord(c)^ord(plain))
	rt=rt+plain
    return rt

each_Nth_byte=[""]*KEY_LEN

content=read_file(sys.argv[1])
# split input by 16-byte chunks:
all_chunks=chunks(content, KEY_LEN)
for c in all_chunks:
    for i in range(KEY_LEN):
        each_Nth_byte[i]=each_Nth_byte[i] + c[i]

# try each byte of key
for N in range(KEY_LEN):
    print "N=", N
    stat={}
    for i in range(256):
        tmp_key=chr(i)
	tmp=XOR_PCBC_step(tmp_key,each_Nth_byte[N], N)
        # count 0, FFs and 8Bs in decrypted buffer:
	important_bytes=tmp.count('\x00')+tmp.count('\xFF')+tmp.count('\x8B')
	stat[i]=important_bytes
    sorted_stat = sorted(stat.iteritems(), key=operator.itemgetter(1), reverse=True)
    print sorted_stat[0]
\end{lstlisting}

(Source code can be downloaded \href{\GitHubBlobMasterURL/examples/simple_exec_crypto/files/decrypt.py}{here}.)

I run it and here is a key for which 00/FF/8B bytes presence in decrypted buffer is maximal:

\begin{lstlisting}
N= 0
(147, 1224)
N= 1
(94, 1327)
N= 2
(252, 1223)
N= 3
(218, 1266)
N= 4
(38, 1209)
N= 5
(192, 1378)
N= 6
(199, 1204)
N= 7
(213, 1332)
N= 8
(225, 1251)
N= 9
(112, 1223)
N= 10
(143, 1177)
N= 11
(108, 1286)
N= 12
(10, 1164)
N= 13
(3, 1271)
N= 14
(128, 1253)
N= 15
(232, 1330)
\end{lstlisting}

Let's write decryption utility with the key we got:

\begin{lstlisting}[style=custompy]
#!/usr/bin/env python
import sys, hexdump, array

def xor_strings(s,t):
    # \verb|https://en.wikipedia.org/wiki/XOR_cipher#Example_implementation|
    """xor two strings together"""
    return "".join(chr(ord(a)^ord(b)) for a,b in zip(s,t))

IV=array.array('B', [147, 94, 252, 218, 38, 192, 199, 213, 225, 112, 143, 108, 10, 3, 128, 232]).tostring()

def chunks(l, n):
    n = max(1, n)
    return [l[i:i + n] for i in range(0, len(l), n)]

def read_file(fname):
    file=open(fname, mode='rb')
    content=file.read()
    file.close()
    return content

def decrypt_byte(i, k):
    return chr ((ord(i)-k) % 256)

def decrypt(buf):
    return "".join(decrypt_byte(buf[i], i) for i in range(16))

fout=open(sys.argv[2], mode='wb')

prev=IV
content=read_file(sys.argv[1])
tmp=chunks(content, 16)
for c in tmp:
    new_c=decrypt(c)
    p=xor_strings (new_c, prev)
    prev=xor_strings(c, p)
    fout.write(p)
fout.close()
\end{lstlisting}

(Source code can be downloaded \href{\GitHubBlobMasterURL/examples/simple_exec_crypto/files/decrypt2.py}{here}.)

Let's check resulting file:

\lstinputlisting{examples/simple_exec_crypto/objdump_result.txt}

Yes, this is seems correctly disassembled piece of x86 code.
The whole decryped file can be downloaded \href{\GitHubBlobMasterURL/examples/simple_exec_crypto/files/decrypted.bin}{here}.

In fact, this is text section from regedit.exe from Windows 7.
But this example is based on a real case I encountered, so just executable is different (and key), algorithm is the same.

\subsection{Other ideas to consider}

What if I would fail with such simple frequency analysis?
There are other ideas on how to measure correctness of decrypted/decompressed x86 code:

\begin{itemize}

\item Many modern compilers aligns functions on 0x10 border.
So the space left before is filled with NOPs (0x90) or other NOP instructions with known opcodes: \myref{sec:npad}.

\item Perhaps, the most frequent pattern in any assembly language is function call:\\
\TT{PUSH chain / CALL / ADD ESP, X}.
This sequence can easily detected and found.
I've even gathered statistics about average number of function arguments: \myref{args_stat}.
(Hence, this is average length of PUSH chain.)

\end{itemize}

Read more about incorrectly/correctly disassembled code: \myref{ISA_detect}.
}\RU{\subsection{Простое шифрование используя XOR-маску}
\label{XOR_mask_1}

Я нашел одну старую игру в стиле interactive fiction в архиве \emph{if-archive}\footnote{\url{http://www.ifarchive.org/}}:

\begin{lstlisting}
The New Castle v3.5 - Text/Adventure Game
in the style of the original Infocom (tm)
type games, Zork, Collosal Cave (Adventure),
etc.  Can you solve the mystery of the
abandoned castle?
Shareware from Software Customization.
Software Customization [ASP] Version 3.5 Feb. 2000
\end{lstlisting}

Можно скачать здесь: \url{\GitHubBlobMasterURL/ff/XOR/mask_1/files/newcastle.tgz}.

Там внутри есть файл (с названием \emph{castle.dbf}), который явно зашифрован, но не настоящим криптоалгоритмом,
и оне сжат, это что-то куда проще.
Я бы даже не стал измерять уровень энтропии (\myref{entropy}) этого файла, потому что я итак уверен, что он низкий.
Вот как он выглядит в Midnight Commander:

\begin{figure}[H]
\centering
\myincludegraphics{ff/XOR/mask_1/mc_encrypted.png}
\caption{Зашифрованный файл в Midnight Commander}
\end{figure}

Зашифрованный файл можно скачать здесь:
\url{\GitHubBlobMasterURL/ff/XOR/mask_1/files/castle.dbf.bz2}.

Можно ли расшифровать его без доступа к программе, используя просто этот файл?

Тут явно просматривается повторяющаяся строка. 
Если использовалось простое шифрование с XOR-маской, такие повторяющиеся строки это явное свидетельство,
потому что, вероятно, тут были длинные лакуны с нулевыми байтами, которые, в свою очередь, присутствуют
во мноигих исполняемых файлах, и в остальных бинарных файлах.

\myindex{UNIX!xxd}
Вот дам начала этого файла используя утилиту \emph{xxd} из UNIX:

\lstinputlisting{ff/XOR/mask_1/xxd_result.txt}

Давайте держаться за повторяющуюся строку \TT{iubgv}.
Глядя на этот дамп, мы можем легко увидеть, что период повторений этой строки это 0x51 или 81.
Вероятно, 81 это длина блока?
Длина файла 1658961, и она может быть поделена на 81 без остатка (и тогда там 20481 блоков).

Теперь я буду использовать Mathematica для анализа, есть ли тут повторяющиеся 81-байтные блоки в файле?
Я разделю входной файл на 81-байтные блоки и затем использую ф-цию
\emph{Tally[]}\footnote{\url{https://reference.wolfram.com/language/ref/Tally.html}}
которая просто считает, сколько раз каждый элемент встретился во входном списке.
Вывод Tally не отсортирован, так что я также добавлю ф-цию \emph{Sort[]} для сортировки его по кол-ву вхождений
в нисходящем порядке.

\begin{lstlisting}[style=custommath]
input = BinaryReadList["/home/dennis/.../castle.dbf"];

blocks = Partition[input, 81];

stat = Sort[Tally[blocks], #1[[2]] > #2[[2]] &]
\end{lstlisting}

И вот вывод:

\begin{lstlisting}[style=custommath]
{{{80, 103, 2, 116, 113, 102, 118, 25, 99, 8, 19, 23, 116, 125, 107, 
   25, 99, 109, 114, 102, 14, 121, 115, 31, 9, 117, 113, 111, 5, 4, 
   127, 28, 122, 101, 8, 110, 14, 18, 124, 106, 16, 20, 104, 119, 8, 
   109, 26, 106, 9, 97, 13, 99, 15, 119, 20, 105, 117, 98, 103, 118, 
   1, 126, 29, 97, 122, 17, 15, 114, 110, 3, 5, 125, 125, 99, 126, 
   119, 102, 30, 122, 2, 117}, 1739}, 
{{80, 100, 2, 116, 113, 102, 118, 25, 99, 8, 19, 23, 116, 
   125, 107, 25, 99, 109, 114, 102, 14, 121, 115, 31, 9, 117, 113, 
   111, 5, 4, 127, 28, 122, 101, 8, 110, 14, 18, 124, 106, 16, 20, 
   104, 119, 8, 109, 26, 106, 9, 97, 13, 99, 15, 119, 20, 105, 117, 
   98, 103, 118, 1, 126, 29, 97, 122, 17, 15, 114, 110, 3, 5, 125, 
   125, 99, 126, 119, 102, 30, 122, 2, 117}, 1422}, 
{{80, 101, 2, 116, 113, 102, 118, 25, 99, 8, 19, 23, 116, 
   125, 107, 25, 99, 109, 114, 102, 14, 121, 115, 31, 9, 117, 113, 
   111, 5, 4, 127, 28, 122, 101, 8, 110, 14, 18, 124, 106, 16, 20, 
   104, 119, 8, 109, 26, 106, 9, 97, 13, 99, 15, 119, 20, 105, 117, 
   98, 103, 118, 1, 126, 29, 97, 122, 17, 15, 114, 110, 3, 5, 125, 
   125, 99, 126, 119, 102, 30, 122, 2, 117}, 1012},
{{80, 120, 2, 116, 113, 102, 118, 25, 99, 8, 19, 23, 116, 
   125, 107, 25, 99, 109, 114, 102, 14, 121, 115, 31, 9, 117, 113, 
   111, 5, 4, 127, 28, 122, 101, 8, 110, 14, 18, 124, 106, 16, 20, 
   104, 119, 8, 109, 26, 106, 9, 97, 13, 99, 15, 119, 20, 105, 117, 
   98, 103, 118, 1, 126, 29, 97, 122, 17, 15, 114, 110, 3, 5, 125, 
   125, 99, 126, 119, 102, 30, 122, 2, 117}, 377},

...

{{80, 2, 74, 49, 113, 21, 62, 88, 39, 71, 68, 23, 63, 51, 36, 78, 48, 
   108, 114, 102, 14, 121, 115, 31, 9, 117, 113, 111, 5, 4, 127, 28, 
   122, 101, 8, 110, 14, 18, 124, 106, 16, 20, 104, 119, 8, 109, 26, 
   106, 9, 97, 13, 99, 15, 119, 20, 105, 117, 98, 103, 118, 1, 126, 
   29, 97, 122, 17, 15, 114, 110, 3, 5, 125, 125, 99, 126, 119, 102, 
   30, 122, 2, 117}, 1},
{{80, 1, 74, 59, 113, 45, 56, 86, 52, 91, 19, 64, 60, 60, 63, 
   25, 38, 59, 59, 42, 14, 53, 38, 77, 66, 38, 113, 38, 75, 4, 43, 84,
    63, 101, 64, 43, 79, 64, 40, 57, 16, 91, 46, 119, 69, 40, 84, 117,
    9, 97, 13, 99, 15, 119, 20, 105, 117, 98, 103, 118, 1, 126, 29, 
   97, 122, 17, 15, 114, 110, 3, 5, 125, 125, 99, 126, 119, 102, 30, 
   122, 2, 117}, 1},
{{80, 2, 74, 49, 113, 49, 51, 92, 39, 8, 92, 81, 116, 62, 57, 
   80, 46, 40, 114, 36, 75, 56, 33, 76, 9, 55, 56, 59, 81, 65, 45, 28,
    60, 55, 93, 39, 90, 28, 124, 106, 16, 20, 104, 119, 8, 109, 26, 
   106, 9, 97, 13, 99, 15, 119, 20, 105, 117, 98, 103, 118, 1, 126, 
   29, 97, 122, 17, 15, 114, 110, 3, 5, 125, 125, 99, 126, 119, 102, 
   30, 122, 2, 117}, 1}}
\end{lstlisting}

Вывод Tally это список пар, каждая пара это 81-байтный блок и количество раз, сколько он встретился в файле.
Мы видим, что наиболее частно встречающийся блок это первый, он встретился 1739 раз.
Второй встретился 1422 раза. Есть и другие: 1012 раза, 377 раз, итд.
81-байтные блоки, встреченные лишь один раз, находятся в конце вывода.

Попробуем сравнить эти блоки. Первый и второй.
Есть ли в Mathematica ф-ция для сравнения списков/массивов?
Наверняка есть, но в педагогических целях, я буду использоват операцию XOR для сравнения.
Действительно: если байты во входных массивах равны друг другу, результат операции XOR это 0.
Если не равны, результат будет ненулевой.

Сравним первый блок (встречается 1739 раз) и второй (встречается 1422 раз):

\begin{lstlisting}[style=custommath]
In[]:= BitXor[stat[[1]][[1]], stat[[2]][[1]]]
Out[]= {0, 3, 0, 0, 0, 0, 0, 0, 0, 0, 0, 0, 0, 0, 0, 0, 0, 0, 0, \
0, 0, 0, 0, 0, 0, 0, 0, 0, 0, 0, 0, 0, 0, 0, 0, 0, 0, 0, 0, 0, 0, 0, \
0, 0, 0, 0, 0, 0, 0, 0, 0, 0, 0, 0, 0, 0, 0, 0, 0, 0, 0, 0, 0, 0, 0, \
0, 0, 0, 0, 0, 0, 0, 0, 0, 0, 0, 0, 0, 0, 0, 0}
\end{lstlisting}

Они отличаются только вторым байтом.

Сравним второй блок (встречается 1422 раза) и третий (встречается 1012 раз):

\begin{lstlisting}[style=custommath]
In[]:= BitXor[stat[[2]][[1]], stat[[3]][[1]]]
Out[]= {0, 1, 0, 0, 0, 0, 0, 0, 0, 0, 0, 0, 0, 0, 0, 0, 0, 0, 0, \
0, 0, 0, 0, 0, 0, 0, 0, 0, 0, 0, 0, 0, 0, 0, 0, 0, 0, 0, 0, 0, 0, 0, \
0, 0, 0, 0, 0, 0, 0, 0, 0, 0, 0, 0, 0, 0, 0, 0, 0, 0, 0, 0, 0, 0, 0, \
0, 0, 0, 0, 0, 0, 0, 0, 0, 0, 0, 0, 0, 0, 0, 0}
\end{lstlisting}

Они тоже отличаются только вторым байтом.

Так или иначе, попробуем использовать самый встречающийся блок как XOR-ключ и попробуем расшифровать первые 4 81-байтных
блока в файле:

\begin{lstlisting}[style=custommath]
In[]:= key = stat[[1]][[1]]
Out[]= {80, 103, 2, 116, 113, 102, 118, 25, 99, 8, 19, 23, 116, \
125, 107, 25, 99, 109, 114, 102, 14, 121, 115, 31, 9, 117, 113, 111, \
5, 4, 127, 28, 122, 101, 8, 110, 14, 18, 124, 106, 16, 20, 104, 119, \
8, 109, 26, 106, 9, 97, 13, 99, 15, 119, 20, 105, 117, 98, 103, 118, \
1, 126, 29, 97, 122, 17, 15, 114, 110, 3, 5, 125, 125, 99, 126, 119, \
102, 30, 122, 2, 117}

In[]:= ToASCII[val_] := If[val == 0, " ", FromCharacterCode[val, "PrintableASCII"]]

In[]:= DecryptBlockASCII[blk_] := Map[ToASCII[#] &, BitXor[key, blk]]

In[]:= DecryptBlockASCII[blocks[[1]]]
Out[]= {" ", " ", " ", " ", " ", " ", " ", " ", " ", " ", " ", " \
", " ", " ", " ", " ", " ", " ", " ", " ", " ", " ", " ", " ", " ", " \
", " ", " ", " ", " ", " ", " ", " ", " ", " ", " ", " ", " ", " ", " \
", " ", " ", " ", " ", " ", " ", " ", " ", " ", " ", " ", " ", " ", " \
", " ", " ", " ", " ", " ", " ", " ", " ", " ", " ", " ", " ", " ", " \
", " ", " ", " ", " ", " ", " ", " ", " ", " ", " ", " ", " ", " "}

In[]:= DecryptBlockASCII[blocks[[2]]]
Out[]= {" ", "e", "H", "E", " ", "W", "E", "E", "D", " ", "O", \
"F", " ", "C", "R", "I", "M", "E", " ", "B", "E", "A", "R", "S", " ", \
"B", "I", "T", "T", "E", "R", " ", "F", "R", "U", "I", "T", "?", \
" ", " ", " ", " ", " ", " ", " ", " ", " ", " ", " ", " ", " ", " ", \
" ", " ", " ", " ", " ", " ", " ", " ", " ", " ", " ", " ", " ", " ", \
" ", " ", " ", " ", " ", " ", " ", " ", " ", " ", " ", " ", " ", " ", \
" "}

In[]:= DecryptBlockASCII[blocks[[3]]]
Out[]= {" ", "?", " ", " ", " ", " ", " ", " ", " ", " ", " \
", " ", " ", " ", " ", " ", " ", " ", " ", " ", " ", " ", " ", " ", " \
", " ", " ", " ", " ", " ", " ", " ", " ", " ", " ", " ", " ", " ", " \
", " ", " ", " ", " ", " ", " ", " ", " ", " ", " ", " ", " ", " ", " \
", " ", " ", " ", " ", " ", " ", " ", " ", " ", " ", " ", " ", " ", " \
", " ", " ", " ", " ", " ", " ", " ", " ", " ", " ", " ", " ", " ", " \
"}

In[]:= DecryptBlockASCII[blocks[[4]]]
Out[]= {" ", "f", "H", "O", " ", "K", "N", "O", "W", "S", " ", \
"W", "H", "A", "T", " ", "E", "V", "I", "L", " ", "L", "U", "R", "K", \
"S", " ", "I", "N", " ", "T", "H", "E", " ", "H", "E", "A", "R", "T", \
"S", " ", "O", "F", " ", "M", "E", "N", "?", " ", " ", " ", " ", \
" ", " ", " ", " ", " ", " ", " ", " ", " ", " ", " ", " ", " ", " ", \
" ", " ", " ", " ", " ", " ", " ", " ", " ", " ", " ", " ", " ", " ", \
" "}
\end{lstlisting}

(Я заменил непечатаемые символы на \q{?}.)

Мы видим что первый и третий блоки пустые (или почти пустые),
но второй и четвертый имеют ясно различимые английские слова/фразы.
Похоже что наше предположение насчет ключа верно (как минимум частично).
Это означает, что самый встречающийся 81-байтный блок в файле находится в местах лакун с нулевыми байтами
или что-то в этом роде.

Попробуем расшифровать весь файл:

\begin{lstlisting}[style=custommath]
DecryptBlock[blk_] := BitXor[key, blk]

decrypted = Map[DecryptBlock[#] &, blocks];

BinaryWrite["/home/dennis/.../tmp", Flatten[decrypted]]

Close["/home/dennis/.../tmp"]
\end{lstlisting}

\begin{figure}[H]
\centering
\myincludegraphics{ff/XOR/mask_1/mc_decrypted1.png}
\caption{Расшифрованный файл в Midnight Commander, первая попытка}
\end{figure}

Выглядит как английские фразы для какой-то игры, но что-то не так.
Прежде всего, регистр инвертирован: фразы и некоторые слова начинаются со строчных букв,
в то время как остальные буквы заглавные.
Также, некоторые фразы начинаются с не тех букв.
Посмотрите на самую первую фразу: \q{eHE WEED OF CRIME BEARS BITTER FRUIT}.
Что такое \q{eHE}? Разве не \q{tHE} тут должно быть?
Возможно ли что наш ключ для дешифрования имеет неверный байт в этом месте?

Посмотрим снова на второй блок в файле, на ключ и на результат дешифрования:

\begin{lstlisting}[style=custommath]
In[]:= blocks[[2]]
Out[]= {80, 2, 74, 49, 113, 49, 51, 92, 39, 8, 92, 81, 116, 62, \
57, 80, 46, 40, 114, 36, 75, 56, 33, 76, 9, 55, 56, 59, 81, 65, 45, \
28, 60, 55, 93, 39, 90, 28, 124, 106, 16, 20, 104, 119, 8, 109, 26, \
106, 9, 97, 13, 99, 15, 119, 20, 105, 117, 98, 103, 118, 1, 126, 29, \
97, 122, 17, 15, 114, 110, 3, 5, 125, 125, 99, 126, 119, 102, 30, \
122, 2, 117}

In[]:= key
Out[]= {80, 103, 2, 116, 113, 102, 118, 25, 99, 8, 19, 23, 116, \
125, 107, 25, 99, 109, 114, 102, 14, 121, 115, 31, 9, 117, 113, 111, \
5, 4, 127, 28, 122, 101, 8, 110, 14, 18, 124, 106, 16, 20, 104, 119, \
8, 109, 26, 106, 9, 97, 13, 99, 15, 119, 20, 105, 117, 98, 103, 118, \
1, 126, 29, 97, 122, 17, 15, 114, 110, 3, 5, 125, 125, 99, 126, 119, \
102, 30, 122, 2, 117}

In[]:= BitXor[key, blocks[[2]]]
Out[]= {0, 101, 72, 69, 0, 87, 69, 69, 68, 0, 79, 70, 0, 67, 82, \
73, 77, 69, 0, 66, 69, 65, 82, 83, 0, 66, 73, 84, 84, 69, 82, 0, 70, \
82, 85, 73, 84, 14, 0, 0, 0, 0, 0, 0, 0, 0, 0, 0, 0, 0, 0, 0, 0, 0, \
0, 0, 0, 0, 0, 0, 0, 0, 0, 0, 0, 0, 0, 0, 0, 0, 0, 0, 0, 0, 0, 0, 0, \
0, 0, 0, 0}
\end{lstlisting}

Зашифрованный байт это 2, байт из ключа это 103, $2 \oplus 103=101$ и 101 это ASCII-код символа \q{e}.
Чему должен равнятся этот байт ключа, чтобы ASCII-код был 116 (для символа  \q{t})?
$2 \oplus 116=118$, присвоим 118 второму байту в ключе \dots

\begin{lstlisting}[style=custommath]
key = {80, 118, 2, 116, 113, 102, 118, 25, 99, 8, 19, 23, 116, 125, 
  107, 25, 99, 109, 114, 102, 14, 121, 115, 31, 9, 117, 113, 111, 5, 
  4, 127, 28, 122, 101, 8, 110, 14, 18, 124, 106, 16, 20, 104, 119, 8,
   109, 26, 106, 9, 97, 13, 99, 15, 119, 20, 105, 117, 98, 103, 118, 
  1, 126, 29, 97, 122, 17, 15, 114, 110, 3, 5, 125, 125, 99, 126, 119,
   102, 30, 122, 2, 117}
\end{lstlisting}

\dots и снова дешифруем весь файл.

\begin{figure}[H]
\centering
\myincludegraphics{ff/XOR/mask_1/mc_decrypted2.png}
\caption{Дешифрованный файл в Midnight Commander, вторая попытка}
\end{figure}

Ух ты, теперь грамматика корректна, и все фразы начинаются с корректных букв.
Но все таки, регистр подозрителен.
С чего бы разработчику игры записывать их в такой манере?
Может быть наш ключ все еще неправилен?

% TODO ASCII table somewhere in the book
Изучая таблицу ASCII мы можем заметить что ASCII-коды для букв в верхнем и нижнем регистре отличаются только на один бит
(6-й бит, если считать с первого, 0b100000):

\begin{figure}[H]
\centering
\includegraphics[width=0.7\textwidth]{ascii.png}
\caption{7-битная таблица \ac{ASCII} в Emacs}
\end{figure}

6-й бит, выставленный в нулевом байте, В десятичном виде это будет 32.
Но 32 это ASCII-код пробела!

Действительно, можно менять регистр просто применяя XOR к ASCII-коду, с 32 (больше об этом: \myref{toupper_bit}).

Возможно ли, что пустые лакуны в файле это не нулевые байты, а скорее содержащие пробелы?
Еще раз модифицируем наш XOR-ключ (я про-XOR-ю каждый байт ключа с 32):

\begin{lstlisting}[style=custommath]
(* "32" это скаляр, и "key" это вектор, но это OK *)

In[]:= key3 = BitXor[32, key]
Out[]= {112, 86, 34, 84, 81, 70, 86, 57, 67, 40, 51, 55, 84, 93, 75, \
57, 67, 77, 82, 70, 46, 89, 83, 63, 41, 85, 81, 79, 37, 36, 95, 60, \
90, 69, 40, 78, 46, 50, 92, 74, 48, 52, 72, 87, 40, 77, 58, 74, 41, \
65, 45, 67, 47, 87, 52, 73, 85, 66, 71, 86, 33, 94, 61, 65, 90, 49, \
47, 82, 78, 35, 37, 93, 93, 67, 94, 87, 70, 62, 90, 34, 85}

In[]:= DecryptBlock[blk_] := BitXor[key3, blk]
\end{lstlisting}

И снова дешифруем входной файл:

\begin{figure}[H]
\centering
\myincludegraphics{ff/XOR/mask_1/mc_decrypted.png}
\caption{Дешифрованный файл в Midnight Commander, последняя попытка}
\end{figure}

(Расшифрованный файл доступен здесь:
\url{\GitHubBlobMasterURL/ff/XOR/mask_1/files/decrypted.dat.bz2}.)

Несомненно, это корректный исходный файл.
Да, и мы видим числа в начале каждого блока. Должно быть это и есть источник некорректного XOR-ключа.
Как выходит, самый встречающийся 81-байтный блок в файле это блок заполненный пробелами и содержащий символ \q{1} на месте
второго байта.
Действительно, как-то так получилось что многие блоки здесь перемежаются с этим блоком.
Может быть это что-то вроде выравнивания (padding) для коротких фраз/сообщений?
Другой часто встречающийся 81-байтный блок также заполнен пробелами, но с другой цифрой, следовательно,
они отличаются только вторым байтом.

Вот и всё! Теперь мы можем написать утилиту для зашифрования файла назад, и, может быть, модифицировать его перед этим

Файл для Mathematica можно скачать здесь:\\
\url{\GitHubBlobMasterURL/ff/XOR/mask_1/files/XOR_mask_1.nb}.

Итог: XOR-шифрование не надежно вообще. Вероятно, разработчик игры хотел просто скрыть внутренности игры от игрока,
ничего более серьезного.
Все же, шифрование вроде этого крайне популярно вследствии его простоты, так что многие реверс инженеры обычно хорошо
с этим знакомы.

}\FR{\mysection{Fonction presque vide}
\label{Boolector}
\myindex{Boolector}
\myindex{x86!\Instructions!JMP}

Ceci est un morceau de code réel que j'ai trouvé dans Boolector\footnote{\url{https://boolector.github.io/}}:

\lstinputlisting[style=customc]{patterns/025_almost_empty/boolectormain.c}

Pourquoi quelqu'un ferait-il comme ça?
Je ne sais pas mais mon hypothèse est que \verb|boolector_main()| peut être compilée
dans une sorte de DLL ou bibliothèque dynamique, et appelée depuis une suite de test.
Certainement qu'une suite de test peut préparer les variables argc/argv comme
le ferait \ac{CRT}.

Il est intéressant de voir comment c'est compilé:

\lstinputlisting[caption=GCC 8.2 x64 \NonOptimizing (\assemblyOutput),style=customasmx86]{patterns/025_almost_empty/boolectormain_O0.s}

Ceci est OK, le prologue (non optimisé) déplace inutilement deux arguments,
\INS{CALL}, épilogue, \INS{RET}.
Mais regardons la version optimisée:

\lstinputlisting[caption=GCC 8.2 x64 \Optimizing (\assemblyOutput),style=customasmx86]{patterns/025_almost_empty/boolectormain_O3.s}

Aussi simple que ça: la pile et les registres ne sont pas touchés et \verb|boolector_main()|
a le même ensemble d'arguments.
Donc, tout ce que nous avons à faire est de passer l'exécution à une autre adresse.

Ceci est proche d'une \glslink{thunk function}{fonction thunk}.

Nous verons queelque chose de plus avancé plus tard: \myref{ARM_B_to_printf}, \myref{JMP_instead_of_RET}.
}

\renewcommand{\CURPATH}{advanced/310_obfuscation}
\EN{% TODO translate
\mysection{Breaking simple executable cryptor}

I've got an executable file which is encrypted by relatively simple encryption.
\href{\GitHubBlobMasterURL/examples/simple_exec_crypto/files/cipher.bin}{Here is it} (only executable section is left here).

First, all encryption function does is just adds number of position in buffer to the byte.
Here is how this can be encoded in Python:

\begin{lstlisting}[caption=Python script,style=custompy]
#!/usr/bin/env python
def e(i, k):
    return chr ((ord(i)+k) % 256)

def encrypt(buf):
    return e(buf[0], 0)+ e(buf[1], 1)+ e(buf[2], 2) + e(buf[3], 3)+ e(buf[4], 4)+ e(buf[5], 5)+ e(buf[6], 6)+ e(buf[7], 7)+
           e(buf[8], 8)+ e(buf[9], 9)+ e(buf[10], 10)+ e(buf[11], 11)+ e(buf[12], 12)+ e(buf[13], 13)+ e(buf[14], 14)+ e(buf[15], 15)
\end{lstlisting}

Hence, if you encrypt buffer with 16 zeros, you'll get \emph{0, 1, 2, 3 ... 12, 13, 14, 15}.

\myindex{Propagating Cipher Block Chaining}
Propagating Cipher Block Chaining (PCBC) is also used, here is how it works:

\begin{figure}[H]
\centering
\myincludegraphics{examples/simple_exec_crypto/601px-PCBC_encryption.png}
\caption{Propagating Cipher Block Chaining encryption (image is taken from Wikipedia article)}
\end{figure}

The problem is that it's too boring to recover IV (Initialization Vector) each time.
Brute-force is also not an option, because IV is too long (16 bytes).
Let's see, if it's possible to recover IV for arbitrary encrypted executable file?

Let's try simple frequency analysis.
This is 32-bit x86 executable code, so let's gather statistics about most frequent bytes and opcodes.
I tried huge oracle.exe file from Oracle RDBMS version 11.2 for windows x86 and I've found that the most frequent byte (no surprise) is zero (~10\%).
The next most frequent byte is (again, no surprise) 0xFF (~5\%).
The next is 0x8B (~5\%).

\myindex{x86!\Instructions!MOV}
0x8B is opcode for \INS{MOV}, this is indeed one of the most busy x86 instructions.
Now what about popularity of zero byte?
If compiler needs to encode value bigger than 127, it has to use 32-bit displacement instead of 8-bit one, but large values are very rare,
so it is padded by zeros.
\myindex{x86!\Instructions!LEA}
\myindex{x86!\Instructions!PUSH}
\myindex{x86!\Instructions!CALL}
This is at least in \INS{LEA}, \INS{MOV}, \INS{PUSH}, \INS{CALL}.

For example:

\begin{lstlisting}[style=customasmx86]
8D B0 28 01 00 00                 lea     esi, [eax+128h]
8D BF 40 38 00 00                 lea     edi, [edi+3840h]
\end{lstlisting}

Displacements bigger than 127 are very popular, but they are rarely exceeds 0x10000
(indeed, such large memory buffers/structures are also rare).

Same story with \INS{MOV}, large constants are rare, the most heavily used are 0, 1, 10, 100, $2^n$, and so on.
Compiler has to pad small constants by zeros to represent them as 32-bit values:

\begin{lstlisting}[style=customasmx86]
BF 02 00 00 00                    mov     edi, 2
BF 01 00 00 00                    mov     edi, 1
\end{lstlisting}

Now about 00 and FF bytes combined: jumps (including conditional) and calls can pass execution flow forward or backwards, but very often,
within the limits of the current executable module.
If forward, displacement is not very big and also padded with zeros.
If backwards, displacement is represented as negative value, so padded with FF bytes.
For example, transfer execution flow forward:

\begin{lstlisting}[style=customasmx86]
E8 43 0C 00 00                    call    _function1
E8 5C 00 00 00                    call    _function2
0F 84 F0 0A 00 00                 jz      loc_4F09A0
0F 84 EB 00 00 00                 jz      loc_4EFBB8
\end{lstlisting}

Backwards:

\begin{lstlisting}[style=customasmx86]
E8 79 0C FE FF                    call    _function1
E8 F4 16 FF FF                    call    _function2
0F 84 F8 FB FF FF                 jz      loc_8212BC
0F 84 06 FD FF FF                 jz      loc_FF1E7D
\end{lstlisting}

FF byte is also very often occurred in negative displacements like these:

\begin{lstlisting}[style=customasmx86]
8D 85 1E FF FF FF                 lea     eax, [ebp-0E2h]
8D 95 F8 5C FF FF                 lea     edx, [ebp-0A308h]
\end{lstlisting}

So far so good. Now we have to try various 16-byte keys, decrypt executable section and measure how often 00, FF and 8B bytes are occurred.
Let's also keep in sight how PCBC decryption works:

\begin{figure}[H]
\centering
\myincludegraphics{examples/simple_exec_crypto/640px-PCBC_decryption.png}
\caption{Propagating Cipher Block Chaining decryption (image is taken from Wikipedia article)}
\end{figure}

The good news is that we don't really have to decrypt whole piece of data, but only slice by slice, this is exactly how I did in my previous example: \myref{XOR_mask_2}.

Now I'm trying all possible bytes (0..255) for each byte in key and just pick the byte producing maximal amount of 00/FF/8B bytes in a decrypted slice:

\begin{lstlisting}[style=custompy]
#!/usr/bin/env python
import sys, hexdump, array, string, operator

KEY_LEN=16

def chunks(l, n):
    # split n by l-byte chunks
    # https://stackoverflow.com/q/312443
    n = max(1, n)
    return [l[i:i + n] for i in range(0, len(l), n)]

def read_file(fname):
    file=open(fname, mode='rb')
    content=file.read()
    file.close()
    return content

def decrypt_byte (c, key):
    return chr((ord(c)-key) % 256)

def XOR_PCBC_step (IV, buf, k):
    prev=IV
    rt=""
    for c in buf:
	new_c=decrypt_byte(c, k)
        plain=chr(ord(new_c)^ord(prev))
	prev=chr(ord(c)^ord(plain))
	rt=rt+plain
    return rt

each_Nth_byte=[""]*KEY_LEN

content=read_file(sys.argv[1])
# split input by 16-byte chunks:
all_chunks=chunks(content, KEY_LEN)
for c in all_chunks:
    for i in range(KEY_LEN):
        each_Nth_byte[i]=each_Nth_byte[i] + c[i]

# try each byte of key
for N in range(KEY_LEN):
    print "N=", N
    stat={}
    for i in range(256):
        tmp_key=chr(i)
	tmp=XOR_PCBC_step(tmp_key,each_Nth_byte[N], N)
        # count 0, FFs and 8Bs in decrypted buffer:
	important_bytes=tmp.count('\x00')+tmp.count('\xFF')+tmp.count('\x8B')
	stat[i]=important_bytes
    sorted_stat = sorted(stat.iteritems(), key=operator.itemgetter(1), reverse=True)
    print sorted_stat[0]
\end{lstlisting}

(Source code can be downloaded \href{\GitHubBlobMasterURL/examples/simple_exec_crypto/files/decrypt.py}{here}.)

I run it and here is a key for which 00/FF/8B bytes presence in decrypted buffer is maximal:

\begin{lstlisting}
N= 0
(147, 1224)
N= 1
(94, 1327)
N= 2
(252, 1223)
N= 3
(218, 1266)
N= 4
(38, 1209)
N= 5
(192, 1378)
N= 6
(199, 1204)
N= 7
(213, 1332)
N= 8
(225, 1251)
N= 9
(112, 1223)
N= 10
(143, 1177)
N= 11
(108, 1286)
N= 12
(10, 1164)
N= 13
(3, 1271)
N= 14
(128, 1253)
N= 15
(232, 1330)
\end{lstlisting}

Let's write decryption utility with the key we got:

\begin{lstlisting}[style=custompy]
#!/usr/bin/env python
import sys, hexdump, array

def xor_strings(s,t):
    # \verb|https://en.wikipedia.org/wiki/XOR_cipher#Example_implementation|
    """xor two strings together"""
    return "".join(chr(ord(a)^ord(b)) for a,b in zip(s,t))

IV=array.array('B', [147, 94, 252, 218, 38, 192, 199, 213, 225, 112, 143, 108, 10, 3, 128, 232]).tostring()

def chunks(l, n):
    n = max(1, n)
    return [l[i:i + n] for i in range(0, len(l), n)]

def read_file(fname):
    file=open(fname, mode='rb')
    content=file.read()
    file.close()
    return content

def decrypt_byte(i, k):
    return chr ((ord(i)-k) % 256)

def decrypt(buf):
    return "".join(decrypt_byte(buf[i], i) for i in range(16))

fout=open(sys.argv[2], mode='wb')

prev=IV
content=read_file(sys.argv[1])
tmp=chunks(content, 16)
for c in tmp:
    new_c=decrypt(c)
    p=xor_strings (new_c, prev)
    prev=xor_strings(c, p)
    fout.write(p)
fout.close()
\end{lstlisting}

(Source code can be downloaded \href{\GitHubBlobMasterURL/examples/simple_exec_crypto/files/decrypt2.py}{here}.)

Let's check resulting file:

\lstinputlisting{examples/simple_exec_crypto/objdump_result.txt}

Yes, this is seems correctly disassembled piece of x86 code.
The whole decryped file can be downloaded \href{\GitHubBlobMasterURL/examples/simple_exec_crypto/files/decrypted.bin}{here}.

In fact, this is text section from regedit.exe from Windows 7.
But this example is based on a real case I encountered, so just executable is different (and key), algorithm is the same.

\subsection{Other ideas to consider}

What if I would fail with such simple frequency analysis?
There are other ideas on how to measure correctness of decrypted/decompressed x86 code:

\begin{itemize}

\item Many modern compilers aligns functions on 0x10 border.
So the space left before is filled with NOPs (0x90) or other NOP instructions with known opcodes: \myref{sec:npad}.

\item Perhaps, the most frequent pattern in any assembly language is function call:\\
\TT{PUSH chain / CALL / ADD ESP, X}.
This sequence can easily detected and found.
I've even gathered statistics about average number of function arguments: \myref{args_stat}.
(Hence, this is average length of PUSH chain.)

\end{itemize}

Read more about incorrectly/correctly disassembled code: \myref{ISA_detect}.
}\RU{\subsection{Простое шифрование используя XOR-маску}
\label{XOR_mask_1}

Я нашел одну старую игру в стиле interactive fiction в архиве \emph{if-archive}\footnote{\url{http://www.ifarchive.org/}}:

\begin{lstlisting}
The New Castle v3.5 - Text/Adventure Game
in the style of the original Infocom (tm)
type games, Zork, Collosal Cave (Adventure),
etc.  Can you solve the mystery of the
abandoned castle?
Shareware from Software Customization.
Software Customization [ASP] Version 3.5 Feb. 2000
\end{lstlisting}

Можно скачать здесь: \url{\GitHubBlobMasterURL/ff/XOR/mask_1/files/newcastle.tgz}.

Там внутри есть файл (с названием \emph{castle.dbf}), который явно зашифрован, но не настоящим криптоалгоритмом,
и оне сжат, это что-то куда проще.
Я бы даже не стал измерять уровень энтропии (\myref{entropy}) этого файла, потому что я итак уверен, что он низкий.
Вот как он выглядит в Midnight Commander:

\begin{figure}[H]
\centering
\myincludegraphics{ff/XOR/mask_1/mc_encrypted.png}
\caption{Зашифрованный файл в Midnight Commander}
\end{figure}

Зашифрованный файл можно скачать здесь:
\url{\GitHubBlobMasterURL/ff/XOR/mask_1/files/castle.dbf.bz2}.

Можно ли расшифровать его без доступа к программе, используя просто этот файл?

Тут явно просматривается повторяющаяся строка. 
Если использовалось простое шифрование с XOR-маской, такие повторяющиеся строки это явное свидетельство,
потому что, вероятно, тут были длинные лакуны с нулевыми байтами, которые, в свою очередь, присутствуют
во мноигих исполняемых файлах, и в остальных бинарных файлах.

\myindex{UNIX!xxd}
Вот дам начала этого файла используя утилиту \emph{xxd} из UNIX:

\lstinputlisting{ff/XOR/mask_1/xxd_result.txt}

Давайте держаться за повторяющуюся строку \TT{iubgv}.
Глядя на этот дамп, мы можем легко увидеть, что период повторений этой строки это 0x51 или 81.
Вероятно, 81 это длина блока?
Длина файла 1658961, и она может быть поделена на 81 без остатка (и тогда там 20481 блоков).

Теперь я буду использовать Mathematica для анализа, есть ли тут повторяющиеся 81-байтные блоки в файле?
Я разделю входной файл на 81-байтные блоки и затем использую ф-цию
\emph{Tally[]}\footnote{\url{https://reference.wolfram.com/language/ref/Tally.html}}
которая просто считает, сколько раз каждый элемент встретился во входном списке.
Вывод Tally не отсортирован, так что я также добавлю ф-цию \emph{Sort[]} для сортировки его по кол-ву вхождений
в нисходящем порядке.

\begin{lstlisting}[style=custommath]
input = BinaryReadList["/home/dennis/.../castle.dbf"];

blocks = Partition[input, 81];

stat = Sort[Tally[blocks], #1[[2]] > #2[[2]] &]
\end{lstlisting}

И вот вывод:

\begin{lstlisting}[style=custommath]
{{{80, 103, 2, 116, 113, 102, 118, 25, 99, 8, 19, 23, 116, 125, 107, 
   25, 99, 109, 114, 102, 14, 121, 115, 31, 9, 117, 113, 111, 5, 4, 
   127, 28, 122, 101, 8, 110, 14, 18, 124, 106, 16, 20, 104, 119, 8, 
   109, 26, 106, 9, 97, 13, 99, 15, 119, 20, 105, 117, 98, 103, 118, 
   1, 126, 29, 97, 122, 17, 15, 114, 110, 3, 5, 125, 125, 99, 126, 
   119, 102, 30, 122, 2, 117}, 1739}, 
{{80, 100, 2, 116, 113, 102, 118, 25, 99, 8, 19, 23, 116, 
   125, 107, 25, 99, 109, 114, 102, 14, 121, 115, 31, 9, 117, 113, 
   111, 5, 4, 127, 28, 122, 101, 8, 110, 14, 18, 124, 106, 16, 20, 
   104, 119, 8, 109, 26, 106, 9, 97, 13, 99, 15, 119, 20, 105, 117, 
   98, 103, 118, 1, 126, 29, 97, 122, 17, 15, 114, 110, 3, 5, 125, 
   125, 99, 126, 119, 102, 30, 122, 2, 117}, 1422}, 
{{80, 101, 2, 116, 113, 102, 118, 25, 99, 8, 19, 23, 116, 
   125, 107, 25, 99, 109, 114, 102, 14, 121, 115, 31, 9, 117, 113, 
   111, 5, 4, 127, 28, 122, 101, 8, 110, 14, 18, 124, 106, 16, 20, 
   104, 119, 8, 109, 26, 106, 9, 97, 13, 99, 15, 119, 20, 105, 117, 
   98, 103, 118, 1, 126, 29, 97, 122, 17, 15, 114, 110, 3, 5, 125, 
   125, 99, 126, 119, 102, 30, 122, 2, 117}, 1012},
{{80, 120, 2, 116, 113, 102, 118, 25, 99, 8, 19, 23, 116, 
   125, 107, 25, 99, 109, 114, 102, 14, 121, 115, 31, 9, 117, 113, 
   111, 5, 4, 127, 28, 122, 101, 8, 110, 14, 18, 124, 106, 16, 20, 
   104, 119, 8, 109, 26, 106, 9, 97, 13, 99, 15, 119, 20, 105, 117, 
   98, 103, 118, 1, 126, 29, 97, 122, 17, 15, 114, 110, 3, 5, 125, 
   125, 99, 126, 119, 102, 30, 122, 2, 117}, 377},

...

{{80, 2, 74, 49, 113, 21, 62, 88, 39, 71, 68, 23, 63, 51, 36, 78, 48, 
   108, 114, 102, 14, 121, 115, 31, 9, 117, 113, 111, 5, 4, 127, 28, 
   122, 101, 8, 110, 14, 18, 124, 106, 16, 20, 104, 119, 8, 109, 26, 
   106, 9, 97, 13, 99, 15, 119, 20, 105, 117, 98, 103, 118, 1, 126, 
   29, 97, 122, 17, 15, 114, 110, 3, 5, 125, 125, 99, 126, 119, 102, 
   30, 122, 2, 117}, 1},
{{80, 1, 74, 59, 113, 45, 56, 86, 52, 91, 19, 64, 60, 60, 63, 
   25, 38, 59, 59, 42, 14, 53, 38, 77, 66, 38, 113, 38, 75, 4, 43, 84,
    63, 101, 64, 43, 79, 64, 40, 57, 16, 91, 46, 119, 69, 40, 84, 117,
    9, 97, 13, 99, 15, 119, 20, 105, 117, 98, 103, 118, 1, 126, 29, 
   97, 122, 17, 15, 114, 110, 3, 5, 125, 125, 99, 126, 119, 102, 30, 
   122, 2, 117}, 1},
{{80, 2, 74, 49, 113, 49, 51, 92, 39, 8, 92, 81, 116, 62, 57, 
   80, 46, 40, 114, 36, 75, 56, 33, 76, 9, 55, 56, 59, 81, 65, 45, 28,
    60, 55, 93, 39, 90, 28, 124, 106, 16, 20, 104, 119, 8, 109, 26, 
   106, 9, 97, 13, 99, 15, 119, 20, 105, 117, 98, 103, 118, 1, 126, 
   29, 97, 122, 17, 15, 114, 110, 3, 5, 125, 125, 99, 126, 119, 102, 
   30, 122, 2, 117}, 1}}
\end{lstlisting}

Вывод Tally это список пар, каждая пара это 81-байтный блок и количество раз, сколько он встретился в файле.
Мы видим, что наиболее частно встречающийся блок это первый, он встретился 1739 раз.
Второй встретился 1422 раза. Есть и другие: 1012 раза, 377 раз, итд.
81-байтные блоки, встреченные лишь один раз, находятся в конце вывода.

Попробуем сравнить эти блоки. Первый и второй.
Есть ли в Mathematica ф-ция для сравнения списков/массивов?
Наверняка есть, но в педагогических целях, я буду использоват операцию XOR для сравнения.
Действительно: если байты во входных массивах равны друг другу, результат операции XOR это 0.
Если не равны, результат будет ненулевой.

Сравним первый блок (встречается 1739 раз) и второй (встречается 1422 раз):

\begin{lstlisting}[style=custommath]
In[]:= BitXor[stat[[1]][[1]], stat[[2]][[1]]]
Out[]= {0, 3, 0, 0, 0, 0, 0, 0, 0, 0, 0, 0, 0, 0, 0, 0, 0, 0, 0, \
0, 0, 0, 0, 0, 0, 0, 0, 0, 0, 0, 0, 0, 0, 0, 0, 0, 0, 0, 0, 0, 0, 0, \
0, 0, 0, 0, 0, 0, 0, 0, 0, 0, 0, 0, 0, 0, 0, 0, 0, 0, 0, 0, 0, 0, 0, \
0, 0, 0, 0, 0, 0, 0, 0, 0, 0, 0, 0, 0, 0, 0, 0}
\end{lstlisting}

Они отличаются только вторым байтом.

Сравним второй блок (встречается 1422 раза) и третий (встречается 1012 раз):

\begin{lstlisting}[style=custommath]
In[]:= BitXor[stat[[2]][[1]], stat[[3]][[1]]]
Out[]= {0, 1, 0, 0, 0, 0, 0, 0, 0, 0, 0, 0, 0, 0, 0, 0, 0, 0, 0, \
0, 0, 0, 0, 0, 0, 0, 0, 0, 0, 0, 0, 0, 0, 0, 0, 0, 0, 0, 0, 0, 0, 0, \
0, 0, 0, 0, 0, 0, 0, 0, 0, 0, 0, 0, 0, 0, 0, 0, 0, 0, 0, 0, 0, 0, 0, \
0, 0, 0, 0, 0, 0, 0, 0, 0, 0, 0, 0, 0, 0, 0, 0}
\end{lstlisting}

Они тоже отличаются только вторым байтом.

Так или иначе, попробуем использовать самый встречающийся блок как XOR-ключ и попробуем расшифровать первые 4 81-байтных
блока в файле:

\begin{lstlisting}[style=custommath]
In[]:= key = stat[[1]][[1]]
Out[]= {80, 103, 2, 116, 113, 102, 118, 25, 99, 8, 19, 23, 116, \
125, 107, 25, 99, 109, 114, 102, 14, 121, 115, 31, 9, 117, 113, 111, \
5, 4, 127, 28, 122, 101, 8, 110, 14, 18, 124, 106, 16, 20, 104, 119, \
8, 109, 26, 106, 9, 97, 13, 99, 15, 119, 20, 105, 117, 98, 103, 118, \
1, 126, 29, 97, 122, 17, 15, 114, 110, 3, 5, 125, 125, 99, 126, 119, \
102, 30, 122, 2, 117}

In[]:= ToASCII[val_] := If[val == 0, " ", FromCharacterCode[val, "PrintableASCII"]]

In[]:= DecryptBlockASCII[blk_] := Map[ToASCII[#] &, BitXor[key, blk]]

In[]:= DecryptBlockASCII[blocks[[1]]]
Out[]= {" ", " ", " ", " ", " ", " ", " ", " ", " ", " ", " ", " \
", " ", " ", " ", " ", " ", " ", " ", " ", " ", " ", " ", " ", " ", " \
", " ", " ", " ", " ", " ", " ", " ", " ", " ", " ", " ", " ", " ", " \
", " ", " ", " ", " ", " ", " ", " ", " ", " ", " ", " ", " ", " ", " \
", " ", " ", " ", " ", " ", " ", " ", " ", " ", " ", " ", " ", " ", " \
", " ", " ", " ", " ", " ", " ", " ", " ", " ", " ", " ", " ", " "}

In[]:= DecryptBlockASCII[blocks[[2]]]
Out[]= {" ", "e", "H", "E", " ", "W", "E", "E", "D", " ", "O", \
"F", " ", "C", "R", "I", "M", "E", " ", "B", "E", "A", "R", "S", " ", \
"B", "I", "T", "T", "E", "R", " ", "F", "R", "U", "I", "T", "?", \
" ", " ", " ", " ", " ", " ", " ", " ", " ", " ", " ", " ", " ", " ", \
" ", " ", " ", " ", " ", " ", " ", " ", " ", " ", " ", " ", " ", " ", \
" ", " ", " ", " ", " ", " ", " ", " ", " ", " ", " ", " ", " ", " ", \
" "}

In[]:= DecryptBlockASCII[blocks[[3]]]
Out[]= {" ", "?", " ", " ", " ", " ", " ", " ", " ", " ", " \
", " ", " ", " ", " ", " ", " ", " ", " ", " ", " ", " ", " ", " ", " \
", " ", " ", " ", " ", " ", " ", " ", " ", " ", " ", " ", " ", " ", " \
", " ", " ", " ", " ", " ", " ", " ", " ", " ", " ", " ", " ", " ", " \
", " ", " ", " ", " ", " ", " ", " ", " ", " ", " ", " ", " ", " ", " \
", " ", " ", " ", " ", " ", " ", " ", " ", " ", " ", " ", " ", " ", " \
"}

In[]:= DecryptBlockASCII[blocks[[4]]]
Out[]= {" ", "f", "H", "O", " ", "K", "N", "O", "W", "S", " ", \
"W", "H", "A", "T", " ", "E", "V", "I", "L", " ", "L", "U", "R", "K", \
"S", " ", "I", "N", " ", "T", "H", "E", " ", "H", "E", "A", "R", "T", \
"S", " ", "O", "F", " ", "M", "E", "N", "?", " ", " ", " ", " ", \
" ", " ", " ", " ", " ", " ", " ", " ", " ", " ", " ", " ", " ", " ", \
" ", " ", " ", " ", " ", " ", " ", " ", " ", " ", " ", " ", " ", " ", \
" "}
\end{lstlisting}

(Я заменил непечатаемые символы на \q{?}.)

Мы видим что первый и третий блоки пустые (или почти пустые),
но второй и четвертый имеют ясно различимые английские слова/фразы.
Похоже что наше предположение насчет ключа верно (как минимум частично).
Это означает, что самый встречающийся 81-байтный блок в файле находится в местах лакун с нулевыми байтами
или что-то в этом роде.

Попробуем расшифровать весь файл:

\begin{lstlisting}[style=custommath]
DecryptBlock[blk_] := BitXor[key, blk]

decrypted = Map[DecryptBlock[#] &, blocks];

BinaryWrite["/home/dennis/.../tmp", Flatten[decrypted]]

Close["/home/dennis/.../tmp"]
\end{lstlisting}

\begin{figure}[H]
\centering
\myincludegraphics{ff/XOR/mask_1/mc_decrypted1.png}
\caption{Расшифрованный файл в Midnight Commander, первая попытка}
\end{figure}

Выглядит как английские фразы для какой-то игры, но что-то не так.
Прежде всего, регистр инвертирован: фразы и некоторые слова начинаются со строчных букв,
в то время как остальные буквы заглавные.
Также, некоторые фразы начинаются с не тех букв.
Посмотрите на самую первую фразу: \q{eHE WEED OF CRIME BEARS BITTER FRUIT}.
Что такое \q{eHE}? Разве не \q{tHE} тут должно быть?
Возможно ли что наш ключ для дешифрования имеет неверный байт в этом месте?

Посмотрим снова на второй блок в файле, на ключ и на результат дешифрования:

\begin{lstlisting}[style=custommath]
In[]:= blocks[[2]]
Out[]= {80, 2, 74, 49, 113, 49, 51, 92, 39, 8, 92, 81, 116, 62, \
57, 80, 46, 40, 114, 36, 75, 56, 33, 76, 9, 55, 56, 59, 81, 65, 45, \
28, 60, 55, 93, 39, 90, 28, 124, 106, 16, 20, 104, 119, 8, 109, 26, \
106, 9, 97, 13, 99, 15, 119, 20, 105, 117, 98, 103, 118, 1, 126, 29, \
97, 122, 17, 15, 114, 110, 3, 5, 125, 125, 99, 126, 119, 102, 30, \
122, 2, 117}

In[]:= key
Out[]= {80, 103, 2, 116, 113, 102, 118, 25, 99, 8, 19, 23, 116, \
125, 107, 25, 99, 109, 114, 102, 14, 121, 115, 31, 9, 117, 113, 111, \
5, 4, 127, 28, 122, 101, 8, 110, 14, 18, 124, 106, 16, 20, 104, 119, \
8, 109, 26, 106, 9, 97, 13, 99, 15, 119, 20, 105, 117, 98, 103, 118, \
1, 126, 29, 97, 122, 17, 15, 114, 110, 3, 5, 125, 125, 99, 126, 119, \
102, 30, 122, 2, 117}

In[]:= BitXor[key, blocks[[2]]]
Out[]= {0, 101, 72, 69, 0, 87, 69, 69, 68, 0, 79, 70, 0, 67, 82, \
73, 77, 69, 0, 66, 69, 65, 82, 83, 0, 66, 73, 84, 84, 69, 82, 0, 70, \
82, 85, 73, 84, 14, 0, 0, 0, 0, 0, 0, 0, 0, 0, 0, 0, 0, 0, 0, 0, 0, \
0, 0, 0, 0, 0, 0, 0, 0, 0, 0, 0, 0, 0, 0, 0, 0, 0, 0, 0, 0, 0, 0, 0, \
0, 0, 0, 0}
\end{lstlisting}

Зашифрованный байт это 2, байт из ключа это 103, $2 \oplus 103=101$ и 101 это ASCII-код символа \q{e}.
Чему должен равнятся этот байт ключа, чтобы ASCII-код был 116 (для символа  \q{t})?
$2 \oplus 116=118$, присвоим 118 второму байту в ключе \dots

\begin{lstlisting}[style=custommath]
key = {80, 118, 2, 116, 113, 102, 118, 25, 99, 8, 19, 23, 116, 125, 
  107, 25, 99, 109, 114, 102, 14, 121, 115, 31, 9, 117, 113, 111, 5, 
  4, 127, 28, 122, 101, 8, 110, 14, 18, 124, 106, 16, 20, 104, 119, 8,
   109, 26, 106, 9, 97, 13, 99, 15, 119, 20, 105, 117, 98, 103, 118, 
  1, 126, 29, 97, 122, 17, 15, 114, 110, 3, 5, 125, 125, 99, 126, 119,
   102, 30, 122, 2, 117}
\end{lstlisting}

\dots и снова дешифруем весь файл.

\begin{figure}[H]
\centering
\myincludegraphics{ff/XOR/mask_1/mc_decrypted2.png}
\caption{Дешифрованный файл в Midnight Commander, вторая попытка}
\end{figure}

Ух ты, теперь грамматика корректна, и все фразы начинаются с корректных букв.
Но все таки, регистр подозрителен.
С чего бы разработчику игры записывать их в такой манере?
Может быть наш ключ все еще неправилен?

% TODO ASCII table somewhere in the book
Изучая таблицу ASCII мы можем заметить что ASCII-коды для букв в верхнем и нижнем регистре отличаются только на один бит
(6-й бит, если считать с первого, 0b100000):

\begin{figure}[H]
\centering
\includegraphics[width=0.7\textwidth]{ascii.png}
\caption{7-битная таблица \ac{ASCII} в Emacs}
\end{figure}

6-й бит, выставленный в нулевом байте, В десятичном виде это будет 32.
Но 32 это ASCII-код пробела!

Действительно, можно менять регистр просто применяя XOR к ASCII-коду, с 32 (больше об этом: \myref{toupper_bit}).

Возможно ли, что пустые лакуны в файле это не нулевые байты, а скорее содержащие пробелы?
Еще раз модифицируем наш XOR-ключ (я про-XOR-ю каждый байт ключа с 32):

\begin{lstlisting}[style=custommath]
(* "32" это скаляр, и "key" это вектор, но это OK *)

In[]:= key3 = BitXor[32, key]
Out[]= {112, 86, 34, 84, 81, 70, 86, 57, 67, 40, 51, 55, 84, 93, 75, \
57, 67, 77, 82, 70, 46, 89, 83, 63, 41, 85, 81, 79, 37, 36, 95, 60, \
90, 69, 40, 78, 46, 50, 92, 74, 48, 52, 72, 87, 40, 77, 58, 74, 41, \
65, 45, 67, 47, 87, 52, 73, 85, 66, 71, 86, 33, 94, 61, 65, 90, 49, \
47, 82, 78, 35, 37, 93, 93, 67, 94, 87, 70, 62, 90, 34, 85}

In[]:= DecryptBlock[blk_] := BitXor[key3, blk]
\end{lstlisting}

И снова дешифруем входной файл:

\begin{figure}[H]
\centering
\myincludegraphics{ff/XOR/mask_1/mc_decrypted.png}
\caption{Дешифрованный файл в Midnight Commander, последняя попытка}
\end{figure}

(Расшифрованный файл доступен здесь:
\url{\GitHubBlobMasterURL/ff/XOR/mask_1/files/decrypted.dat.bz2}.)

Несомненно, это корректный исходный файл.
Да, и мы видим числа в начале каждого блока. Должно быть это и есть источник некорректного XOR-ключа.
Как выходит, самый встречающийся 81-байтный блок в файле это блок заполненный пробелами и содержащий символ \q{1} на месте
второго байта.
Действительно, как-то так получилось что многие блоки здесь перемежаются с этим блоком.
Может быть это что-то вроде выравнивания (padding) для коротких фраз/сообщений?
Другой часто встречающийся 81-байтный блок также заполнен пробелами, но с другой цифрой, следовательно,
они отличаются только вторым байтом.

Вот и всё! Теперь мы можем написать утилиту для зашифрования файла назад, и, может быть, модифицировать его перед этим

Файл для Mathematica можно скачать здесь:\\
\url{\GitHubBlobMasterURL/ff/XOR/mask_1/files/XOR_mask_1.nb}.

Итог: XOR-шифрование не надежно вообще. Вероятно, разработчик игры хотел просто скрыть внутренности игры от игрока,
ничего более серьезного.
Все же, шифрование вроде этого крайне популярно вследствии его простоты, так что многие реверс инженеры обычно хорошо
с этим знакомы.

}\FR{\mysection{Fonction presque vide}
\label{Boolector}
\myindex{Boolector}
\myindex{x86!\Instructions!JMP}

Ceci est un morceau de code réel que j'ai trouvé dans Boolector\footnote{\url{https://boolector.github.io/}}:

\lstinputlisting[style=customc]{patterns/025_almost_empty/boolectormain.c}

Pourquoi quelqu'un ferait-il comme ça?
Je ne sais pas mais mon hypothèse est que \verb|boolector_main()| peut être compilée
dans une sorte de DLL ou bibliothèque dynamique, et appelée depuis une suite de test.
Certainement qu'une suite de test peut préparer les variables argc/argv comme
le ferait \ac{CRT}.

Il est intéressant de voir comment c'est compilé:

\lstinputlisting[caption=GCC 8.2 x64 \NonOptimizing (\assemblyOutput),style=customasmx86]{patterns/025_almost_empty/boolectormain_O0.s}

Ceci est OK, le prologue (non optimisé) déplace inutilement deux arguments,
\INS{CALL}, épilogue, \INS{RET}.
Mais regardons la version optimisée:

\lstinputlisting[caption=GCC 8.2 x64 \Optimizing (\assemblyOutput),style=customasmx86]{patterns/025_almost_empty/boolectormain_O3.s}

Aussi simple que ça: la pile et les registres ne sont pas touchés et \verb|boolector_main()|
a le même ensemble d'arguments.
Donc, tout ce que nous avons à faire est de passer l'exécution à une autre adresse.

Ceci est proche d'une \glslink{thunk function}{fonction thunk}.

Nous verons queelque chose de plus avancé plus tard: \myref{ARM_B_to_printf}, \myref{JMP_instead_of_RET}.
}

\renewcommand{\CURPATH}{advanced/350_cpp}
\EN{\EN{\input{patterns/016_empty_redux/main_EN}}%
\FR{\input{patterns/016_empty_redux/main_FR}}
}\RU{\EN{\input{patterns/016_empty_redux/main_EN}}%
\FR{\input{patterns/016_empty_redux/main_FR}}
}\FR{\EN{\input{patterns/016_empty_redux/main_EN}}%
\FR{\input{patterns/016_empty_redux/main_FR}}
}

\renewcommand{\CURPATH}{advanced/370_neg_arrays}
\EN{% TODO translate
\mysection{Breaking simple executable cryptor}

I've got an executable file which is encrypted by relatively simple encryption.
\href{\GitHubBlobMasterURL/examples/simple_exec_crypto/files/cipher.bin}{Here is it} (only executable section is left here).

First, all encryption function does is just adds number of position in buffer to the byte.
Here is how this can be encoded in Python:

\begin{lstlisting}[caption=Python script,style=custompy]
#!/usr/bin/env python
def e(i, k):
    return chr ((ord(i)+k) % 256)

def encrypt(buf):
    return e(buf[0], 0)+ e(buf[1], 1)+ e(buf[2], 2) + e(buf[3], 3)+ e(buf[4], 4)+ e(buf[5], 5)+ e(buf[6], 6)+ e(buf[7], 7)+
           e(buf[8], 8)+ e(buf[9], 9)+ e(buf[10], 10)+ e(buf[11], 11)+ e(buf[12], 12)+ e(buf[13], 13)+ e(buf[14], 14)+ e(buf[15], 15)
\end{lstlisting}

Hence, if you encrypt buffer with 16 zeros, you'll get \emph{0, 1, 2, 3 ... 12, 13, 14, 15}.

\myindex{Propagating Cipher Block Chaining}
Propagating Cipher Block Chaining (PCBC) is also used, here is how it works:

\begin{figure}[H]
\centering
\myincludegraphics{examples/simple_exec_crypto/601px-PCBC_encryption.png}
\caption{Propagating Cipher Block Chaining encryption (image is taken from Wikipedia article)}
\end{figure}

The problem is that it's too boring to recover IV (Initialization Vector) each time.
Brute-force is also not an option, because IV is too long (16 bytes).
Let's see, if it's possible to recover IV for arbitrary encrypted executable file?

Let's try simple frequency analysis.
This is 32-bit x86 executable code, so let's gather statistics about most frequent bytes and opcodes.
I tried huge oracle.exe file from Oracle RDBMS version 11.2 for windows x86 and I've found that the most frequent byte (no surprise) is zero (~10\%).
The next most frequent byte is (again, no surprise) 0xFF (~5\%).
The next is 0x8B (~5\%).

\myindex{x86!\Instructions!MOV}
0x8B is opcode for \INS{MOV}, this is indeed one of the most busy x86 instructions.
Now what about popularity of zero byte?
If compiler needs to encode value bigger than 127, it has to use 32-bit displacement instead of 8-bit one, but large values are very rare,
so it is padded by zeros.
\myindex{x86!\Instructions!LEA}
\myindex{x86!\Instructions!PUSH}
\myindex{x86!\Instructions!CALL}
This is at least in \INS{LEA}, \INS{MOV}, \INS{PUSH}, \INS{CALL}.

For example:

\begin{lstlisting}[style=customasmx86]
8D B0 28 01 00 00                 lea     esi, [eax+128h]
8D BF 40 38 00 00                 lea     edi, [edi+3840h]
\end{lstlisting}

Displacements bigger than 127 are very popular, but they are rarely exceeds 0x10000
(indeed, such large memory buffers/structures are also rare).

Same story with \INS{MOV}, large constants are rare, the most heavily used are 0, 1, 10, 100, $2^n$, and so on.
Compiler has to pad small constants by zeros to represent them as 32-bit values:

\begin{lstlisting}[style=customasmx86]
BF 02 00 00 00                    mov     edi, 2
BF 01 00 00 00                    mov     edi, 1
\end{lstlisting}

Now about 00 and FF bytes combined: jumps (including conditional) and calls can pass execution flow forward or backwards, but very often,
within the limits of the current executable module.
If forward, displacement is not very big and also padded with zeros.
If backwards, displacement is represented as negative value, so padded with FF bytes.
For example, transfer execution flow forward:

\begin{lstlisting}[style=customasmx86]
E8 43 0C 00 00                    call    _function1
E8 5C 00 00 00                    call    _function2
0F 84 F0 0A 00 00                 jz      loc_4F09A0
0F 84 EB 00 00 00                 jz      loc_4EFBB8
\end{lstlisting}

Backwards:

\begin{lstlisting}[style=customasmx86]
E8 79 0C FE FF                    call    _function1
E8 F4 16 FF FF                    call    _function2
0F 84 F8 FB FF FF                 jz      loc_8212BC
0F 84 06 FD FF FF                 jz      loc_FF1E7D
\end{lstlisting}

FF byte is also very often occurred in negative displacements like these:

\begin{lstlisting}[style=customasmx86]
8D 85 1E FF FF FF                 lea     eax, [ebp-0E2h]
8D 95 F8 5C FF FF                 lea     edx, [ebp-0A308h]
\end{lstlisting}

So far so good. Now we have to try various 16-byte keys, decrypt executable section and measure how often 00, FF and 8B bytes are occurred.
Let's also keep in sight how PCBC decryption works:

\begin{figure}[H]
\centering
\myincludegraphics{examples/simple_exec_crypto/640px-PCBC_decryption.png}
\caption{Propagating Cipher Block Chaining decryption (image is taken from Wikipedia article)}
\end{figure}

The good news is that we don't really have to decrypt whole piece of data, but only slice by slice, this is exactly how I did in my previous example: \myref{XOR_mask_2}.

Now I'm trying all possible bytes (0..255) for each byte in key and just pick the byte producing maximal amount of 00/FF/8B bytes in a decrypted slice:

\begin{lstlisting}[style=custompy]
#!/usr/bin/env python
import sys, hexdump, array, string, operator

KEY_LEN=16

def chunks(l, n):
    # split n by l-byte chunks
    # https://stackoverflow.com/q/312443
    n = max(1, n)
    return [l[i:i + n] for i in range(0, len(l), n)]

def read_file(fname):
    file=open(fname, mode='rb')
    content=file.read()
    file.close()
    return content

def decrypt_byte (c, key):
    return chr((ord(c)-key) % 256)

def XOR_PCBC_step (IV, buf, k):
    prev=IV
    rt=""
    for c in buf:
	new_c=decrypt_byte(c, k)
        plain=chr(ord(new_c)^ord(prev))
	prev=chr(ord(c)^ord(plain))
	rt=rt+plain
    return rt

each_Nth_byte=[""]*KEY_LEN

content=read_file(sys.argv[1])
# split input by 16-byte chunks:
all_chunks=chunks(content, KEY_LEN)
for c in all_chunks:
    for i in range(KEY_LEN):
        each_Nth_byte[i]=each_Nth_byte[i] + c[i]

# try each byte of key
for N in range(KEY_LEN):
    print "N=", N
    stat={}
    for i in range(256):
        tmp_key=chr(i)
	tmp=XOR_PCBC_step(tmp_key,each_Nth_byte[N], N)
        # count 0, FFs and 8Bs in decrypted buffer:
	important_bytes=tmp.count('\x00')+tmp.count('\xFF')+tmp.count('\x8B')
	stat[i]=important_bytes
    sorted_stat = sorted(stat.iteritems(), key=operator.itemgetter(1), reverse=True)
    print sorted_stat[0]
\end{lstlisting}

(Source code can be downloaded \href{\GitHubBlobMasterURL/examples/simple_exec_crypto/files/decrypt.py}{here}.)

I run it and here is a key for which 00/FF/8B bytes presence in decrypted buffer is maximal:

\begin{lstlisting}
N= 0
(147, 1224)
N= 1
(94, 1327)
N= 2
(252, 1223)
N= 3
(218, 1266)
N= 4
(38, 1209)
N= 5
(192, 1378)
N= 6
(199, 1204)
N= 7
(213, 1332)
N= 8
(225, 1251)
N= 9
(112, 1223)
N= 10
(143, 1177)
N= 11
(108, 1286)
N= 12
(10, 1164)
N= 13
(3, 1271)
N= 14
(128, 1253)
N= 15
(232, 1330)
\end{lstlisting}

Let's write decryption utility with the key we got:

\begin{lstlisting}[style=custompy]
#!/usr/bin/env python
import sys, hexdump, array

def xor_strings(s,t):
    # \verb|https://en.wikipedia.org/wiki/XOR_cipher#Example_implementation|
    """xor two strings together"""
    return "".join(chr(ord(a)^ord(b)) for a,b in zip(s,t))

IV=array.array('B', [147, 94, 252, 218, 38, 192, 199, 213, 225, 112, 143, 108, 10, 3, 128, 232]).tostring()

def chunks(l, n):
    n = max(1, n)
    return [l[i:i + n] for i in range(0, len(l), n)]

def read_file(fname):
    file=open(fname, mode='rb')
    content=file.read()
    file.close()
    return content

def decrypt_byte(i, k):
    return chr ((ord(i)-k) % 256)

def decrypt(buf):
    return "".join(decrypt_byte(buf[i], i) for i in range(16))

fout=open(sys.argv[2], mode='wb')

prev=IV
content=read_file(sys.argv[1])
tmp=chunks(content, 16)
for c in tmp:
    new_c=decrypt(c)
    p=xor_strings (new_c, prev)
    prev=xor_strings(c, p)
    fout.write(p)
fout.close()
\end{lstlisting}

(Source code can be downloaded \href{\GitHubBlobMasterURL/examples/simple_exec_crypto/files/decrypt2.py}{here}.)

Let's check resulting file:

\lstinputlisting{examples/simple_exec_crypto/objdump_result.txt}

Yes, this is seems correctly disassembled piece of x86 code.
The whole decryped file can be downloaded \href{\GitHubBlobMasterURL/examples/simple_exec_crypto/files/decrypted.bin}{here}.

In fact, this is text section from regedit.exe from Windows 7.
But this example is based on a real case I encountered, so just executable is different (and key), algorithm is the same.

\subsection{Other ideas to consider}

What if I would fail with such simple frequency analysis?
There are other ideas on how to measure correctness of decrypted/decompressed x86 code:

\begin{itemize}

\item Many modern compilers aligns functions on 0x10 border.
So the space left before is filled with NOPs (0x90) or other NOP instructions with known opcodes: \myref{sec:npad}.

\item Perhaps, the most frequent pattern in any assembly language is function call:\\
\TT{PUSH chain / CALL / ADD ESP, X}.
This sequence can easily detected and found.
I've even gathered statistics about average number of function arguments: \myref{args_stat}.
(Hence, this is average length of PUSH chain.)

\end{itemize}

Read more about incorrectly/correctly disassembled code: \myref{ISA_detect}.
}\RU{\subsection{Простое шифрование используя XOR-маску}
\label{XOR_mask_1}

Я нашел одну старую игру в стиле interactive fiction в архиве \emph{if-archive}\footnote{\url{http://www.ifarchive.org/}}:

\begin{lstlisting}
The New Castle v3.5 - Text/Adventure Game
in the style of the original Infocom (tm)
type games, Zork, Collosal Cave (Adventure),
etc.  Can you solve the mystery of the
abandoned castle?
Shareware from Software Customization.
Software Customization [ASP] Version 3.5 Feb. 2000
\end{lstlisting}

Можно скачать здесь: \url{\GitHubBlobMasterURL/ff/XOR/mask_1/files/newcastle.tgz}.

Там внутри есть файл (с названием \emph{castle.dbf}), который явно зашифрован, но не настоящим криптоалгоритмом,
и оне сжат, это что-то куда проще.
Я бы даже не стал измерять уровень энтропии (\myref{entropy}) этого файла, потому что я итак уверен, что он низкий.
Вот как он выглядит в Midnight Commander:

\begin{figure}[H]
\centering
\myincludegraphics{ff/XOR/mask_1/mc_encrypted.png}
\caption{Зашифрованный файл в Midnight Commander}
\end{figure}

Зашифрованный файл можно скачать здесь:
\url{\GitHubBlobMasterURL/ff/XOR/mask_1/files/castle.dbf.bz2}.

Можно ли расшифровать его без доступа к программе, используя просто этот файл?

Тут явно просматривается повторяющаяся строка. 
Если использовалось простое шифрование с XOR-маской, такие повторяющиеся строки это явное свидетельство,
потому что, вероятно, тут были длинные лакуны с нулевыми байтами, которые, в свою очередь, присутствуют
во мноигих исполняемых файлах, и в остальных бинарных файлах.

\myindex{UNIX!xxd}
Вот дам начала этого файла используя утилиту \emph{xxd} из UNIX:

\lstinputlisting{ff/XOR/mask_1/xxd_result.txt}

Давайте держаться за повторяющуюся строку \TT{iubgv}.
Глядя на этот дамп, мы можем легко увидеть, что период повторений этой строки это 0x51 или 81.
Вероятно, 81 это длина блока?
Длина файла 1658961, и она может быть поделена на 81 без остатка (и тогда там 20481 блоков).

Теперь я буду использовать Mathematica для анализа, есть ли тут повторяющиеся 81-байтные блоки в файле?
Я разделю входной файл на 81-байтные блоки и затем использую ф-цию
\emph{Tally[]}\footnote{\url{https://reference.wolfram.com/language/ref/Tally.html}}
которая просто считает, сколько раз каждый элемент встретился во входном списке.
Вывод Tally не отсортирован, так что я также добавлю ф-цию \emph{Sort[]} для сортировки его по кол-ву вхождений
в нисходящем порядке.

\begin{lstlisting}[style=custommath]
input = BinaryReadList["/home/dennis/.../castle.dbf"];

blocks = Partition[input, 81];

stat = Sort[Tally[blocks], #1[[2]] > #2[[2]] &]
\end{lstlisting}

И вот вывод:

\begin{lstlisting}[style=custommath]
{{{80, 103, 2, 116, 113, 102, 118, 25, 99, 8, 19, 23, 116, 125, 107, 
   25, 99, 109, 114, 102, 14, 121, 115, 31, 9, 117, 113, 111, 5, 4, 
   127, 28, 122, 101, 8, 110, 14, 18, 124, 106, 16, 20, 104, 119, 8, 
   109, 26, 106, 9, 97, 13, 99, 15, 119, 20, 105, 117, 98, 103, 118, 
   1, 126, 29, 97, 122, 17, 15, 114, 110, 3, 5, 125, 125, 99, 126, 
   119, 102, 30, 122, 2, 117}, 1739}, 
{{80, 100, 2, 116, 113, 102, 118, 25, 99, 8, 19, 23, 116, 
   125, 107, 25, 99, 109, 114, 102, 14, 121, 115, 31, 9, 117, 113, 
   111, 5, 4, 127, 28, 122, 101, 8, 110, 14, 18, 124, 106, 16, 20, 
   104, 119, 8, 109, 26, 106, 9, 97, 13, 99, 15, 119, 20, 105, 117, 
   98, 103, 118, 1, 126, 29, 97, 122, 17, 15, 114, 110, 3, 5, 125, 
   125, 99, 126, 119, 102, 30, 122, 2, 117}, 1422}, 
{{80, 101, 2, 116, 113, 102, 118, 25, 99, 8, 19, 23, 116, 
   125, 107, 25, 99, 109, 114, 102, 14, 121, 115, 31, 9, 117, 113, 
   111, 5, 4, 127, 28, 122, 101, 8, 110, 14, 18, 124, 106, 16, 20, 
   104, 119, 8, 109, 26, 106, 9, 97, 13, 99, 15, 119, 20, 105, 117, 
   98, 103, 118, 1, 126, 29, 97, 122, 17, 15, 114, 110, 3, 5, 125, 
   125, 99, 126, 119, 102, 30, 122, 2, 117}, 1012},
{{80, 120, 2, 116, 113, 102, 118, 25, 99, 8, 19, 23, 116, 
   125, 107, 25, 99, 109, 114, 102, 14, 121, 115, 31, 9, 117, 113, 
   111, 5, 4, 127, 28, 122, 101, 8, 110, 14, 18, 124, 106, 16, 20, 
   104, 119, 8, 109, 26, 106, 9, 97, 13, 99, 15, 119, 20, 105, 117, 
   98, 103, 118, 1, 126, 29, 97, 122, 17, 15, 114, 110, 3, 5, 125, 
   125, 99, 126, 119, 102, 30, 122, 2, 117}, 377},

...

{{80, 2, 74, 49, 113, 21, 62, 88, 39, 71, 68, 23, 63, 51, 36, 78, 48, 
   108, 114, 102, 14, 121, 115, 31, 9, 117, 113, 111, 5, 4, 127, 28, 
   122, 101, 8, 110, 14, 18, 124, 106, 16, 20, 104, 119, 8, 109, 26, 
   106, 9, 97, 13, 99, 15, 119, 20, 105, 117, 98, 103, 118, 1, 126, 
   29, 97, 122, 17, 15, 114, 110, 3, 5, 125, 125, 99, 126, 119, 102, 
   30, 122, 2, 117}, 1},
{{80, 1, 74, 59, 113, 45, 56, 86, 52, 91, 19, 64, 60, 60, 63, 
   25, 38, 59, 59, 42, 14, 53, 38, 77, 66, 38, 113, 38, 75, 4, 43, 84,
    63, 101, 64, 43, 79, 64, 40, 57, 16, 91, 46, 119, 69, 40, 84, 117,
    9, 97, 13, 99, 15, 119, 20, 105, 117, 98, 103, 118, 1, 126, 29, 
   97, 122, 17, 15, 114, 110, 3, 5, 125, 125, 99, 126, 119, 102, 30, 
   122, 2, 117}, 1},
{{80, 2, 74, 49, 113, 49, 51, 92, 39, 8, 92, 81, 116, 62, 57, 
   80, 46, 40, 114, 36, 75, 56, 33, 76, 9, 55, 56, 59, 81, 65, 45, 28,
    60, 55, 93, 39, 90, 28, 124, 106, 16, 20, 104, 119, 8, 109, 26, 
   106, 9, 97, 13, 99, 15, 119, 20, 105, 117, 98, 103, 118, 1, 126, 
   29, 97, 122, 17, 15, 114, 110, 3, 5, 125, 125, 99, 126, 119, 102, 
   30, 122, 2, 117}, 1}}
\end{lstlisting}

Вывод Tally это список пар, каждая пара это 81-байтный блок и количество раз, сколько он встретился в файле.
Мы видим, что наиболее частно встречающийся блок это первый, он встретился 1739 раз.
Второй встретился 1422 раза. Есть и другие: 1012 раза, 377 раз, итд.
81-байтные блоки, встреченные лишь один раз, находятся в конце вывода.

Попробуем сравнить эти блоки. Первый и второй.
Есть ли в Mathematica ф-ция для сравнения списков/массивов?
Наверняка есть, но в педагогических целях, я буду использоват операцию XOR для сравнения.
Действительно: если байты во входных массивах равны друг другу, результат операции XOR это 0.
Если не равны, результат будет ненулевой.

Сравним первый блок (встречается 1739 раз) и второй (встречается 1422 раз):

\begin{lstlisting}[style=custommath]
In[]:= BitXor[stat[[1]][[1]], stat[[2]][[1]]]
Out[]= {0, 3, 0, 0, 0, 0, 0, 0, 0, 0, 0, 0, 0, 0, 0, 0, 0, 0, 0, \
0, 0, 0, 0, 0, 0, 0, 0, 0, 0, 0, 0, 0, 0, 0, 0, 0, 0, 0, 0, 0, 0, 0, \
0, 0, 0, 0, 0, 0, 0, 0, 0, 0, 0, 0, 0, 0, 0, 0, 0, 0, 0, 0, 0, 0, 0, \
0, 0, 0, 0, 0, 0, 0, 0, 0, 0, 0, 0, 0, 0, 0, 0}
\end{lstlisting}

Они отличаются только вторым байтом.

Сравним второй блок (встречается 1422 раза) и третий (встречается 1012 раз):

\begin{lstlisting}[style=custommath]
In[]:= BitXor[stat[[2]][[1]], stat[[3]][[1]]]
Out[]= {0, 1, 0, 0, 0, 0, 0, 0, 0, 0, 0, 0, 0, 0, 0, 0, 0, 0, 0, \
0, 0, 0, 0, 0, 0, 0, 0, 0, 0, 0, 0, 0, 0, 0, 0, 0, 0, 0, 0, 0, 0, 0, \
0, 0, 0, 0, 0, 0, 0, 0, 0, 0, 0, 0, 0, 0, 0, 0, 0, 0, 0, 0, 0, 0, 0, \
0, 0, 0, 0, 0, 0, 0, 0, 0, 0, 0, 0, 0, 0, 0, 0}
\end{lstlisting}

Они тоже отличаются только вторым байтом.

Так или иначе, попробуем использовать самый встречающийся блок как XOR-ключ и попробуем расшифровать первые 4 81-байтных
блока в файле:

\begin{lstlisting}[style=custommath]
In[]:= key = stat[[1]][[1]]
Out[]= {80, 103, 2, 116, 113, 102, 118, 25, 99, 8, 19, 23, 116, \
125, 107, 25, 99, 109, 114, 102, 14, 121, 115, 31, 9, 117, 113, 111, \
5, 4, 127, 28, 122, 101, 8, 110, 14, 18, 124, 106, 16, 20, 104, 119, \
8, 109, 26, 106, 9, 97, 13, 99, 15, 119, 20, 105, 117, 98, 103, 118, \
1, 126, 29, 97, 122, 17, 15, 114, 110, 3, 5, 125, 125, 99, 126, 119, \
102, 30, 122, 2, 117}

In[]:= ToASCII[val_] := If[val == 0, " ", FromCharacterCode[val, "PrintableASCII"]]

In[]:= DecryptBlockASCII[blk_] := Map[ToASCII[#] &, BitXor[key, blk]]

In[]:= DecryptBlockASCII[blocks[[1]]]
Out[]= {" ", " ", " ", " ", " ", " ", " ", " ", " ", " ", " ", " \
", " ", " ", " ", " ", " ", " ", " ", " ", " ", " ", " ", " ", " ", " \
", " ", " ", " ", " ", " ", " ", " ", " ", " ", " ", " ", " ", " ", " \
", " ", " ", " ", " ", " ", " ", " ", " ", " ", " ", " ", " ", " ", " \
", " ", " ", " ", " ", " ", " ", " ", " ", " ", " ", " ", " ", " ", " \
", " ", " ", " ", " ", " ", " ", " ", " ", " ", " ", " ", " ", " "}

In[]:= DecryptBlockASCII[blocks[[2]]]
Out[]= {" ", "e", "H", "E", " ", "W", "E", "E", "D", " ", "O", \
"F", " ", "C", "R", "I", "M", "E", " ", "B", "E", "A", "R", "S", " ", \
"B", "I", "T", "T", "E", "R", " ", "F", "R", "U", "I", "T", "?", \
" ", " ", " ", " ", " ", " ", " ", " ", " ", " ", " ", " ", " ", " ", \
" ", " ", " ", " ", " ", " ", " ", " ", " ", " ", " ", " ", " ", " ", \
" ", " ", " ", " ", " ", " ", " ", " ", " ", " ", " ", " ", " ", " ", \
" "}

In[]:= DecryptBlockASCII[blocks[[3]]]
Out[]= {" ", "?", " ", " ", " ", " ", " ", " ", " ", " ", " \
", " ", " ", " ", " ", " ", " ", " ", " ", " ", " ", " ", " ", " ", " \
", " ", " ", " ", " ", " ", " ", " ", " ", " ", " ", " ", " ", " ", " \
", " ", " ", " ", " ", " ", " ", " ", " ", " ", " ", " ", " ", " ", " \
", " ", " ", " ", " ", " ", " ", " ", " ", " ", " ", " ", " ", " ", " \
", " ", " ", " ", " ", " ", " ", " ", " ", " ", " ", " ", " ", " ", " \
"}

In[]:= DecryptBlockASCII[blocks[[4]]]
Out[]= {" ", "f", "H", "O", " ", "K", "N", "O", "W", "S", " ", \
"W", "H", "A", "T", " ", "E", "V", "I", "L", " ", "L", "U", "R", "K", \
"S", " ", "I", "N", " ", "T", "H", "E", " ", "H", "E", "A", "R", "T", \
"S", " ", "O", "F", " ", "M", "E", "N", "?", " ", " ", " ", " ", \
" ", " ", " ", " ", " ", " ", " ", " ", " ", " ", " ", " ", " ", " ", \
" ", " ", " ", " ", " ", " ", " ", " ", " ", " ", " ", " ", " ", " ", \
" "}
\end{lstlisting}

(Я заменил непечатаемые символы на \q{?}.)

Мы видим что первый и третий блоки пустые (или почти пустые),
но второй и четвертый имеют ясно различимые английские слова/фразы.
Похоже что наше предположение насчет ключа верно (как минимум частично).
Это означает, что самый встречающийся 81-байтный блок в файле находится в местах лакун с нулевыми байтами
или что-то в этом роде.

Попробуем расшифровать весь файл:

\begin{lstlisting}[style=custommath]
DecryptBlock[blk_] := BitXor[key, blk]

decrypted = Map[DecryptBlock[#] &, blocks];

BinaryWrite["/home/dennis/.../tmp", Flatten[decrypted]]

Close["/home/dennis/.../tmp"]
\end{lstlisting}

\begin{figure}[H]
\centering
\myincludegraphics{ff/XOR/mask_1/mc_decrypted1.png}
\caption{Расшифрованный файл в Midnight Commander, первая попытка}
\end{figure}

Выглядит как английские фразы для какой-то игры, но что-то не так.
Прежде всего, регистр инвертирован: фразы и некоторые слова начинаются со строчных букв,
в то время как остальные буквы заглавные.
Также, некоторые фразы начинаются с не тех букв.
Посмотрите на самую первую фразу: \q{eHE WEED OF CRIME BEARS BITTER FRUIT}.
Что такое \q{eHE}? Разве не \q{tHE} тут должно быть?
Возможно ли что наш ключ для дешифрования имеет неверный байт в этом месте?

Посмотрим снова на второй блок в файле, на ключ и на результат дешифрования:

\begin{lstlisting}[style=custommath]
In[]:= blocks[[2]]
Out[]= {80, 2, 74, 49, 113, 49, 51, 92, 39, 8, 92, 81, 116, 62, \
57, 80, 46, 40, 114, 36, 75, 56, 33, 76, 9, 55, 56, 59, 81, 65, 45, \
28, 60, 55, 93, 39, 90, 28, 124, 106, 16, 20, 104, 119, 8, 109, 26, \
106, 9, 97, 13, 99, 15, 119, 20, 105, 117, 98, 103, 118, 1, 126, 29, \
97, 122, 17, 15, 114, 110, 3, 5, 125, 125, 99, 126, 119, 102, 30, \
122, 2, 117}

In[]:= key
Out[]= {80, 103, 2, 116, 113, 102, 118, 25, 99, 8, 19, 23, 116, \
125, 107, 25, 99, 109, 114, 102, 14, 121, 115, 31, 9, 117, 113, 111, \
5, 4, 127, 28, 122, 101, 8, 110, 14, 18, 124, 106, 16, 20, 104, 119, \
8, 109, 26, 106, 9, 97, 13, 99, 15, 119, 20, 105, 117, 98, 103, 118, \
1, 126, 29, 97, 122, 17, 15, 114, 110, 3, 5, 125, 125, 99, 126, 119, \
102, 30, 122, 2, 117}

In[]:= BitXor[key, blocks[[2]]]
Out[]= {0, 101, 72, 69, 0, 87, 69, 69, 68, 0, 79, 70, 0, 67, 82, \
73, 77, 69, 0, 66, 69, 65, 82, 83, 0, 66, 73, 84, 84, 69, 82, 0, 70, \
82, 85, 73, 84, 14, 0, 0, 0, 0, 0, 0, 0, 0, 0, 0, 0, 0, 0, 0, 0, 0, \
0, 0, 0, 0, 0, 0, 0, 0, 0, 0, 0, 0, 0, 0, 0, 0, 0, 0, 0, 0, 0, 0, 0, \
0, 0, 0, 0}
\end{lstlisting}

Зашифрованный байт это 2, байт из ключа это 103, $2 \oplus 103=101$ и 101 это ASCII-код символа \q{e}.
Чему должен равнятся этот байт ключа, чтобы ASCII-код был 116 (для символа  \q{t})?
$2 \oplus 116=118$, присвоим 118 второму байту в ключе \dots

\begin{lstlisting}[style=custommath]
key = {80, 118, 2, 116, 113, 102, 118, 25, 99, 8, 19, 23, 116, 125, 
  107, 25, 99, 109, 114, 102, 14, 121, 115, 31, 9, 117, 113, 111, 5, 
  4, 127, 28, 122, 101, 8, 110, 14, 18, 124, 106, 16, 20, 104, 119, 8,
   109, 26, 106, 9, 97, 13, 99, 15, 119, 20, 105, 117, 98, 103, 118, 
  1, 126, 29, 97, 122, 17, 15, 114, 110, 3, 5, 125, 125, 99, 126, 119,
   102, 30, 122, 2, 117}
\end{lstlisting}

\dots и снова дешифруем весь файл.

\begin{figure}[H]
\centering
\myincludegraphics{ff/XOR/mask_1/mc_decrypted2.png}
\caption{Дешифрованный файл в Midnight Commander, вторая попытка}
\end{figure}

Ух ты, теперь грамматика корректна, и все фразы начинаются с корректных букв.
Но все таки, регистр подозрителен.
С чего бы разработчику игры записывать их в такой манере?
Может быть наш ключ все еще неправилен?

% TODO ASCII table somewhere in the book
Изучая таблицу ASCII мы можем заметить что ASCII-коды для букв в верхнем и нижнем регистре отличаются только на один бит
(6-й бит, если считать с первого, 0b100000):

\begin{figure}[H]
\centering
\includegraphics[width=0.7\textwidth]{ascii.png}
\caption{7-битная таблица \ac{ASCII} в Emacs}
\end{figure}

6-й бит, выставленный в нулевом байте, В десятичном виде это будет 32.
Но 32 это ASCII-код пробела!

Действительно, можно менять регистр просто применяя XOR к ASCII-коду, с 32 (больше об этом: \myref{toupper_bit}).

Возможно ли, что пустые лакуны в файле это не нулевые байты, а скорее содержащие пробелы?
Еще раз модифицируем наш XOR-ключ (я про-XOR-ю каждый байт ключа с 32):

\begin{lstlisting}[style=custommath]
(* "32" это скаляр, и "key" это вектор, но это OK *)

In[]:= key3 = BitXor[32, key]
Out[]= {112, 86, 34, 84, 81, 70, 86, 57, 67, 40, 51, 55, 84, 93, 75, \
57, 67, 77, 82, 70, 46, 89, 83, 63, 41, 85, 81, 79, 37, 36, 95, 60, \
90, 69, 40, 78, 46, 50, 92, 74, 48, 52, 72, 87, 40, 77, 58, 74, 41, \
65, 45, 67, 47, 87, 52, 73, 85, 66, 71, 86, 33, 94, 61, 65, 90, 49, \
47, 82, 78, 35, 37, 93, 93, 67, 94, 87, 70, 62, 90, 34, 85}

In[]:= DecryptBlock[blk_] := BitXor[key3, blk]
\end{lstlisting}

И снова дешифруем входной файл:

\begin{figure}[H]
\centering
\myincludegraphics{ff/XOR/mask_1/mc_decrypted.png}
\caption{Дешифрованный файл в Midnight Commander, последняя попытка}
\end{figure}

(Расшифрованный файл доступен здесь:
\url{\GitHubBlobMasterURL/ff/XOR/mask_1/files/decrypted.dat.bz2}.)

Несомненно, это корректный исходный файл.
Да, и мы видим числа в начале каждого блока. Должно быть это и есть источник некорректного XOR-ключа.
Как выходит, самый встречающийся 81-байтный блок в файле это блок заполненный пробелами и содержащий символ \q{1} на месте
второго байта.
Действительно, как-то так получилось что многие блоки здесь перемежаются с этим блоком.
Может быть это что-то вроде выравнивания (padding) для коротких фраз/сообщений?
Другой часто встречающийся 81-байтный блок также заполнен пробелами, но с другой цифрой, следовательно,
они отличаются только вторым байтом.

Вот и всё! Теперь мы можем написать утилиту для зашифрования файла назад, и, может быть, модифицировать его перед этим

Файл для Mathematica можно скачать здесь:\\
\url{\GitHubBlobMasterURL/ff/XOR/mask_1/files/XOR_mask_1.nb}.

Итог: XOR-шифрование не надежно вообще. Вероятно, разработчик игры хотел просто скрыть внутренности игры от игрока,
ничего более серьезного.
Все же, шифрование вроде этого крайне популярно вследствии его простоты, так что многие реверс инженеры обычно хорошо
с этим знакомы.

}\FR{\mysection{Fonction presque vide}
\label{Boolector}
\myindex{Boolector}
\myindex{x86!\Instructions!JMP}

Ceci est un morceau de code réel que j'ai trouvé dans Boolector\footnote{\url{https://boolector.github.io/}}:

\lstinputlisting[style=customc]{patterns/025_almost_empty/boolectormain.c}

Pourquoi quelqu'un ferait-il comme ça?
Je ne sais pas mais mon hypothèse est que \verb|boolector_main()| peut être compilée
dans une sorte de DLL ou bibliothèque dynamique, et appelée depuis une suite de test.
Certainement qu'une suite de test peut préparer les variables argc/argv comme
le ferait \ac{CRT}.

Il est intéressant de voir comment c'est compilé:

\lstinputlisting[caption=GCC 8.2 x64 \NonOptimizing (\assemblyOutput),style=customasmx86]{patterns/025_almost_empty/boolectormain_O0.s}

Ceci est OK, le prologue (non optimisé) déplace inutilement deux arguments,
\INS{CALL}, épilogue, \INS{RET}.
Mais regardons la version optimisée:

\lstinputlisting[caption=GCC 8.2 x64 \Optimizing (\assemblyOutput),style=customasmx86]{patterns/025_almost_empty/boolectormain_O3.s}

Aussi simple que ça: la pile et les registres ne sont pas touchés et \verb|boolector_main()|
a le même ensemble d'arguments.
Donc, tout ce que nous avons à faire est de passer l'exécution à une autre adresse.

Ceci est proche d'une \glslink{thunk function}{fonction thunk}.

Nous verons queelque chose de plus avancé plus tard: \myref{ARM_B_to_printf}, \myref{JMP_instead_of_RET}.
}

% too heavy. I never liked it... maybe I'll change my mind
%\renewcommand{\CURPATH}{advanced/380_FAT12}
%\EN{% TODO translate
\mysection{Breaking simple executable cryptor}

I've got an executable file which is encrypted by relatively simple encryption.
\href{\GitHubBlobMasterURL/examples/simple_exec_crypto/files/cipher.bin}{Here is it} (only executable section is left here).

First, all encryption function does is just adds number of position in buffer to the byte.
Here is how this can be encoded in Python:

\begin{lstlisting}[caption=Python script,style=custompy]
#!/usr/bin/env python
def e(i, k):
    return chr ((ord(i)+k) % 256)

def encrypt(buf):
    return e(buf[0], 0)+ e(buf[1], 1)+ e(buf[2], 2) + e(buf[3], 3)+ e(buf[4], 4)+ e(buf[5], 5)+ e(buf[6], 6)+ e(buf[7], 7)+
           e(buf[8], 8)+ e(buf[9], 9)+ e(buf[10], 10)+ e(buf[11], 11)+ e(buf[12], 12)+ e(buf[13], 13)+ e(buf[14], 14)+ e(buf[15], 15)
\end{lstlisting}

Hence, if you encrypt buffer with 16 zeros, you'll get \emph{0, 1, 2, 3 ... 12, 13, 14, 15}.

\myindex{Propagating Cipher Block Chaining}
Propagating Cipher Block Chaining (PCBC) is also used, here is how it works:

\begin{figure}[H]
\centering
\myincludegraphics{examples/simple_exec_crypto/601px-PCBC_encryption.png}
\caption{Propagating Cipher Block Chaining encryption (image is taken from Wikipedia article)}
\end{figure}

The problem is that it's too boring to recover IV (Initialization Vector) each time.
Brute-force is also not an option, because IV is too long (16 bytes).
Let's see, if it's possible to recover IV for arbitrary encrypted executable file?

Let's try simple frequency analysis.
This is 32-bit x86 executable code, so let's gather statistics about most frequent bytes and opcodes.
I tried huge oracle.exe file from Oracle RDBMS version 11.2 for windows x86 and I've found that the most frequent byte (no surprise) is zero (~10\%).
The next most frequent byte is (again, no surprise) 0xFF (~5\%).
The next is 0x8B (~5\%).

\myindex{x86!\Instructions!MOV}
0x8B is opcode for \INS{MOV}, this is indeed one of the most busy x86 instructions.
Now what about popularity of zero byte?
If compiler needs to encode value bigger than 127, it has to use 32-bit displacement instead of 8-bit one, but large values are very rare,
so it is padded by zeros.
\myindex{x86!\Instructions!LEA}
\myindex{x86!\Instructions!PUSH}
\myindex{x86!\Instructions!CALL}
This is at least in \INS{LEA}, \INS{MOV}, \INS{PUSH}, \INS{CALL}.

For example:

\begin{lstlisting}[style=customasmx86]
8D B0 28 01 00 00                 lea     esi, [eax+128h]
8D BF 40 38 00 00                 lea     edi, [edi+3840h]
\end{lstlisting}

Displacements bigger than 127 are very popular, but they are rarely exceeds 0x10000
(indeed, such large memory buffers/structures are also rare).

Same story with \INS{MOV}, large constants are rare, the most heavily used are 0, 1, 10, 100, $2^n$, and so on.
Compiler has to pad small constants by zeros to represent them as 32-bit values:

\begin{lstlisting}[style=customasmx86]
BF 02 00 00 00                    mov     edi, 2
BF 01 00 00 00                    mov     edi, 1
\end{lstlisting}

Now about 00 and FF bytes combined: jumps (including conditional) and calls can pass execution flow forward or backwards, but very often,
within the limits of the current executable module.
If forward, displacement is not very big and also padded with zeros.
If backwards, displacement is represented as negative value, so padded with FF bytes.
For example, transfer execution flow forward:

\begin{lstlisting}[style=customasmx86]
E8 43 0C 00 00                    call    _function1
E8 5C 00 00 00                    call    _function2
0F 84 F0 0A 00 00                 jz      loc_4F09A0
0F 84 EB 00 00 00                 jz      loc_4EFBB8
\end{lstlisting}

Backwards:

\begin{lstlisting}[style=customasmx86]
E8 79 0C FE FF                    call    _function1
E8 F4 16 FF FF                    call    _function2
0F 84 F8 FB FF FF                 jz      loc_8212BC
0F 84 06 FD FF FF                 jz      loc_FF1E7D
\end{lstlisting}

FF byte is also very often occurred in negative displacements like these:

\begin{lstlisting}[style=customasmx86]
8D 85 1E FF FF FF                 lea     eax, [ebp-0E2h]
8D 95 F8 5C FF FF                 lea     edx, [ebp-0A308h]
\end{lstlisting}

So far so good. Now we have to try various 16-byte keys, decrypt executable section and measure how often 00, FF and 8B bytes are occurred.
Let's also keep in sight how PCBC decryption works:

\begin{figure}[H]
\centering
\myincludegraphics{examples/simple_exec_crypto/640px-PCBC_decryption.png}
\caption{Propagating Cipher Block Chaining decryption (image is taken from Wikipedia article)}
\end{figure}

The good news is that we don't really have to decrypt whole piece of data, but only slice by slice, this is exactly how I did in my previous example: \myref{XOR_mask_2}.

Now I'm trying all possible bytes (0..255) for each byte in key and just pick the byte producing maximal amount of 00/FF/8B bytes in a decrypted slice:

\begin{lstlisting}[style=custompy]
#!/usr/bin/env python
import sys, hexdump, array, string, operator

KEY_LEN=16

def chunks(l, n):
    # split n by l-byte chunks
    # https://stackoverflow.com/q/312443
    n = max(1, n)
    return [l[i:i + n] for i in range(0, len(l), n)]

def read_file(fname):
    file=open(fname, mode='rb')
    content=file.read()
    file.close()
    return content

def decrypt_byte (c, key):
    return chr((ord(c)-key) % 256)

def XOR_PCBC_step (IV, buf, k):
    prev=IV
    rt=""
    for c in buf:
	new_c=decrypt_byte(c, k)
        plain=chr(ord(new_c)^ord(prev))
	prev=chr(ord(c)^ord(plain))
	rt=rt+plain
    return rt

each_Nth_byte=[""]*KEY_LEN

content=read_file(sys.argv[1])
# split input by 16-byte chunks:
all_chunks=chunks(content, KEY_LEN)
for c in all_chunks:
    for i in range(KEY_LEN):
        each_Nth_byte[i]=each_Nth_byte[i] + c[i]

# try each byte of key
for N in range(KEY_LEN):
    print "N=", N
    stat={}
    for i in range(256):
        tmp_key=chr(i)
	tmp=XOR_PCBC_step(tmp_key,each_Nth_byte[N], N)
        # count 0, FFs and 8Bs in decrypted buffer:
	important_bytes=tmp.count('\x00')+tmp.count('\xFF')+tmp.count('\x8B')
	stat[i]=important_bytes
    sorted_stat = sorted(stat.iteritems(), key=operator.itemgetter(1), reverse=True)
    print sorted_stat[0]
\end{lstlisting}

(Source code can be downloaded \href{\GitHubBlobMasterURL/examples/simple_exec_crypto/files/decrypt.py}{here}.)

I run it and here is a key for which 00/FF/8B bytes presence in decrypted buffer is maximal:

\begin{lstlisting}
N= 0
(147, 1224)
N= 1
(94, 1327)
N= 2
(252, 1223)
N= 3
(218, 1266)
N= 4
(38, 1209)
N= 5
(192, 1378)
N= 6
(199, 1204)
N= 7
(213, 1332)
N= 8
(225, 1251)
N= 9
(112, 1223)
N= 10
(143, 1177)
N= 11
(108, 1286)
N= 12
(10, 1164)
N= 13
(3, 1271)
N= 14
(128, 1253)
N= 15
(232, 1330)
\end{lstlisting}

Let's write decryption utility with the key we got:

\begin{lstlisting}[style=custompy]
#!/usr/bin/env python
import sys, hexdump, array

def xor_strings(s,t):
    # \verb|https://en.wikipedia.org/wiki/XOR_cipher#Example_implementation|
    """xor two strings together"""
    return "".join(chr(ord(a)^ord(b)) for a,b in zip(s,t))

IV=array.array('B', [147, 94, 252, 218, 38, 192, 199, 213, 225, 112, 143, 108, 10, 3, 128, 232]).tostring()

def chunks(l, n):
    n = max(1, n)
    return [l[i:i + n] for i in range(0, len(l), n)]

def read_file(fname):
    file=open(fname, mode='rb')
    content=file.read()
    file.close()
    return content

def decrypt_byte(i, k):
    return chr ((ord(i)-k) % 256)

def decrypt(buf):
    return "".join(decrypt_byte(buf[i], i) for i in range(16))

fout=open(sys.argv[2], mode='wb')

prev=IV
content=read_file(sys.argv[1])
tmp=chunks(content, 16)
for c in tmp:
    new_c=decrypt(c)
    p=xor_strings (new_c, prev)
    prev=xor_strings(c, p)
    fout.write(p)
fout.close()
\end{lstlisting}

(Source code can be downloaded \href{\GitHubBlobMasterURL/examples/simple_exec_crypto/files/decrypt2.py}{here}.)

Let's check resulting file:

\lstinputlisting{examples/simple_exec_crypto/objdump_result.txt}

Yes, this is seems correctly disassembled piece of x86 code.
The whole decryped file can be downloaded \href{\GitHubBlobMasterURL/examples/simple_exec_crypto/files/decrypted.bin}{here}.

In fact, this is text section from regedit.exe from Windows 7.
But this example is based on a real case I encountered, so just executable is different (and key), algorithm is the same.

\subsection{Other ideas to consider}

What if I would fail with such simple frequency analysis?
There are other ideas on how to measure correctness of decrypted/decompressed x86 code:

\begin{itemize}

\item Many modern compilers aligns functions on 0x10 border.
So the space left before is filled with NOPs (0x90) or other NOP instructions with known opcodes: \myref{sec:npad}.

\item Perhaps, the most frequent pattern in any assembly language is function call:\\
\TT{PUSH chain / CALL / ADD ESP, X}.
This sequence can easily detected and found.
I've even gathered statistics about average number of function arguments: \myref{args_stat}.
(Hence, this is average length of PUSH chain.)

\end{itemize}

Read more about incorrectly/correctly disassembled code: \myref{ISA_detect}.
}\FR{\mysection{Fonction presque vide}
\label{Boolector}
\myindex{Boolector}
\myindex{x86!\Instructions!JMP}

Ceci est un morceau de code réel que j'ai trouvé dans Boolector\footnote{\url{https://boolector.github.io/}}:

\lstinputlisting[style=customc]{patterns/025_almost_empty/boolectormain.c}

Pourquoi quelqu'un ferait-il comme ça?
Je ne sais pas mais mon hypothèse est que \verb|boolector_main()| peut être compilée
dans une sorte de DLL ou bibliothèque dynamique, et appelée depuis une suite de test.
Certainement qu'une suite de test peut préparer les variables argc/argv comme
le ferait \ac{CRT}.

Il est intéressant de voir comment c'est compilé:

\lstinputlisting[caption=GCC 8.2 x64 \NonOptimizing (\assemblyOutput),style=customasmx86]{patterns/025_almost_empty/boolectormain_O0.s}

Ceci est OK, le prologue (non optimisé) déplace inutilement deux arguments,
\INS{CALL}, épilogue, \INS{RET}.
Mais regardons la version optimisée:

\lstinputlisting[caption=GCC 8.2 x64 \Optimizing (\assemblyOutput),style=customasmx86]{patterns/025_almost_empty/boolectormain_O3.s}

Aussi simple que ça: la pile et les registres ne sont pas touchés et \verb|boolector_main()|
a le même ensemble d'arguments.
Donc, tout ce que nous avons à faire est de passer l'exécution à une autre adresse.

Ceci est proche d'une \glslink{thunk function}{fonction thunk}.

Nous verons queelque chose de plus avancé plus tard: \myref{ARM_B_to_printf}, \myref{JMP_instead_of_RET}.
}

\EN{\mysection{More about pointers}
\myindex{\CLanguageElements!\Pointers}
\label{label_pointers}

\epigraph{The way C handles pointers, for example, was a brilliant innovation;
it solved a lot of problems that we had before in data structuring and
made the programs look good afterwards.}{Donald Knuth, interview (1993)}

For those, who still have hard time understanding \CCpp pointers, here are more examples.
Some of them are weird and serves only demonstration purpose:
use them in production code only if you really know what you're doing.

% subsections:
\input{advanced/450_more_ptrs/1_EN}
\input{advanced/450_more_ptrs/2_EN}
\input{advanced/450_more_ptrs/3_EN}
\input{advanced/450_more_ptrs/4_EN}
\input{advanced/450_more_ptrs/5_EN}
\input{advanced/450_more_ptrs/6_EN}
\input{advanced/450_more_ptrs/61_EN}
\input{advanced/450_more_ptrs/62_EN}
\input{advanced/450_more_ptrs/7_EN}
\input{advanced/450_more_ptrs/Oracle_and_GC_EN}

}\RU{\mysection{Больше об указателях}
\myindex{\CLanguageElements!\Pointers}
\label{label_pointers}

\epigraph{The way C handles pointers, for example, was a brilliant innovation;
it solved a lot of problems that we had before in data structuring and
made the programs look good afterwards.}{Дональд Кнут, интервью (1993)}

Для тех, кому все еще трудно понимать указатели в \CCpp{}, вот еще примеры.
Некоторые из них крайне странные и служат только демонстрационным целям:
использовать подобное в production-коде можно только если вы действительно понимаете, что вы делаете.

\input{advanced/450_more_ptrs/1_RU}
\input{advanced/450_more_ptrs/2_RU}
\input{advanced/450_more_ptrs/3_RU}
\input{advanced/450_more_ptrs/4_RU}
\input{advanced/450_more_ptrs/5_RU}
\input{advanced/450_more_ptrs/6_RU}
\input{advanced/450_more_ptrs/61_RU}
\input{advanced/450_more_ptrs/62_RU}
\input{advanced/450_more_ptrs/7_RU}
% \input{advanced/450_more_ptrs/Oracle_and_GC_RU}

}\FR{\mysection{Plus loin avec les pointeurs}
\myindex{\CLanguageElements!\Pointers}
\label{label_pointers}

\epigraph{The way C handles pointers, for example, was a brilliant innovation;
it solved a lot of problems that we had before in data structuring and
made the programs look good afterwards.}{Donald Knuth, interview (1993)}

Pour ceux qui veulent se casser la tête à comprendre les pointeurs \CCpp, voici plus
d'exemples.
Certains d'entre eux sont bizarres et ne servent qu'à des fins de démonstration:
utilisez-les en production uniquement si vous savez vraiment ce que vous faites.

\input{advanced/450_more_ptrs/1_FR}
\input{advanced/450_more_ptrs/2_FR}
\input{advanced/450_more_ptrs/3_FR}
\input{advanced/450_more_ptrs/4_FR}
\input{advanced/450_more_ptrs/5_FR}
\input{advanced/450_more_ptrs/6_FR}
\input{advanced/450_more_ptrs/61_FR}
%\input{advanced/450_more_ptrs/62_FR}
\input{advanced/450_more_ptrs/7_FR}
\input{advanced/450_more_ptrs/Oracle_and_GC_FR}

}

\EN{% TODO translate
\mysection{Breaking simple executable cryptor}

I've got an executable file which is encrypted by relatively simple encryption.
\href{\GitHubBlobMasterURL/examples/simple_exec_crypto/files/cipher.bin}{Here is it} (only executable section is left here).

First, all encryption function does is just adds number of position in buffer to the byte.
Here is how this can be encoded in Python:

\begin{lstlisting}[caption=Python script,style=custompy]
#!/usr/bin/env python
def e(i, k):
    return chr ((ord(i)+k) % 256)

def encrypt(buf):
    return e(buf[0], 0)+ e(buf[1], 1)+ e(buf[2], 2) + e(buf[3], 3)+ e(buf[4], 4)+ e(buf[5], 5)+ e(buf[6], 6)+ e(buf[7], 7)+
           e(buf[8], 8)+ e(buf[9], 9)+ e(buf[10], 10)+ e(buf[11], 11)+ e(buf[12], 12)+ e(buf[13], 13)+ e(buf[14], 14)+ e(buf[15], 15)
\end{lstlisting}

Hence, if you encrypt buffer with 16 zeros, you'll get \emph{0, 1, 2, 3 ... 12, 13, 14, 15}.

\myindex{Propagating Cipher Block Chaining}
Propagating Cipher Block Chaining (PCBC) is also used, here is how it works:

\begin{figure}[H]
\centering
\myincludegraphics{examples/simple_exec_crypto/601px-PCBC_encryption.png}
\caption{Propagating Cipher Block Chaining encryption (image is taken from Wikipedia article)}
\end{figure}

The problem is that it's too boring to recover IV (Initialization Vector) each time.
Brute-force is also not an option, because IV is too long (16 bytes).
Let's see, if it's possible to recover IV for arbitrary encrypted executable file?

Let's try simple frequency analysis.
This is 32-bit x86 executable code, so let's gather statistics about most frequent bytes and opcodes.
I tried huge oracle.exe file from Oracle RDBMS version 11.2 for windows x86 and I've found that the most frequent byte (no surprise) is zero (~10\%).
The next most frequent byte is (again, no surprise) 0xFF (~5\%).
The next is 0x8B (~5\%).

\myindex{x86!\Instructions!MOV}
0x8B is opcode for \INS{MOV}, this is indeed one of the most busy x86 instructions.
Now what about popularity of zero byte?
If compiler needs to encode value bigger than 127, it has to use 32-bit displacement instead of 8-bit one, but large values are very rare,
so it is padded by zeros.
\myindex{x86!\Instructions!LEA}
\myindex{x86!\Instructions!PUSH}
\myindex{x86!\Instructions!CALL}
This is at least in \INS{LEA}, \INS{MOV}, \INS{PUSH}, \INS{CALL}.

For example:

\begin{lstlisting}[style=customasmx86]
8D B0 28 01 00 00                 lea     esi, [eax+128h]
8D BF 40 38 00 00                 lea     edi, [edi+3840h]
\end{lstlisting}

Displacements bigger than 127 are very popular, but they are rarely exceeds 0x10000
(indeed, such large memory buffers/structures are also rare).

Same story with \INS{MOV}, large constants are rare, the most heavily used are 0, 1, 10, 100, $2^n$, and so on.
Compiler has to pad small constants by zeros to represent them as 32-bit values:

\begin{lstlisting}[style=customasmx86]
BF 02 00 00 00                    mov     edi, 2
BF 01 00 00 00                    mov     edi, 1
\end{lstlisting}

Now about 00 and FF bytes combined: jumps (including conditional) and calls can pass execution flow forward or backwards, but very often,
within the limits of the current executable module.
If forward, displacement is not very big and also padded with zeros.
If backwards, displacement is represented as negative value, so padded with FF bytes.
For example, transfer execution flow forward:

\begin{lstlisting}[style=customasmx86]
E8 43 0C 00 00                    call    _function1
E8 5C 00 00 00                    call    _function2
0F 84 F0 0A 00 00                 jz      loc_4F09A0
0F 84 EB 00 00 00                 jz      loc_4EFBB8
\end{lstlisting}

Backwards:

\begin{lstlisting}[style=customasmx86]
E8 79 0C FE FF                    call    _function1
E8 F4 16 FF FF                    call    _function2
0F 84 F8 FB FF FF                 jz      loc_8212BC
0F 84 06 FD FF FF                 jz      loc_FF1E7D
\end{lstlisting}

FF byte is also very often occurred in negative displacements like these:

\begin{lstlisting}[style=customasmx86]
8D 85 1E FF FF FF                 lea     eax, [ebp-0E2h]
8D 95 F8 5C FF FF                 lea     edx, [ebp-0A308h]
\end{lstlisting}

So far so good. Now we have to try various 16-byte keys, decrypt executable section and measure how often 00, FF and 8B bytes are occurred.
Let's also keep in sight how PCBC decryption works:

\begin{figure}[H]
\centering
\myincludegraphics{examples/simple_exec_crypto/640px-PCBC_decryption.png}
\caption{Propagating Cipher Block Chaining decryption (image is taken from Wikipedia article)}
\end{figure}

The good news is that we don't really have to decrypt whole piece of data, but only slice by slice, this is exactly how I did in my previous example: \myref{XOR_mask_2}.

Now I'm trying all possible bytes (0..255) for each byte in key and just pick the byte producing maximal amount of 00/FF/8B bytes in a decrypted slice:

\begin{lstlisting}[style=custompy]
#!/usr/bin/env python
import sys, hexdump, array, string, operator

KEY_LEN=16

def chunks(l, n):
    # split n by l-byte chunks
    # https://stackoverflow.com/q/312443
    n = max(1, n)
    return [l[i:i + n] for i in range(0, len(l), n)]

def read_file(fname):
    file=open(fname, mode='rb')
    content=file.read()
    file.close()
    return content

def decrypt_byte (c, key):
    return chr((ord(c)-key) % 256)

def XOR_PCBC_step (IV, buf, k):
    prev=IV
    rt=""
    for c in buf:
	new_c=decrypt_byte(c, k)
        plain=chr(ord(new_c)^ord(prev))
	prev=chr(ord(c)^ord(plain))
	rt=rt+plain
    return rt

each_Nth_byte=[""]*KEY_LEN

content=read_file(sys.argv[1])
# split input by 16-byte chunks:
all_chunks=chunks(content, KEY_LEN)
for c in all_chunks:
    for i in range(KEY_LEN):
        each_Nth_byte[i]=each_Nth_byte[i] + c[i]

# try each byte of key
for N in range(KEY_LEN):
    print "N=", N
    stat={}
    for i in range(256):
        tmp_key=chr(i)
	tmp=XOR_PCBC_step(tmp_key,each_Nth_byte[N], N)
        # count 0, FFs and 8Bs in decrypted buffer:
	important_bytes=tmp.count('\x00')+tmp.count('\xFF')+tmp.count('\x8B')
	stat[i]=important_bytes
    sorted_stat = sorted(stat.iteritems(), key=operator.itemgetter(1), reverse=True)
    print sorted_stat[0]
\end{lstlisting}

(Source code can be downloaded \href{\GitHubBlobMasterURL/examples/simple_exec_crypto/files/decrypt.py}{here}.)

I run it and here is a key for which 00/FF/8B bytes presence in decrypted buffer is maximal:

\begin{lstlisting}
N= 0
(147, 1224)
N= 1
(94, 1327)
N= 2
(252, 1223)
N= 3
(218, 1266)
N= 4
(38, 1209)
N= 5
(192, 1378)
N= 6
(199, 1204)
N= 7
(213, 1332)
N= 8
(225, 1251)
N= 9
(112, 1223)
N= 10
(143, 1177)
N= 11
(108, 1286)
N= 12
(10, 1164)
N= 13
(3, 1271)
N= 14
(128, 1253)
N= 15
(232, 1330)
\end{lstlisting}

Let's write decryption utility with the key we got:

\begin{lstlisting}[style=custompy]
#!/usr/bin/env python
import sys, hexdump, array

def xor_strings(s,t):
    # \verb|https://en.wikipedia.org/wiki/XOR_cipher#Example_implementation|
    """xor two strings together"""
    return "".join(chr(ord(a)^ord(b)) for a,b in zip(s,t))

IV=array.array('B', [147, 94, 252, 218, 38, 192, 199, 213, 225, 112, 143, 108, 10, 3, 128, 232]).tostring()

def chunks(l, n):
    n = max(1, n)
    return [l[i:i + n] for i in range(0, len(l), n)]

def read_file(fname):
    file=open(fname, mode='rb')
    content=file.read()
    file.close()
    return content

def decrypt_byte(i, k):
    return chr ((ord(i)-k) % 256)

def decrypt(buf):
    return "".join(decrypt_byte(buf[i], i) for i in range(16))

fout=open(sys.argv[2], mode='wb')

prev=IV
content=read_file(sys.argv[1])
tmp=chunks(content, 16)
for c in tmp:
    new_c=decrypt(c)
    p=xor_strings (new_c, prev)
    prev=xor_strings(c, p)
    fout.write(p)
fout.close()
\end{lstlisting}

(Source code can be downloaded \href{\GitHubBlobMasterURL/examples/simple_exec_crypto/files/decrypt2.py}{here}.)

Let's check resulting file:

\lstinputlisting{examples/simple_exec_crypto/objdump_result.txt}

Yes, this is seems correctly disassembled piece of x86 code.
The whole decryped file can be downloaded \href{\GitHubBlobMasterURL/examples/simple_exec_crypto/files/decrypted.bin}{here}.

In fact, this is text section from regedit.exe from Windows 7.
But this example is based on a real case I encountered, so just executable is different (and key), algorithm is the same.

\subsection{Other ideas to consider}

What if I would fail with such simple frequency analysis?
There are other ideas on how to measure correctness of decrypted/decompressed x86 code:

\begin{itemize}

\item Many modern compilers aligns functions on 0x10 border.
So the space left before is filled with NOPs (0x90) or other NOP instructions with known opcodes: \myref{sec:npad}.

\item Perhaps, the most frequent pattern in any assembly language is function call:\\
\TT{PUSH chain / CALL / ADD ESP, X}.
This sequence can easily detected and found.
I've even gathered statistics about average number of function arguments: \myref{args_stat}.
(Hence, this is average length of PUSH chain.)

\end{itemize}

Read more about incorrectly/correctly disassembled code: \myref{ISA_detect}.
}\RU{\subsection{Простое шифрование используя XOR-маску}
\label{XOR_mask_1}

Я нашел одну старую игру в стиле interactive fiction в архиве \emph{if-archive}\footnote{\url{http://www.ifarchive.org/}}:

\begin{lstlisting}
The New Castle v3.5 - Text/Adventure Game
in the style of the original Infocom (tm)
type games, Zork, Collosal Cave (Adventure),
etc.  Can you solve the mystery of the
abandoned castle?
Shareware from Software Customization.
Software Customization [ASP] Version 3.5 Feb. 2000
\end{lstlisting}

Можно скачать здесь: \url{\GitHubBlobMasterURL/ff/XOR/mask_1/files/newcastle.tgz}.

Там внутри есть файл (с названием \emph{castle.dbf}), который явно зашифрован, но не настоящим криптоалгоритмом,
и оне сжат, это что-то куда проще.
Я бы даже не стал измерять уровень энтропии (\myref{entropy}) этого файла, потому что я итак уверен, что он низкий.
Вот как он выглядит в Midnight Commander:

\begin{figure}[H]
\centering
\myincludegraphics{ff/XOR/mask_1/mc_encrypted.png}
\caption{Зашифрованный файл в Midnight Commander}
\end{figure}

Зашифрованный файл можно скачать здесь:
\url{\GitHubBlobMasterURL/ff/XOR/mask_1/files/castle.dbf.bz2}.

Можно ли расшифровать его без доступа к программе, используя просто этот файл?

Тут явно просматривается повторяющаяся строка. 
Если использовалось простое шифрование с XOR-маской, такие повторяющиеся строки это явное свидетельство,
потому что, вероятно, тут были длинные лакуны с нулевыми байтами, которые, в свою очередь, присутствуют
во мноигих исполняемых файлах, и в остальных бинарных файлах.

\myindex{UNIX!xxd}
Вот дам начала этого файла используя утилиту \emph{xxd} из UNIX:

\lstinputlisting{ff/XOR/mask_1/xxd_result.txt}

Давайте держаться за повторяющуюся строку \TT{iubgv}.
Глядя на этот дамп, мы можем легко увидеть, что период повторений этой строки это 0x51 или 81.
Вероятно, 81 это длина блока?
Длина файла 1658961, и она может быть поделена на 81 без остатка (и тогда там 20481 блоков).

Теперь я буду использовать Mathematica для анализа, есть ли тут повторяющиеся 81-байтные блоки в файле?
Я разделю входной файл на 81-байтные блоки и затем использую ф-цию
\emph{Tally[]}\footnote{\url{https://reference.wolfram.com/language/ref/Tally.html}}
которая просто считает, сколько раз каждый элемент встретился во входном списке.
Вывод Tally не отсортирован, так что я также добавлю ф-цию \emph{Sort[]} для сортировки его по кол-ву вхождений
в нисходящем порядке.

\begin{lstlisting}[style=custommath]
input = BinaryReadList["/home/dennis/.../castle.dbf"];

blocks = Partition[input, 81];

stat = Sort[Tally[blocks], #1[[2]] > #2[[2]] &]
\end{lstlisting}

И вот вывод:

\begin{lstlisting}[style=custommath]
{{{80, 103, 2, 116, 113, 102, 118, 25, 99, 8, 19, 23, 116, 125, 107, 
   25, 99, 109, 114, 102, 14, 121, 115, 31, 9, 117, 113, 111, 5, 4, 
   127, 28, 122, 101, 8, 110, 14, 18, 124, 106, 16, 20, 104, 119, 8, 
   109, 26, 106, 9, 97, 13, 99, 15, 119, 20, 105, 117, 98, 103, 118, 
   1, 126, 29, 97, 122, 17, 15, 114, 110, 3, 5, 125, 125, 99, 126, 
   119, 102, 30, 122, 2, 117}, 1739}, 
{{80, 100, 2, 116, 113, 102, 118, 25, 99, 8, 19, 23, 116, 
   125, 107, 25, 99, 109, 114, 102, 14, 121, 115, 31, 9, 117, 113, 
   111, 5, 4, 127, 28, 122, 101, 8, 110, 14, 18, 124, 106, 16, 20, 
   104, 119, 8, 109, 26, 106, 9, 97, 13, 99, 15, 119, 20, 105, 117, 
   98, 103, 118, 1, 126, 29, 97, 122, 17, 15, 114, 110, 3, 5, 125, 
   125, 99, 126, 119, 102, 30, 122, 2, 117}, 1422}, 
{{80, 101, 2, 116, 113, 102, 118, 25, 99, 8, 19, 23, 116, 
   125, 107, 25, 99, 109, 114, 102, 14, 121, 115, 31, 9, 117, 113, 
   111, 5, 4, 127, 28, 122, 101, 8, 110, 14, 18, 124, 106, 16, 20, 
   104, 119, 8, 109, 26, 106, 9, 97, 13, 99, 15, 119, 20, 105, 117, 
   98, 103, 118, 1, 126, 29, 97, 122, 17, 15, 114, 110, 3, 5, 125, 
   125, 99, 126, 119, 102, 30, 122, 2, 117}, 1012},
{{80, 120, 2, 116, 113, 102, 118, 25, 99, 8, 19, 23, 116, 
   125, 107, 25, 99, 109, 114, 102, 14, 121, 115, 31, 9, 117, 113, 
   111, 5, 4, 127, 28, 122, 101, 8, 110, 14, 18, 124, 106, 16, 20, 
   104, 119, 8, 109, 26, 106, 9, 97, 13, 99, 15, 119, 20, 105, 117, 
   98, 103, 118, 1, 126, 29, 97, 122, 17, 15, 114, 110, 3, 5, 125, 
   125, 99, 126, 119, 102, 30, 122, 2, 117}, 377},

...

{{80, 2, 74, 49, 113, 21, 62, 88, 39, 71, 68, 23, 63, 51, 36, 78, 48, 
   108, 114, 102, 14, 121, 115, 31, 9, 117, 113, 111, 5, 4, 127, 28, 
   122, 101, 8, 110, 14, 18, 124, 106, 16, 20, 104, 119, 8, 109, 26, 
   106, 9, 97, 13, 99, 15, 119, 20, 105, 117, 98, 103, 118, 1, 126, 
   29, 97, 122, 17, 15, 114, 110, 3, 5, 125, 125, 99, 126, 119, 102, 
   30, 122, 2, 117}, 1},
{{80, 1, 74, 59, 113, 45, 56, 86, 52, 91, 19, 64, 60, 60, 63, 
   25, 38, 59, 59, 42, 14, 53, 38, 77, 66, 38, 113, 38, 75, 4, 43, 84,
    63, 101, 64, 43, 79, 64, 40, 57, 16, 91, 46, 119, 69, 40, 84, 117,
    9, 97, 13, 99, 15, 119, 20, 105, 117, 98, 103, 118, 1, 126, 29, 
   97, 122, 17, 15, 114, 110, 3, 5, 125, 125, 99, 126, 119, 102, 30, 
   122, 2, 117}, 1},
{{80, 2, 74, 49, 113, 49, 51, 92, 39, 8, 92, 81, 116, 62, 57, 
   80, 46, 40, 114, 36, 75, 56, 33, 76, 9, 55, 56, 59, 81, 65, 45, 28,
    60, 55, 93, 39, 90, 28, 124, 106, 16, 20, 104, 119, 8, 109, 26, 
   106, 9, 97, 13, 99, 15, 119, 20, 105, 117, 98, 103, 118, 1, 126, 
   29, 97, 122, 17, 15, 114, 110, 3, 5, 125, 125, 99, 126, 119, 102, 
   30, 122, 2, 117}, 1}}
\end{lstlisting}

Вывод Tally это список пар, каждая пара это 81-байтный блок и количество раз, сколько он встретился в файле.
Мы видим, что наиболее частно встречающийся блок это первый, он встретился 1739 раз.
Второй встретился 1422 раза. Есть и другие: 1012 раза, 377 раз, итд.
81-байтные блоки, встреченные лишь один раз, находятся в конце вывода.

Попробуем сравнить эти блоки. Первый и второй.
Есть ли в Mathematica ф-ция для сравнения списков/массивов?
Наверняка есть, но в педагогических целях, я буду использоват операцию XOR для сравнения.
Действительно: если байты во входных массивах равны друг другу, результат операции XOR это 0.
Если не равны, результат будет ненулевой.

Сравним первый блок (встречается 1739 раз) и второй (встречается 1422 раз):

\begin{lstlisting}[style=custommath]
In[]:= BitXor[stat[[1]][[1]], stat[[2]][[1]]]
Out[]= {0, 3, 0, 0, 0, 0, 0, 0, 0, 0, 0, 0, 0, 0, 0, 0, 0, 0, 0, \
0, 0, 0, 0, 0, 0, 0, 0, 0, 0, 0, 0, 0, 0, 0, 0, 0, 0, 0, 0, 0, 0, 0, \
0, 0, 0, 0, 0, 0, 0, 0, 0, 0, 0, 0, 0, 0, 0, 0, 0, 0, 0, 0, 0, 0, 0, \
0, 0, 0, 0, 0, 0, 0, 0, 0, 0, 0, 0, 0, 0, 0, 0}
\end{lstlisting}

Они отличаются только вторым байтом.

Сравним второй блок (встречается 1422 раза) и третий (встречается 1012 раз):

\begin{lstlisting}[style=custommath]
In[]:= BitXor[stat[[2]][[1]], stat[[3]][[1]]]
Out[]= {0, 1, 0, 0, 0, 0, 0, 0, 0, 0, 0, 0, 0, 0, 0, 0, 0, 0, 0, \
0, 0, 0, 0, 0, 0, 0, 0, 0, 0, 0, 0, 0, 0, 0, 0, 0, 0, 0, 0, 0, 0, 0, \
0, 0, 0, 0, 0, 0, 0, 0, 0, 0, 0, 0, 0, 0, 0, 0, 0, 0, 0, 0, 0, 0, 0, \
0, 0, 0, 0, 0, 0, 0, 0, 0, 0, 0, 0, 0, 0, 0, 0}
\end{lstlisting}

Они тоже отличаются только вторым байтом.

Так или иначе, попробуем использовать самый встречающийся блок как XOR-ключ и попробуем расшифровать первые 4 81-байтных
блока в файле:

\begin{lstlisting}[style=custommath]
In[]:= key = stat[[1]][[1]]
Out[]= {80, 103, 2, 116, 113, 102, 118, 25, 99, 8, 19, 23, 116, \
125, 107, 25, 99, 109, 114, 102, 14, 121, 115, 31, 9, 117, 113, 111, \
5, 4, 127, 28, 122, 101, 8, 110, 14, 18, 124, 106, 16, 20, 104, 119, \
8, 109, 26, 106, 9, 97, 13, 99, 15, 119, 20, 105, 117, 98, 103, 118, \
1, 126, 29, 97, 122, 17, 15, 114, 110, 3, 5, 125, 125, 99, 126, 119, \
102, 30, 122, 2, 117}

In[]:= ToASCII[val_] := If[val == 0, " ", FromCharacterCode[val, "PrintableASCII"]]

In[]:= DecryptBlockASCII[blk_] := Map[ToASCII[#] &, BitXor[key, blk]]

In[]:= DecryptBlockASCII[blocks[[1]]]
Out[]= {" ", " ", " ", " ", " ", " ", " ", " ", " ", " ", " ", " \
", " ", " ", " ", " ", " ", " ", " ", " ", " ", " ", " ", " ", " ", " \
", " ", " ", " ", " ", " ", " ", " ", " ", " ", " ", " ", " ", " ", " \
", " ", " ", " ", " ", " ", " ", " ", " ", " ", " ", " ", " ", " ", " \
", " ", " ", " ", " ", " ", " ", " ", " ", " ", " ", " ", " ", " ", " \
", " ", " ", " ", " ", " ", " ", " ", " ", " ", " ", " ", " ", " "}

In[]:= DecryptBlockASCII[blocks[[2]]]
Out[]= {" ", "e", "H", "E", " ", "W", "E", "E", "D", " ", "O", \
"F", " ", "C", "R", "I", "M", "E", " ", "B", "E", "A", "R", "S", " ", \
"B", "I", "T", "T", "E", "R", " ", "F", "R", "U", "I", "T", "?", \
" ", " ", " ", " ", " ", " ", " ", " ", " ", " ", " ", " ", " ", " ", \
" ", " ", " ", " ", " ", " ", " ", " ", " ", " ", " ", " ", " ", " ", \
" ", " ", " ", " ", " ", " ", " ", " ", " ", " ", " ", " ", " ", " ", \
" "}

In[]:= DecryptBlockASCII[blocks[[3]]]
Out[]= {" ", "?", " ", " ", " ", " ", " ", " ", " ", " ", " \
", " ", " ", " ", " ", " ", " ", " ", " ", " ", " ", " ", " ", " ", " \
", " ", " ", " ", " ", " ", " ", " ", " ", " ", " ", " ", " ", " ", " \
", " ", " ", " ", " ", " ", " ", " ", " ", " ", " ", " ", " ", " ", " \
", " ", " ", " ", " ", " ", " ", " ", " ", " ", " ", " ", " ", " ", " \
", " ", " ", " ", " ", " ", " ", " ", " ", " ", " ", " ", " ", " ", " \
"}

In[]:= DecryptBlockASCII[blocks[[4]]]
Out[]= {" ", "f", "H", "O", " ", "K", "N", "O", "W", "S", " ", \
"W", "H", "A", "T", " ", "E", "V", "I", "L", " ", "L", "U", "R", "K", \
"S", " ", "I", "N", " ", "T", "H", "E", " ", "H", "E", "A", "R", "T", \
"S", " ", "O", "F", " ", "M", "E", "N", "?", " ", " ", " ", " ", \
" ", " ", " ", " ", " ", " ", " ", " ", " ", " ", " ", " ", " ", " ", \
" ", " ", " ", " ", " ", " ", " ", " ", " ", " ", " ", " ", " ", " ", \
" "}
\end{lstlisting}

(Я заменил непечатаемые символы на \q{?}.)

Мы видим что первый и третий блоки пустые (или почти пустые),
но второй и четвертый имеют ясно различимые английские слова/фразы.
Похоже что наше предположение насчет ключа верно (как минимум частично).
Это означает, что самый встречающийся 81-байтный блок в файле находится в местах лакун с нулевыми байтами
или что-то в этом роде.

Попробуем расшифровать весь файл:

\begin{lstlisting}[style=custommath]
DecryptBlock[blk_] := BitXor[key, blk]

decrypted = Map[DecryptBlock[#] &, blocks];

BinaryWrite["/home/dennis/.../tmp", Flatten[decrypted]]

Close["/home/dennis/.../tmp"]
\end{lstlisting}

\begin{figure}[H]
\centering
\myincludegraphics{ff/XOR/mask_1/mc_decrypted1.png}
\caption{Расшифрованный файл в Midnight Commander, первая попытка}
\end{figure}

Выглядит как английские фразы для какой-то игры, но что-то не так.
Прежде всего, регистр инвертирован: фразы и некоторые слова начинаются со строчных букв,
в то время как остальные буквы заглавные.
Также, некоторые фразы начинаются с не тех букв.
Посмотрите на самую первую фразу: \q{eHE WEED OF CRIME BEARS BITTER FRUIT}.
Что такое \q{eHE}? Разве не \q{tHE} тут должно быть?
Возможно ли что наш ключ для дешифрования имеет неверный байт в этом месте?

Посмотрим снова на второй блок в файле, на ключ и на результат дешифрования:

\begin{lstlisting}[style=custommath]
In[]:= blocks[[2]]
Out[]= {80, 2, 74, 49, 113, 49, 51, 92, 39, 8, 92, 81, 116, 62, \
57, 80, 46, 40, 114, 36, 75, 56, 33, 76, 9, 55, 56, 59, 81, 65, 45, \
28, 60, 55, 93, 39, 90, 28, 124, 106, 16, 20, 104, 119, 8, 109, 26, \
106, 9, 97, 13, 99, 15, 119, 20, 105, 117, 98, 103, 118, 1, 126, 29, \
97, 122, 17, 15, 114, 110, 3, 5, 125, 125, 99, 126, 119, 102, 30, \
122, 2, 117}

In[]:= key
Out[]= {80, 103, 2, 116, 113, 102, 118, 25, 99, 8, 19, 23, 116, \
125, 107, 25, 99, 109, 114, 102, 14, 121, 115, 31, 9, 117, 113, 111, \
5, 4, 127, 28, 122, 101, 8, 110, 14, 18, 124, 106, 16, 20, 104, 119, \
8, 109, 26, 106, 9, 97, 13, 99, 15, 119, 20, 105, 117, 98, 103, 118, \
1, 126, 29, 97, 122, 17, 15, 114, 110, 3, 5, 125, 125, 99, 126, 119, \
102, 30, 122, 2, 117}

In[]:= BitXor[key, blocks[[2]]]
Out[]= {0, 101, 72, 69, 0, 87, 69, 69, 68, 0, 79, 70, 0, 67, 82, \
73, 77, 69, 0, 66, 69, 65, 82, 83, 0, 66, 73, 84, 84, 69, 82, 0, 70, \
82, 85, 73, 84, 14, 0, 0, 0, 0, 0, 0, 0, 0, 0, 0, 0, 0, 0, 0, 0, 0, \
0, 0, 0, 0, 0, 0, 0, 0, 0, 0, 0, 0, 0, 0, 0, 0, 0, 0, 0, 0, 0, 0, 0, \
0, 0, 0, 0}
\end{lstlisting}

Зашифрованный байт это 2, байт из ключа это 103, $2 \oplus 103=101$ и 101 это ASCII-код символа \q{e}.
Чему должен равнятся этот байт ключа, чтобы ASCII-код был 116 (для символа  \q{t})?
$2 \oplus 116=118$, присвоим 118 второму байту в ключе \dots

\begin{lstlisting}[style=custommath]
key = {80, 118, 2, 116, 113, 102, 118, 25, 99, 8, 19, 23, 116, 125, 
  107, 25, 99, 109, 114, 102, 14, 121, 115, 31, 9, 117, 113, 111, 5, 
  4, 127, 28, 122, 101, 8, 110, 14, 18, 124, 106, 16, 20, 104, 119, 8,
   109, 26, 106, 9, 97, 13, 99, 15, 119, 20, 105, 117, 98, 103, 118, 
  1, 126, 29, 97, 122, 17, 15, 114, 110, 3, 5, 125, 125, 99, 126, 119,
   102, 30, 122, 2, 117}
\end{lstlisting}

\dots и снова дешифруем весь файл.

\begin{figure}[H]
\centering
\myincludegraphics{ff/XOR/mask_1/mc_decrypted2.png}
\caption{Дешифрованный файл в Midnight Commander, вторая попытка}
\end{figure}

Ух ты, теперь грамматика корректна, и все фразы начинаются с корректных букв.
Но все таки, регистр подозрителен.
С чего бы разработчику игры записывать их в такой манере?
Может быть наш ключ все еще неправилен?

% TODO ASCII table somewhere in the book
Изучая таблицу ASCII мы можем заметить что ASCII-коды для букв в верхнем и нижнем регистре отличаются только на один бит
(6-й бит, если считать с первого, 0b100000):

\begin{figure}[H]
\centering
\includegraphics[width=0.7\textwidth]{ascii.png}
\caption{7-битная таблица \ac{ASCII} в Emacs}
\end{figure}

6-й бит, выставленный в нулевом байте, В десятичном виде это будет 32.
Но 32 это ASCII-код пробела!

Действительно, можно менять регистр просто применяя XOR к ASCII-коду, с 32 (больше об этом: \myref{toupper_bit}).

Возможно ли, что пустые лакуны в файле это не нулевые байты, а скорее содержащие пробелы?
Еще раз модифицируем наш XOR-ключ (я про-XOR-ю каждый байт ключа с 32):

\begin{lstlisting}[style=custommath]
(* "32" это скаляр, и "key" это вектор, но это OK *)

In[]:= key3 = BitXor[32, key]
Out[]= {112, 86, 34, 84, 81, 70, 86, 57, 67, 40, 51, 55, 84, 93, 75, \
57, 67, 77, 82, 70, 46, 89, 83, 63, 41, 85, 81, 79, 37, 36, 95, 60, \
90, 69, 40, 78, 46, 50, 92, 74, 48, 52, 72, 87, 40, 77, 58, 74, 41, \
65, 45, 67, 47, 87, 52, 73, 85, 66, 71, 86, 33, 94, 61, 65, 90, 49, \
47, 82, 78, 35, 37, 93, 93, 67, 94, 87, 70, 62, 90, 34, 85}

In[]:= DecryptBlock[blk_] := BitXor[key3, blk]
\end{lstlisting}

И снова дешифруем входной файл:

\begin{figure}[H]
\centering
\myincludegraphics{ff/XOR/mask_1/mc_decrypted.png}
\caption{Дешифрованный файл в Midnight Commander, последняя попытка}
\end{figure}

(Расшифрованный файл доступен здесь:
\url{\GitHubBlobMasterURL/ff/XOR/mask_1/files/decrypted.dat.bz2}.)

Несомненно, это корректный исходный файл.
Да, и мы видим числа в начале каждого блока. Должно быть это и есть источник некорректного XOR-ключа.
Как выходит, самый встречающийся 81-байтный блок в файле это блок заполненный пробелами и содержащий символ \q{1} на месте
второго байта.
Действительно, как-то так получилось что многие блоки здесь перемежаются с этим блоком.
Может быть это что-то вроде выравнивания (padding) для коротких фраз/сообщений?
Другой часто встречающийся 81-байтный блок также заполнен пробелами, но с другой цифрой, следовательно,
они отличаются только вторым байтом.

Вот и всё! Теперь мы можем написать утилиту для зашифрования файла назад, и, может быть, модифицировать его перед этим

Файл для Mathematica можно скачать здесь:\\
\url{\GitHubBlobMasterURL/ff/XOR/mask_1/files/XOR_mask_1.nb}.

Итог: XOR-шифрование не надежно вообще. Вероятно, разработчик игры хотел просто скрыть внутренности игры от игрока,
ничего более серьезного.
Все же, шифрование вроде этого крайне популярно вследствии его простоты, так что многие реверс инженеры обычно хорошо
с этим знакомы.

}\FR{\mysection{Fonction presque vide}
\label{Boolector}
\myindex{Boolector}
\myindex{x86!\Instructions!JMP}

Ceci est un morceau de code réel que j'ai trouvé dans Boolector\footnote{\url{https://boolector.github.io/}}:

\lstinputlisting[style=customc]{patterns/025_almost_empty/boolectormain.c}

Pourquoi quelqu'un ferait-il comme ça?
Je ne sais pas mais mon hypothèse est que \verb|boolector_main()| peut être compilée
dans une sorte de DLL ou bibliothèque dynamique, et appelée depuis une suite de test.
Certainement qu'une suite de test peut préparer les variables argc/argv comme
le ferait \ac{CRT}.

Il est intéressant de voir comment c'est compilé:

\lstinputlisting[caption=GCC 8.2 x64 \NonOptimizing (\assemblyOutput),style=customasmx86]{patterns/025_almost_empty/boolectormain_O0.s}

Ceci est OK, le prologue (non optimisé) déplace inutilement deux arguments,
\INS{CALL}, épilogue, \INS{RET}.
Mais regardons la version optimisée:

\lstinputlisting[caption=GCC 8.2 x64 \Optimizing (\assemblyOutput),style=customasmx86]{patterns/025_almost_empty/boolectormain_O3.s}

Aussi simple que ça: la pile et les registres ne sont pas touchés et \verb|boolector_main()|
a le même ensemble d'arguments.
Donc, tout ce que nous avons à faire est de passer l'exécution à une autre adresse.

Ceci est proche d'une \glslink{thunk function}{fonction thunk}.

Nous verons queelque chose de plus avancé plus tard: \myref{ARM_B_to_printf}, \myref{JMP_instead_of_RET}.
}

\EN{% TODO translate
\mysection{Breaking simple executable cryptor}

I've got an executable file which is encrypted by relatively simple encryption.
\href{\GitHubBlobMasterURL/examples/simple_exec_crypto/files/cipher.bin}{Here is it} (only executable section is left here).

First, all encryption function does is just adds number of position in buffer to the byte.
Here is how this can be encoded in Python:

\begin{lstlisting}[caption=Python script,style=custompy]
#!/usr/bin/env python
def e(i, k):
    return chr ((ord(i)+k) % 256)

def encrypt(buf):
    return e(buf[0], 0)+ e(buf[1], 1)+ e(buf[2], 2) + e(buf[3], 3)+ e(buf[4], 4)+ e(buf[5], 5)+ e(buf[6], 6)+ e(buf[7], 7)+
           e(buf[8], 8)+ e(buf[9], 9)+ e(buf[10], 10)+ e(buf[11], 11)+ e(buf[12], 12)+ e(buf[13], 13)+ e(buf[14], 14)+ e(buf[15], 15)
\end{lstlisting}

Hence, if you encrypt buffer with 16 zeros, you'll get \emph{0, 1, 2, 3 ... 12, 13, 14, 15}.

\myindex{Propagating Cipher Block Chaining}
Propagating Cipher Block Chaining (PCBC) is also used, here is how it works:

\begin{figure}[H]
\centering
\myincludegraphics{examples/simple_exec_crypto/601px-PCBC_encryption.png}
\caption{Propagating Cipher Block Chaining encryption (image is taken from Wikipedia article)}
\end{figure}

The problem is that it's too boring to recover IV (Initialization Vector) each time.
Brute-force is also not an option, because IV is too long (16 bytes).
Let's see, if it's possible to recover IV for arbitrary encrypted executable file?

Let's try simple frequency analysis.
This is 32-bit x86 executable code, so let's gather statistics about most frequent bytes and opcodes.
I tried huge oracle.exe file from Oracle RDBMS version 11.2 for windows x86 and I've found that the most frequent byte (no surprise) is zero (~10\%).
The next most frequent byte is (again, no surprise) 0xFF (~5\%).
The next is 0x8B (~5\%).

\myindex{x86!\Instructions!MOV}
0x8B is opcode for \INS{MOV}, this is indeed one of the most busy x86 instructions.
Now what about popularity of zero byte?
If compiler needs to encode value bigger than 127, it has to use 32-bit displacement instead of 8-bit one, but large values are very rare,
so it is padded by zeros.
\myindex{x86!\Instructions!LEA}
\myindex{x86!\Instructions!PUSH}
\myindex{x86!\Instructions!CALL}
This is at least in \INS{LEA}, \INS{MOV}, \INS{PUSH}, \INS{CALL}.

For example:

\begin{lstlisting}[style=customasmx86]
8D B0 28 01 00 00                 lea     esi, [eax+128h]
8D BF 40 38 00 00                 lea     edi, [edi+3840h]
\end{lstlisting}

Displacements bigger than 127 are very popular, but they are rarely exceeds 0x10000
(indeed, such large memory buffers/structures are also rare).

Same story with \INS{MOV}, large constants are rare, the most heavily used are 0, 1, 10, 100, $2^n$, and so on.
Compiler has to pad small constants by zeros to represent them as 32-bit values:

\begin{lstlisting}[style=customasmx86]
BF 02 00 00 00                    mov     edi, 2
BF 01 00 00 00                    mov     edi, 1
\end{lstlisting}

Now about 00 and FF bytes combined: jumps (including conditional) and calls can pass execution flow forward or backwards, but very often,
within the limits of the current executable module.
If forward, displacement is not very big and also padded with zeros.
If backwards, displacement is represented as negative value, so padded with FF bytes.
For example, transfer execution flow forward:

\begin{lstlisting}[style=customasmx86]
E8 43 0C 00 00                    call    _function1
E8 5C 00 00 00                    call    _function2
0F 84 F0 0A 00 00                 jz      loc_4F09A0
0F 84 EB 00 00 00                 jz      loc_4EFBB8
\end{lstlisting}

Backwards:

\begin{lstlisting}[style=customasmx86]
E8 79 0C FE FF                    call    _function1
E8 F4 16 FF FF                    call    _function2
0F 84 F8 FB FF FF                 jz      loc_8212BC
0F 84 06 FD FF FF                 jz      loc_FF1E7D
\end{lstlisting}

FF byte is also very often occurred in negative displacements like these:

\begin{lstlisting}[style=customasmx86]
8D 85 1E FF FF FF                 lea     eax, [ebp-0E2h]
8D 95 F8 5C FF FF                 lea     edx, [ebp-0A308h]
\end{lstlisting}

So far so good. Now we have to try various 16-byte keys, decrypt executable section and measure how often 00, FF and 8B bytes are occurred.
Let's also keep in sight how PCBC decryption works:

\begin{figure}[H]
\centering
\myincludegraphics{examples/simple_exec_crypto/640px-PCBC_decryption.png}
\caption{Propagating Cipher Block Chaining decryption (image is taken from Wikipedia article)}
\end{figure}

The good news is that we don't really have to decrypt whole piece of data, but only slice by slice, this is exactly how I did in my previous example: \myref{XOR_mask_2}.

Now I'm trying all possible bytes (0..255) for each byte in key and just pick the byte producing maximal amount of 00/FF/8B bytes in a decrypted slice:

\begin{lstlisting}[style=custompy]
#!/usr/bin/env python
import sys, hexdump, array, string, operator

KEY_LEN=16

def chunks(l, n):
    # split n by l-byte chunks
    # https://stackoverflow.com/q/312443
    n = max(1, n)
    return [l[i:i + n] for i in range(0, len(l), n)]

def read_file(fname):
    file=open(fname, mode='rb')
    content=file.read()
    file.close()
    return content

def decrypt_byte (c, key):
    return chr((ord(c)-key) % 256)

def XOR_PCBC_step (IV, buf, k):
    prev=IV
    rt=""
    for c in buf:
	new_c=decrypt_byte(c, k)
        plain=chr(ord(new_c)^ord(prev))
	prev=chr(ord(c)^ord(plain))
	rt=rt+plain
    return rt

each_Nth_byte=[""]*KEY_LEN

content=read_file(sys.argv[1])
# split input by 16-byte chunks:
all_chunks=chunks(content, KEY_LEN)
for c in all_chunks:
    for i in range(KEY_LEN):
        each_Nth_byte[i]=each_Nth_byte[i] + c[i]

# try each byte of key
for N in range(KEY_LEN):
    print "N=", N
    stat={}
    for i in range(256):
        tmp_key=chr(i)
	tmp=XOR_PCBC_step(tmp_key,each_Nth_byte[N], N)
        # count 0, FFs and 8Bs in decrypted buffer:
	important_bytes=tmp.count('\x00')+tmp.count('\xFF')+tmp.count('\x8B')
	stat[i]=important_bytes
    sorted_stat = sorted(stat.iteritems(), key=operator.itemgetter(1), reverse=True)
    print sorted_stat[0]
\end{lstlisting}

(Source code can be downloaded \href{\GitHubBlobMasterURL/examples/simple_exec_crypto/files/decrypt.py}{here}.)

I run it and here is a key for which 00/FF/8B bytes presence in decrypted buffer is maximal:

\begin{lstlisting}
N= 0
(147, 1224)
N= 1
(94, 1327)
N= 2
(252, 1223)
N= 3
(218, 1266)
N= 4
(38, 1209)
N= 5
(192, 1378)
N= 6
(199, 1204)
N= 7
(213, 1332)
N= 8
(225, 1251)
N= 9
(112, 1223)
N= 10
(143, 1177)
N= 11
(108, 1286)
N= 12
(10, 1164)
N= 13
(3, 1271)
N= 14
(128, 1253)
N= 15
(232, 1330)
\end{lstlisting}

Let's write decryption utility with the key we got:

\begin{lstlisting}[style=custompy]
#!/usr/bin/env python
import sys, hexdump, array

def xor_strings(s,t):
    # \verb|https://en.wikipedia.org/wiki/XOR_cipher#Example_implementation|
    """xor two strings together"""
    return "".join(chr(ord(a)^ord(b)) for a,b in zip(s,t))

IV=array.array('B', [147, 94, 252, 218, 38, 192, 199, 213, 225, 112, 143, 108, 10, 3, 128, 232]).tostring()

def chunks(l, n):
    n = max(1, n)
    return [l[i:i + n] for i in range(0, len(l), n)]

def read_file(fname):
    file=open(fname, mode='rb')
    content=file.read()
    file.close()
    return content

def decrypt_byte(i, k):
    return chr ((ord(i)-k) % 256)

def decrypt(buf):
    return "".join(decrypt_byte(buf[i], i) for i in range(16))

fout=open(sys.argv[2], mode='wb')

prev=IV
content=read_file(sys.argv[1])
tmp=chunks(content, 16)
for c in tmp:
    new_c=decrypt(c)
    p=xor_strings (new_c, prev)
    prev=xor_strings(c, p)
    fout.write(p)
fout.close()
\end{lstlisting}

(Source code can be downloaded \href{\GitHubBlobMasterURL/examples/simple_exec_crypto/files/decrypt2.py}{here}.)

Let's check resulting file:

\lstinputlisting{examples/simple_exec_crypto/objdump_result.txt}

Yes, this is seems correctly disassembled piece of x86 code.
The whole decryped file can be downloaded \href{\GitHubBlobMasterURL/examples/simple_exec_crypto/files/decrypted.bin}{here}.

In fact, this is text section from regedit.exe from Windows 7.
But this example is based on a real case I encountered, so just executable is different (and key), algorithm is the same.

\subsection{Other ideas to consider}

What if I would fail with such simple frequency analysis?
There are other ideas on how to measure correctness of decrypted/decompressed x86 code:

\begin{itemize}

\item Many modern compilers aligns functions on 0x10 border.
So the space left before is filled with NOPs (0x90) or other NOP instructions with known opcodes: \myref{sec:npad}.

\item Perhaps, the most frequent pattern in any assembly language is function call:\\
\TT{PUSH chain / CALL / ADD ESP, X}.
This sequence can easily detected and found.
I've even gathered statistics about average number of function arguments: \myref{args_stat}.
(Hence, this is average length of PUSH chain.)

\end{itemize}

Read more about incorrectly/correctly disassembled code: \myref{ISA_detect}.
}\RU{\subsection{Простое шифрование используя XOR-маску}
\label{XOR_mask_1}

Я нашел одну старую игру в стиле interactive fiction в архиве \emph{if-archive}\footnote{\url{http://www.ifarchive.org/}}:

\begin{lstlisting}
The New Castle v3.5 - Text/Adventure Game
in the style of the original Infocom (tm)
type games, Zork, Collosal Cave (Adventure),
etc.  Can you solve the mystery of the
abandoned castle?
Shareware from Software Customization.
Software Customization [ASP] Version 3.5 Feb. 2000
\end{lstlisting}

Можно скачать здесь: \url{\GitHubBlobMasterURL/ff/XOR/mask_1/files/newcastle.tgz}.

Там внутри есть файл (с названием \emph{castle.dbf}), который явно зашифрован, но не настоящим криптоалгоритмом,
и оне сжат, это что-то куда проще.
Я бы даже не стал измерять уровень энтропии (\myref{entropy}) этого файла, потому что я итак уверен, что он низкий.
Вот как он выглядит в Midnight Commander:

\begin{figure}[H]
\centering
\myincludegraphics{ff/XOR/mask_1/mc_encrypted.png}
\caption{Зашифрованный файл в Midnight Commander}
\end{figure}

Зашифрованный файл можно скачать здесь:
\url{\GitHubBlobMasterURL/ff/XOR/mask_1/files/castle.dbf.bz2}.

Можно ли расшифровать его без доступа к программе, используя просто этот файл?

Тут явно просматривается повторяющаяся строка. 
Если использовалось простое шифрование с XOR-маской, такие повторяющиеся строки это явное свидетельство,
потому что, вероятно, тут были длинные лакуны с нулевыми байтами, которые, в свою очередь, присутствуют
во мноигих исполняемых файлах, и в остальных бинарных файлах.

\myindex{UNIX!xxd}
Вот дам начала этого файла используя утилиту \emph{xxd} из UNIX:

\lstinputlisting{ff/XOR/mask_1/xxd_result.txt}

Давайте держаться за повторяющуюся строку \TT{iubgv}.
Глядя на этот дамп, мы можем легко увидеть, что период повторений этой строки это 0x51 или 81.
Вероятно, 81 это длина блока?
Длина файла 1658961, и она может быть поделена на 81 без остатка (и тогда там 20481 блоков).

Теперь я буду использовать Mathematica для анализа, есть ли тут повторяющиеся 81-байтные блоки в файле?
Я разделю входной файл на 81-байтные блоки и затем использую ф-цию
\emph{Tally[]}\footnote{\url{https://reference.wolfram.com/language/ref/Tally.html}}
которая просто считает, сколько раз каждый элемент встретился во входном списке.
Вывод Tally не отсортирован, так что я также добавлю ф-цию \emph{Sort[]} для сортировки его по кол-ву вхождений
в нисходящем порядке.

\begin{lstlisting}[style=custommath]
input = BinaryReadList["/home/dennis/.../castle.dbf"];

blocks = Partition[input, 81];

stat = Sort[Tally[blocks], #1[[2]] > #2[[2]] &]
\end{lstlisting}

И вот вывод:

\begin{lstlisting}[style=custommath]
{{{80, 103, 2, 116, 113, 102, 118, 25, 99, 8, 19, 23, 116, 125, 107, 
   25, 99, 109, 114, 102, 14, 121, 115, 31, 9, 117, 113, 111, 5, 4, 
   127, 28, 122, 101, 8, 110, 14, 18, 124, 106, 16, 20, 104, 119, 8, 
   109, 26, 106, 9, 97, 13, 99, 15, 119, 20, 105, 117, 98, 103, 118, 
   1, 126, 29, 97, 122, 17, 15, 114, 110, 3, 5, 125, 125, 99, 126, 
   119, 102, 30, 122, 2, 117}, 1739}, 
{{80, 100, 2, 116, 113, 102, 118, 25, 99, 8, 19, 23, 116, 
   125, 107, 25, 99, 109, 114, 102, 14, 121, 115, 31, 9, 117, 113, 
   111, 5, 4, 127, 28, 122, 101, 8, 110, 14, 18, 124, 106, 16, 20, 
   104, 119, 8, 109, 26, 106, 9, 97, 13, 99, 15, 119, 20, 105, 117, 
   98, 103, 118, 1, 126, 29, 97, 122, 17, 15, 114, 110, 3, 5, 125, 
   125, 99, 126, 119, 102, 30, 122, 2, 117}, 1422}, 
{{80, 101, 2, 116, 113, 102, 118, 25, 99, 8, 19, 23, 116, 
   125, 107, 25, 99, 109, 114, 102, 14, 121, 115, 31, 9, 117, 113, 
   111, 5, 4, 127, 28, 122, 101, 8, 110, 14, 18, 124, 106, 16, 20, 
   104, 119, 8, 109, 26, 106, 9, 97, 13, 99, 15, 119, 20, 105, 117, 
   98, 103, 118, 1, 126, 29, 97, 122, 17, 15, 114, 110, 3, 5, 125, 
   125, 99, 126, 119, 102, 30, 122, 2, 117}, 1012},
{{80, 120, 2, 116, 113, 102, 118, 25, 99, 8, 19, 23, 116, 
   125, 107, 25, 99, 109, 114, 102, 14, 121, 115, 31, 9, 117, 113, 
   111, 5, 4, 127, 28, 122, 101, 8, 110, 14, 18, 124, 106, 16, 20, 
   104, 119, 8, 109, 26, 106, 9, 97, 13, 99, 15, 119, 20, 105, 117, 
   98, 103, 118, 1, 126, 29, 97, 122, 17, 15, 114, 110, 3, 5, 125, 
   125, 99, 126, 119, 102, 30, 122, 2, 117}, 377},

...

{{80, 2, 74, 49, 113, 21, 62, 88, 39, 71, 68, 23, 63, 51, 36, 78, 48, 
   108, 114, 102, 14, 121, 115, 31, 9, 117, 113, 111, 5, 4, 127, 28, 
   122, 101, 8, 110, 14, 18, 124, 106, 16, 20, 104, 119, 8, 109, 26, 
   106, 9, 97, 13, 99, 15, 119, 20, 105, 117, 98, 103, 118, 1, 126, 
   29, 97, 122, 17, 15, 114, 110, 3, 5, 125, 125, 99, 126, 119, 102, 
   30, 122, 2, 117}, 1},
{{80, 1, 74, 59, 113, 45, 56, 86, 52, 91, 19, 64, 60, 60, 63, 
   25, 38, 59, 59, 42, 14, 53, 38, 77, 66, 38, 113, 38, 75, 4, 43, 84,
    63, 101, 64, 43, 79, 64, 40, 57, 16, 91, 46, 119, 69, 40, 84, 117,
    9, 97, 13, 99, 15, 119, 20, 105, 117, 98, 103, 118, 1, 126, 29, 
   97, 122, 17, 15, 114, 110, 3, 5, 125, 125, 99, 126, 119, 102, 30, 
   122, 2, 117}, 1},
{{80, 2, 74, 49, 113, 49, 51, 92, 39, 8, 92, 81, 116, 62, 57, 
   80, 46, 40, 114, 36, 75, 56, 33, 76, 9, 55, 56, 59, 81, 65, 45, 28,
    60, 55, 93, 39, 90, 28, 124, 106, 16, 20, 104, 119, 8, 109, 26, 
   106, 9, 97, 13, 99, 15, 119, 20, 105, 117, 98, 103, 118, 1, 126, 
   29, 97, 122, 17, 15, 114, 110, 3, 5, 125, 125, 99, 126, 119, 102, 
   30, 122, 2, 117}, 1}}
\end{lstlisting}

Вывод Tally это список пар, каждая пара это 81-байтный блок и количество раз, сколько он встретился в файле.
Мы видим, что наиболее частно встречающийся блок это первый, он встретился 1739 раз.
Второй встретился 1422 раза. Есть и другие: 1012 раза, 377 раз, итд.
81-байтные блоки, встреченные лишь один раз, находятся в конце вывода.

Попробуем сравнить эти блоки. Первый и второй.
Есть ли в Mathematica ф-ция для сравнения списков/массивов?
Наверняка есть, но в педагогических целях, я буду использоват операцию XOR для сравнения.
Действительно: если байты во входных массивах равны друг другу, результат операции XOR это 0.
Если не равны, результат будет ненулевой.

Сравним первый блок (встречается 1739 раз) и второй (встречается 1422 раз):

\begin{lstlisting}[style=custommath]
In[]:= BitXor[stat[[1]][[1]], stat[[2]][[1]]]
Out[]= {0, 3, 0, 0, 0, 0, 0, 0, 0, 0, 0, 0, 0, 0, 0, 0, 0, 0, 0, \
0, 0, 0, 0, 0, 0, 0, 0, 0, 0, 0, 0, 0, 0, 0, 0, 0, 0, 0, 0, 0, 0, 0, \
0, 0, 0, 0, 0, 0, 0, 0, 0, 0, 0, 0, 0, 0, 0, 0, 0, 0, 0, 0, 0, 0, 0, \
0, 0, 0, 0, 0, 0, 0, 0, 0, 0, 0, 0, 0, 0, 0, 0}
\end{lstlisting}

Они отличаются только вторым байтом.

Сравним второй блок (встречается 1422 раза) и третий (встречается 1012 раз):

\begin{lstlisting}[style=custommath]
In[]:= BitXor[stat[[2]][[1]], stat[[3]][[1]]]
Out[]= {0, 1, 0, 0, 0, 0, 0, 0, 0, 0, 0, 0, 0, 0, 0, 0, 0, 0, 0, \
0, 0, 0, 0, 0, 0, 0, 0, 0, 0, 0, 0, 0, 0, 0, 0, 0, 0, 0, 0, 0, 0, 0, \
0, 0, 0, 0, 0, 0, 0, 0, 0, 0, 0, 0, 0, 0, 0, 0, 0, 0, 0, 0, 0, 0, 0, \
0, 0, 0, 0, 0, 0, 0, 0, 0, 0, 0, 0, 0, 0, 0, 0}
\end{lstlisting}

Они тоже отличаются только вторым байтом.

Так или иначе, попробуем использовать самый встречающийся блок как XOR-ключ и попробуем расшифровать первые 4 81-байтных
блока в файле:

\begin{lstlisting}[style=custommath]
In[]:= key = stat[[1]][[1]]
Out[]= {80, 103, 2, 116, 113, 102, 118, 25, 99, 8, 19, 23, 116, \
125, 107, 25, 99, 109, 114, 102, 14, 121, 115, 31, 9, 117, 113, 111, \
5, 4, 127, 28, 122, 101, 8, 110, 14, 18, 124, 106, 16, 20, 104, 119, \
8, 109, 26, 106, 9, 97, 13, 99, 15, 119, 20, 105, 117, 98, 103, 118, \
1, 126, 29, 97, 122, 17, 15, 114, 110, 3, 5, 125, 125, 99, 126, 119, \
102, 30, 122, 2, 117}

In[]:= ToASCII[val_] := If[val == 0, " ", FromCharacterCode[val, "PrintableASCII"]]

In[]:= DecryptBlockASCII[blk_] := Map[ToASCII[#] &, BitXor[key, blk]]

In[]:= DecryptBlockASCII[blocks[[1]]]
Out[]= {" ", " ", " ", " ", " ", " ", " ", " ", " ", " ", " ", " \
", " ", " ", " ", " ", " ", " ", " ", " ", " ", " ", " ", " ", " ", " \
", " ", " ", " ", " ", " ", " ", " ", " ", " ", " ", " ", " ", " ", " \
", " ", " ", " ", " ", " ", " ", " ", " ", " ", " ", " ", " ", " ", " \
", " ", " ", " ", " ", " ", " ", " ", " ", " ", " ", " ", " ", " ", " \
", " ", " ", " ", " ", " ", " ", " ", " ", " ", " ", " ", " ", " "}

In[]:= DecryptBlockASCII[blocks[[2]]]
Out[]= {" ", "e", "H", "E", " ", "W", "E", "E", "D", " ", "O", \
"F", " ", "C", "R", "I", "M", "E", " ", "B", "E", "A", "R", "S", " ", \
"B", "I", "T", "T", "E", "R", " ", "F", "R", "U", "I", "T", "?", \
" ", " ", " ", " ", " ", " ", " ", " ", " ", " ", " ", " ", " ", " ", \
" ", " ", " ", " ", " ", " ", " ", " ", " ", " ", " ", " ", " ", " ", \
" ", " ", " ", " ", " ", " ", " ", " ", " ", " ", " ", " ", " ", " ", \
" "}

In[]:= DecryptBlockASCII[blocks[[3]]]
Out[]= {" ", "?", " ", " ", " ", " ", " ", " ", " ", " ", " \
", " ", " ", " ", " ", " ", " ", " ", " ", " ", " ", " ", " ", " ", " \
", " ", " ", " ", " ", " ", " ", " ", " ", " ", " ", " ", " ", " ", " \
", " ", " ", " ", " ", " ", " ", " ", " ", " ", " ", " ", " ", " ", " \
", " ", " ", " ", " ", " ", " ", " ", " ", " ", " ", " ", " ", " ", " \
", " ", " ", " ", " ", " ", " ", " ", " ", " ", " ", " ", " ", " ", " \
"}

In[]:= DecryptBlockASCII[blocks[[4]]]
Out[]= {" ", "f", "H", "O", " ", "K", "N", "O", "W", "S", " ", \
"W", "H", "A", "T", " ", "E", "V", "I", "L", " ", "L", "U", "R", "K", \
"S", " ", "I", "N", " ", "T", "H", "E", " ", "H", "E", "A", "R", "T", \
"S", " ", "O", "F", " ", "M", "E", "N", "?", " ", " ", " ", " ", \
" ", " ", " ", " ", " ", " ", " ", " ", " ", " ", " ", " ", " ", " ", \
" ", " ", " ", " ", " ", " ", " ", " ", " ", " ", " ", " ", " ", " ", \
" "}
\end{lstlisting}

(Я заменил непечатаемые символы на \q{?}.)

Мы видим что первый и третий блоки пустые (или почти пустые),
но второй и четвертый имеют ясно различимые английские слова/фразы.
Похоже что наше предположение насчет ключа верно (как минимум частично).
Это означает, что самый встречающийся 81-байтный блок в файле находится в местах лакун с нулевыми байтами
или что-то в этом роде.

Попробуем расшифровать весь файл:

\begin{lstlisting}[style=custommath]
DecryptBlock[blk_] := BitXor[key, blk]

decrypted = Map[DecryptBlock[#] &, blocks];

BinaryWrite["/home/dennis/.../tmp", Flatten[decrypted]]

Close["/home/dennis/.../tmp"]
\end{lstlisting}

\begin{figure}[H]
\centering
\myincludegraphics{ff/XOR/mask_1/mc_decrypted1.png}
\caption{Расшифрованный файл в Midnight Commander, первая попытка}
\end{figure}

Выглядит как английские фразы для какой-то игры, но что-то не так.
Прежде всего, регистр инвертирован: фразы и некоторые слова начинаются со строчных букв,
в то время как остальные буквы заглавные.
Также, некоторые фразы начинаются с не тех букв.
Посмотрите на самую первую фразу: \q{eHE WEED OF CRIME BEARS BITTER FRUIT}.
Что такое \q{eHE}? Разве не \q{tHE} тут должно быть?
Возможно ли что наш ключ для дешифрования имеет неверный байт в этом месте?

Посмотрим снова на второй блок в файле, на ключ и на результат дешифрования:

\begin{lstlisting}[style=custommath]
In[]:= blocks[[2]]
Out[]= {80, 2, 74, 49, 113, 49, 51, 92, 39, 8, 92, 81, 116, 62, \
57, 80, 46, 40, 114, 36, 75, 56, 33, 76, 9, 55, 56, 59, 81, 65, 45, \
28, 60, 55, 93, 39, 90, 28, 124, 106, 16, 20, 104, 119, 8, 109, 26, \
106, 9, 97, 13, 99, 15, 119, 20, 105, 117, 98, 103, 118, 1, 126, 29, \
97, 122, 17, 15, 114, 110, 3, 5, 125, 125, 99, 126, 119, 102, 30, \
122, 2, 117}

In[]:= key
Out[]= {80, 103, 2, 116, 113, 102, 118, 25, 99, 8, 19, 23, 116, \
125, 107, 25, 99, 109, 114, 102, 14, 121, 115, 31, 9, 117, 113, 111, \
5, 4, 127, 28, 122, 101, 8, 110, 14, 18, 124, 106, 16, 20, 104, 119, \
8, 109, 26, 106, 9, 97, 13, 99, 15, 119, 20, 105, 117, 98, 103, 118, \
1, 126, 29, 97, 122, 17, 15, 114, 110, 3, 5, 125, 125, 99, 126, 119, \
102, 30, 122, 2, 117}

In[]:= BitXor[key, blocks[[2]]]
Out[]= {0, 101, 72, 69, 0, 87, 69, 69, 68, 0, 79, 70, 0, 67, 82, \
73, 77, 69, 0, 66, 69, 65, 82, 83, 0, 66, 73, 84, 84, 69, 82, 0, 70, \
82, 85, 73, 84, 14, 0, 0, 0, 0, 0, 0, 0, 0, 0, 0, 0, 0, 0, 0, 0, 0, \
0, 0, 0, 0, 0, 0, 0, 0, 0, 0, 0, 0, 0, 0, 0, 0, 0, 0, 0, 0, 0, 0, 0, \
0, 0, 0, 0}
\end{lstlisting}

Зашифрованный байт это 2, байт из ключа это 103, $2 \oplus 103=101$ и 101 это ASCII-код символа \q{e}.
Чему должен равнятся этот байт ключа, чтобы ASCII-код был 116 (для символа  \q{t})?
$2 \oplus 116=118$, присвоим 118 второму байту в ключе \dots

\begin{lstlisting}[style=custommath]
key = {80, 118, 2, 116, 113, 102, 118, 25, 99, 8, 19, 23, 116, 125, 
  107, 25, 99, 109, 114, 102, 14, 121, 115, 31, 9, 117, 113, 111, 5, 
  4, 127, 28, 122, 101, 8, 110, 14, 18, 124, 106, 16, 20, 104, 119, 8,
   109, 26, 106, 9, 97, 13, 99, 15, 119, 20, 105, 117, 98, 103, 118, 
  1, 126, 29, 97, 122, 17, 15, 114, 110, 3, 5, 125, 125, 99, 126, 119,
   102, 30, 122, 2, 117}
\end{lstlisting}

\dots и снова дешифруем весь файл.

\begin{figure}[H]
\centering
\myincludegraphics{ff/XOR/mask_1/mc_decrypted2.png}
\caption{Дешифрованный файл в Midnight Commander, вторая попытка}
\end{figure}

Ух ты, теперь грамматика корректна, и все фразы начинаются с корректных букв.
Но все таки, регистр подозрителен.
С чего бы разработчику игры записывать их в такой манере?
Может быть наш ключ все еще неправилен?

% TODO ASCII table somewhere in the book
Изучая таблицу ASCII мы можем заметить что ASCII-коды для букв в верхнем и нижнем регистре отличаются только на один бит
(6-й бит, если считать с первого, 0b100000):

\begin{figure}[H]
\centering
\includegraphics[width=0.7\textwidth]{ascii.png}
\caption{7-битная таблица \ac{ASCII} в Emacs}
\end{figure}

6-й бит, выставленный в нулевом байте, В десятичном виде это будет 32.
Но 32 это ASCII-код пробела!

Действительно, можно менять регистр просто применяя XOR к ASCII-коду, с 32 (больше об этом: \myref{toupper_bit}).

Возможно ли, что пустые лакуны в файле это не нулевые байты, а скорее содержащие пробелы?
Еще раз модифицируем наш XOR-ключ (я про-XOR-ю каждый байт ключа с 32):

\begin{lstlisting}[style=custommath]
(* "32" это скаляр, и "key" это вектор, но это OK *)

In[]:= key3 = BitXor[32, key]
Out[]= {112, 86, 34, 84, 81, 70, 86, 57, 67, 40, 51, 55, 84, 93, 75, \
57, 67, 77, 82, 70, 46, 89, 83, 63, 41, 85, 81, 79, 37, 36, 95, 60, \
90, 69, 40, 78, 46, 50, 92, 74, 48, 52, 72, 87, 40, 77, 58, 74, 41, \
65, 45, 67, 47, 87, 52, 73, 85, 66, 71, 86, 33, 94, 61, 65, 90, 49, \
47, 82, 78, 35, 37, 93, 93, 67, 94, 87, 70, 62, 90, 34, 85}

In[]:= DecryptBlock[blk_] := BitXor[key3, blk]
\end{lstlisting}

И снова дешифруем входной файл:

\begin{figure}[H]
\centering
\myincludegraphics{ff/XOR/mask_1/mc_decrypted.png}
\caption{Дешифрованный файл в Midnight Commander, последняя попытка}
\end{figure}

(Расшифрованный файл доступен здесь:
\url{\GitHubBlobMasterURL/ff/XOR/mask_1/files/decrypted.dat.bz2}.)

Несомненно, это корректный исходный файл.
Да, и мы видим числа в начале каждого блока. Должно быть это и есть источник некорректного XOR-ключа.
Как выходит, самый встречающийся 81-байтный блок в файле это блок заполненный пробелами и содержащий символ \q{1} на месте
второго байта.
Действительно, как-то так получилось что многие блоки здесь перемежаются с этим блоком.
Может быть это что-то вроде выравнивания (padding) для коротких фраз/сообщений?
Другой часто встречающийся 81-байтный блок также заполнен пробелами, но с другой цифрой, следовательно,
они отличаются только вторым байтом.

Вот и всё! Теперь мы можем написать утилиту для зашифрования файла назад, и, может быть, модифицировать его перед этим

Файл для Mathematica можно скачать здесь:\\
\url{\GitHubBlobMasterURL/ff/XOR/mask_1/files/XOR_mask_1.nb}.

Итог: XOR-шифрование не надежно вообще. Вероятно, разработчик игры хотел просто скрыть внутренности игры от игрока,
ничего более серьезного.
Все же, шифрование вроде этого крайне популярно вследствии его простоты, так что многие реверс инженеры обычно хорошо
с этим знакомы.

}\FR{\mysection{Fonction presque vide}
\label{Boolector}
\myindex{Boolector}
\myindex{x86!\Instructions!JMP}

Ceci est un morceau de code réel que j'ai trouvé dans Boolector\footnote{\url{https://boolector.github.io/}}:

\lstinputlisting[style=customc]{patterns/025_almost_empty/boolectormain.c}

Pourquoi quelqu'un ferait-il comme ça?
Je ne sais pas mais mon hypothèse est que \verb|boolector_main()| peut être compilée
dans une sorte de DLL ou bibliothèque dynamique, et appelée depuis une suite de test.
Certainement qu'une suite de test peut préparer les variables argc/argv comme
le ferait \ac{CRT}.

Il est intéressant de voir comment c'est compilé:

\lstinputlisting[caption=GCC 8.2 x64 \NonOptimizing (\assemblyOutput),style=customasmx86]{patterns/025_almost_empty/boolectormain_O0.s}

Ceci est OK, le prologue (non optimisé) déplace inutilement deux arguments,
\INS{CALL}, épilogue, \INS{RET}.
Mais regardons la version optimisée:

\lstinputlisting[caption=GCC 8.2 x64 \Optimizing (\assemblyOutput),style=customasmx86]{patterns/025_almost_empty/boolectormain_O3.s}

Aussi simple que ça: la pile et les registres ne sont pas touchés et \verb|boolector_main()|
a le même ensemble d'arguments.
Donc, tout ce que nous avons à faire est de passer l'exécution à une autre adresse.

Ceci est proche d'une \glslink{thunk function}{fonction thunk}.

Nous verons queelque chose de plus avancé plus tard: \myref{ARM_B_to_printf}, \myref{JMP_instead_of_RET}.
}

\EN{% TODO translate
\mysection{Breaking simple executable cryptor}

I've got an executable file which is encrypted by relatively simple encryption.
\href{\GitHubBlobMasterURL/examples/simple_exec_crypto/files/cipher.bin}{Here is it} (only executable section is left here).

First, all encryption function does is just adds number of position in buffer to the byte.
Here is how this can be encoded in Python:

\begin{lstlisting}[caption=Python script,style=custompy]
#!/usr/bin/env python
def e(i, k):
    return chr ((ord(i)+k) % 256)

def encrypt(buf):
    return e(buf[0], 0)+ e(buf[1], 1)+ e(buf[2], 2) + e(buf[3], 3)+ e(buf[4], 4)+ e(buf[5], 5)+ e(buf[6], 6)+ e(buf[7], 7)+
           e(buf[8], 8)+ e(buf[9], 9)+ e(buf[10], 10)+ e(buf[11], 11)+ e(buf[12], 12)+ e(buf[13], 13)+ e(buf[14], 14)+ e(buf[15], 15)
\end{lstlisting}

Hence, if you encrypt buffer with 16 zeros, you'll get \emph{0, 1, 2, 3 ... 12, 13, 14, 15}.

\myindex{Propagating Cipher Block Chaining}
Propagating Cipher Block Chaining (PCBC) is also used, here is how it works:

\begin{figure}[H]
\centering
\myincludegraphics{examples/simple_exec_crypto/601px-PCBC_encryption.png}
\caption{Propagating Cipher Block Chaining encryption (image is taken from Wikipedia article)}
\end{figure}

The problem is that it's too boring to recover IV (Initialization Vector) each time.
Brute-force is also not an option, because IV is too long (16 bytes).
Let's see, if it's possible to recover IV for arbitrary encrypted executable file?

Let's try simple frequency analysis.
This is 32-bit x86 executable code, so let's gather statistics about most frequent bytes and opcodes.
I tried huge oracle.exe file from Oracle RDBMS version 11.2 for windows x86 and I've found that the most frequent byte (no surprise) is zero (~10\%).
The next most frequent byte is (again, no surprise) 0xFF (~5\%).
The next is 0x8B (~5\%).

\myindex{x86!\Instructions!MOV}
0x8B is opcode for \INS{MOV}, this is indeed one of the most busy x86 instructions.
Now what about popularity of zero byte?
If compiler needs to encode value bigger than 127, it has to use 32-bit displacement instead of 8-bit one, but large values are very rare,
so it is padded by zeros.
\myindex{x86!\Instructions!LEA}
\myindex{x86!\Instructions!PUSH}
\myindex{x86!\Instructions!CALL}
This is at least in \INS{LEA}, \INS{MOV}, \INS{PUSH}, \INS{CALL}.

For example:

\begin{lstlisting}[style=customasmx86]
8D B0 28 01 00 00                 lea     esi, [eax+128h]
8D BF 40 38 00 00                 lea     edi, [edi+3840h]
\end{lstlisting}

Displacements bigger than 127 are very popular, but they are rarely exceeds 0x10000
(indeed, such large memory buffers/structures are also rare).

Same story with \INS{MOV}, large constants are rare, the most heavily used are 0, 1, 10, 100, $2^n$, and so on.
Compiler has to pad small constants by zeros to represent them as 32-bit values:

\begin{lstlisting}[style=customasmx86]
BF 02 00 00 00                    mov     edi, 2
BF 01 00 00 00                    mov     edi, 1
\end{lstlisting}

Now about 00 and FF bytes combined: jumps (including conditional) and calls can pass execution flow forward or backwards, but very often,
within the limits of the current executable module.
If forward, displacement is not very big and also padded with zeros.
If backwards, displacement is represented as negative value, so padded with FF bytes.
For example, transfer execution flow forward:

\begin{lstlisting}[style=customasmx86]
E8 43 0C 00 00                    call    _function1
E8 5C 00 00 00                    call    _function2
0F 84 F0 0A 00 00                 jz      loc_4F09A0
0F 84 EB 00 00 00                 jz      loc_4EFBB8
\end{lstlisting}

Backwards:

\begin{lstlisting}[style=customasmx86]
E8 79 0C FE FF                    call    _function1
E8 F4 16 FF FF                    call    _function2
0F 84 F8 FB FF FF                 jz      loc_8212BC
0F 84 06 FD FF FF                 jz      loc_FF1E7D
\end{lstlisting}

FF byte is also very often occurred in negative displacements like these:

\begin{lstlisting}[style=customasmx86]
8D 85 1E FF FF FF                 lea     eax, [ebp-0E2h]
8D 95 F8 5C FF FF                 lea     edx, [ebp-0A308h]
\end{lstlisting}

So far so good. Now we have to try various 16-byte keys, decrypt executable section and measure how often 00, FF and 8B bytes are occurred.
Let's also keep in sight how PCBC decryption works:

\begin{figure}[H]
\centering
\myincludegraphics{examples/simple_exec_crypto/640px-PCBC_decryption.png}
\caption{Propagating Cipher Block Chaining decryption (image is taken from Wikipedia article)}
\end{figure}

The good news is that we don't really have to decrypt whole piece of data, but only slice by slice, this is exactly how I did in my previous example: \myref{XOR_mask_2}.

Now I'm trying all possible bytes (0..255) for each byte in key and just pick the byte producing maximal amount of 00/FF/8B bytes in a decrypted slice:

\begin{lstlisting}[style=custompy]
#!/usr/bin/env python
import sys, hexdump, array, string, operator

KEY_LEN=16

def chunks(l, n):
    # split n by l-byte chunks
    # https://stackoverflow.com/q/312443
    n = max(1, n)
    return [l[i:i + n] for i in range(0, len(l), n)]

def read_file(fname):
    file=open(fname, mode='rb')
    content=file.read()
    file.close()
    return content

def decrypt_byte (c, key):
    return chr((ord(c)-key) % 256)

def XOR_PCBC_step (IV, buf, k):
    prev=IV
    rt=""
    for c in buf:
	new_c=decrypt_byte(c, k)
        plain=chr(ord(new_c)^ord(prev))
	prev=chr(ord(c)^ord(plain))
	rt=rt+plain
    return rt

each_Nth_byte=[""]*KEY_LEN

content=read_file(sys.argv[1])
# split input by 16-byte chunks:
all_chunks=chunks(content, KEY_LEN)
for c in all_chunks:
    for i in range(KEY_LEN):
        each_Nth_byte[i]=each_Nth_byte[i] + c[i]

# try each byte of key
for N in range(KEY_LEN):
    print "N=", N
    stat={}
    for i in range(256):
        tmp_key=chr(i)
	tmp=XOR_PCBC_step(tmp_key,each_Nth_byte[N], N)
        # count 0, FFs and 8Bs in decrypted buffer:
	important_bytes=tmp.count('\x00')+tmp.count('\xFF')+tmp.count('\x8B')
	stat[i]=important_bytes
    sorted_stat = sorted(stat.iteritems(), key=operator.itemgetter(1), reverse=True)
    print sorted_stat[0]
\end{lstlisting}

(Source code can be downloaded \href{\GitHubBlobMasterURL/examples/simple_exec_crypto/files/decrypt.py}{here}.)

I run it and here is a key for which 00/FF/8B bytes presence in decrypted buffer is maximal:

\begin{lstlisting}
N= 0
(147, 1224)
N= 1
(94, 1327)
N= 2
(252, 1223)
N= 3
(218, 1266)
N= 4
(38, 1209)
N= 5
(192, 1378)
N= 6
(199, 1204)
N= 7
(213, 1332)
N= 8
(225, 1251)
N= 9
(112, 1223)
N= 10
(143, 1177)
N= 11
(108, 1286)
N= 12
(10, 1164)
N= 13
(3, 1271)
N= 14
(128, 1253)
N= 15
(232, 1330)
\end{lstlisting}

Let's write decryption utility with the key we got:

\begin{lstlisting}[style=custompy]
#!/usr/bin/env python
import sys, hexdump, array

def xor_strings(s,t):
    # \verb|https://en.wikipedia.org/wiki/XOR_cipher#Example_implementation|
    """xor two strings together"""
    return "".join(chr(ord(a)^ord(b)) for a,b in zip(s,t))

IV=array.array('B', [147, 94, 252, 218, 38, 192, 199, 213, 225, 112, 143, 108, 10, 3, 128, 232]).tostring()

def chunks(l, n):
    n = max(1, n)
    return [l[i:i + n] for i in range(0, len(l), n)]

def read_file(fname):
    file=open(fname, mode='rb')
    content=file.read()
    file.close()
    return content

def decrypt_byte(i, k):
    return chr ((ord(i)-k) % 256)

def decrypt(buf):
    return "".join(decrypt_byte(buf[i], i) for i in range(16))

fout=open(sys.argv[2], mode='wb')

prev=IV
content=read_file(sys.argv[1])
tmp=chunks(content, 16)
for c in tmp:
    new_c=decrypt(c)
    p=xor_strings (new_c, prev)
    prev=xor_strings(c, p)
    fout.write(p)
fout.close()
\end{lstlisting}

(Source code can be downloaded \href{\GitHubBlobMasterURL/examples/simple_exec_crypto/files/decrypt2.py}{here}.)

Let's check resulting file:

\lstinputlisting{examples/simple_exec_crypto/objdump_result.txt}

Yes, this is seems correctly disassembled piece of x86 code.
The whole decryped file can be downloaded \href{\GitHubBlobMasterURL/examples/simple_exec_crypto/files/decrypted.bin}{here}.

In fact, this is text section from regedit.exe from Windows 7.
But this example is based on a real case I encountered, so just executable is different (and key), algorithm is the same.

\subsection{Other ideas to consider}

What if I would fail with such simple frequency analysis?
There are other ideas on how to measure correctness of decrypted/decompressed x86 code:

\begin{itemize}

\item Many modern compilers aligns functions on 0x10 border.
So the space left before is filled with NOPs (0x90) or other NOP instructions with known opcodes: \myref{sec:npad}.

\item Perhaps, the most frequent pattern in any assembly language is function call:\\
\TT{PUSH chain / CALL / ADD ESP, X}.
This sequence can easily detected and found.
I've even gathered statistics about average number of function arguments: \myref{args_stat}.
(Hence, this is average length of PUSH chain.)

\end{itemize}

Read more about incorrectly/correctly disassembled code: \myref{ISA_detect}.
}\RU{\subsection{Простое шифрование используя XOR-маску}
\label{XOR_mask_1}

Я нашел одну старую игру в стиле interactive fiction в архиве \emph{if-archive}\footnote{\url{http://www.ifarchive.org/}}:

\begin{lstlisting}
The New Castle v3.5 - Text/Adventure Game
in the style of the original Infocom (tm)
type games, Zork, Collosal Cave (Adventure),
etc.  Can you solve the mystery of the
abandoned castle?
Shareware from Software Customization.
Software Customization [ASP] Version 3.5 Feb. 2000
\end{lstlisting}

Можно скачать здесь: \url{\GitHubBlobMasterURL/ff/XOR/mask_1/files/newcastle.tgz}.

Там внутри есть файл (с названием \emph{castle.dbf}), который явно зашифрован, но не настоящим криптоалгоритмом,
и оне сжат, это что-то куда проще.
Я бы даже не стал измерять уровень энтропии (\myref{entropy}) этого файла, потому что я итак уверен, что он низкий.
Вот как он выглядит в Midnight Commander:

\begin{figure}[H]
\centering
\myincludegraphics{ff/XOR/mask_1/mc_encrypted.png}
\caption{Зашифрованный файл в Midnight Commander}
\end{figure}

Зашифрованный файл можно скачать здесь:
\url{\GitHubBlobMasterURL/ff/XOR/mask_1/files/castle.dbf.bz2}.

Можно ли расшифровать его без доступа к программе, используя просто этот файл?

Тут явно просматривается повторяющаяся строка. 
Если использовалось простое шифрование с XOR-маской, такие повторяющиеся строки это явное свидетельство,
потому что, вероятно, тут были длинные лакуны с нулевыми байтами, которые, в свою очередь, присутствуют
во мноигих исполняемых файлах, и в остальных бинарных файлах.

\myindex{UNIX!xxd}
Вот дам начала этого файла используя утилиту \emph{xxd} из UNIX:

\lstinputlisting{ff/XOR/mask_1/xxd_result.txt}

Давайте держаться за повторяющуюся строку \TT{iubgv}.
Глядя на этот дамп, мы можем легко увидеть, что период повторений этой строки это 0x51 или 81.
Вероятно, 81 это длина блока?
Длина файла 1658961, и она может быть поделена на 81 без остатка (и тогда там 20481 блоков).

Теперь я буду использовать Mathematica для анализа, есть ли тут повторяющиеся 81-байтные блоки в файле?
Я разделю входной файл на 81-байтные блоки и затем использую ф-цию
\emph{Tally[]}\footnote{\url{https://reference.wolfram.com/language/ref/Tally.html}}
которая просто считает, сколько раз каждый элемент встретился во входном списке.
Вывод Tally не отсортирован, так что я также добавлю ф-цию \emph{Sort[]} для сортировки его по кол-ву вхождений
в нисходящем порядке.

\begin{lstlisting}[style=custommath]
input = BinaryReadList["/home/dennis/.../castle.dbf"];

blocks = Partition[input, 81];

stat = Sort[Tally[blocks], #1[[2]] > #2[[2]] &]
\end{lstlisting}

И вот вывод:

\begin{lstlisting}[style=custommath]
{{{80, 103, 2, 116, 113, 102, 118, 25, 99, 8, 19, 23, 116, 125, 107, 
   25, 99, 109, 114, 102, 14, 121, 115, 31, 9, 117, 113, 111, 5, 4, 
   127, 28, 122, 101, 8, 110, 14, 18, 124, 106, 16, 20, 104, 119, 8, 
   109, 26, 106, 9, 97, 13, 99, 15, 119, 20, 105, 117, 98, 103, 118, 
   1, 126, 29, 97, 122, 17, 15, 114, 110, 3, 5, 125, 125, 99, 126, 
   119, 102, 30, 122, 2, 117}, 1739}, 
{{80, 100, 2, 116, 113, 102, 118, 25, 99, 8, 19, 23, 116, 
   125, 107, 25, 99, 109, 114, 102, 14, 121, 115, 31, 9, 117, 113, 
   111, 5, 4, 127, 28, 122, 101, 8, 110, 14, 18, 124, 106, 16, 20, 
   104, 119, 8, 109, 26, 106, 9, 97, 13, 99, 15, 119, 20, 105, 117, 
   98, 103, 118, 1, 126, 29, 97, 122, 17, 15, 114, 110, 3, 5, 125, 
   125, 99, 126, 119, 102, 30, 122, 2, 117}, 1422}, 
{{80, 101, 2, 116, 113, 102, 118, 25, 99, 8, 19, 23, 116, 
   125, 107, 25, 99, 109, 114, 102, 14, 121, 115, 31, 9, 117, 113, 
   111, 5, 4, 127, 28, 122, 101, 8, 110, 14, 18, 124, 106, 16, 20, 
   104, 119, 8, 109, 26, 106, 9, 97, 13, 99, 15, 119, 20, 105, 117, 
   98, 103, 118, 1, 126, 29, 97, 122, 17, 15, 114, 110, 3, 5, 125, 
   125, 99, 126, 119, 102, 30, 122, 2, 117}, 1012},
{{80, 120, 2, 116, 113, 102, 118, 25, 99, 8, 19, 23, 116, 
   125, 107, 25, 99, 109, 114, 102, 14, 121, 115, 31, 9, 117, 113, 
   111, 5, 4, 127, 28, 122, 101, 8, 110, 14, 18, 124, 106, 16, 20, 
   104, 119, 8, 109, 26, 106, 9, 97, 13, 99, 15, 119, 20, 105, 117, 
   98, 103, 118, 1, 126, 29, 97, 122, 17, 15, 114, 110, 3, 5, 125, 
   125, 99, 126, 119, 102, 30, 122, 2, 117}, 377},

...

{{80, 2, 74, 49, 113, 21, 62, 88, 39, 71, 68, 23, 63, 51, 36, 78, 48, 
   108, 114, 102, 14, 121, 115, 31, 9, 117, 113, 111, 5, 4, 127, 28, 
   122, 101, 8, 110, 14, 18, 124, 106, 16, 20, 104, 119, 8, 109, 26, 
   106, 9, 97, 13, 99, 15, 119, 20, 105, 117, 98, 103, 118, 1, 126, 
   29, 97, 122, 17, 15, 114, 110, 3, 5, 125, 125, 99, 126, 119, 102, 
   30, 122, 2, 117}, 1},
{{80, 1, 74, 59, 113, 45, 56, 86, 52, 91, 19, 64, 60, 60, 63, 
   25, 38, 59, 59, 42, 14, 53, 38, 77, 66, 38, 113, 38, 75, 4, 43, 84,
    63, 101, 64, 43, 79, 64, 40, 57, 16, 91, 46, 119, 69, 40, 84, 117,
    9, 97, 13, 99, 15, 119, 20, 105, 117, 98, 103, 118, 1, 126, 29, 
   97, 122, 17, 15, 114, 110, 3, 5, 125, 125, 99, 126, 119, 102, 30, 
   122, 2, 117}, 1},
{{80, 2, 74, 49, 113, 49, 51, 92, 39, 8, 92, 81, 116, 62, 57, 
   80, 46, 40, 114, 36, 75, 56, 33, 76, 9, 55, 56, 59, 81, 65, 45, 28,
    60, 55, 93, 39, 90, 28, 124, 106, 16, 20, 104, 119, 8, 109, 26, 
   106, 9, 97, 13, 99, 15, 119, 20, 105, 117, 98, 103, 118, 1, 126, 
   29, 97, 122, 17, 15, 114, 110, 3, 5, 125, 125, 99, 126, 119, 102, 
   30, 122, 2, 117}, 1}}
\end{lstlisting}

Вывод Tally это список пар, каждая пара это 81-байтный блок и количество раз, сколько он встретился в файле.
Мы видим, что наиболее частно встречающийся блок это первый, он встретился 1739 раз.
Второй встретился 1422 раза. Есть и другие: 1012 раза, 377 раз, итд.
81-байтные блоки, встреченные лишь один раз, находятся в конце вывода.

Попробуем сравнить эти блоки. Первый и второй.
Есть ли в Mathematica ф-ция для сравнения списков/массивов?
Наверняка есть, но в педагогических целях, я буду использоват операцию XOR для сравнения.
Действительно: если байты во входных массивах равны друг другу, результат операции XOR это 0.
Если не равны, результат будет ненулевой.

Сравним первый блок (встречается 1739 раз) и второй (встречается 1422 раз):

\begin{lstlisting}[style=custommath]
In[]:= BitXor[stat[[1]][[1]], stat[[2]][[1]]]
Out[]= {0, 3, 0, 0, 0, 0, 0, 0, 0, 0, 0, 0, 0, 0, 0, 0, 0, 0, 0, \
0, 0, 0, 0, 0, 0, 0, 0, 0, 0, 0, 0, 0, 0, 0, 0, 0, 0, 0, 0, 0, 0, 0, \
0, 0, 0, 0, 0, 0, 0, 0, 0, 0, 0, 0, 0, 0, 0, 0, 0, 0, 0, 0, 0, 0, 0, \
0, 0, 0, 0, 0, 0, 0, 0, 0, 0, 0, 0, 0, 0, 0, 0}
\end{lstlisting}

Они отличаются только вторым байтом.

Сравним второй блок (встречается 1422 раза) и третий (встречается 1012 раз):

\begin{lstlisting}[style=custommath]
In[]:= BitXor[stat[[2]][[1]], stat[[3]][[1]]]
Out[]= {0, 1, 0, 0, 0, 0, 0, 0, 0, 0, 0, 0, 0, 0, 0, 0, 0, 0, 0, \
0, 0, 0, 0, 0, 0, 0, 0, 0, 0, 0, 0, 0, 0, 0, 0, 0, 0, 0, 0, 0, 0, 0, \
0, 0, 0, 0, 0, 0, 0, 0, 0, 0, 0, 0, 0, 0, 0, 0, 0, 0, 0, 0, 0, 0, 0, \
0, 0, 0, 0, 0, 0, 0, 0, 0, 0, 0, 0, 0, 0, 0, 0}
\end{lstlisting}

Они тоже отличаются только вторым байтом.

Так или иначе, попробуем использовать самый встречающийся блок как XOR-ключ и попробуем расшифровать первые 4 81-байтных
блока в файле:

\begin{lstlisting}[style=custommath]
In[]:= key = stat[[1]][[1]]
Out[]= {80, 103, 2, 116, 113, 102, 118, 25, 99, 8, 19, 23, 116, \
125, 107, 25, 99, 109, 114, 102, 14, 121, 115, 31, 9, 117, 113, 111, \
5, 4, 127, 28, 122, 101, 8, 110, 14, 18, 124, 106, 16, 20, 104, 119, \
8, 109, 26, 106, 9, 97, 13, 99, 15, 119, 20, 105, 117, 98, 103, 118, \
1, 126, 29, 97, 122, 17, 15, 114, 110, 3, 5, 125, 125, 99, 126, 119, \
102, 30, 122, 2, 117}

In[]:= ToASCII[val_] := If[val == 0, " ", FromCharacterCode[val, "PrintableASCII"]]

In[]:= DecryptBlockASCII[blk_] := Map[ToASCII[#] &, BitXor[key, blk]]

In[]:= DecryptBlockASCII[blocks[[1]]]
Out[]= {" ", " ", " ", " ", " ", " ", " ", " ", " ", " ", " ", " \
", " ", " ", " ", " ", " ", " ", " ", " ", " ", " ", " ", " ", " ", " \
", " ", " ", " ", " ", " ", " ", " ", " ", " ", " ", " ", " ", " ", " \
", " ", " ", " ", " ", " ", " ", " ", " ", " ", " ", " ", " ", " ", " \
", " ", " ", " ", " ", " ", " ", " ", " ", " ", " ", " ", " ", " ", " \
", " ", " ", " ", " ", " ", " ", " ", " ", " ", " ", " ", " ", " "}

In[]:= DecryptBlockASCII[blocks[[2]]]
Out[]= {" ", "e", "H", "E", " ", "W", "E", "E", "D", " ", "O", \
"F", " ", "C", "R", "I", "M", "E", " ", "B", "E", "A", "R", "S", " ", \
"B", "I", "T", "T", "E", "R", " ", "F", "R", "U", "I", "T", "?", \
" ", " ", " ", " ", " ", " ", " ", " ", " ", " ", " ", " ", " ", " ", \
" ", " ", " ", " ", " ", " ", " ", " ", " ", " ", " ", " ", " ", " ", \
" ", " ", " ", " ", " ", " ", " ", " ", " ", " ", " ", " ", " ", " ", \
" "}

In[]:= DecryptBlockASCII[blocks[[3]]]
Out[]= {" ", "?", " ", " ", " ", " ", " ", " ", " ", " ", " \
", " ", " ", " ", " ", " ", " ", " ", " ", " ", " ", " ", " ", " ", " \
", " ", " ", " ", " ", " ", " ", " ", " ", " ", " ", " ", " ", " ", " \
", " ", " ", " ", " ", " ", " ", " ", " ", " ", " ", " ", " ", " ", " \
", " ", " ", " ", " ", " ", " ", " ", " ", " ", " ", " ", " ", " ", " \
", " ", " ", " ", " ", " ", " ", " ", " ", " ", " ", " ", " ", " ", " \
"}

In[]:= DecryptBlockASCII[blocks[[4]]]
Out[]= {" ", "f", "H", "O", " ", "K", "N", "O", "W", "S", " ", \
"W", "H", "A", "T", " ", "E", "V", "I", "L", " ", "L", "U", "R", "K", \
"S", " ", "I", "N", " ", "T", "H", "E", " ", "H", "E", "A", "R", "T", \
"S", " ", "O", "F", " ", "M", "E", "N", "?", " ", " ", " ", " ", \
" ", " ", " ", " ", " ", " ", " ", " ", " ", " ", " ", " ", " ", " ", \
" ", " ", " ", " ", " ", " ", " ", " ", " ", " ", " ", " ", " ", " ", \
" "}
\end{lstlisting}

(Я заменил непечатаемые символы на \q{?}.)

Мы видим что первый и третий блоки пустые (или почти пустые),
но второй и четвертый имеют ясно различимые английские слова/фразы.
Похоже что наше предположение насчет ключа верно (как минимум частично).
Это означает, что самый встречающийся 81-байтный блок в файле находится в местах лакун с нулевыми байтами
или что-то в этом роде.

Попробуем расшифровать весь файл:

\begin{lstlisting}[style=custommath]
DecryptBlock[blk_] := BitXor[key, blk]

decrypted = Map[DecryptBlock[#] &, blocks];

BinaryWrite["/home/dennis/.../tmp", Flatten[decrypted]]

Close["/home/dennis/.../tmp"]
\end{lstlisting}

\begin{figure}[H]
\centering
\myincludegraphics{ff/XOR/mask_1/mc_decrypted1.png}
\caption{Расшифрованный файл в Midnight Commander, первая попытка}
\end{figure}

Выглядит как английские фразы для какой-то игры, но что-то не так.
Прежде всего, регистр инвертирован: фразы и некоторые слова начинаются со строчных букв,
в то время как остальные буквы заглавные.
Также, некоторые фразы начинаются с не тех букв.
Посмотрите на самую первую фразу: \q{eHE WEED OF CRIME BEARS BITTER FRUIT}.
Что такое \q{eHE}? Разве не \q{tHE} тут должно быть?
Возможно ли что наш ключ для дешифрования имеет неверный байт в этом месте?

Посмотрим снова на второй блок в файле, на ключ и на результат дешифрования:

\begin{lstlisting}[style=custommath]
In[]:= blocks[[2]]
Out[]= {80, 2, 74, 49, 113, 49, 51, 92, 39, 8, 92, 81, 116, 62, \
57, 80, 46, 40, 114, 36, 75, 56, 33, 76, 9, 55, 56, 59, 81, 65, 45, \
28, 60, 55, 93, 39, 90, 28, 124, 106, 16, 20, 104, 119, 8, 109, 26, \
106, 9, 97, 13, 99, 15, 119, 20, 105, 117, 98, 103, 118, 1, 126, 29, \
97, 122, 17, 15, 114, 110, 3, 5, 125, 125, 99, 126, 119, 102, 30, \
122, 2, 117}

In[]:= key
Out[]= {80, 103, 2, 116, 113, 102, 118, 25, 99, 8, 19, 23, 116, \
125, 107, 25, 99, 109, 114, 102, 14, 121, 115, 31, 9, 117, 113, 111, \
5, 4, 127, 28, 122, 101, 8, 110, 14, 18, 124, 106, 16, 20, 104, 119, \
8, 109, 26, 106, 9, 97, 13, 99, 15, 119, 20, 105, 117, 98, 103, 118, \
1, 126, 29, 97, 122, 17, 15, 114, 110, 3, 5, 125, 125, 99, 126, 119, \
102, 30, 122, 2, 117}

In[]:= BitXor[key, blocks[[2]]]
Out[]= {0, 101, 72, 69, 0, 87, 69, 69, 68, 0, 79, 70, 0, 67, 82, \
73, 77, 69, 0, 66, 69, 65, 82, 83, 0, 66, 73, 84, 84, 69, 82, 0, 70, \
82, 85, 73, 84, 14, 0, 0, 0, 0, 0, 0, 0, 0, 0, 0, 0, 0, 0, 0, 0, 0, \
0, 0, 0, 0, 0, 0, 0, 0, 0, 0, 0, 0, 0, 0, 0, 0, 0, 0, 0, 0, 0, 0, 0, \
0, 0, 0, 0}
\end{lstlisting}

Зашифрованный байт это 2, байт из ключа это 103, $2 \oplus 103=101$ и 101 это ASCII-код символа \q{e}.
Чему должен равнятся этот байт ключа, чтобы ASCII-код был 116 (для символа  \q{t})?
$2 \oplus 116=118$, присвоим 118 второму байту в ключе \dots

\begin{lstlisting}[style=custommath]
key = {80, 118, 2, 116, 113, 102, 118, 25, 99, 8, 19, 23, 116, 125, 
  107, 25, 99, 109, 114, 102, 14, 121, 115, 31, 9, 117, 113, 111, 5, 
  4, 127, 28, 122, 101, 8, 110, 14, 18, 124, 106, 16, 20, 104, 119, 8,
   109, 26, 106, 9, 97, 13, 99, 15, 119, 20, 105, 117, 98, 103, 118, 
  1, 126, 29, 97, 122, 17, 15, 114, 110, 3, 5, 125, 125, 99, 126, 119,
   102, 30, 122, 2, 117}
\end{lstlisting}

\dots и снова дешифруем весь файл.

\begin{figure}[H]
\centering
\myincludegraphics{ff/XOR/mask_1/mc_decrypted2.png}
\caption{Дешифрованный файл в Midnight Commander, вторая попытка}
\end{figure}

Ух ты, теперь грамматика корректна, и все фразы начинаются с корректных букв.
Но все таки, регистр подозрителен.
С чего бы разработчику игры записывать их в такой манере?
Может быть наш ключ все еще неправилен?

% TODO ASCII table somewhere in the book
Изучая таблицу ASCII мы можем заметить что ASCII-коды для букв в верхнем и нижнем регистре отличаются только на один бит
(6-й бит, если считать с первого, 0b100000):

\begin{figure}[H]
\centering
\includegraphics[width=0.7\textwidth]{ascii.png}
\caption{7-битная таблица \ac{ASCII} в Emacs}
\end{figure}

6-й бит, выставленный в нулевом байте, В десятичном виде это будет 32.
Но 32 это ASCII-код пробела!

Действительно, можно менять регистр просто применяя XOR к ASCII-коду, с 32 (больше об этом: \myref{toupper_bit}).

Возможно ли, что пустые лакуны в файле это не нулевые байты, а скорее содержащие пробелы?
Еще раз модифицируем наш XOR-ключ (я про-XOR-ю каждый байт ключа с 32):

\begin{lstlisting}[style=custommath]
(* "32" это скаляр, и "key" это вектор, но это OK *)

In[]:= key3 = BitXor[32, key]
Out[]= {112, 86, 34, 84, 81, 70, 86, 57, 67, 40, 51, 55, 84, 93, 75, \
57, 67, 77, 82, 70, 46, 89, 83, 63, 41, 85, 81, 79, 37, 36, 95, 60, \
90, 69, 40, 78, 46, 50, 92, 74, 48, 52, 72, 87, 40, 77, 58, 74, 41, \
65, 45, 67, 47, 87, 52, 73, 85, 66, 71, 86, 33, 94, 61, 65, 90, 49, \
47, 82, 78, 35, 37, 93, 93, 67, 94, 87, 70, 62, 90, 34, 85}

In[]:= DecryptBlock[blk_] := BitXor[key3, blk]
\end{lstlisting}

И снова дешифруем входной файл:

\begin{figure}[H]
\centering
\myincludegraphics{ff/XOR/mask_1/mc_decrypted.png}
\caption{Дешифрованный файл в Midnight Commander, последняя попытка}
\end{figure}

(Расшифрованный файл доступен здесь:
\url{\GitHubBlobMasterURL/ff/XOR/mask_1/files/decrypted.dat.bz2}.)

Несомненно, это корректный исходный файл.
Да, и мы видим числа в начале каждого блока. Должно быть это и есть источник некорректного XOR-ключа.
Как выходит, самый встречающийся 81-байтный блок в файле это блок заполненный пробелами и содержащий символ \q{1} на месте
второго байта.
Действительно, как-то так получилось что многие блоки здесь перемежаются с этим блоком.
Может быть это что-то вроде выравнивания (padding) для коротких фраз/сообщений?
Другой часто встречающийся 81-байтный блок также заполнен пробелами, но с другой цифрой, следовательно,
они отличаются только вторым байтом.

Вот и всё! Теперь мы можем написать утилиту для зашифрования файла назад, и, может быть, модифицировать его перед этим

Файл для Mathematica можно скачать здесь:\\
\url{\GitHubBlobMasterURL/ff/XOR/mask_1/files/XOR_mask_1.nb}.

Итог: XOR-шифрование не надежно вообще. Вероятно, разработчик игры хотел просто скрыть внутренности игры от игрока,
ничего более серьезного.
Все же, шифрование вроде этого крайне популярно вследствии его простоты, так что многие реверс инженеры обычно хорошо
с этим знакомы.

}\FR{\mysection{Fonction presque vide}
\label{Boolector}
\myindex{Boolector}
\myindex{x86!\Instructions!JMP}

Ceci est un morceau de code réel que j'ai trouvé dans Boolector\footnote{\url{https://boolector.github.io/}}:

\lstinputlisting[style=customc]{patterns/025_almost_empty/boolectormain.c}

Pourquoi quelqu'un ferait-il comme ça?
Je ne sais pas mais mon hypothèse est que \verb|boolector_main()| peut être compilée
dans une sorte de DLL ou bibliothèque dynamique, et appelée depuis une suite de test.
Certainement qu'une suite de test peut préparer les variables argc/argv comme
le ferait \ac{CRT}.

Il est intéressant de voir comment c'est compilé:

\lstinputlisting[caption=GCC 8.2 x64 \NonOptimizing (\assemblyOutput),style=customasmx86]{patterns/025_almost_empty/boolectormain_O0.s}

Ceci est OK, le prologue (non optimisé) déplace inutilement deux arguments,
\INS{CALL}, épilogue, \INS{RET}.
Mais regardons la version optimisée:

\lstinputlisting[caption=GCC 8.2 x64 \Optimizing (\assemblyOutput),style=customasmx86]{patterns/025_almost_empty/boolectormain_O3.s}

Aussi simple que ça: la pile et les registres ne sont pas touchés et \verb|boolector_main()|
a le même ensemble d'arguments.
Donc, tout ce que nous avons à faire est de passer l'exécution à une autre adresse.

Ceci est proche d'une \glslink{thunk function}{fonction thunk}.

Nous verons queelque chose de plus avancé plus tard: \myref{ARM_B_to_printf}, \myref{JMP_instead_of_RET}.
}

\EN{% TODO translate
\mysection{Breaking simple executable cryptor}

I've got an executable file which is encrypted by relatively simple encryption.
\href{\GitHubBlobMasterURL/examples/simple_exec_crypto/files/cipher.bin}{Here is it} (only executable section is left here).

First, all encryption function does is just adds number of position in buffer to the byte.
Here is how this can be encoded in Python:

\begin{lstlisting}[caption=Python script,style=custompy]
#!/usr/bin/env python
def e(i, k):
    return chr ((ord(i)+k) % 256)

def encrypt(buf):
    return e(buf[0], 0)+ e(buf[1], 1)+ e(buf[2], 2) + e(buf[3], 3)+ e(buf[4], 4)+ e(buf[5], 5)+ e(buf[6], 6)+ e(buf[7], 7)+
           e(buf[8], 8)+ e(buf[9], 9)+ e(buf[10], 10)+ e(buf[11], 11)+ e(buf[12], 12)+ e(buf[13], 13)+ e(buf[14], 14)+ e(buf[15], 15)
\end{lstlisting}

Hence, if you encrypt buffer with 16 zeros, you'll get \emph{0, 1, 2, 3 ... 12, 13, 14, 15}.

\myindex{Propagating Cipher Block Chaining}
Propagating Cipher Block Chaining (PCBC) is also used, here is how it works:

\begin{figure}[H]
\centering
\myincludegraphics{examples/simple_exec_crypto/601px-PCBC_encryption.png}
\caption{Propagating Cipher Block Chaining encryption (image is taken from Wikipedia article)}
\end{figure}

The problem is that it's too boring to recover IV (Initialization Vector) each time.
Brute-force is also not an option, because IV is too long (16 bytes).
Let's see, if it's possible to recover IV for arbitrary encrypted executable file?

Let's try simple frequency analysis.
This is 32-bit x86 executable code, so let's gather statistics about most frequent bytes and opcodes.
I tried huge oracle.exe file from Oracle RDBMS version 11.2 for windows x86 and I've found that the most frequent byte (no surprise) is zero (~10\%).
The next most frequent byte is (again, no surprise) 0xFF (~5\%).
The next is 0x8B (~5\%).

\myindex{x86!\Instructions!MOV}
0x8B is opcode for \INS{MOV}, this is indeed one of the most busy x86 instructions.
Now what about popularity of zero byte?
If compiler needs to encode value bigger than 127, it has to use 32-bit displacement instead of 8-bit one, but large values are very rare,
so it is padded by zeros.
\myindex{x86!\Instructions!LEA}
\myindex{x86!\Instructions!PUSH}
\myindex{x86!\Instructions!CALL}
This is at least in \INS{LEA}, \INS{MOV}, \INS{PUSH}, \INS{CALL}.

For example:

\begin{lstlisting}[style=customasmx86]
8D B0 28 01 00 00                 lea     esi, [eax+128h]
8D BF 40 38 00 00                 lea     edi, [edi+3840h]
\end{lstlisting}

Displacements bigger than 127 are very popular, but they are rarely exceeds 0x10000
(indeed, such large memory buffers/structures are also rare).

Same story with \INS{MOV}, large constants are rare, the most heavily used are 0, 1, 10, 100, $2^n$, and so on.
Compiler has to pad small constants by zeros to represent them as 32-bit values:

\begin{lstlisting}[style=customasmx86]
BF 02 00 00 00                    mov     edi, 2
BF 01 00 00 00                    mov     edi, 1
\end{lstlisting}

Now about 00 and FF bytes combined: jumps (including conditional) and calls can pass execution flow forward or backwards, but very often,
within the limits of the current executable module.
If forward, displacement is not very big and also padded with zeros.
If backwards, displacement is represented as negative value, so padded with FF bytes.
For example, transfer execution flow forward:

\begin{lstlisting}[style=customasmx86]
E8 43 0C 00 00                    call    _function1
E8 5C 00 00 00                    call    _function2
0F 84 F0 0A 00 00                 jz      loc_4F09A0
0F 84 EB 00 00 00                 jz      loc_4EFBB8
\end{lstlisting}

Backwards:

\begin{lstlisting}[style=customasmx86]
E8 79 0C FE FF                    call    _function1
E8 F4 16 FF FF                    call    _function2
0F 84 F8 FB FF FF                 jz      loc_8212BC
0F 84 06 FD FF FF                 jz      loc_FF1E7D
\end{lstlisting}

FF byte is also very often occurred in negative displacements like these:

\begin{lstlisting}[style=customasmx86]
8D 85 1E FF FF FF                 lea     eax, [ebp-0E2h]
8D 95 F8 5C FF FF                 lea     edx, [ebp-0A308h]
\end{lstlisting}

So far so good. Now we have to try various 16-byte keys, decrypt executable section and measure how often 00, FF and 8B bytes are occurred.
Let's also keep in sight how PCBC decryption works:

\begin{figure}[H]
\centering
\myincludegraphics{examples/simple_exec_crypto/640px-PCBC_decryption.png}
\caption{Propagating Cipher Block Chaining decryption (image is taken from Wikipedia article)}
\end{figure}

The good news is that we don't really have to decrypt whole piece of data, but only slice by slice, this is exactly how I did in my previous example: \myref{XOR_mask_2}.

Now I'm trying all possible bytes (0..255) for each byte in key and just pick the byte producing maximal amount of 00/FF/8B bytes in a decrypted slice:

\begin{lstlisting}[style=custompy]
#!/usr/bin/env python
import sys, hexdump, array, string, operator

KEY_LEN=16

def chunks(l, n):
    # split n by l-byte chunks
    # https://stackoverflow.com/q/312443
    n = max(1, n)
    return [l[i:i + n] for i in range(0, len(l), n)]

def read_file(fname):
    file=open(fname, mode='rb')
    content=file.read()
    file.close()
    return content

def decrypt_byte (c, key):
    return chr((ord(c)-key) % 256)

def XOR_PCBC_step (IV, buf, k):
    prev=IV
    rt=""
    for c in buf:
	new_c=decrypt_byte(c, k)
        plain=chr(ord(new_c)^ord(prev))
	prev=chr(ord(c)^ord(plain))
	rt=rt+plain
    return rt

each_Nth_byte=[""]*KEY_LEN

content=read_file(sys.argv[1])
# split input by 16-byte chunks:
all_chunks=chunks(content, KEY_LEN)
for c in all_chunks:
    for i in range(KEY_LEN):
        each_Nth_byte[i]=each_Nth_byte[i] + c[i]

# try each byte of key
for N in range(KEY_LEN):
    print "N=", N
    stat={}
    for i in range(256):
        tmp_key=chr(i)
	tmp=XOR_PCBC_step(tmp_key,each_Nth_byte[N], N)
        # count 0, FFs and 8Bs in decrypted buffer:
	important_bytes=tmp.count('\x00')+tmp.count('\xFF')+tmp.count('\x8B')
	stat[i]=important_bytes
    sorted_stat = sorted(stat.iteritems(), key=operator.itemgetter(1), reverse=True)
    print sorted_stat[0]
\end{lstlisting}

(Source code can be downloaded \href{\GitHubBlobMasterURL/examples/simple_exec_crypto/files/decrypt.py}{here}.)

I run it and here is a key for which 00/FF/8B bytes presence in decrypted buffer is maximal:

\begin{lstlisting}
N= 0
(147, 1224)
N= 1
(94, 1327)
N= 2
(252, 1223)
N= 3
(218, 1266)
N= 4
(38, 1209)
N= 5
(192, 1378)
N= 6
(199, 1204)
N= 7
(213, 1332)
N= 8
(225, 1251)
N= 9
(112, 1223)
N= 10
(143, 1177)
N= 11
(108, 1286)
N= 12
(10, 1164)
N= 13
(3, 1271)
N= 14
(128, 1253)
N= 15
(232, 1330)
\end{lstlisting}

Let's write decryption utility with the key we got:

\begin{lstlisting}[style=custompy]
#!/usr/bin/env python
import sys, hexdump, array

def xor_strings(s,t):
    # \verb|https://en.wikipedia.org/wiki/XOR_cipher#Example_implementation|
    """xor two strings together"""
    return "".join(chr(ord(a)^ord(b)) for a,b in zip(s,t))

IV=array.array('B', [147, 94, 252, 218, 38, 192, 199, 213, 225, 112, 143, 108, 10, 3, 128, 232]).tostring()

def chunks(l, n):
    n = max(1, n)
    return [l[i:i + n] for i in range(0, len(l), n)]

def read_file(fname):
    file=open(fname, mode='rb')
    content=file.read()
    file.close()
    return content

def decrypt_byte(i, k):
    return chr ((ord(i)-k) % 256)

def decrypt(buf):
    return "".join(decrypt_byte(buf[i], i) for i in range(16))

fout=open(sys.argv[2], mode='wb')

prev=IV
content=read_file(sys.argv[1])
tmp=chunks(content, 16)
for c in tmp:
    new_c=decrypt(c)
    p=xor_strings (new_c, prev)
    prev=xor_strings(c, p)
    fout.write(p)
fout.close()
\end{lstlisting}

(Source code can be downloaded \href{\GitHubBlobMasterURL/examples/simple_exec_crypto/files/decrypt2.py}{here}.)

Let's check resulting file:

\lstinputlisting{examples/simple_exec_crypto/objdump_result.txt}

Yes, this is seems correctly disassembled piece of x86 code.
The whole decryped file can be downloaded \href{\GitHubBlobMasterURL/examples/simple_exec_crypto/files/decrypted.bin}{here}.

In fact, this is text section from regedit.exe from Windows 7.
But this example is based on a real case I encountered, so just executable is different (and key), algorithm is the same.

\subsection{Other ideas to consider}

What if I would fail with such simple frequency analysis?
There are other ideas on how to measure correctness of decrypted/decompressed x86 code:

\begin{itemize}

\item Many modern compilers aligns functions on 0x10 border.
So the space left before is filled with NOPs (0x90) or other NOP instructions with known opcodes: \myref{sec:npad}.

\item Perhaps, the most frequent pattern in any assembly language is function call:\\
\TT{PUSH chain / CALL / ADD ESP, X}.
This sequence can easily detected and found.
I've even gathered statistics about average number of function arguments: \myref{args_stat}.
(Hence, this is average length of PUSH chain.)

\end{itemize}

Read more about incorrectly/correctly disassembled code: \myref{ISA_detect}.
}\RU{\subsection{Простое шифрование используя XOR-маску}
\label{XOR_mask_1}

Я нашел одну старую игру в стиле interactive fiction в архиве \emph{if-archive}\footnote{\url{http://www.ifarchive.org/}}:

\begin{lstlisting}
The New Castle v3.5 - Text/Adventure Game
in the style of the original Infocom (tm)
type games, Zork, Collosal Cave (Adventure),
etc.  Can you solve the mystery of the
abandoned castle?
Shareware from Software Customization.
Software Customization [ASP] Version 3.5 Feb. 2000
\end{lstlisting}

Можно скачать здесь: \url{\GitHubBlobMasterURL/ff/XOR/mask_1/files/newcastle.tgz}.

Там внутри есть файл (с названием \emph{castle.dbf}), который явно зашифрован, но не настоящим криптоалгоритмом,
и оне сжат, это что-то куда проще.
Я бы даже не стал измерять уровень энтропии (\myref{entropy}) этого файла, потому что я итак уверен, что он низкий.
Вот как он выглядит в Midnight Commander:

\begin{figure}[H]
\centering
\myincludegraphics{ff/XOR/mask_1/mc_encrypted.png}
\caption{Зашифрованный файл в Midnight Commander}
\end{figure}

Зашифрованный файл можно скачать здесь:
\url{\GitHubBlobMasterURL/ff/XOR/mask_1/files/castle.dbf.bz2}.

Можно ли расшифровать его без доступа к программе, используя просто этот файл?

Тут явно просматривается повторяющаяся строка. 
Если использовалось простое шифрование с XOR-маской, такие повторяющиеся строки это явное свидетельство,
потому что, вероятно, тут были длинные лакуны с нулевыми байтами, которые, в свою очередь, присутствуют
во мноигих исполняемых файлах, и в остальных бинарных файлах.

\myindex{UNIX!xxd}
Вот дам начала этого файла используя утилиту \emph{xxd} из UNIX:

\lstinputlisting{ff/XOR/mask_1/xxd_result.txt}

Давайте держаться за повторяющуюся строку \TT{iubgv}.
Глядя на этот дамп, мы можем легко увидеть, что период повторений этой строки это 0x51 или 81.
Вероятно, 81 это длина блока?
Длина файла 1658961, и она может быть поделена на 81 без остатка (и тогда там 20481 блоков).

Теперь я буду использовать Mathematica для анализа, есть ли тут повторяющиеся 81-байтные блоки в файле?
Я разделю входной файл на 81-байтные блоки и затем использую ф-цию
\emph{Tally[]}\footnote{\url{https://reference.wolfram.com/language/ref/Tally.html}}
которая просто считает, сколько раз каждый элемент встретился во входном списке.
Вывод Tally не отсортирован, так что я также добавлю ф-цию \emph{Sort[]} для сортировки его по кол-ву вхождений
в нисходящем порядке.

\begin{lstlisting}[style=custommath]
input = BinaryReadList["/home/dennis/.../castle.dbf"];

blocks = Partition[input, 81];

stat = Sort[Tally[blocks], #1[[2]] > #2[[2]] &]
\end{lstlisting}

И вот вывод:

\begin{lstlisting}[style=custommath]
{{{80, 103, 2, 116, 113, 102, 118, 25, 99, 8, 19, 23, 116, 125, 107, 
   25, 99, 109, 114, 102, 14, 121, 115, 31, 9, 117, 113, 111, 5, 4, 
   127, 28, 122, 101, 8, 110, 14, 18, 124, 106, 16, 20, 104, 119, 8, 
   109, 26, 106, 9, 97, 13, 99, 15, 119, 20, 105, 117, 98, 103, 118, 
   1, 126, 29, 97, 122, 17, 15, 114, 110, 3, 5, 125, 125, 99, 126, 
   119, 102, 30, 122, 2, 117}, 1739}, 
{{80, 100, 2, 116, 113, 102, 118, 25, 99, 8, 19, 23, 116, 
   125, 107, 25, 99, 109, 114, 102, 14, 121, 115, 31, 9, 117, 113, 
   111, 5, 4, 127, 28, 122, 101, 8, 110, 14, 18, 124, 106, 16, 20, 
   104, 119, 8, 109, 26, 106, 9, 97, 13, 99, 15, 119, 20, 105, 117, 
   98, 103, 118, 1, 126, 29, 97, 122, 17, 15, 114, 110, 3, 5, 125, 
   125, 99, 126, 119, 102, 30, 122, 2, 117}, 1422}, 
{{80, 101, 2, 116, 113, 102, 118, 25, 99, 8, 19, 23, 116, 
   125, 107, 25, 99, 109, 114, 102, 14, 121, 115, 31, 9, 117, 113, 
   111, 5, 4, 127, 28, 122, 101, 8, 110, 14, 18, 124, 106, 16, 20, 
   104, 119, 8, 109, 26, 106, 9, 97, 13, 99, 15, 119, 20, 105, 117, 
   98, 103, 118, 1, 126, 29, 97, 122, 17, 15, 114, 110, 3, 5, 125, 
   125, 99, 126, 119, 102, 30, 122, 2, 117}, 1012},
{{80, 120, 2, 116, 113, 102, 118, 25, 99, 8, 19, 23, 116, 
   125, 107, 25, 99, 109, 114, 102, 14, 121, 115, 31, 9, 117, 113, 
   111, 5, 4, 127, 28, 122, 101, 8, 110, 14, 18, 124, 106, 16, 20, 
   104, 119, 8, 109, 26, 106, 9, 97, 13, 99, 15, 119, 20, 105, 117, 
   98, 103, 118, 1, 126, 29, 97, 122, 17, 15, 114, 110, 3, 5, 125, 
   125, 99, 126, 119, 102, 30, 122, 2, 117}, 377},

...

{{80, 2, 74, 49, 113, 21, 62, 88, 39, 71, 68, 23, 63, 51, 36, 78, 48, 
   108, 114, 102, 14, 121, 115, 31, 9, 117, 113, 111, 5, 4, 127, 28, 
   122, 101, 8, 110, 14, 18, 124, 106, 16, 20, 104, 119, 8, 109, 26, 
   106, 9, 97, 13, 99, 15, 119, 20, 105, 117, 98, 103, 118, 1, 126, 
   29, 97, 122, 17, 15, 114, 110, 3, 5, 125, 125, 99, 126, 119, 102, 
   30, 122, 2, 117}, 1},
{{80, 1, 74, 59, 113, 45, 56, 86, 52, 91, 19, 64, 60, 60, 63, 
   25, 38, 59, 59, 42, 14, 53, 38, 77, 66, 38, 113, 38, 75, 4, 43, 84,
    63, 101, 64, 43, 79, 64, 40, 57, 16, 91, 46, 119, 69, 40, 84, 117,
    9, 97, 13, 99, 15, 119, 20, 105, 117, 98, 103, 118, 1, 126, 29, 
   97, 122, 17, 15, 114, 110, 3, 5, 125, 125, 99, 126, 119, 102, 30, 
   122, 2, 117}, 1},
{{80, 2, 74, 49, 113, 49, 51, 92, 39, 8, 92, 81, 116, 62, 57, 
   80, 46, 40, 114, 36, 75, 56, 33, 76, 9, 55, 56, 59, 81, 65, 45, 28,
    60, 55, 93, 39, 90, 28, 124, 106, 16, 20, 104, 119, 8, 109, 26, 
   106, 9, 97, 13, 99, 15, 119, 20, 105, 117, 98, 103, 118, 1, 126, 
   29, 97, 122, 17, 15, 114, 110, 3, 5, 125, 125, 99, 126, 119, 102, 
   30, 122, 2, 117}, 1}}
\end{lstlisting}

Вывод Tally это список пар, каждая пара это 81-байтный блок и количество раз, сколько он встретился в файле.
Мы видим, что наиболее частно встречающийся блок это первый, он встретился 1739 раз.
Второй встретился 1422 раза. Есть и другие: 1012 раза, 377 раз, итд.
81-байтные блоки, встреченные лишь один раз, находятся в конце вывода.

Попробуем сравнить эти блоки. Первый и второй.
Есть ли в Mathematica ф-ция для сравнения списков/массивов?
Наверняка есть, но в педагогических целях, я буду использоват операцию XOR для сравнения.
Действительно: если байты во входных массивах равны друг другу, результат операции XOR это 0.
Если не равны, результат будет ненулевой.

Сравним первый блок (встречается 1739 раз) и второй (встречается 1422 раз):

\begin{lstlisting}[style=custommath]
In[]:= BitXor[stat[[1]][[1]], stat[[2]][[1]]]
Out[]= {0, 3, 0, 0, 0, 0, 0, 0, 0, 0, 0, 0, 0, 0, 0, 0, 0, 0, 0, \
0, 0, 0, 0, 0, 0, 0, 0, 0, 0, 0, 0, 0, 0, 0, 0, 0, 0, 0, 0, 0, 0, 0, \
0, 0, 0, 0, 0, 0, 0, 0, 0, 0, 0, 0, 0, 0, 0, 0, 0, 0, 0, 0, 0, 0, 0, \
0, 0, 0, 0, 0, 0, 0, 0, 0, 0, 0, 0, 0, 0, 0, 0}
\end{lstlisting}

Они отличаются только вторым байтом.

Сравним второй блок (встречается 1422 раза) и третий (встречается 1012 раз):

\begin{lstlisting}[style=custommath]
In[]:= BitXor[stat[[2]][[1]], stat[[3]][[1]]]
Out[]= {0, 1, 0, 0, 0, 0, 0, 0, 0, 0, 0, 0, 0, 0, 0, 0, 0, 0, 0, \
0, 0, 0, 0, 0, 0, 0, 0, 0, 0, 0, 0, 0, 0, 0, 0, 0, 0, 0, 0, 0, 0, 0, \
0, 0, 0, 0, 0, 0, 0, 0, 0, 0, 0, 0, 0, 0, 0, 0, 0, 0, 0, 0, 0, 0, 0, \
0, 0, 0, 0, 0, 0, 0, 0, 0, 0, 0, 0, 0, 0, 0, 0}
\end{lstlisting}

Они тоже отличаются только вторым байтом.

Так или иначе, попробуем использовать самый встречающийся блок как XOR-ключ и попробуем расшифровать первые 4 81-байтных
блока в файле:

\begin{lstlisting}[style=custommath]
In[]:= key = stat[[1]][[1]]
Out[]= {80, 103, 2, 116, 113, 102, 118, 25, 99, 8, 19, 23, 116, \
125, 107, 25, 99, 109, 114, 102, 14, 121, 115, 31, 9, 117, 113, 111, \
5, 4, 127, 28, 122, 101, 8, 110, 14, 18, 124, 106, 16, 20, 104, 119, \
8, 109, 26, 106, 9, 97, 13, 99, 15, 119, 20, 105, 117, 98, 103, 118, \
1, 126, 29, 97, 122, 17, 15, 114, 110, 3, 5, 125, 125, 99, 126, 119, \
102, 30, 122, 2, 117}

In[]:= ToASCII[val_] := If[val == 0, " ", FromCharacterCode[val, "PrintableASCII"]]

In[]:= DecryptBlockASCII[blk_] := Map[ToASCII[#] &, BitXor[key, blk]]

In[]:= DecryptBlockASCII[blocks[[1]]]
Out[]= {" ", " ", " ", " ", " ", " ", " ", " ", " ", " ", " ", " \
", " ", " ", " ", " ", " ", " ", " ", " ", " ", " ", " ", " ", " ", " \
", " ", " ", " ", " ", " ", " ", " ", " ", " ", " ", " ", " ", " ", " \
", " ", " ", " ", " ", " ", " ", " ", " ", " ", " ", " ", " ", " ", " \
", " ", " ", " ", " ", " ", " ", " ", " ", " ", " ", " ", " ", " ", " \
", " ", " ", " ", " ", " ", " ", " ", " ", " ", " ", " ", " ", " "}

In[]:= DecryptBlockASCII[blocks[[2]]]
Out[]= {" ", "e", "H", "E", " ", "W", "E", "E", "D", " ", "O", \
"F", " ", "C", "R", "I", "M", "E", " ", "B", "E", "A", "R", "S", " ", \
"B", "I", "T", "T", "E", "R", " ", "F", "R", "U", "I", "T", "?", \
" ", " ", " ", " ", " ", " ", " ", " ", " ", " ", " ", " ", " ", " ", \
" ", " ", " ", " ", " ", " ", " ", " ", " ", " ", " ", " ", " ", " ", \
" ", " ", " ", " ", " ", " ", " ", " ", " ", " ", " ", " ", " ", " ", \
" "}

In[]:= DecryptBlockASCII[blocks[[3]]]
Out[]= {" ", "?", " ", " ", " ", " ", " ", " ", " ", " ", " \
", " ", " ", " ", " ", " ", " ", " ", " ", " ", " ", " ", " ", " ", " \
", " ", " ", " ", " ", " ", " ", " ", " ", " ", " ", " ", " ", " ", " \
", " ", " ", " ", " ", " ", " ", " ", " ", " ", " ", " ", " ", " ", " \
", " ", " ", " ", " ", " ", " ", " ", " ", " ", " ", " ", " ", " ", " \
", " ", " ", " ", " ", " ", " ", " ", " ", " ", " ", " ", " ", " ", " \
"}

In[]:= DecryptBlockASCII[blocks[[4]]]
Out[]= {" ", "f", "H", "O", " ", "K", "N", "O", "W", "S", " ", \
"W", "H", "A", "T", " ", "E", "V", "I", "L", " ", "L", "U", "R", "K", \
"S", " ", "I", "N", " ", "T", "H", "E", " ", "H", "E", "A", "R", "T", \
"S", " ", "O", "F", " ", "M", "E", "N", "?", " ", " ", " ", " ", \
" ", " ", " ", " ", " ", " ", " ", " ", " ", " ", " ", " ", " ", " ", \
" ", " ", " ", " ", " ", " ", " ", " ", " ", " ", " ", " ", " ", " ", \
" "}
\end{lstlisting}

(Я заменил непечатаемые символы на \q{?}.)

Мы видим что первый и третий блоки пустые (или почти пустые),
но второй и четвертый имеют ясно различимые английские слова/фразы.
Похоже что наше предположение насчет ключа верно (как минимум частично).
Это означает, что самый встречающийся 81-байтный блок в файле находится в местах лакун с нулевыми байтами
или что-то в этом роде.

Попробуем расшифровать весь файл:

\begin{lstlisting}[style=custommath]
DecryptBlock[blk_] := BitXor[key, blk]

decrypted = Map[DecryptBlock[#] &, blocks];

BinaryWrite["/home/dennis/.../tmp", Flatten[decrypted]]

Close["/home/dennis/.../tmp"]
\end{lstlisting}

\begin{figure}[H]
\centering
\myincludegraphics{ff/XOR/mask_1/mc_decrypted1.png}
\caption{Расшифрованный файл в Midnight Commander, первая попытка}
\end{figure}

Выглядит как английские фразы для какой-то игры, но что-то не так.
Прежде всего, регистр инвертирован: фразы и некоторые слова начинаются со строчных букв,
в то время как остальные буквы заглавные.
Также, некоторые фразы начинаются с не тех букв.
Посмотрите на самую первую фразу: \q{eHE WEED OF CRIME BEARS BITTER FRUIT}.
Что такое \q{eHE}? Разве не \q{tHE} тут должно быть?
Возможно ли что наш ключ для дешифрования имеет неверный байт в этом месте?

Посмотрим снова на второй блок в файле, на ключ и на результат дешифрования:

\begin{lstlisting}[style=custommath]
In[]:= blocks[[2]]
Out[]= {80, 2, 74, 49, 113, 49, 51, 92, 39, 8, 92, 81, 116, 62, \
57, 80, 46, 40, 114, 36, 75, 56, 33, 76, 9, 55, 56, 59, 81, 65, 45, \
28, 60, 55, 93, 39, 90, 28, 124, 106, 16, 20, 104, 119, 8, 109, 26, \
106, 9, 97, 13, 99, 15, 119, 20, 105, 117, 98, 103, 118, 1, 126, 29, \
97, 122, 17, 15, 114, 110, 3, 5, 125, 125, 99, 126, 119, 102, 30, \
122, 2, 117}

In[]:= key
Out[]= {80, 103, 2, 116, 113, 102, 118, 25, 99, 8, 19, 23, 116, \
125, 107, 25, 99, 109, 114, 102, 14, 121, 115, 31, 9, 117, 113, 111, \
5, 4, 127, 28, 122, 101, 8, 110, 14, 18, 124, 106, 16, 20, 104, 119, \
8, 109, 26, 106, 9, 97, 13, 99, 15, 119, 20, 105, 117, 98, 103, 118, \
1, 126, 29, 97, 122, 17, 15, 114, 110, 3, 5, 125, 125, 99, 126, 119, \
102, 30, 122, 2, 117}

In[]:= BitXor[key, blocks[[2]]]
Out[]= {0, 101, 72, 69, 0, 87, 69, 69, 68, 0, 79, 70, 0, 67, 82, \
73, 77, 69, 0, 66, 69, 65, 82, 83, 0, 66, 73, 84, 84, 69, 82, 0, 70, \
82, 85, 73, 84, 14, 0, 0, 0, 0, 0, 0, 0, 0, 0, 0, 0, 0, 0, 0, 0, 0, \
0, 0, 0, 0, 0, 0, 0, 0, 0, 0, 0, 0, 0, 0, 0, 0, 0, 0, 0, 0, 0, 0, 0, \
0, 0, 0, 0}
\end{lstlisting}

Зашифрованный байт это 2, байт из ключа это 103, $2 \oplus 103=101$ и 101 это ASCII-код символа \q{e}.
Чему должен равнятся этот байт ключа, чтобы ASCII-код был 116 (для символа  \q{t})?
$2 \oplus 116=118$, присвоим 118 второму байту в ключе \dots

\begin{lstlisting}[style=custommath]
key = {80, 118, 2, 116, 113, 102, 118, 25, 99, 8, 19, 23, 116, 125, 
  107, 25, 99, 109, 114, 102, 14, 121, 115, 31, 9, 117, 113, 111, 5, 
  4, 127, 28, 122, 101, 8, 110, 14, 18, 124, 106, 16, 20, 104, 119, 8,
   109, 26, 106, 9, 97, 13, 99, 15, 119, 20, 105, 117, 98, 103, 118, 
  1, 126, 29, 97, 122, 17, 15, 114, 110, 3, 5, 125, 125, 99, 126, 119,
   102, 30, 122, 2, 117}
\end{lstlisting}

\dots и снова дешифруем весь файл.

\begin{figure}[H]
\centering
\myincludegraphics{ff/XOR/mask_1/mc_decrypted2.png}
\caption{Дешифрованный файл в Midnight Commander, вторая попытка}
\end{figure}

Ух ты, теперь грамматика корректна, и все фразы начинаются с корректных букв.
Но все таки, регистр подозрителен.
С чего бы разработчику игры записывать их в такой манере?
Может быть наш ключ все еще неправилен?

% TODO ASCII table somewhere in the book
Изучая таблицу ASCII мы можем заметить что ASCII-коды для букв в верхнем и нижнем регистре отличаются только на один бит
(6-й бит, если считать с первого, 0b100000):

\begin{figure}[H]
\centering
\includegraphics[width=0.7\textwidth]{ascii.png}
\caption{7-битная таблица \ac{ASCII} в Emacs}
\end{figure}

6-й бит, выставленный в нулевом байте, В десятичном виде это будет 32.
Но 32 это ASCII-код пробела!

Действительно, можно менять регистр просто применяя XOR к ASCII-коду, с 32 (больше об этом: \myref{toupper_bit}).

Возможно ли, что пустые лакуны в файле это не нулевые байты, а скорее содержащие пробелы?
Еще раз модифицируем наш XOR-ключ (я про-XOR-ю каждый байт ключа с 32):

\begin{lstlisting}[style=custommath]
(* "32" это скаляр, и "key" это вектор, но это OK *)

In[]:= key3 = BitXor[32, key]
Out[]= {112, 86, 34, 84, 81, 70, 86, 57, 67, 40, 51, 55, 84, 93, 75, \
57, 67, 77, 82, 70, 46, 89, 83, 63, 41, 85, 81, 79, 37, 36, 95, 60, \
90, 69, 40, 78, 46, 50, 92, 74, 48, 52, 72, 87, 40, 77, 58, 74, 41, \
65, 45, 67, 47, 87, 52, 73, 85, 66, 71, 86, 33, 94, 61, 65, 90, 49, \
47, 82, 78, 35, 37, 93, 93, 67, 94, 87, 70, 62, 90, 34, 85}

In[]:= DecryptBlock[blk_] := BitXor[key3, blk]
\end{lstlisting}

И снова дешифруем входной файл:

\begin{figure}[H]
\centering
\myincludegraphics{ff/XOR/mask_1/mc_decrypted.png}
\caption{Дешифрованный файл в Midnight Commander, последняя попытка}
\end{figure}

(Расшифрованный файл доступен здесь:
\url{\GitHubBlobMasterURL/ff/XOR/mask_1/files/decrypted.dat.bz2}.)

Несомненно, это корректный исходный файл.
Да, и мы видим числа в начале каждого блока. Должно быть это и есть источник некорректного XOR-ключа.
Как выходит, самый встречающийся 81-байтный блок в файле это блок заполненный пробелами и содержащий символ \q{1} на месте
второго байта.
Действительно, как-то так получилось что многие блоки здесь перемежаются с этим блоком.
Может быть это что-то вроде выравнивания (padding) для коротких фраз/сообщений?
Другой часто встречающийся 81-байтный блок также заполнен пробелами, но с другой цифрой, следовательно,
они отличаются только вторым байтом.

Вот и всё! Теперь мы можем написать утилиту для зашифрования файла назад, и, может быть, модифицировать его перед этим

Файл для Mathematica можно скачать здесь:\\
\url{\GitHubBlobMasterURL/ff/XOR/mask_1/files/XOR_mask_1.nb}.

Итог: XOR-шифрование не надежно вообще. Вероятно, разработчик игры хотел просто скрыть внутренности игры от игрока,
ничего более серьезного.
Все же, шифрование вроде этого крайне популярно вследствии его простоты, так что многие реверс инженеры обычно хорошо
с этим знакомы.

}\FR{\mysection{Fonction presque vide}
\label{Boolector}
\myindex{Boolector}
\myindex{x86!\Instructions!JMP}

Ceci est un morceau de code réel que j'ai trouvé dans Boolector\footnote{\url{https://boolector.github.io/}}:

\lstinputlisting[style=customc]{patterns/025_almost_empty/boolectormain.c}

Pourquoi quelqu'un ferait-il comme ça?
Je ne sais pas mais mon hypothèse est que \verb|boolector_main()| peut être compilée
dans une sorte de DLL ou bibliothèque dynamique, et appelée depuis une suite de test.
Certainement qu'une suite de test peut préparer les variables argc/argv comme
le ferait \ac{CRT}.

Il est intéressant de voir comment c'est compilé:

\lstinputlisting[caption=GCC 8.2 x64 \NonOptimizing (\assemblyOutput),style=customasmx86]{patterns/025_almost_empty/boolectormain_O0.s}

Ceci est OK, le prologue (non optimisé) déplace inutilement deux arguments,
\INS{CALL}, épilogue, \INS{RET}.
Mais regardons la version optimisée:

\lstinputlisting[caption=GCC 8.2 x64 \Optimizing (\assemblyOutput),style=customasmx86]{patterns/025_almost_empty/boolectormain_O3.s}

Aussi simple que ça: la pile et les registres ne sont pas touchés et \verb|boolector_main()|
a le même ensemble d'arguments.
Donc, tout ce que nous avons à faire est de passer l'exécution à une autre adresse.

Ceci est proche d'une \glslink{thunk function}{fonction thunk}.

Nous verons queelque chose de plus avancé plus tard: \myref{ARM_B_to_printf}, \myref{JMP_instead_of_RET}.
}

\EN{\EN{\input{patterns/016_empty_redux/main_EN}}%
\FR{\input{patterns/016_empty_redux/main_FR}}
}\RU{\EN{\input{patterns/016_empty_redux/main_EN}}%
\FR{\input{patterns/016_empty_redux/main_FR}}
}\FR{\EN{\input{patterns/016_empty_redux/main_EN}}%
\FR{\input{patterns/016_empty_redux/main_FR}}
}

\EN{\mysection{OpenMP}
\label{openmp}
\myindex{OpenMP}

OpenMP is one of the simplest ways to parallelize simple algorithms.

\myindex{Bitcoin}

As an example, let's try to build a program to compute a cryptographic \emph{nonce}.

In my simplistic example, 
the \emph{nonce} is a number added to the plain unencrypted text in order to produce a hash with some specific 
features.

For example, at some step, the Bitcoin protocol requires to find such \emph{nonce} so the resulting hash
contains a specific number of consecutive zeros.
This is also called \emph{proof of work}
(i.e., the system proves that it did some intensive calculations and spent some time for it).

\myindex{SHA512}
My example is not related to Bitcoin in any way, 
it will try to add numbers to the \q{hello, world!\_}
string in order to find such number that when 
\q{hello, world!\_<number>} is hashed with the SHA512 algorithm, it will contain at least 3 zero bytes.

Let's limit our brute-force to the interval in
0..INT32\_MAX-1 (i.e., \TT{0x7FFFFFFE} or 2147483646).

The algorithm is pretty straightforward:

\lstinputlisting[style=customc]{advanced/700_openmp/openmp_example.c}

The \TT{check\_nonce()} function just adds a number to the string, 
hashes it with the SHA512 algorithm and checks for 3 zero bytes in the result.

A very important part of the code is:

\begin{lstlisting}[style=customc]
	#pragma omp parallel for
	for (i=0; i<INT32_MAX; i++)
		check_nonce (i);
\end{lstlisting}

Yes, that simple, without \TT{\#pragma} 
we just call \TT{check\_nonce()} for each number from 0 to 
\TT{INT32\_MAX} (\TT{0x7fffffff} or 2147483647).
With \TT{\#pragma}, the compiler adds some special 
code which slices the loop interval into smaller ones,
to run them on all \ac{CPU} cores available
\footnote{N.B.: This is intentionally simplest possible
example, but in practice, the usage of OpenMP can be harder and more complex}.

The example can be compiled
\footnote{sha512.(c|h) and u64.h 
files can be taken from the OpenSSL library:
\url{http://go.yurichev.com/17324}}
in MSVC 2012:
% FIXME1: \footnote{other source code files can be downloaded here: ...} 

\begin{lstlisting}
cl openmp_example.c sha512.obj /openmp /O1 /Zi /Faopenmp_example.asm
\end{lstlisting}

Or in GCC:

\begin{lstlisting}
gcc -fopenmp 2.c sha512.c -S -masm=intel
\end{lstlisting}

\subsection{MSVC}

Now this is how MSVC 2012 generates the main loop:

\lstinputlisting[caption=MSVC 2012,style=customasmx86]{advanced/700_openmp/MSVC_loop.asm}

All functions prefixed by \TT{vcomp} 
are OpenMP-related and are stored in the 
vcomp*.dll file.
So here a group of threads is started.

Let's take a look on \TT{\_main\$omp\$1}:

\lstinputlisting[caption=MSVC 2012,style=customasmx86]{advanced/700_openmp/MSVC_1.asm}

This function is to be started $n$ 
times in parallel, where $n$ is the number of \ac{CPU} cores.\\
\TT{vcomp\_for\_static\_simple\_init()} 
calculates the interval for the for() 
construct for the current
thread, depending on the current thread's number.

The loop's start and end values are stored in the \TT{\$T1} and \TT{\$T2} local variables.
You may also notice \TT{7ffffffeh} (or 2147483646) 
as an argument to the 
\TT{vcomp\_for\_static\_simple\_init()}
function---this is the number of iterations for the whole loop, to be divided evenly.

Then we see a new loop with a call to the 
\TT{check\_nonce()} function, which does all the work.

Let's also add some code at the beginning of the \TT{check\_nonce()} function to gather statistics about
the arguments with which the function has been called.

This is what we see when we run it:

\begin{lstlisting}
threads=4
...
checked=2800000
checked=3000000
checked=3200000
checked=3300000
found (thread 3): [hello, world!_1611446522]. seconds spent=3
__min[0]=0x00000000 __max[0]=0x1fffffff
__min[1]=0x20000000 __max[1]=0x3fffffff
__min[2]=0x40000000 __max[2]=0x5fffffff
__min[3]=0x60000000 __max[3]=0x7ffffffe
\end{lstlisting}

Yes, the result is correct, the first 3 bytes are zeros:

\begin{lstlisting}
C:\...\sha512sum test
000000f4a8fac5a4ed38794da4c1e39f54279ad5d9bb3c5465cdf57adaf60403
df6e3fe6019f5764fc9975e505a7395fed780fee50eb38dd4c0279cb114672e2 *test
\end{lstlisting}

The running time is $\approx2..3$ seconds on 4-core Intel Xeon E3-1220 3.10 GHz.
In the task manager we see 5 threads: 
1 main thread + 4 more.
No further optimizations are done to keep this example as small and clear as possible.
But probably it can be done much faster.
My \ac{CPU} has 4 cores, that is why OpenMP 
started exactly 4 threads.

By looking at the statistics table we can clearly see how the loop has been sliced into 4 even parts.
Oh well, almost even, if we don't consider the last bit.

There are also pragmas for 
\glslink{atomic operation}{atomic operations}.

Let's see how this code is compiled:

\begin{lstlisting}[style=customc]
	#pragma omp atomic
	checked++;

	#pragma omp critical
	if ((checked % 100000)==0)
		printf ("checked=%d\n", checked);
\end{lstlisting}

\lstinputlisting[caption=MSVC 2012,style=customasmx86]{advanced/700_openmp/MSVC_2.asm}

As it turns out, the \TT{vcomp\_atomic\_add\_i4()} 
function in the vcomp*.dll 
is just a tiny function
with the \TT{LOCK XADD} instruction\footnote{
Read more about LOCK prefix: \myref{x86_lock}} in it.

\TT{vcomp\_enter\_critsect()} 
eventually calling win32 \ac{API} function \\
\TT{EnterCriticalSection()}
\footnote{You can read more about critical sections 
here: \myref{critical_sections}}.

\subsection{GCC}

GCC 4.8.1 
produces a program which shows exactly the same statistics table, 

so, GCC's implementation divides the loop in parts in the same fashion.

\begin{lstlisting}[caption=GCC 4.8.1,style=customasmx86]
	mov	edi, OFFSET FLAT:main._omp_fn.0
	call	GOMP_parallel_start
	mov	edi, 0
	call	main._omp_fn.0
	call	GOMP_parallel_end
\end{lstlisting}

Unlike MSVC's implementation, what GCC code does is to start 3 threads,
and run the fourth in the current thread. So there are 4 threads instead of the 5 in MSVC.

Here is the \TT{main.\_omp\_fn.0} function:
 
\lstinputlisting[caption=GCC 4.8.1,style=customasmx86]{advanced/700_openmp/GCC_1.asm}

Here we see the division clearly: by calling 
\TT{omp\_get\_num\_threads()} and \TT{omp\_get\_thread\_num()}

we get the number of threads running, and also the current thread's number, and then determine the loop's interval.
Then we run \TT{check\_nonce()}.

GCC also inserted the \TT{LOCK ADD} 

instruction right in the code, unlike MSVC, which generated a call to a separate DLL function:

\lstinputlisting[caption=GCC 4.8.1,style=customasmx86]{advanced/700_openmp/GCC_2.asm}

The functions prefixed with GOMP are from GNU OpenMP library.
Unlike vcomp*.dll, its source code is freely available: 
\href{http://go.yurichev.com/17102}{GitHub}.

}\RU{\mysection{OpenMP}
\label{openmp}
\myindex{OpenMP}

OpenMP это один из простейших способов распараллелить работу простого алгоритма.

\myindex{Bitcoin}
В качестве примера, попытаемся написать программу для вычисления криптографического \emph{nonce}.
В моем простейшем примере, \emph{nonce} это число, добавляемое к нешифрованному тексту, чтобы получить
хэш с какой-то особенностью.
Например, на одной из стадии, протокол Bitcoin требует найти такую \emph{nonce}, чтобы в результате
хэширования подряд шли определенное количество нулей.

Это еще называется \emph{proof of work}
(т.е. система доказывает, что она произвела какие-то очень ресурсоёмкие вычисления и затратила
время на это).

\myindex{SHA512}
Мой пример не связан с Bitcoin, 
он будет пытаться добавлять числа к строке \q{hello, world!\_}
чтобы найти такое число, при котором строка вида 
\q{hello, world!\_<number>} после хеширования алгоритмом SHA512 будет содержать как минимум 3 нулевых
байта в начале.

Ограничимся перебором всех чисел в интервале
0..INT32\_MAX-1 (т.е., \TT{0x7FFFFFFE} или 2147483646).

Алгоритм очень простой:

\lstinputlisting[style=customc]{advanced/700_openmp/openmp_example.c}

\TT{check\_nonce()} просто добавляет число к строке, хеширует алгоритмом SHA512 и проверяет 
3 нулевых байта в начале.

Очень важная часть кода --- это:

\begin{lstlisting}[style=customc]
	#pragma omp parallel for
	for (i=0; i<INT32_MAX; i++)
		check_nonce (i);
\end{lstlisting}

Да, вот настолько просто, без \TT{\#pragma} мы просто вызываем
 \TT{check\_nonce()} для каждого числа от 0 до 
\TT{INT32\_MAX} (\TT{0x7fffffff} или 2147483647).
С \TT{\#pragma}, компилятор добавляет специальный код, который разрежет интервал цикла
на меньшие интервалы, чтобы запустить их на доступных ядрах \ac{CPU}
\footnote{N.B.: Это намеренно упрощенный пример, но на практике, 
применение OpenMP может быть труднее и сложнее}.

Пример может быть скомпилирован
\footnote{файлы sha512.(c|h) и u64.h можно взять из библиотеки OpenSSL:
\url{http://go.yurichev.com/17324}}
в MSVC 2012:
% FIXME1: \footnote{other source code files can be downloaded here: ...} 

\begin{lstlisting}
cl openmp_example.c sha512.obj /openmp /O1 /Zi /Faopenmp_example.asm
\end{lstlisting}

Или в GCC:

\begin{lstlisting}
gcc -fopenmp 2.c sha512.c -S -masm=intel
\end{lstlisting}

\subsection{MSVC}

Вот как MSVC 2012 генерирует главный цикл:

\lstinputlisting[caption=MSVC 2012,style=customasmx86]{advanced/700_openmp/MSVC_loop.asm}

Функции с префиксом \TT{vcomp} связаны с OpenMP и находятся в файле vcomp*.dll.
Так что тут запускается группа тредов.

Посмотрим на \TT{\_main\$omp\$1}:

\lstinputlisting[caption=MSVC 2012,style=customasmx86]{advanced/700_openmp/MSVC_1.asm}

Эта функция будет запущена $n$ раз параллельно, где $n$ это число ядер \ac{CPU}.\\
\TT{vcomp\_for\_static\_simple\_init()} вычисляет интервал для конструкта
for() для текущего треда, в зависимости от текущего номера треда.
Значения начала и конца цикла записаны в локальных переменных \TT{\$T1} и \TT{\$T2}.
Вы также можете заметить \TT{7ffffffeh} (или 2147483646) как аргумент для функции 
\TT{vcomp\_for\_static\_simple\_init()}
это количество итераций всего цикла, оно будет поделено на равные части.

Потом мы видим новый цикл с вызовом функции \TT{check\_nonce()} делающей всю работу.

Добавил также немного кода в начале функции \TT{check\_nonce()} для сбора статистики,
с какими аргументами эта функция вызывалась.

Вот что мы видим если запустим:

\begin{lstlisting}
threads=4
...
checked=2800000
checked=3000000
checked=3200000
checked=3300000
found (thread 3): [hello, world!_1611446522]. seconds spent=3
__min[0]=0x00000000 __max[0]=0x1fffffff
__min[1]=0x20000000 __max[1]=0x3fffffff
__min[2]=0x40000000 __max[2]=0x5fffffff
__min[3]=0x60000000 __max[3]=0x7ffffffe
\end{lstlisting}

Да, результат правильный, первые 3 байта это нули:

\begin{lstlisting}
C:\...\sha512sum test
000000f4a8fac5a4ed38794da4c1e39f54279ad5d9bb3c5465cdf57adaf60403
df6e3fe6019f5764fc9975e505a7395fed780fee50eb38dd4c0279cb114672e2 *test
\end{lstlisting}

Оно требует $\approx2..3$ секунды на 4-х ядерном Intel Xeon E3-1220 3.10 GHz.

В task manager мы видим 5 тредов: один главный тред + 4 запущенных.
Никаких оптимизаций не было сделано, чтобы оставить этот пример в как можно более простом виде.
Но, наверное, этот алгоритм может работать быстрее.
У моего \ac{CPU} 4 ядра, вот почему OpenMP запустил именно 4 треда.

Глядя на таблицу статистики, можно легко увидеть, что цикл был разделен очень точно на 4 равных части.
Ну хорошо, почти равных, если не учитывать последний бит.

Имеются также прагмы и для \glslink{atomic operation}{атомарных операций}.

Посмотрим, как вот этот код будет скомпилирован:

\begin{lstlisting}[style=customc]
	#pragma omp atomic
	checked++;

	#pragma omp critical
	if ((checked % 100000)==0)
		printf ("checked=%d\n", checked);
\end{lstlisting}

\lstinputlisting[caption=MSVC 2012,style=customasmx86]{advanced/700_openmp/MSVC_2.asm}

Как выясняется, функция \TT{vcomp\_atomic\_add\_i4()} 
в vcomp*.dll это просто крохотная функция имеющая инструкцию
 \TT{LOCK XADD}\footnote{О префиксе LOCK читайте больше: \myref{x86_lock}}.

\TT{vcomp\_enter\_critsect()} в конце концов вызывает функцию win32 \ac{API} \\
\TT{EnterCriticalSection()}
\footnote{О критических секциях читайте больше тут: \myref{critical_sections}}.

\subsection{GCC}

GCC 4.8.1 выдает программу показывающую точно такую же таблицу со статистикой, 
так что, реализация GCC делит цикл на части точно так же.

\begin{lstlisting}[caption=GCC 4.8.1,style=customasmx86]
	mov	edi, OFFSET FLAT:main._omp_fn.0
	call	GOMP_parallel_start
	mov	edi, 0
	call	main._omp_fn.0
	call	GOMP_parallel_end
\end{lstlisting}

В отличие от реализации MSVC, то, что делает код GCC, это запускает 3 треда, но также запускает 
четвертый прямо в текущем треде. Так что здесь всего 4 треда а не 5 как в случае
с MSVC.

Вот функция \TT{main.\_omp\_fn.0}:
 
\lstinputlisting[caption=GCC 4.8.1,style=customasmx86]{advanced/700_openmp/GCC_1.asm}

Здесь мы видим это деление явно: вызывая 
\TT{omp\_get\_num\_threads()} и \TT{omp\_get\_thread\_num()}
мы получаем количество запущенных тредов, а также номер текущего треда, и затем определяем интервал цикла.
И затем запускаем \TT{check\_nonce()}.

GCC также вставляет инструкцию \TT{LOCK ADD} 
прямо в том месте кода, где MSVC сгенерировал вызов отдельной функции в DLL:

\lstinputlisting[caption=GCC 4.8.1,style=customasmx86]{advanced/700_openmp/GCC_2.asm}

Функции с префиксом GOMP это часть библиотеки 
GNU OpenMP.
В отличие от vcomp*.dll, её исходный код свободно доступен: 
\href{http://go.yurichev.com/17102}{GitHub}.

}\FR{\mysection{OpenMP}
\label{openmp}
\myindex{OpenMP}

OpenMP est l'un des moyens les plus simple de paralléliser des algorithmes simples.

\myindex{Bitcoin}

À titre d'exemple, essayons de construire un programme pour calculer une \emph{nonce}
cryptographique.

Dans mon exemple simpliste, le \emph{nonce} est un nombre ajouté au texte non chiffré
afin de produire un hash avec quelques caractéristiques spécifiques.

Par exemple, à certaines étapes, le protocole Bitcoin nécessite de trouver de tels
\emph{nonce} dont le hash résultant contient un nombre spécifique de zéros consécutifs.
Ceci est aussi appelé \q{preuve de travail}
\footnote{\href{http://go.yurichev.com/17101}{Wikipédia}}
(i.e., le système prouve qu'il a fait des calculs intensifs et y a passé du temps).

\myindex{SHA512}
Mon exemple n'est en aucun cas lié au Bitcoin, il va essayer d'ajouter des nombres
à la chaîne afin de trouver un nombre tel que le hash de \q{hello, world!\_<number>}
avec l'algorithme SHA512, contiendra au moins 3 octets à zéro.

Limitons notre recherche brute-force dans l'intervalle 0..INT32\_MAX-1 (i.e., \TT{0x7FFFFFFE}
ou 2147483646).

L'algorithme est assez direct:

\lstinputlisting[style=customc]{advanced/700_openmp/openmp_example.c}

La fonction \TT{check\_nonce()} ajoute simplement un nombre à la chaîne, hashe le
résultat avec l'algorithme SHA12 et teste si il y a 3 octets à zéro dans le résultat.

Une partie très importante du code est:

\begin{lstlisting}[style=customc]
	#pragma omp parallel for
	for (i=0; i<INT32_MAX; i++)
		check_nonce (i);
\end{lstlisting}

Oui, c'est simple, sans le \TT{\#pragma} nous appelons \TT{check\_nonce()} pour chaque
nombre de 0 à \TT{INT32\_MAX} (\TT{0x7fffffff} ou 2147483647).
Avec le \TT{\#pragma}, le compilateur ajoute du code particulier qui découpe l'intervalle
de la boucle en des plus petits, afin de les lancer sur tous les c\oe{}urs de \ac{CPU}
disponible\footnote{N.B.: Ceci est intentionnellement l'exemple le plus simple possible,
mais en pratique, l'utilisation de OpenMP peut être plus difficile et plus complexe.}.

L'exemple peut être compilé\footnote{Les fichiers sha512.(c|h) et u64.h peuvent être
pris de la bibliothèque OpenSSL:
\url{http://go.yurichev.com/17324}}
dans MSVC 2012:
% FIXME1: \footnote{other source code files can be downloaded here: ...}

\begin{lstlisting}
cl openmp_example.c sha512.obj /openmp /O1 /Zi /Faopenmp_example.asm
\end{lstlisting}

Ou dans GCC:

\begin{lstlisting}
gcc -fopenmp 2.c sha512.c -S -masm=intel
\end{lstlisting}

\subsection{MSVC}

Maintenant voici comment MSVC 2012 génère la boucle principale:

\lstinputlisting[caption=MSVC 2012,style=customasmx86]{advanced/700_openmp/MSVC_loop.asm}

Toutes les fonctions préfixées par \TT{vcomp} sont relatives à OpenMP et sont stockées
dans le fichier vcomp*.dll.
Donc, ici un groupe de threads est démarré.

Regardons \TT{\_main\$omp\$1}:

\lstinputlisting[caption=MSVC 2012,style=customasmx86]{advanced/700_openmp/MSVC_1.asm}

Cette fonction va être démarrée $n$ fois en parallèle, où $n$ est le nombre de c\oe{}urs
du \ac{CPU}.\\
\TT{vcomp\_for\_static\_simple\_init()}
calcul l'intervalle pour la construction for() du thread courant, dépendant du numéro
du thread courant.

Les valeurs de début et de fin sont stockées dans les variables locales \TT{\$T1}
et \TT{\$T2}.
Vous pouvez également remarquer l'argument \TT{7ffffffeh} (ou 2147483646) comme argument
de la fonction\\
\TT{vcomp\_for\_static\_simple\_init()}---ceci est le nombre d'itérations
de la boucle complète,\\
 qui doit éventuellement être divisée.

Puis nous voyons une nouvelle boucle avec un appel à la fonction \TT{check\_nonce()},
qui fait tout le travail.

Ajoutons du code au début de la fonction \TT{check\_nonce()} pour collecter des statistiques
sur les arguments avec lesquels la fonction a été appelée.

Voici ce que nous pouvons voir lorsque nous le lançons:

\begin{lstlisting}
threads=4
...
checked=2800000
checked=3000000
checked=3200000
checked=3300000
found (thread 3): [hello, world!_1611446522]. seconds spent=3
__min[0]=0x00000000 __max[0]=0x1fffffff
__min[1]=0x20000000 __max[1]=0x3fffffff
__min[2]=0x40000000 __max[2]=0x5fffffff
__min[3]=0x60000000 __max[3]=0x7ffffffe
\end{lstlisting}

Oui, le résultat est correct, les 3 premiers octets sont des zéros:

\begin{lstlisting}
C:\...\sha512sum test
000000f4a8fac5a4ed38794da4c1e39f54279ad5d9bb3c5465cdf57adaf60403
df6e3fe6019f5764fc9975e505a7395fed780fee50eb38dd4c0279cb114672e2 *test
\end{lstlisting}

Le temps de traitement est $\approx2..3$ secondes sur un Intel Xeon E3-1220 3.10 GHz 4-core.
Dans le gestionnaire de tâches nous voyons 5 threads:
1 thread principal + 4 autres.
Il n'y a pas d'optimisations faites afin de garder cet exemple aussi petit et clair
que possible.
Mais probablement qu'on pourrait le rendre plus rapide.
Mon \ac{CPU} a 4 c\oe{}urs, c'est pourquoi OpenMP a démarré exactement 4 threads.

En regardant la table des statistiques, nous voyons clairement comment la boucle
a été découpée en 4 parties égales.
Oui bon, presque égales, si nous ne tenons pas compte du dernier bit.

Il y a aussi des pragmas pour les \glslink{atomic operation}{operations atomiques.}.

Voyons comment ce code est compilé:

\begin{lstlisting}[style=customc]
	#pragma omp atomic
	checked++;

	#pragma omp critical
	if ((checked % 100000)==0)
		printf ("checked=%d\n", checked);
\end{lstlisting}

\lstinputlisting[caption=MSVC 2012,style=customasmx86]{advanced/700_openmp/MSVC_2.asm}

Il semble que la fonction \TT{vcomp\_atomic\_add\_i4()} dans vcomp*.dll soit juste
une minuscule fonction avec l'instruction \TT{LOCK XADD}\footnote{En savoir plus sur le préfixe LOCK: \myref{x86_lock}}
dedans.

\TT{vcomp\_enter\_critsect()}
appelle finalement la fonction de l'\ac{API} win32 \\
\TT{EnterCriticalSection()}
\footnote{Vous pouvez en lire plus sur les sections critiques ici: \myref{critical_sections}}.

\subsection{GCC}

GCC 4.8.1 produit un programme qui montre exactement la même table de statistique,

donc, l'implémentation de GCC divise la boucle en parties de la même manière.

\begin{lstlisting}[caption=GCC 4.8.1,style=customasmx86]
	mov	edi, OFFSET FLAT:main._omp_fn.0
	call	GOMP_parallel_start
	mov	edi, 0
	call	main._omp_fn.0
	call	GOMP_parallel_end
\end{lstlisting}

Contrairement à l'implémentation de MSVC, ce que le code de GCC fait est de démarrer
3 threads et lance la quatrième dans le thread courant. Il y a donc 4 threads au
lieu de 5 dans MSVC.

Voici les fonctions \TT{main.\_omp\_fn.0}:

\lstinputlisting[caption=GCC 4.8.1,style=customasmx86]{advanced/700_openmp/GCC_1.asm}

Ici nous voyons la division clairement: en appelant
\TT{omp\_get\_num\_threads()} et \TT{omp\_get\_thread\_num()}

nous obtenons le nombre de threads lancés, le nombre courant de threads,
et ainsi détermine l'intervalle de la boucle.
Ensuite nous lançons \TT{check\_nonce()}.

GCC a également inséré l'instruction \TT{LOCK ADD} directement dans le code, contrairement
à MSVC, qui génère un appel à une fonction DLL séparée:

\lstinputlisting[caption=GCC 4.8.1,style=customasmx86]{advanced/700_openmp/GCC_2.asm}

Les fonctions préfixées par GOMP sont de la bibliothèque GNU OpenMP.
Contrairement à vcomp*.dll, son code source est librement disponible:
\href{http://go.yurichev.com/17102}{GitHub}.
}

\EN{\mysection{Signed division using shifts}

Unsigned division by $2^n$ numbers is easy, just use bit shift right by $n$.
Signed division by $2^n$ is easy as well, but some correction needs to be done before or after shift opeartion.

\myindex{x86!\Instructions!SHR}
\myindex{x86!\Instructions!SAR}
First, most CPU architectures support two right shift operations: logical and arithmetical.
During logical shift right, free bit(s) at left are set to zero bit(s).
This is \INS{SHR} in x86.
During arithmetical shift right, free bit(s) at left are set equal to the bit which was at the same place.
Thus, it preserves sign bit while shifting.
This is \INS{SAR} in x86.

\myindex{x86!\Instructions!SAL}
\myindex{x86!\Instructions!SHL}
Interesting to know, there is no special instruction for arithmetical shift left, because it works just as logical shift left.
So, \INS{SAL} and \INS{SHL} instructions in x86 are mapped to the same opcode.
Many disassemblers don't even know about SAL instruction and decode this opcode as \INS{SHL}.

Hence, arithmetical shift right is used for signed numbers.
For example, if you shift -4 (11111100b) by 1 bit right, logical shift right operation will produce 01111110b, which is 126.
Arithmetical shift right will produce 11111110b, which is -2.
So far so good.

What if we need to divide -5 by 2? This is -2.5, or just -2 in integer arithmetic.
-5 is 11111011b, by shifting this value by 1 bit right, we'll get 11111101b, which is -3.
This is slightly incorrect.

Another example: $-\frac{1}{2}=-0.5$ or just 0 in integer arithmetic.
But -1 is 11111111b, and 11111111b >> 1 = 11111111b, which is -1 again.
This is also incorrect.

One solution is to add 1 to the input value if it's negative.

That is why, if we compile \TT{x/2} expression, where x is \textit{signed int}, GCC 4.8 will produce something like that:

\begin{lstlisting}[style=customasmx86]
	mov	eax, edi
	shr	eax, 31  ; isolate leftmost bit, which is 1 if the number is negative and 0 if positive
	add	eax, edi ; add 1 to the input value if it's negative, do nothing otherwise
	sar	eax      ; arithmetical shift right by one bit
	ret
\end{lstlisting}

If you divide by 4, 3 needs to be added to the input value if it's negative. So this is what GCC 4.8 does for \TT{x/4}:

\begin{lstlisting}[style=customasmx86]
	lea	eax, [rdi+3] ; prepare x+3 value ahead of time
	test	edi, edi
	
	; if the sign is not negative (i.e., positive), move input value to EAX
	; if the sign is negative, x+3 value is left in EAX untouched
	cmovns	eax, edi
	; do arithmetical shift right by 2 bits
	sar	eax, 2
	ret
\end{lstlisting}

If you divide by 8, 7 will be added to the input value, etc.

MSVC 2013 is slightly different. This is division by 2:

\begin{lstlisting}[style=customasmx86]
	mov	eax, DWORD PTR _a$[esp-4]
	; sign-extend input value to 64-bit value into EDX:EAX
	; effectively, that means EDX will be set to 0FFFFFFFFh if the input value is negative
	; ... or to 0 if positive
	cdq
	; subtract -1 from input value if it's negative
	; this is the same as adding 1
	sub	eax, edx 
	; do arithmetical shift right
	sar	eax, 1
	ret	0
\end{lstlisting}

Division by 4 in MSVC 2013 is little more complex:

\begin{lstlisting}[style=customasmx86]
	mov	eax, DWORD PTR _a$[esp-4]
	cdq
	; now EDX is 0FFFFFFFFh if input value is negative
	; EDX is 0 if it's positive
	and	edx, 3
	; now EDX is 3 if input is negative or 0 otherwise
	; add 3 to input value if it's negative or do nothing otherwise:
	add	eax, edx
	; do arithmetical shift
	sar	eax, 2
	ret	0
\end{lstlisting}

Division by 8 in MSVC 2013 is similar, but 3 bits from \EDX is taken instead of 2, producing correction value of 7 instead of 3.

\myindex{Hex-Rays}
Sometimes, Hex-Rays 6.8 can't handle such code correctly, and it may produce something like this:

\begin{lstlisting}
int v0;
...
__int64 v14
...
          
v14 = ...;
v0 = ((signed int)v14 - HIDWORD(v14)) >> 1;
\end{lstlisting}

... it can be safely rewritten to \TT{v0=v14/2}.

Hex-Rays 6.8 can also handle signed division by 4 like that:

\begin{lstlisting}
result = ((BYTE4(v25) & 3) + (signed int)v25) >> 2;
\end{lstlisting}

... can be rewritten to \TT{v25 / 4}.

\myhrule{}

Also, such correction code is used often when division is replaced by multiplication by \textit{magic numbers}:
read \MathForProg about multiplicative inverse.
And sometimes, additional shifting is used after multiplication.
For example, when GCC optimizes $\frac{x}{10}$, it can't find multiplicative inverse for 10, because diophantine equation has no solutions.
So it generates code for $\frac{x}{5}$ and then adds arithmetical shift right operation by 1 bit, to divide the result by 2.
Of course, this is true only for signed integers.

So here is division by 10 by GCC 4.8:

\begin{lstlisting}[style=customasmx86]
	mov	eax, edi
	mov	edx, 1717986919 ; magic number
	sar	edi, 31         ; isolate leftmost bit (which reflects sign)
	imul	edx             ; multiplication by magic number (calculate x/5)
	sar	edx, 2          ; now calculate (x/5)/2

        ; subtract -1 (or add 1) if the input value is negative.
	; do nothing otherwise:
	sub	edx, edi        
	mov	eax, edx
	ret
\end{lstlisting}

Summary: $2^n-1$ must be added to input value before arithmetical shift, or 1 must be added to the final result after shift.
Both operations are equivalent to each other, so compiler developers may choose what is more suitable to them.
From the reverse engineer's point of view, this correction is a clear evidence that the value has signed type.

}%
\FR{\mysection{Division signée en utilisant des décalages}

La division non signée par des nombres $2^n$ est facile, il sufft d'utiliser le
décalage de $n$ bit à droite.
La division signée par $2^n$ est aussi facile, mais des corrections doivent être
faites avant ou après l'opération de décalage.

\myindex{x86!\Instructions!SHR}
\myindex{x86!\Instructions!SAR}
D'abord, la plupart des architectures CPU supportent deux opérations de décalage
à droite: logique et arithmétique.
Lors d'un décalage logique à droite, le bit libre est mis à zéro.
C'est \INS{SHR} en x86.
Lors d'un décalage arithmétique à droite, le bit à gauche est mis avec celui qui
était à cette position avant le décalage.
Ainsi, le signe est conservé lors du décalage.
C'est \INS{SAR} en x86.

\myindex{x86!\Instructions!SAL}
\myindex{x86!\Instructions!SHL}
Il est intéressant de noter qu'il n'y a pas d'instruction spéciale pour le décalage
arithmétique à gauche, car il fonctionne tout simplement comme le décalage logique.
Donc, les instructions \INS{SAL} et \INS{SHL} en x86 sont mappées sur le même opcode.
De nombreux désassembleurs ne connaissent même pas l'instruction \INS{SAL} et la
décode comme \INS{SHL}.

De ce fait, le décalages arithmétique à droite est utilisé pour les nombres signés.
Par exemple, si vous décalez -4 (11111100b) d'1 bit à droite, l'opération de décalage
logique produira 01111110b, qui est 126.
Le décalage arithmétique à droite produira 11111110b, qui est -2.
Jusqu'ici, tout va bien.

Et si nous devons diviser -5 par 2? Ça vaut 2,5 ou juste -2 en arithmétique entière.
-5 est 11111011b, en décalant cette valeur de 1 bit à droite, nous obtenons 11111101b,
qui est -3.
Ceci est légèrement incorrect.

Un autre exemple: $-\frac{1}{2}=-0.5$ ou 0 en arithmétique entière.
Mais -1 est 11111111b, et 11111111b >> 1 = 11111111b, qui est encore -1.
Ceci est aussi incorrect.

Une solution est d'ajouter 1 à la valeur en entrée si elle est négative.

C'est pourquoi, si nous compilons l'expression \TT{x/2}, où x est un \textit{signed int},
GCC 4.8 produira quelque chose comme cela:

\begin{lstlisting}[style=customasmx86]
	mov	eax, edi
	shr	eax, 31  ; isoler le bit le plus à gauche, qui est 1 si la valeur est négative et 0 si elle est positive
	add	eax, edi ; ajouter 1 à la valeur en entrée si elle est négative, ne rien faire autrement
	sar	eax      ; décalage arithmétique à droite de un bit
	ret
\end{lstlisting}

Si vous divisez par 4, il faut ajouter 3 à la valeur en entrée si elle est négative.
Donc ceci est ce que GCC 4.8 génère pour \TT{x/4}:

\begin{lstlisting}[style=customasmx86]
	lea	eax, [rdi+3] ; préparer la valeur x+3 en avance
	test	edi, edi
	
	; si le signe n'est pas négatif (i.e., positif), déplacer la valeur en entrée dans EAX
	; si le signe est négatif, la valeur x+3 est laissée telle quelle dans EAX
	cmovns	eax, edi
	; effectuer un décalage arithmétique à droite de 2 bits
	sar	eax, 2
	ret
\end{lstlisting}

Si vous divisez par 8, il faut ajouter 7 à la valeur en entrée, etc.

MSVC 2013 est légèrement différent. Ceci est la division par 2:

\begin{lstlisting}[style=customasmx86]
	mov	eax, DWORD PTR _a$[esp-4]
	; étendre la valeur d'entrée à 64-bit dans EDX:EAX
	; concrètement, cela signifie que EDX sera mis à 0FFFFFFFFh si la valeur en entrée est négative
	; ... ou à 0 si elle est positive
	cdq
	; soustraire -1 de la valeur en entrée si elle est négative
	; ceci est la même chose qu'ajouter 1
	sub	eax, edx 
	; effectuer le décalage arithmétique à droite
	sar	eax, 1
	ret	0
\end{lstlisting}

La division par 4 dans MSVC 2013 est encore plus complexe:

\begin{lstlisting}[style=customasmx86]
	mov	eax, DWORD PTR _a$[esp-4]
	cdq
	; maintenant EDX contient 0FFFFFFFFh si la valeur en entrée est négative
	; EDX contient 0 si elle est positive
	and	edx, 3
	; maintenant EDX contient 3 si la valeur en entrée est négative et 0 autrement
	; ajouter 3 à la valeur en entrée si elle est négative ou ne rien faire autrement:
	add	eax, edx
	; effectuer le décalage arithmétique à droite
	sar	eax, 2
	ret	0
\end{lstlisting}

La division par 8 dans MSVC 2013 est similaire, mais 3 bits de \EDX sont pris au
lieu de 2, produisant une correction de 7 au lieu de 3.

\myindex{Hex-Rays}
Parfois, Hex-Rays 6.8 ne gère pas correctement un tel code, et il peut produire
quelque chose comme ceci:

\begin{lstlisting}
int v0;
...
__int64 v14
...
          
v14 = ...;
v0 = ((signed int)v14 - HIDWORD(v14)) >> 1;
\end{lstlisting}

\dots ce qui peut sans risque être récrit en \TT{v0=v14/2}.

Hex-Rays 6.8 peut aussi gérer les divisions signées par 4 comme cela:

\begin{lstlisting}
result = ((BYTE4(v25) & 3) + (signed int)v25) >> 2;
\end{lstlisting}

\dots peut être récrit en \TT{v25 / 4}.

\myhrule{}

En outre, une telle correction est souvent utilisé lorsque la division est remplacée
par la multiplication par des \textit{nombres magiques}:
lire \MathForProg à propos de la multiplication inverse.
Et parfois, un décalage additionnel est utilisé après la multiplication.
Par exemple, lorsque GCC optimise $\frac{x}{10}$, il ne peut pas trouver la multiplication
inverse pour 10, car l'équation diophantienne n'a pas de solution.
Donc il génère du code pour $\frac{x}{5}$ et puis ajoute une opération de décalage
arithmétique à droite de 1 bit, pour diviser le résultat par 2.
Bien sûr, ceci est seulement vrai pour les entiers signés.

Donc, voici la division par 10 de GCC 4.8:

\begin{lstlisting}[style=customasmx86]
	mov	eax, edi
	mov	edx, 1717986919 ; nombre magique
	sar	edi, 31         ; isoler le bit le plus à gauche (qui reflète le signe)
	imul	edx         ; multiplication par le nombre magique (calculer x/5)
	sar	edx, 2          ; mainenant calculer (x/5)/2

    ; soustraire -1 (ou ajouter 1) si la valeur en entrée est négative
	; ne rien faire autrement
	sub	edx, edi        
	mov	eax, edx
	ret
\end{lstlisting}

Résumé: $2^n-1$ doit être ajouté à la valeur en entrée avant un décalage arithmétique,
ou il faut ajouter 1 au résultat final après le décalage.
Les deux opérations sont équivalentes l'une à l'autre, donc les développeurs du
compilateur doivent choisir ce qui est la plus adapté pour eux.
Du point de vue du rétro-ingénieur, cette correction est une preuve manifeste que
la valeur a un type signé.

}

\EN{% TODO translate
\mysection{Breaking simple executable cryptor}

I've got an executable file which is encrypted by relatively simple encryption.
\href{\GitHubBlobMasterURL/examples/simple_exec_crypto/files/cipher.bin}{Here is it} (only executable section is left here).

First, all encryption function does is just adds number of position in buffer to the byte.
Here is how this can be encoded in Python:

\begin{lstlisting}[caption=Python script,style=custompy]
#!/usr/bin/env python
def e(i, k):
    return chr ((ord(i)+k) % 256)

def encrypt(buf):
    return e(buf[0], 0)+ e(buf[1], 1)+ e(buf[2], 2) + e(buf[3], 3)+ e(buf[4], 4)+ e(buf[5], 5)+ e(buf[6], 6)+ e(buf[7], 7)+
           e(buf[8], 8)+ e(buf[9], 9)+ e(buf[10], 10)+ e(buf[11], 11)+ e(buf[12], 12)+ e(buf[13], 13)+ e(buf[14], 14)+ e(buf[15], 15)
\end{lstlisting}

Hence, if you encrypt buffer with 16 zeros, you'll get \emph{0, 1, 2, 3 ... 12, 13, 14, 15}.

\myindex{Propagating Cipher Block Chaining}
Propagating Cipher Block Chaining (PCBC) is also used, here is how it works:

\begin{figure}[H]
\centering
\myincludegraphics{examples/simple_exec_crypto/601px-PCBC_encryption.png}
\caption{Propagating Cipher Block Chaining encryption (image is taken from Wikipedia article)}
\end{figure}

The problem is that it's too boring to recover IV (Initialization Vector) each time.
Brute-force is also not an option, because IV is too long (16 bytes).
Let's see, if it's possible to recover IV for arbitrary encrypted executable file?

Let's try simple frequency analysis.
This is 32-bit x86 executable code, so let's gather statistics about most frequent bytes and opcodes.
I tried huge oracle.exe file from Oracle RDBMS version 11.2 for windows x86 and I've found that the most frequent byte (no surprise) is zero (~10\%).
The next most frequent byte is (again, no surprise) 0xFF (~5\%).
The next is 0x8B (~5\%).

\myindex{x86!\Instructions!MOV}
0x8B is opcode for \INS{MOV}, this is indeed one of the most busy x86 instructions.
Now what about popularity of zero byte?
If compiler needs to encode value bigger than 127, it has to use 32-bit displacement instead of 8-bit one, but large values are very rare,
so it is padded by zeros.
\myindex{x86!\Instructions!LEA}
\myindex{x86!\Instructions!PUSH}
\myindex{x86!\Instructions!CALL}
This is at least in \INS{LEA}, \INS{MOV}, \INS{PUSH}, \INS{CALL}.

For example:

\begin{lstlisting}[style=customasmx86]
8D B0 28 01 00 00                 lea     esi, [eax+128h]
8D BF 40 38 00 00                 lea     edi, [edi+3840h]
\end{lstlisting}

Displacements bigger than 127 are very popular, but they are rarely exceeds 0x10000
(indeed, such large memory buffers/structures are also rare).

Same story with \INS{MOV}, large constants are rare, the most heavily used are 0, 1, 10, 100, $2^n$, and so on.
Compiler has to pad small constants by zeros to represent them as 32-bit values:

\begin{lstlisting}[style=customasmx86]
BF 02 00 00 00                    mov     edi, 2
BF 01 00 00 00                    mov     edi, 1
\end{lstlisting}

Now about 00 and FF bytes combined: jumps (including conditional) and calls can pass execution flow forward or backwards, but very often,
within the limits of the current executable module.
If forward, displacement is not very big and also padded with zeros.
If backwards, displacement is represented as negative value, so padded with FF bytes.
For example, transfer execution flow forward:

\begin{lstlisting}[style=customasmx86]
E8 43 0C 00 00                    call    _function1
E8 5C 00 00 00                    call    _function2
0F 84 F0 0A 00 00                 jz      loc_4F09A0
0F 84 EB 00 00 00                 jz      loc_4EFBB8
\end{lstlisting}

Backwards:

\begin{lstlisting}[style=customasmx86]
E8 79 0C FE FF                    call    _function1
E8 F4 16 FF FF                    call    _function2
0F 84 F8 FB FF FF                 jz      loc_8212BC
0F 84 06 FD FF FF                 jz      loc_FF1E7D
\end{lstlisting}

FF byte is also very often occurred in negative displacements like these:

\begin{lstlisting}[style=customasmx86]
8D 85 1E FF FF FF                 lea     eax, [ebp-0E2h]
8D 95 F8 5C FF FF                 lea     edx, [ebp-0A308h]
\end{lstlisting}

So far so good. Now we have to try various 16-byte keys, decrypt executable section and measure how often 00, FF and 8B bytes are occurred.
Let's also keep in sight how PCBC decryption works:

\begin{figure}[H]
\centering
\myincludegraphics{examples/simple_exec_crypto/640px-PCBC_decryption.png}
\caption{Propagating Cipher Block Chaining decryption (image is taken from Wikipedia article)}
\end{figure}

The good news is that we don't really have to decrypt whole piece of data, but only slice by slice, this is exactly how I did in my previous example: \myref{XOR_mask_2}.

Now I'm trying all possible bytes (0..255) for each byte in key and just pick the byte producing maximal amount of 00/FF/8B bytes in a decrypted slice:

\begin{lstlisting}[style=custompy]
#!/usr/bin/env python
import sys, hexdump, array, string, operator

KEY_LEN=16

def chunks(l, n):
    # split n by l-byte chunks
    # https://stackoverflow.com/q/312443
    n = max(1, n)
    return [l[i:i + n] for i in range(0, len(l), n)]

def read_file(fname):
    file=open(fname, mode='rb')
    content=file.read()
    file.close()
    return content

def decrypt_byte (c, key):
    return chr((ord(c)-key) % 256)

def XOR_PCBC_step (IV, buf, k):
    prev=IV
    rt=""
    for c in buf:
	new_c=decrypt_byte(c, k)
        plain=chr(ord(new_c)^ord(prev))
	prev=chr(ord(c)^ord(plain))
	rt=rt+plain
    return rt

each_Nth_byte=[""]*KEY_LEN

content=read_file(sys.argv[1])
# split input by 16-byte chunks:
all_chunks=chunks(content, KEY_LEN)
for c in all_chunks:
    for i in range(KEY_LEN):
        each_Nth_byte[i]=each_Nth_byte[i] + c[i]

# try each byte of key
for N in range(KEY_LEN):
    print "N=", N
    stat={}
    for i in range(256):
        tmp_key=chr(i)
	tmp=XOR_PCBC_step(tmp_key,each_Nth_byte[N], N)
        # count 0, FFs and 8Bs in decrypted buffer:
	important_bytes=tmp.count('\x00')+tmp.count('\xFF')+tmp.count('\x8B')
	stat[i]=important_bytes
    sorted_stat = sorted(stat.iteritems(), key=operator.itemgetter(1), reverse=True)
    print sorted_stat[0]
\end{lstlisting}

(Source code can be downloaded \href{\GitHubBlobMasterURL/examples/simple_exec_crypto/files/decrypt.py}{here}.)

I run it and here is a key for which 00/FF/8B bytes presence in decrypted buffer is maximal:

\begin{lstlisting}
N= 0
(147, 1224)
N= 1
(94, 1327)
N= 2
(252, 1223)
N= 3
(218, 1266)
N= 4
(38, 1209)
N= 5
(192, 1378)
N= 6
(199, 1204)
N= 7
(213, 1332)
N= 8
(225, 1251)
N= 9
(112, 1223)
N= 10
(143, 1177)
N= 11
(108, 1286)
N= 12
(10, 1164)
N= 13
(3, 1271)
N= 14
(128, 1253)
N= 15
(232, 1330)
\end{lstlisting}

Let's write decryption utility with the key we got:

\begin{lstlisting}[style=custompy]
#!/usr/bin/env python
import sys, hexdump, array

def xor_strings(s,t):
    # \verb|https://en.wikipedia.org/wiki/XOR_cipher#Example_implementation|
    """xor two strings together"""
    return "".join(chr(ord(a)^ord(b)) for a,b in zip(s,t))

IV=array.array('B', [147, 94, 252, 218, 38, 192, 199, 213, 225, 112, 143, 108, 10, 3, 128, 232]).tostring()

def chunks(l, n):
    n = max(1, n)
    return [l[i:i + n] for i in range(0, len(l), n)]

def read_file(fname):
    file=open(fname, mode='rb')
    content=file.read()
    file.close()
    return content

def decrypt_byte(i, k):
    return chr ((ord(i)-k) % 256)

def decrypt(buf):
    return "".join(decrypt_byte(buf[i], i) for i in range(16))

fout=open(sys.argv[2], mode='wb')

prev=IV
content=read_file(sys.argv[1])
tmp=chunks(content, 16)
for c in tmp:
    new_c=decrypt(c)
    p=xor_strings (new_c, prev)
    prev=xor_strings(c, p)
    fout.write(p)
fout.close()
\end{lstlisting}

(Source code can be downloaded \href{\GitHubBlobMasterURL/examples/simple_exec_crypto/files/decrypt2.py}{here}.)

Let's check resulting file:

\lstinputlisting{examples/simple_exec_crypto/objdump_result.txt}

Yes, this is seems correctly disassembled piece of x86 code.
The whole decryped file can be downloaded \href{\GitHubBlobMasterURL/examples/simple_exec_crypto/files/decrypted.bin}{here}.

In fact, this is text section from regedit.exe from Windows 7.
But this example is based on a real case I encountered, so just executable is different (and key), algorithm is the same.

\subsection{Other ideas to consider}

What if I would fail with such simple frequency analysis?
There are other ideas on how to measure correctness of decrypted/decompressed x86 code:

\begin{itemize}

\item Many modern compilers aligns functions on 0x10 border.
So the space left before is filled with NOPs (0x90) or other NOP instructions with known opcodes: \myref{sec:npad}.

\item Perhaps, the most frequent pattern in any assembly language is function call:\\
\TT{PUSH chain / CALL / ADD ESP, X}.
This sequence can easily detected and found.
I've even gathered statistics about average number of function arguments: \myref{args_stat}.
(Hence, this is average length of PUSH chain.)

\end{itemize}

Read more about incorrectly/correctly disassembled code: \myref{ISA_detect}.
}
\FR{\mysection{Fonction presque vide}
\label{Boolector}
\myindex{Boolector}
\myindex{x86!\Instructions!JMP}

Ceci est un morceau de code réel que j'ai trouvé dans Boolector\footnote{\url{https://boolector.github.io/}}:

\lstinputlisting[style=customc]{patterns/025_almost_empty/boolectormain.c}

Pourquoi quelqu'un ferait-il comme ça?
Je ne sais pas mais mon hypothèse est que \verb|boolector_main()| peut être compilée
dans une sorte de DLL ou bibliothèque dynamique, et appelée depuis une suite de test.
Certainement qu'une suite de test peut préparer les variables argc/argv comme
le ferait \ac{CRT}.

Il est intéressant de voir comment c'est compilé:

\lstinputlisting[caption=GCC 8.2 x64 \NonOptimizing (\assemblyOutput),style=customasmx86]{patterns/025_almost_empty/boolectormain_O0.s}

Ceci est OK, le prologue (non optimisé) déplace inutilement deux arguments,
\INS{CALL}, épilogue, \INS{RET}.
Mais regardons la version optimisée:

\lstinputlisting[caption=GCC 8.2 x64 \Optimizing (\assemblyOutput),style=customasmx86]{patterns/025_almost_empty/boolectormain_O3.s}

Aussi simple que ça: la pile et les registres ne sont pas touchés et \verb|boolector_main()|
a le même ensemble d'arguments.
Donc, tout ce que nous avons à faire est de passer l'exécution à une autre adresse.

Ceci est proche d'une \glslink{thunk function}{fonction thunk}.

Nous verons queelque chose de plus avancé plus tard: \myref{ARM_B_to_printf}, \myref{JMP_instead_of_RET}.
}
\RU{\subsection{Простое шифрование используя XOR-маску}
\label{XOR_mask_1}

Я нашел одну старую игру в стиле interactive fiction в архиве \emph{if-archive}\footnote{\url{http://www.ifarchive.org/}}:

\begin{lstlisting}
The New Castle v3.5 - Text/Adventure Game
in the style of the original Infocom (tm)
type games, Zork, Collosal Cave (Adventure),
etc.  Can you solve the mystery of the
abandoned castle?
Shareware from Software Customization.
Software Customization [ASP] Version 3.5 Feb. 2000
\end{lstlisting}

Можно скачать здесь: \url{\GitHubBlobMasterURL/ff/XOR/mask_1/files/newcastle.tgz}.

Там внутри есть файл (с названием \emph{castle.dbf}), который явно зашифрован, но не настоящим криптоалгоритмом,
и оне сжат, это что-то куда проще.
Я бы даже не стал измерять уровень энтропии (\myref{entropy}) этого файла, потому что я итак уверен, что он низкий.
Вот как он выглядит в Midnight Commander:

\begin{figure}[H]
\centering
\myincludegraphics{ff/XOR/mask_1/mc_encrypted.png}
\caption{Зашифрованный файл в Midnight Commander}
\end{figure}

Зашифрованный файл можно скачать здесь:
\url{\GitHubBlobMasterURL/ff/XOR/mask_1/files/castle.dbf.bz2}.

Можно ли расшифровать его без доступа к программе, используя просто этот файл?

Тут явно просматривается повторяющаяся строка. 
Если использовалось простое шифрование с XOR-маской, такие повторяющиеся строки это явное свидетельство,
потому что, вероятно, тут были длинные лакуны с нулевыми байтами, которые, в свою очередь, присутствуют
во мноигих исполняемых файлах, и в остальных бинарных файлах.

\myindex{UNIX!xxd}
Вот дам начала этого файла используя утилиту \emph{xxd} из UNIX:

\lstinputlisting{ff/XOR/mask_1/xxd_result.txt}

Давайте держаться за повторяющуюся строку \TT{iubgv}.
Глядя на этот дамп, мы можем легко увидеть, что период повторений этой строки это 0x51 или 81.
Вероятно, 81 это длина блока?
Длина файла 1658961, и она может быть поделена на 81 без остатка (и тогда там 20481 блоков).

Теперь я буду использовать Mathematica для анализа, есть ли тут повторяющиеся 81-байтные блоки в файле?
Я разделю входной файл на 81-байтные блоки и затем использую ф-цию
\emph{Tally[]}\footnote{\url{https://reference.wolfram.com/language/ref/Tally.html}}
которая просто считает, сколько раз каждый элемент встретился во входном списке.
Вывод Tally не отсортирован, так что я также добавлю ф-цию \emph{Sort[]} для сортировки его по кол-ву вхождений
в нисходящем порядке.

\begin{lstlisting}[style=custommath]
input = BinaryReadList["/home/dennis/.../castle.dbf"];

blocks = Partition[input, 81];

stat = Sort[Tally[blocks], #1[[2]] > #2[[2]] &]
\end{lstlisting}

И вот вывод:

\begin{lstlisting}[style=custommath]
{{{80, 103, 2, 116, 113, 102, 118, 25, 99, 8, 19, 23, 116, 125, 107, 
   25, 99, 109, 114, 102, 14, 121, 115, 31, 9, 117, 113, 111, 5, 4, 
   127, 28, 122, 101, 8, 110, 14, 18, 124, 106, 16, 20, 104, 119, 8, 
   109, 26, 106, 9, 97, 13, 99, 15, 119, 20, 105, 117, 98, 103, 118, 
   1, 126, 29, 97, 122, 17, 15, 114, 110, 3, 5, 125, 125, 99, 126, 
   119, 102, 30, 122, 2, 117}, 1739}, 
{{80, 100, 2, 116, 113, 102, 118, 25, 99, 8, 19, 23, 116, 
   125, 107, 25, 99, 109, 114, 102, 14, 121, 115, 31, 9, 117, 113, 
   111, 5, 4, 127, 28, 122, 101, 8, 110, 14, 18, 124, 106, 16, 20, 
   104, 119, 8, 109, 26, 106, 9, 97, 13, 99, 15, 119, 20, 105, 117, 
   98, 103, 118, 1, 126, 29, 97, 122, 17, 15, 114, 110, 3, 5, 125, 
   125, 99, 126, 119, 102, 30, 122, 2, 117}, 1422}, 
{{80, 101, 2, 116, 113, 102, 118, 25, 99, 8, 19, 23, 116, 
   125, 107, 25, 99, 109, 114, 102, 14, 121, 115, 31, 9, 117, 113, 
   111, 5, 4, 127, 28, 122, 101, 8, 110, 14, 18, 124, 106, 16, 20, 
   104, 119, 8, 109, 26, 106, 9, 97, 13, 99, 15, 119, 20, 105, 117, 
   98, 103, 118, 1, 126, 29, 97, 122, 17, 15, 114, 110, 3, 5, 125, 
   125, 99, 126, 119, 102, 30, 122, 2, 117}, 1012},
{{80, 120, 2, 116, 113, 102, 118, 25, 99, 8, 19, 23, 116, 
   125, 107, 25, 99, 109, 114, 102, 14, 121, 115, 31, 9, 117, 113, 
   111, 5, 4, 127, 28, 122, 101, 8, 110, 14, 18, 124, 106, 16, 20, 
   104, 119, 8, 109, 26, 106, 9, 97, 13, 99, 15, 119, 20, 105, 117, 
   98, 103, 118, 1, 126, 29, 97, 122, 17, 15, 114, 110, 3, 5, 125, 
   125, 99, 126, 119, 102, 30, 122, 2, 117}, 377},

...

{{80, 2, 74, 49, 113, 21, 62, 88, 39, 71, 68, 23, 63, 51, 36, 78, 48, 
   108, 114, 102, 14, 121, 115, 31, 9, 117, 113, 111, 5, 4, 127, 28, 
   122, 101, 8, 110, 14, 18, 124, 106, 16, 20, 104, 119, 8, 109, 26, 
   106, 9, 97, 13, 99, 15, 119, 20, 105, 117, 98, 103, 118, 1, 126, 
   29, 97, 122, 17, 15, 114, 110, 3, 5, 125, 125, 99, 126, 119, 102, 
   30, 122, 2, 117}, 1},
{{80, 1, 74, 59, 113, 45, 56, 86, 52, 91, 19, 64, 60, 60, 63, 
   25, 38, 59, 59, 42, 14, 53, 38, 77, 66, 38, 113, 38, 75, 4, 43, 84,
    63, 101, 64, 43, 79, 64, 40, 57, 16, 91, 46, 119, 69, 40, 84, 117,
    9, 97, 13, 99, 15, 119, 20, 105, 117, 98, 103, 118, 1, 126, 29, 
   97, 122, 17, 15, 114, 110, 3, 5, 125, 125, 99, 126, 119, 102, 30, 
   122, 2, 117}, 1},
{{80, 2, 74, 49, 113, 49, 51, 92, 39, 8, 92, 81, 116, 62, 57, 
   80, 46, 40, 114, 36, 75, 56, 33, 76, 9, 55, 56, 59, 81, 65, 45, 28,
    60, 55, 93, 39, 90, 28, 124, 106, 16, 20, 104, 119, 8, 109, 26, 
   106, 9, 97, 13, 99, 15, 119, 20, 105, 117, 98, 103, 118, 1, 126, 
   29, 97, 122, 17, 15, 114, 110, 3, 5, 125, 125, 99, 126, 119, 102, 
   30, 122, 2, 117}, 1}}
\end{lstlisting}

Вывод Tally это список пар, каждая пара это 81-байтный блок и количество раз, сколько он встретился в файле.
Мы видим, что наиболее частно встречающийся блок это первый, он встретился 1739 раз.
Второй встретился 1422 раза. Есть и другие: 1012 раза, 377 раз, итд.
81-байтные блоки, встреченные лишь один раз, находятся в конце вывода.

Попробуем сравнить эти блоки. Первый и второй.
Есть ли в Mathematica ф-ция для сравнения списков/массивов?
Наверняка есть, но в педагогических целях, я буду использоват операцию XOR для сравнения.
Действительно: если байты во входных массивах равны друг другу, результат операции XOR это 0.
Если не равны, результат будет ненулевой.

Сравним первый блок (встречается 1739 раз) и второй (встречается 1422 раз):

\begin{lstlisting}[style=custommath]
In[]:= BitXor[stat[[1]][[1]], stat[[2]][[1]]]
Out[]= {0, 3, 0, 0, 0, 0, 0, 0, 0, 0, 0, 0, 0, 0, 0, 0, 0, 0, 0, \
0, 0, 0, 0, 0, 0, 0, 0, 0, 0, 0, 0, 0, 0, 0, 0, 0, 0, 0, 0, 0, 0, 0, \
0, 0, 0, 0, 0, 0, 0, 0, 0, 0, 0, 0, 0, 0, 0, 0, 0, 0, 0, 0, 0, 0, 0, \
0, 0, 0, 0, 0, 0, 0, 0, 0, 0, 0, 0, 0, 0, 0, 0}
\end{lstlisting}

Они отличаются только вторым байтом.

Сравним второй блок (встречается 1422 раза) и третий (встречается 1012 раз):

\begin{lstlisting}[style=custommath]
In[]:= BitXor[stat[[2]][[1]], stat[[3]][[1]]]
Out[]= {0, 1, 0, 0, 0, 0, 0, 0, 0, 0, 0, 0, 0, 0, 0, 0, 0, 0, 0, \
0, 0, 0, 0, 0, 0, 0, 0, 0, 0, 0, 0, 0, 0, 0, 0, 0, 0, 0, 0, 0, 0, 0, \
0, 0, 0, 0, 0, 0, 0, 0, 0, 0, 0, 0, 0, 0, 0, 0, 0, 0, 0, 0, 0, 0, 0, \
0, 0, 0, 0, 0, 0, 0, 0, 0, 0, 0, 0, 0, 0, 0, 0}
\end{lstlisting}

Они тоже отличаются только вторым байтом.

Так или иначе, попробуем использовать самый встречающийся блок как XOR-ключ и попробуем расшифровать первые 4 81-байтных
блока в файле:

\begin{lstlisting}[style=custommath]
In[]:= key = stat[[1]][[1]]
Out[]= {80, 103, 2, 116, 113, 102, 118, 25, 99, 8, 19, 23, 116, \
125, 107, 25, 99, 109, 114, 102, 14, 121, 115, 31, 9, 117, 113, 111, \
5, 4, 127, 28, 122, 101, 8, 110, 14, 18, 124, 106, 16, 20, 104, 119, \
8, 109, 26, 106, 9, 97, 13, 99, 15, 119, 20, 105, 117, 98, 103, 118, \
1, 126, 29, 97, 122, 17, 15, 114, 110, 3, 5, 125, 125, 99, 126, 119, \
102, 30, 122, 2, 117}

In[]:= ToASCII[val_] := If[val == 0, " ", FromCharacterCode[val, "PrintableASCII"]]

In[]:= DecryptBlockASCII[blk_] := Map[ToASCII[#] &, BitXor[key, blk]]

In[]:= DecryptBlockASCII[blocks[[1]]]
Out[]= {" ", " ", " ", " ", " ", " ", " ", " ", " ", " ", " ", " \
", " ", " ", " ", " ", " ", " ", " ", " ", " ", " ", " ", " ", " ", " \
", " ", " ", " ", " ", " ", " ", " ", " ", " ", " ", " ", " ", " ", " \
", " ", " ", " ", " ", " ", " ", " ", " ", " ", " ", " ", " ", " ", " \
", " ", " ", " ", " ", " ", " ", " ", " ", " ", " ", " ", " ", " ", " \
", " ", " ", " ", " ", " ", " ", " ", " ", " ", " ", " ", " ", " "}

In[]:= DecryptBlockASCII[blocks[[2]]]
Out[]= {" ", "e", "H", "E", " ", "W", "E", "E", "D", " ", "O", \
"F", " ", "C", "R", "I", "M", "E", " ", "B", "E", "A", "R", "S", " ", \
"B", "I", "T", "T", "E", "R", " ", "F", "R", "U", "I", "T", "?", \
" ", " ", " ", " ", " ", " ", " ", " ", " ", " ", " ", " ", " ", " ", \
" ", " ", " ", " ", " ", " ", " ", " ", " ", " ", " ", " ", " ", " ", \
" ", " ", " ", " ", " ", " ", " ", " ", " ", " ", " ", " ", " ", " ", \
" "}

In[]:= DecryptBlockASCII[blocks[[3]]]
Out[]= {" ", "?", " ", " ", " ", " ", " ", " ", " ", " ", " \
", " ", " ", " ", " ", " ", " ", " ", " ", " ", " ", " ", " ", " ", " \
", " ", " ", " ", " ", " ", " ", " ", " ", " ", " ", " ", " ", " ", " \
", " ", " ", " ", " ", " ", " ", " ", " ", " ", " ", " ", " ", " ", " \
", " ", " ", " ", " ", " ", " ", " ", " ", " ", " ", " ", " ", " ", " \
", " ", " ", " ", " ", " ", " ", " ", " ", " ", " ", " ", " ", " ", " \
"}

In[]:= DecryptBlockASCII[blocks[[4]]]
Out[]= {" ", "f", "H", "O", " ", "K", "N", "O", "W", "S", " ", \
"W", "H", "A", "T", " ", "E", "V", "I", "L", " ", "L", "U", "R", "K", \
"S", " ", "I", "N", " ", "T", "H", "E", " ", "H", "E", "A", "R", "T", \
"S", " ", "O", "F", " ", "M", "E", "N", "?", " ", " ", " ", " ", \
" ", " ", " ", " ", " ", " ", " ", " ", " ", " ", " ", " ", " ", " ", \
" ", " ", " ", " ", " ", " ", " ", " ", " ", " ", " ", " ", " ", " ", \
" "}
\end{lstlisting}

(Я заменил непечатаемые символы на \q{?}.)

Мы видим что первый и третий блоки пустые (или почти пустые),
но второй и четвертый имеют ясно различимые английские слова/фразы.
Похоже что наше предположение насчет ключа верно (как минимум частично).
Это означает, что самый встречающийся 81-байтный блок в файле находится в местах лакун с нулевыми байтами
или что-то в этом роде.

Попробуем расшифровать весь файл:

\begin{lstlisting}[style=custommath]
DecryptBlock[blk_] := BitXor[key, blk]

decrypted = Map[DecryptBlock[#] &, blocks];

BinaryWrite["/home/dennis/.../tmp", Flatten[decrypted]]

Close["/home/dennis/.../tmp"]
\end{lstlisting}

\begin{figure}[H]
\centering
\myincludegraphics{ff/XOR/mask_1/mc_decrypted1.png}
\caption{Расшифрованный файл в Midnight Commander, первая попытка}
\end{figure}

Выглядит как английские фразы для какой-то игры, но что-то не так.
Прежде всего, регистр инвертирован: фразы и некоторые слова начинаются со строчных букв,
в то время как остальные буквы заглавные.
Также, некоторые фразы начинаются с не тех букв.
Посмотрите на самую первую фразу: \q{eHE WEED OF CRIME BEARS BITTER FRUIT}.
Что такое \q{eHE}? Разве не \q{tHE} тут должно быть?
Возможно ли что наш ключ для дешифрования имеет неверный байт в этом месте?

Посмотрим снова на второй блок в файле, на ключ и на результат дешифрования:

\begin{lstlisting}[style=custommath]
In[]:= blocks[[2]]
Out[]= {80, 2, 74, 49, 113, 49, 51, 92, 39, 8, 92, 81, 116, 62, \
57, 80, 46, 40, 114, 36, 75, 56, 33, 76, 9, 55, 56, 59, 81, 65, 45, \
28, 60, 55, 93, 39, 90, 28, 124, 106, 16, 20, 104, 119, 8, 109, 26, \
106, 9, 97, 13, 99, 15, 119, 20, 105, 117, 98, 103, 118, 1, 126, 29, \
97, 122, 17, 15, 114, 110, 3, 5, 125, 125, 99, 126, 119, 102, 30, \
122, 2, 117}

In[]:= key
Out[]= {80, 103, 2, 116, 113, 102, 118, 25, 99, 8, 19, 23, 116, \
125, 107, 25, 99, 109, 114, 102, 14, 121, 115, 31, 9, 117, 113, 111, \
5, 4, 127, 28, 122, 101, 8, 110, 14, 18, 124, 106, 16, 20, 104, 119, \
8, 109, 26, 106, 9, 97, 13, 99, 15, 119, 20, 105, 117, 98, 103, 118, \
1, 126, 29, 97, 122, 17, 15, 114, 110, 3, 5, 125, 125, 99, 126, 119, \
102, 30, 122, 2, 117}

In[]:= BitXor[key, blocks[[2]]]
Out[]= {0, 101, 72, 69, 0, 87, 69, 69, 68, 0, 79, 70, 0, 67, 82, \
73, 77, 69, 0, 66, 69, 65, 82, 83, 0, 66, 73, 84, 84, 69, 82, 0, 70, \
82, 85, 73, 84, 14, 0, 0, 0, 0, 0, 0, 0, 0, 0, 0, 0, 0, 0, 0, 0, 0, \
0, 0, 0, 0, 0, 0, 0, 0, 0, 0, 0, 0, 0, 0, 0, 0, 0, 0, 0, 0, 0, 0, 0, \
0, 0, 0, 0}
\end{lstlisting}

Зашифрованный байт это 2, байт из ключа это 103, $2 \oplus 103=101$ и 101 это ASCII-код символа \q{e}.
Чему должен равнятся этот байт ключа, чтобы ASCII-код был 116 (для символа  \q{t})?
$2 \oplus 116=118$, присвоим 118 второму байту в ключе \dots

\begin{lstlisting}[style=custommath]
key = {80, 118, 2, 116, 113, 102, 118, 25, 99, 8, 19, 23, 116, 125, 
  107, 25, 99, 109, 114, 102, 14, 121, 115, 31, 9, 117, 113, 111, 5, 
  4, 127, 28, 122, 101, 8, 110, 14, 18, 124, 106, 16, 20, 104, 119, 8,
   109, 26, 106, 9, 97, 13, 99, 15, 119, 20, 105, 117, 98, 103, 118, 
  1, 126, 29, 97, 122, 17, 15, 114, 110, 3, 5, 125, 125, 99, 126, 119,
   102, 30, 122, 2, 117}
\end{lstlisting}

\dots и снова дешифруем весь файл.

\begin{figure}[H]
\centering
\myincludegraphics{ff/XOR/mask_1/mc_decrypted2.png}
\caption{Дешифрованный файл в Midnight Commander, вторая попытка}
\end{figure}

Ух ты, теперь грамматика корректна, и все фразы начинаются с корректных букв.
Но все таки, регистр подозрителен.
С чего бы разработчику игры записывать их в такой манере?
Может быть наш ключ все еще неправилен?

% TODO ASCII table somewhere in the book
Изучая таблицу ASCII мы можем заметить что ASCII-коды для букв в верхнем и нижнем регистре отличаются только на один бит
(6-й бит, если считать с первого, 0b100000):

\begin{figure}[H]
\centering
\includegraphics[width=0.7\textwidth]{ascii.png}
\caption{7-битная таблица \ac{ASCII} в Emacs}
\end{figure}

6-й бит, выставленный в нулевом байте, В десятичном виде это будет 32.
Но 32 это ASCII-код пробела!

Действительно, можно менять регистр просто применяя XOR к ASCII-коду, с 32 (больше об этом: \myref{toupper_bit}).

Возможно ли, что пустые лакуны в файле это не нулевые байты, а скорее содержащие пробелы?
Еще раз модифицируем наш XOR-ключ (я про-XOR-ю каждый байт ключа с 32):

\begin{lstlisting}[style=custommath]
(* "32" это скаляр, и "key" это вектор, но это OK *)

In[]:= key3 = BitXor[32, key]
Out[]= {112, 86, 34, 84, 81, 70, 86, 57, 67, 40, 51, 55, 84, 93, 75, \
57, 67, 77, 82, 70, 46, 89, 83, 63, 41, 85, 81, 79, 37, 36, 95, 60, \
90, 69, 40, 78, 46, 50, 92, 74, 48, 52, 72, 87, 40, 77, 58, 74, 41, \
65, 45, 67, 47, 87, 52, 73, 85, 66, 71, 86, 33, 94, 61, 65, 90, 49, \
47, 82, 78, 35, 37, 93, 93, 67, 94, 87, 70, 62, 90, 34, 85}

In[]:= DecryptBlock[blk_] := BitXor[key3, blk]
\end{lstlisting}

И снова дешифруем входной файл:

\begin{figure}[H]
\centering
\myincludegraphics{ff/XOR/mask_1/mc_decrypted.png}
\caption{Дешифрованный файл в Midnight Commander, последняя попытка}
\end{figure}

(Расшифрованный файл доступен здесь:
\url{\GitHubBlobMasterURL/ff/XOR/mask_1/files/decrypted.dat.bz2}.)

Несомненно, это корректный исходный файл.
Да, и мы видим числа в начале каждого блока. Должно быть это и есть источник некорректного XOR-ключа.
Как выходит, самый встречающийся 81-байтный блок в файле это блок заполненный пробелами и содержащий символ \q{1} на месте
второго байта.
Действительно, как-то так получилось что многие блоки здесь перемежаются с этим блоком.
Может быть это что-то вроде выравнивания (padding) для коротких фраз/сообщений?
Другой часто встречающийся 81-байтный блок также заполнен пробелами, но с другой цифрой, следовательно,
они отличаются только вторым байтом.

Вот и всё! Теперь мы можем написать утилиту для зашифрования файла назад, и, может быть, модифицировать его перед этим

Файл для Mathematica можно скачать здесь:\\
\url{\GitHubBlobMasterURL/ff/XOR/mask_1/files/XOR_mask_1.nb}.

Итог: XOR-шифрование не надежно вообще. Вероятно, разработчик игры хотел просто скрыть внутренности игры от игрока,
ничего более серьезного.
Все же, шифрование вроде этого крайне популярно вследствии его простоты, так что многие реверс инженеры обычно хорошо
с этим знакомы.

}

\renewcommand{\CURPATH}{advanced/760_no_return}
\EN{\mysection{The case of forgotten return}
\label{ForgottenReturn}

Let's revisit the ``attempt to use the result of a function returning \Tvoid'' part: \label{UseResultOfVoidFunc}.

This is a bug I once hit.

And this is also yet another demonstration, how C/C++ places return value into \EAX/\RAX register.

In the piece of code like that, I forgot to add \TT{return}:

\lstinputlisting[style=customc]{\CURPATH/1.c}

Non-optimizing GCC 5.4 silently compiles this with no warnings.
And the code works!
Let's see, why:

\lstinputlisting[style=customasmx86,caption=Non-optimizing GCC 5.4]{\CURPATH/O0_no_return_works.lst}

If I add \TT{return rt;}, the only instruction is added at the end, which is redundant:

\lstinputlisting[style=customasmx86,caption=Non-optimizing GCC 5.4]{\CURPATH/O0_return_works.lst}

Bugs like that are very dangerous, sometimes they appear, sometimes hide.
It's like Heisenbug.

Now I'm trying optimizing GCC:

\lstinputlisting[style=customasmx86,caption=Optimizing GCC 5.4]{\CURPATH/O3_no_return_crash.lst}

Compiler deducing that nothing returns from the function, so it optimizes it away.
And it assumes, that is returns 0 by default. The zero is then used as an address to a structure in main()..
Of course, this code crashes.

GCC is C++ mode silent about it as well.

Let's try non-optimizing MSVC 2015 x86.
It warns about the problem:

\begin{lstlisting}
c:\tmp\3.c(19) : warning C4716: 'create_color': must return a value                                                               
\end{lstlisting}

And generates crashing code:

\lstinputlisting[style=customasmx86,caption=Non-optimizing MSVC 2015 x86]{\CURPATH/MSVC_x86_crash.lst}

Now optimizing MSVC 2015 x86 generates crashing code as well, but for the different reason:

\lstinputlisting[style=customasmx86,caption=Optimizing MSVC 2015 x86]{\CURPATH/MSVC_Ox_x86_crash.lst}

However, non-optimizing MSVC 2015 x64 generates working code:

\lstinputlisting[style=customasmx86,caption=Non-optimizing MSVC 2015 x64]{\CURPATH/MSVC_x64_works.lst}

Optimizing MSVC 2015 x64 also inlines the function, as in case of x86, and the resulting code also crashes.

\myhrule{}

This is a real piece of code from my \emph{octothorpe} library\footnote{\url{https://github.com/DennisYurichev/octothorpe}}, that worked and all tests passed.
It was so, without \verb|return| for quite a time...

\begin{lstlisting}
uint32_t LPHM_u32_hash(void *key)
{
        jenkins_one_at_a_time_hash_u32((uint32_t)key);
}
\end{lstlisting}

\myhrule{}

The moral of the story: warnings are very important, use \TT{-Wall}, etc, etc...
When \TT{return} statement is absent, compiler can just silently do nothing at that point.

Such a bug left unnoticed can ruin a day.

Also, \emph{shotgun debugging}
is bad, because again, such a bug can left unnoticed (``everything works now, so be it'').

}%
\FR{\mysection{Fonction presque vide}
\label{Boolector}
\myindex{Boolector}
\myindex{x86!\Instructions!JMP}

Ceci est un morceau de code réel que j'ai trouvé dans Boolector\footnote{\url{https://boolector.github.io/}}:

\lstinputlisting[style=customc]{patterns/025_almost_empty/boolectormain.c}

Pourquoi quelqu'un ferait-il comme ça?
Je ne sais pas mais mon hypothèse est que \verb|boolector_main()| peut être compilée
dans une sorte de DLL ou bibliothèque dynamique, et appelée depuis une suite de test.
Certainement qu'une suite de test peut préparer les variables argc/argv comme
le ferait \ac{CRT}.

Il est intéressant de voir comment c'est compilé:

\lstinputlisting[caption=GCC 8.2 x64 \NonOptimizing (\assemblyOutput),style=customasmx86]{patterns/025_almost_empty/boolectormain_O0.s}

Ceci est OK, le prologue (non optimisé) déplace inutilement deux arguments,
\INS{CALL}, épilogue, \INS{RET}.
Mais regardons la version optimisée:

\lstinputlisting[caption=GCC 8.2 x64 \Optimizing (\assemblyOutput),style=customasmx86]{patterns/025_almost_empty/boolectormain_O3.s}

Aussi simple que ça: la pile et les registres ne sont pas touchés et \verb|boolector_main()|
a le même ensemble d'arguments.
Donc, tout ce que nous avons à faire est de passer l'exécution à une autre adresse.

Ceci est proche d'une \glslink{thunk function}{fonction thunk}.

Nous verons queelque chose de plus avancé plus tard: \myref{ARM_B_to_printf}, \myref{JMP_instead_of_RET}.
}

\renewcommand{\CURPATH}{advanced/770_more_func_ptrs}
\EN{% TODO translate
\mysection{Breaking simple executable cryptor}

I've got an executable file which is encrypted by relatively simple encryption.
\href{\GitHubBlobMasterURL/examples/simple_exec_crypto/files/cipher.bin}{Here is it} (only executable section is left here).

First, all encryption function does is just adds number of position in buffer to the byte.
Here is how this can be encoded in Python:

\begin{lstlisting}[caption=Python script,style=custompy]
#!/usr/bin/env python
def e(i, k):
    return chr ((ord(i)+k) % 256)

def encrypt(buf):
    return e(buf[0], 0)+ e(buf[1], 1)+ e(buf[2], 2) + e(buf[3], 3)+ e(buf[4], 4)+ e(buf[5], 5)+ e(buf[6], 6)+ e(buf[7], 7)+
           e(buf[8], 8)+ e(buf[9], 9)+ e(buf[10], 10)+ e(buf[11], 11)+ e(buf[12], 12)+ e(buf[13], 13)+ e(buf[14], 14)+ e(buf[15], 15)
\end{lstlisting}

Hence, if you encrypt buffer with 16 zeros, you'll get \emph{0, 1, 2, 3 ... 12, 13, 14, 15}.

\myindex{Propagating Cipher Block Chaining}
Propagating Cipher Block Chaining (PCBC) is also used, here is how it works:

\begin{figure}[H]
\centering
\myincludegraphics{examples/simple_exec_crypto/601px-PCBC_encryption.png}
\caption{Propagating Cipher Block Chaining encryption (image is taken from Wikipedia article)}
\end{figure}

The problem is that it's too boring to recover IV (Initialization Vector) each time.
Brute-force is also not an option, because IV is too long (16 bytes).
Let's see, if it's possible to recover IV for arbitrary encrypted executable file?

Let's try simple frequency analysis.
This is 32-bit x86 executable code, so let's gather statistics about most frequent bytes and opcodes.
I tried huge oracle.exe file from Oracle RDBMS version 11.2 for windows x86 and I've found that the most frequent byte (no surprise) is zero (~10\%).
The next most frequent byte is (again, no surprise) 0xFF (~5\%).
The next is 0x8B (~5\%).

\myindex{x86!\Instructions!MOV}
0x8B is opcode for \INS{MOV}, this is indeed one of the most busy x86 instructions.
Now what about popularity of zero byte?
If compiler needs to encode value bigger than 127, it has to use 32-bit displacement instead of 8-bit one, but large values are very rare,
so it is padded by zeros.
\myindex{x86!\Instructions!LEA}
\myindex{x86!\Instructions!PUSH}
\myindex{x86!\Instructions!CALL}
This is at least in \INS{LEA}, \INS{MOV}, \INS{PUSH}, \INS{CALL}.

For example:

\begin{lstlisting}[style=customasmx86]
8D B0 28 01 00 00                 lea     esi, [eax+128h]
8D BF 40 38 00 00                 lea     edi, [edi+3840h]
\end{lstlisting}

Displacements bigger than 127 are very popular, but they are rarely exceeds 0x10000
(indeed, such large memory buffers/structures are also rare).

Same story with \INS{MOV}, large constants are rare, the most heavily used are 0, 1, 10, 100, $2^n$, and so on.
Compiler has to pad small constants by zeros to represent them as 32-bit values:

\begin{lstlisting}[style=customasmx86]
BF 02 00 00 00                    mov     edi, 2
BF 01 00 00 00                    mov     edi, 1
\end{lstlisting}

Now about 00 and FF bytes combined: jumps (including conditional) and calls can pass execution flow forward or backwards, but very often,
within the limits of the current executable module.
If forward, displacement is not very big and also padded with zeros.
If backwards, displacement is represented as negative value, so padded with FF bytes.
For example, transfer execution flow forward:

\begin{lstlisting}[style=customasmx86]
E8 43 0C 00 00                    call    _function1
E8 5C 00 00 00                    call    _function2
0F 84 F0 0A 00 00                 jz      loc_4F09A0
0F 84 EB 00 00 00                 jz      loc_4EFBB8
\end{lstlisting}

Backwards:

\begin{lstlisting}[style=customasmx86]
E8 79 0C FE FF                    call    _function1
E8 F4 16 FF FF                    call    _function2
0F 84 F8 FB FF FF                 jz      loc_8212BC
0F 84 06 FD FF FF                 jz      loc_FF1E7D
\end{lstlisting}

FF byte is also very often occurred in negative displacements like these:

\begin{lstlisting}[style=customasmx86]
8D 85 1E FF FF FF                 lea     eax, [ebp-0E2h]
8D 95 F8 5C FF FF                 lea     edx, [ebp-0A308h]
\end{lstlisting}

So far so good. Now we have to try various 16-byte keys, decrypt executable section and measure how often 00, FF and 8B bytes are occurred.
Let's also keep in sight how PCBC decryption works:

\begin{figure}[H]
\centering
\myincludegraphics{examples/simple_exec_crypto/640px-PCBC_decryption.png}
\caption{Propagating Cipher Block Chaining decryption (image is taken from Wikipedia article)}
\end{figure}

The good news is that we don't really have to decrypt whole piece of data, but only slice by slice, this is exactly how I did in my previous example: \myref{XOR_mask_2}.

Now I'm trying all possible bytes (0..255) for each byte in key and just pick the byte producing maximal amount of 00/FF/8B bytes in a decrypted slice:

\begin{lstlisting}[style=custompy]
#!/usr/bin/env python
import sys, hexdump, array, string, operator

KEY_LEN=16

def chunks(l, n):
    # split n by l-byte chunks
    # https://stackoverflow.com/q/312443
    n = max(1, n)
    return [l[i:i + n] for i in range(0, len(l), n)]

def read_file(fname):
    file=open(fname, mode='rb')
    content=file.read()
    file.close()
    return content

def decrypt_byte (c, key):
    return chr((ord(c)-key) % 256)

def XOR_PCBC_step (IV, buf, k):
    prev=IV
    rt=""
    for c in buf:
	new_c=decrypt_byte(c, k)
        plain=chr(ord(new_c)^ord(prev))
	prev=chr(ord(c)^ord(plain))
	rt=rt+plain
    return rt

each_Nth_byte=[""]*KEY_LEN

content=read_file(sys.argv[1])
# split input by 16-byte chunks:
all_chunks=chunks(content, KEY_LEN)
for c in all_chunks:
    for i in range(KEY_LEN):
        each_Nth_byte[i]=each_Nth_byte[i] + c[i]

# try each byte of key
for N in range(KEY_LEN):
    print "N=", N
    stat={}
    for i in range(256):
        tmp_key=chr(i)
	tmp=XOR_PCBC_step(tmp_key,each_Nth_byte[N], N)
        # count 0, FFs and 8Bs in decrypted buffer:
	important_bytes=tmp.count('\x00')+tmp.count('\xFF')+tmp.count('\x8B')
	stat[i]=important_bytes
    sorted_stat = sorted(stat.iteritems(), key=operator.itemgetter(1), reverse=True)
    print sorted_stat[0]
\end{lstlisting}

(Source code can be downloaded \href{\GitHubBlobMasterURL/examples/simple_exec_crypto/files/decrypt.py}{here}.)

I run it and here is a key for which 00/FF/8B bytes presence in decrypted buffer is maximal:

\begin{lstlisting}
N= 0
(147, 1224)
N= 1
(94, 1327)
N= 2
(252, 1223)
N= 3
(218, 1266)
N= 4
(38, 1209)
N= 5
(192, 1378)
N= 6
(199, 1204)
N= 7
(213, 1332)
N= 8
(225, 1251)
N= 9
(112, 1223)
N= 10
(143, 1177)
N= 11
(108, 1286)
N= 12
(10, 1164)
N= 13
(3, 1271)
N= 14
(128, 1253)
N= 15
(232, 1330)
\end{lstlisting}

Let's write decryption utility with the key we got:

\begin{lstlisting}[style=custompy]
#!/usr/bin/env python
import sys, hexdump, array

def xor_strings(s,t):
    # \verb|https://en.wikipedia.org/wiki/XOR_cipher#Example_implementation|
    """xor two strings together"""
    return "".join(chr(ord(a)^ord(b)) for a,b in zip(s,t))

IV=array.array('B', [147, 94, 252, 218, 38, 192, 199, 213, 225, 112, 143, 108, 10, 3, 128, 232]).tostring()

def chunks(l, n):
    n = max(1, n)
    return [l[i:i + n] for i in range(0, len(l), n)]

def read_file(fname):
    file=open(fname, mode='rb')
    content=file.read()
    file.close()
    return content

def decrypt_byte(i, k):
    return chr ((ord(i)-k) % 256)

def decrypt(buf):
    return "".join(decrypt_byte(buf[i], i) for i in range(16))

fout=open(sys.argv[2], mode='wb')

prev=IV
content=read_file(sys.argv[1])
tmp=chunks(content, 16)
for c in tmp:
    new_c=decrypt(c)
    p=xor_strings (new_c, prev)
    prev=xor_strings(c, p)
    fout.write(p)
fout.close()
\end{lstlisting}

(Source code can be downloaded \href{\GitHubBlobMasterURL/examples/simple_exec_crypto/files/decrypt2.py}{here}.)

Let's check resulting file:

\lstinputlisting{examples/simple_exec_crypto/objdump_result.txt}

Yes, this is seems correctly disassembled piece of x86 code.
The whole decryped file can be downloaded \href{\GitHubBlobMasterURL/examples/simple_exec_crypto/files/decrypted.bin}{here}.

In fact, this is text section from regedit.exe from Windows 7.
But this example is based on a real case I encountered, so just executable is different (and key), algorithm is the same.

\subsection{Other ideas to consider}

What if I would fail with such simple frequency analysis?
There are other ideas on how to measure correctness of decrypted/decompressed x86 code:

\begin{itemize}

\item Many modern compilers aligns functions on 0x10 border.
So the space left before is filled with NOPs (0x90) or other NOP instructions with known opcodes: \myref{sec:npad}.

\item Perhaps, the most frequent pattern in any assembly language is function call:\\
\TT{PUSH chain / CALL / ADD ESP, X}.
This sequence can easily detected and found.
I've even gathered statistics about average number of function arguments: \myref{args_stat}.
(Hence, this is average length of PUSH chain.)

\end{itemize}

Read more about incorrectly/correctly disassembled code: \myref{ISA_detect}.
}%
\FR{\mysection{Fonction presque vide}
\label{Boolector}
\myindex{Boolector}
\myindex{x86!\Instructions!JMP}

Ceci est un morceau de code réel que j'ai trouvé dans Boolector\footnote{\url{https://boolector.github.io/}}:

\lstinputlisting[style=customc]{patterns/025_almost_empty/boolectormain.c}

Pourquoi quelqu'un ferait-il comme ça?
Je ne sais pas mais mon hypothèse est que \verb|boolector_main()| peut être compilée
dans une sorte de DLL ou bibliothèque dynamique, et appelée depuis une suite de test.
Certainement qu'une suite de test peut préparer les variables argc/argv comme
le ferait \ac{CRT}.

Il est intéressant de voir comment c'est compilé:

\lstinputlisting[caption=GCC 8.2 x64 \NonOptimizing (\assemblyOutput),style=customasmx86]{patterns/025_almost_empty/boolectormain_O0.s}

Ceci est OK, le prologue (non optimisé) déplace inutilement deux arguments,
\INS{CALL}, épilogue, \INS{RET}.
Mais regardons la version optimisée:

\lstinputlisting[caption=GCC 8.2 x64 \Optimizing (\assemblyOutput),style=customasmx86]{patterns/025_almost_empty/boolectormain_O3.s}

Aussi simple que ça: la pile et les registres ne sont pas touchés et \verb|boolector_main()|
a le même ensemble d'arguments.
Donc, tout ce que nous avons à faire est de passer l'exécution à une autre adresse.

Ceci est proche d'une \glslink{thunk function}{fonction thunk}.

Nous verons queelque chose de plus avancé plus tard: \myref{ARM_B_to_printf}, \myref{JMP_instead_of_RET}.
}

\renewcommand{\CURPATH}{advanced/800_win16}
\EN{\EN{\input{patterns/016_empty_redux/main_EN}}%
\FR{\input{patterns/016_empty_redux/main_FR}}
}\RU{\EN{\input{patterns/016_empty_redux/main_EN}}%
\FR{\input{patterns/016_empty_redux/main_FR}}
}\FR{\mysection{Fonction presque vide}
\label{Boolector}
\myindex{Boolector}
\myindex{x86!\Instructions!JMP}

Ceci est un morceau de code réel que j'ai trouvé dans Boolector\footnote{\url{https://boolector.github.io/}}:

\lstinputlisting[style=customc]{patterns/025_almost_empty/boolectormain.c}

Pourquoi quelqu'un ferait-il comme ça?
Je ne sais pas mais mon hypothèse est que \verb|boolector_main()| peut être compilée
dans une sorte de DLL ou bibliothèque dynamique, et appelée depuis une suite de test.
Certainement qu'une suite de test peut préparer les variables argc/argv comme
le ferait \ac{CRT}.

Il est intéressant de voir comment c'est compilé:

\lstinputlisting[caption=GCC 8.2 x64 \NonOptimizing (\assemblyOutput),style=customasmx86]{patterns/025_almost_empty/boolectormain_O0.s}

Ceci est OK, le prologue (non optimisé) déplace inutilement deux arguments,
\INS{CALL}, épilogue, \INS{RET}.
Mais regardons la version optimisée:

\lstinputlisting[caption=GCC 8.2 x64 \Optimizing (\assemblyOutput),style=customasmx86]{patterns/025_almost_empty/boolectormain_O3.s}

Aussi simple que ça: la pile et les registres ne sont pas touchés et \verb|boolector_main()|
a le même ensemble d'arguments.
Donc, tout ce que nous avons à faire est de passer l'exécution à une autre adresse.

Ceci est proche d'une \glslink{thunk function}{fonction thunk}.

Nous verons queelque chose de plus avancé plus tard: \myref{ARM_B_to_printf}, \myref{JMP_instead_of_RET}.
}



\EN{\EN{% TODO translate
\mysection{Breaking simple executable cryptor}

I've got an executable file which is encrypted by relatively simple encryption.
\href{\GitHubBlobMasterURL/examples/simple_exec_crypto/files/cipher.bin}{Here is it} (only executable section is left here).

First, all encryption function does is just adds number of position in buffer to the byte.
Here is how this can be encoded in Python:

\begin{lstlisting}[caption=Python script,style=custompy]
#!/usr/bin/env python
def e(i, k):
    return chr ((ord(i)+k) % 256)

def encrypt(buf):
    return e(buf[0], 0)+ e(buf[1], 1)+ e(buf[2], 2) + e(buf[3], 3)+ e(buf[4], 4)+ e(buf[5], 5)+ e(buf[6], 6)+ e(buf[7], 7)+
           e(buf[8], 8)+ e(buf[9], 9)+ e(buf[10], 10)+ e(buf[11], 11)+ e(buf[12], 12)+ e(buf[13], 13)+ e(buf[14], 14)+ e(buf[15], 15)
\end{lstlisting}

Hence, if you encrypt buffer with 16 zeros, you'll get \emph{0, 1, 2, 3 ... 12, 13, 14, 15}.

\myindex{Propagating Cipher Block Chaining}
Propagating Cipher Block Chaining (PCBC) is also used, here is how it works:

\begin{figure}[H]
\centering
\myincludegraphics{examples/simple_exec_crypto/601px-PCBC_encryption.png}
\caption{Propagating Cipher Block Chaining encryption (image is taken from Wikipedia article)}
\end{figure}

The problem is that it's too boring to recover IV (Initialization Vector) each time.
Brute-force is also not an option, because IV is too long (16 bytes).
Let's see, if it's possible to recover IV for arbitrary encrypted executable file?

Let's try simple frequency analysis.
This is 32-bit x86 executable code, so let's gather statistics about most frequent bytes and opcodes.
I tried huge oracle.exe file from Oracle RDBMS version 11.2 for windows x86 and I've found that the most frequent byte (no surprise) is zero (~10\%).
The next most frequent byte is (again, no surprise) 0xFF (~5\%).
The next is 0x8B (~5\%).

\myindex{x86!\Instructions!MOV}
0x8B is opcode for \INS{MOV}, this is indeed one of the most busy x86 instructions.
Now what about popularity of zero byte?
If compiler needs to encode value bigger than 127, it has to use 32-bit displacement instead of 8-bit one, but large values are very rare,
so it is padded by zeros.
\myindex{x86!\Instructions!LEA}
\myindex{x86!\Instructions!PUSH}
\myindex{x86!\Instructions!CALL}
This is at least in \INS{LEA}, \INS{MOV}, \INS{PUSH}, \INS{CALL}.

For example:

\begin{lstlisting}[style=customasmx86]
8D B0 28 01 00 00                 lea     esi, [eax+128h]
8D BF 40 38 00 00                 lea     edi, [edi+3840h]
\end{lstlisting}

Displacements bigger than 127 are very popular, but they are rarely exceeds 0x10000
(indeed, such large memory buffers/structures are also rare).

Same story with \INS{MOV}, large constants are rare, the most heavily used are 0, 1, 10, 100, $2^n$, and so on.
Compiler has to pad small constants by zeros to represent them as 32-bit values:

\begin{lstlisting}[style=customasmx86]
BF 02 00 00 00                    mov     edi, 2
BF 01 00 00 00                    mov     edi, 1
\end{lstlisting}

Now about 00 and FF bytes combined: jumps (including conditional) and calls can pass execution flow forward or backwards, but very often,
within the limits of the current executable module.
If forward, displacement is not very big and also padded with zeros.
If backwards, displacement is represented as negative value, so padded with FF bytes.
For example, transfer execution flow forward:

\begin{lstlisting}[style=customasmx86]
E8 43 0C 00 00                    call    _function1
E8 5C 00 00 00                    call    _function2
0F 84 F0 0A 00 00                 jz      loc_4F09A0
0F 84 EB 00 00 00                 jz      loc_4EFBB8
\end{lstlisting}

Backwards:

\begin{lstlisting}[style=customasmx86]
E8 79 0C FE FF                    call    _function1
E8 F4 16 FF FF                    call    _function2
0F 84 F8 FB FF FF                 jz      loc_8212BC
0F 84 06 FD FF FF                 jz      loc_FF1E7D
\end{lstlisting}

FF byte is also very often occurred in negative displacements like these:

\begin{lstlisting}[style=customasmx86]
8D 85 1E FF FF FF                 lea     eax, [ebp-0E2h]
8D 95 F8 5C FF FF                 lea     edx, [ebp-0A308h]
\end{lstlisting}

So far so good. Now we have to try various 16-byte keys, decrypt executable section and measure how often 00, FF and 8B bytes are occurred.
Let's also keep in sight how PCBC decryption works:

\begin{figure}[H]
\centering
\myincludegraphics{examples/simple_exec_crypto/640px-PCBC_decryption.png}
\caption{Propagating Cipher Block Chaining decryption (image is taken from Wikipedia article)}
\end{figure}

The good news is that we don't really have to decrypt whole piece of data, but only slice by slice, this is exactly how I did in my previous example: \myref{XOR_mask_2}.

Now I'm trying all possible bytes (0..255) for each byte in key and just pick the byte producing maximal amount of 00/FF/8B bytes in a decrypted slice:

\begin{lstlisting}[style=custompy]
#!/usr/bin/env python
import sys, hexdump, array, string, operator

KEY_LEN=16

def chunks(l, n):
    # split n by l-byte chunks
    # https://stackoverflow.com/q/312443
    n = max(1, n)
    return [l[i:i + n] for i in range(0, len(l), n)]

def read_file(fname):
    file=open(fname, mode='rb')
    content=file.read()
    file.close()
    return content

def decrypt_byte (c, key):
    return chr((ord(c)-key) % 256)

def XOR_PCBC_step (IV, buf, k):
    prev=IV
    rt=""
    for c in buf:
	new_c=decrypt_byte(c, k)
        plain=chr(ord(new_c)^ord(prev))
	prev=chr(ord(c)^ord(plain))
	rt=rt+plain
    return rt

each_Nth_byte=[""]*KEY_LEN

content=read_file(sys.argv[1])
# split input by 16-byte chunks:
all_chunks=chunks(content, KEY_LEN)
for c in all_chunks:
    for i in range(KEY_LEN):
        each_Nth_byte[i]=each_Nth_byte[i] + c[i]

# try each byte of key
for N in range(KEY_LEN):
    print "N=", N
    stat={}
    for i in range(256):
        tmp_key=chr(i)
	tmp=XOR_PCBC_step(tmp_key,each_Nth_byte[N], N)
        # count 0, FFs and 8Bs in decrypted buffer:
	important_bytes=tmp.count('\x00')+tmp.count('\xFF')+tmp.count('\x8B')
	stat[i]=important_bytes
    sorted_stat = sorted(stat.iteritems(), key=operator.itemgetter(1), reverse=True)
    print sorted_stat[0]
\end{lstlisting}

(Source code can be downloaded \href{\GitHubBlobMasterURL/examples/simple_exec_crypto/files/decrypt.py}{here}.)

I run it and here is a key for which 00/FF/8B bytes presence in decrypted buffer is maximal:

\begin{lstlisting}
N= 0
(147, 1224)
N= 1
(94, 1327)
N= 2
(252, 1223)
N= 3
(218, 1266)
N= 4
(38, 1209)
N= 5
(192, 1378)
N= 6
(199, 1204)
N= 7
(213, 1332)
N= 8
(225, 1251)
N= 9
(112, 1223)
N= 10
(143, 1177)
N= 11
(108, 1286)
N= 12
(10, 1164)
N= 13
(3, 1271)
N= 14
(128, 1253)
N= 15
(232, 1330)
\end{lstlisting}

Let's write decryption utility with the key we got:

\begin{lstlisting}[style=custompy]
#!/usr/bin/env python
import sys, hexdump, array

def xor_strings(s,t):
    # \verb|https://en.wikipedia.org/wiki/XOR_cipher#Example_implementation|
    """xor two strings together"""
    return "".join(chr(ord(a)^ord(b)) for a,b in zip(s,t))

IV=array.array('B', [147, 94, 252, 218, 38, 192, 199, 213, 225, 112, 143, 108, 10, 3, 128, 232]).tostring()

def chunks(l, n):
    n = max(1, n)
    return [l[i:i + n] for i in range(0, len(l), n)]

def read_file(fname):
    file=open(fname, mode='rb')
    content=file.read()
    file.close()
    return content

def decrypt_byte(i, k):
    return chr ((ord(i)-k) % 256)

def decrypt(buf):
    return "".join(decrypt_byte(buf[i], i) for i in range(16))

fout=open(sys.argv[2], mode='wb')

prev=IV
content=read_file(sys.argv[1])
tmp=chunks(content, 16)
for c in tmp:
    new_c=decrypt(c)
    p=xor_strings (new_c, prev)
    prev=xor_strings(c, p)
    fout.write(p)
fout.close()
\end{lstlisting}

(Source code can be downloaded \href{\GitHubBlobMasterURL/examples/simple_exec_crypto/files/decrypt2.py}{here}.)

Let's check resulting file:

\lstinputlisting{examples/simple_exec_crypto/objdump_result.txt}

Yes, this is seems correctly disassembled piece of x86 code.
The whole decryped file can be downloaded \href{\GitHubBlobMasterURL/examples/simple_exec_crypto/files/decrypted.bin}{here}.

In fact, this is text section from regedit.exe from Windows 7.
But this example is based on a real case I encountered, so just executable is different (and key), algorithm is the same.

\subsection{Other ideas to consider}

What if I would fail with such simple frequency analysis?
There are other ideas on how to measure correctness of decrypted/decompressed x86 code:

\begin{itemize}

\item Many modern compilers aligns functions on 0x10 border.
So the space left before is filled with NOPs (0x90) or other NOP instructions with known opcodes: \myref{sec:npad}.

\item Perhaps, the most frequent pattern in any assembly language is function call:\\
\TT{PUSH chain / CALL / ADD ESP, X}.
This sequence can easily detected and found.
I've even gathered statistics about average number of function arguments: \myref{args_stat}.
(Hence, this is average length of PUSH chain.)

\end{itemize}

Read more about incorrectly/correctly disassembled code: \myref{ISA_detect}.
}%
\FR{\mysection{Une fonction vide: redux}

Revenons sur l'exemple de la fonction vide \myref{empty_func}.
Maintenant que nous connaissons le prologue et l'épilogue de fonction, ceci est
une fonction vide \myref{lst:empty_func} compilée par GCC sans optimisation:

\lstinputlisting[caption=GCC 8.2 x64 \NonOptimizing (\assemblyOutput),style=customasmx86]{patterns/016_empty_redux/1.s}

C'est \INS{RET}, mais le prologue et l'épilogue de la fonction, probablement, n'ont
pas été optimisés et laissés tels quels.
\INS{NOP} semble être un autre artefact du compilateur.
De toutes façons, la seule instruction effective ici est \INS{RET}.
Toutes les autres instructions peuvent être supprimées (ou optimisées).

}
}
\RU{\EN{% TODO translate
\mysection{Breaking simple executable cryptor}

I've got an executable file which is encrypted by relatively simple encryption.
\href{\GitHubBlobMasterURL/examples/simple_exec_crypto/files/cipher.bin}{Here is it} (only executable section is left here).

First, all encryption function does is just adds number of position in buffer to the byte.
Here is how this can be encoded in Python:

\begin{lstlisting}[caption=Python script,style=custompy]
#!/usr/bin/env python
def e(i, k):
    return chr ((ord(i)+k) % 256)

def encrypt(buf):
    return e(buf[0], 0)+ e(buf[1], 1)+ e(buf[2], 2) + e(buf[3], 3)+ e(buf[4], 4)+ e(buf[5], 5)+ e(buf[6], 6)+ e(buf[7], 7)+
           e(buf[8], 8)+ e(buf[9], 9)+ e(buf[10], 10)+ e(buf[11], 11)+ e(buf[12], 12)+ e(buf[13], 13)+ e(buf[14], 14)+ e(buf[15], 15)
\end{lstlisting}

Hence, if you encrypt buffer with 16 zeros, you'll get \emph{0, 1, 2, 3 ... 12, 13, 14, 15}.

\myindex{Propagating Cipher Block Chaining}
Propagating Cipher Block Chaining (PCBC) is also used, here is how it works:

\begin{figure}[H]
\centering
\myincludegraphics{examples/simple_exec_crypto/601px-PCBC_encryption.png}
\caption{Propagating Cipher Block Chaining encryption (image is taken from Wikipedia article)}
\end{figure}

The problem is that it's too boring to recover IV (Initialization Vector) each time.
Brute-force is also not an option, because IV is too long (16 bytes).
Let's see, if it's possible to recover IV for arbitrary encrypted executable file?

Let's try simple frequency analysis.
This is 32-bit x86 executable code, so let's gather statistics about most frequent bytes and opcodes.
I tried huge oracle.exe file from Oracle RDBMS version 11.2 for windows x86 and I've found that the most frequent byte (no surprise) is zero (~10\%).
The next most frequent byte is (again, no surprise) 0xFF (~5\%).
The next is 0x8B (~5\%).

\myindex{x86!\Instructions!MOV}
0x8B is opcode for \INS{MOV}, this is indeed one of the most busy x86 instructions.
Now what about popularity of zero byte?
If compiler needs to encode value bigger than 127, it has to use 32-bit displacement instead of 8-bit one, but large values are very rare,
so it is padded by zeros.
\myindex{x86!\Instructions!LEA}
\myindex{x86!\Instructions!PUSH}
\myindex{x86!\Instructions!CALL}
This is at least in \INS{LEA}, \INS{MOV}, \INS{PUSH}, \INS{CALL}.

For example:

\begin{lstlisting}[style=customasmx86]
8D B0 28 01 00 00                 lea     esi, [eax+128h]
8D BF 40 38 00 00                 lea     edi, [edi+3840h]
\end{lstlisting}

Displacements bigger than 127 are very popular, but they are rarely exceeds 0x10000
(indeed, such large memory buffers/structures are also rare).

Same story with \INS{MOV}, large constants are rare, the most heavily used are 0, 1, 10, 100, $2^n$, and so on.
Compiler has to pad small constants by zeros to represent them as 32-bit values:

\begin{lstlisting}[style=customasmx86]
BF 02 00 00 00                    mov     edi, 2
BF 01 00 00 00                    mov     edi, 1
\end{lstlisting}

Now about 00 and FF bytes combined: jumps (including conditional) and calls can pass execution flow forward or backwards, but very often,
within the limits of the current executable module.
If forward, displacement is not very big and also padded with zeros.
If backwards, displacement is represented as negative value, so padded with FF bytes.
For example, transfer execution flow forward:

\begin{lstlisting}[style=customasmx86]
E8 43 0C 00 00                    call    _function1
E8 5C 00 00 00                    call    _function2
0F 84 F0 0A 00 00                 jz      loc_4F09A0
0F 84 EB 00 00 00                 jz      loc_4EFBB8
\end{lstlisting}

Backwards:

\begin{lstlisting}[style=customasmx86]
E8 79 0C FE FF                    call    _function1
E8 F4 16 FF FF                    call    _function2
0F 84 F8 FB FF FF                 jz      loc_8212BC
0F 84 06 FD FF FF                 jz      loc_FF1E7D
\end{lstlisting}

FF byte is also very often occurred in negative displacements like these:

\begin{lstlisting}[style=customasmx86]
8D 85 1E FF FF FF                 lea     eax, [ebp-0E2h]
8D 95 F8 5C FF FF                 lea     edx, [ebp-0A308h]
\end{lstlisting}

So far so good. Now we have to try various 16-byte keys, decrypt executable section and measure how often 00, FF and 8B bytes are occurred.
Let's also keep in sight how PCBC decryption works:

\begin{figure}[H]
\centering
\myincludegraphics{examples/simple_exec_crypto/640px-PCBC_decryption.png}
\caption{Propagating Cipher Block Chaining decryption (image is taken from Wikipedia article)}
\end{figure}

The good news is that we don't really have to decrypt whole piece of data, but only slice by slice, this is exactly how I did in my previous example: \myref{XOR_mask_2}.

Now I'm trying all possible bytes (0..255) for each byte in key and just pick the byte producing maximal amount of 00/FF/8B bytes in a decrypted slice:

\begin{lstlisting}[style=custompy]
#!/usr/bin/env python
import sys, hexdump, array, string, operator

KEY_LEN=16

def chunks(l, n):
    # split n by l-byte chunks
    # https://stackoverflow.com/q/312443
    n = max(1, n)
    return [l[i:i + n] for i in range(0, len(l), n)]

def read_file(fname):
    file=open(fname, mode='rb')
    content=file.read()
    file.close()
    return content

def decrypt_byte (c, key):
    return chr((ord(c)-key) % 256)

def XOR_PCBC_step (IV, buf, k):
    prev=IV
    rt=""
    for c in buf:
	new_c=decrypt_byte(c, k)
        plain=chr(ord(new_c)^ord(prev))
	prev=chr(ord(c)^ord(plain))
	rt=rt+plain
    return rt

each_Nth_byte=[""]*KEY_LEN

content=read_file(sys.argv[1])
# split input by 16-byte chunks:
all_chunks=chunks(content, KEY_LEN)
for c in all_chunks:
    for i in range(KEY_LEN):
        each_Nth_byte[i]=each_Nth_byte[i] + c[i]

# try each byte of key
for N in range(KEY_LEN):
    print "N=", N
    stat={}
    for i in range(256):
        tmp_key=chr(i)
	tmp=XOR_PCBC_step(tmp_key,each_Nth_byte[N], N)
        # count 0, FFs and 8Bs in decrypted buffer:
	important_bytes=tmp.count('\x00')+tmp.count('\xFF')+tmp.count('\x8B')
	stat[i]=important_bytes
    sorted_stat = sorted(stat.iteritems(), key=operator.itemgetter(1), reverse=True)
    print sorted_stat[0]
\end{lstlisting}

(Source code can be downloaded \href{\GitHubBlobMasterURL/examples/simple_exec_crypto/files/decrypt.py}{here}.)

I run it and here is a key for which 00/FF/8B bytes presence in decrypted buffer is maximal:

\begin{lstlisting}
N= 0
(147, 1224)
N= 1
(94, 1327)
N= 2
(252, 1223)
N= 3
(218, 1266)
N= 4
(38, 1209)
N= 5
(192, 1378)
N= 6
(199, 1204)
N= 7
(213, 1332)
N= 8
(225, 1251)
N= 9
(112, 1223)
N= 10
(143, 1177)
N= 11
(108, 1286)
N= 12
(10, 1164)
N= 13
(3, 1271)
N= 14
(128, 1253)
N= 15
(232, 1330)
\end{lstlisting}

Let's write decryption utility with the key we got:

\begin{lstlisting}[style=custompy]
#!/usr/bin/env python
import sys, hexdump, array

def xor_strings(s,t):
    # \verb|https://en.wikipedia.org/wiki/XOR_cipher#Example_implementation|
    """xor two strings together"""
    return "".join(chr(ord(a)^ord(b)) for a,b in zip(s,t))

IV=array.array('B', [147, 94, 252, 218, 38, 192, 199, 213, 225, 112, 143, 108, 10, 3, 128, 232]).tostring()

def chunks(l, n):
    n = max(1, n)
    return [l[i:i + n] for i in range(0, len(l), n)]

def read_file(fname):
    file=open(fname, mode='rb')
    content=file.read()
    file.close()
    return content

def decrypt_byte(i, k):
    return chr ((ord(i)-k) % 256)

def decrypt(buf):
    return "".join(decrypt_byte(buf[i], i) for i in range(16))

fout=open(sys.argv[2], mode='wb')

prev=IV
content=read_file(sys.argv[1])
tmp=chunks(content, 16)
for c in tmp:
    new_c=decrypt(c)
    p=xor_strings (new_c, prev)
    prev=xor_strings(c, p)
    fout.write(p)
fout.close()
\end{lstlisting}

(Source code can be downloaded \href{\GitHubBlobMasterURL/examples/simple_exec_crypto/files/decrypt2.py}{here}.)

Let's check resulting file:

\lstinputlisting{examples/simple_exec_crypto/objdump_result.txt}

Yes, this is seems correctly disassembled piece of x86 code.
The whole decryped file can be downloaded \href{\GitHubBlobMasterURL/examples/simple_exec_crypto/files/decrypted.bin}{here}.

In fact, this is text section from regedit.exe from Windows 7.
But this example is based on a real case I encountered, so just executable is different (and key), algorithm is the same.

\subsection{Other ideas to consider}

What if I would fail with such simple frequency analysis?
There are other ideas on how to measure correctness of decrypted/decompressed x86 code:

\begin{itemize}

\item Many modern compilers aligns functions on 0x10 border.
So the space left before is filled with NOPs (0x90) or other NOP instructions with known opcodes: \myref{sec:npad}.

\item Perhaps, the most frequent pattern in any assembly language is function call:\\
\TT{PUSH chain / CALL / ADD ESP, X}.
This sequence can easily detected and found.
I've even gathered statistics about average number of function arguments: \myref{args_stat}.
(Hence, this is average length of PUSH chain.)

\end{itemize}

Read more about incorrectly/correctly disassembled code: \myref{ISA_detect}.
}%
\FR{\mysection{Une fonction vide: redux}

Revenons sur l'exemple de la fonction vide \myref{empty_func}.
Maintenant que nous connaissons le prologue et l'épilogue de fonction, ceci est
une fonction vide \myref{lst:empty_func} compilée par GCC sans optimisation:

\lstinputlisting[caption=GCC 8.2 x64 \NonOptimizing (\assemblyOutput),style=customasmx86]{patterns/016_empty_redux/1.s}

C'est \INS{RET}, mais le prologue et l'épilogue de la fonction, probablement, n'ont
pas été optimisés et laissés tels quels.
\INS{NOP} semble être un autre artefact du compilateur.
De toutes façons, la seule instruction effective ici est \INS{RET}.
Toutes les autres instructions peuvent être supprimées (ou optimisées).

}
}
\DE{\EN{% TODO translate
\mysection{Breaking simple executable cryptor}

I've got an executable file which is encrypted by relatively simple encryption.
\href{\GitHubBlobMasterURL/examples/simple_exec_crypto/files/cipher.bin}{Here is it} (only executable section is left here).

First, all encryption function does is just adds number of position in buffer to the byte.
Here is how this can be encoded in Python:

\begin{lstlisting}[caption=Python script,style=custompy]
#!/usr/bin/env python
def e(i, k):
    return chr ((ord(i)+k) % 256)

def encrypt(buf):
    return e(buf[0], 0)+ e(buf[1], 1)+ e(buf[2], 2) + e(buf[3], 3)+ e(buf[4], 4)+ e(buf[5], 5)+ e(buf[6], 6)+ e(buf[7], 7)+
           e(buf[8], 8)+ e(buf[9], 9)+ e(buf[10], 10)+ e(buf[11], 11)+ e(buf[12], 12)+ e(buf[13], 13)+ e(buf[14], 14)+ e(buf[15], 15)
\end{lstlisting}

Hence, if you encrypt buffer with 16 zeros, you'll get \emph{0, 1, 2, 3 ... 12, 13, 14, 15}.

\myindex{Propagating Cipher Block Chaining}
Propagating Cipher Block Chaining (PCBC) is also used, here is how it works:

\begin{figure}[H]
\centering
\myincludegraphics{examples/simple_exec_crypto/601px-PCBC_encryption.png}
\caption{Propagating Cipher Block Chaining encryption (image is taken from Wikipedia article)}
\end{figure}

The problem is that it's too boring to recover IV (Initialization Vector) each time.
Brute-force is also not an option, because IV is too long (16 bytes).
Let's see, if it's possible to recover IV for arbitrary encrypted executable file?

Let's try simple frequency analysis.
This is 32-bit x86 executable code, so let's gather statistics about most frequent bytes and opcodes.
I tried huge oracle.exe file from Oracle RDBMS version 11.2 for windows x86 and I've found that the most frequent byte (no surprise) is zero (~10\%).
The next most frequent byte is (again, no surprise) 0xFF (~5\%).
The next is 0x8B (~5\%).

\myindex{x86!\Instructions!MOV}
0x8B is opcode for \INS{MOV}, this is indeed one of the most busy x86 instructions.
Now what about popularity of zero byte?
If compiler needs to encode value bigger than 127, it has to use 32-bit displacement instead of 8-bit one, but large values are very rare,
so it is padded by zeros.
\myindex{x86!\Instructions!LEA}
\myindex{x86!\Instructions!PUSH}
\myindex{x86!\Instructions!CALL}
This is at least in \INS{LEA}, \INS{MOV}, \INS{PUSH}, \INS{CALL}.

For example:

\begin{lstlisting}[style=customasmx86]
8D B0 28 01 00 00                 lea     esi, [eax+128h]
8D BF 40 38 00 00                 lea     edi, [edi+3840h]
\end{lstlisting}

Displacements bigger than 127 are very popular, but they are rarely exceeds 0x10000
(indeed, such large memory buffers/structures are also rare).

Same story with \INS{MOV}, large constants are rare, the most heavily used are 0, 1, 10, 100, $2^n$, and so on.
Compiler has to pad small constants by zeros to represent them as 32-bit values:

\begin{lstlisting}[style=customasmx86]
BF 02 00 00 00                    mov     edi, 2
BF 01 00 00 00                    mov     edi, 1
\end{lstlisting}

Now about 00 and FF bytes combined: jumps (including conditional) and calls can pass execution flow forward or backwards, but very often,
within the limits of the current executable module.
If forward, displacement is not very big and also padded with zeros.
If backwards, displacement is represented as negative value, so padded with FF bytes.
For example, transfer execution flow forward:

\begin{lstlisting}[style=customasmx86]
E8 43 0C 00 00                    call    _function1
E8 5C 00 00 00                    call    _function2
0F 84 F0 0A 00 00                 jz      loc_4F09A0
0F 84 EB 00 00 00                 jz      loc_4EFBB8
\end{lstlisting}

Backwards:

\begin{lstlisting}[style=customasmx86]
E8 79 0C FE FF                    call    _function1
E8 F4 16 FF FF                    call    _function2
0F 84 F8 FB FF FF                 jz      loc_8212BC
0F 84 06 FD FF FF                 jz      loc_FF1E7D
\end{lstlisting}

FF byte is also very often occurred in negative displacements like these:

\begin{lstlisting}[style=customasmx86]
8D 85 1E FF FF FF                 lea     eax, [ebp-0E2h]
8D 95 F8 5C FF FF                 lea     edx, [ebp-0A308h]
\end{lstlisting}

So far so good. Now we have to try various 16-byte keys, decrypt executable section and measure how often 00, FF and 8B bytes are occurred.
Let's also keep in sight how PCBC decryption works:

\begin{figure}[H]
\centering
\myincludegraphics{examples/simple_exec_crypto/640px-PCBC_decryption.png}
\caption{Propagating Cipher Block Chaining decryption (image is taken from Wikipedia article)}
\end{figure}

The good news is that we don't really have to decrypt whole piece of data, but only slice by slice, this is exactly how I did in my previous example: \myref{XOR_mask_2}.

Now I'm trying all possible bytes (0..255) for each byte in key and just pick the byte producing maximal amount of 00/FF/8B bytes in a decrypted slice:

\begin{lstlisting}[style=custompy]
#!/usr/bin/env python
import sys, hexdump, array, string, operator

KEY_LEN=16

def chunks(l, n):
    # split n by l-byte chunks
    # https://stackoverflow.com/q/312443
    n = max(1, n)
    return [l[i:i + n] for i in range(0, len(l), n)]

def read_file(fname):
    file=open(fname, mode='rb')
    content=file.read()
    file.close()
    return content

def decrypt_byte (c, key):
    return chr((ord(c)-key) % 256)

def XOR_PCBC_step (IV, buf, k):
    prev=IV
    rt=""
    for c in buf:
	new_c=decrypt_byte(c, k)
        plain=chr(ord(new_c)^ord(prev))
	prev=chr(ord(c)^ord(plain))
	rt=rt+plain
    return rt

each_Nth_byte=[""]*KEY_LEN

content=read_file(sys.argv[1])
# split input by 16-byte chunks:
all_chunks=chunks(content, KEY_LEN)
for c in all_chunks:
    for i in range(KEY_LEN):
        each_Nth_byte[i]=each_Nth_byte[i] + c[i]

# try each byte of key
for N in range(KEY_LEN):
    print "N=", N
    stat={}
    for i in range(256):
        tmp_key=chr(i)
	tmp=XOR_PCBC_step(tmp_key,each_Nth_byte[N], N)
        # count 0, FFs and 8Bs in decrypted buffer:
	important_bytes=tmp.count('\x00')+tmp.count('\xFF')+tmp.count('\x8B')
	stat[i]=important_bytes
    sorted_stat = sorted(stat.iteritems(), key=operator.itemgetter(1), reverse=True)
    print sorted_stat[0]
\end{lstlisting}

(Source code can be downloaded \href{\GitHubBlobMasterURL/examples/simple_exec_crypto/files/decrypt.py}{here}.)

I run it and here is a key for which 00/FF/8B bytes presence in decrypted buffer is maximal:

\begin{lstlisting}
N= 0
(147, 1224)
N= 1
(94, 1327)
N= 2
(252, 1223)
N= 3
(218, 1266)
N= 4
(38, 1209)
N= 5
(192, 1378)
N= 6
(199, 1204)
N= 7
(213, 1332)
N= 8
(225, 1251)
N= 9
(112, 1223)
N= 10
(143, 1177)
N= 11
(108, 1286)
N= 12
(10, 1164)
N= 13
(3, 1271)
N= 14
(128, 1253)
N= 15
(232, 1330)
\end{lstlisting}

Let's write decryption utility with the key we got:

\begin{lstlisting}[style=custompy]
#!/usr/bin/env python
import sys, hexdump, array

def xor_strings(s,t):
    # \verb|https://en.wikipedia.org/wiki/XOR_cipher#Example_implementation|
    """xor two strings together"""
    return "".join(chr(ord(a)^ord(b)) for a,b in zip(s,t))

IV=array.array('B', [147, 94, 252, 218, 38, 192, 199, 213, 225, 112, 143, 108, 10, 3, 128, 232]).tostring()

def chunks(l, n):
    n = max(1, n)
    return [l[i:i + n] for i in range(0, len(l), n)]

def read_file(fname):
    file=open(fname, mode='rb')
    content=file.read()
    file.close()
    return content

def decrypt_byte(i, k):
    return chr ((ord(i)-k) % 256)

def decrypt(buf):
    return "".join(decrypt_byte(buf[i], i) for i in range(16))

fout=open(sys.argv[2], mode='wb')

prev=IV
content=read_file(sys.argv[1])
tmp=chunks(content, 16)
for c in tmp:
    new_c=decrypt(c)
    p=xor_strings (new_c, prev)
    prev=xor_strings(c, p)
    fout.write(p)
fout.close()
\end{lstlisting}

(Source code can be downloaded \href{\GitHubBlobMasterURL/examples/simple_exec_crypto/files/decrypt2.py}{here}.)

Let's check resulting file:

\lstinputlisting{examples/simple_exec_crypto/objdump_result.txt}

Yes, this is seems correctly disassembled piece of x86 code.
The whole decryped file can be downloaded \href{\GitHubBlobMasterURL/examples/simple_exec_crypto/files/decrypted.bin}{here}.

In fact, this is text section from regedit.exe from Windows 7.
But this example is based on a real case I encountered, so just executable is different (and key), algorithm is the same.

\subsection{Other ideas to consider}

What if I would fail with such simple frequency analysis?
There are other ideas on how to measure correctness of decrypted/decompressed x86 code:

\begin{itemize}

\item Many modern compilers aligns functions on 0x10 border.
So the space left before is filled with NOPs (0x90) or other NOP instructions with known opcodes: \myref{sec:npad}.

\item Perhaps, the most frequent pattern in any assembly language is function call:\\
\TT{PUSH chain / CALL / ADD ESP, X}.
This sequence can easily detected and found.
I've even gathered statistics about average number of function arguments: \myref{args_stat}.
(Hence, this is average length of PUSH chain.)

\end{itemize}

Read more about incorrectly/correctly disassembled code: \myref{ISA_detect}.
}%
\FR{\mysection{Une fonction vide: redux}

Revenons sur l'exemple de la fonction vide \myref{empty_func}.
Maintenant que nous connaissons le prologue et l'épilogue de fonction, ceci est
une fonction vide \myref{lst:empty_func} compilée par GCC sans optimisation:

\lstinputlisting[caption=GCC 8.2 x64 \NonOptimizing (\assemblyOutput),style=customasmx86]{patterns/016_empty_redux/1.s}

C'est \INS{RET}, mais le prologue et l'épilogue de la fonction, probablement, n'ont
pas été optimisés et laissés tels quels.
\INS{NOP} semble être un autre artefact du compilateur.
De toutes façons, la seule instruction effective ici est \INS{RET}.
Toutes les autres instructions peuvent être supprimées (ou optimisées).

}
}
\FR{\EN{% TODO translate
\mysection{Breaking simple executable cryptor}

I've got an executable file which is encrypted by relatively simple encryption.
\href{\GitHubBlobMasterURL/examples/simple_exec_crypto/files/cipher.bin}{Here is it} (only executable section is left here).

First, all encryption function does is just adds number of position in buffer to the byte.
Here is how this can be encoded in Python:

\begin{lstlisting}[caption=Python script,style=custompy]
#!/usr/bin/env python
def e(i, k):
    return chr ((ord(i)+k) % 256)

def encrypt(buf):
    return e(buf[0], 0)+ e(buf[1], 1)+ e(buf[2], 2) + e(buf[3], 3)+ e(buf[4], 4)+ e(buf[5], 5)+ e(buf[6], 6)+ e(buf[7], 7)+
           e(buf[8], 8)+ e(buf[9], 9)+ e(buf[10], 10)+ e(buf[11], 11)+ e(buf[12], 12)+ e(buf[13], 13)+ e(buf[14], 14)+ e(buf[15], 15)
\end{lstlisting}

Hence, if you encrypt buffer with 16 zeros, you'll get \emph{0, 1, 2, 3 ... 12, 13, 14, 15}.

\myindex{Propagating Cipher Block Chaining}
Propagating Cipher Block Chaining (PCBC) is also used, here is how it works:

\begin{figure}[H]
\centering
\myincludegraphics{examples/simple_exec_crypto/601px-PCBC_encryption.png}
\caption{Propagating Cipher Block Chaining encryption (image is taken from Wikipedia article)}
\end{figure}

The problem is that it's too boring to recover IV (Initialization Vector) each time.
Brute-force is also not an option, because IV is too long (16 bytes).
Let's see, if it's possible to recover IV for arbitrary encrypted executable file?

Let's try simple frequency analysis.
This is 32-bit x86 executable code, so let's gather statistics about most frequent bytes and opcodes.
I tried huge oracle.exe file from Oracle RDBMS version 11.2 for windows x86 and I've found that the most frequent byte (no surprise) is zero (~10\%).
The next most frequent byte is (again, no surprise) 0xFF (~5\%).
The next is 0x8B (~5\%).

\myindex{x86!\Instructions!MOV}
0x8B is opcode for \INS{MOV}, this is indeed one of the most busy x86 instructions.
Now what about popularity of zero byte?
If compiler needs to encode value bigger than 127, it has to use 32-bit displacement instead of 8-bit one, but large values are very rare,
so it is padded by zeros.
\myindex{x86!\Instructions!LEA}
\myindex{x86!\Instructions!PUSH}
\myindex{x86!\Instructions!CALL}
This is at least in \INS{LEA}, \INS{MOV}, \INS{PUSH}, \INS{CALL}.

For example:

\begin{lstlisting}[style=customasmx86]
8D B0 28 01 00 00                 lea     esi, [eax+128h]
8D BF 40 38 00 00                 lea     edi, [edi+3840h]
\end{lstlisting}

Displacements bigger than 127 are very popular, but they are rarely exceeds 0x10000
(indeed, such large memory buffers/structures are also rare).

Same story with \INS{MOV}, large constants are rare, the most heavily used are 0, 1, 10, 100, $2^n$, and so on.
Compiler has to pad small constants by zeros to represent them as 32-bit values:

\begin{lstlisting}[style=customasmx86]
BF 02 00 00 00                    mov     edi, 2
BF 01 00 00 00                    mov     edi, 1
\end{lstlisting}

Now about 00 and FF bytes combined: jumps (including conditional) and calls can pass execution flow forward or backwards, but very often,
within the limits of the current executable module.
If forward, displacement is not very big and also padded with zeros.
If backwards, displacement is represented as negative value, so padded with FF bytes.
For example, transfer execution flow forward:

\begin{lstlisting}[style=customasmx86]
E8 43 0C 00 00                    call    _function1
E8 5C 00 00 00                    call    _function2
0F 84 F0 0A 00 00                 jz      loc_4F09A0
0F 84 EB 00 00 00                 jz      loc_4EFBB8
\end{lstlisting}

Backwards:

\begin{lstlisting}[style=customasmx86]
E8 79 0C FE FF                    call    _function1
E8 F4 16 FF FF                    call    _function2
0F 84 F8 FB FF FF                 jz      loc_8212BC
0F 84 06 FD FF FF                 jz      loc_FF1E7D
\end{lstlisting}

FF byte is also very often occurred in negative displacements like these:

\begin{lstlisting}[style=customasmx86]
8D 85 1E FF FF FF                 lea     eax, [ebp-0E2h]
8D 95 F8 5C FF FF                 lea     edx, [ebp-0A308h]
\end{lstlisting}

So far so good. Now we have to try various 16-byte keys, decrypt executable section and measure how often 00, FF and 8B bytes are occurred.
Let's also keep in sight how PCBC decryption works:

\begin{figure}[H]
\centering
\myincludegraphics{examples/simple_exec_crypto/640px-PCBC_decryption.png}
\caption{Propagating Cipher Block Chaining decryption (image is taken from Wikipedia article)}
\end{figure}

The good news is that we don't really have to decrypt whole piece of data, but only slice by slice, this is exactly how I did in my previous example: \myref{XOR_mask_2}.

Now I'm trying all possible bytes (0..255) for each byte in key and just pick the byte producing maximal amount of 00/FF/8B bytes in a decrypted slice:

\begin{lstlisting}[style=custompy]
#!/usr/bin/env python
import sys, hexdump, array, string, operator

KEY_LEN=16

def chunks(l, n):
    # split n by l-byte chunks
    # https://stackoverflow.com/q/312443
    n = max(1, n)
    return [l[i:i + n] for i in range(0, len(l), n)]

def read_file(fname):
    file=open(fname, mode='rb')
    content=file.read()
    file.close()
    return content

def decrypt_byte (c, key):
    return chr((ord(c)-key) % 256)

def XOR_PCBC_step (IV, buf, k):
    prev=IV
    rt=""
    for c in buf:
	new_c=decrypt_byte(c, k)
        plain=chr(ord(new_c)^ord(prev))
	prev=chr(ord(c)^ord(plain))
	rt=rt+plain
    return rt

each_Nth_byte=[""]*KEY_LEN

content=read_file(sys.argv[1])
# split input by 16-byte chunks:
all_chunks=chunks(content, KEY_LEN)
for c in all_chunks:
    for i in range(KEY_LEN):
        each_Nth_byte[i]=each_Nth_byte[i] + c[i]

# try each byte of key
for N in range(KEY_LEN):
    print "N=", N
    stat={}
    for i in range(256):
        tmp_key=chr(i)
	tmp=XOR_PCBC_step(tmp_key,each_Nth_byte[N], N)
        # count 0, FFs and 8Bs in decrypted buffer:
	important_bytes=tmp.count('\x00')+tmp.count('\xFF')+tmp.count('\x8B')
	stat[i]=important_bytes
    sorted_stat = sorted(stat.iteritems(), key=operator.itemgetter(1), reverse=True)
    print sorted_stat[0]
\end{lstlisting}

(Source code can be downloaded \href{\GitHubBlobMasterURL/examples/simple_exec_crypto/files/decrypt.py}{here}.)

I run it and here is a key for which 00/FF/8B bytes presence in decrypted buffer is maximal:

\begin{lstlisting}
N= 0
(147, 1224)
N= 1
(94, 1327)
N= 2
(252, 1223)
N= 3
(218, 1266)
N= 4
(38, 1209)
N= 5
(192, 1378)
N= 6
(199, 1204)
N= 7
(213, 1332)
N= 8
(225, 1251)
N= 9
(112, 1223)
N= 10
(143, 1177)
N= 11
(108, 1286)
N= 12
(10, 1164)
N= 13
(3, 1271)
N= 14
(128, 1253)
N= 15
(232, 1330)
\end{lstlisting}

Let's write decryption utility with the key we got:

\begin{lstlisting}[style=custompy]
#!/usr/bin/env python
import sys, hexdump, array

def xor_strings(s,t):
    # \verb|https://en.wikipedia.org/wiki/XOR_cipher#Example_implementation|
    """xor two strings together"""
    return "".join(chr(ord(a)^ord(b)) for a,b in zip(s,t))

IV=array.array('B', [147, 94, 252, 218, 38, 192, 199, 213, 225, 112, 143, 108, 10, 3, 128, 232]).tostring()

def chunks(l, n):
    n = max(1, n)
    return [l[i:i + n] for i in range(0, len(l), n)]

def read_file(fname):
    file=open(fname, mode='rb')
    content=file.read()
    file.close()
    return content

def decrypt_byte(i, k):
    return chr ((ord(i)-k) % 256)

def decrypt(buf):
    return "".join(decrypt_byte(buf[i], i) for i in range(16))

fout=open(sys.argv[2], mode='wb')

prev=IV
content=read_file(sys.argv[1])
tmp=chunks(content, 16)
for c in tmp:
    new_c=decrypt(c)
    p=xor_strings (new_c, prev)
    prev=xor_strings(c, p)
    fout.write(p)
fout.close()
\end{lstlisting}

(Source code can be downloaded \href{\GitHubBlobMasterURL/examples/simple_exec_crypto/files/decrypt2.py}{here}.)

Let's check resulting file:

\lstinputlisting{examples/simple_exec_crypto/objdump_result.txt}

Yes, this is seems correctly disassembled piece of x86 code.
The whole decryped file can be downloaded \href{\GitHubBlobMasterURL/examples/simple_exec_crypto/files/decrypted.bin}{here}.

In fact, this is text section from regedit.exe from Windows 7.
But this example is based on a real case I encountered, so just executable is different (and key), algorithm is the same.

\subsection{Other ideas to consider}

What if I would fail with such simple frequency analysis?
There are other ideas on how to measure correctness of decrypted/decompressed x86 code:

\begin{itemize}

\item Many modern compilers aligns functions on 0x10 border.
So the space left before is filled with NOPs (0x90) or other NOP instructions with known opcodes: \myref{sec:npad}.

\item Perhaps, the most frequent pattern in any assembly language is function call:\\
\TT{PUSH chain / CALL / ADD ESP, X}.
This sequence can easily detected and found.
I've even gathered statistics about average number of function arguments: \myref{args_stat}.
(Hence, this is average length of PUSH chain.)

\end{itemize}

Read more about incorrectly/correctly disassembled code: \myref{ISA_detect}.
}%
\FR{\mysection{Une fonction vide: redux}

Revenons sur l'exemple de la fonction vide \myref{empty_func}.
Maintenant que nous connaissons le prologue et l'épilogue de fonction, ceci est
une fonction vide \myref{lst:empty_func} compilée par GCC sans optimisation:

\lstinputlisting[caption=GCC 8.2 x64 \NonOptimizing (\assemblyOutput),style=customasmx86]{patterns/016_empty_redux/1.s}

C'est \INS{RET}, mais le prologue et l'épilogue de la fonction, probablement, n'ont
pas été optimisés et laissés tels quels.
\INS{NOP} semble être un autre artefact du compilateur.
De toutes façons, la seule instruction effective ici est \INS{RET}.
Toutes les autres instructions peuvent être supprimées (ou optimisées).

}
}

\EN{\chapter{Finding important/interesting stuff in the code}

Minimalism it is not a prominent feature of modern software.

\myindex{\Cpp!STL}

But not because the programmers are writing a lot, but because a lot of libraries are commonly linked statically
to executable files.
If all external libraries were shifted into an external DLL files, the world would be different.
(Another reason for C++ are the \ac{STL} and other template libraries.)

\newcommand{\FOOTNOTEBOOST}{\footnote{\url{http://go.yurichev.com/17036}}}
\newcommand{\FOOTNOTELIBPNG}{\footnote{\url{http://go.yurichev.com/17037}}}

Thus, it is very important to determine the origin of a function, if it is from standard library or 
well-known library (like Boost\FOOTNOTEBOOST, libpng\FOOTNOTELIBPNG),
or if it is related to what we are trying to find in the code.

It is just absurd to rewrite all code in \CCpp to find what we're looking for.

One of the primary tasks of a reverse engineer is to find quickly the code he/she needs, and what is not that important.

\myindex{\GrepUsage}

The \IDA disassembler allow us to search among text strings, byte sequences and constants.
It is even possible to export the code to .lst or .asm text files and then use \TT{grep}, \TT{awk}, etc.

When you try to understand what some code is doing, this easily could be some open-source library like libpng.
So when you see some constants or text strings which look familiar, it is always worth to \emph{google} them.
And if you find the opensource project where they are used, 
then it's enough just to compare the functions.
It may solve some part of the problem.

For example, if a program uses XML files, the first step may be determining which
XML library is used for processing, since the standard (or well-known) libraries are usually used
instead of self-made one.

\myindex{SAP}
\myindex{Windows!PDB}

For example, the author of these lines once tried to understand how the compression/decompression of network packets works in SAP 6.0. 
It is a huge software, but a detailed .\gls{PDB} with debugging information is present, 
and that is convenient.
He finally came to the idea that one of the functions, that was called \emph{CsDecomprLZC}, was doing the decompression of network packets.
Immediately he tried to google its name and he quickly found the function was used in MaxDB
(it is an open-source SAP project) \footnote{More about it in relevant section~(\myref{sec:SAPGUI})}.

\url{http://www.google.com/search?q=CsDecomprLZC}

Astoundingly, MaxDB and SAP 6.0 software shared likewise code for the compression/decompression of network packets.

\input{digging_into_code/identification/exec_EN}

% binary files might be also here

\mysection{Communication with outer world (function level)}
It's often advisable to track function arguments and return values in debugger or \ac{DBI}.
For example, the author once tried to understand meaning of some obscure function, which happens to be incorrectly
implemented bubble sort\footnote{\url{https://yurichev.com/blog/weird_sort_KLEE/}}.
(It worked correctly, but slower.)
Meanwhile, watching inputs and outputs of this function helps instantly to understand what it does.

Often, when you see division by multiplication (\myref{sec:divisionbymult}),
but forgot all details about its mechanics, you can just observe input
and output and quickly find divisor.

% sections:
\input{digging_into_code/communication_win32_EN}
\input{digging_into_code/strings_EN}
\input{digging_into_code/assert_EN}
\mysection{Constants}

Humans, including programmers, often use round numbers like 10, 100, 1000, 
in real life as well as in the code.

The practicing reverse engineer usually know them well in hexadecimal representation:
10=0xA, 100=0x64, 1000=0x3E8, 10000=0x2710.

The constants \TT{0xAAAAAAAA} (0b10101010101010101010101010101010) and \\
\TT{0x55555555} (0b01010101010101010101010101010101)  are also popular---those
are composed of alternating bits.

That may help to distinguish some signal from a signal where all bits are turned on (0b1111 \dots) or off (0b0000 \dots).
For example, the \TT{0x55AA} constant
is used at least in the boot sector, \ac{MBR}, 
and in the \ac{ROM} of IBM-compatible extension cards.

Some algorithms, especially cryptographical ones use distinct constants, which are easy to find
in code using \IDA.

\myindex{MD5}

For example, the MD5 algorithm initializes its own internal variables like this:

\begin{verbatim}
var int h0 := 0x67452301
var int h1 := 0xEFCDAB89
var int h2 := 0x98BADCFE
var int h3 := 0x10325476
\end{verbatim}

If you find these four constants used in the code in a row, it is highly probable that this function is related to MD5.

\par Another example are the CRC16/CRC32 algorithms, 
whose calculation algorithms often use precomputed tables like this one:

\begin{lstlisting}[caption=linux/lib/crc16.c,style=customc]
/** CRC table for the CRC-16. The poly is 0x8005 (x^16 + x^15 + x^2 + 1) */
u16 const crc16_table[256] = {
	0x0000, 0xC0C1, 0xC181, 0x0140, 0xC301, 0x03C0, 0x0280, 0xC241,
	0xC601, 0x06C0, 0x0780, 0xC741, 0x0500, 0xC5C1, 0xC481, 0x0440,
	0xCC01, 0x0CC0, 0x0D80, 0xCD41, 0x0F00, 0xCFC1, 0xCE81, 0x0E40,
	...
\end{lstlisting}

See also the precomputed table for CRC32: \myref{sec:CRC32}.

In tableless CRC algorithms well-known polynomials are used, for example, 0xEDB88320 for CRC32.

\subsection{Magic numbers}
\label{magic_numbers}

A lot of file formats define a standard file header where a \emph{magic number(s)} is used, single one or even several.

\myindex{MS-DOS}

For example, all Win32 and MS-DOS executables start with the two characters \q{MZ}.

\myindex{MIDI}

At the beginning of a MIDI file the \q{MThd} signature must be present. 
If we have a program which uses MIDI files for something,
it's very likely that it must check the file for validity by checking at least the first 4 bytes.

This could be done like this:
(\emph{buf} points to the beginning of the loaded file in memory)

\begin{lstlisting}[style=customasmx86]
cmp [buf], 0x6468544D ; "MThd"
jnz _error_not_a_MIDI_file
\end{lstlisting}

\myindex{\CStandardLibrary!memcmp()}
\myindex{x86!\Instructions!CMPSB}

\dots or by calling a function for comparing memory blocks like \TT{memcmp()} or any other equivalent code
up to a \TT{CMPSB} (\myref{REPE_CMPSx}) instruction.

When you find such point you already can say where the loading of the MIDI file starts,
also, we could see the location
of the buffer with the contents of the MIDI file, what is used from the buffer, and how.

\subsubsection{Dates}

\myindex{UFS2}
\myindex{FreeBSD}
\myindex{HASP}

Often, one may encounter number like \TT{0x19870116}, which is clearly looks like a date (year 1987, 1th month (January), 16th day).
This may be someone's birthday date (a programmer, his/her relative, child), or some other important date.
The date may also be written in a reverse order, like \TT{0x16011987}.
American-style dates are also popular, like \TT{0x01161987}.

Well-known example is \TT{0x19540119} (magic number used in UFS2 superblock structure), which is a birthday date of Marshall Kirk McKusick, prominent FreeBSD contributor.

\myindex{Stuxnet}
Stuxnet uses the number ``19790509'' (not as 32-bit number, but as string, though), and this led to speculation
that the malware is connected to Israel%
\footnote{This is a date of execution of Habib Elghanian, persian jew.}.

Also, numbers like those are very popular in amateur-grade cryptography, for example, excerpt from the \emph{secret function} internals from HASP3 dongle
\footnote{\url{https://web.archive.org/web/20160311231616/http://www.woodmann.com/fravia/bayu3.htm}}:

\begin{lstlisting}[style=customc]
void xor_pwd(void) 
{ 
	int i; 
	
	pwd^=0x09071966;
	for(i=0;i<8;i++) 
	{ 
		al_buf[i]= pwd & 7; pwd = pwd >> 3; 
	} 
};

void emulate_func2(unsigned short seed)
{ 
	int i, j; 
	for(i=0;i<8;i++) 
	{ 
		ch[i] = 0; 
		
		for(j=0;j<8;j++)
		{ 
			seed *= 0x1989; 
			seed += 5; 
			ch[i] |= (tab[(seed>>9)&0x3f]) << (7-j); 
		}
	} 
}
\end{lstlisting}

\subsubsection{DHCP}

This applies to network protocols as well.
For example, the DHCP protocol's network packets contains the so-called \emph{magic cookie}: \TT{0x63538263}.
Any code that generates DHCP packets somewhere must embed this constant into the packet.
If we find it in the code we may find where this happens and, not only that.
Any program which can receive DHCP packet must verify the \emph{magic cookie}, comparing it with the constant.

For example, let's take the dhcpcore.dll file from Windows 7 x64 and search for the constant.
And we can find it, twice:
it seems that the constant is used in two functions with descriptive names\\
\TT{DhcpExtractOptionsForValidation()} and \TT{DhcpExtractFullOptions()}:

\begin{lstlisting}[caption=dhcpcore.dll (Windows 7 x64),style=customasmx86]
.rdata:000007FF6483CBE8 dword_7FF6483CBE8 dd 63538263h          ; DATA XREF: DhcpExtractOptionsForValidation+79
.rdata:000007FF6483CBEC dword_7FF6483CBEC dd 63538263h          ; DATA XREF: DhcpExtractFullOptions+97
\end{lstlisting}

And here are the places where these constants are accessed:

\begin{lstlisting}[caption=dhcpcore.dll (Windows 7 x64),style=customasmx86]
.text:000007FF6480875F  mov     eax, [rsi]
.text:000007FF64808761  cmp     eax, cs:dword_7FF6483CBE8
.text:000007FF64808767  jnz     loc_7FF64817179
\end{lstlisting}

And:

\begin{lstlisting}[caption=dhcpcore.dll (Windows 7 x64),style=customasmx86]
.text:000007FF648082C7  mov     eax, [r12]
.text:000007FF648082CB  cmp     eax, cs:dword_7FF6483CBEC
.text:000007FF648082D1  jnz     loc_7FF648173AF
\end{lstlisting}

\subsection{Specific constants}

Sometimes, there is a specific constant for some type of code.
For example, the author once dug into a code, where number 12 was encountered suspiciously often.
Size of many arrays is 12, or multiple of 12 (24, etc).
As it turned out, that code takes 12-channel audio file at input and process it.

And vice versa: for example, if a program works with text field which has length of 120 bytes,
there has to be a constant 120 or 119 somewhere in the code.
If UTF-16 is used, then $2 \cdot 120$.
If a code works with network packets of fixed size, it's good idea to search for this constant in the code as well.

This is also true for amateur cryptography (license keys, etc).
If encrypted block has size of $n$ bytes, you may want to try to find occurences of this number throughout the code.
Also, if you see a piece of code which is been repeated $n$ times in loop during execution,
this may be encryption/decryption routine.

\subsection{Searching for constants}

It is easy in \IDA: Alt-B or Alt-I.
\myindex{binary grep}
And for searching for a constant in a big pile of files, or for searching in non-executable files,
there is a small utility called \emph{binary grep}\footnote{\BGREPURL}.


\input{digging_into_code/instructions_EN}
\mysection{Suspicious code patterns}

\subsection{XOR instructions}
\myindex{x86!\Instructions!XOR}

Instructions like \TT{XOR op, op} (for example, \TT{XOR EAX, EAX}) 
are usually used for setting the register value
to zero, but if the operands are different, the \q{exclusive or} operation
is executed.

This operation is rare in common programming, but widespread in cryptography,
including amateur one.
It's especially suspicious if the
second operand is a big number.

This may point to encrypting/decrypting, checksum computing, etc.\\
\\

One exception to this observation worth noting is the \q{canary} (\myref{subsec:BO_protection}). 
Its generation and checking are often done using the \XOR instruction. \\
\\
\myindex{AWK}

This AWK script can be used for processing \IDA{} listing (.lst) files:

\lstinputlisting{digging_into_code/awk.sh}

It is also worth noting that this kind of script can also match incorrectly disassembled code 
(\myref{sec:incorrectly_disasmed_code}).

\subsection{Hand-written assembly code}

\myindex{Function prologue}
\myindex{Function epilogue}
\myindex{x86!\Instructions!LOOP}
\myindex{x86!\Instructions!RCL}

Modern compilers do not emit the \TT{LOOP} and \TT{RCL} instructions.
On the other hand, these instructions are well-known to coders who like to code directly in assembly language.
If you spot these, it can be said that there is a high probability that this fragment of code was hand-written.
Such instructions are marked as (M) in the instructions list in this appendix: \myref{sec:x86_instructions}.

\par
Also the function prologue/epilogue are not commonly present in hand-written assembly.

\par
Commonly there is no fixed system for passing arguments to functions in the hand-written code.

\par
Example from the Windows 2003 kernel 
(ntoskrnl.exe file):

\lstinputlisting[style=customasmx86]{digging_into_code/ntoskrnl.lst}

Indeed, if we look in the 
\ac{WRK} v1.2 source code, this code
can be found easily in file \\
\emph{WRK-v1.2\textbackslash{}base\textbackslash{}ntos\textbackslash{}ke\textbackslash{}i386\textbackslash{}cpu.asm}.

\par 
As of \INS{RCL}, I could find it in ntoskrnl.exe file from Windows 2003 x86 (MS Visual C compiler).
It is occurred only once, in \TT{RtlExtendedLargeIntegerDivide()} function, and this might be inline assembler code case.


\input{digging_into_code/magic_numbers_tracing_EN}
\input{digging_into_code/loops_EN}
% TODO move section...

\subsection{Some binary file patterns}

All examples here were prepared on the Windows with active code page 437
in console.
Binary files internally may look visually different if another code page is set.

\clearpage
\subsubsection{Arrays}

Sometimes, we can clearly spot an array of 16/32/64-bit values visually, in hex editor.

Here is an example of array of 16-bit values.
We see that the first byte in pair is 7 or 8, and the second looks random:

\begin{figure}[H]
\centering
\myincludegraphics{digging_into_code/binary/16bit_array.png}
\caption{FAR: array of 16-bit values}
\end{figure}

I used a file containing 12-channel signal digitized using 16-bit \ac{ADC}.

\clearpage
\myindex{MIPS}
\par And here is an example of very typical MIPS code.

As we may recall, every MIPS (and also ARM in ARM mode or ARM64) instruction has size of 32 bits (or 4 bytes), 
so such code is array of 32-bit values.

By looking at this screenshot, we may see some kind of pattern.

Vertical red lines are added for clarity:

\begin{figure}[H]
\centering
\myincludegraphics{digging_into_code/binary/typical_MIPS_code.png}
\caption{Hiew: very typical MIPS code}
\end{figure}

Another example of such pattern here is book: 
\myref{Oracle_SYM_files_example}.

\clearpage
\subsubsection{Sparse files}

This is sparse file with data scattered amidst almost empty file.
Each space character here is in fact zero byte (which is looks like space).
This is a file to program FPGA (Altera Stratix GX device).
Of course, files like these can be compressed easily, but formats like this one are very popular in scientific and engineering software where efficient access is important while compactness is not.

\begin{figure}[H]
\centering
\myincludegraphics{digging_into_code/binary/sparse_FPGA.png}
\caption{FAR: Sparse file}
\end{figure}

\clearpage
\subsubsection{Compressed file}

% FIXME \ref{} ->
This file is just some compressed archive.
It has relatively high entropy and visually looks just chaotic.
This is how compressed and/or encrypted files looks like.

\begin{figure}[H]
\centering
\myincludegraphics{digging_into_code/binary/compressed.png}
\caption{FAR: Compressed file}
\end{figure}

\clearpage
\subsubsection{\ac{CDFS}}

\ac{OS} installations are usually distributed as ISO files which are copies of CD/DVD discs.
Filesystem used is named \ac{CDFS}, here is you see file names mixed with some additional data.
This can be file sizes, pointers to another directories, file attributes, etc.
This is how typical filesystems may look internally.

\begin{figure}[H]
\centering
\myincludegraphics{digging_into_code/binary/cdfs.png}
\caption{FAR: ISO file: Ubuntu 15 installation \ac{CD}}
\end{figure}

\clearpage
\subsubsection{32-bit x86 executable code}

This is how 32-bit x86 executable code looks like.
It has not very high entropy, because some bytes occurred more often than others.

\begin{figure}[H]
\centering
\myincludegraphics{digging_into_code/binary/x86_32.png}
\caption{FAR: Executable 32-bit x86 code}
\end{figure}

% TODO: Read more about x86 statistics: \ref{}. % FIXME blog post about decryption...

\clearpage
\subsubsection{BMP graphics files}

% TODO: bitmap, bit, group of bits...

BMP files are not compressed, so each byte (or group of bytes) describes each pixel.
I've found this picture somewhere inside my installed Windows 8.1:

\begin{figure}[H]
\centering
\myincludegraphicsSmall{digging_into_code/binary/bmp.png}
\caption{Example picture}
\end{figure}

You see that this picture has some pixels which unlikely can be compressed very good (around center), 
but there are long one-color lines at top and bottom.
Indeed, lines like these also looks as lines during viewing the file:

\begin{figure}[H]
\centering
\myincludegraphics{digging_into_code/binary/bmp_FAR.png}
\caption{BMP file fragment}
\end{figure}


% FIXME comparison!
\subsection{Memory \q{snapshots} comparing}
\label{snapshots_comparing}

The technique of the straightforward comparison of two memory snapshots in order to see changes was often used to hack
8-bit computer games and for hacking \q{high score} files.

For example, if you had a loaded game on an 8-bit computer (there isn't much memory on these, but the game usually
consumes even less memory) and you know that you have now, let's say, 100 bullets, you can do a \q{snapshot}
of all memory and back it up to some place. Then shoot once, the bullet count goes to 99, do a second \q{snapshot}
and then compare both: it must be a byte somewhere which has been 100 at the beginning, and now it is 99.

Considering the fact that these 8-bit games were often written in assembly language and such variables were global,
it can be said for sure which address in memory has holding the bullet count. If you searched for all references to the
address in the disassembled game code, it was not very hard to find a piece of code \glslink{decrement}{decrementing} the bullet count,
then to write a \gls{NOP} instruction there, or a couple of \gls{NOP}-s, 
and then have a game with 100 bullets forever.
\myindex{BASIC!POKE}
Games on these 8-bit computers were commonly loaded at the constant
address, also, there were not much different versions of each game (commonly just one version was popular for a long span of time),
so enthusiastic gamers knew which bytes must be overwritten (using the BASIC's instruction \gls{POKE}) at which address in
order to hack it. This led to \q{cheat} lists that contained \gls{POKE} instructions, published in magazines related to
8-bit games. See also: \href{http://go.yurichev.com/17114}{wikipedia}.

\myindex{MS-DOS}

Likewise, it is easy to modify \q{high score} files, this does not work with just 8-bit games. Notice 
your score count and back up the file somewhere. When the \q{high score} count gets different, just compare the two files,
it can even be done with the DOS utility FC\footnote{MS-DOS utility for comparing binary files} (\q{high score} files
are often in binary form).

There will be a point where a couple of bytes are different and it is easy to see which ones are
holding the score number.
However, game developers are fully aware of such tricks and may defend the program against it.

Somewhat similar example in this book is: \myref{Millenium_DOS_game}.

% TODO: пример с какой-то простой игрушкой?

\subsubsection{A real story from 1999}

\myindex{ICQ}
There was a time of ICQ messenger's popularity, at least in ex-USSR countries.
The messenger had a peculiarity --- some users didn't want to share their online status with everyone.
And you had to ask an \emph{authorization} from that user.
That user could allow you seeing his/her status, or maybe not.

This is what the author of these lines did:

\begin{itemize}
\item Added a user.
\item A user appeared in a contact-list, in a ``wait for authorization'' section.
\item Closed ICQ.
\item Backed up the ICQ database.
\item Loaded ICQ again.
\item User \emph{authorized}.
\item Closed ICQ and compared two databases.
\end{itemize}

It turned out: two database differed by only one byte.
In the first version: \verb|RESU\x03|, in the second: \verb|RESU\x02|.
(``RESU'', presumably, means ``USER'', i.e., a header of a structure where all the information about user was stored.)
That means the information about authorization was stored not at the server, but at the client.
Presumably, 2/3 value reflected \emph{authorization} status.

\subsubsection{Windows registry}

It is also possible to compare the Windows registry before and after a program installation.

It is a very popular method of finding which registry elements are used by the program.
Perhaps, this is the reason why the \q{windows registry cleaner} shareware is so popular.

By the way, this is how to dump Windows registry to text files:

\begin{lstlisting}
reg export HKLM HKLM.reg
reg export HKCU HKCU.reg
reg export HKCR HKCR.reg
reg export HKU HKU.reg
reg export HKCC HKCC.reg
\end{lstlisting}

\myindex{UNIX!diff}
They can be compared using diff...

\subsubsection{Engineering software, CADs, etc}

If a software uses proprietary files, you can also investigate something here as well.
You save file.
Then you add a dot or line or another primitive.
Save file, compare.
Or move dot, save file, compare.

\subsubsection{Blink-comparator}

Comparison of files or memory snapshots remind us blink-comparator
\footnote{\url{http://go.yurichev.com/17348}}:
a device used by astronomers in past, intended to find moving celestial objects.

Blink-comparator allows to switch quickly between two photographies shot in different time,
so astronomer would spot the difference visually.

By the way, Pluto was discovered by blink-comparator in 1930.

\input{digging_into_code/ISA_detect_EN}

\mysection{Other things}

\subsection{General idea}

A reverse engineer should try to be in programmer's shoes as often as possible. 
To take his/her viewpoint and ask himself, how would one solve some task the specific case.

\subsection{Order of functions in binary code}

All functions located in a single .c or .cpp-file are compiled into corresponding object (.o) file.
Later, a linker puts all object files it needs together, not changing order of functions in them.
As a consequence, if you see two or more consecutive functions, it means, that they were placed together
in a single source code file (unless you're on border of two object files, of course.)
This means these functions have something in common, that they are from the same \ac{API} level, from the same library, etc.

\myindex{CryptoPP}
This is a real story from practice: once upon a time, the author searched for Twofish-related functions in
a program with CryptoPP library linked, especially encryption/decryption functions.\\
I found the \verb|Twofish::Base::UncheckedSetKey()| function, but not others.
After peeking into the \verb|twofish.cpp| source code
\footnote{\url{https://github.com/weidai11/cryptopp/blob/b613522794a7633aa2bd81932a98a0b0a51bc04f/twofish.cpp}}, it became clear that all functions are located in one module (\verb|twofish.cpp|).\\
So I tried all function that followed \verb|Twofish::Base::UncheckedSetKey()|---as it happened,\\
one was \verb|Twofish::Enc::ProcessAndXorBlock()|, another---\verb|Twofish::Dec::ProcessAndXorBlock()|.

\subsection{Tiny functions}

Tiny functions like empty functions (\myref{empty_func})
or function which returns just ``true'' (1) or ``false'' (0) (\myref{ret_val_func}) are very common,
and almost all decent compilers tend to put only one such function into resulting executable code even if there were several
similar functions in source code.
So, whenever you see a tiny function consisting just of \TT{mov eax, 1 / ret}
which is referenced (and can be called) from many places,
which are seems unconnected to each other, this may be a result of such optimization.%

\subsection{\Cpp}

\ac{RTTI}~(\myref{RTTI})-data may be also useful for \Cpp class identification.

\subsection{Crash on purpose}

Often you need to know, which function has been executed, and which is not.
You can use a debugger, but on exotic architectures there may not be the one, so easiest way is to put there an invalid opcode,
or something like \INS{INT3} (0xCC).
The crash would signal about the very fact this instruction has been executed.

Another example of crashing on purpose: \myref{dmalloc_KILL_PROCESS}.

}
\RU{\chapter{Поиск в коде того что нужно}

Современное ПО, в общем-то, минимализмом не отличается.

\myindex{\Cpp!STL}
Но не потому, что программисты слишком много пишут, 
а потому что к исполняемым файлам обыкновенно прикомпилируют все подряд библиотеки. 
Если бы все вспомогательные библиотеки всегда выносили во внешние DLL, мир был бы иным.
(Еще одна причина для Си++ --- \ac{STL} и прочие библиотеки шаблонов.)

\newcommand{\FOOTNOTEBOOST}{\footnote{\url{http://go.yurichev.com/17036}}}
\newcommand{\FOOTNOTELIBPNG}{\footnote{\url{http://go.yurichev.com/17037}}}

Таким образом, очень полезно сразу понимать, какая функция из стандартной библиотеки или 
более-менее известной (как Boost\FOOTNOTEBOOST, libpng\FOOTNOTELIBPNG), 
а какая --- имеет отношение к тому что мы пытаемся найти в коде.

Переписывать весь код на \CCpp, чтобы разобраться в нем, безусловно, не имеет никакого смысла.

Одна из важных задач reverse engineer-а это быстрый поиск в коде того что собственно его интересует, а что -- второстепенно.

\myindex{\GrepUsage}
Дизассемблер \IDA позволяет делать поиск как минимум строк, последовательностей байт, констант.
Можно даже сделать экспорт кода в текстовый файл .lst или .asm и затем натравить на него \TT{grep}, \TT{awk}, итд.

Когда вы пытаетесь понять, что делает тот или иной код, это запросто может быть какая-то 
опенсорсная библиотека вроде libpng. Поэтому, когда находите константы, или текстовые строки, которые 
выглядят явно знакомыми, всегда полезно их \emph{погуглить}.
А если вы найдете искомый опенсорсный проект где это используется, 
то тогда будет достаточно будет просто сравнить вашу функцию с ней. 
Это решит часть проблем.

К примеру, если программа использует какие-то XML-файлы, первым шагом может быть
установление, какая именно XML-библиотека для этого используется, ведь часто используется какая-то
стандартная (или очень известная) вместо самодельной.

\myindex{SAP}
\myindex{Windows!PDB}
К примеру, автор этих строк однажды пытался разобраться как происходит компрессия/декомпрессия сетевых пакетов в SAP 6.0. 
Это очень большая программа, но к ней идет подробный .\gls{PDB}-файл с отладочной информацией, и это очень удобно. 
Он в конце концов пришел к тому что одна из функций декомпрессирующая пакеты называется CsDecomprLZC(). 
Не сильно раздумывая, он решил погуглить и оказалось, что функция с таким же названием имеется в MaxDB
(это опен-сорсный проект SAP) \footnote{Больше об этом в соответствующей секции~(\myref{sec:SAPGUI})}.

\url{http://www.google.com/search?q=CsDecomprLZC}

Каково же было мое удивление, когда оказалось, что в MaxDB используется точно такой же алгоритм, 
скорее всего, с таким же исходником.

\input{digging_into_code/identification/exec_RU}
% binary files might be also here

\mysection{Связь с внешним миром (на уровне функции)}

Очень желательно следить за аргументами ф-ции и возвращаемыми значениями, в отладчике или \ac{DBI}.
Например, автор этих строк однажды пытался понять значение некоторой очень запутанной ф-ции, которая, как потом оказалось,
была неверно реализованной пузырьковой сортировкой\footnote{\url{https://yurichev.com/blog/weird_sort_KLEE/}}.
(Она работала правильно, но медленнее.)
В то же время, наблюдение за входами и выходами этой ф-ции помогает мгновенно понять, что она делает.

Часто, когда вы видите деление через умножение (\myref{sec:divisionbymult}),
но забыли все детали о том, как оно работает, вы можете просто наблюдать за входом и выходом, и так быстро найти делитель.

% sections:
\input{digging_into_code/communication_win32_RU}
\input{digging_into_code/strings_RU}
\input{digging_into_code/assert_RU}
\mysection{Константы}

Люди, включая программистов, часто используют круглые числа вроде 10, 100, 1000, в т.ч. и в коде.

Практикующие реверсеры, обычно, хорошо знают их в шестнадцатеричном представлении:
10=0xA, 100=0x64, 1000=0x3E8, 10000=0x2710.

Иногда попадаются константы \TT{0xAAAAAAAA} \\
(0b10101010101010101010101010101010) и
\TT{0x55555555} (0b01010101010101010101010101010101) --- это чередующиеся биты.
Это помогает отличить некоторый сигнал от сигнала где все биты включены (0b1111 \dots) или выключены (0b0000 \dots).

Например, константа \TT{0x55AA} используется как минимум в бут-секторе, \ac{MBR}, 
и в \ac{ROM} плат-расширений IBM-компьютеров.

Некоторые алгоритмы, особенно криптографические, используют хорошо различимые константы, 
которые при помощи \IDA легко находить в коде.

\myindex{MD5}

Например, алгоритм MD5 инициализирует свои внутренние переменные так:

\begin{verbatim}
var int h0 := 0x67452301
var int h1 := 0xEFCDAB89
var int h2 := 0x98BADCFE
var int h3 := 0x10325476
\end{verbatim}

Если в коде найти использование этих четырех констант подряд --- очень высокая вероятность что эта функция имеет отношение к MD5.

\par
Еще такой пример это алгоритмы CRC16/CRC32, часто, алгоритмы вычисления контрольной суммы по CRC 
используют заранее заполненные таблицы, вроде:

\begin{lstlisting}[caption=linux/lib/crc16.c,style=customc]
/** CRC table for the CRC-16. The poly is 0x8005 (x^16 + x^15 + x^2 + 1) */
u16 const crc16_table[256] = {
	0x0000, 0xC0C1, 0xC181, 0x0140, 0xC301, 0x03C0, 0x0280, 0xC241,
	0xC601, 0x06C0, 0x0780, 0xC741, 0x0500, 0xC5C1, 0xC481, 0x0440,
	0xCC01, 0x0CC0, 0x0D80, 0xCD41, 0x0F00, 0xCFC1, 0xCE81, 0x0E40,
	...
\end{lstlisting}

См. также таблицу CRC32: \myref{sec:CRC32}.

В бестабличных алгоритмах CRC используются хорошо известные полиномы, например 0xEDB88320 для CRC32.

\subsection{Магические числа}
\label{magic_numbers}

Немало форматов файлов определяет стандартный заголовок файла где используются \emph{магическое число} (magic number), один или даже несколько.

\myindex{MS-DOS}
Скажем, все исполняемые файлы для Win32 и MS-DOS начинаются с двух символов \q{MZ}.

\myindex{MIDI}
В начале MIDI-файла должно быть \q{MThd}. Если у нас есть использующая для чего-нибудь MIDI-файлы программа,
наверняка она будет проверять MIDI-файлы на правильность хотя бы проверяя первые 4 байта.

Это можно сделать при помощи:
(\emph{buf} указывает на начало загруженного в память файла)

\begin{lstlisting}[style=customasmx86]
cmp [buf], 0x6468544D ; "MThd"
jnz _error_not_a_MIDI_file
\end{lstlisting}

\myindex{\CStandardLibrary!memcmp()}
\myindex{x86!\Instructions!CMPSB}
\dots либо вызвав функцию сравнения блоков памяти \TT{memcmp()} или любой аналогичный код, 
вплоть до инструкции \TT{CMPSB} (\myref{REPE_CMPSx}).

Найдя такое место мы получаем как минимум информацию о том, где начинается загрузка MIDI-файла, во-вторых, 
мы можем увидеть где располагается буфер с содержимым файла, и что еще оттуда берется, и как используется.

\subsubsection{Даты}

\myindex{UFS2}
\myindex{FreeBSD}
\myindex{HASP}

Часто, можно встретить число вроде \TT{0x19870116}, которое явно выглядит как дата (1987-й год, 1-й месяц (январь), 16-й день).
Это может быть чей-то день рождения (программиста, его/её родственника, ребенка), либо какая-то другая важная дата.
Дата может быть записана и в другом порядке, например \TT{0x16011987}.
Даты в американском стиле также популярны, например \TT{0x01161987}.

Известный пример это \TT{0x19540119} (магическое число используемое в структуре суперблока UFS2), это день рождения Маршала Кирка МакКузика, видного разработчика FreeBSD.

\myindex{Stuxnet}
В Stuxnet используется число ``19790509'' (хотя и не как 32-битное число, а как строка), и это привело к догадкам,
что этот зловред связан с Израелем\footnote{Это дата казни персидского еврея Habib Elghanian-а}.

Также, числа вроде таких очень популярны в любительской криптографии, например, это отрывок из внутренностей \emph{секретной функции} донглы HASP3
\footnote{\url{https://web.archive.org/web/20160311231616/http://www.woodmann.com/fravia/bayu3.htm}}:

\begin{lstlisting}[style=customc]
void xor_pwd(void) 
{ 
	int i; 
	
	pwd^=0x09071966;
	for(i=0;i<8;i++) 
	{ 
		al_buf[i]= pwd & 7; pwd = pwd >> 3; 
	} 
};

void emulate_func2(unsigned short seed)
{ 
	int i, j; 
	for(i=0;i<8;i++) 
	{ 
		ch[i] = 0; 
		
		for(j=0;j<8;j++)
		{ 
			seed *= 0x1989; 
			seed += 5; 
			ch[i] |= (tab[(seed>>9)&0x3f]) << (7-j); 
		}
	} 
}
\end{lstlisting}

\subsubsection{DHCP}

Это касается также и сетевых протоколов. 
Например, сетевые пакеты протокола DHCP содержат так называемую \emph{magic cookie}: \TT{0x63538263}. 
Какой-либо код, генерирующий пакеты по протоколу DHCP где-то и как-то должен внедрять в пакет также и эту константу. 
Найдя её в коде мы сможем найти место где происходит это и не только это. 
Любая программа, получающая DHCP-пакеты, должна где-то как-то проверять \emph{magic cookie}, 
сравнивая это поле пакета с константой.

Например, берем файл dhcpcore.dll из Windows 7 x64 и ищем эту константу. 
И находим, два раза: оказывается, эта константа используется в функциях с красноречивыми названиями \\
\TT{DhcpExtractOptionsForValidation()} и \TT{DhcpExtractFullOptions()}:

\begin{lstlisting}[caption=dhcpcore.dll (Windows 7 x64),style=customasmx86]
.rdata:000007FF6483CBE8 dword_7FF6483CBE8 dd 63538263h          ; DATA XREF: DhcpExtractOptionsForValidation+79
.rdata:000007FF6483CBEC dword_7FF6483CBEC dd 63538263h          ; DATA XREF: DhcpExtractFullOptions+97
\end{lstlisting}

А вот те места в функциях где происходит обращение к константам:

\begin{lstlisting}[caption=dhcpcore.dll (Windows 7 x64),style=customasmx86]
.text:000007FF6480875F  mov     eax, [rsi]
.text:000007FF64808761  cmp     eax, cs:dword_7FF6483CBE8
.text:000007FF64808767  jnz     loc_7FF64817179
\end{lstlisting}

И:

\begin{lstlisting}[caption=dhcpcore.dll (Windows 7 x64),style=customasmx86]
.text:000007FF648082C7  mov     eax, [r12]
.text:000007FF648082CB  cmp     eax, cs:dword_7FF6483CBEC
.text:000007FF648082D1  jnz     loc_7FF648173AF
\end{lstlisting}

\subsection{Специфические константы}

Иногда, бывают какие-то специфические константы для некоторого типа кода.
Например, однажды автор сих строк пытался разобраться с кодом, где подозрительно часто встречалось число 12.
Размеры многих массивов также были 12, или кратные 12 (24, итд).
Оказалось, этот код брал на вход 12-канальный аудиофайл и обрабатывал его.

И наоборот: например, если программа работает с текстовым полем длиной 120 байт, значит где-то в коде должна
быть константа 120, или 119.
Если используется UTF-16, то тогда $2 \cdot 120$.
Если код работает с сетевыми пакетами фиксированной длины, то хорошо бы и такую константу поискать в коде.

Это также справедливо для любительской криптографии (ключи с лицензией, итд).
Если зашифрованный блок имеет размер в $n$ байт, вы можете попробовать поискать это число в коде.
Также, если вы видите фрагмент кода, который при исполнении, повторяется $n$ раз в цикле,
это может быть ф-ция шифрования/дешифрования.

\subsection{Поиск констант}

В \IDA это очень просто, Alt-B или Alt-I.

\myindex{binary grep}
А для поиска константы в большом количестве файлов, либо для поиска их в неисполняемых файлах, имеется небольшая утилита
\emph{binary grep}\footnote{\BGREPURL}.


\input{digging_into_code/instructions_RU}
\mysection{Подозрительные паттерны кода}

\subsection{Инструкции XOR}
\myindex{x86!\Instructions!XOR}

Инструкции вроде \TT{XOR op, op} (например, \TT{XOR EAX, EAX}) 
обычно используются для обнуления регистра,
однако, если операнды разные, то применяется операция именно \q{исключающего или}.
Эта операция очень редко применяется в обычном программировании, но применяется очень часто в криптографии,
включая любительскую.

Особенно подозрительно, если второй операнд --- это большое число.
Это может указывать на шифрование, вычисление контрольной суммы, итд.  \\
\\
Одно из исключений из этого наблюдения о котором стоит сказать, то, что генерация и проверка значения \q{канарейки}
(\myref{subsec:BO_protection}) часто происходит, используя инструкцию \XOR.  \\
\\
\myindex{AWK}
Этот AWK-скрипт можно использовать для обработки листингов (.lst) созданных \IDA{}:

\lstinputlisting{digging_into_code/awk.sh}

Нельзя также забывать, что подобный скрипт может захватить и неверно дизассемблированный код 
(\myref{sec:incorrectly_disasmed_code}).

\subsection{Вручную написанный код на ассемблере}

\myindex{Function prologue}
\myindex{Function epilogue}
\myindex{x86!\Instructions!LOOP}
\myindex{x86!\Instructions!RCL}
Современные компиляторы не генерируют инструкции \TT{LOOP} и \TT{RCL}. 
С другой стороны, эти инструкции хорошо знакомы кодерам, предпочитающим писать прямо на ассемблере. 
Подобные инструкции отмечены как (M) в списке инструкций в приложении: 
\myref{sec:x86_instructions}.
Если такие инструкции встретились, можно сказать с какой-то вероятностью, что этот фрагмент кода написан вручную.

\par
Также, пролог/эпилог функции обычно не встречается в ассемблерном коде, написанном вручную.

\par
Как правило, во вручную написанном коде, нет никакого четкого метода передачи аргументов в функцию.

\par
Пример из ядра Windows 2003 (файл ntoskrnl.exe):

\lstinputlisting[style=customasmx86]{digging_into_code/ntoskrnl.lst}

Действительно, если заглянуть в исходные коды 
\ac{WRK} v1.2, данный код можно найти в файле \\
\emph{WRK-v1.2\textbackslash{}base\textbackslash{}ntos\textbackslash{}ke\textbackslash{}i386\textbackslash{}cpu.asm}.

\input{digging_into_code/magic_numbers_tracing_RU}
\input{digging_into_code/loops_RU}
% TODO move section...

\subsection{Некоторые паттерны в бинарных файлах}

Все примеры здесь были подготовлены в Windows с активной кодовой страницей 437
в консоли.
Двоичные файлы внутри могут визуально выглядеть иначе если установлена другая кодовая страница.

\clearpage
\subsubsection{Массивы}

Иногда мы можем легко заметить массив 16/32/64-битных значений визуально, в шестнадцатеричном 
редакторе.

Вот пример массива 16-битных значений.
Мы видим, что каждый первый байт в паре всегда равен 7 или 8, а второй выглядит случайным:

\begin{figure}[H]
\centering
\myincludegraphics{digging_into_code/binary/16bit_array.png}
\caption{FAR: массив 16-битных значений}
\end{figure}

Для примера я использовал файл содержащий 12-канальный сигнал оцифрованный при помощи 16-битного \ac{ADC}.

\clearpage
\myindex{MIPS}
\par А вот пример очень типичного MIPS-кода.

Как мы наверное помним, каждая инструкция в MIPS (а также в ARM в режиме ARM, или ARM64) имеет 
длину 32 бита (или 4 байта),
так что такой код это массив 32-битных значений.

Глядя на этот скриншот, можно увидеть некий узор.
Вертикальные красные линии добавлены для ясности:

\begin{figure}[H]
\centering
\myincludegraphics{digging_into_code/binary/typical_MIPS_code.png}
\caption{Hiew: очень типичный код для MIPS}
\end{figure}

Еще пример таких файлов в этой книге: 
\myref{Oracle_SYM_files_example}.

\clearpage
\subsubsection{Разреженные файлы}

Это разреженный файл, в котором данные разбросаны посреди почти пустого файла.
Каждый символ пробела здесь на самом деле нулевой байт (который выглядит как пробел).
Это файл для программирования FPGA (чип Altera Stratix GX).
Конечно, такие файлы легко сжимаются, но подобные форматы очень популярны в научном и инженерном ПО, где быстрый доступ важен, а компактность --- не очень.

\begin{figure}[H]
\centering
\myincludegraphics{digging_into_code/binary/sparse_FPGA.png}
\caption{FAR: Разреженный файл}
\end{figure}

\clearpage
\subsubsection{Сжатый файл}

% FIXME \ref{} ->
Этот файл это просто некий сжатый архив.
Он имеет довольно высокую энтропию и визуально выглядит просто хаотичным.
Так выглядят сжатые и/или зашифрованные файлы.

\begin{figure}[H]
\centering
\myincludegraphics{digging_into_code/binary/compressed.png}
\caption{FAR: Сжатый файл}
\end{figure}

\clearpage
\subsubsection{\ac{CDFS}}

Инсталляции \ac{OS} обычно распространяются в ISO-файлах, которые суть копии CD/DVD-дисков.
Используемая файловая система называется \ac{CDFS}, здесь видны имена файлов и какие-то допольнительные данные.
Это могут быть длины файлов, указатели на другие директории, атрибуты файлов, итд.
Так может выглядеть типичная файловая система внутри.

\begin{figure}[H]
\centering
\myincludegraphics{digging_into_code/binary/cdfs.png}
\caption{FAR: ISO-файл: инсталляционный \ac{CD} Ubuntu 15}
\end{figure}

\clearpage
\subsubsection{32-битный x86 исполняемый код}

Так выглядит 32-битный x86 исполняемый код.
У него не очень высокая энтропия, потому что некоторые байты встречаются чаще других.

\begin{figure}[H]
\centering
\myincludegraphics{digging_into_code/binary/x86_32.png}
\caption{FAR: Исполняемый 32-битных x86 код}
\end{figure}

% TODO: Read more about x86 statistics: \ref{}. % FIXME blog post about decryption...

\clearpage
\subsubsection{Графические BMP-файлы}

% TODO: bitmap, bit, group of bits...

BMP-файлы не сжаты, так что каждый байт (или группа байт) описывают каждый пиксель.
Я нашел эту картинку где-то внутри заинсталлированной Windows 8.1:

\begin{figure}[H]
\centering
\myincludegraphicsSmall{digging_into_code/binary/bmp.png}
\caption{Пример картинки}
\end{figure}

Вы видите, что эта картинка имеет пиксели, которые вряд ли могут быть хорошо сжаты (в районе центра),
но здесь есть длинные одноцветные линии вверху и внизу.
Действительно, линии вроде этих выглядят как линии при просмотре этого файла:

\begin{figure}[H]
\centering
\myincludegraphics{digging_into_code/binary/bmp_FAR.png}
\caption{Фрагмент BMP-файла}
\end{figure}


% FIXME comparison!
\subsection{Сравнение \q{снимков} памяти}
\label{snapshots_comparing}

Метод простого сравнения двух снимков памяти для поиска изменений часто применялся для взлома игр 
на 8-битных компьютерах и взлома файлов с записанными рекордными очками.

К примеру, если вы имеете загруженную игру на 8-битном компьютере (где самой памяти не очень много, но игра
занимает еще меньше), и вы знаете что сейчас у вас, условно, 100 пуль, вы можете сделать \q{снимок} всей
памяти и сохранить где-то. Затем просто стреляете куда угодно, у вас станет 99 пуль, сделать второй \q{снимок},
и затем сравнить эти два снимка: где-то наверняка должен быть байт, который в начале был 100, а затем стал 99.

Если учесть, что игры на тех маломощных домашних компьютерах обычно были написаны на ассемблере и подобные
переменные там были глобальные, то можно с уверенностью сказать, какой адрес в памяти всегда отвечает за количество
пуль. Если поискать в дизассемблированном коде игры все обращения по этому адресу, несложно найти код,
отвечающий за уменьшение пуль и записать туда инструкцию \gls{NOP}
или несколько \gls{NOP}-в, так мы получим игру в которой у игрока всегда будет 100 пуль, например.

\myindex{BASIC!POKE}
А так как игры на тех домашних 8-битных 
компьютерах всегда загружались по одним и тем же адресам, и версий одной игры редко когда было больше одной продолжительное время,
то геймеры-энтузиасты знали, по какому адресу (используя инструкцию языка BASIC \gls{POKE}) что записать после загрузки
игры, чтобы хакнуть её. Это привело к появлению списков \q{читов} состоящих из инструкций \gls{POKE}, публикуемых
в журналах посвященным 8-битным играм.

\myindex{MS-DOS}
Точно так же легко модифицировать файлы с сохраненными рекордами (кто сколько очков набрал), впрочем, это может
сработать не только с 8-битными играми. Нужно заметить, какой у вас сейчас рекорд и где-то сохранить файл
с очками. Затем, когда очков станет другое количество, просто сравнить два файла, можно даже
DOS-утилитой FC\footnote{утилита MS-DOS для сравнения двух файлов побайтово} (файлы рекордов, часто, бинарные).

Где-то будут отличаться несколько байт, и легко будет увидеть, какие именно отвечают за количество очков. 
Впрочем, разработчики игр полностью осведомлены о таких хитростях и могут защититься от этого.

В каком-то смысле похожий пример в этой книге здесь: \myref{Millenium_DOS_game}.

% TODO: пример с какой-то простой игрушкой?

\subsubsection{Реальная история из 1999}

\myindex{ICQ}
В то время был популярен мессенджер ICQ, по крайней мере, в странах бывшего СССР.
У мессенджера была особенность --- некоторые пользователи не хотели, чтобы все знали, в онлайне они или нет.
И для начала у того пользователя нужно было запросить \emph{авторизацию}.
Тот человек мог разрешить вам видеть свой статус, а мог и не разрешить.

Автор сих строк сделал следующее.

\begin{itemize}
\item Добавил человека. Он появился в контакт-листе, в разделе ``wait for authorization''.
\item Выгрузил ICQ.
\item Сохранил базу ICQ в другом месте.
\item Загрузил ICQ снова.
\item Человек \emph{авторизировал}.
\item Выгрузил ICQ и сравнил две базы.
\end{itemize}

Выяснилось: базы отличались только одним байтом.
В первой версии: \verb|RESU\x03|, во второй \verb|RESU\x02|.
(``RESU'', надо думать, означало ``USER'', т.е., заголовок структуры, где хранилась информация о пользователе.)
Это означало, что информация об авторизации хранилась не на сервере, а в клиенте.
Вероятно, значение 2/3 отражало статус \emph{авторизированности}.

\subsubsection{Реестр Windows}

А еще можно вспомнить сравнение реестра Windows до инсталляции программы и после.
Это также популярный метод поиска, какие элементы реестра программа использует.

Наверное это причина, почему так популярны shareware-программы для очистки реестра в Windows.

Кстати, вот как сдампить реестр в Windows в текстовые файлы:

\begin{lstlisting}
reg export HKLM HKLM.reg
reg export HKCU HKCU.reg
reg export HKCR HKCR.reg
reg export HKU HKU.reg
reg export HKCC HKCC.reg
\end{lstlisting}

\myindex{UNIX!diff}
Затем их можно сравнивать используя diff...

\subsubsection{Инженерное ПО, CAD-ы, итд}

Если некое ПО использует закрытые (проприетарные) файлы, то и тут можно попытаться что-то выяснить.
Сохраняете файл.
Затем добавили точку, или линию, или еще какой примитив.
Сохранили файл, сравнили.
Или сдвинули точку в сторону, сохранили файл, сравнили.

\subsubsection{Блинк-компаратор}

Сравнение файлов или слепков памяти вообще, немного напоминает блинк-компаратор
\footnote{\url{http://go.yurichev.com/17349}}:
устройство, которое раньше использовали астрономы для поиска движущихся небесных объектов.

Блинк-компаратор позволял быстро переключаться между двух отснятых в разное время кадров,
и астроном мог увидеть разницу визуально.

Кстати, при помощи блинк-компаратора, в 1930 был открыт Плутон.


\input{digging_into_code/ISA_detect_RU}

\mysection{Прочее}

\subsection{Общая идея}

Нужно стараться как можно чаще ставить себя на место программиста и задавать себе вопрос, 
как бы вы сделали ту или иную вещь в этом случае и в этой программе.

\subsection{Порядок функций в бинарном коде}

Все функции расположеные в одном .c или .cpp файле компилируются в соответствующий объектный (.o) файл.
Линкер впоследствии складывает все нужные объектные файлы вместе, не меняя порядок ф-ций в них.
Как следствие, если вы видите в коде две или более идущих подряд ф-ций, то это означает, что и в исходном коде они 
были расположены в одном и том же файле (если только вы не на границе двух объектных файлов, конечно).
Это может означать, что эти ф-ции имеют что-то общее между собой, что они из одного слоя \ac{API}, из одной библиотеки, итд.

\myindex{CryptoPP}
Это реальная история из практики: однажды автор искал в прикомпилированной библиотеке CryptoPP ф-ции связанные
с алгоритмом Twofish, особенно шифрования/дешифрования.\\
Я нашел ф-цию \verb|Twofish::Base::UncheckedSetKey()|, но не остальные.
Заглянув в исходники \verb|twofish.cpp|
\footnote{\url{https://github.com/weidai11/cryptopp/blob/b613522794a7633aa2bd81932a98a0b0a51bc04f/twofish.cpp}}, стало ясно, что все ф-ции расположены в одном модуле (\verb|twofish.cpp|).
Так что я просто попробовал посмотреть ф-ции следующие за \\
\verb|Twofish::Base::UncheckedSetKey()| --- так и оказалось,\\
одна из них была \verb|Twofish::Enc::ProcessAndXorBlock()|, другая --- \verb|Twofish::Dec::ProcessAndXorBlock()|.

\subsection{Крохотные функции}

Крохотные ф-ции, такие как пустые ф-ции (\myref{empty_func})
или ф-ции возвращающие только ``true'' (1) или ``false'' (0) (\myref{ret_val_func}) очень часто встречаются,
и почти все современные компиляторы, как правило, помещают только одну такую ф-цию в исполняемый код,
даже если в исходном их было много одинаковых.
Так что если вы видите ф-цию состояющую только из \TT{mov eax, 1 / ret}, которая может вызываться из разных мест,
которые, судя по всему, друг с другом никак не связаны, это может быть результат подобной оптимизации.

\subsection{\Cpp}

\ac{RTTI}~(\myref{RTTI})-информация также может быть полезна для идентификации 
классов в \Cpp.

\subsection{Намеренный сбой}

Часто, нужно знать, какая ф-ция была исполнена, а какая --- нет.
Вы можете использовать отладчик, но на экзотических архитектурах его может и не быть, так что простейший способ это вписать
туда неверный опкод или что-то вроде \INS{INT3} (0xCC).
Сбой будет сигнализировать о том, что эта инструкция была исполнена.

Еще один пример намеренного сбоя: \myref{dmalloc_KILL_PROCESS}.

}
\DE{\chapter{Finden von wichtigen / interessanten Stellen im Code}

Minimalismus ist kein beliebtes Feature moderner Software.

\myindex{\Cpp!STL}

Aber nicht weil die Programmierer so viel Code schreiben, sondern weil die Libaries
allgemein statisch zu ausf\"uhrbaren Dateien gelinkt werden. Wenn alle externen
Libraries in externe DLL Dateien verschoben werden w\"urden, w\"are die Welt ein
anderer Ort. (Ein weiterer Grund f\"ur C++ sind die \ac{STL} und andere Template-Libraries.)

\newcommand{\FOOTNOTEBOOST}{\footnote{\url{http://go.yurichev.com/17036}}}
\newcommand{\FOOTNOTELIBPNG}{\footnote{\url{http://go.yurichev.com/17037}}}

Deshalb ist es sehr wichtig den Ursprung einer Funktion zu bestimmen, wenn die
Funktion aus einer Standard-Library oder aus einer sehr bekannten Library stammt
(wie z.B Boost\FOOTNOTEBOOST, libpng\FOOTNOTELIBPNG), oder ob die Funktion sich
auf das bezieht was wir im Code versuchen zu finden.

Es ist ein wenig absurd s\"amtlichen Code in \CCpp neu zu schreiben, um das zu
finden was wir suchen.

Eine der Hauptaufgaben eines Reverse Enigneers ist es schnell Code zu finden den
er/sie sucht.

\myindex{\GrepUsage}

Der \IDA-Disassembler erlaubt es durch Textstrings, Byte-Sequenzen und Konstanten
zu suchen.  Es ist sogar m\"oglich den Code in .lst oder .asm Text Dateien zu
exportieren und diese mit \TT{grep}, \TT{awk}, etc. zu untersuchen.

Wenn man versucht zu verstehen wie ein bestimmter Code funktioniert, kann auch
eine einfache Open-Source-Library wie libpng als Beispiel dienen.
Wenn man also eine Konstante oder Textstrings findet die vertraut erscheinen, ist
es immer einen Versuch wert diese zu \emph{google}n .
Und wenn man ein Opensource Projekt findet in dem diese Funktion benutzt wird, 
reicht es meist aus diese Funktionen miteinander zu vergleichen.
Es k\"onnte helfen Teile des Problems zu l\"osen.

% When you try to understand what some code is doing, this easily could be some open-source library like libpng.
% So when you see some constants or text strings which look familiar, it is always worth to \emph{google} them.
% And if you find the opensource project where they are used, 
% then it's enough just to compare the functions.
% It may solve some part of the problem.

Zum Beispiel, wenn ein Programm XML Dateien benutzt, w\"are der erste Schritt zu ermitteln welche
XML-Library benutzt wird f\"ur die Verarbeitung, da die Standard (oder am weitesten verbreitete) libraries
normal benutzt werden anstatt selbst geschriebene librarys.

\myindex{SAP}
\myindex{Windows!PDB}

Zum Beispiel, der Autor dieser Zeilen wollte verstehen wie die Kompression/Dekompression von Netzwerkpaketen in SAP 6.0 funktioniert.
SAP ist ein gewaltiges St\"uck Software, aber detaillierte -\gls{PDB} Dateien mit Debug Informationen sind vorhanden, was sehr praktisch 
ist. Der Autor hat schließlich eine Ahnung gehabt, das eine Funktion genannt \emph{CsDecomprLZC} die Dekompression der Netzwerkpakete \"ubernahm.
Er hat nach dem Namen der Funktion auf google gesucht und ist schnell zum schluss gekommen das diese Funktion in 
MaxDB benutzt wurde (Das ist ein Open-Source SAP Projekt) \footnote{Mehr dar\"uber in der relevanten Sektion~(\myref{sec:SAPGbUI})}. 

\url{http://www.google.com/search?q=CsDecomprLZC}

Erstaunlich, das MaxDB und die SAP 6.0 Software den selben Code geteilt haben f\"ur die Kompression/Dekompression der Netzwerkpakete.

\input{digging_into_code/identification/exec_DE} 

\mysection{Kommunikation mit der außen Welt (Funktion Level)} 
Oft ist es empfehlenswert die Funktionsargumente und die R\"uckgabewerte im
Debugger oder \ac{DBI} zu \"uberwachen. Zum Beispiel hat der Autor einmal
versucht die Bedeutung einer obskuren Funktion zu verstehen, die einen inkorrekten
Bubblesort-Algorithmus implementiert hatte\footnote{\url{https://yurichev.com/blog/weird_sort_KLEE/}}
(Sie hat funktioniert, jedoch viel langsamer als normal). Die Eingaben und Ausgaben zur Laufzeit 
der Funktion zu \"uberwachen hilft sofort zu verstehen was die Funktion tut.

% TBT

% sections:
\input{digging_into_code/communication_win32_DE}
\input{digging_into_code/strings_DE}
\input{digging_into_code/assert_DE}
\mysection{Konstanten}

Menschen, Programmierer eingeschlossen, neigen dazu Zahlen zu runden wie z.B 10, 100, 1000,
im realen Leben so wie in ihrem Code.

Der angehende Reverse Engineer kennt diese Werte und ihre hexadezimale Repr\"asentation sehr gut:
10=0xA, 100=0x64, 1000=0x3E8, 10000=0x2710.

Die Konstanten \TT{0xAAAAAAAA} (0b10101010101010101010101010101010) und \\
\TT{0x55555555} (0b01010101010101010101010101010101) sind auch sehr popul\"ar---
sie sind zusammengesetzt aus ver\"andernden Bits. % <-- Findest vielleicht noch ne bessere Bezeichnung

Dies hilft Signale voneinander zu unterscheiden bei denen alle Bits eingeschaltet (0b1111 \dots) oder ausgeschaltet (0b0000 \dots) werden .
Zum Beispiel wird die Konstante \TT{0x55AA} beim Boot Sektor, \ac{MBR},
und im \ac{ROM} von IBM-Kompatiblen Erweiterung Karten benutzt.

Manche Algorithmen, speziell die Kryptografischen benutzen eindeutige Konstanten, die mit der Hilfe von \IDA einfach im Code zu finden sind.

\myindex{MD5}

Zum Beispiel, der MD5 Algorithmus initialisiert seine Internen Variablen wie folgt:

\begin{verbatim}
var int h0 := 0x67452301
var int h1 := 0xEFCDAB89
var int h2 := 0x98BADCFE
var int h3 := 0x10325476
\end{verbatim}

Wenn man diese vier Konstanten im Code hintereinander benutzt findet, dann ist die Wahrscheinlichkeit das diese Funktion 
sich auf MD5 bezieht.

\par Ein weiteres Beispiel sind die CRC16/CRC32 Algorithmen,
ihre Berechnungs Algorithmen benutzen oft vorberechnete Tabellen wie diese:

\begin{lstlisting}[caption=linux/lib/crc16.c,style=customc]
/** CRC table for the CRC-16. The poly is 0x8005 (x^16 + x^15 + x^2 + 1) */
u16 const crc16_table[256] = {
	0x0000, 0xC0C1, 0xC181, 0x0140, 0xC301, 0x03C0, 0x0280, 0xC241,
	0xC601, 0x06C0, 0x0780, 0xC741, 0x0500, 0xC5C1, 0xC481, 0x0440,
	0xCC01, 0x0CC0, 0x0D80, 0xCD41, 0x0F00, 0xCFC1, 0xCE81, 0x0E40,
	...
\end{lstlisting}

Man beachte auch die vorberechnete Tabelle f\"ur CRC32: \myref{sec:CRC32}.

In tabellenlosen CRC-Algorithmen werden bekannte Polynome benutzt, zum Beispiel, 0xEDB88320 f\"ur CRC32.

\subsection{Magic numbers}
\label{magic_numbers}

Viele Datei-Formate definieren einen Standard-Dateiheader in dem eine \emph{magic number(s)} benutzt wird, einzelne oder sogar mehrere. 

\myindex{MS-DOS}

Zum Beispiel, alle Win32 und MS-DOS executable starten mit zwei Zeichen \q{MZ}.

\myindex{MIDI}

Am Anfang einer MIDI Datei muss die \q{MThd} Signatur vorhanden sein.
Wenn wir ein Programm haben das auf MIDI Dateien zugreift um sonst was zu machen,
ist es sehr wahrscheinlich das das Programm die Datei validieren muss in dem es
mindestens die ersten 4 Bytes pr\"uft.

Das kann man wie folgt realisieren:
(\emph{buf} Zeigt auf den Anfang der geladenen Datei im Speicher) 

\begin{lstlisting}[style=customasmx86]
cmp [buf], 0x6468544D ; "MThd"
jnz _error_not_a_MIDI_file
\end{lstlisting}

\myindex{\CStandardLibrary!memcmp()}
\myindex{x86!\Instructions!CMPSB}

\dots oder durch das Aufrufen der Funktion f\"ur das vergleichen von Speicherbl\"ocken wie z.B \TT{memcmp()} oder 
beliebigen anderen Code bis hin zu einer \TT{CMPSB} (\myref{REPE_CMPSx}) Instruktion.

Wenn man so einen Punkt findet kann man bereits sagen das eine MIDI Datei geladen wird, % <-- \"Andern?
wir k\"onnen auch sehen wo der Puffer mit den Inhalten der MIDI Datei liegt und was/wie aus diesem
Puffer verwendet wird.

\subsubsection{Daten}

\myindex{UFS2}
\myindex{FreeBSD}
\myindex{HASP}

Oft findet man auch nur eine Zahl wie \TT{0x19870116}, was ganz klar nach einem Jahres Datum aussieht (Tag 16,  1 Monat (Januar),  Jahr 1987).
Das ist vielleicht das Geburtsdatum von jemandem (ein Programmierer. ihre/seine bekannte, Kind), oder ein anderes wichtiges Datum.
Das Datum kann auch in umgekehrter folge auftreten, wie z.B \TT{0x16011987}. 
Datums angaben im Amerikanischen-Stil sind auch weit verbreitet wie \TT{0x01161987}.

Ein ziemlich bekanntes Beispiel ist  \TT{0x19540119} (magic number wird in der UFS2 Superblock Struktur benutzt), das 
Geburtsdatum von Marschall Kirk McKusick ist, einem Prominenten FreeBSD Entwickler. 


\myindex{Stuxnet}
Stuxnet benutzt die Zahl ``19790509'' (nicht als 32-Bit Zahl, aber als String), was zu Spekulationen gef\"uhrt hat
weil die malware Verbindungen nach Israel aufzeigt%
\footnote{Das ist das Datum der Hinrichtung von Habib Elghanian, persischer Jude.}.

Solche Zahlen sind auch sehr beliebt in Amateur Kryptografie, zum Beispiel, ein Ausschnitt aus den \emph{secret function} Interna aus dem HASP3 Dongle %  <-- Vielleicht bessere formulierung?
\footnote{\url{https://web.archive.org/web/20160311231616/http://www.woodmann.com/fravia/bayu3.htm}}:

\begin{lstlisting}[style=customc]
void xor_pwd(void) 
{ 
	int i; 
	
	pwd^=0x09071966;
	for(i=0;i<8;i++) 
	{ 
		al_buf[i]= pwd & 7; pwd = pwd >> 3; 
	} 
};

void emulate_func2(unsigned short seed)
{ 
	int i, j; 
	for(i=0;i<8;i++) 
	{ 
		ch[i] = 0; 
		
		for(j=0;j<8;j++)
		{ 
			seed *= 0x1989; 
			seed += 5; 
			ch[i] |= (tab[(seed>>9)&0x3f]) << (7-j); 
		}
	} 
}
\end{lstlisting}

\subsubsection{DHCP}

Das Trifft auf Netzwerk Protokolle ebenso zu. 
Zum Beispiel, die Pakete des DHCP Protokoll's beinhalten so genannte \emph{magic cookie}: \TT{0x63538263}.
Jeder Code der ein DHCP Pakete generiert, muss diese Konstante in das Pakete einbetten.
Wenn wir diesen Code finden, wissen wir auch wo es passiert und nicht nur was passiert.
Jedes Programm das DHCP Pakete empfangen kann muss verifizieren das der \emph{magic cookie} mit der Konstante 
\"ubereinstimmt. 

Zum Beispiel, lasst uns die dhcpcore.dll Datei aus Windows 7 x64 analysieren die nach der Konstante suchen.
Wir k\"onnen die Konstante zweimal finden:
Es sieht danach aus als w\"are die Konstante in zwei Funktionen benutzt mit dem selbst redenden Namen\\
\TT{DhcpExtractOptionsForValidation()} und \TT{DhcpExtractFullOptions()}:

\begin{lstlisting}[caption=dhcpcore.dll (Windows 7 x64),style=customasmx86]
.rdata:000007FF6483CBE8 dword_7FF6483CBE8 dd 63538263h          ; DATA XREF: DhcpExtractOptionsForValidation+79
.rdata:000007FF6483CBEC dword_7FF6483CBEC dd 63538263h          ; DATA XREF: DhcpExtractFullOptions+97
\end{lstlisting}

Und hier die (Speicher) Orte an denen auf die Konstante zugegriffen wird:

\begin{lstlisting}[caption=dhcpcore.dll (Windows 7 x64),style=customasmx86]
.text:000007FF6480875F  mov     eax, [rsi]
.text:000007FF64808761  cmp     eax, cs:dword_7FF6483CBE8
.text:000007FF64808767  jnz     loc_7FF64817179
\end{lstlisting}

Und:

\begin{lstlisting}[caption=dhcpcore.dll (Windows 7 x64),style=customasmx86]
.text:000007FF648082C7  mov     eax, [r12]
.text:000007FF648082CB  cmp     eax, cs:dword_7FF6483CBEC
.text:000007FF648082D1  jnz     loc_7FF648173AF
\end{lstlisting}

\subsection{Spezifische Konstanten}

Manchmal, gibt es spezifische Konstanten f\"ur gewissen Code % <-- Besser? 
Zum Beispiel, einmal hat der Autor sich in ein St\"uck Code gegraben wo die Nummer 12 verd\"achtig
oft vor kam. Arrays haben oft eine Gr\"oße von 12 oder ein vielfaches von 12 (24, etc). 
Wie sich raus stellte, hat der Code eine 12-Kanal Audiodatei an der Eingabe entgegen genommen und
sie verarbeitet.

Und umgekehrt: zum Beispiel, wenn ein Programm ein Textfeld verarbeitet das eine L\"ange von 120 Bytes hat,
dann gibt es auch eine Konstante 120 oder 119 irgendwo im Code.
Wenn UTF-16 Benutzt wird, dann $2 \cdot 120$. Wenn Code mit Netzwerkpaketen arbeitet die von fester Gr\"oße
sind, ist es eine gute Idee nach dieser Konstante im Code zu suchen.

Das trifft auch auf Amateur Kryptografie zu (Lizenz Schl\"ussel, etc). 
Bei einem verschl\"usselten Block von $n$ Bytes, will man versuchen die vorkommen dieser Nummer im Code zu suchen,
auch, wenn man ein St\"uck Code sieht der sich $n$ mal w\"ahrend einer Schleifen Ausf\"uhrung wiederholt, ist das vielleicht
eine ver-/Entschl\"usselung Routine.

\subsection{Nach Konstanten suchen}

Das ist einfach mit \IDA: Alt-B oder Alt-I.
\myindex{bin\"ar grep}
Und f\"ur das suchen von Konstanten in einem Haufen großer Dateien, oder f\"ur das suchen in nicht ausf\"uhrbaren Dateien,
gibt es ein kleines Utility genannt \emph{binary grep}\footnote{\BGREPURL}.

\input{digging_into_code/instructions_DE}
\mysection{Verd\"achtige Code muster}

\subsection{XOR Instruktionen}
\myindex{x86!\Instructions!XOR}

Instruktionen wie \TT{XOR op, op} (zum Beispiel, \TT{XOR EAX, EAX})
werden normal daf\"ur benutzt Register Werte auf Null zu setzen, wenn jedoch
einer der Operanden sich unterscheidet wird die \q{exclusive or} Operation 
ausgef\"uhrt.

Diese Operation wird allgemeinen selten benutzt beim programmieren, aber ist
weit verbreitet in der Kryptografie, besonders bei Amateuren der Kryptografie.
Sowas ist besonders Verd\"achtig wenn der zweite Operand eine große Zahl ist.

Das k\"onnte ein Hinweis sein das etwas ver-/entschl\"usselt wird oder Checksumme berechnet werden, etc.

Eine Ausnahme dieser Beobachtung ist der \q{canary} (\myref{subsec:BO_protection}). 
Die Generierung und das pr\"ufen des \q{canary} werden oft mit Hilfe der \XOR Instruktion gemacht. 

\myindex{AWK}

Dieses AWK Skript kann benutzt werden um \IDA{} listing (.lst) Dateien zu parsen:

\lstinputlisting{digging_into_code/awk.sh}

Es sollte auch noch erw\"ahnt werden das diese Art von Skript in der Lage ist inkorrekt disassemblierten Code zu erkennen
(\myref{sec:incorrectly_disasmed_code}).

\subsection{Hand geschriebener Assembler code}

\myindex{Function prologue}
\myindex{Function epilogue}
\myindex{x86!\Instructions!LOOP}
\myindex{x86!\Instructions!RCL}

Moderne Compiler benutzen keine \TT{LOOP} und \TT{RCL} Instruktionen.
Auf der anderen Seite sind diese Instruktionen sehr beliebt bei Programmieren die Code direkt in Assembler schreiben.
Wenn man diese Instruktionen sieht, kann man mit hoher Sicherheit sagen das dieses Code Fragment h\"andisch geschrieben wurde.,
Diese Instruktionen sind in der Instruktionsliste im Anhang mit (M) markiert: \myref{sec:x86_instructions}.

\par Die Funktions Prolog und Epilog sind allgemein nicht vorhanden bei handgeschriebenen Assembler Code.

\par Tats\"achlich gibt es kein bestimmtes System um Argumente an Funktionen zu \"ubergeben wenn der Code handgeschrieben wurde. 

\par Beispiel aus dem Windows 2003 Kernel (ntoskrnl.exe file):

\lstinputlisting[style=customasmx86]{digging_into_code/ntoskrnl.lst}

Tats\"achlich, wenn wir in den \ac{WRK} v1.2 source code schauen, kann dieser Code einfach in der Datei
\emph{WRK-v1.2\textbackslash{}base\textbackslash{}ntos\textbackslash{}ke\textbackslash{}i386\textbackslash{}cpu.asm} gefunden werden.

% TBT
%\par 
%As of \INS{RCL}, I could find it in ntoskrnl.exe file from Windows 2003 x86 (MS Visual C compiler).
%It is occurred only once, in \TT{RtlExtendedLargeIntegerDivide()} function, and this might be inline assembler code case.

\input{digging_into_code/magic_numbers_tracing_DE}
\input{digging_into_code/loops_DE}
% TODO move section...

\subsection{ Muster in Bin\"ardatein finden}

Alle Beispiele hier wurden vorbereitet mit Windows mit aktiver Code Page 437
in der Konsole.
Bin\"ar Dateien sehen intern etwas anders aus wenn eine andere Code page gesetzt ist.

\clearpage
\subsubsection{Arrays}

Manchmal kann man klar ein Array von 16/32/64-Bit Werten mit bloßem Auge im hex Editor erkennen.

Hier ist ein Beispiel eines 16-Bit Wertes.
Wir sehen das das erste Byte ein paar aus 7 oder 8 ist und das zweite sieht
zuf\"allig aus:

\begin{figure}[H]
\centering
\myincludegraphics{digging_into_code/binary/16bit_array.png}
\caption{FAR: array von 16-Bit Werten}
\end{figure}

Ich habe eine Datei benutzt die ein 12 Kanal Signal digitalisiert mit 16-Bit nutzt \ac{ADC}.

\clearpage
\myindex{MIPS}
\par Und hier ist ein Beispiel von einem Typischen MIPS Code.

Wie wir uns vielleicht erinnern, jede MIPS ( also auch ARM in ARM Mode oder ARM64 ) Instruktion hat eine Gr\"oße von 32 Bits (oder 4 Bytes),
also ist solcher Code ein Array von 32-Bit Werten. 

Wenn man den Screenshot anschaut, sehen wir eine Art Muster.

Vertikale und rote Linien wurden zur besseren Lesbarkeit eingef\"ugt:

\begin{figure}[H]
\centering
\myincludegraphics{digging_into_code/binary/typical_MIPS_code.png}
\caption{Hiew: sehr typischer MIPS code}
\end{figure}

Ein weiteres Beispiel eines solchen Musters ist Buch:
\myref{Oracle_SYM_files_example}.

\clearpage
\subsubsection{Sparse Dateien} 

Diese d\"urftige Datei mit zerstreuten Daten inmitten einer fast leeren Datei.
Jedes Space Zeichen hier ist in der tat ein Zero Byte (das wie ein space aussieht). % <-- findet man sicher was besseres
Das ist eine Datei mit der ein FPGA Programmiert wird (Ein Altera Stratix GX Ger\"at).
Sicher k\"onnen Dateien wie diese einfach Komprimiert werden, aber diese Formate sind in 
der Wissenschaft und im Ingenieurs Wesen so wie in der Softwareentwicklung sehr verbreitet.
Wo es oft um effizienten Zugriff geht und weniger um die Komprimierung der Daten.

% This is sparse file with data scattered amidst almost empty file.
% Each space character here is in fact zero byte (which is looks like space).
% This is a file to program FPGA (Altera Stratix GX device).
% Of course, files like these can be compressed easily, but formats like this one are very popular in scientific and engineering software where efficient access is important while compactness is not.

\begin{figure}[H]
\centering
\myincludegraphics{digging_into_code/binary/sparse_FPGA.png}
\caption{FAR: Sparse file}
\end{figure}

\clearpage
\subsubsection{Komprimierte Dateien}

% FIXME \ref{} ->
Diese Datei ist einfach ein komprimiertes Archiv. 
Es hat eine relativ hohe Entropie und visuell betrachtet sieht es 
eher Chaotisch aus. So sehen komprimierte oder verschl\"usselte Dateien aus.

\begin{figure}[H]
\centering
\myincludegraphics{digging_into_code/binary/compressed.png}
\caption{FAR: Komprimierte Datei}
\end{figure}

\clearpage
\subsubsection{\ac{CDFS}}

\ac{OS} Installationen werden \"ublicherweise als ISO Datei bereit gestellt, die Kopien von CD/DVD Disks sind. 
Das Dateisystem das benutzt wird heißt \ac{cdfs}, hier sieht man wie Dateinamen mit zus\"atzlichen Daten vermischt sind.
Das k\"onnen Datei Gr\"oßen, Pointer auf andere Verzeichnisse, Datei Attribute und anderes sein. 
So sehen Dateisysteme typischerweise auch von innen aus.

\begin{figure}[H]
\centering
\myincludegraphics{digging_into_code/binary/cdfs.png}
\caption{FAR: ISO file: Ubuntu 15 Installation \ac{CD}}
\end{figure}

\clearpage
\subsubsection{32-bit x86 ausf\"uhrbarer Code} 

So sieht 32-Bit x86 ausf\"uhrbarer Code aus. 
Der Code hat nicht wirklich viel Entropie, weil manche Bytes \"ofters vorkommen als andere.

\begin{figure}[H]
\centering
\myincludegraphics{digging_into_code/binary/x86_32.png}
\caption{FAR: Executable 32-bit x86 code}
\end{figure}

% TODO: Read more about x86 statistics: \ref{}. % FIXME blog post about decryption...

\clearpage
\subsubsection{BMP graphics files}

% TODO: bitmap, bit, group of bits...

BMP Dateien sind nicht komprimiert, also ist jedes Byte ( oder Gruppen von Bytes ) beschrieben als
ein Pixel. Diese Bild habe ich irgendwo in meiner Windows 8.1 Installation gefunden: 

\begin{figure}[H]
\centering
\myincludegraphicsSmall{digging_into_code/binary/bmp.png}
\caption{Example picture}
\end{figure}

Man kann sehen das dieses Bild Pixel hat, die nicht wirklich gut komprimiert werden k\"onne (um das Zentrum herum),
aber es sind lange ein-Farben Linien am Anfang und am ende der Datei. Tats\"achlich Linien wie diese sehen wie Linien aus
wenn man sich die Datei anschaut:

\begin{figure}[H]
\centering
\myincludegraphics{digging_into_code/binary/bmp_FAR.png}
\caption{BMP file fragment}
\end{figure}


% FIXME comparison!
\subsection{Memory \q{snapshots} comparing}
\label{snapshots_comparing}

Die Technik zwei Memory Snapshots zu vergleichen ist recht einfach, das hat man auch oft benutzt um 8-Bit Computerspiele und
\q{high score}'s  zu hacken.

Zum Beispiel, wenn man ein geladenes Spiel auf einem 8-Bit Computer hat ( auf den Maschinen ist nicht viel Speicher 
vorhanden, jedoch braucht das Spiel noch weniger Speicher) und du weißt was du im Spiel hast, sagen wir 100 Patronen, 
nun kann man einen \q{snapshot} vom gesamten Speicher machen und diesen Irgendwohin speichern. Dann verschiesst man 
eine Patrone, dann geht der Patronen Z\"ahler auf 99, nun erstellt man den zweiten Snapshot und Vergleich die beiden: 
Nun muss es irgendwo ein Byte geben das vorher 100 war und jetzt 99 ist. 

Betrachtet man den Fakt das diese 8-Bit Spiele oftmals in Assembler geschrieben wurden und diese Variablen meist global 
waren, konnte man ziemlich einfach bestimmen welche Adressen im Speicher den Kugelz\"ahler beinhalten. Wenn man nach allen 
Referenzen der Adresse im dissassembelten Spiel code sucht, ist es nicht schwer den Code \glslink{decrement}{decrementing} 
zu finden und dann eine \gls{NOP} Instruktion an diese Stelle zu schreiben, oder gar mehrere \gls{NOP}-s, und dann hat man 
ein Spiel bei dem man f\"ur immer 100 Kugeln hat. %<-- das kacke der ganze block
\myindex{BASIC!POKE}
Spiele auf 8-Bit Computern wurden allgemein an konstanten Adressen geladen, zus\"atzlich gab es nicht viele unterschiedliche
Versionen des Spiels (  Es war meist eine Version f\"ur lange Zeit popul\"ar ), dadurch wussten enthusiastische Gamer welche
Bytes (durch das benutzen von Basic Instruktionen wie \gls{POKE}) \"uberschrieben werden mussten um das Spiel zu hacken. 
Das hat wiederum zu \q{cheat} listen gef\"uhrt die in Magazinen f\"ur 8-Bit Games erschienen, die dann \gls{POKE} Instruktionen enthielten.

% Considering the fact that these 8-bit games were often written in assembly language and such variables were global,
% it can be said for sure which address in memory has holding the bullet count. If you searched for all references to the
% address in the disassembled game code, it was not very hard to find a piece of code \glslink{decrement}{decrementing} the bullet count,
% then to write a \gls{NOP} instruction there, or a couple of \gls{NOP}-s, 
% and then have a game with 100 bullets forever.
% \myindex{BASIC!POKE}
% Games on these 8-bit computers were commonly loaded at the constant
% address, also, there were not much different versions of each game (commonly just one version was popular for a long span of time),
% so enthusiastic gamers knew which bytes must be overwritten (using the BASIC's instruction \gls{POKE}) at which address in
% order to hack it. This led to \q{cheat} lists that contained \gls{POKE} instructions, published in magazines related to
% 8-bit games.

\myindex{MS-DOS}

Es ist auch einfach \q{high score} Dateien zu modifizieren, das funktioniert nicht nur bei 8-Bit Spielen. Man achte 
auf seinen Highscore Z\"ahler, dann macht man ein Backup der Datei. Wenn sich der \q{high score} Z\"ahler \"andert, vergleicht man die 
zwei Dateien miteinander, das kann man sogar mit dem DOS Tool FC\footnote{MS-DOS Utility zum vergleichen von  Dateien} (\q{high score} Dateien,
sind oft in Bin\"arer Form). 

Es wird beim Vergleichen der Dateien einen Punkt geben wo einige Bytes sich unterscheiden und 
es wird leicht sein, die Punkte zu sehen die die Bytes des Punktez\"ahler beinhalten. 
Jedoch sind sich die Spiele Entwickler solcher Tricks bewusst und bauen Wege ein um das Programm
vor solchen Manipulationen zu sch\"utzen. 

Ein \"ahnliches Beispiel findet man auch in dem Buch \myref{Millenium_DOS_game}.

% TODO: пример с какой-то простой игрушкой?

% TBT 

\subsubsection{Windows registry}

Es ist auch m\"oglich die Windows Regestry zu vergleichen vor und nach der Programm Installation.

Es ist eine sehr popul\"are Methode Regestry Elemente zu finden die vom Programm benutzt werden.
Vielleicht ist das auch der Grund warum die \q{windows regestry cleaner} Shareware so popul\"ar ist.

% TBT

\subsubsection{Blink-comparator}

Der Vergleich von Datei- oder Speichersnapshots erinnert ein wenig an einen Blinkkomparator
\footnote{\url{http://go.yurichev.com/17348}}
ein Ger\"at das in der Vergangenheit von Astronomen benutzt wurde, um sich bewegende Astronomische
Objekte zu finden.

Ein Blinkkomperator erlaubt es schnell zwischen Photographie zu wechseln die zu unterschiedlicher
Zeit aufgenommen wurden, so kann ein Astronom Unterschiede zwischen Fotografien visuell erkennen.

Ach \"ubrigens, Pluto wurde durch einen solchen Blink-Komparator 1930 entdeckt.

% TBT \input{digging_into_code/ISA_detect_DE}

\mysection{Andere Dinge}

\subsection{Die Idee}  

Ein Reverse Engineer sollte versuchen so oft wie m\"oglich in den Schuhen des
Programmierers zu laufen. Um ihren/seinen Standpunkt zu betrachten uns sich
selbst zu Fragen wie man einen Task in spezifischen F\"allen l\"osen w\"urde.

\subsection{Anordnung von Funktionen in Bin\"ar Code}  

S\"amtliche Funktionen die in einer einzelnen .c oder .cpp-Datei gefunden werden,
werden zu den entsprechenden Objekt Dateien (.o) kompiliert. Sp\"ater, f\"ugt
der Linker alle Objektdatein die er braucht zusammen, ohne die Reihenfolge oder
die Funktionen in Ihnen zu ver\"andern. Als eine Konsequenz, ergibt sich daraus
wenn man zwei oder mehr aufeinander folgende Funktionen sieht, bedeutet dass das
sie in der gleichen Source Code Datei platziert waren (Außer nat\"urlich man bewegt
sich an der Grenze zwischen zwei Dateien.). Das bedeutet das diese Funktionen etwas
gemeinsam haben, das sie aus dem gleichen \ac{API}-Level stammen oder aus der
gleichen Library, etc.

% TBT
%\myindex{CryptoPP}
%This is a real story from practice: once upon a time, the author searched for Twofish-related functions in
%a program with CryptoPP library linked, especially encryption/decryption functions.\\
%I found the \verb|Twofish::Base::UncheckedSetKey()| function, but not others.
%After peeking into the \verb|twofish.cpp| source code
%\footnote{\url{https://github.com/weidai11/cryptopp/blob/b613522794a7633aa2bd81932a98a0b0a51bc04f/twofish.cpp}}, it became clear that all functions are located in one module (\verb|twofish.cpp|).\\
%So I tried all function that followed \verb|Twofish::Base::UncheckedSetKey()|---as it happened,\\
%one was \verb|Twofish::Enc::ProcessAndXorBlock()|, another---\verb|Twofish::Dec::ProcessAndXorBlock()|.

\subsection{kleine Funktionen} 

Sehr kleine oder leere Funktionen  (\myref{empty_func})
oder Funktionen die nur ``true'' (1) oder ``false'' (0) (\myref{ret_val_func}) sind weit verbreitet,
und fast jeder ordentlicher Compiler tendiert dazu nur solche Funktionen in den resultierenden ausf\"uhrbaren Code zu stecken,
sogar wenn es mehrere gleiche Funktionen im Source Code bereits gibt. 
Also, wann immer man solche kleinen Funktionen sieht die z.B nur aus \TT{mov eax, 1 / ret} bestehen und von mehreren 
Orten aus referenziert werden (und aufgerufen werden k\"onnen), und scheinbar keine Verbindung zu einander haben, dann 
ist das wahrscheinlich das Ergebnis einer Optimierung. 

\subsection{\Cpp}

\ac{RTTI}~(\myref{RTTI})-data ist vielleicht auch n\"utzlich f\"ur die \Cpp Klassen Identifikation.
}
\FR{\chapter{Trouver des choses importantes/intéressantes dans le code}

Le minimalisme n'est pas une caractéristique prépondérante des logiciels modernes.

\myindex{\Cpp!STL}

Pas parce que les programmeurs écrivent beaucoup, mais parce que de nombreuses bibliothèques
sont couramment liées statiquement aux fichiers exécutable.
Si toutes les bibliothèques externes étaient déplacées dans des fichiers DLL externes,
le monde serait différent. (Une autre raison pour C++ sont la \ac{STL} et autres
bibliothèques templates.)

\newcommand{\FOOTNOTEBOOST}{\footnote{\url{http://go.yurichev.com/17036}}}
\newcommand{\FOOTNOTELIBPNG}{\footnote{\url{http://go.yurichev.com/17037}}}

Ainsi, il est très important de déterminer l'origine de la fonction, si elle provient
d'une bibliothèque standard ou d'une bibliothèque bien connue (comme Boost\FOOTNOTEBOOST,
libpng\FOOTNOTELIBPNG), ou si elle est liée à ce que l'on essaye de trouver dans
le code.

Il est simplement absurde de tout récrire le code en \CCpp pour trouver ce que l'on
cherche.

Une des premières tâches d'un rétro-ingénieur est de trouver rapidement le code dont
il a besoin.

\myindex{\GrepUsage}

Le dés-assembleur \IDA nous permet de chercher parmi les chaînes de texte, les séquences
d'octets et les constantes.
Il est même possible d'exporter le code dans un fichier texte .lst ou .asm et d'utiliser
\TT{grep}, \TT{awk}, etc.

Lorsque vous essayez de comprendre ce que fait un certain code, ceci peut être facile
avec une bibliothèque open-source comme libpng.
Donc, lorsque vous voyez certaines constantes ou chaînes de texte qui vous semblent
familières, il vaut toujours la peine de les \emph{googler}.
Et si vous trouvez le projet open-source où elles sont utilisées, alors il suffit
de comparer les fonctions.
Ceci peut permettre de résoudre certaines parties du problème.

Par exemple, si un programme utilise des fichiers XML, la premières étape peut-être
de déterminer quelle bibliothèque XML est utilisée pour le traitement, puisque les
bibliothèques standards (ou bien connues) sont en général utilisées au lieu de code
fait maison.

\myindex{SAP}
\myindex{Windows!PDB}

Par exemple, j'ai essayé une fois de comprendre comment la compression/décompression
des paquets réseau fonctionne dans SAP 6.0.
C'est un logiciel gigantesque, mais un .\gls{PDB} détaillé avec des informations
de débogage est présent, et c'est pratique.
J'en suis finalement arrivé à l'idée que l'une des fonctions, qui était appelée par
\emph{CsDecomprLZC}, effectuait la décompression des paquets réseau.
Immédiatement, j'ai essayé de googler le nom et rapidement trouvé que la fonction
était utilisée dans MaxDB (c'est un projet open-source de SAP)
\footnote{Plus sur ce sujet dans la section concernée~(\myref{sec:SAPGUI})}.

\url{http://www.google.com/search?q=CsDecomprLZC}

Étonnement, les logiciels MaxDB et SAP 6.0 partagent du code comme ceci pour la compression/
décompression des paquets réseau.

\mysection{Identification de fichiers exécutables}

\subsection{Microsoft Visual C++}
\label{MSVC_versions}

Les versions de MSVC et des DLLs peuvent être importées:

%\small
\begin{center}
\begin{tabular}{ | l | l | l | l | l | }
\hline
\HeaderColor Marketing ver. &
\HeaderColor Internal ver. &
\HeaderColor CL.EXE ver. &
\HeaderColor DLLs imported &
\HeaderColor Release date \\
\hline
% 4.0, April 1995
% 97 & 5.0 & February 1997
6		&  6.0	& 12.00	& msvcrt.dll	& June 1998		\\
		&	&	& msvcp60.dll	&			\\
\hline
.NET (2002)	&  7.0	& 13.00	& msvcr70.dll	& February 13, 2002	\\
		&	&	& msvcp70.dll	&			\\
\hline
.NET 2003	&  7.1	& 13.10 & msvcr71.dll	& April 24, 2003	\\
		&	&	& msvcp71.dll	&			\\
\hline
2005		&  8.0	& 14.00 & msvcr80.dll	& November 7, 2005	\\
		&	&	& msvcp80.dll	&			\\
\hline
2008		&  9.0	& 15.00 & msvcr90.dll	& November 19, 2007	\\
		&	&	& msvcp90.dll	&			\\
\hline
2010		& 10.0	& 16.00 & msvcr100.dll	& April 12, 2010 	\\
		&	&	& msvcp100.dll	&			\\
\hline
2012		& 11.0	& 17.00 & msvcr110.dll	& September 12, 2012 	\\
		&	&	& msvcp110.dll	&			\\
\hline
2013		& 12.0	& 18.00 & msvcr120.dll	& October 17, 2013 	\\
		&	&	& msvcp120.dll	&			\\
\hline
\end{tabular}
\end{center}
%\normalsize

msvcp*.dll contient des fonctions relatives à \Cpp{}, donc si elle est importées,
il s'agit probablement d'un programme \Cpp.

\subsubsection{Mangling de nom}

Les noms commencent en général par le symbole \TT{?}.

Vous trouverez plus d'informations le \glslink{name mangling}{mangling de nom} de
MSVC ici: \myref{namemangling}.

\subsection{GCC}
\myindex{GCC}

À part les cibles *NIX, GCC est aussi présent dans l'environnement win32, sous la
forme de Cygwin et MinGW.

\subsubsection{Mangling de nom}

Les noms commencent en général par le symbole \TT{\_Z}.
Vous trouverez plus d'informations le \glslink{name mangling}{mangling de nom} de
GCC ici: \myref{namemangling}.
\subsubsection{Cygwin}
\myindex{Cygwin}

cygwin1.dll est souvent importée.

\subsubsection{MinGW}
\myindex{MinGW}

msvcrt.dll peut être importée.

\subsection{Intel Fortran}
\myindex{Fortran}

libifcoremd.dll, libifportmd.dll et libiomp5md.dll (support OpenMP) peuvent être importées.

libifcoremd.dll a beaucoup de fonctions préfixées par \TT{for\_}, qui signifie \emph{Fortran}.

\subsection{Watcom, OpenWatcom}
\myindex{Watcom}
\myindex{OpenWatcom}

\subsubsection{Mangling de nom}

Les noms commencent usuellement par le symbole \TT{W}.

Par exemple, ceci est la façon dont la méthode nommées \q{method} de la classe \q{class}
qui n'a pas d'argument et qui renvoie \Tvoid est encodée:

\begin{lstlisting}
W?method$_class$n__v
\end{lstlisting}

\subsection{Borland}
\myindex{Borland Delphi}
\myindex{Borland C++Builder}

Voici un exemple de \glslink{name mangling}{mangling de nom} de Delphi de Borland
et de C++Builder:

\lstinputlisting{digging_into_code/identification/borland_mangling.txt}

Les noms commencent toujours avec le symbole \TT{@}, puis nous avons le nom de la
classe, de la méthode et les types des arguments de méthode encodés.

Ces noms peuvent être dans des imports .exe, des exports .dll, des données de débogage,
etc.

Les Borland Visual Component Libraries (VCL) sont stockées dans des fichiers .bpl
au lieu de .dll, par exemple, vcl50.dll, rtl60.dll.

Une autre DLL qui peut être importée: BORLNDMM.DLL.

\subsubsection{Delphi}

Presque tous les exécutables Delpi ont la chaîne de texte \q{Boolean} au début de
leur segment de code, ainsi que d'autres noms de type.

Ceci est le début très typique du segment \TT{CODE} d'un programme Delphi, ce bloc
vient juste après l'entête de fichier win32 PE:

\lstinputlisting{digging_into_code/identification/delphi.txt}

Les 4 premiers octets du segment de données (\TT{DATA}) peuvent être \TT{00 00 00 00},
\TT{32 13 8B C0} ou \TT{FF FF FF FF}.%

Cette information peut être utile lorsque l'on fait face à des exécutables Delphi
préparés/chiffrés.

\subsection{Autres DLLs connues}

\myindex{OpenMP}
\begin{itemize}
\item vcomp*.dll---implémentation d'OpenMP de Microsoft.
\end{itemize}


% binary files might be also here

\mysection{Communication avec le monde extérieur (niveau fonction)}
Il est souvent recommandé de suivre les arguments de la fonction et sa valeur de
retour dans un débogueur ou \ac{DBI}.
Par exemple, l'auteur a essayé une fois de comprendre la signification d'une fonction
obscure, qui s'est avérée être un tri à bulles mal implémenté\footnote{\url{https://yurichev.com/blog/weird_sort_KLEE/}}.
(Il fonctionnait correctement, mais plus lentement.)
En même temps, regarder les entrées et sorties de cette fonction aide instantanément
à comprendre ce quelle fait.

Souvent, lorsque vous voyez une division par la multiplication (\myref{sec:divisionbymult}),
mais avez oublié tous les détails du mécanisme, vous pouvez seulement observer l'entrée
et la sortie, et trouver le diviseur rapidement.

% sections:
\input{digging_into_code/communication_win32_FR}
\mysection{Chaînes}
\label{sec:digging_strings}

\input{digging_into_code/strings/main_FR}

\subsection{Trouver des chaînes dans un binaire}

\epigraph{Actually, the best form of Unix documentation is frequently running the
\textbf{strings} command over a program’s object code. Using \textbf{strings}, you can get
a complete list of the program’s hard-coded file name, environment variables,
undocumented options, obscure error messages, and so forth.}{The Unix-Haters Handbook}
En fait, la meilleure forme de documentation Unix est de lancer la commande
\textbf{strings} sur le code objet d'un programme. En utilisant \textbf{strings},
vous obtenez une liste complète des noms de fichiers codés en dur dans le programme,
les variables d'environnement, les options non documentées, les messages d'erreurs
méconnus et ainsi de suite.

\myindex{UNIX!strings}
L'utilitaire standard UNIX \emph{strings} est un moyen rapide et facile de voir les
chaînes dans un fichier.
Par exemple, voici quelques chaînes du fichier exécutable sshd d'OpenSSH 7.2:

\lstinputlisting{digging_into_code/sshd_strings.txt}

Il y a des options, des messages d'erreur, des chemins de fichier, des modules et
des fonctions importés dynamiquement, ainsi que d'autres chaînes étranges (clefs?).
Il y a aussi du bruit illisible---le code x86 à parfois des fragments constitués de
caractères ASCII imprimables, jusqu'à ~8 caractères.

Bien sûr, OpenSSH est un programme open-source.
Mais regarder les chaînes lisibles dans un binaire inconnu est souvent une première
étape d'analyse.
\myindex{UNIX!grep}

\emph{grep} peut aussi être utilisé.

\myindex{Hiew}
\myindex{Sysinternals}
Hiew a la même capacité (Alt-F6), ainsi que ProcessMonitor de Sysinternals.

\subsection{Messages d'erreur/de débogage}

Les messages de débogage sont très utiles s'il sont présents.
Dans un certain sens, les messages de débogage rapportent ce qui est en train de
se passer dans le programme. Souvent, ce sont des fonctions \printf-like, qui écrivent
des fichiers de log, ou parfois elles n'écrivent rien du tout mais les appels sont
toujours présents puisque le build n'est pas un de débogage mais de \emph{release}.
\myindex{\oracle}

Si des variables locales ou globales sont affichées dans les messages, ça peut être
aussi utile, puisqu'il est possible d'obtenir au moins le nom de la variable.
Par exemple, une telle fonction dans \oracle est \TT{ksdwrt()}.

Des chaînes de texte significatives sont souvent utiles.
Le dés-assembleur \IDA peut montrer depuis quelles fonctions et depuis quel endroit
cette chaîne particulière est utilisée.
Des cas drôles arrivent parfois\footnote{\href{http://go.yurichev.com/17223}{blog.yurichev.com}}.

Le message d'erreur peut aussi nous aider.
Dans \oracle, les erreurs sont rapportées en utilisant un groupe de fonctions.\\
Vous pouvez en lire plus ici: \href{http://go.yurichev.com/17224}{blog.yurichev.com}.

\myindex{Error messages}

Il est possible de trouver rapidement quelle fonction signale une erreur et dans
quelles conditions.

À propos, ceci est souvent la raison pour laquelle les systèmes de protection contre
la copie utilisent des messages d'erreur inintelligibles ou juste des numéros d'erreur.
Personne n'est content lorsque le copieur de logiciel comprend comment fonctionne
la protection contre la copie seulement en lisant les messages d'erreur.

Un exemple de messages d'erreur chiffrés se trouve ici: \myref{examples_SCO}.

\subsection{Chaînes magiques suspectes}

Certaines chaînes magique sont d'habitude utilisées dans les porte dérobées semblent
vraiment suspectes.

Par exemple, il y avait une porte dérobée dans le routeur personnel TP-Link WR740%
\footnote{\url{http://sekurak.pl/tp-link-httptftp-backdoor/}}.
La porte dérobée était activée en utilisant l'URL suivante:\\
\url{http://192.168.0.1/userRpmNatDebugRpm26525557/start_art.html}.\\

En effet, la chaîne \q{userRpmNatDebugRpm26525557} est présente dans le firmware.

Cette chaîne n'était pas googlable jusqu'à la large révélation d'information concernant
la porte dérobée.

Vous ne trouverez ceci dans aucun \ac{RFC}.

Vous ne trouverez pas d'algorithme informatique qui utilise une séquence d'octets
aussi étrange.

Et elle ne ressemble pas à une erreur ou un message de débogage.

Donc, c'est une bonne idée d'inspecter l'utilisation de ce genre de chaînes bizarres.\\
\\
\myindex{base64}

Parfois, de telles chaînes sont encodées en utilisant base64.

Donc, c'est une bonne idée de toutes les décoder et de les inspecter visuellement,
même un coup d'\oe{}il doit suffire.\\
\\
\myindex{Sécurité par l'obscurité}
Plus précisément, cette méthode de cacher des accès non documentés est appelée \q{sécurité par l'obscurité}.


\input{digging_into_code/assert_FR}
\mysection{Constantes}

Les humains, programmeurs inclus, utilisent souvent des nombres ronds, comme 10, 100,
1000, dans la vie courante comme dans le code.

Le rétro ingénieur pratiquant connaît en général bien leur représentation décimale:
10=0xA, 100=0x64, 1000=0x3E8, 10000=0x2710.

Les constantes \TT{0xAAAAAAAA} (0b10101010101010101010101010101010) et \\
\TT{0x55555555} (0b01010101010101010101010101010101)  sont aussi répandues---elles
sont composées d'alternance de bits.

Cela peut aider à distinguer un signal d'un signal dans lequel tous les bits sont
à 1 (0b1111 \dots) ou à 0 (0b0000 \dots).
Par exemple, la constante \TT{0x55AA} est utilisée au moins dans le secteur de boot,
\ac{MBR}, et dans la \ac{ROM} de cartes d'extention de compatible IBM.

Certains algorithmes, particulièrement ceux de chiffrement, utilisent des constantes
distinctes, qui sont faciles à trouver dans le code en utilisant \IDA.

\myindex{MD5}

Par exemple, l'algorithme MD5 initialise ses propres
variables internes comme ceci:

\begin{verbatim}
var int h0 := 0x67452301
var int h1 := 0xEFCDAB89
var int h2 := 0x98BADCFE
var int h3 := 0x10325476
\end{verbatim}

Si vous trouvez ces quatre constantes utilisées à la suite dans du code, il est
très probable que cette fonction soit relatives à MD5.

\par Un autre exemple sont les algorithmes CRC16/CRC32, ces algorithmes de calcul
utilisent souvent des tables pré-calculées comme celle-ci:

\begin{lstlisting}[caption=linux/lib/crc16.c,style=customc]
/** CRC table for the CRC-16. The poly is 0x8005 (x^16 + x^15 + x^2 + 1) */
u16 const crc16_table[256] = {
	0x0000, 0xC0C1, 0xC181, 0x0140, 0xC301, 0x03C0, 0x0280, 0xC241,
	0xC601, 0x06C0, 0x0780, 0xC741, 0x0500, 0xC5C1, 0xC481, 0x0440,
	0xCC01, 0x0CC0, 0x0D80, 0xCD41, 0x0F00, 0xCFC1, 0xCE81, 0x0E40,
	...
\end{lstlisting}

Voir aussi la table pré-calculée pour CRC32: \myref{sec:CRC32}.

Dans les algorithmes CRC sans table, des polynômes bien connus sont utilisés, par
exemple 0xEDB88320 pour CRC32.

\subsection{Nombres magiques}
\label{magic_numbers}

De nombreux formats de fichier définissent un entête standard où un \emph{nombre(s) magique}
est utilisé, unique ou même plusieurs.

\myindex{MS-DOS}

Par exemple, tous les exécutables Win32 et MS-DOS débutent par ces deux caractères \q{MZ}\footnote{\href{http://go.yurichev.com/17113}{Wikipédia}}.

\myindex{MIDI}

Au début d'un fichier MIDI, la signature \q{MThd} doit être présente.
Si nous avons un programme qui utilise des fichiers MIDI pour quelque chose, il
est très probable qu'il doit vérifier la validité du fichier en testant au moins
les 4 premiers octets.

Ça peut être fait comme ceci:
(\emph{buf} pointe sur le début du fichier chargé en mémoire)

\begin{lstlisting}[style=customasmx86]
cmp [buf], 0x6468544D ; "MThd"
jnz _error_not_a_MIDI_file
\end{lstlisting}

\myindex{\CStandardLibrary!memcmp()}
\myindex{x86!\Instructions!CMPSB}

\dots ou en appelant une fonction pour comparer des blocs de mémoire comme \TT{memcmp()}
ou tout autre code équivalent jusqu'à une instruction \TT{CMPSB} (\myref{REPE_CMPSx}).

Lorsque vous trouvez un tel point, vous pouvez déjà dire que le chargement du fichier
MIDI commence, ainsi, vous pouvez voir l'endroit où se trouve le buffer avec le contenu
du fichier MIDI, ce qui est utilisé dans le buffer et comment.

\subsubsection{Dates}

\myindex{UFS2}
\myindex{FreeBSD}
\myindex{HASP}

Souvent, on peut rencontrer des nombres comme \TT{0x19870116}, qui ressemble clairement
à une date (année 1987, 1er mois (janvier), 16ème jour).
Ça peut être la date de naissance de quelqu'un (un programmeur, une de ses relations,
un enfant), ou une autre date importante.
La date peut aussi être écrite dans l'ordre inverse, comme \TT{0x16011987}.
Les dates au format américain sont aussi courante, comme \TT{0x01161987}.

Un exemple célèbre est \TT{0x19540119} (nombre magique utilisé dans la structure
du super-bloc UFS2), qui est la date de naissance de Marshall Kirk McKusick, éminent
contributeur FreeBSD.

\myindex{Stuxnet}
Stuxnet utilise le nombre ``19790509'' (pas comme un nombre 32-bit, mais comme une
chaîne, toutefois), et ça a conduit à spéculer que le malware était relié à Israël%
\footnote{C'est la date d'exécution de Habib Elghanian, juif persan.}.

Aussi, des nombres comme ceux-ci sont très répandus dans dans le chiffrement niveau
amateur, par exemple, extrait de la \emph{fonction secrète} des entrailles du dongle
HASP3\footnote{\url{https://web.archive.org/web/20160311231616/http://www.woodmann.com/fravia/bayu3.htm}}:

\begin{lstlisting}[style=customc]
void xor_pwd(void)
{
	int i;

	pwd^=0x09071966;
	for(i=0;i<8;i++)
	{
		al_buf[i]= pwd & 7; pwd = pwd >> 3;
	}
};

void emulate_func2(unsigned short seed)
{
	int i, j;
	for(i=0;i<8;i++)
	{
		ch[i] = 0;

		for(j=0;j<8;j++)
		{
			seed *= 0x1989;
			seed += 5;
			ch[i] |= (tab[(seed>>9)&0x3f]) << (7-j);
		}
	}
}
\end{lstlisting}

\subsubsection{DHCP}

Ceci s'applique aussi aux protocoles réseaux.
Par exemple, les paquets réseau du protocole DHCP contiennent un soi-disant \emph{nombre
magique}: \TT{0x63538263}.
Tout code qui génère des paquets DHCP doit contenir quelque part cette constante
à insérer dans les paquets.
Si nous la trouvons dans du code, nous pouvons trouver ce qui s'y passe, et pas seulement ça.
Tout programme qui peut recevoir des paquet DHCP doit vérifier le \emph{cookie magique},
et le comparer à cette constante.

Par exemple, prenons le fichier dhcpcore.dll de Windows 7 x64 et cherchons cette constante.
Et nous la trouvons, deux fois:
Il semble que la constante soit utilisée dans deux fonctions avec des noms parlants\\
\TT{DhcpExtractOptionsForValidation()} et \TT{DhcpExtractFullOptions()}:

\begin{lstlisting}[caption=dhcpcore.dll (Windows 7 x64),style=customasmx86]
.rdata:000007FF6483CBE8 dword_7FF6483CBE8 dd 63538263h          ; DATA XREF: DhcpExtractOptionsForValidation+79
.rdata:000007FF6483CBEC dword_7FF6483CBEC dd 63538263h          ; DATA XREF: DhcpExtractFullOptions+97
\end{lstlisting}

Et ici sont les endroits où ces constantes sont accédées:

\begin{lstlisting}[caption=dhcpcore.dll (Windows 7 x64),style=customasmx86]
.text:000007FF6480875F  mov     eax, [rsi]
.text:000007FF64808761  cmp     eax, cs:dword_7FF6483CBE8
.text:000007FF64808767  jnz     loc_7FF64817179
\end{lstlisting}

Et:

\begin{lstlisting}[caption=dhcpcore.dll (Windows 7 x64),style=customasmx86]
.text:000007FF648082C7  mov     eax, [r12]
.text:000007FF648082CB  cmp     eax, cs:dword_7FF6483CBEC
.text:000007FF648082D1  jnz     loc_7FF648173AF
\end{lstlisting}

\subsection{Constantes spécifiques}

Parfois, il y a une constante spécifique pour un certain type de code.
Par exemple, je me suis plongé une fois dans du code, où le nombre 12 était rencontré
anormalement souvent.
La taille de nombreux tableaux était 12 ou un multiple de 12 (24, etc.).
Il s'est avéré que ce code prenait des fichiers audio de 12 canaux en entrée et les
traitait.

Et vice versa: par exemple, si un programme fonctionne avec des champs de texte qui
ont une longueur de 120 octets, il doit y avoir une constante 120 ou 119 quelque
part dans le code.
Si UTF-16 est utilisé, alors $2 \cdot 120$.
Si le code fonctionne avec des paquets réseau de taille fixe, c'est une bonne idée
de chercher cette constante dans le code.

C'est aussi vrai pour le chiffrement amateur (clefs de licence, etc.).
Si le bloc chiffré a une taille de $n$ octets, vous pouvez essayer de trouver des
occurrences de ce nombre à travers le code.
Aussi, si vous voyez un morceau de code qui est répété $n$ fois dans une boucle durant
l'exécution, ceci peut être une routine de chiffrement/déchiffrement.

\subsection{Chercher des constantes}

C'est facile dans \IDA: Alt-B or Alt-I.
\myindex{binary grep}
Et pour chercher une constante dans un grand nombre de fichiers, ou pour chercher
dans des fichiers non exécutables, il y a un petit utilitaire appelé \emph{binary grep}\footnote{\BGREPURL}.


\input{digging_into_code/instructions_FR}
\mysection{Patterns de code suspect}

\subsection{instructions XOR}
\myindex{x86!\Instructions!XOR}

Des instructions comme \TT{XOR op, op} (par exemple, \TT{XOR EAX, EAX}) sont utilisées
en général pour mettre la valeur d'un registre à zéro, mais si les opérandes sont
différentes, l'opération \q{ou exclusif} est exécutée.

Cette opération est rare en programmation courante, mais répandu en cryptographie,
y compris amateur.
C'est particulièrement suspect si le second opérande est un grand nombre.

Ceci peut indiquer du chiffrement/déchiffrement, du calcul de somme de contrôle, etc.\\
\\

Une exception à cette observation, qu'il est utile de noter, est le \q{canari} (\myref{subsec:BO_protection}).
Sa génération et sa vérification sont souvent effectuées en utilisant des instructions
\XOR. \\
\\
\myindex{AWK}

Ce script awk peut être utilisé pour traité les fichiers listing (.lst) d'\IDA:

\lstinputlisting{digging_into_code/awk.sh}

Il est aussi utile de noter que ce type de script peut aussi rapporter du code mal
désassemblé
(\myref{sec:incorrectly_disasmed_code}).

\subsection{Code assembleur écrit à la main}

\myindex{Function prologue}
\myindex{Function epilogue}
\myindex{x86!\Instructions!LOOP}
\myindex{x86!\Instructions!RCL}

Les compilateurs modernes ne génèrent pas les instructions \TT{LOOP} et \TT{RCL}.
D'un autre côté, ces instructions sont très connues des codeurs qui aiment écrire
directement en langage d'assemblage.
Si vous les rencontrez, on peut dire qu'il est très probable que ce morceau de code
ait été écrit à la main.
De telles instructions sont marquées avec un (M) dans la liste des instructions de
cet appendice: \myref{sec:x86_instructions}.

\par
De même, les prologue/épilogue de fonction sont rares dans de l'assembleur écrit
à la main.

\par
Il n'y a généralement pas de système fixé pour le passage des arguments aux fonctions
dans du code écrit à la main.

\par
Exemple du noyau de Windows 2003
(ntoskrnl.exe file):

\lstinputlisting[style=customasmx86]{digging_into_code/ntoskrnl.lst}

En effet, si nous regardons dans le code source de \ac{WRK} v1.2, ce code peut être
trouvé facilement dans le fichier \\
\emph{WRK-v1.2\textbackslash{}base\textbackslash{}ntos\textbackslash{}ke\textbackslash{}i386\textbackslash{}cpu.asm}.

\input{digging_into_code/magic_numbers_tracing_FR}
\mysection{Boucles}

À chaque fois que votre programme travaille avec des sortes de fichier, ou un buffer
d'une certaine taille, il doit s'agir d'un sorte de boucle de déchiffrement/traitement
à l'intérieur du code.

Ceci est un exemple réel de sortie de l'outil \tracer.
Il y avait un code qui chargeait une sorte de fichier chiffré de 258 octets.
Je l'ai lancé dans l'intention d'obtenir le nombre d'exécution de chaque instruction
(l'outil \ac{DBI} irait beaucoup mieux de nos jours).
Et j'ai rapidement trouvé un morceau de code qui était exécuté 259/258 fois:

\lstinputlisting{digging_into_code/crypto_loop.txt}

Il s'avère qu'il s'agit de la boucle de déchiffrement.


% TODO move section...

\subsection{Quelques schémas de fichier binaire}

Tous les exemples ici ont été préparé sur Windows, avec la page de code 437 activée%
dans la console.
L'intérieur des fichiers binaires peut avoir l'air différent avec une autre page
de code.

\clearpage
\subsubsection{Tableaux}

Parfois, nous pouvons clairement localiser visuellement un tableau de valeurs 16/32/64-bit,
dans un éditeur hexadécimal.

Voici un exemple de tableau de valeurs 16-bit.
Nous voyons que le premier octet d'une paire est 7 ou 8, et que le second semble
aléatoire:

\begin{figure}[H]
\centering
\myincludegraphics{digging_into_code/binary/16bit_array.png}
\caption{FAR: tableau de valeurs 16-bit}
\end{figure}

J'ai utilisé un fichier contenant un signal 12-canaux numérisé en utilisant 16-bit \ac{ADC}.

\clearpage
\myindex{MIPS}
\par Et voici un exemple de code MIPS très typique.

Comme nous pouvons nous en souvenir, chaque instruction MIPS (et aussi ARM en mode
ARM ou ARM64) a une taille de 32 bits (ou 4 octets), donc un tel code est un tableau
de valeurs 32-bit.

En regardant cette copie d'écran, nous voyons des sortes de schémas.

Les lignes rouge verticales ont été ajoutées pour la clarté:

\begin{figure}[H]
\centering
\myincludegraphics{digging_into_code/binary/typical_MIPS_code.png}
\caption{Hiew: code MIPS très typique}
\end{figure}

Il y a un autre exemple de tel schéma ici dans le livre:
\myref{Oracle_SYM_files_example}.

\clearpage
\subsubsection{Fichiers clairsemés}

Ceci est un fichier clairsemé avec des données éparpillées dans un fichier presque vide.
Chaque caractère espace est en fait l'octet zéro (qui rend comme un espace).
Ceci est un fichier pour programmer des FPGA (Altera Stratix GX device).
Bien sûr, de tels fichiers peuvent être compressés facilement, mais des formats comme
celui-ci sont très populaire dans les logiciels scientifiques et d'ingénierie, où
l'efficience des accès est importante, tandis que la compacité ne l'est pas.

\begin{figure}[H]
\centering
\myincludegraphics{digging_into_code/binary/sparse_FPGA.png}
\caption{FAR: Fichier clairsemé}
\end{figure}

\clearpage
\subsubsection{Fichiers compressés}

% FIXME \ref{} ->
Ce fichier est juste une archive compressée.
Il a une entropie relativement haute et visuellement, il à l'air chaotique.
Ceci est ce à quoi ressemble les fichiers compressés et/ou chiffrés.

\begin{figure}[H]
\centering
\myincludegraphics{digging_into_code/binary/compressed.png}
\caption{FAR: Fichier compressé}
\end{figure}

\clearpage
\subsubsection{\ac{CDFS}}

Les fichiers d'installation d'un \ac{OS} sont en général distribués sous forme de
fichiers ISO, qui sont des copies de disques CD/DVD.
Le système de fichiers utilisé est appelé \ac{CDFS}, ce que vous voyez ici sont des
noms de fichiers mixés avec des données additionnelles.
Ceci peut-être la taille des fichiers, des pointeurs sur d'autres répertoires, des
attributs de fichier, etc.
C'est l'aspect typique de ce à quoi ressemble un système de fichiers en interne.

\begin{figure}[H]
\centering
\myincludegraphics{digging_into_code/binary/cdfs.png}
\caption{FAR: Fichier ISO: \ac{CD} d'installation d'Ubuntu 15}
\end{figure}

\clearpage
\subsubsection{Code exécutable x86 32-bit}

Voici l'allure de code exécutable x86 32-bit.
Il n'a pas une grande entropie, car certains octets reviennent plus souvent que d'autres.

\begin{figure}[H]
\centering
\myincludegraphics{digging_into_code/binary/x86_32.png}
\caption{FAR: Code exécutable x86 32-bit}
\end{figure}

% TODO: Read more about x86 statistics: \ref{}. % FIXME blog post about decryption...

\clearpage
\subsubsection{Fichiers graphique BMP}

% TODO: bitmap, bit, group of bits...

Les fichiers BMP ne sont pas compressés, donc chaque octet (ou groupe d'octet) représente
chaque pixel.
J'ai trouvé cette image quelque part dans mon installation de Windows 8.1:

\begin{figure}[H]
\centering
\myincludegraphicsSmall{digging_into_code/binary/bmp.png}
\caption{Image exemple}
\end{figure}

Vous voyez que cette image a des pixels qui ne doivent pas pouvoir être compressés
beaucoup (autour du centre), mais il y a de longues lignes monochromes au haut et
en bas.
En effet, de telles lignes ressemblent à des lignes lorsque l'on regarde le fichier:

\begin{figure}[H]
\centering
\myincludegraphics{digging_into_code/binary/bmp_FAR.png}
\caption{Fragment de fichier BMP}
\end{figure}


% FIXME comparison!
\subsection{Comparer des \q{snapshots} mémoire}
\label{snapshots_comparing}

La technique consistant à comparer directement deux états mémoire afin de voir les
changements était souvent utilisée pour tricher avec les jeux sur ordinateurs 8-bit
et pour modifier le fichiers des \q{meilleurs scores}.

Par exemple, si vous avez chargé un jeu sur un ordinateur 8-bit (il n'y a pas beaucoup
de mémoire dedans, mais le jeu utilise en général encore moins de mémoire), et que
vous savez que vous avez maintenant, disons, 100 balles, vous pouvez faire un \q{snapshot}
de toute la mémoire et le sauver quelque part. Puis, vous tirez une fois, le compteur
de balles descend à 99, faites un second \q{snapshot} et puis comparer les deux:
il doit y avoir quelque part un octet qui était à 100 au début, et qui est maintenant
à 99.

En considérant le fait que ces jeux 8-bit étaient souvent écrits en langage d'assemblage
et que de telles variables étaient globales, on peut déterminer avec certitude quelle
adresse en mémoire contenait le compteur de balles. Si vous cherchiez toutes les références
à cette adresse dans le code du jeu désassemblé, il n'était pas très difficile de
trouver un morceau de code \glslink{decrement}{décrémentant} le compteur de balles,
puis d'y écrire une, ou plusieurs, instruction \gls{NOP}, et d'avoir un jeu avec
toujours 100 balles.
\myindex{BASIC!POKE}
Les jeux sur ces ordinateurs 8-bit étaient en général chargés à une adresse constante,
aussi, il n'y avait pas beaucoup de versions ce chaque jeu (souvent, une seule version
était répandue pour un long moment), donc les joueurs enthousiastes savaient à quelles
adresses se trouvaient les octets qui devaient être modifiés (en utilisant l'instruction
BASIC \gls{POKE}) pour le bidouiller. Ceci à conduit à des listes de \q{cheat} qui
contenaient les instructions \gls{POKE} publiées dans des magazines relatifs aux
jeux 8-bit. Voir aussi: \href{http://go.yurichev.com/17114}{Wikipédia}.

\myindex{MS-DOS}

De même, il est facile de modifier le fichier des \q{meilleurs scores}, ceci ne fonctionne
pas seulement avec des jeux 8-bit. Notez votre score et sauvez le fichier quelque part.
Lorsque le décompte des \q{meilleurs scores} devient différent, comparez juste les
deux fichiers, ça peut même être fait avec l'utilitaire DOS FC\footnote{Utilitaire
MS-DOS pour comparer des fichiers binaires.} (les fichiers des \q{meilleurs scores}
sont souvent au format binaire).

Il y aura un endroit où quelques octets seront différents et il est facile de voir
lesquels contiennent le score.
Toutefois, les développeurs de jeux étaient conscient de ces trucs et pouvaient protéger
le programme contre ça.

Exemple quelque peu similaire dans ce livre: \myref{Millenium_DOS_game}.

% TODO: пример с какой-то простой игрушкой?

\subsubsection{Une histoire vraie de 1999}

\myindex{ICQ}
C'était un temps de l'engouement pour la messagerie ICQ, au moins dans les pays de
l'ex-URSS.
Cette messagerie avait une particularité --- certains utilisateurs ne voulaient pas
partager leur état en ligne avec tout le monde.
Et vous deviez demander une \emph{autorisation} à cet utilisateur.
Il pouvait vous autoriser à voir son état, ou pas.

Voici ce que j'ai fait:

\begin{itemize}
\item Ajouté un utilisateur.
\item Un utiliseur est apparu dans la liste de contact, dans la section ``attente d'autorisation''.
\item Fermé ICQ.
\item Sauvegardé la base de données ICQ.
\item Ouvert à nouveau ICQ.
\item L'utilisateur m'a \emph{autorisé}.
\item Refermé ICQ et comparé les deux base de données.
\end{itemize}

Il s'est avéré que: les deux bases de données ne différaient que d'un octet.
Dans la première version: \verb|RESU\x03|, dans la seconde: \verb|RESU\x02|.
(``RESU'', signifie probablement ``USER'', i.e., un entête d'une structure où toutes
les informations à propos d'un utilisateur étaient stockées.)
Cela signifie que l'information sur l'autorisation n'était pas stockée sur le serveur,
mais sur le client. Vraisemblablement, la valeur 2/3 reflétait l'état de l'\q{autorisation}.

\subsubsection{Registres de Windows}

Il est aussi possible de comparer les registres de Windows avant et après l'installation
d'un programme.

C'est une méthode courante  que de trouver quels sont les éléments des registres
utilisés par le programme. Peut-être que ceci est la raison pour laquelle le shareware
de \q{nettoyage des registres windows} est si apprécié.

À propos, voici comment sauver les registres de Windows dans des fichiers texte:

\begin{lstlisting}
reg export HKLM HKLM.reg
reg export HKCU HKCU.reg
reg export HKCR HKCR.reg
reg export HKU HKU.reg
reg export HKCC HKCC.reg
\end{lstlisting}

\myindex{UNIX!diff}
Ils peuvent être comparés en utilisant diff...

\subsubsection{Logiciels d'ingénierie, de CAO, etc.}

Si un logiciel utilise des fichiers propriétaires, vous pouvez aussi les examiner.
Sauvez un fichier.
Puis, ajouter un point ou une ligne ou une autre primitive.
Sauvez le fichier, comparez.
Ou déplacez un point, sauvez le fichier, comparez.

\subsubsection{Comparateur à clignotement}

La comparaison de fichiers ou d'images mémoire nous rappelle le comparateur à clignotement
\footnote{\url{http://go.yurichev.com/17348}}:
Un dispositif utilisé autrefois par les astronomes pour trouver les objets célestes
changeant de position.

Les comparateurs à clignotement permettent d'alterner rapidement entre deux photographies
prisent à des moments différents, de façon à faire apparaître les différences visuellement.

À propos, Pluton a été découverte avec un comparateur à clignotement en 1930.

\mysection{Détection de l'\ac{ISA}}
\label{ISA_detect}

Souvent, vous avez à faire à un binaire avec un \ac{ISA} inconnu.
Peut-être que la manière la plus facile de détecter l'\ac{ISA} est d'en essayer plusieurs
dans \IDA, objdump ou un autre désassembleur.

Pour réussir ceci, il faut comprendre la différence entre du code incorrectement
et celui correctement désassemblé.

% subsection:
\renewcommand{\CURPATH}{digging_into_code/incorrect_disassembly}
\input{digging_into_code/incorrect_disassembly/main_FR}

\subsection{Code désassemblé correctement}
\label{correctly_disasmed_code}

Chaque \ac{ISA} a une douzaine d'instructions les plus utilisées, toutes les autres
le sont beaucoup moins souvent.

Concernant le x86, il est intéressant de savoir le fait que les instructions d'appel
de fonctions (\PUSH/\CALL/\ADD) et \MOV sont les morceaux de code les plus fréquemment
exécutées dans presque tous les programmes que nous utilisons.
Autrement dit, le \ac{CPU} est très occupé à passer de l'information entre les niveaux
d'abstraction, ou, on peut dire qu'il est très occupé à commuter entre ces niveaux.
Indépendamment du type d'\ac{ISA}.
Ceci a un coût de diviser les problèmes entre plusieurs niveaux d'abstraction (ainsi
les êtres humain peuvent travailler plus facilement avec).



\mysection{Autres choses}

\subsection{Idée générale}

Un rétro-ingénieur doit essayer se se mettre dans la peau d'un programmeur aussi
souvent que possible.
Pour adopter son point de vue et se demander comment il aurait résolu des tâches
d'un cas spécifique.

\subsection{Ordre des fonctions dans le code binaire}

Toutes les fonctions situées dans un unique fichier .c ou .cpp sont compilées dans
le fichier objet (.o) correspondant.
Plus tard, l'éditeur de liens mets tous les fichiers dont il a besoin ensemble, sans
changer l'ordre ni les fonctions.
Par conséquent, si vous voyez deux ou plus fonctions consécutives, cela signifie
qu'elles étaient situées dans le même fichier source (à moins que vous ne soyez en
limite de deux fichiers objet, bien sûr).
Ceci signifie que ces fonctions ont quelque chose en commun, qu'elles sont des fonctions
du même niveau d'\ac{API}, de la même bibliothèque, etc.

\myindex{CryptoPP}
Ceci est une histoire vraie de pratique: il était une fois, alors que je cherchais
des fonctions relatives à Twofish dans un programme lié à la bibliothèque CryptoPP,
en particulier des fonctions de chiffrement/déchiffrement.\\
J'ai trouvé la fonction \verb|Twofish::Base::UncheckedSetKey()| mais pas d'autres.
Après avoir cherché dans le code source
\verb|twofish.cpp|\footnote{\url{https://github.com/weidai11/cryptopp/blob/b613522794a7633aa2bd81932a98a0b0a51bc04f/twofish.cpp}},
il devint clair que toutes les fonctions étaient situées dans ce module (\verb|twofish.cpp|).\\
Donc j'ai essayé toutes les fonctions qui suivaient \verb|Twofish::Base::UncheckedSetKey()|---comme elles arrivaient,\\
une a été \verb|Twofish::Enc::ProcessAndXorBlock()|, une autre---\verb|Twofish::Dec::ProcessAndXorBlock()|.

\subsection{Fonctions minuscules}

Les fonctions minuscules comme les fonctions vides (\myref{empty_func})
ou les fonctions qui renvoient juste ``true'' (1) ou ``false'' (0) (\myref{ret_val_func})
sont très communes, et presque tous les compilateurs corrects tendent à ne mettre
qu'une seule fonction de ce genre dans le code de l'exécutable résultant, même si
il y avait plusieurs fonctions similaires dans le code source.
Donc, à chaque fois que vous voyez une fonction minuscule consistant seulement en
\TT{mov eax, 1 / ret} qui est référencée (et peut être appelée) dans plusieurs endroits
qui ne semblent pas reliés les uns au autres, ceci peut résulter d'une telle optimisation.%

\subsection{\Cpp}

Les données \ac{RTTI}~(\myref{RTTI})- peuvent être utiles pour l'identification des
classes \Cpp.

%\subsection{Crash on purpose}
\subsection{Crash délibéré}

Souvent, vous voulez savoir quelle fonction a été exécutée, et laquelle ne l'a pas
été.
Vous pouvez utiliser un débogueur, mais sur des architectures exotiques, il peut
ne pas en avoir, donc la façon la plus simple est d'y mettre un opcode invalide,
ou quelque chose comme \INS{INT3} (0xCC).
Le crash signalera le fait que l'instruction a été exécutée.

Un autre exemple de crash délibéré: \myref{dmalloc_KILL_PROCESS}.

}

\EN{% TODO translate
\mysection{Breaking simple executable cryptor}

I've got an executable file which is encrypted by relatively simple encryption.
\href{\GitHubBlobMasterURL/examples/simple_exec_crypto/files/cipher.bin}{Here is it} (only executable section is left here).

First, all encryption function does is just adds number of position in buffer to the byte.
Here is how this can be encoded in Python:

\begin{lstlisting}[caption=Python script,style=custompy]
#!/usr/bin/env python
def e(i, k):
    return chr ((ord(i)+k) % 256)

def encrypt(buf):
    return e(buf[0], 0)+ e(buf[1], 1)+ e(buf[2], 2) + e(buf[3], 3)+ e(buf[4], 4)+ e(buf[5], 5)+ e(buf[6], 6)+ e(buf[7], 7)+
           e(buf[8], 8)+ e(buf[9], 9)+ e(buf[10], 10)+ e(buf[11], 11)+ e(buf[12], 12)+ e(buf[13], 13)+ e(buf[14], 14)+ e(buf[15], 15)
\end{lstlisting}

Hence, if you encrypt buffer with 16 zeros, you'll get \emph{0, 1, 2, 3 ... 12, 13, 14, 15}.

\myindex{Propagating Cipher Block Chaining}
Propagating Cipher Block Chaining (PCBC) is also used, here is how it works:

\begin{figure}[H]
\centering
\myincludegraphics{examples/simple_exec_crypto/601px-PCBC_encryption.png}
\caption{Propagating Cipher Block Chaining encryption (image is taken from Wikipedia article)}
\end{figure}

The problem is that it's too boring to recover IV (Initialization Vector) each time.
Brute-force is also not an option, because IV is too long (16 bytes).
Let's see, if it's possible to recover IV for arbitrary encrypted executable file?

Let's try simple frequency analysis.
This is 32-bit x86 executable code, so let's gather statistics about most frequent bytes and opcodes.
I tried huge oracle.exe file from Oracle RDBMS version 11.2 for windows x86 and I've found that the most frequent byte (no surprise) is zero (~10\%).
The next most frequent byte is (again, no surprise) 0xFF (~5\%).
The next is 0x8B (~5\%).

\myindex{x86!\Instructions!MOV}
0x8B is opcode for \INS{MOV}, this is indeed one of the most busy x86 instructions.
Now what about popularity of zero byte?
If compiler needs to encode value bigger than 127, it has to use 32-bit displacement instead of 8-bit one, but large values are very rare,
so it is padded by zeros.
\myindex{x86!\Instructions!LEA}
\myindex{x86!\Instructions!PUSH}
\myindex{x86!\Instructions!CALL}
This is at least in \INS{LEA}, \INS{MOV}, \INS{PUSH}, \INS{CALL}.

For example:

\begin{lstlisting}[style=customasmx86]
8D B0 28 01 00 00                 lea     esi, [eax+128h]
8D BF 40 38 00 00                 lea     edi, [edi+3840h]
\end{lstlisting}

Displacements bigger than 127 are very popular, but they are rarely exceeds 0x10000
(indeed, such large memory buffers/structures are also rare).

Same story with \INS{MOV}, large constants are rare, the most heavily used are 0, 1, 10, 100, $2^n$, and so on.
Compiler has to pad small constants by zeros to represent them as 32-bit values:

\begin{lstlisting}[style=customasmx86]
BF 02 00 00 00                    mov     edi, 2
BF 01 00 00 00                    mov     edi, 1
\end{lstlisting}

Now about 00 and FF bytes combined: jumps (including conditional) and calls can pass execution flow forward or backwards, but very often,
within the limits of the current executable module.
If forward, displacement is not very big and also padded with zeros.
If backwards, displacement is represented as negative value, so padded with FF bytes.
For example, transfer execution flow forward:

\begin{lstlisting}[style=customasmx86]
E8 43 0C 00 00                    call    _function1
E8 5C 00 00 00                    call    _function2
0F 84 F0 0A 00 00                 jz      loc_4F09A0
0F 84 EB 00 00 00                 jz      loc_4EFBB8
\end{lstlisting}

Backwards:

\begin{lstlisting}[style=customasmx86]
E8 79 0C FE FF                    call    _function1
E8 F4 16 FF FF                    call    _function2
0F 84 F8 FB FF FF                 jz      loc_8212BC
0F 84 06 FD FF FF                 jz      loc_FF1E7D
\end{lstlisting}

FF byte is also very often occurred in negative displacements like these:

\begin{lstlisting}[style=customasmx86]
8D 85 1E FF FF FF                 lea     eax, [ebp-0E2h]
8D 95 F8 5C FF FF                 lea     edx, [ebp-0A308h]
\end{lstlisting}

So far so good. Now we have to try various 16-byte keys, decrypt executable section and measure how often 00, FF and 8B bytes are occurred.
Let's also keep in sight how PCBC decryption works:

\begin{figure}[H]
\centering
\myincludegraphics{examples/simple_exec_crypto/640px-PCBC_decryption.png}
\caption{Propagating Cipher Block Chaining decryption (image is taken from Wikipedia article)}
\end{figure}

The good news is that we don't really have to decrypt whole piece of data, but only slice by slice, this is exactly how I did in my previous example: \myref{XOR_mask_2}.

Now I'm trying all possible bytes (0..255) for each byte in key and just pick the byte producing maximal amount of 00/FF/8B bytes in a decrypted slice:

\begin{lstlisting}[style=custompy]
#!/usr/bin/env python
import sys, hexdump, array, string, operator

KEY_LEN=16

def chunks(l, n):
    # split n by l-byte chunks
    # https://stackoverflow.com/q/312443
    n = max(1, n)
    return [l[i:i + n] for i in range(0, len(l), n)]

def read_file(fname):
    file=open(fname, mode='rb')
    content=file.read()
    file.close()
    return content

def decrypt_byte (c, key):
    return chr((ord(c)-key) % 256)

def XOR_PCBC_step (IV, buf, k):
    prev=IV
    rt=""
    for c in buf:
	new_c=decrypt_byte(c, k)
        plain=chr(ord(new_c)^ord(prev))
	prev=chr(ord(c)^ord(plain))
	rt=rt+plain
    return rt

each_Nth_byte=[""]*KEY_LEN

content=read_file(sys.argv[1])
# split input by 16-byte chunks:
all_chunks=chunks(content, KEY_LEN)
for c in all_chunks:
    for i in range(KEY_LEN):
        each_Nth_byte[i]=each_Nth_byte[i] + c[i]

# try each byte of key
for N in range(KEY_LEN):
    print "N=", N
    stat={}
    for i in range(256):
        tmp_key=chr(i)
	tmp=XOR_PCBC_step(tmp_key,each_Nth_byte[N], N)
        # count 0, FFs and 8Bs in decrypted buffer:
	important_bytes=tmp.count('\x00')+tmp.count('\xFF')+tmp.count('\x8B')
	stat[i]=important_bytes
    sorted_stat = sorted(stat.iteritems(), key=operator.itemgetter(1), reverse=True)
    print sorted_stat[0]
\end{lstlisting}

(Source code can be downloaded \href{\GitHubBlobMasterURL/examples/simple_exec_crypto/files/decrypt.py}{here}.)

I run it and here is a key for which 00/FF/8B bytes presence in decrypted buffer is maximal:

\begin{lstlisting}
N= 0
(147, 1224)
N= 1
(94, 1327)
N= 2
(252, 1223)
N= 3
(218, 1266)
N= 4
(38, 1209)
N= 5
(192, 1378)
N= 6
(199, 1204)
N= 7
(213, 1332)
N= 8
(225, 1251)
N= 9
(112, 1223)
N= 10
(143, 1177)
N= 11
(108, 1286)
N= 12
(10, 1164)
N= 13
(3, 1271)
N= 14
(128, 1253)
N= 15
(232, 1330)
\end{lstlisting}

Let's write decryption utility with the key we got:

\begin{lstlisting}[style=custompy]
#!/usr/bin/env python
import sys, hexdump, array

def xor_strings(s,t):
    # \verb|https://en.wikipedia.org/wiki/XOR_cipher#Example_implementation|
    """xor two strings together"""
    return "".join(chr(ord(a)^ord(b)) for a,b in zip(s,t))

IV=array.array('B', [147, 94, 252, 218, 38, 192, 199, 213, 225, 112, 143, 108, 10, 3, 128, 232]).tostring()

def chunks(l, n):
    n = max(1, n)
    return [l[i:i + n] for i in range(0, len(l), n)]

def read_file(fname):
    file=open(fname, mode='rb')
    content=file.read()
    file.close()
    return content

def decrypt_byte(i, k):
    return chr ((ord(i)-k) % 256)

def decrypt(buf):
    return "".join(decrypt_byte(buf[i], i) for i in range(16))

fout=open(sys.argv[2], mode='wb')

prev=IV
content=read_file(sys.argv[1])
tmp=chunks(content, 16)
for c in tmp:
    new_c=decrypt(c)
    p=xor_strings (new_c, prev)
    prev=xor_strings(c, p)
    fout.write(p)
fout.close()
\end{lstlisting}

(Source code can be downloaded \href{\GitHubBlobMasterURL/examples/simple_exec_crypto/files/decrypt2.py}{here}.)

Let's check resulting file:

\lstinputlisting{examples/simple_exec_crypto/objdump_result.txt}

Yes, this is seems correctly disassembled piece of x86 code.
The whole decryped file can be downloaded \href{\GitHubBlobMasterURL/examples/simple_exec_crypto/files/decrypted.bin}{here}.

In fact, this is text section from regedit.exe from Windows 7.
But this example is based on a real case I encountered, so just executable is different (and key), algorithm is the same.

\subsection{Other ideas to consider}

What if I would fail with such simple frequency analysis?
There are other ideas on how to measure correctness of decrypted/decompressed x86 code:

\begin{itemize}

\item Many modern compilers aligns functions on 0x10 border.
So the space left before is filled with NOPs (0x90) or other NOP instructions with known opcodes: \myref{sec:npad}.

\item Perhaps, the most frequent pattern in any assembly language is function call:\\
\TT{PUSH chain / CALL / ADD ESP, X}.
This sequence can easily detected and found.
I've even gathered statistics about average number of function arguments: \myref{args_stat}.
(Hence, this is average length of PUSH chain.)

\end{itemize}

Read more about incorrectly/correctly disassembled code: \myref{ISA_detect}.
}%
\FR{\mysection{Une fonction vide: redux}

Revenons sur l'exemple de la fonction vide \myref{empty_func}.
Maintenant que nous connaissons le prologue et l'épilogue de fonction, ceci est
une fonction vide \myref{lst:empty_func} compilée par GCC sans optimisation:

\lstinputlisting[caption=GCC 8.2 x64 \NonOptimizing (\assemblyOutput),style=customasmx86]{patterns/016_empty_redux/1.s}

C'est \INS{RET}, mais le prologue et l'épilogue de la fonction, probablement, n'ont
pas été optimisés et laissés tels quels.
\INS{NOP} semble être un autre artefact du compilateur.
De toutes façons, la seule instruction effective ici est \INS{RET}.
Toutes les autres instructions peuvent être supprimées (ou optimisées).

}

\input{tools}
\EN{\chapter{Case studies}

\input{examples/Knuth_EN}

% sections here
% TODO translate
\mysection{Breaking simple executable cryptor}

I've got an executable file which is encrypted by relatively simple encryption.
\href{\GitHubBlobMasterURL/examples/simple_exec_crypto/files/cipher.bin}{Here is it} (only executable section is left here).

First, all encryption function does is just adds number of position in buffer to the byte.
Here is how this can be encoded in Python:

\begin{lstlisting}[caption=Python script,style=custompy]
#!/usr/bin/env python
def e(i, k):
    return chr ((ord(i)+k) % 256)

def encrypt(buf):
    return e(buf[0], 0)+ e(buf[1], 1)+ e(buf[2], 2) + e(buf[3], 3)+ e(buf[4], 4)+ e(buf[5], 5)+ e(buf[6], 6)+ e(buf[7], 7)+
           e(buf[8], 8)+ e(buf[9], 9)+ e(buf[10], 10)+ e(buf[11], 11)+ e(buf[12], 12)+ e(buf[13], 13)+ e(buf[14], 14)+ e(buf[15], 15)
\end{lstlisting}

Hence, if you encrypt buffer with 16 zeros, you'll get \emph{0, 1, 2, 3 ... 12, 13, 14, 15}.

\myindex{Propagating Cipher Block Chaining}
Propagating Cipher Block Chaining (PCBC) is also used, here is how it works:

\begin{figure}[H]
\centering
\myincludegraphics{examples/simple_exec_crypto/601px-PCBC_encryption.png}
\caption{Propagating Cipher Block Chaining encryption (image is taken from Wikipedia article)}
\end{figure}

The problem is that it's too boring to recover IV (Initialization Vector) each time.
Brute-force is also not an option, because IV is too long (16 bytes).
Let's see, if it's possible to recover IV for arbitrary encrypted executable file?

Let's try simple frequency analysis.
This is 32-bit x86 executable code, so let's gather statistics about most frequent bytes and opcodes.
I tried huge oracle.exe file from Oracle RDBMS version 11.2 for windows x86 and I've found that the most frequent byte (no surprise) is zero (~10\%).
The next most frequent byte is (again, no surprise) 0xFF (~5\%).
The next is 0x8B (~5\%).

\myindex{x86!\Instructions!MOV}
0x8B is opcode for \INS{MOV}, this is indeed one of the most busy x86 instructions.
Now what about popularity of zero byte?
If compiler needs to encode value bigger than 127, it has to use 32-bit displacement instead of 8-bit one, but large values are very rare,
so it is padded by zeros.
\myindex{x86!\Instructions!LEA}
\myindex{x86!\Instructions!PUSH}
\myindex{x86!\Instructions!CALL}
This is at least in \INS{LEA}, \INS{MOV}, \INS{PUSH}, \INS{CALL}.

For example:

\begin{lstlisting}[style=customasmx86]
8D B0 28 01 00 00                 lea     esi, [eax+128h]
8D BF 40 38 00 00                 lea     edi, [edi+3840h]
\end{lstlisting}

Displacements bigger than 127 are very popular, but they are rarely exceeds 0x10000
(indeed, such large memory buffers/structures are also rare).

Same story with \INS{MOV}, large constants are rare, the most heavily used are 0, 1, 10, 100, $2^n$, and so on.
Compiler has to pad small constants by zeros to represent them as 32-bit values:

\begin{lstlisting}[style=customasmx86]
BF 02 00 00 00                    mov     edi, 2
BF 01 00 00 00                    mov     edi, 1
\end{lstlisting}

Now about 00 and FF bytes combined: jumps (including conditional) and calls can pass execution flow forward or backwards, but very often,
within the limits of the current executable module.
If forward, displacement is not very big and also padded with zeros.
If backwards, displacement is represented as negative value, so padded with FF bytes.
For example, transfer execution flow forward:

\begin{lstlisting}[style=customasmx86]
E8 43 0C 00 00                    call    _function1
E8 5C 00 00 00                    call    _function2
0F 84 F0 0A 00 00                 jz      loc_4F09A0
0F 84 EB 00 00 00                 jz      loc_4EFBB8
\end{lstlisting}

Backwards:

\begin{lstlisting}[style=customasmx86]
E8 79 0C FE FF                    call    _function1
E8 F4 16 FF FF                    call    _function2
0F 84 F8 FB FF FF                 jz      loc_8212BC
0F 84 06 FD FF FF                 jz      loc_FF1E7D
\end{lstlisting}

FF byte is also very often occurred in negative displacements like these:

\begin{lstlisting}[style=customasmx86]
8D 85 1E FF FF FF                 lea     eax, [ebp-0E2h]
8D 95 F8 5C FF FF                 lea     edx, [ebp-0A308h]
\end{lstlisting}

So far so good. Now we have to try various 16-byte keys, decrypt executable section and measure how often 00, FF and 8B bytes are occurred.
Let's also keep in sight how PCBC decryption works:

\begin{figure}[H]
\centering
\myincludegraphics{examples/simple_exec_crypto/640px-PCBC_decryption.png}
\caption{Propagating Cipher Block Chaining decryption (image is taken from Wikipedia article)}
\end{figure}

The good news is that we don't really have to decrypt whole piece of data, but only slice by slice, this is exactly how I did in my previous example: \myref{XOR_mask_2}.

Now I'm trying all possible bytes (0..255) for each byte in key and just pick the byte producing maximal amount of 00/FF/8B bytes in a decrypted slice:

\begin{lstlisting}[style=custompy]
#!/usr/bin/env python
import sys, hexdump, array, string, operator

KEY_LEN=16

def chunks(l, n):
    # split n by l-byte chunks
    # https://stackoverflow.com/q/312443
    n = max(1, n)
    return [l[i:i + n] for i in range(0, len(l), n)]

def read_file(fname):
    file=open(fname, mode='rb')
    content=file.read()
    file.close()
    return content

def decrypt_byte (c, key):
    return chr((ord(c)-key) % 256)

def XOR_PCBC_step (IV, buf, k):
    prev=IV
    rt=""
    for c in buf:
	new_c=decrypt_byte(c, k)
        plain=chr(ord(new_c)^ord(prev))
	prev=chr(ord(c)^ord(plain))
	rt=rt+plain
    return rt

each_Nth_byte=[""]*KEY_LEN

content=read_file(sys.argv[1])
# split input by 16-byte chunks:
all_chunks=chunks(content, KEY_LEN)
for c in all_chunks:
    for i in range(KEY_LEN):
        each_Nth_byte[i]=each_Nth_byte[i] + c[i]

# try each byte of key
for N in range(KEY_LEN):
    print "N=", N
    stat={}
    for i in range(256):
        tmp_key=chr(i)
	tmp=XOR_PCBC_step(tmp_key,each_Nth_byte[N], N)
        # count 0, FFs and 8Bs in decrypted buffer:
	important_bytes=tmp.count('\x00')+tmp.count('\xFF')+tmp.count('\x8B')
	stat[i]=important_bytes
    sorted_stat = sorted(stat.iteritems(), key=operator.itemgetter(1), reverse=True)
    print sorted_stat[0]
\end{lstlisting}

(Source code can be downloaded \href{\GitHubBlobMasterURL/examples/simple_exec_crypto/files/decrypt.py}{here}.)

I run it and here is a key for which 00/FF/8B bytes presence in decrypted buffer is maximal:

\begin{lstlisting}
N= 0
(147, 1224)
N= 1
(94, 1327)
N= 2
(252, 1223)
N= 3
(218, 1266)
N= 4
(38, 1209)
N= 5
(192, 1378)
N= 6
(199, 1204)
N= 7
(213, 1332)
N= 8
(225, 1251)
N= 9
(112, 1223)
N= 10
(143, 1177)
N= 11
(108, 1286)
N= 12
(10, 1164)
N= 13
(3, 1271)
N= 14
(128, 1253)
N= 15
(232, 1330)
\end{lstlisting}

Let's write decryption utility with the key we got:

\begin{lstlisting}[style=custompy]
#!/usr/bin/env python
import sys, hexdump, array

def xor_strings(s,t):
    # \verb|https://en.wikipedia.org/wiki/XOR_cipher#Example_implementation|
    """xor two strings together"""
    return "".join(chr(ord(a)^ord(b)) for a,b in zip(s,t))

IV=array.array('B', [147, 94, 252, 218, 38, 192, 199, 213, 225, 112, 143, 108, 10, 3, 128, 232]).tostring()

def chunks(l, n):
    n = max(1, n)
    return [l[i:i + n] for i in range(0, len(l), n)]

def read_file(fname):
    file=open(fname, mode='rb')
    content=file.read()
    file.close()
    return content

def decrypt_byte(i, k):
    return chr ((ord(i)-k) % 256)

def decrypt(buf):
    return "".join(decrypt_byte(buf[i], i) for i in range(16))

fout=open(sys.argv[2], mode='wb')

prev=IV
content=read_file(sys.argv[1])
tmp=chunks(content, 16)
for c in tmp:
    new_c=decrypt(c)
    p=xor_strings (new_c, prev)
    prev=xor_strings(c, p)
    fout.write(p)
fout.close()
\end{lstlisting}

(Source code can be downloaded \href{\GitHubBlobMasterURL/examples/simple_exec_crypto/files/decrypt2.py}{here}.)

Let's check resulting file:

\lstinputlisting{examples/simple_exec_crypto/objdump_result.txt}

Yes, this is seems correctly disassembled piece of x86 code.
The whole decryped file can be downloaded \href{\GitHubBlobMasterURL/examples/simple_exec_crypto/files/decrypted.bin}{here}.

In fact, this is text section from regedit.exe from Windows 7.
But this example is based on a real case I encountered, so just executable is different (and key), algorithm is the same.

\subsection{Other ideas to consider}

What if I would fail with such simple frequency analysis?
There are other ideas on how to measure correctness of decrypted/decompressed x86 code:

\begin{itemize}

\item Many modern compilers aligns functions on 0x10 border.
So the space left before is filled with NOPs (0x90) or other NOP instructions with known opcodes: \myref{sec:npad}.

\item Perhaps, the most frequent pattern in any assembly language is function call:\\
\TT{PUSH chain / CALL / ADD ESP, X}.
This sequence can easily detected and found.
I've even gathered statistics about average number of function arguments: \myref{args_stat}.
(Hence, this is average length of PUSH chain.)

\end{itemize}

Read more about incorrectly/correctly disassembled code: \myref{ISA_detect}.

% TODO translate
\mysection{Breaking simple executable cryptor}

I've got an executable file which is encrypted by relatively simple encryption.
\href{\GitHubBlobMasterURL/examples/simple_exec_crypto/files/cipher.bin}{Here is it} (only executable section is left here).

First, all encryption function does is just adds number of position in buffer to the byte.
Here is how this can be encoded in Python:

\begin{lstlisting}[caption=Python script,style=custompy]
#!/usr/bin/env python
def e(i, k):
    return chr ((ord(i)+k) % 256)

def encrypt(buf):
    return e(buf[0], 0)+ e(buf[1], 1)+ e(buf[2], 2) + e(buf[3], 3)+ e(buf[4], 4)+ e(buf[5], 5)+ e(buf[6], 6)+ e(buf[7], 7)+
           e(buf[8], 8)+ e(buf[9], 9)+ e(buf[10], 10)+ e(buf[11], 11)+ e(buf[12], 12)+ e(buf[13], 13)+ e(buf[14], 14)+ e(buf[15], 15)
\end{lstlisting}

Hence, if you encrypt buffer with 16 zeros, you'll get \emph{0, 1, 2, 3 ... 12, 13, 14, 15}.

\myindex{Propagating Cipher Block Chaining}
Propagating Cipher Block Chaining (PCBC) is also used, here is how it works:

\begin{figure}[H]
\centering
\myincludegraphics{examples/simple_exec_crypto/601px-PCBC_encryption.png}
\caption{Propagating Cipher Block Chaining encryption (image is taken from Wikipedia article)}
\end{figure}

The problem is that it's too boring to recover IV (Initialization Vector) each time.
Brute-force is also not an option, because IV is too long (16 bytes).
Let's see, if it's possible to recover IV for arbitrary encrypted executable file?

Let's try simple frequency analysis.
This is 32-bit x86 executable code, so let's gather statistics about most frequent bytes and opcodes.
I tried huge oracle.exe file from Oracle RDBMS version 11.2 for windows x86 and I've found that the most frequent byte (no surprise) is zero (~10\%).
The next most frequent byte is (again, no surprise) 0xFF (~5\%).
The next is 0x8B (~5\%).

\myindex{x86!\Instructions!MOV}
0x8B is opcode for \INS{MOV}, this is indeed one of the most busy x86 instructions.
Now what about popularity of zero byte?
If compiler needs to encode value bigger than 127, it has to use 32-bit displacement instead of 8-bit one, but large values are very rare,
so it is padded by zeros.
\myindex{x86!\Instructions!LEA}
\myindex{x86!\Instructions!PUSH}
\myindex{x86!\Instructions!CALL}
This is at least in \INS{LEA}, \INS{MOV}, \INS{PUSH}, \INS{CALL}.

For example:

\begin{lstlisting}[style=customasmx86]
8D B0 28 01 00 00                 lea     esi, [eax+128h]
8D BF 40 38 00 00                 lea     edi, [edi+3840h]
\end{lstlisting}

Displacements bigger than 127 are very popular, but they are rarely exceeds 0x10000
(indeed, such large memory buffers/structures are also rare).

Same story with \INS{MOV}, large constants are rare, the most heavily used are 0, 1, 10, 100, $2^n$, and so on.
Compiler has to pad small constants by zeros to represent them as 32-bit values:

\begin{lstlisting}[style=customasmx86]
BF 02 00 00 00                    mov     edi, 2
BF 01 00 00 00                    mov     edi, 1
\end{lstlisting}

Now about 00 and FF bytes combined: jumps (including conditional) and calls can pass execution flow forward or backwards, but very often,
within the limits of the current executable module.
If forward, displacement is not very big and also padded with zeros.
If backwards, displacement is represented as negative value, so padded with FF bytes.
For example, transfer execution flow forward:

\begin{lstlisting}[style=customasmx86]
E8 43 0C 00 00                    call    _function1
E8 5C 00 00 00                    call    _function2
0F 84 F0 0A 00 00                 jz      loc_4F09A0
0F 84 EB 00 00 00                 jz      loc_4EFBB8
\end{lstlisting}

Backwards:

\begin{lstlisting}[style=customasmx86]
E8 79 0C FE FF                    call    _function1
E8 F4 16 FF FF                    call    _function2
0F 84 F8 FB FF FF                 jz      loc_8212BC
0F 84 06 FD FF FF                 jz      loc_FF1E7D
\end{lstlisting}

FF byte is also very often occurred in negative displacements like these:

\begin{lstlisting}[style=customasmx86]
8D 85 1E FF FF FF                 lea     eax, [ebp-0E2h]
8D 95 F8 5C FF FF                 lea     edx, [ebp-0A308h]
\end{lstlisting}

So far so good. Now we have to try various 16-byte keys, decrypt executable section and measure how often 00, FF and 8B bytes are occurred.
Let's also keep in sight how PCBC decryption works:

\begin{figure}[H]
\centering
\myincludegraphics{examples/simple_exec_crypto/640px-PCBC_decryption.png}
\caption{Propagating Cipher Block Chaining decryption (image is taken from Wikipedia article)}
\end{figure}

The good news is that we don't really have to decrypt whole piece of data, but only slice by slice, this is exactly how I did in my previous example: \myref{XOR_mask_2}.

Now I'm trying all possible bytes (0..255) for each byte in key and just pick the byte producing maximal amount of 00/FF/8B bytes in a decrypted slice:

\begin{lstlisting}[style=custompy]
#!/usr/bin/env python
import sys, hexdump, array, string, operator

KEY_LEN=16

def chunks(l, n):
    # split n by l-byte chunks
    # https://stackoverflow.com/q/312443
    n = max(1, n)
    return [l[i:i + n] for i in range(0, len(l), n)]

def read_file(fname):
    file=open(fname, mode='rb')
    content=file.read()
    file.close()
    return content

def decrypt_byte (c, key):
    return chr((ord(c)-key) % 256)

def XOR_PCBC_step (IV, buf, k):
    prev=IV
    rt=""
    for c in buf:
	new_c=decrypt_byte(c, k)
        plain=chr(ord(new_c)^ord(prev))
	prev=chr(ord(c)^ord(plain))
	rt=rt+plain
    return rt

each_Nth_byte=[""]*KEY_LEN

content=read_file(sys.argv[1])
# split input by 16-byte chunks:
all_chunks=chunks(content, KEY_LEN)
for c in all_chunks:
    for i in range(KEY_LEN):
        each_Nth_byte[i]=each_Nth_byte[i] + c[i]

# try each byte of key
for N in range(KEY_LEN):
    print "N=", N
    stat={}
    for i in range(256):
        tmp_key=chr(i)
	tmp=XOR_PCBC_step(tmp_key,each_Nth_byte[N], N)
        # count 0, FFs and 8Bs in decrypted buffer:
	important_bytes=tmp.count('\x00')+tmp.count('\xFF')+tmp.count('\x8B')
	stat[i]=important_bytes
    sorted_stat = sorted(stat.iteritems(), key=operator.itemgetter(1), reverse=True)
    print sorted_stat[0]
\end{lstlisting}

(Source code can be downloaded \href{\GitHubBlobMasterURL/examples/simple_exec_crypto/files/decrypt.py}{here}.)

I run it and here is a key for which 00/FF/8B bytes presence in decrypted buffer is maximal:

\begin{lstlisting}
N= 0
(147, 1224)
N= 1
(94, 1327)
N= 2
(252, 1223)
N= 3
(218, 1266)
N= 4
(38, 1209)
N= 5
(192, 1378)
N= 6
(199, 1204)
N= 7
(213, 1332)
N= 8
(225, 1251)
N= 9
(112, 1223)
N= 10
(143, 1177)
N= 11
(108, 1286)
N= 12
(10, 1164)
N= 13
(3, 1271)
N= 14
(128, 1253)
N= 15
(232, 1330)
\end{lstlisting}

Let's write decryption utility with the key we got:

\begin{lstlisting}[style=custompy]
#!/usr/bin/env python
import sys, hexdump, array

def xor_strings(s,t):
    # \verb|https://en.wikipedia.org/wiki/XOR_cipher#Example_implementation|
    """xor two strings together"""
    return "".join(chr(ord(a)^ord(b)) for a,b in zip(s,t))

IV=array.array('B', [147, 94, 252, 218, 38, 192, 199, 213, 225, 112, 143, 108, 10, 3, 128, 232]).tostring()

def chunks(l, n):
    n = max(1, n)
    return [l[i:i + n] for i in range(0, len(l), n)]

def read_file(fname):
    file=open(fname, mode='rb')
    content=file.read()
    file.close()
    return content

def decrypt_byte(i, k):
    return chr ((ord(i)-k) % 256)

def decrypt(buf):
    return "".join(decrypt_byte(buf[i], i) for i in range(16))

fout=open(sys.argv[2], mode='wb')

prev=IV
content=read_file(sys.argv[1])
tmp=chunks(content, 16)
for c in tmp:
    new_c=decrypt(c)
    p=xor_strings (new_c, prev)
    prev=xor_strings(c, p)
    fout.write(p)
fout.close()
\end{lstlisting}

(Source code can be downloaded \href{\GitHubBlobMasterURL/examples/simple_exec_crypto/files/decrypt2.py}{here}.)

Let's check resulting file:

\lstinputlisting{examples/simple_exec_crypto/objdump_result.txt}

Yes, this is seems correctly disassembled piece of x86 code.
The whole decryped file can be downloaded \href{\GitHubBlobMasterURL/examples/simple_exec_crypto/files/decrypted.bin}{here}.

In fact, this is text section from regedit.exe from Windows 7.
But this example is based on a real case I encountered, so just executable is different (and key), algorithm is the same.

\subsection{Other ideas to consider}

What if I would fail with such simple frequency analysis?
There are other ideas on how to measure correctness of decrypted/decompressed x86 code:

\begin{itemize}

\item Many modern compilers aligns functions on 0x10 border.
So the space left before is filled with NOPs (0x90) or other NOP instructions with known opcodes: \myref{sec:npad}.

\item Perhaps, the most frequent pattern in any assembly language is function call:\\
\TT{PUSH chain / CALL / ADD ESP, X}.
This sequence can easily detected and found.
I've even gathered statistics about average number of function arguments: \myref{args_stat}.
(Hence, this is average length of PUSH chain.)

\end{itemize}

Read more about incorrectly/correctly disassembled code: \myref{ISA_detect}.

\mysection{\MinesweeperWinXPExampleChapterName}
\label{minesweeper_winxp}
\myindex{Windows!Windows XP}

For those who are not very good at playing Minesweeper, we could try to reveal the hidden mines in the debugger.

\myindex{\CStandardLibrary!rand()}
\myindex{Windows!PDB}

As we know, Minesweeper places mines randomly, so there has to be some kind of random number generator or
a call to the standard \TT{rand()} C-function.

What is really cool about reversing Microsoft products is that there are \gls{PDB} 
file with symbols (function names, etc).
When we load \TT{winmine.exe} into \IDA, it downloads the 
\gls{PDB} file exactly for this 
executable and shows all names.

So here it is, the only call to \TT{rand()} is this function:

\lstinputlisting[style=customasmx86]{examples/minesweeper/tmp1.lst}

\IDA named it so, and it was the name given to it by Minesweeper's developers.

The function is very simple:

\begin{lstlisting}[style=customc]
int Rnd(int limit)
{
    return rand() % limit;
};
\end{lstlisting}

(There is no \q{limit} name in the \gls{PDB} file; we manually named this argument like this.)

So it returns 
a random value from 0 to a specified limit.

\TT{Rnd()} is called only from one place, 
a function called \TT{StartGame()}, 
and as it seems, this is exactly 
the code which place the mines:

\begin{lstlisting}[style=customasmx86]
.text:010036C7                 push    _xBoxMac
.text:010036CD                 call    _Rnd@4          ; Rnd(x)
.text:010036D2                 push    _yBoxMac
.text:010036D8                 mov     esi, eax
.text:010036DA                 inc     esi
.text:010036DB                 call    _Rnd@4          ; Rnd(x)
.text:010036E0                 inc     eax
.text:010036E1                 mov     ecx, eax
.text:010036E3                 shl     ecx, 5          ; ECX=ECX*32
.text:010036E6                 test    _rgBlk[ecx+esi], 80h
.text:010036EE                 jnz     short loc_10036C7
.text:010036F0                 shl     eax, 5          ; EAX=EAX*32
.text:010036F3                 lea     eax, _rgBlk[eax+esi]
.text:010036FA                 or      byte ptr [eax], 80h
.text:010036FD                 dec     _cBombStart
.text:01003703                 jnz     short loc_10036C7
\end{lstlisting}

Minesweeper allows you to set the board size, so the X (xBoxMac) and Y (yBoxMac) of the board are global variables.
They are passed to \TT{Rnd()} and random 
coordinates are generated.
A mine is placed by the \TT{OR} instruction at \TT{0x010036FA}. 
And if it has been placed before 
(it's possible if the pair of \TT{Rnd()} 
generates a coordinates pair which has been already 
generated), 
then \TT{TEST} and \TT{JNZ} at \TT{0x010036E6} 
jumps to the generation routine again.

\TT{cBombStart} is the global variable containing total number of mines. So this is loop.

The width of the array is 32 
(we can conclude this by looking at the \TT{SHL} instruction, which multiplies one of the coordinates by 32).

The size of the \TT{rgBlk} 
global array can be easily determined by the difference 
between the \TT{rgBlk} 
label in the data segment and the next known one. 
It is 0x360 (864):

\begin{lstlisting}[style=customasmx86]
.data:01005340 _rgBlk          db 360h dup(?)          ; DATA XREF: MainWndProc(x,x,x,x)+574
.data:01005340                                         ; DisplayBlk(x,x)+23
.data:010056A0 _Preferences    dd ?                    ; DATA XREF: FixMenus()+2
...
\end{lstlisting}

$864/32=27$.

So the array size is $27*32$?
It is close to what we know: when we try to set board size to $100*100$ in Minesweeper settings, it fallbacks to a board of size $24*30$.
So this is the maximal board size here.
And the array has a fixed size for any board size.

So let's see all this in \olly.
We will ran Minesweeper, attaching \olly to it and now we can see the memory dump at the address of the \TT{rgBlk} array (\TT{0x01005340})
\footnote{All addresses here are for Minesweeper for Windows XP SP3 English. 
They may differ for other service packs.}.

So we got this memory dump of the array:

\lstinputlisting[style=customasmx86]{examples/minesweeper/1.lst}

\olly, like any other hexadecimal editor, shows 16 bytes per line.
So each 32-byte array row occupies exactly 2 lines here.

This is beginner level (9*9 board).

There is some square 
structure can be seen visually (0x10 bytes).

We will click \q{Run} in \olly to unfreeze the Minesweeper process, then we'll clicked randomly at the Minesweeper window 
and trapped into mine, but now all mines are visible:

\begin{figure}[H]
\centering
\myincludegraphicsSmall{examples/minesweeper/1.png}
\caption{Mines}
\label{fig:minesweeper1}
\end{figure}

By comparing the mine places and the dump, we can conclude that 0x10 stands for border, 0x0F---empty block, 0x8F---mine.
Perhaps, 0x10 is just a \emph{sentinel value}.

Now we'll add comments and also enclose all 0x8F bytes into square brackets:

\lstinputlisting[style=customasmx86]{examples/minesweeper/2.lst}

Now we'll remove all \emph{border bytes} (0x10) and what's beyond those:

\lstinputlisting[style=customasmx86]{examples/minesweeper/3.lst}

Yes, these are mines, now it can be clearly seen and compared with the screenshot.

\clearpage
What is interesting is that we can modify the array right in \olly.
We can remove all mines by changing all 0x8F bytes by 0x0F, and here is what we'll get in Minesweeper:

\begin{figure}[H]
\centering
\myincludegraphicsSmall{examples/minesweeper/3.png}
\caption{All mines are removed in debugger}
\label{fig:minesweeper3}
\end{figure}

We can also move all of them to the first line: 

\begin{figure}[H]
\centering
\myincludegraphicsSmall{examples/minesweeper/2.png}
\caption{Mines set in debugger}
\label{fig:minesweeper2}
\end{figure}

Well, the debugger is not very convenient for eavesdropping (which is our goal anyway), so we'll write a small utility
to dump the contents of the board:

\lstinputlisting[style=customc]{examples/minesweeper/minesweeper_cheater.c}

Just set the \ac{PID}
\footnote{PID it can be seen in Task Manager 
(enable it in \q{View $\rightarrow$ Select Columns})} 
and the address of the array (\TT{0x01005340} for Windows XP SP3 English) 
and it will dump it
\footnote{The compiled executable is here: 
\href{http://go.yurichev.com/17165}{beginners.re}}.

It attaches itself to a win32 process by \ac{PID} and just reads process memory at the address.

\subsection{Finding grid automatically}

This is kind of nuisance to set address each time when we run our utility.
Also, various Minesweeper versions may have the array on different address.
Knowing the fact that there is always a border (0x10 bytes), we can just find it in memory:

\lstinputlisting[style=customc]{examples/minesweeper/cheater2_fragment.c}

Full source code: \url{\RepoURL/examples/minesweeper/minesweeper_cheater2.c}.

\subsection{\Exercises}

\begin{itemize}

\item 
Why do the \emph{border bytes} (or \emph{sentinel values}) (0x10) exist in the array?

What they are for if they are not visible in Minesweeper's interface?
How could it work without them?

\item 
As it turns out, there are more values possible (for open blocks, for flagged by user, etc).
Try to find the meaning of each one.

\item 
Modify my utility so it can remove all mines or set them in a fixed pattern that you want in the Minesweeper
process currently running.

\end{itemize}

\mysection{Hacking Windows clock}

Sometimes I do some kind of first April prank for my coworkers.

Let's find, if we could do something with Windows clock?
Can we force to go clock hands backwards?

First of all, when you click on date/time in status bar,\\
a \emph{C:\textbackslash{}WINDOWS\textbackslash{}SYSTEM32\textbackslash{}TIMEDATE.CPL} module gets executed,
which is usual executable \ac{PE}-file.

Let's see, how it draw hands?
When I open the file (from Windows 7) in Resource Hacker, there are clock faces, but with no hands:

\begin{figure}[H]
\centering
\myincludegraphics{examples/timedate/reshack.png}
\caption{Resource Hacker}
\end{figure}

OK, what we know? How to draw a clock hand? All they are started at the middle of circle, ending with its border.
Hence, we must calculate coordinates of a point on circle's border.
From school-level mathematics we may recall that we have to use sine/cosine functions to draw circle, or at least
square root.
There are no such things in \emph{TIMEDATE.CPL}, at least at first glance.
But, thanks to Microsoft debugging PDB files, I can find a function named \emph{CAnalogClock::DrawHand()}, which calls
\emph{Gdiplus::Graphics::DrawLine()} at least twice.

Here is its code:

\lstinputlisting[style=customasmx86]{examples/timedate/1.lst}

\myindex{Windows!Win32!MulDiv()}
We can see that \emph{DrawLine()} arguments are dependent on result of \emph{MulDiv()} function
and a \emph{table[]} table (name is mine),
which has 8-byte elements (look at \INS{LEA}'s second operand).

What is inside of table[]?

\lstinputlisting[style=customasmx86]{examples/timedate/2.lst}

It's referenced only from \emph{DrawHand()} function.
It has 120 32-bit words or 60 32-bit pairs... wait, 60?
Let's take a closer look at these values.
First of all, I'll zap 6 pairs or 12 32-bit words with zeros, and then I'll put patched \emph{TIMEDATE.CPL}
into \emph{C:\textbackslash{}WINDOWS\textbackslash{}SYSTEM32}.
(You may need to set owner of the *TIMEDATE.CPL* file to your primary user account (instead of \emph{TrustedInstaller}),
and also, boot in safe mode with command prompt so you can copy the file, which is usually locked.)

\begin{figure}[H]
\centering
\includegraphics[width=0.5\textwidth]{examples/timedate/6_pairs_zeroed.png}
\caption{Attempt to run}
\end{figure}

Now when any hand is located at 0..5 seconds/minutes, it's invisible! However, opposite (shorter) part of second hand
is visible and moving.
When any hand is outside of this area, hand is visible as usual.

\myindex{Mathematica}
Let's take even closer look at the table in Mathematica.
I have copypasted table from the \emph{TIMEDATE.CPL} to a \emph{tbl} file (480 bytes).
We will take for granted the fact that these are signed values, because half of elements are below zero (0FFFFE0C1h, etc.).
If these values would be unsigned, they would be suspiciously huge.

\begin{lstlisting}[style=custommath]
In[]:= tbl = BinaryReadList["~/.../tbl", "Integer32"]

Out[]= {0, -7999, 836, -7956, 1663, -7825, 2472, -7608, 3253, -7308, 3999, \
-6928, 4702, -6472, 5353, -5945, 5945, -5353, 6472, -4702, 6928, \
-4000, 7308, -3253, 7608, -2472, 7825, -1663, 7956, -836, 8000, 0, \
7956, 836, 7825, 1663, 7608, 2472, 7308, 3253, 6928, 4000, 6472, \
4702, 5945, 5353, 5353, 5945, 4702, 6472, 3999, 6928, 3253, 7308, \
2472, 7608, 1663, 7825, 836, 7956, 0, 7999, -836, 7956, -1663, 7825, \
-2472, 7608, -3253, 7308, -4000, 6928, -4702, 6472, -5353, 5945, \
-5945, 5353, -6472, 4702, -6928, 3999, -7308, 3253, -7608, 2472, \
-7825, 1663, -7956, 836, -7999, 0, -7956, -836, -7825, -1663, -7608, \
-2472, -7308, -3253, -6928, -4000, -6472, -4702, -5945, -5353, -5353, \
-5945, -4702, -6472, -3999, -6928, -3253, -7308, -2472, -7608, -1663, \
-7825, -836, -7956}

In[]:= Length[tbl]
Out[]= 120
\end{lstlisting}

Let's treat two consecutive 32-bit values as pair:

\begin{lstlisting}[style=custommath]
In[]:= pairs = Partition[tbl, 2]
Out[]= {{0, -7999}, {836, -7956}, {1663, -7825}, {2472, -7608}, \
{3253, -7308}, {3999, -6928}, {4702, -6472}, {5353, -5945}, {5945, \
-5353}, {6472, -4702}, {6928, -4000}, {7308, -3253}, {7608, -2472}, \
{7825, -1663}, {7956, -836}, {8000, 0}, {7956, 836}, {7825, 
1663}, {7608, 2472}, {7308, 3253}, {6928, 4000}, {6472, 
4702}, {5945, 5353}, {5353, 5945}, {4702, 6472}, {3999, 
6928}, {3253, 7308}, {2472, 7608}, {1663, 7825}, {836, 7956}, {0, 
7999}, {-836, 7956}, {-1663, 7825}, {-2472, 7608}, {-3253, 
7308}, {-4000, 6928}, {-4702, 6472}, {-5353, 5945}, {-5945, 
5353}, {-6472, 4702}, {-6928, 3999}, {-7308, 3253}, {-7608, 
2472}, {-7825, 1663}, {-7956, 836}, {-7999, 
0}, {-7956, -836}, {-7825, -1663}, {-7608, -2472}, {-7308, -3253}, \
{-6928, -4000}, {-6472, -4702}, {-5945, -5353}, {-5353, -5945}, \
{-4702, -6472}, {-3999, -6928}, {-3253, -7308}, {-2472, -7608}, \
{-1663, -7825}, {-836, -7956}}

In[]:= Length[pairs]
Out[]= 60
\end{lstlisting}

Let's try to treat each pair as X/Y coordinate and draw all 60 pairs, and also first 15 pairs:

\begin{figure}[H]
\centering
\myincludegraphics{examples/timedate/math.png}
\caption{Mathematica}
\end{figure}

Now this is something!
Each pair is just coordinate.
First 15 pairs are coordinates for $\frac{1}{4}$ of circle.

Perhaps, Microsoft developers precalculated all coordinates and put them into table.
\myindex{Memoization}
This is widespread, though somewhat old school practice -- precalculated table access is faster than calling relatively slow sine/cosine functions\footnote{Today this is known
as \emph{memoization}}.
Sine/cosine operations are not that expensive anymore...

Now I can understand why when I zapped first 6 pairs, hands were invisible at that area: in fact, hands were drawn,
they just had zero length, because hand started at 0:0 coordinate and ended there.

\subsubsection{The prank (practical joke)}

Given all that, how would we force hands to go counterclockwise?
In fact, this is simple, we need just to rotate the table, so each hand, instead of drawing at place of zeroth second,
would be drawing at place of 59th second.

I made the patcher a long time ago, at the very beginning of 2000s, for Windows 2000.
Hard to believe, it still works for Windows 7, perhaps, the table hasn't been changed since then!

The patcher source code: \url{\RepoURL/examples/timedate/time_pt.c}.

Now I can see all hands goes backwards:

\begin{figure}[H]
\centering
\includegraphics[width=0.5\textwidth]{examples/timedate/counterclockwise.png}
\caption{Now it works}
\end{figure}

Well, there is no animation in this book, but if you look closer, you can see, that hands are in fact shows correct
time, but the whole clock face is rotated vertically, like we see it from the inside of clock.

\subsubsection{Windows 2000 leaked source code}

So I did the patcher and then Windows 2000 source code has been leaked (I can't force you to trust me, though).
Let's take a look on source code if that function and table.\\
The file is \emph{win2k/private/shell/cpls/utc/clock.c}:

\begin{lstlisting}[style=customc]
//
//  Array containing the sine and cosine values for hand positions.
//
POINT rCircleTable[] =
{
    { 0,     -7999},
    { 836,   -7956},
    { 1663,  -7825},
    { 2472,  -7608},
    { 3253,  -7308},
...
    { -4702, -6472},
    { -3999, -6928},
    { -3253, -7308},
    { -2472, -7608},
    { -1663, -7825},
    { -836 , -7956},
};

////////////////////////////////////////////////////////////////////////////
//
//  DrawHand
//
//  Draws the hands of the clock.
//
////////////////////////////////////////////////////////////////////////////

void DrawHand(
    HDC hDC,
    int pos,
    HPEN hPen,
    int scale,
    int patMode,
    PCLOCKSTR np)
{
    LPPOINT lppt;
    int radius;

    MoveTo(hDC, np->clockCenter.x, np->clockCenter.y);
    radius = MulDiv(np->clockRadius, scale, 100);
    lppt = rCircleTable + pos;
    SetROP2(hDC, patMode);
    SelectObject(hDC, hPen);

    LineTo( hDC,
            np->clockCenter.x + MulDiv(lppt->x, radius, 8000),
            np->clockCenter.y + MulDiv(lppt->y, radius, 8000) );
}
\end{lstlisting}

Now it's clear: coordinates has been precalculated as if clock face has height and width of $2 \cdot 8000$,
and then it's rescaled to current clock face radius using \emph{MulDiv()} function.

POINT structure\footnote{\url{https://msdn.microsoft.com/en-us/library/windows/desktop/dd162805(v=vs.85).aspx}}
is a structure of two 32-bit values, first is \emph{x}, second is \emph{y}.


\mysection{(Windows 7) Solitaire: practical jokes}

\input{examples/solitaire/51/main_EN}
\input{examples/solitaire/53/main_EN}


\EN{\input{patterns/016_empty_redux/main_EN}}%
\FR{\input{patterns/016_empty_redux/main_FR}}


% I never liked this part:
% \input{examples/qr9/qr9_EN}

% TODO translate
% TODO: OpenSSL tool, URLs, etc
\mysection{Encrypted database case \#1}
\label{encrypted_DB1}

(This part has been first appeared in my blog at 26-Aug-2015.
Some discussion: \url{https://news.ycombinator.com/item?id=10128684}.)

\subsection{Base64 and entropy}

\myindex{XML}
I've got the \ac{XML} file containing some encrypted data.
Perhaps, it's related to some orders and/or customers information.

\begin{lstlisting}
<?xml version = "1.0" encoding = "UTF-8"?>
<Orders>
	<Order>
		<OrderID>1</OrderID>
		<Data>yjmxhXUbhB/5MV45chPsXZWAJwIh1S0aD9lFn3XuJMSxJ3/E+UE3hsnH</Data>
	</Order>
	<Order>
		<OrderID>2</OrderID>
		<Data>0KGe/wnypFBjsy+U0C2P9fC5nDZP3XDZLMPCRaiBw9OjIk6Tu5U=</Data>
	</Order>
	<Order>
		<OrderID>3</OrderID>
		<Data>mqkXfdzvQKvEArdzh+zD9oETVGBFvcTBLs2ph1b5bYddExzp</Data>
	</Order>
	<Order>
		<OrderID>4</OrderID>
		<Data>FCx6JhIDqnESyT3HAepyE1BJ3cJd7wCk+APCRUeuNtZdpCvQ2MR/7kLXtfUHuA==</Data>
	</Order>
...
\end{lstlisting}

The file is available \href{https://raw.githubusercontent.com/DennisYurichev/RE-for-beginners/master/examples/encrypted_DB1/encrypted.xml}{here}.

\myindex{base64}
This is clearly base64-encoded data, because all strings consisting of Latin characters, digits,
plus (+) and slash (/) symbols.
There can be 1 or 2 padding symbols (=), but they are never occurred in the middle of string.
Keeping in mind these base64 properties, it's very easy to recognize them.

Let's decode them and calculate entropies (\myref{entropy}) of these blocks in Wolfram Mathematica:

\begin{lstlisting}
In[]:= ListOfBase64Strings =
  Map[First[#[[3]]] &, Cases[Import["encrypted.xml"], XMLElement["Data", _, _], Infinity]];

In[]:= BinaryStrings =
  Map[ImportString[#, {"Base64", "String"}] &, ListOfBase64Strings];

In[]:= Entropies = Map[N[Entropy[2, #]] &, BinaryStrings];

In[]:= Variance[Entropies]
Out[]= 0.0238614
\end{lstlisting}

\myindex{Variance}
Variance is low.
This means the entropy values are not very different from each other.
This is visible on graph:

\begin{lstlisting}
In[]:= ListPlot[Entropies]
\end{lstlisting}

\begin{figure}[H]
\centering
\myincludegraphics{examples/encrypted_DB1/entropy.png}
\end{figure}

Most values are between 5.0 and 5.4.
This is a sign that the data is compressed and/or encrypted.

To understand variance, let's calculate entropies of all lines in Conan Doyle's \emph{The Hound of the Baskervilles} book:

\begin{lstlisting}
In[]:= BaskervillesLines = Import["http://www.gutenberg.org/cache/epub/2852/pg2852.txt", "List"];

In[]:= EntropiesT = Map[N[Entropy[2, #]] &, BaskervillesLines];

In[]:= Variance[EntropiesT]
Out[]= 2.73883

In[]:= ListPlot[EntropiesT]
\end{lstlisting}

\begin{figure}[H]
\centering
\myincludegraphics{examples/encrypted_DB1/conan_doyle.png}
\end{figure}

Most values are gathered around value of 4, but there are also values which are smaller,
and they are influenced final variance value.

Perhaps, shortest strings has smaller entropy, let's take short string from the Conan Doyle's book:

\begin{lstlisting}
In[]:= Entropy[2, "Yes, sir."] // N
Out[]= 2.9477
\end{lstlisting}

Let's try even shorter:

\begin{lstlisting}
In[]:= Entropy[2, "Yes"] // N
Out[]= 1.58496

In[]:= Entropy[2, "No"] // N
Out[]= 1.
\end{lstlisting}

\subsection{Is data compressed?}

OK, so our data is compressed and/or encrypted.
Is it compressed? Almost all data compressors put some header at the start, signature, or something like that.
As we can see, there are no consistent data at the start of each block.
It's still possible that this is a handmade data compressor, but they are very rare.
On the other hand, handmade cryptoalgorithms are much more popular, because it's very easy to make it work.
\myindex{memfrob()}
\myindex{ROT13}
Even primitive keyless cryptosystems like \emph{memfrob()}\footnote{\url{http://linux.die.net/man/3/memfrob}}
and ROT13 works fine without errors.
It's a serious challenge to write data compressor from scratch using only fantasy and imagination in a way so it will have no evident bugs.
Some programmers implements data compression functions by reading textbooks, but this is also rare.
The most popular two ways are:
\myindex{zlib}
1) just take open-source library like zlib;
2) copy\&paste something from somewhere.
Open-source data compressions algorithms usually puts some kind of header, and so do
algorithms from sites like \url{http://www.codeproject.com/}.

\subsection{Is data encrypted?}

Major data encryption algorithms process data in blocks. DES---8 bytes, AES---16 bytes.
If the input buffer is not divided evenly by block size, it's padded by zeroes (or something else),
so encrypted data will be aligned by cryptoalgorithm's block size.
This is not our case.

Using Wolfram Mathematica, I analyzed block's lengths:

\begin{lstlisting}
In[]:= Counts[Map[StringLength[#] &, BinaryStrings]]
Out[]= <|42 -> 1858, 38 -> 1235, 36 -> 699, 46 -> 1151, 40 -> 1784,
 44 -> 1558, 50 -> 366, 34 -> 291, 32 -> 74, 56 -> 15, 48 -> 716,
 30 -> 13, 52 -> 156, 54 -> 71, 60 -> 3, 58 -> 6, 28 -> 4|>
\end{lstlisting}

1858 blocks has size of 42 bytes, 1235 blocks has size of 38 bytes, etc.

I made a graph:

\begin{lstlisting}
ListPlot[Counts[Map[StringLength[#] &, BinaryStrings]]]
\end{lstlisting}

\begin{figure}[H]
\centering
\myincludegraphics{examples/encrypted_DB1/lengths.png}
\end{figure}

So, most blocks has size between $\textasciitilde{}36$ and $\textasciitilde{}48$.
There is also another thing to notice: all block sizes are even.
No single block with odd size.

There are, however, stream ciphers which can operate on byte level or even on bit level.

\subsection{CryptoPP}
\myindex{CryptoPP}

The program which can browse this encrypted database is written C\# and the .NET code
is heavily obfuscated.
Nevertheless, there is DLL with x86 code, which, after brief examination,
has parts of the CryptoPP popular open-source library!
(I just spotted \q{CryptoPP} strings inside.)
Now it's very easy to find all functions inside of DLL because CryptoPP library is open-source.

\myindex{AES}
CryptoPP library has a lot of crypto-functions, including AES (AKA Rijndael).
Newer x86 CPUs has AES helper instructions like \INS{AESENC}, \INS{AESDEC} and \INS{AESKEYGENASSIST}
\footnote{\url{https://en.wikipedia.org/wiki/AES_instruction_set}}.
They are not performing encryption/decryption completely, but they do significant amount of job.
And newer CryptoPP versions use them.
For example, here:
\href{https://github.com/mmoss/cryptopp/blob/2772f7b57182b31a41659b48d5f35a7b6cedd34d/src/rijndael.cpp#L1034}{1},
\href{https://github.com/mmoss/cryptopp/blob/2772f7b57182b31a41659b48d5f35a7b6cedd34d/src/rijndael.cpp#L1000}{2}.
\myindex{x86!\Instructions!AESENC}
\myindex{x86!\Instructions!AESDEC}
\myindex{tracer}
To my surprise, during decryption, \INS{AESENC} gets executed, while \INS{AESDEC} is not
(I just checked with my tracer utility, but any debugger can be used).
I checked, if my CPU really supports AES instructions. Some Intel i3 CPUs are not.
And if not, CryptoPP library falling back to AES functions implemented in old way
\footnote{\url{https://github.com/mmoss/cryptopp/blob/2772f7b57182b31a41659b48d5f35a7b6cedd34d/src/rijndael.cpp#L355}}.
But my CPU supports them.
Why \INS{AESDEC} is still not executed?
Why the program use AES encryption in order to decrypt database?

OK, it's not a problem to find a function which encrypts block.
It is called \\
\emph{CryptoPP::Rijndael::Enc::ProcessAndXorBlock}:
\href{https://github.com/mmoss/cryptopp/blob/2772f7b57182b31a41659b48d5f35a7b6cedd34d/src/rijndael.cpp#L349}{src},
and it can call another function: \\
\emph{Rijndael::Enc::AdvancedProcessBlocks()}
\href{https://github.com/mmoss/cryptopp/blob/2772f7b57182b31a41659b48d5f35a7b6cedd34d/src/rijndael.cpp#L1179}{src},
which, in turn, can call two other functions (
\href{https://github.com/mmoss/cryptopp/blob/2772f7b57182b31a41659b48d5f35a7b6cedd34d/src/rijndael.cpp#L1000}{AESNI\_Enc\_Block}
and
\href{https://github.com/mmoss/cryptopp/blob/2772f7b57182b31a41659b48d5f35a7b6cedd34d/src/rijndael.cpp#L1012}{AESNI\_Enc\_4\_Blocks}
)
which has \INS{AESENC} instructions.

So, judging by CryptoPP internals, \\
\emph{CryptoPP::Rijndael::Enc::ProcessAndXorBlock()} encrypts one 16-byte block.
Let's set breakpoint on it and see, what happens during decryption.
I use my simple tracer tool again.
The software must decrypt first data block now.
Oh, by the way, here is the first data block converted from base64 encoding to hexadecimal data,
let's have it at hand:

\lstinputlisting{examples/encrypted_DB1/1.lst}

These are also arguments of the function from CryptoPP source files:

\begin{lstlisting}
size_t Rijndael::Enc::AdvancedProcessBlocks(const byte *inBlocks, const byte *xorBlocks, byte *outBlocks, size_t length, word32 flags);
\end{lstlisting}

So it has 5 arguments. Possible flags are:

\begin{lstlisting}
enum {BT_InBlockIsCounter=1, BT_DontIncrementInOutPointers=2, BT_XorInput=4, BT_ReverseDirection=8, BT_AllowParallel=16} FlagsForAdvancedProcessBlocks;
\end{lstlisting}

OK, run tracer on \emph{ProcessAndXorBlock()} function:

\lstinputlisting{examples/encrypted_DB1/2.lst}

Here we can see inputs to the \emph{ProcessAndXorBlock()} function, and outputs from it.

This is output from the function during first call:

\begin{lstlisting}
00000000: C7 39 4E 7B 33 1B D6 1F-B8 31 10 39 39 13 A5 5D ".9N{3....1.99..]"
\end{lstlisting}

Then the \emph{ProcessAndXorBlock()} is called with zero-length block, but with 8 flag (\emph{BT\_ReverseDirection}).

Second call:

\begin{lstlisting}
00000000: 45 00 20 00 4A 00 4F 00-48 00 4E 00 53 00 00 00 "E. .J.O.H.N.S..."
\end{lstlisting}

Wow, there is some string familiar to us!

Third call:

\begin{lstlisting}
00000000: B1 27 7F E4 9F 01 E3 81-CF C6 12 FB B9 7C F1 BC ".'...........|.."
\end{lstlisting}

The first output is very similar to the first 16 bytes of the encrypted buffer.

Output of the first call of \emph{ProcessAndXorBlock()}:

\begin{lstlisting}
00000000: C7 39 4E 7B 33 1B D6 1F-B8 31 10 39 39 13 A5 5D ".9N{3....1.99..]"
\end{lstlisting}

First 16 bytes of encrypted buffer:

\begin{lstlisting}
00000000: CA 39 B1 85 75 1B 84 1F F9 31 5E 39 72 13 EC 5D  .9..u....1^9r..]
\end{lstlisting}

There are too much equal bytes!
How come AES encryption result can be very similar to the encrypted buffer while this is not
encryption but rather decryption?!

\subsection{Cipher Feedback mode}

\myindex{Cipher Feedback mode}
\myindex{XOR}
The answer is \ac{CFB}:
in this mode, AES algorithm used not as encryption algorithm, but as a device which generates cryptographically secure random data.
The actual encryption is happening using simple XOR operation.

Here is encryption algorithm (images are taken from Wikipedia):

\begin{figure}[H]
\centering
\myincludegraphics{examples/encrypted_DB1/601px-CFB_encryption.png}
\end{figure}

And decryption:

\begin{figure}[H]
\centering
\myincludegraphics{examples/encrypted_DB1/601px-CFB_decryption.png}
\label{fig:CFB_decryption}
\end{figure}

Now let's see: AES encryption operation generates 16 bytes (or 128 bits) of \emph{random} data
to be used while XOR-ing, who forces us to use all 16 bytes?
If at the last iteration we've got 1 byte of data, let's xor 1 byte of data with 1 byte of generated
\emph{random} data?
This leads to important property of \ac{CFB} mode: data can be not padded, data of arbitrary size
can be encrypted and decrypted.

Oh, that's why all encrypted blocks are not padded.
And that's why \INS{AESDEC} instruction is never called.

Let's try to decrypt first block manually, using Python.
\ac{CFB} mode also use \ac{IV}, as a \emph{seed} for \ac{CSPRNG}.
In our case, \ac{IV} is the block which is encrypted at first iteration:

\begin{lstlisting}
0038B920: 01 00 00 00 FF FF FF FF-79 C1 69 0B 67 C1 04 7D "........y.i.g..}"
\end{lstlisting}

Oh, and we also have to recover encryption key.
\myindex{x86!\Instructions!AESKEYGENASSIST}
There is \INS{AESKEYGENASSIST} is DLL, and it is called, and it is used in the \\
\emph{Rijndael::Base::UncheckedSetKey()} function:
\href{https://github.com/mmoss/cryptopp/blob/2772f7b57182b31a41659b48d5f35a7b6cedd34d/src/rijndael.cpp#L198}{src}.
It's easy to find it in IDA and set breakpoint. Let's see:

\begin{lstlisting}
... tracer.exe -l:filename.exe bpf=filename.exe!0x435c30,args:3,dump_args:0x10

Warning: no tracer.cfg file.
PID=2068|New process software.exe
no module registered with image base 0x77320000
no module registered with image base 0x76e20000
no module registered with image base 0x77320000
no module registered with image base 0x77220000
Warning: unknown (to us) INT3 breakpoint at ntdll.dll!LdrVerifyImageMatchesChecksum+0x96c (0x776c103b)
(0) software.exe!0x435c30(0x15e8000, 0x10, 0x14f808) (called from software.exe!.text+0x22fa1 (0x13d3fa1))
Argument 1/3
015E8000: CD C5 7E AD 28 5F 6D E1-CE 8F CC 29 B1 21 88 8E "..~.(_m....).!.."
Argument 3/3
0014F808: 38 82 58 01 C8 B9 46 00-01 D1 3C 01 00 F8 14 00 "8.X...F...<....."
Argument 3/3 +0x0: software.exe!.rdata+0x5238
Argument 3/3 +0x8: software.exe!.text+0x1c101
(0) software.exe!0x435c30() -> 0x13c2801
PID=2068|Process software.exe exited. ExitCode=0 (0x0)
\end{lstlisting}

So this is the key: \emph{CD C5 7E AD 28 5F 6D E1-CE 8F CC 29 B1 21 88 8E}.

During manual decryption we've got this:

\begin{lstlisting}
00000000: 0D 00 FF FE 46 00 52 00  41 00 4E 00 4B 00 49 00  ....F.R.A.N.K.I.
00000010: 45 00 20 00 4A 00 4F 00  48 00 4E 00 53 00 66 66  E. .J.O.H.N.S.ff
00000020: 66 66 66 9E 61 40 D4 07  06 01                    fff.a@....
\end{lstlisting}

Now this is something readable!
And now we can see why there were so many equal bytes at the first decryption iteration:
because plaintext has so many zero bytes!

Let's decrypt the second block:

\begin{lstlisting}
00000000: 17 98 D0 84 3A E9 72 4F  DB 82 3F AD E9 3E 2A A8  ....:.rO..?..>*.
00000010: 41 00 52 00 52 00 4F 00  4E 00 CD CC CC CC CC CC  A.R.R.O.N.......
00000020: 1B 40 D4 07 06 01                                 .@....
\end{lstlisting}

Third, fourth and fifth:

\begin{lstlisting}
00000000: 5D 90 59 06 EF F4 96 B4  7C 33 A7 4A BE FF 66 AB  ].Y.....|3.J..f.
00000010: 49 00 47 00 47 00 53 00  00 00 00 00 00 C0 65 40  I.G.G.S.......e@
00000020: D4 07 06 01                                       ....
\end{lstlisting}

\begin{lstlisting}
00000000: D3 15 34 5D 21 18 7C 6E  AA F8 2D FE 38 F9 D7 4E  ..4]!.|n..-.8..N
00000010: 41 00 20 00 44 00 4F 00  48 00 45 00 52 00 54 00  A. .D.O.H.E.R.T.
00000020: 59 00 48 E1 7A 14 AE FF  68 40 D4 07 06 02        Y.H.z...h@....
\end{lstlisting}

\begin{lstlisting}
00000000: 1E 8B 90 0A 17 7B C5 52  31 6C 4E 2F DE 1B 27 19  .....{.R1lN...'.
00000010: 41 00 52 00 43 00 55 00  53 00 00 00 00 00 00 60  A.R.C.U.S.......
00000020: 66 40 D4 07 06 03                                 f@....
\end{lstlisting}

All blocks decrypted seems correct except of first 16 bytes part.

\subsection{Initializing Vector}

What can affect first 16 bytes?

Let's back to \ac{CFB} decryption algorithm again: \myref{fig:CFB_decryption}.

We can see that \ac{IV} can affect to first block decryption operation, but not the second,
because during the second iteration, ciphertext from the first iteration is used, and in case of decryption,
it's the same, no matter what \ac{IV} has!

So probably, \ac{IV} is different each time.
Using my tracer, I'll take a look at the first input during decryption of the second block
of \ac{XML} file:

\begin{lstlisting}
0038B920: 02 00 00 00 FE FF FF FF-79 C1 69 0B 67 C1 04 7D "........y.i.g..}"
\end{lstlisting}

\dots third:

\begin{lstlisting}
0038B920: 03 00 00 00 FD FF FF FF-79 C1 69 0B 67 C1 04 7D "........y.i.g..}"
\end{lstlisting}

It seems, first and fifth byte are changed each time.
I finally concluded that the first 32-bit integer is just OrderID from the \ac{XML} file,
and the second 32-bit integer is also OrderID, but negated. All other 8 bytes are same for each operation.
Now I have decrypted the whole database:
\url{https://raw.githubusercontent.com/DennisYurichev/RE-for-beginners/master/examples/encrypted_DB1/decrypted.full.txt}.

The Python script used for this is:
\url{\GitHubBlobMasterURL/examples/encrypted_DB1/decrypt_blocks.py}.

Perhaps, the author wanted each block encrypted differently, so he/she used OrderID as part of key.
It would be also possible to make different AES key instead of \ac{IV}.

So now we know that \ac{IV} only affects first block during decryption in \ac{CFB} mode, this is
feature of it.
All other blocks can be decrypted without knowledge \ac{IV}, but using the key.

OK, so why \ac{CFB} mode? Apparently, because the very first AES example on CryptoPP wiki
uses \ac{CFB} mode:
\url{http://www.cryptopp.com/wiki/Advanced_Encryption_Standard#Encrypting_and_Decrypting_Using_AES}.
Supposedly, developer choose it for simplicity:
the example can encrypt/decrypt text strings with arbitrary lengths, without padding.

It is very likely, program's author(s) just copypasted the example from CryptoPP wiki page.
Many programmers do so.

The only difference that \ac{IV} is chosen randomly in CryptoPP wiki example, while this indeterminism
wasn't allowable to programmers of the software we are dissecting now,
so they choose to initialize \ac{IV} using Order ID.

Now we can proceed to analyzing matter of each byte in the decrypted block.

\subsection{Structure of the buffer}

Let's take first four decrypted blocks:

\begin{lstlisting}
00000000: 0D 00 FF FE 46 00 52 00  41 00 4E 00 4B 00 49 00  ....F.R.A.N.K.I.
00000010: 45 00 20 00 4A 00 4F 00  48 00 4E 00 53 00 66 66  E. .J.O.H.N.S.ff
00000020: 66 66 66 9E 61 40 D4 07  06 01                    fff.a@....

00000000: 0B 00 FF FE 4C 00 4F 00  52 00 49 00 20 00 42 00  ....L.O.R.I. .B.
00000010: 41 00 52 00 52 00 4F 00  4E 00 CD CC CC CC CC CC  A.R.R.O.N.......
00000020: 1B 40 D4 07 06 01                                 .@....

00000000: 0A 00 FF FE 47 00 41 00  52 00 59 00 20 00 42 00  ....G.A.R.Y. .B.
00000010: 49 00 47 00 47 00 53 00  00 00 00 00 00 C0 65 40  I.G.G.S.......e@
00000020: D4 07 06 01                                       ....

00000000: 0F 00 FF FE 4D 00 45 00  4C 00 49 00 4E 00 44 00  ....M.E.L.I.N.D.
00000010: 41 00 20 00 44 00 4F 00  48 00 45 00 52 00 54 00  A. .D.O.H.E.R.T.
00000020: 59 00 48 E1 7A 14 AE FF  68 40 D4 07 06 02        Y.H.z...h@....
\end{lstlisting}

UTF-16 encoded text strings are clearly visible, these are names and surnames.
The first byte (or 16-bit word) is seems string length, we can visually check it.
\emph{FF FE} is seems Unicode \ac{BOM}.

There are 12 more bytes after each string.

Using this script
(\url{\GitHubBlobMasterURL/examples/encrypted_DB1/dump_buffer_rest.py})
I've got random selection of the \emph{tails}:

\lstinputlisting{examples/encrypted_DB1/tails.lst}

We first see the 0x40 and 0x07 bytes present in each \emph{tail}.
The very last byte s always in 1..0x1F (1..31) range, I've checked.
The penultimate byte is always in 1..0xC (1..12) range.
Wow, that looks like a date!
Year can be represented as 16-bit value, and maybe last 4 bytes is date (16 bits for year, 8 bits
for month and 8 more for day)?
0x7DD is 2013, 0x7D5 is 2005, etc. Seems fine. This is a date.
There are 8 more bytes.
Judging by the fact this is database named \emph{orders}, maybe some kind of sum is present here?
I made attempt to interpret it as double-precision IEEE 754 floating point and dump all values!

Some are:

\begin{lstlisting}
71.0
134.0
51.95
53.0
121.99
96.95
98.95
15.95
85.95
184.99
94.95
29.95
85.0
36.0
130.99
115.95
87.99
127.95
114.0
150.95
\end{lstlisting}

Looks like real!

Now we can dump names, sums and dates.

\begin{lstlisting}
plain:
00000000: 0D 00 FF FE 46 00 52 00  41 00 4E 00 4B 00 49 00  ....F.R.A.N.K.I.
00000010: 45 00 20 00 4A 00 4F 00  48 00 4E 00 53 00 66 66  E. .J.O.H.N.S.ff
00000020: 66 66 66 9E 61 40 D4 07  06 01                    fff.a@....
OrderID= 1 name= FRANKIE JOHNS sum= 140.95 date= 2004 / 6 / 1

plain:
00000000: 0B 00 FF FE 4C 00 4F 00  52 00 49 00 20 00 42 00  ....L.O.R.I. .B.
00000010: 41 00 52 00 52 00 4F 00  4E 00 CD CC CC CC CC CC  A.R.R.O.N.......
00000020: 1B 40 D4 07 06 01                                 .@....
OrderID= 2 name= LORI BARRON sum= 6.95 date= 2004 / 6 / 1

plain:
00000000: 0A 00 FF FE 47 00 41 00  52 00 59 00 20 00 42 00  ....G.A.R.Y. .B.
00000010: 49 00 47 00 47 00 53 00  00 00 00 00 00 C0 65 40  I.G.G.S.......e@
00000020: D4 07 06 01                                       ....
OrderID= 3 name= GARY BIGGS sum= 174.0 date= 2004 / 6 / 1

plain:
00000000: 0F 00 FF FE 4D 00 45 00  4C 00 49 00 4E 00 44 00  ....M.E.L.I.N.D.
00000010: 41 00 20 00 44 00 4F 00  48 00 45 00 52 00 54 00  A. .D.O.H.E.R.T.
00000020: 59 00 48 E1 7A 14 AE FF  68 40 D4 07 06 02        Y.H.z...h@....
OrderID= 4 name= MELINDA DOHERTY sum= 199.99 date= 2004 / 6 / 2

plain:
00000000: 0B 00 FF FE 4C 00 45 00  4E 00 41 00 20 00 4D 00  ....L.E.N.A. .M.
00000010: 41 00 52 00 43 00 55 00  53 00 00 00 00 00 00 60  A.R.C.U.S.......
00000020: 66 40 D4 07 06 03                                 f@....
OrderID= 5 name= LENA MARCUS sum= 179.0 date= 2004 / 6 / 3
\end{lstlisting}

See more: \url{https://raw.githubusercontent.com/DennisYurichev/RE-for-beginners/master/examples/encrypted_DB1/decrypted.full.with_data.txt}.
Or filtered: \url{\GitHubBlobMasterURL/examples/encrypted_DB1/decrypted.short.txt}.
Seems correct.

This is some kind of \ac{OOP} serialization, i.e., packing differently typed values into binary buffer for storing and/or transmitting.

\subsection{Noise at the end}

The only question remaining is that sometimes, \emph{tail} is bigger:

\begin{lstlisting}
00000000: 0E 00 FF FE 54 00 48 00  45 00 52 00 45 00 53 00  ....T.H.E.R.E.S.
00000010: 45 00 20 00 54 00 55 00  54 00 54 00 4C 00 45 00  E. .T.U.T.T.L.E.
00000020: 66 66 66 66 66 1E 63 40  D4 07 07 1A 00 07 07 19  fffff.c@........
OrderID= 172 name= THERESE TUTTLE sum= 152.95 date= 2004 / 7 / 26
\end{lstlisting}

(\emph{00 07 07 19} bytes are not used and is ballast.)

\begin{lstlisting}
00000000: 0C 00 FF FE 4D 00 45 00  4C 00 41 00 4E 00 49 00  ....M.E.L.A.N.I.
00000010: 45 00 20 00 4B 00 49 00  52 00 4B 00 00 00 00 00  E. .K.I.R.K.....
00000020: 00 20 64 40 D4 07 09 02  00 02                    . d@......
OrderID= 286 name= MELANIE KIRK sum= 161.0 date= 2004 / 9 / 2
\end{lstlisting}

(\emph{00 02} are not used.)

After close examination, we can see, that the noise at the end of \emph{tail} is just left
from previous encryption!

Here are two subsequent buffers:

\begin{lstlisting}
00000000: 10 00 FF FE 42 00 4F 00  4E 00 4E 00 49 00 45 00  ....B.O.N.N.I.E.
00000010: 20 00 47 00 4F 00 4C 00  44 00 53 00 54 00 45 00   .G.O.L.D.S.T.E.
00000020: 49 00 4E 00 9A 99 99 99  99 79 46 40 D4 07 07 19  I.N......yF@....
OrderID= 171 name= BONNIE GOLDSTEIN sum= 44.95 date= 2004 / 7 / 25

00000000: 0E 00 FF FE 54 00 48 00  45 00 52 00 45 00 53 00  ....T.H.E.R.E.S.
00000010: 45 00 20 00 54 00 55 00  54 00 54 00 4C 00 45 00  E. .T.U.T.T.L.E.
00000020: 66 66 66 66 66 1E 63 40  D4 07 07 1A 00 07 07 19  fffff.c@........
OrderID= 172 name= THERESE TUTTLE sum= 152.95 date= 2004 / 7 / 26
\end{lstlisting}

(The last \emph{07 07 19} bytes are copied from the previous plaintext buffer.)

Another two subsequent buffers:

\begin{lstlisting}
00000000: 0D 00 FF FE 4C 00 4F 00  52 00 45 00 4E 00 45 00  ....L.O.R.E.N.E.
00000010: 20 00 4F 00 54 00 4F 00  4F 00 4C 00 45 00 CD CC   .O.T.O.O.L.E...
00000020: CC CC CC 3C 5E 40 D4 07  09 02                    ...<^@....
OrderID= 285 name= LORENE OTOOLE sum= 120.95 date= 2004 / 9 / 2

00000000: 0C 00 FF FE 4D 00 45 00  4C 00 41 00 4E 00 49 00  ....M.E.L.A.N.I.
00000010: 45 00 20 00 4B 00 49 00  52 00 4B 00 00 00 00 00  E. .K.I.R.K.....
00000020: 00 20 64 40 D4 07 09 02  00 02                    . d@......
OrderID= 286 name= MELANIE KIRK sum= 161.0 date= 2004 / 9 / 2
\end{lstlisting}

The last 02 byte has been copied from the previous plaintext buffer.

It's possible if the buffer used while encrypting is global and/or isn't clearing before
each encryption.
The final buffer size is also chaotic, nevertheless, the bug left uncaught
because it doesn't affect decrypting process, which just ignores noise at the end.
This is common mistake.
\myindex{OpenSSL}
\myindex{Heartbleed}
It's been present in OpenSSL (Heartbleed bug).

\subsection{Conclusion}

Summary:
every practicing reverse engineer should be familiar with major crypto algorithms and
also major cryptographical modes.
Some books about it: \myref{crypto_books}.

\emph{Encrypted} database contents has been artificially constructed by me for the sake of demonstration.
I've got most popular USA names and surnames from there: \url{http://stackoverflow.com/questions/1803628/raw-list-of-person-names},
and combined them randomly.
Dates and sums were also generated randomly.

All files used in this part are here: \url{\GitHubTreeMasterURL/examples/encrypted_DB1}.

Nevertheless, many features like these I've observed in real-world software applications.
This example is based on them.

\subsection{Post Scriptum: brute-forcing \ac{IV}}

The case you have just seen has been artificially constructed, but is based on a real application I've reverse engineered.
When I've been working on it, I first noticed that \ac{IV} has been generating using some 32-bit number,
and I wasn't able to find a link between this value and OrderID.
So I prepared to use brute-force, which is indeed possible here.

It's not a problem to enumerate all 32-bit values and try each as a base for \ac{IV}.
Then you decrypt the first 16-byte block and check for zero bytes, which are always at fixed places.

\mysection{Overclocking Cointerra Bitcoin miner}
\myindex{Bitcoin}
\myindex{BeagleBone}

There was Cointerra Bitcoin miner, looking like that:

\begin{figure}[H]
\centering
\myincludegraphics{examples/bitcoin_miner/board.jpg}
\caption{Board}
\end{figure}

And there was also (possibly leaked) utility\footnote{Can be downloaded here: \url{\GitHubURL/raw/master/examples/bitcoin_miner/files/cointool-overclock}}
which can set clock rate for the board.
It runs on additional BeagleBone Linux ARM board (small board at bottom of the picture).

And the author was once asked, is it possible to hack this utility to see, which frequency can be set and which are not.
And it is possible to tweak it?

The utility must be executed like that: \TT{./cointool-overclock 0 0 900}, where 900 is frequency in MHz.
If the frequency is too high, utility will print \q{Error with arguments} and exit.

This is a fragment of code around reference to \q{Error with arguments} text string:

\begin{lstlisting}[style=customasmARM]

...

.text:0000ABC4         STR      R3, [R11,#var_28]
.text:0000ABC8         MOV      R3, #optind
.text:0000ABD0         LDR      R3, [R3]
.text:0000ABD4         ADD      R3, R3, #1
.text:0000ABD8         MOV      R3, R3,LSL#2
.text:0000ABDC         LDR      R2, [R11,#argv]
.text:0000ABE0         ADD      R3, R2, R3
.text:0000ABE4         LDR      R3, [R3]
.text:0000ABE8         MOV      R0, R3  ; nptr
.text:0000ABEC         MOV      R1, #0  ; endptr
.text:0000ABF0         MOV      R2, #0  ; base
.text:0000ABF4         BL       strtoll
.text:0000ABF8         MOV      R2, R0
.text:0000ABFC         MOV      R3, R1
.text:0000AC00         MOV      R3, R2
.text:0000AC04         STR      R3, [R11,#var_2C]
.text:0000AC08         MOV      R3, #optind
.text:0000AC10         LDR      R3, [R3]
.text:0000AC14         ADD      R3, R3, #2
.text:0000AC18         MOV      R3, R3,LSL#2
.text:0000AC1C         LDR      R2, [R11,#argv]
.text:0000AC20         ADD      R3, R2, R3
.text:0000AC24         LDR      R3, [R3]
.text:0000AC28         MOV      R0, R3  ; nptr
.text:0000AC2C         MOV      R1, #0  ; endptr
.text:0000AC30         MOV      R2, #0  ; base
.text:0000AC34         BL       strtoll
.text:0000AC38         MOV      R2, R0
.text:0000AC3C         MOV      R3, R1
.text:0000AC40         MOV      R3, R2
.text:0000AC44         STR      R3, [R11,#third_argument]
.text:0000AC48         LDR      R3, [R11,#var_28]
.text:0000AC4C         CMP      R3, #0
.text:0000AC50         BLT      errors_with_arguments
.text:0000AC54         LDR      R3, [R11,#var_28]
.text:0000AC58         CMP      R3, #1
.text:0000AC5C         BGT      errors_with_arguments
.text:0000AC60         LDR      R3, [R11,#var_2C]
.text:0000AC64         CMP      R3, #0
.text:0000AC68         BLT      errors_with_arguments
.text:0000AC6C         LDR      R3, [R11,#var_2C]
.text:0000AC70         CMP      R3, #3
.text:0000AC74         BGT      errors_with_arguments
.text:0000AC78         LDR      R3, [R11,#third_argument]
.text:0000AC7C         CMP      R3, #0x31
.text:0000AC80         BLE      errors_with_arguments
.text:0000AC84         LDR      R2, [R11,#third_argument]
.text:0000AC88         MOV      R3, #950
.text:0000AC8C         CMP      R2, R3
.text:0000AC90         BGT      errors_with_arguments
.text:0000AC94         LDR      R2, [R11,#third_argument]
.text:0000AC98         MOV      R3, #0x51EB851F
.text:0000ACA0         SMULL    R1, R3, R3, R2
.text:0000ACA4         MOV      R1, R3,ASR#4
.text:0000ACA8         MOV      R3, R2,ASR#31
.text:0000ACAC         RSB      R3, R3, R1
.text:0000ACB0         MOV      R1, #50
.text:0000ACB4         MUL      R3, R1, R3
.text:0000ACB8         RSB      R3, R3, R2
.text:0000ACBC         CMP      R3, #0
.text:0000ACC0         BEQ      loc_ACEC
.text:0000ACC4
.text:0000ACC4 errors_with_arguments
.text:0000ACC4                                         
.text:0000ACC4         LDR      R3, [R11,#argv]
.text:0000ACC8         LDR      R3, [R3]
.text:0000ACCC         MOV      R0, R3  ; path
.text:0000ACD0         BL       __xpg_basename
.text:0000ACD4         MOV      R3, R0
.text:0000ACD8         MOV      R0, #aSErrorWithArgu ; format
.text:0000ACE0         MOV      R1, R3
.text:0000ACE4         BL       printf
.text:0000ACE8         B        loc_ADD4
.text:0000ACEC ; ------------------------------------------------------------
.text:0000ACEC
.text:0000ACEC loc_ACEC                 ; CODE XREF: main+66C
.text:0000ACEC         LDR      R2, [R11,#third_argument]
.text:0000ACF0         MOV      R3, #499
.text:0000ACF4         CMP      R2, R3
.text:0000ACF8         BGT      loc_AD08
.text:0000ACFC         MOV      R3, #0x64
.text:0000AD00         STR      R3, [R11,#unk_constant]
.text:0000AD04         B        jump_to_write_power
.text:0000AD08 ; ------------------------------------------------------------
.text:0000AD08
.text:0000AD08 loc_AD08                 ; CODE XREF: main+6A4
.text:0000AD08         LDR      R2, [R11,#third_argument]
.text:0000AD0C         MOV      R3, #799
.text:0000AD10         CMP      R2, R3
.text:0000AD14         BGT      loc_AD24
.text:0000AD18         MOV      R3, #0x5F
.text:0000AD1C         STR      R3, [R11,#unk_constant]
.text:0000AD20         B        jump_to_write_power
.text:0000AD24 ; ------------------------------------------------------------
.text:0000AD24
.text:0000AD24 loc_AD24                 ; CODE XREF: main+6C0
.text:0000AD24         LDR      R2, [R11,#third_argument]
.text:0000AD28         MOV      R3, #899
.text:0000AD2C         CMP      R2, R3
.text:0000AD30         BGT      loc_AD40
.text:0000AD34         MOV      R3, #0x5A
.text:0000AD38         STR      R3, [R11,#unk_constant]
.text:0000AD3C         B        jump_to_write_power
.text:0000AD40 ; ------------------------------------------------------------
.text:0000AD40
.text:0000AD40 loc_AD40                 ; CODE XREF: main+6DC
.text:0000AD40         LDR      R2, [R11,#third_argument]
.text:0000AD44         MOV      R3, #999
.text:0000AD48         CMP      R2, R3
.text:0000AD4C         BGT      loc_AD5C
.text:0000AD50         MOV      R3, #0x55
.text:0000AD54         STR      R3, [R11,#unk_constant]
.text:0000AD58         B        jump_to_write_power
.text:0000AD5C ; ------------------------------------------------------------
.text:0000AD5C
.text:0000AD5C loc_AD5C                 ; CODE XREF: main+6F8
.text:0000AD5C         LDR      R2, [R11,#third_argument]
.text:0000AD60         MOV      R3, #1099
.text:0000AD64         CMP      R2, R3
.text:0000AD68         BGT      jump_to_write_power
.text:0000AD6C         MOV      R3, #0x50
.text:0000AD70         STR      R3, [R11,#unk_constant]
.text:0000AD74
.text:0000AD74 jump_to_write_power                     ; CODE XREF: main+6B0
.text:0000AD74                                         ; main+6CC ...
.text:0000AD74         LDR      R3, [R11,#var_28]
.text:0000AD78         UXTB     R1, R3
.text:0000AD7C         LDR      R3, [R11,#var_2C]
.text:0000AD80         UXTB     R2, R3
.text:0000AD84         LDR      R3, [R11,#unk_constant]
.text:0000AD88         UXTB     R3, R3
.text:0000AD8C         LDR      R0, [R11,#third_argument]
.text:0000AD90         UXTH     R0, R0
.text:0000AD94         STR      R0, [SP,#0x44+var_44]
.text:0000AD98         LDR      R0, [R11,#var_24]
.text:0000AD9C         BL       write_power
.text:0000ADA0         LDR      R0, [R11,#var_24]
.text:0000ADA4         MOV      R1, #0x5A
.text:0000ADA8         BL       read_loop
.text:0000ADAC         B        loc_ADD4

...

.rodata:0000B378 aSErrorWithArgu DCB "%s: Error with arguments",0xA,0 ; DATA XREF: main+684

...

\end{lstlisting}

Function names were present in debugging information of the original binary, like \TT{write\_power}, \TT{read\_loop}.
But labels inside functions were named by me.

\myindex{UNIX!getopt}
\myindex{strtoll()}
\TT{optind} name looks familiar. It is from \emph{getopt} *NIX library intended for command-line parsing---well,
this is exactly what happens inside.
Then, the 3rd argument (where frequency value is to be passed) is converted from a string to a number using
a call to \emph{strtoll()} function.

The value is then checked against various constants.
At 0xACEC, it's checked, if it is lesser or equal to 499, and if it is so,
0x64 is to be passed to \TT{write\_power()} function (which sends a command through USB using \TT{send\_msg()}).
If it is greater than 499, jump to 0xAD08 is occurred.

At 0xAD08 it's checked, if it's lesser or equal to 799. 0x5F is then passed to \TT{write\_power()} function in case of success.

There are more checks: for 899 at 0xAD24, for 0x999 at 0xAD40 and finally, for 1099 at 0xAD5C.
If the input frequency is lesser or equal to 1099, 0x50 will be passed (at 0xAD6C) to \TT{write\_power()} function.
And there is some kind of bug.
If the value is still greater than 1099, the value itself is passed into \TT{write\_power()} function.
Oh, it's not a bug, because we can't get here: value is checked first against 950 at 0xAC88, and if it is greater, error message will be displayed and the utility will finish.

Now the table between frequency in MHz and value passed to \TT{write\_power()} function:

\begin{center}
\begin{longtable}{ | l | l | l | }
\hline
\HeaderColor MHz & \HeaderColor hexadecimal & \HeaderColor decimal \\
\hline
499MHz & 0x64 & 100 \\
\hline
799MHz & 0x5f & 95 \\
\hline
899MHz & 0x5a & 90 \\
\hline
999MHz & 0x55 & 85 \\
\hline
1099MHz & 0x50 & 80 \\
\hline
\end{longtable}
\end{center}

As it seems, a value passed to the board is gradually decreasing during frequency increasing.

Now we see that value of 950MHz is a hardcoded limit, at least in this utility. Can we trick it?

Let's back to this piece of code:

\begin{lstlisting}[style=customasmARM]
.text:0000AC84      LDR     R2, [R11,#third_argument]
.text:0000AC88      MOV     R3, #950
.text:0000AC8C      CMP     R2, R3
.text:0000AC90      BGT     errors_with_arguments ; I've patched here to 00 00 00 00
\end{lstlisting}

We must disable \INS{BGT} branch instruction at 0xAC90 somehow. And this is ARM in ARM mode, because, as we see, all addresses are increasing by 4, i.e., each instruction has size of 4 bytes.
\TT{NOP} (no operation) instruction in ARM mode is just four zero bytes: \TT{00 00 00 00}.
So by writing four zeros at 0xAC90 address (or physical offset in file 0x2C90) we can disable the check.

Now it's possible to set frequencies up to 1050MHz. Even more is possible, but due to the bug, if input value is greater than 1099, a value \emph{as is} in MHz will be passed to the board, which is incorrect.

I didn't go further, but if I had to, I would try to decrease a value which is passed to \TT{write\_power()} function.

Now the scary piece of code which I skipped at first:

\lstinputlisting[style=customasmARM]{examples/bitcoin_miner/tmp1.lst}

Division via multiplication is used here, and constant is 0x51EB851F.
I wrote a simple programmer's calculator\footnote{\url{https://github.com/DennisYurichev/progcalc}} for myself.
And I have there a feature to calculate modulo inverse.

\begin{lstlisting}
modinv32(0x51EB851F)
Warning, result is not integer: 3.125000
(unsigned) dec: 3 hex: 0x3 bin: 11
\end{lstlisting}

That means that \INS{SMULL} instruction at 0xACA0 is basically divides 3rd argument by 3.125.
In fact, all \TT{modinv32()} function in my calculator does, is this:

\[
\frac{1}{\frac{input}{2^{32}}} = \frac{2^{32}}{input}
\]

Then there are additional shifts and now we see than 3rg argument is just divided by 50.
And then it's multiplied by 50 again.
Why?
This is simplest check, if the input value is can be divided by 50 evenly.
If the value of this expression is non-zero, $x$ can't be divided by 50 evenly:

\[
x-((\frac{x}{50}) \cdot 50)
\]

This is in fact simple way to calculate remainder of division.

And then, if the remainder is non-zero, error message is displayed.
So this utility takes frequency values in form like 850, 900, 950, 1000, etc., but not 855 or 911.

That's it! If you do something like that, please be warned that you may damage your board, just as in case of overclocking other devices like \ac{CPU}s, \ac{GPU}s, etc.
If you have a Cointerra board, do this on your own risk!


% TODO translate
\mysection{Breaking simple executable cryptor}

I've got an executable file which is encrypted by relatively simple encryption.
\href{\GitHubBlobMasterURL/examples/simple_exec_crypto/files/cipher.bin}{Here is it} (only executable section is left here).

First, all encryption function does is just adds number of position in buffer to the byte.
Here is how this can be encoded in Python:

\begin{lstlisting}[caption=Python script,style=custompy]
#!/usr/bin/env python
def e(i, k):
    return chr ((ord(i)+k) % 256)

def encrypt(buf):
    return e(buf[0], 0)+ e(buf[1], 1)+ e(buf[2], 2) + e(buf[3], 3)+ e(buf[4], 4)+ e(buf[5], 5)+ e(buf[6], 6)+ e(buf[7], 7)+
           e(buf[8], 8)+ e(buf[9], 9)+ e(buf[10], 10)+ e(buf[11], 11)+ e(buf[12], 12)+ e(buf[13], 13)+ e(buf[14], 14)+ e(buf[15], 15)
\end{lstlisting}

Hence, if you encrypt buffer with 16 zeros, you'll get \emph{0, 1, 2, 3 ... 12, 13, 14, 15}.

\myindex{Propagating Cipher Block Chaining}
Propagating Cipher Block Chaining (PCBC) is also used, here is how it works:

\begin{figure}[H]
\centering
\myincludegraphics{examples/simple_exec_crypto/601px-PCBC_encryption.png}
\caption{Propagating Cipher Block Chaining encryption (image is taken from Wikipedia article)}
\end{figure}

The problem is that it's too boring to recover IV (Initialization Vector) each time.
Brute-force is also not an option, because IV is too long (16 bytes).
Let's see, if it's possible to recover IV for arbitrary encrypted executable file?

Let's try simple frequency analysis.
This is 32-bit x86 executable code, so let's gather statistics about most frequent bytes and opcodes.
I tried huge oracle.exe file from Oracle RDBMS version 11.2 for windows x86 and I've found that the most frequent byte (no surprise) is zero (~10\%).
The next most frequent byte is (again, no surprise) 0xFF (~5\%).
The next is 0x8B (~5\%).

\myindex{x86!\Instructions!MOV}
0x8B is opcode for \INS{MOV}, this is indeed one of the most busy x86 instructions.
Now what about popularity of zero byte?
If compiler needs to encode value bigger than 127, it has to use 32-bit displacement instead of 8-bit one, but large values are very rare,
so it is padded by zeros.
\myindex{x86!\Instructions!LEA}
\myindex{x86!\Instructions!PUSH}
\myindex{x86!\Instructions!CALL}
This is at least in \INS{LEA}, \INS{MOV}, \INS{PUSH}, \INS{CALL}.

For example:

\begin{lstlisting}[style=customasmx86]
8D B0 28 01 00 00                 lea     esi, [eax+128h]
8D BF 40 38 00 00                 lea     edi, [edi+3840h]
\end{lstlisting}

Displacements bigger than 127 are very popular, but they are rarely exceeds 0x10000
(indeed, such large memory buffers/structures are also rare).

Same story with \INS{MOV}, large constants are rare, the most heavily used are 0, 1, 10, 100, $2^n$, and so on.
Compiler has to pad small constants by zeros to represent them as 32-bit values:

\begin{lstlisting}[style=customasmx86]
BF 02 00 00 00                    mov     edi, 2
BF 01 00 00 00                    mov     edi, 1
\end{lstlisting}

Now about 00 and FF bytes combined: jumps (including conditional) and calls can pass execution flow forward or backwards, but very often,
within the limits of the current executable module.
If forward, displacement is not very big and also padded with zeros.
If backwards, displacement is represented as negative value, so padded with FF bytes.
For example, transfer execution flow forward:

\begin{lstlisting}[style=customasmx86]
E8 43 0C 00 00                    call    _function1
E8 5C 00 00 00                    call    _function2
0F 84 F0 0A 00 00                 jz      loc_4F09A0
0F 84 EB 00 00 00                 jz      loc_4EFBB8
\end{lstlisting}

Backwards:

\begin{lstlisting}[style=customasmx86]
E8 79 0C FE FF                    call    _function1
E8 F4 16 FF FF                    call    _function2
0F 84 F8 FB FF FF                 jz      loc_8212BC
0F 84 06 FD FF FF                 jz      loc_FF1E7D
\end{lstlisting}

FF byte is also very often occurred in negative displacements like these:

\begin{lstlisting}[style=customasmx86]
8D 85 1E FF FF FF                 lea     eax, [ebp-0E2h]
8D 95 F8 5C FF FF                 lea     edx, [ebp-0A308h]
\end{lstlisting}

So far so good. Now we have to try various 16-byte keys, decrypt executable section and measure how often 00, FF and 8B bytes are occurred.
Let's also keep in sight how PCBC decryption works:

\begin{figure}[H]
\centering
\myincludegraphics{examples/simple_exec_crypto/640px-PCBC_decryption.png}
\caption{Propagating Cipher Block Chaining decryption (image is taken from Wikipedia article)}
\end{figure}

The good news is that we don't really have to decrypt whole piece of data, but only slice by slice, this is exactly how I did in my previous example: \myref{XOR_mask_2}.

Now I'm trying all possible bytes (0..255) for each byte in key and just pick the byte producing maximal amount of 00/FF/8B bytes in a decrypted slice:

\begin{lstlisting}[style=custompy]
#!/usr/bin/env python
import sys, hexdump, array, string, operator

KEY_LEN=16

def chunks(l, n):
    # split n by l-byte chunks
    # https://stackoverflow.com/q/312443
    n = max(1, n)
    return [l[i:i + n] for i in range(0, len(l), n)]

def read_file(fname):
    file=open(fname, mode='rb')
    content=file.read()
    file.close()
    return content

def decrypt_byte (c, key):
    return chr((ord(c)-key) % 256)

def XOR_PCBC_step (IV, buf, k):
    prev=IV
    rt=""
    for c in buf:
	new_c=decrypt_byte(c, k)
        plain=chr(ord(new_c)^ord(prev))
	prev=chr(ord(c)^ord(plain))
	rt=rt+plain
    return rt

each_Nth_byte=[""]*KEY_LEN

content=read_file(sys.argv[1])
# split input by 16-byte chunks:
all_chunks=chunks(content, KEY_LEN)
for c in all_chunks:
    for i in range(KEY_LEN):
        each_Nth_byte[i]=each_Nth_byte[i] + c[i]

# try each byte of key
for N in range(KEY_LEN):
    print "N=", N
    stat={}
    for i in range(256):
        tmp_key=chr(i)
	tmp=XOR_PCBC_step(tmp_key,each_Nth_byte[N], N)
        # count 0, FFs and 8Bs in decrypted buffer:
	important_bytes=tmp.count('\x00')+tmp.count('\xFF')+tmp.count('\x8B')
	stat[i]=important_bytes
    sorted_stat = sorted(stat.iteritems(), key=operator.itemgetter(1), reverse=True)
    print sorted_stat[0]
\end{lstlisting}

(Source code can be downloaded \href{\GitHubBlobMasterURL/examples/simple_exec_crypto/files/decrypt.py}{here}.)

I run it and here is a key for which 00/FF/8B bytes presence in decrypted buffer is maximal:

\begin{lstlisting}
N= 0
(147, 1224)
N= 1
(94, 1327)
N= 2
(252, 1223)
N= 3
(218, 1266)
N= 4
(38, 1209)
N= 5
(192, 1378)
N= 6
(199, 1204)
N= 7
(213, 1332)
N= 8
(225, 1251)
N= 9
(112, 1223)
N= 10
(143, 1177)
N= 11
(108, 1286)
N= 12
(10, 1164)
N= 13
(3, 1271)
N= 14
(128, 1253)
N= 15
(232, 1330)
\end{lstlisting}

Let's write decryption utility with the key we got:

\begin{lstlisting}[style=custompy]
#!/usr/bin/env python
import sys, hexdump, array

def xor_strings(s,t):
    # \verb|https://en.wikipedia.org/wiki/XOR_cipher#Example_implementation|
    """xor two strings together"""
    return "".join(chr(ord(a)^ord(b)) for a,b in zip(s,t))

IV=array.array('B', [147, 94, 252, 218, 38, 192, 199, 213, 225, 112, 143, 108, 10, 3, 128, 232]).tostring()

def chunks(l, n):
    n = max(1, n)
    return [l[i:i + n] for i in range(0, len(l), n)]

def read_file(fname):
    file=open(fname, mode='rb')
    content=file.read()
    file.close()
    return content

def decrypt_byte(i, k):
    return chr ((ord(i)-k) % 256)

def decrypt(buf):
    return "".join(decrypt_byte(buf[i], i) for i in range(16))

fout=open(sys.argv[2], mode='wb')

prev=IV
content=read_file(sys.argv[1])
tmp=chunks(content, 16)
for c in tmp:
    new_c=decrypt(c)
    p=xor_strings (new_c, prev)
    prev=xor_strings(c, p)
    fout.write(p)
fout.close()
\end{lstlisting}

(Source code can be downloaded \href{\GitHubBlobMasterURL/examples/simple_exec_crypto/files/decrypt2.py}{here}.)

Let's check resulting file:

\lstinputlisting{examples/simple_exec_crypto/objdump_result.txt}

Yes, this is seems correctly disassembled piece of x86 code.
The whole decryped file can be downloaded \href{\GitHubBlobMasterURL/examples/simple_exec_crypto/files/decrypted.bin}{here}.

In fact, this is text section from regedit.exe from Windows 7.
But this example is based on a real case I encountered, so just executable is different (and key), algorithm is the same.

\subsection{Other ideas to consider}

What if I would fail with such simple frequency analysis?
There are other ideas on how to measure correctness of decrypted/decompressed x86 code:

\begin{itemize}

\item Many modern compilers aligns functions on 0x10 border.
So the space left before is filled with NOPs (0x90) or other NOP instructions with known opcodes: \myref{sec:npad}.

\item Perhaps, the most frequent pattern in any assembly language is function call:\\
\TT{PUSH chain / CALL / ADD ESP, X}.
This sequence can easily detected and found.
I've even gathered statistics about average number of function arguments: \myref{args_stat}.
(Hence, this is average length of PUSH chain.)

\end{itemize}

Read more about incorrectly/correctly disassembled code: \myref{ISA_detect}.

\EN{\input{patterns/016_empty_redux/main_EN}}%
\FR{\input{patterns/016_empty_redux/main_FR}}

\EN{\input{patterns/016_empty_redux/main_EN}}%
\FR{\input{patterns/016_empty_redux/main_FR}}

\mysection{Handwritten assembly code}

\subsection{ EICAR test file}
\label{subsec:EICAR}

\myindex{MS-DOS}
\myindex{EICAR}
This .COM-file is intended for testing antivirus software, it is possible to run in
in MS-DOS and it prints this string: \q{EICAR-STANDARD-ANTIVIRUS-TEST-FILE!}.
% FIXME1 \myref{} -> about .COM files

Its important property is that it consists entirely of printable 
ASCII-symbols, which, in turn, makes it possible to create it in any text editor:

\begin{lstlisting}
X5O!P%@AP[4\PZX54(P^)7CC)7}$EICAR-STANDARD-ANTIVIRUS-TEST-FILE!$H+H*
\end{lstlisting}

Let's decompile it:

\lstinputlisting[style=customasmx86]{examples/handcoding/EICAR_EN.lst}

We will add comments about the registers and stack after each instruction.

Essentially, all these
instructions are here only to execute this code:

\begin{lstlisting}[style=customasmx86]
B4 09     MOV AH, 9
BA 1C 01  MOV DX, 11Ch
CD 21     INT 21h
CD 20     INT 20h
\end{lstlisting}

\myindex{x86!\Instructions!INT}
\TT{INT 21h} with 9th
function (passed in \TT{AH}) just prints a string, the address of which is passed in \TT{DS:DX}.
By the way, the string has to be terminated
with the '\$' sign.
Apparently, it's inherited from \gls{CP/M} 
and this function was left in DOS for compatibility.
\TT{INT 20h} exits to DOS.

But as we can see, these instruction's
opcodes are not strictly printable.
So the main part of EICAR file is:

\begin{itemize}
\item preparing the register (AH and DX) values that we need;
\item preparing INT 21 and INT 20 opcodes in memory;
\item executing INT 21 and INT 20.
\end{itemize}

\myindex{Shellcode}

By the way, this technique is widely used in shellcode construction, when one have to pass x86 code
in string form.

Here is also a list of all 
x86 instructions which have printable opcodes: \myref{printable_x86_opcodes}.

\EN{\input{patterns/016_empty_redux/main_EN}}%
\FR{\input{patterns/016_empty_redux/main_FR}}

%% TODO translate
\mysection{Breaking simple executable cryptor}

I've got an executable file which is encrypted by relatively simple encryption.
\href{\GitHubBlobMasterURL/examples/simple_exec_crypto/files/cipher.bin}{Here is it} (only executable section is left here).

First, all encryption function does is just adds number of position in buffer to the byte.
Here is how this can be encoded in Python:

\begin{lstlisting}[caption=Python script,style=custompy]
#!/usr/bin/env python
def e(i, k):
    return chr ((ord(i)+k) % 256)

def encrypt(buf):
    return e(buf[0], 0)+ e(buf[1], 1)+ e(buf[2], 2) + e(buf[3], 3)+ e(buf[4], 4)+ e(buf[5], 5)+ e(buf[6], 6)+ e(buf[7], 7)+
           e(buf[8], 8)+ e(buf[9], 9)+ e(buf[10], 10)+ e(buf[11], 11)+ e(buf[12], 12)+ e(buf[13], 13)+ e(buf[14], 14)+ e(buf[15], 15)
\end{lstlisting}

Hence, if you encrypt buffer with 16 zeros, you'll get \emph{0, 1, 2, 3 ... 12, 13, 14, 15}.

\myindex{Propagating Cipher Block Chaining}
Propagating Cipher Block Chaining (PCBC) is also used, here is how it works:

\begin{figure}[H]
\centering
\myincludegraphics{examples/simple_exec_crypto/601px-PCBC_encryption.png}
\caption{Propagating Cipher Block Chaining encryption (image is taken from Wikipedia article)}
\end{figure}

The problem is that it's too boring to recover IV (Initialization Vector) each time.
Brute-force is also not an option, because IV is too long (16 bytes).
Let's see, if it's possible to recover IV for arbitrary encrypted executable file?

Let's try simple frequency analysis.
This is 32-bit x86 executable code, so let's gather statistics about most frequent bytes and opcodes.
I tried huge oracle.exe file from Oracle RDBMS version 11.2 for windows x86 and I've found that the most frequent byte (no surprise) is zero (~10\%).
The next most frequent byte is (again, no surprise) 0xFF (~5\%).
The next is 0x8B (~5\%).

\myindex{x86!\Instructions!MOV}
0x8B is opcode for \INS{MOV}, this is indeed one of the most busy x86 instructions.
Now what about popularity of zero byte?
If compiler needs to encode value bigger than 127, it has to use 32-bit displacement instead of 8-bit one, but large values are very rare,
so it is padded by zeros.
\myindex{x86!\Instructions!LEA}
\myindex{x86!\Instructions!PUSH}
\myindex{x86!\Instructions!CALL}
This is at least in \INS{LEA}, \INS{MOV}, \INS{PUSH}, \INS{CALL}.

For example:

\begin{lstlisting}[style=customasmx86]
8D B0 28 01 00 00                 lea     esi, [eax+128h]
8D BF 40 38 00 00                 lea     edi, [edi+3840h]
\end{lstlisting}

Displacements bigger than 127 are very popular, but they are rarely exceeds 0x10000
(indeed, such large memory buffers/structures are also rare).

Same story with \INS{MOV}, large constants are rare, the most heavily used are 0, 1, 10, 100, $2^n$, and so on.
Compiler has to pad small constants by zeros to represent them as 32-bit values:

\begin{lstlisting}[style=customasmx86]
BF 02 00 00 00                    mov     edi, 2
BF 01 00 00 00                    mov     edi, 1
\end{lstlisting}

Now about 00 and FF bytes combined: jumps (including conditional) and calls can pass execution flow forward or backwards, but very often,
within the limits of the current executable module.
If forward, displacement is not very big and also padded with zeros.
If backwards, displacement is represented as negative value, so padded with FF bytes.
For example, transfer execution flow forward:

\begin{lstlisting}[style=customasmx86]
E8 43 0C 00 00                    call    _function1
E8 5C 00 00 00                    call    _function2
0F 84 F0 0A 00 00                 jz      loc_4F09A0
0F 84 EB 00 00 00                 jz      loc_4EFBB8
\end{lstlisting}

Backwards:

\begin{lstlisting}[style=customasmx86]
E8 79 0C FE FF                    call    _function1
E8 F4 16 FF FF                    call    _function2
0F 84 F8 FB FF FF                 jz      loc_8212BC
0F 84 06 FD FF FF                 jz      loc_FF1E7D
\end{lstlisting}

FF byte is also very often occurred in negative displacements like these:

\begin{lstlisting}[style=customasmx86]
8D 85 1E FF FF FF                 lea     eax, [ebp-0E2h]
8D 95 F8 5C FF FF                 lea     edx, [ebp-0A308h]
\end{lstlisting}

So far so good. Now we have to try various 16-byte keys, decrypt executable section and measure how often 00, FF and 8B bytes are occurred.
Let's also keep in sight how PCBC decryption works:

\begin{figure}[H]
\centering
\myincludegraphics{examples/simple_exec_crypto/640px-PCBC_decryption.png}
\caption{Propagating Cipher Block Chaining decryption (image is taken from Wikipedia article)}
\end{figure}

The good news is that we don't really have to decrypt whole piece of data, but only slice by slice, this is exactly how I did in my previous example: \myref{XOR_mask_2}.

Now I'm trying all possible bytes (0..255) for each byte in key and just pick the byte producing maximal amount of 00/FF/8B bytes in a decrypted slice:

\begin{lstlisting}[style=custompy]
#!/usr/bin/env python
import sys, hexdump, array, string, operator

KEY_LEN=16

def chunks(l, n):
    # split n by l-byte chunks
    # https://stackoverflow.com/q/312443
    n = max(1, n)
    return [l[i:i + n] for i in range(0, len(l), n)]

def read_file(fname):
    file=open(fname, mode='rb')
    content=file.read()
    file.close()
    return content

def decrypt_byte (c, key):
    return chr((ord(c)-key) % 256)

def XOR_PCBC_step (IV, buf, k):
    prev=IV
    rt=""
    for c in buf:
	new_c=decrypt_byte(c, k)
        plain=chr(ord(new_c)^ord(prev))
	prev=chr(ord(c)^ord(plain))
	rt=rt+plain
    return rt

each_Nth_byte=[""]*KEY_LEN

content=read_file(sys.argv[1])
# split input by 16-byte chunks:
all_chunks=chunks(content, KEY_LEN)
for c in all_chunks:
    for i in range(KEY_LEN):
        each_Nth_byte[i]=each_Nth_byte[i] + c[i]

# try each byte of key
for N in range(KEY_LEN):
    print "N=", N
    stat={}
    for i in range(256):
        tmp_key=chr(i)
	tmp=XOR_PCBC_step(tmp_key,each_Nth_byte[N], N)
        # count 0, FFs and 8Bs in decrypted buffer:
	important_bytes=tmp.count('\x00')+tmp.count('\xFF')+tmp.count('\x8B')
	stat[i]=important_bytes
    sorted_stat = sorted(stat.iteritems(), key=operator.itemgetter(1), reverse=True)
    print sorted_stat[0]
\end{lstlisting}

(Source code can be downloaded \href{\GitHubBlobMasterURL/examples/simple_exec_crypto/files/decrypt.py}{here}.)

I run it and here is a key for which 00/FF/8B bytes presence in decrypted buffer is maximal:

\begin{lstlisting}
N= 0
(147, 1224)
N= 1
(94, 1327)
N= 2
(252, 1223)
N= 3
(218, 1266)
N= 4
(38, 1209)
N= 5
(192, 1378)
N= 6
(199, 1204)
N= 7
(213, 1332)
N= 8
(225, 1251)
N= 9
(112, 1223)
N= 10
(143, 1177)
N= 11
(108, 1286)
N= 12
(10, 1164)
N= 13
(3, 1271)
N= 14
(128, 1253)
N= 15
(232, 1330)
\end{lstlisting}

Let's write decryption utility with the key we got:

\begin{lstlisting}[style=custompy]
#!/usr/bin/env python
import sys, hexdump, array

def xor_strings(s,t):
    # \verb|https://en.wikipedia.org/wiki/XOR_cipher#Example_implementation|
    """xor two strings together"""
    return "".join(chr(ord(a)^ord(b)) for a,b in zip(s,t))

IV=array.array('B', [147, 94, 252, 218, 38, 192, 199, 213, 225, 112, 143, 108, 10, 3, 128, 232]).tostring()

def chunks(l, n):
    n = max(1, n)
    return [l[i:i + n] for i in range(0, len(l), n)]

def read_file(fname):
    file=open(fname, mode='rb')
    content=file.read()
    file.close()
    return content

def decrypt_byte(i, k):
    return chr ((ord(i)-k) % 256)

def decrypt(buf):
    return "".join(decrypt_byte(buf[i], i) for i in range(16))

fout=open(sys.argv[2], mode='wb')

prev=IV
content=read_file(sys.argv[1])
tmp=chunks(content, 16)
for c in tmp:
    new_c=decrypt(c)
    p=xor_strings (new_c, prev)
    prev=xor_strings(c, p)
    fout.write(p)
fout.close()
\end{lstlisting}

(Source code can be downloaded \href{\GitHubBlobMasterURL/examples/simple_exec_crypto/files/decrypt2.py}{here}.)

Let's check resulting file:

\lstinputlisting{examples/simple_exec_crypto/objdump_result.txt}

Yes, this is seems correctly disassembled piece of x86 code.
The whole decryped file can be downloaded \href{\GitHubBlobMasterURL/examples/simple_exec_crypto/files/decrypted.bin}{here}.

In fact, this is text section from regedit.exe from Windows 7.
But this example is based on a real case I encountered, so just executable is different (and key), algorithm is the same.

\subsection{Other ideas to consider}

What if I would fail with such simple frequency analysis?
There are other ideas on how to measure correctness of decrypted/decompressed x86 code:

\begin{itemize}

\item Many modern compilers aligns functions on 0x10 border.
So the space left before is filled with NOPs (0x90) or other NOP instructions with known opcodes: \myref{sec:npad}.

\item Perhaps, the most frequent pattern in any assembly language is function call:\\
\TT{PUSH chain / CALL / ADD ESP, X}.
This sequence can easily detected and found.
I've even gathered statistics about average number of function arguments: \myref{args_stat}.
(Hence, this is average length of PUSH chain.)

\end{itemize}

Read more about incorrectly/correctly disassembled code: \myref{ISA_detect}.

%% TODO translate
\mysection{Breaking simple executable cryptor}

I've got an executable file which is encrypted by relatively simple encryption.
\href{\GitHubBlobMasterURL/examples/simple_exec_crypto/files/cipher.bin}{Here is it} (only executable section is left here).

First, all encryption function does is just adds number of position in buffer to the byte.
Here is how this can be encoded in Python:

\begin{lstlisting}[caption=Python script,style=custompy]
#!/usr/bin/env python
def e(i, k):
    return chr ((ord(i)+k) % 256)

def encrypt(buf):
    return e(buf[0], 0)+ e(buf[1], 1)+ e(buf[2], 2) + e(buf[3], 3)+ e(buf[4], 4)+ e(buf[5], 5)+ e(buf[6], 6)+ e(buf[7], 7)+
           e(buf[8], 8)+ e(buf[9], 9)+ e(buf[10], 10)+ e(buf[11], 11)+ e(buf[12], 12)+ e(buf[13], 13)+ e(buf[14], 14)+ e(buf[15], 15)
\end{lstlisting}

Hence, if you encrypt buffer with 16 zeros, you'll get \emph{0, 1, 2, 3 ... 12, 13, 14, 15}.

\myindex{Propagating Cipher Block Chaining}
Propagating Cipher Block Chaining (PCBC) is also used, here is how it works:

\begin{figure}[H]
\centering
\myincludegraphics{examples/simple_exec_crypto/601px-PCBC_encryption.png}
\caption{Propagating Cipher Block Chaining encryption (image is taken from Wikipedia article)}
\end{figure}

The problem is that it's too boring to recover IV (Initialization Vector) each time.
Brute-force is also not an option, because IV is too long (16 bytes).
Let's see, if it's possible to recover IV for arbitrary encrypted executable file?

Let's try simple frequency analysis.
This is 32-bit x86 executable code, so let's gather statistics about most frequent bytes and opcodes.
I tried huge oracle.exe file from Oracle RDBMS version 11.2 for windows x86 and I've found that the most frequent byte (no surprise) is zero (~10\%).
The next most frequent byte is (again, no surprise) 0xFF (~5\%).
The next is 0x8B (~5\%).

\myindex{x86!\Instructions!MOV}
0x8B is opcode for \INS{MOV}, this is indeed one of the most busy x86 instructions.
Now what about popularity of zero byte?
If compiler needs to encode value bigger than 127, it has to use 32-bit displacement instead of 8-bit one, but large values are very rare,
so it is padded by zeros.
\myindex{x86!\Instructions!LEA}
\myindex{x86!\Instructions!PUSH}
\myindex{x86!\Instructions!CALL}
This is at least in \INS{LEA}, \INS{MOV}, \INS{PUSH}, \INS{CALL}.

For example:

\begin{lstlisting}[style=customasmx86]
8D B0 28 01 00 00                 lea     esi, [eax+128h]
8D BF 40 38 00 00                 lea     edi, [edi+3840h]
\end{lstlisting}

Displacements bigger than 127 are very popular, but they are rarely exceeds 0x10000
(indeed, such large memory buffers/structures are also rare).

Same story with \INS{MOV}, large constants are rare, the most heavily used are 0, 1, 10, 100, $2^n$, and so on.
Compiler has to pad small constants by zeros to represent them as 32-bit values:

\begin{lstlisting}[style=customasmx86]
BF 02 00 00 00                    mov     edi, 2
BF 01 00 00 00                    mov     edi, 1
\end{lstlisting}

Now about 00 and FF bytes combined: jumps (including conditional) and calls can pass execution flow forward or backwards, but very often,
within the limits of the current executable module.
If forward, displacement is not very big and also padded with zeros.
If backwards, displacement is represented as negative value, so padded with FF bytes.
For example, transfer execution flow forward:

\begin{lstlisting}[style=customasmx86]
E8 43 0C 00 00                    call    _function1
E8 5C 00 00 00                    call    _function2
0F 84 F0 0A 00 00                 jz      loc_4F09A0
0F 84 EB 00 00 00                 jz      loc_4EFBB8
\end{lstlisting}

Backwards:

\begin{lstlisting}[style=customasmx86]
E8 79 0C FE FF                    call    _function1
E8 F4 16 FF FF                    call    _function2
0F 84 F8 FB FF FF                 jz      loc_8212BC
0F 84 06 FD FF FF                 jz      loc_FF1E7D
\end{lstlisting}

FF byte is also very often occurred in negative displacements like these:

\begin{lstlisting}[style=customasmx86]
8D 85 1E FF FF FF                 lea     eax, [ebp-0E2h]
8D 95 F8 5C FF FF                 lea     edx, [ebp-0A308h]
\end{lstlisting}

So far so good. Now we have to try various 16-byte keys, decrypt executable section and measure how often 00, FF and 8B bytes are occurred.
Let's also keep in sight how PCBC decryption works:

\begin{figure}[H]
\centering
\myincludegraphics{examples/simple_exec_crypto/640px-PCBC_decryption.png}
\caption{Propagating Cipher Block Chaining decryption (image is taken from Wikipedia article)}
\end{figure}

The good news is that we don't really have to decrypt whole piece of data, but only slice by slice, this is exactly how I did in my previous example: \myref{XOR_mask_2}.

Now I'm trying all possible bytes (0..255) for each byte in key and just pick the byte producing maximal amount of 00/FF/8B bytes in a decrypted slice:

\begin{lstlisting}[style=custompy]
#!/usr/bin/env python
import sys, hexdump, array, string, operator

KEY_LEN=16

def chunks(l, n):
    # split n by l-byte chunks
    # https://stackoverflow.com/q/312443
    n = max(1, n)
    return [l[i:i + n] for i in range(0, len(l), n)]

def read_file(fname):
    file=open(fname, mode='rb')
    content=file.read()
    file.close()
    return content

def decrypt_byte (c, key):
    return chr((ord(c)-key) % 256)

def XOR_PCBC_step (IV, buf, k):
    prev=IV
    rt=""
    for c in buf:
	new_c=decrypt_byte(c, k)
        plain=chr(ord(new_c)^ord(prev))
	prev=chr(ord(c)^ord(plain))
	rt=rt+plain
    return rt

each_Nth_byte=[""]*KEY_LEN

content=read_file(sys.argv[1])
# split input by 16-byte chunks:
all_chunks=chunks(content, KEY_LEN)
for c in all_chunks:
    for i in range(KEY_LEN):
        each_Nth_byte[i]=each_Nth_byte[i] + c[i]

# try each byte of key
for N in range(KEY_LEN):
    print "N=", N
    stat={}
    for i in range(256):
        tmp_key=chr(i)
	tmp=XOR_PCBC_step(tmp_key,each_Nth_byte[N], N)
        # count 0, FFs and 8Bs in decrypted buffer:
	important_bytes=tmp.count('\x00')+tmp.count('\xFF')+tmp.count('\x8B')
	stat[i]=important_bytes
    sorted_stat = sorted(stat.iteritems(), key=operator.itemgetter(1), reverse=True)
    print sorted_stat[0]
\end{lstlisting}

(Source code can be downloaded \href{\GitHubBlobMasterURL/examples/simple_exec_crypto/files/decrypt.py}{here}.)

I run it and here is a key for which 00/FF/8B bytes presence in decrypted buffer is maximal:

\begin{lstlisting}
N= 0
(147, 1224)
N= 1
(94, 1327)
N= 2
(252, 1223)
N= 3
(218, 1266)
N= 4
(38, 1209)
N= 5
(192, 1378)
N= 6
(199, 1204)
N= 7
(213, 1332)
N= 8
(225, 1251)
N= 9
(112, 1223)
N= 10
(143, 1177)
N= 11
(108, 1286)
N= 12
(10, 1164)
N= 13
(3, 1271)
N= 14
(128, 1253)
N= 15
(232, 1330)
\end{lstlisting}

Let's write decryption utility with the key we got:

\begin{lstlisting}[style=custompy]
#!/usr/bin/env python
import sys, hexdump, array

def xor_strings(s,t):
    # \verb|https://en.wikipedia.org/wiki/XOR_cipher#Example_implementation|
    """xor two strings together"""
    return "".join(chr(ord(a)^ord(b)) for a,b in zip(s,t))

IV=array.array('B', [147, 94, 252, 218, 38, 192, 199, 213, 225, 112, 143, 108, 10, 3, 128, 232]).tostring()

def chunks(l, n):
    n = max(1, n)
    return [l[i:i + n] for i in range(0, len(l), n)]

def read_file(fname):
    file=open(fname, mode='rb')
    content=file.read()
    file.close()
    return content

def decrypt_byte(i, k):
    return chr ((ord(i)-k) % 256)

def decrypt(buf):
    return "".join(decrypt_byte(buf[i], i) for i in range(16))

fout=open(sys.argv[2], mode='wb')

prev=IV
content=read_file(sys.argv[1])
tmp=chunks(content, 16)
for c in tmp:
    new_c=decrypt(c)
    p=xor_strings (new_c, prev)
    prev=xor_strings(c, p)
    fout.write(p)
fout.close()
\end{lstlisting}

(Source code can be downloaded \href{\GitHubBlobMasterURL/examples/simple_exec_crypto/files/decrypt2.py}{here}.)

Let's check resulting file:

\lstinputlisting{examples/simple_exec_crypto/objdump_result.txt}

Yes, this is seems correctly disassembled piece of x86 code.
The whole decryped file can be downloaded \href{\GitHubBlobMasterURL/examples/simple_exec_crypto/files/decrypted.bin}{here}.

In fact, this is text section from regedit.exe from Windows 7.
But this example is based on a real case I encountered, so just executable is different (and key), algorithm is the same.

\subsection{Other ideas to consider}

What if I would fail with such simple frequency analysis?
There are other ideas on how to measure correctness of decrypted/decompressed x86 code:

\begin{itemize}

\item Many modern compilers aligns functions on 0x10 border.
So the space left before is filled with NOPs (0x90) or other NOP instructions with known opcodes: \myref{sec:npad}.

\item Perhaps, the most frequent pattern in any assembly language is function call:\\
\TT{PUSH chain / CALL / ADD ESP, X}.
This sequence can easily detected and found.
I've even gathered statistics about average number of function arguments: \myref{args_stat}.
(Hence, this is average length of PUSH chain.)

\end{itemize}

Read more about incorrectly/correctly disassembled code: \myref{ISA_detect}.

\mysection{A nasty bug in MSVCRT.DLL}
\myindex{MinGW}
\myindex{MSVCRT.DLL}

This is the bug that costed me several hours of debugging.

In 2013 I was using MinGW, my C project seems to be very unstable and I saw the
``Invalid parameter passed to C runtime function.'' error message in debugger.

The error message was also visible using Sysinternals DebugView.
And my project has no such error messages or strings.
So I started to search it in the whole Windows and found in MSVCRT.DLL file.
(Needless to say I was using Windows 7.)

So here it is, the error message in MSVCRT.DLL file supplied with Windows 7:

\begin{lstlisting}[style=customasmx86]
.text:6FFB69D0 OutputString    db 'Invalid parameter passed to C runtime function.',0Ah,0
.text:6FFB69D0                                         ; DATA XREF: \verb|sub_6FFB6930+83|
\end{lstlisting}

Where it is referenced?

\begin{lstlisting}[style=customasmx86,label=msvcrt_output]
.text:6FFB6930 sub_6FFB6930    proc near               ; \verb|CODE XREF: _wfindfirst64+203FC|
.text:6FFB6930                                         ; \verb|sub_6FF62563+319AD|
.text:6FFB6930
.text:6FFB6930 var_2D0         = dword ptr -2D0h
.text:6FFB6930 var_244         = word ptr -244h
.text:6FFB6930 var_240         = word ptr -240h
.text:6FFB6930 var_23C         = word ptr -23Ch
.text:6FFB6930 var_238         = word ptr -238h
.text:6FFB6930 var_234         = dword ptr -234h
.text:6FFB6930 var_230         = dword ptr -230h
.text:6FFB6930 var_22C         = dword ptr -22Ch
.text:6FFB6930 var_228         = dword ptr -228h
.text:6FFB6930 var_224         = dword ptr -224h
.text:6FFB6930 var_220         = dword ptr -220h
.text:6FFB6930 var_21C         = dword ptr -21Ch
.text:6FFB6930 var_218         = dword ptr -218h
.text:6FFB6930 var_214         = word ptr -214h
.text:6FFB6930 var_210         = dword ptr -210h
.text:6FFB6930 var_20C         = dword ptr -20Ch
.text:6FFB6930 var_208         = word ptr -208h
.text:6FFB6930 var_4           = dword ptr -4
.text:6FFB6930
.text:6FFB6930                 mov     edi, edi
.text:6FFB6932                 push    ebp
.text:6FFB6933                 mov     ebp, esp
.text:6FFB6935                 sub     esp, 2D0h
.text:6FFB693B                 mov     eax, ___security_cookie
.text:6FFB6940                 xor     eax, ebp
.text:6FFB6942                 mov     [ebp+var_4], eax
.text:6FFB6945                 mov     [ebp+var_220], eax
.text:6FFB694B                 mov     [ebp+var_224], ecx
.text:6FFB6951                 mov     [ebp+var_228], edx
.text:6FFB6957                 mov     [ebp+var_22C], ebx
.text:6FFB695D                 mov     [ebp+var_230], esi
.text:6FFB6963                 mov     [ebp+var_234], edi
.text:6FFB6969                 mov     [ebp+var_208], ss
.text:6FFB696F                 mov     [ebp+var_214], cs
.text:6FFB6975                 mov     [ebp+var_238], ds
.text:6FFB697B                 mov     [ebp+var_23C], es
.text:6FFB6981                 mov     [ebp+var_240], fs
.text:6FFB6987                 mov     [ebp+var_244], gs
.text:6FFB698D                 pushf
.text:6FFB698E                 pop     [ebp+var_210]
.text:6FFB6994                 mov     eax, [ebp+4]
.text:6FFB6997                 mov     [ebp+var_218], eax
.text:6FFB699D                 lea     eax, [ebp+4]
.text:6FFB69A0                 mov     [ebp+var_2D0], 10001h
.text:6FFB69AA                 mov     [ebp+var_20C], eax
.text:6FFB69B0                 mov     eax, [eax-4]
.text:6FFB69B3                 push    offset OutputString ; "Invalid parameter passed to C runtime f"...
.text:6FFB69B8                 mov     [ebp+var_21C], eax
.text:6FFB69BE                 call    ds:OutputDebugStringA
.text:6FFB69C4                 mov     ecx, [ebp+var_4]
.text:6FFB69C7                 xor     ecx, ebp
.text:6FFB69C9                 call    @__security_check_cookie@4 ; \verb|__security_check_cookie(x)|
.text:6FFB69CE                 leave
.text:6FFB69CF                 retn
.text:6FFB69CF sub_6FFB6930    endp
\end{lstlisting}

The string it reported into debugger or DebugView utility using the standard \verb|OutputDebugStringA()| function.
How the \verb|sub_6FFB6930| can be called?
IDA shows at least 280 references.

Using my \tracer, I set a breakpoint at \verb|sub_6FFB6930| to see, when it's called in my case:

\begin{lstlisting}
tracer.exe -l:1.exe bpf=msvcrt.dll!0x6FFB6930 -s

...

PID=3560|New process 1.exe
(0) msvcrt.dll!0x6ffb6930() (called from msvcrt.dll!_ftol2_sse_excpt+0x1b467 (0x759ed222))
Call stack:
return address=0x401010 (1.exe!.text+0x10), arguments in stack: 0x12ff14, 0x401010, 0x403010("asd"), 0x0, 0x12ff88, 0x4010f8
return address=0x4010f8 (1.exe!OEP+0xe3), arguments in stack: 0x12ff88, 0x4010f8, 0x1, 0x2e0ea8, 0x2e1640, 0x403000
return address=0x75b6ef3c (KERNEL32.dll!BaseThreadInitThunk+0x12), arguments in stack: 0x12ff94, 0x75b6ef3c, 0x7ffdf000, 0x12ffd4, 0x77523688, 0x7ffdf000
return address=0x77523688 (ntdll.dll!RtlInitializeExceptionChain+0xef), arguments in stack: 0x12ffd4, 0x77523688, 0x7ffdf000, 0x74117ec7, 0x0, 0x0
return address=0x7752365b (ntdll.dll!RtlInitializeExceptionChain+0xc2), arguments in stack: 0x12ffec, 0x7752365b, 0x401015, 0x7ffdf000, 0x0, 0x0
(0) msvcrt.dll!0x6ffb6930() -> 0x12f94c
PID=3560|Process 1.exe exited. ExitCode=2147483647 (0x7fffffff)
\end{lstlisting}

I found that my code was calling stricmp() function with NULL as one argument.
In fact, I made up this example when writing this:

\begin{lstlisting}[style=customc]
#include <stdio.h>
#include <string.h>

int main()
{
	stricmp ("asd", NULL);
};
\end{lstlisting}

If this piece of code is compiled using old MinGW or old MSVC 6.0, it is linked against MSVCRT.DLL file.
Which, as of Windows 7, silently sends the ``Invalid parameter passed to C runtime function.''
error message to the debugger and then does nothing!

Let's see how stricmp() is implemented in MSVCRT.DLL:

\begin{lstlisting}[style=customasmx86]
.text:6FF5DB38 ; Exported entry 855. \_strcmpi
.text:6FF5DB38 ; Exported entry 863. \_stricmp
.text:6FF5DB38
.text:6FF5DB38 ; =============== S U B R O U T I N E =======================================
.text:6FF5DB38
.text:6FF5DB38 ; Attributes: bp-based frame
.text:6FF5DB38
.text:6FF5DB38 ; int \_\_cdecl strcmpi(const char *, const char *)
.text:6FF5DB38                 public _strcmpi
.text:6FF5DB38 _strcmpi        proc near               ; CODE XREF: LocaleEnumProc-2B
.text:6FF5DB38                                         ; LocaleEnumProc+5E
.text:6FF5DB38
.text:6FF5DB38 arg_0           = dword ptr  8
.text:6FF5DB38 arg_4           = dword ptr  0Ch
.text:6FF5DB38
.text:6FF5DB38 ; FUNCTION CHUNK AT .text:6FF68CFD SIZE 00000012 BYTES
.text:6FF5DB38 ; FUNCTION CHUNK AT .text:6FF9D20D SIZE 00000022 BYTES
.text:6FF5DB38
.text:6FF5DB38                 mov     edi, edi        ; \_strcmpi
.text:6FF5DB3A                 push    ebp
.text:6FF5DB3B                 mov     ebp, esp
.text:6FF5DB3D                 push    esi
.text:6FF5DB3E                 xor     esi, esi
.text:6FF5DB40                 cmp     dword_6FFF0000, esi
.text:6FF5DB46                 jnz     loc_6FF68CFD
.text:6FF5DB4C                 cmp     [ebp+arg_0], esi  ; is arg\_0==NULL?
.text:6FF5DB4F                 jz      loc_6FF9D20D
.text:6FF5DB55                 cmp     [ebp+arg_4], esi  ; is arg\_0==NULL?
.text:6FF5DB58                 jz      loc_6FF9D20D
.text:6FF5DB5E                 pop     esi
.text:6FF5DB5F                 pop     ebp
.text:6FF5DB5F _strcmpi        endp ; sp-analysis failed
\end{lstlisting}

Actual strings comparison here:

\begin{lstlisting}[style=customasmx86]
.text:6FF5DB60 sub_6FF5DB60    proc near               ; CODE XREF: \verb|_stricmp_l+16C7F|
.text:6FF5DB60                                         ; \verb|sub_6FFD19CD+229|
.text:6FF5DB60
.text:6FF5DB60 arg_0           = dword ptr  8
.text:6FF5DB60 arg_4           = dword ptr  0Ch
.text:6FF5DB60
.text:6FF5DB60                 push    ebp
.text:6FF5DB61                 mov     ebp, esp
.text:6FF5DB63                 push    edi
.text:6FF5DB64                 push    esi
.text:6FF5DB65                 push    ebx
.text:6FF5DB66                 mov     esi, [ebp+arg_4]
.text:6FF5DB69                 mov     edi, [ebp+arg_0]
.text:6FF5DB6C                 mov     al, 0FFh
.text:6FF5DB6E                 mov     edi, edi
.text:6FF5DB70
.text:6FF5DB70 loc_6FF5DB70:                           ; CODE XREF: \verb|sub_6FF5DB60+20|
.text:6FF5DB70                                         ; \verb|sub_6FF5DB60+40|
.text:6FF5DB70                 or      al, al
.text:6FF5DB72                 jz      short loc_6FF5DBA6
.text:6FF5DB74                 mov     al, [esi]
.text:6FF5DB76                 add     esi, 1
.text:6FF5DB79                 mov     ah, [edi]
.text:6FF5DB7B                 add     edi, 1
.text:6FF5DB7E                 cmp     ah, al
.text:6FF5DB80                 jz      short loc_6FF5DB70
.text:6FF5DB82                 sub     al, 41h
.text:6FF5DB84                 cmp     al, 1Ah
.text:6FF5DB86                 sbb     cl, cl
.text:6FF5DB88                 and     cl, 20h
.text:6FF5DB8B                 add     al, cl
.text:6FF5DB8D                 add     al, 41h
.text:6FF5DB8F                 xchg    ah, al
.text:6FF5DB91                 sub     al, 41h
.text:6FF5DB93                 cmp     al, 1Ah
.text:6FF5DB95                 sbb     cl, cl
.text:6FF5DB97                 and     cl, 20h
.text:6FF5DB9A                 add     al, cl
.text:6FF5DB9C                 add     al, 41h
.text:6FF5DB9E                 cmp     al, ah
.text:6FF5DBA0                 jz      short loc_6FF5DB70
.text:6FF5DBA2                 sbb     al, al
.text:6FF5DBA4                 sbb     al, 0FFh
.text:6FF5DBA6
.text:6FF5DBA6 loc_6FF5DBA6:                           ; CODE XREF: \verb|sub_6FF5DB60+12|
.text:6FF5DBA6                 movsx   eax, al
.text:6FF5DBA9                 pop     ebx
.text:6FF5DBAA                 pop     esi
.text:6FF5DBAB                 pop     edi
.text:6FF5DBAC                 leave
.text:6FF5DBAD                 retn
.text:6FF5DBAD sub_6FF5DB60    endp

.text:6FF68D0C loc_6FF68D0C:                           ; CODE XREF: \verb|_strcmpi+3F6F2|
.text:6FF68D0C                 pop     esi
.text:6FF68D0D                 pop     ebp
.text:6FF68D0E                 retn

.text:6FF9D20D loc_6FF9D20D:                           ; CODE XREF: \verb|_strcmpi+17|
.text:6FF9D20D                                         ; \verb|_strcmpi+20|
.text:6FF9D20D                 call    near ptr _errno
.text:6FF9D212                 push    esi
.text:6FF9D213                 push    esi
.text:6FF9D214                 push    esi
.text:6FF9D215                 push    esi
.text:6FF9D216                 push    esi
.text:6FF9D217                 mov     dword ptr [eax], 16h
.text:6FF9D21D                 call    _invalid_parameter
.text:6FF9D222                 add     esp, 14h
.text:6FF9D225                 mov     eax, 7FFFFFFFh
.text:6FF9D22A                 jmp     loc_6FF68D0C
\end{lstlisting}

Now the \verb|invalid_parameter()| function:

\begin{lstlisting}[style=customasmx86]
.text:6FFB6A06                 public _invalid_parameter
.text:6FFB6A06 _invalid_parameter proc near            ; CODE XREF: \verb|sub_6FF5B494:loc_6FF5B618|
.text:6FFB6A06                                         ; \verb|sub_6FF5CCFD:loc_6FF5C8A2|
.text:6FFB6A06                 mov     edi, edi
.text:6FFB6A08                 push    ebp
.text:6FFB6A09                 mov     ebp, esp
.text:6FFB6A0B                 pop     ebp
.text:6FFB6A0C                 jmp     sub_6FFB6930
.text:6FFB6A0C _invalid_parameter endp
\end{lstlisting}

The \verb|invalid_parameter()| function eventually calls the function with \verb|OutputDebugStringA()|:
\myref{msvcrt_output}.

You see, the stricmp() code is like:

\begin{lstlisting}[style=customc]
int stricmp(const char *s1, const char *s2, size_t len)
{
	if (s1==NULL || s2==NULL)
	{
		// print error message AND exit:
		return 0x7FFFFFFFh;
	};
	// do comparison
};
\end{lstlisting}

How come this error is rare? Because newer MSVC versions links against MSVCR120.DLL file, etc (where 120 is version number).

Let's peek inside the newer MSVCR120.DLL from Windows 7:

\begin{lstlisting}[style=customasmx86]
.text:1002A0D4                 public _stricmp_l
.text:1002A0D4 _stricmp_l      proc near               ; CODE XREF: \verb|_stricmp+18|
.text:1002A0D4                                         ; \verb|_mbsicmp_l+47|
.text:1002A0D4                                         ; DATA XREF: ...
.text:1002A0D4
.text:1002A0D4 var_10          = dword ptr -10h
.text:1002A0D4 var_8           = dword ptr -8
.text:1002A0D4 var_4           = byte ptr -4
.text:1002A0D4 arg_0           = dword ptr  8
.text:1002A0D4 arg_4           = dword ptr  0Ch
.text:1002A0D4 arg_8           = dword ptr  10h
.text:1002A0D4
.text:1002A0D4 ; FUNCTION CHUNK AT .text:1005AA7B SIZE 0000002A BYTES
.text:1002A0D4
.text:1002A0D4                 push    ebp
.text:1002A0D5                 mov     ebp, esp
.text:1002A0D7                 sub     esp, 10h
.text:1002A0DA                 lea     ecx, [ebp+var_10]
.text:1002A0DD                 push    ebx
.text:1002A0DE                 push    esi
.text:1002A0DF                 push    edi
.text:1002A0E0                 push    [ebp+arg_8]
.text:1002A0E3                 call    sub_1000F764
.text:1002A0E8                 mov     edi, [ebp+arg_0] ; arg==NULL?
.text:1002A0EB                 test    edi, edi
.text:1002A0ED                 jz      loc_1005AA7B
.text:1002A0F3                 mov     ebx, [ebp+arg_4] ; arg==NULL?
.text:1002A0F6                 test    ebx, ebx
.text:1002A0F8                 jz      loc_1005AA7B
.text:1002A0FE                 mov     eax, [ebp+var_10]
.text:1002A101                 cmp     dword ptr [eax+0A8h], 0
.text:1002A108                 jz      loc_1005AA95
.text:1002A10E                 sub     edi, ebx

...

.text:1005AA7B loc_1005AA7B:                           ; CODE XREF: \verb|_stricmp_l+19|
.text:1005AA7B                                         ; \verb|_stricmp_l+24|
.text:1005AA7B                 call    _errno
.text:1005AA80                 mov     dword ptr [eax], 16h
.text:1005AA86                 call    _invalid_parameter_noinfo
.text:1005AA8B                 mov     esi, 7FFFFFFFh
.text:1005AA90                 jmp     loc_1002A13B

...

.text:100A4670 _invalid_parameter_noinfo proc near     ; CODE XREF: \verb|sub_10013BEC-10F|
.text:100A4670                                         ; \verb|sub_10016C0F-10F|
.text:100A4670                 xor     eax, eax
.text:100A4672                 push    eax
.text:100A4673                 push    eax
.text:100A4674                 push    eax
.text:100A4675                 push    eax
.text:100A4676                 push    eax
.text:100A4677                 call    _invalid_parameter
.text:100A467C                 add     esp, 14h
.text:100A467F                 retn
.text:100A467F _invalid_parameter_noinfo endp

...

.text:100A4645 _invalid_parameter proc near            ; CODE XREF: \_invalid\_parameter(ushort const *,ushort const *,ushort const *,uint,uint)
.text:100A4645                                         ; \verb|_invalid_parameter_noinfo+7|
.text:100A4645
.text:100A4645 arg_0           = dword ptr  8
.text:100A4645 arg_4           = dword ptr  0Ch
.text:100A4645 arg_8           = dword ptr  10h
.text:100A4645 arg_C           = dword ptr  14h
.text:100A4645 arg_10          = dword ptr  18h
.text:100A4645
.text:100A4645                 push    ebp
.text:100A4646                 mov     ebp, esp
.text:100A4648                 push    dword_100E0ED8  ; Ptr
.text:100A464E                 call    ds:DecodePointer
.text:100A4654                 test    eax, eax
.text:100A4656                 jz      short loc_100A465B
.text:100A4658                 pop     ebp
.text:100A4659                 jmp     eax
.text:100A465B ; ---------------------------------------------------------------------------
.text:100A465B
.text:100A465B loc_100A465B:                           ; CODE XREF: \verb|_invalid_parameter+11|
.text:100A465B                 push    [ebp+arg_10]
.text:100A465E                 push    [ebp+arg_C]
.text:100A4661                 push    [ebp+arg_8]
.text:100A4664                 push    [ebp+arg_4]
.text:100A4667                 push    [ebp+arg_0]
.text:100A466A                 call    _invoke_watson
.text:100A466F                 int     3               ; Trap to Debugger
.text:100A466F _invalid_parameter endp

.text:100A469B _invoke_watson  proc near               ; CODE XREF: \verb|sub_1002CDB0+27068|
.text:100A469B                                         ; \verb|sub_10029704+2A792|
.text:100A469B                 push    17h             ; ProcessorFeature
.text:100A469D                 call    IsProcessorFeaturePresent
.text:100A46A2                 test    eax, eax
.text:100A46A4                 jz      short loc_100A46AB
.text:100A46A6                 push    5
.text:100A46A8                 pop     ecx
.text:100A46A9                 int     29h             ; Win8: RtlFailFast(ecx)
.text:100A46AB ; ---------------------------------------------------------------------------
.text:100A46AB
.text:100A46AB loc_100A46AB:                           ; CODE XREF: \verb|_invoke_watson+9|
.text:100A46AB                 push    esi
.text:100A46AC                 push    1
.text:100A46AE                 mov     esi, 0C0000417h
.text:100A46B3                 push    esi
.text:100A46B4                 push    2
.text:100A46B6                 call    sub_100A4519
.text:100A46BB                 push    esi             ; uExitCode
.text:100A46BC                 call    __crtTerminateProcess
.text:100A46C1                 add     esp, 10h
.text:100A46C4                 pop     esi
.text:100A46C5                 retn
.text:100A46C5 _invoke_watson  endp
\end{lstlisting}

Now the \verb|invalid_parameter()| function is rewritten in newer MSVCR*.DLL version,
it shows the message box, if you want to kill the process
or call debugger.
Of course, this is much better than silently return.
Perhaps, Microsoft forgot to fix MSVCRT.DLL since then.

But how it was working in the era of Windows XP? It wasn't: MSVCRT.DLL from Windows XP doesn't check arguments against NULL.
So under Windows XP my \verb|stricmp ("asd", NULL)| code will crash, and this is good.

My hypothesis: Microsoft upgraded MSVCR*.DLL files (including MSVCRT.DLL) for Windows 7 by adding sanitizing checks everywhere.
However, since MSVCRT.DLL wasn't used much since MSVS .NET (year 2002), it wasn't properly tested and the bug left here.
But compilers like MinGW can still use this DLL.

What would I do without my reverse engineering skills?

The MSVCRT.DLL from Windows 8.1 has the same bug.



\mysection{Other examples}

An example about Z3 and manual decompilation was here.
It is moved there:
\url{https://yurichev.com/writings/SAT_SMT_by_example.pdf}.

}
\RU{\chapter{Примеры из практики}

\input{examples/Knuth_RU}

% sections here
\subsection{Простое шифрование используя XOR-маску}
\label{XOR_mask_1}

Я нашел одну старую игру в стиле interactive fiction в архиве \emph{if-archive}\footnote{\url{http://www.ifarchive.org/}}:

\begin{lstlisting}
The New Castle v3.5 - Text/Adventure Game
in the style of the original Infocom (tm)
type games, Zork, Collosal Cave (Adventure),
etc.  Can you solve the mystery of the
abandoned castle?
Shareware from Software Customization.
Software Customization [ASP] Version 3.5 Feb. 2000
\end{lstlisting}

Можно скачать здесь: \url{\GitHubBlobMasterURL/ff/XOR/mask_1/files/newcastle.tgz}.

Там внутри есть файл (с названием \emph{castle.dbf}), который явно зашифрован, но не настоящим криптоалгоритмом,
и оне сжат, это что-то куда проще.
Я бы даже не стал измерять уровень энтропии (\myref{entropy}) этого файла, потому что я итак уверен, что он низкий.
Вот как он выглядит в Midnight Commander:

\begin{figure}[H]
\centering
\myincludegraphics{ff/XOR/mask_1/mc_encrypted.png}
\caption{Зашифрованный файл в Midnight Commander}
\end{figure}

Зашифрованный файл можно скачать здесь:
\url{\GitHubBlobMasterURL/ff/XOR/mask_1/files/castle.dbf.bz2}.

Можно ли расшифровать его без доступа к программе, используя просто этот файл?

Тут явно просматривается повторяющаяся строка. 
Если использовалось простое шифрование с XOR-маской, такие повторяющиеся строки это явное свидетельство,
потому что, вероятно, тут были длинные лакуны с нулевыми байтами, которые, в свою очередь, присутствуют
во мноигих исполняемых файлах, и в остальных бинарных файлах.

\myindex{UNIX!xxd}
Вот дам начала этого файла используя утилиту \emph{xxd} из UNIX:

\lstinputlisting{ff/XOR/mask_1/xxd_result.txt}

Давайте держаться за повторяющуюся строку \TT{iubgv}.
Глядя на этот дамп, мы можем легко увидеть, что период повторений этой строки это 0x51 или 81.
Вероятно, 81 это длина блока?
Длина файла 1658961, и она может быть поделена на 81 без остатка (и тогда там 20481 блоков).

Теперь я буду использовать Mathematica для анализа, есть ли тут повторяющиеся 81-байтные блоки в файле?
Я разделю входной файл на 81-байтные блоки и затем использую ф-цию
\emph{Tally[]}\footnote{\url{https://reference.wolfram.com/language/ref/Tally.html}}
которая просто считает, сколько раз каждый элемент встретился во входном списке.
Вывод Tally не отсортирован, так что я также добавлю ф-цию \emph{Sort[]} для сортировки его по кол-ву вхождений
в нисходящем порядке.

\begin{lstlisting}[style=custommath]
input = BinaryReadList["/home/dennis/.../castle.dbf"];

blocks = Partition[input, 81];

stat = Sort[Tally[blocks], #1[[2]] > #2[[2]] &]
\end{lstlisting}

И вот вывод:

\begin{lstlisting}[style=custommath]
{{{80, 103, 2, 116, 113, 102, 118, 25, 99, 8, 19, 23, 116, 125, 107, 
   25, 99, 109, 114, 102, 14, 121, 115, 31, 9, 117, 113, 111, 5, 4, 
   127, 28, 122, 101, 8, 110, 14, 18, 124, 106, 16, 20, 104, 119, 8, 
   109, 26, 106, 9, 97, 13, 99, 15, 119, 20, 105, 117, 98, 103, 118, 
   1, 126, 29, 97, 122, 17, 15, 114, 110, 3, 5, 125, 125, 99, 126, 
   119, 102, 30, 122, 2, 117}, 1739}, 
{{80, 100, 2, 116, 113, 102, 118, 25, 99, 8, 19, 23, 116, 
   125, 107, 25, 99, 109, 114, 102, 14, 121, 115, 31, 9, 117, 113, 
   111, 5, 4, 127, 28, 122, 101, 8, 110, 14, 18, 124, 106, 16, 20, 
   104, 119, 8, 109, 26, 106, 9, 97, 13, 99, 15, 119, 20, 105, 117, 
   98, 103, 118, 1, 126, 29, 97, 122, 17, 15, 114, 110, 3, 5, 125, 
   125, 99, 126, 119, 102, 30, 122, 2, 117}, 1422}, 
{{80, 101, 2, 116, 113, 102, 118, 25, 99, 8, 19, 23, 116, 
   125, 107, 25, 99, 109, 114, 102, 14, 121, 115, 31, 9, 117, 113, 
   111, 5, 4, 127, 28, 122, 101, 8, 110, 14, 18, 124, 106, 16, 20, 
   104, 119, 8, 109, 26, 106, 9, 97, 13, 99, 15, 119, 20, 105, 117, 
   98, 103, 118, 1, 126, 29, 97, 122, 17, 15, 114, 110, 3, 5, 125, 
   125, 99, 126, 119, 102, 30, 122, 2, 117}, 1012},
{{80, 120, 2, 116, 113, 102, 118, 25, 99, 8, 19, 23, 116, 
   125, 107, 25, 99, 109, 114, 102, 14, 121, 115, 31, 9, 117, 113, 
   111, 5, 4, 127, 28, 122, 101, 8, 110, 14, 18, 124, 106, 16, 20, 
   104, 119, 8, 109, 26, 106, 9, 97, 13, 99, 15, 119, 20, 105, 117, 
   98, 103, 118, 1, 126, 29, 97, 122, 17, 15, 114, 110, 3, 5, 125, 
   125, 99, 126, 119, 102, 30, 122, 2, 117}, 377},

...

{{80, 2, 74, 49, 113, 21, 62, 88, 39, 71, 68, 23, 63, 51, 36, 78, 48, 
   108, 114, 102, 14, 121, 115, 31, 9, 117, 113, 111, 5, 4, 127, 28, 
   122, 101, 8, 110, 14, 18, 124, 106, 16, 20, 104, 119, 8, 109, 26, 
   106, 9, 97, 13, 99, 15, 119, 20, 105, 117, 98, 103, 118, 1, 126, 
   29, 97, 122, 17, 15, 114, 110, 3, 5, 125, 125, 99, 126, 119, 102, 
   30, 122, 2, 117}, 1},
{{80, 1, 74, 59, 113, 45, 56, 86, 52, 91, 19, 64, 60, 60, 63, 
   25, 38, 59, 59, 42, 14, 53, 38, 77, 66, 38, 113, 38, 75, 4, 43, 84,
    63, 101, 64, 43, 79, 64, 40, 57, 16, 91, 46, 119, 69, 40, 84, 117,
    9, 97, 13, 99, 15, 119, 20, 105, 117, 98, 103, 118, 1, 126, 29, 
   97, 122, 17, 15, 114, 110, 3, 5, 125, 125, 99, 126, 119, 102, 30, 
   122, 2, 117}, 1},
{{80, 2, 74, 49, 113, 49, 51, 92, 39, 8, 92, 81, 116, 62, 57, 
   80, 46, 40, 114, 36, 75, 56, 33, 76, 9, 55, 56, 59, 81, 65, 45, 28,
    60, 55, 93, 39, 90, 28, 124, 106, 16, 20, 104, 119, 8, 109, 26, 
   106, 9, 97, 13, 99, 15, 119, 20, 105, 117, 98, 103, 118, 1, 126, 
   29, 97, 122, 17, 15, 114, 110, 3, 5, 125, 125, 99, 126, 119, 102, 
   30, 122, 2, 117}, 1}}
\end{lstlisting}

Вывод Tally это список пар, каждая пара это 81-байтный блок и количество раз, сколько он встретился в файле.
Мы видим, что наиболее частно встречающийся блок это первый, он встретился 1739 раз.
Второй встретился 1422 раза. Есть и другие: 1012 раза, 377 раз, итд.
81-байтные блоки, встреченные лишь один раз, находятся в конце вывода.

Попробуем сравнить эти блоки. Первый и второй.
Есть ли в Mathematica ф-ция для сравнения списков/массивов?
Наверняка есть, но в педагогических целях, я буду использоват операцию XOR для сравнения.
Действительно: если байты во входных массивах равны друг другу, результат операции XOR это 0.
Если не равны, результат будет ненулевой.

Сравним первый блок (встречается 1739 раз) и второй (встречается 1422 раз):

\begin{lstlisting}[style=custommath]
In[]:= BitXor[stat[[1]][[1]], stat[[2]][[1]]]
Out[]= {0, 3, 0, 0, 0, 0, 0, 0, 0, 0, 0, 0, 0, 0, 0, 0, 0, 0, 0, \
0, 0, 0, 0, 0, 0, 0, 0, 0, 0, 0, 0, 0, 0, 0, 0, 0, 0, 0, 0, 0, 0, 0, \
0, 0, 0, 0, 0, 0, 0, 0, 0, 0, 0, 0, 0, 0, 0, 0, 0, 0, 0, 0, 0, 0, 0, \
0, 0, 0, 0, 0, 0, 0, 0, 0, 0, 0, 0, 0, 0, 0, 0}
\end{lstlisting}

Они отличаются только вторым байтом.

Сравним второй блок (встречается 1422 раза) и третий (встречается 1012 раз):

\begin{lstlisting}[style=custommath]
In[]:= BitXor[stat[[2]][[1]], stat[[3]][[1]]]
Out[]= {0, 1, 0, 0, 0, 0, 0, 0, 0, 0, 0, 0, 0, 0, 0, 0, 0, 0, 0, \
0, 0, 0, 0, 0, 0, 0, 0, 0, 0, 0, 0, 0, 0, 0, 0, 0, 0, 0, 0, 0, 0, 0, \
0, 0, 0, 0, 0, 0, 0, 0, 0, 0, 0, 0, 0, 0, 0, 0, 0, 0, 0, 0, 0, 0, 0, \
0, 0, 0, 0, 0, 0, 0, 0, 0, 0, 0, 0, 0, 0, 0, 0}
\end{lstlisting}

Они тоже отличаются только вторым байтом.

Так или иначе, попробуем использовать самый встречающийся блок как XOR-ключ и попробуем расшифровать первые 4 81-байтных
блока в файле:

\begin{lstlisting}[style=custommath]
In[]:= key = stat[[1]][[1]]
Out[]= {80, 103, 2, 116, 113, 102, 118, 25, 99, 8, 19, 23, 116, \
125, 107, 25, 99, 109, 114, 102, 14, 121, 115, 31, 9, 117, 113, 111, \
5, 4, 127, 28, 122, 101, 8, 110, 14, 18, 124, 106, 16, 20, 104, 119, \
8, 109, 26, 106, 9, 97, 13, 99, 15, 119, 20, 105, 117, 98, 103, 118, \
1, 126, 29, 97, 122, 17, 15, 114, 110, 3, 5, 125, 125, 99, 126, 119, \
102, 30, 122, 2, 117}

In[]:= ToASCII[val_] := If[val == 0, " ", FromCharacterCode[val, "PrintableASCII"]]

In[]:= DecryptBlockASCII[blk_] := Map[ToASCII[#] &, BitXor[key, blk]]

In[]:= DecryptBlockASCII[blocks[[1]]]
Out[]= {" ", " ", " ", " ", " ", " ", " ", " ", " ", " ", " ", " \
", " ", " ", " ", " ", " ", " ", " ", " ", " ", " ", " ", " ", " ", " \
", " ", " ", " ", " ", " ", " ", " ", " ", " ", " ", " ", " ", " ", " \
", " ", " ", " ", " ", " ", " ", " ", " ", " ", " ", " ", " ", " ", " \
", " ", " ", " ", " ", " ", " ", " ", " ", " ", " ", " ", " ", " ", " \
", " ", " ", " ", " ", " ", " ", " ", " ", " ", " ", " ", " ", " "}

In[]:= DecryptBlockASCII[blocks[[2]]]
Out[]= {" ", "e", "H", "E", " ", "W", "E", "E", "D", " ", "O", \
"F", " ", "C", "R", "I", "M", "E", " ", "B", "E", "A", "R", "S", " ", \
"B", "I", "T", "T", "E", "R", " ", "F", "R", "U", "I", "T", "?", \
" ", " ", " ", " ", " ", " ", " ", " ", " ", " ", " ", " ", " ", " ", \
" ", " ", " ", " ", " ", " ", " ", " ", " ", " ", " ", " ", " ", " ", \
" ", " ", " ", " ", " ", " ", " ", " ", " ", " ", " ", " ", " ", " ", \
" "}

In[]:= DecryptBlockASCII[blocks[[3]]]
Out[]= {" ", "?", " ", " ", " ", " ", " ", " ", " ", " ", " \
", " ", " ", " ", " ", " ", " ", " ", " ", " ", " ", " ", " ", " ", " \
", " ", " ", " ", " ", " ", " ", " ", " ", " ", " ", " ", " ", " ", " \
", " ", " ", " ", " ", " ", " ", " ", " ", " ", " ", " ", " ", " ", " \
", " ", " ", " ", " ", " ", " ", " ", " ", " ", " ", " ", " ", " ", " \
", " ", " ", " ", " ", " ", " ", " ", " ", " ", " ", " ", " ", " ", " \
"}

In[]:= DecryptBlockASCII[blocks[[4]]]
Out[]= {" ", "f", "H", "O", " ", "K", "N", "O", "W", "S", " ", \
"W", "H", "A", "T", " ", "E", "V", "I", "L", " ", "L", "U", "R", "K", \
"S", " ", "I", "N", " ", "T", "H", "E", " ", "H", "E", "A", "R", "T", \
"S", " ", "O", "F", " ", "M", "E", "N", "?", " ", " ", " ", " ", \
" ", " ", " ", " ", " ", " ", " ", " ", " ", " ", " ", " ", " ", " ", \
" ", " ", " ", " ", " ", " ", " ", " ", " ", " ", " ", " ", " ", " ", \
" "}
\end{lstlisting}

(Я заменил непечатаемые символы на \q{?}.)

Мы видим что первый и третий блоки пустые (или почти пустые),
но второй и четвертый имеют ясно различимые английские слова/фразы.
Похоже что наше предположение насчет ключа верно (как минимум частично).
Это означает, что самый встречающийся 81-байтный блок в файле находится в местах лакун с нулевыми байтами
или что-то в этом роде.

Попробуем расшифровать весь файл:

\begin{lstlisting}[style=custommath]
DecryptBlock[blk_] := BitXor[key, blk]

decrypted = Map[DecryptBlock[#] &, blocks];

BinaryWrite["/home/dennis/.../tmp", Flatten[decrypted]]

Close["/home/dennis/.../tmp"]
\end{lstlisting}

\begin{figure}[H]
\centering
\myincludegraphics{ff/XOR/mask_1/mc_decrypted1.png}
\caption{Расшифрованный файл в Midnight Commander, первая попытка}
\end{figure}

Выглядит как английские фразы для какой-то игры, но что-то не так.
Прежде всего, регистр инвертирован: фразы и некоторые слова начинаются со строчных букв,
в то время как остальные буквы заглавные.
Также, некоторые фразы начинаются с не тех букв.
Посмотрите на самую первую фразу: \q{eHE WEED OF CRIME BEARS BITTER FRUIT}.
Что такое \q{eHE}? Разве не \q{tHE} тут должно быть?
Возможно ли что наш ключ для дешифрования имеет неверный байт в этом месте?

Посмотрим снова на второй блок в файле, на ключ и на результат дешифрования:

\begin{lstlisting}[style=custommath]
In[]:= blocks[[2]]
Out[]= {80, 2, 74, 49, 113, 49, 51, 92, 39, 8, 92, 81, 116, 62, \
57, 80, 46, 40, 114, 36, 75, 56, 33, 76, 9, 55, 56, 59, 81, 65, 45, \
28, 60, 55, 93, 39, 90, 28, 124, 106, 16, 20, 104, 119, 8, 109, 26, \
106, 9, 97, 13, 99, 15, 119, 20, 105, 117, 98, 103, 118, 1, 126, 29, \
97, 122, 17, 15, 114, 110, 3, 5, 125, 125, 99, 126, 119, 102, 30, \
122, 2, 117}

In[]:= key
Out[]= {80, 103, 2, 116, 113, 102, 118, 25, 99, 8, 19, 23, 116, \
125, 107, 25, 99, 109, 114, 102, 14, 121, 115, 31, 9, 117, 113, 111, \
5, 4, 127, 28, 122, 101, 8, 110, 14, 18, 124, 106, 16, 20, 104, 119, \
8, 109, 26, 106, 9, 97, 13, 99, 15, 119, 20, 105, 117, 98, 103, 118, \
1, 126, 29, 97, 122, 17, 15, 114, 110, 3, 5, 125, 125, 99, 126, 119, \
102, 30, 122, 2, 117}

In[]:= BitXor[key, blocks[[2]]]
Out[]= {0, 101, 72, 69, 0, 87, 69, 69, 68, 0, 79, 70, 0, 67, 82, \
73, 77, 69, 0, 66, 69, 65, 82, 83, 0, 66, 73, 84, 84, 69, 82, 0, 70, \
82, 85, 73, 84, 14, 0, 0, 0, 0, 0, 0, 0, 0, 0, 0, 0, 0, 0, 0, 0, 0, \
0, 0, 0, 0, 0, 0, 0, 0, 0, 0, 0, 0, 0, 0, 0, 0, 0, 0, 0, 0, 0, 0, 0, \
0, 0, 0, 0}
\end{lstlisting}

Зашифрованный байт это 2, байт из ключа это 103, $2 \oplus 103=101$ и 101 это ASCII-код символа \q{e}.
Чему должен равнятся этот байт ключа, чтобы ASCII-код был 116 (для символа  \q{t})?
$2 \oplus 116=118$, присвоим 118 второму байту в ключе \dots

\begin{lstlisting}[style=custommath]
key = {80, 118, 2, 116, 113, 102, 118, 25, 99, 8, 19, 23, 116, 125, 
  107, 25, 99, 109, 114, 102, 14, 121, 115, 31, 9, 117, 113, 111, 5, 
  4, 127, 28, 122, 101, 8, 110, 14, 18, 124, 106, 16, 20, 104, 119, 8,
   109, 26, 106, 9, 97, 13, 99, 15, 119, 20, 105, 117, 98, 103, 118, 
  1, 126, 29, 97, 122, 17, 15, 114, 110, 3, 5, 125, 125, 99, 126, 119,
   102, 30, 122, 2, 117}
\end{lstlisting}

\dots и снова дешифруем весь файл.

\begin{figure}[H]
\centering
\myincludegraphics{ff/XOR/mask_1/mc_decrypted2.png}
\caption{Дешифрованный файл в Midnight Commander, вторая попытка}
\end{figure}

Ух ты, теперь грамматика корректна, и все фразы начинаются с корректных букв.
Но все таки, регистр подозрителен.
С чего бы разработчику игры записывать их в такой манере?
Может быть наш ключ все еще неправилен?

% TODO ASCII table somewhere in the book
Изучая таблицу ASCII мы можем заметить что ASCII-коды для букв в верхнем и нижнем регистре отличаются только на один бит
(6-й бит, если считать с первого, 0b100000):

\begin{figure}[H]
\centering
\includegraphics[width=0.7\textwidth]{ascii.png}
\caption{7-битная таблица \ac{ASCII} в Emacs}
\end{figure}

6-й бит, выставленный в нулевом байте, В десятичном виде это будет 32.
Но 32 это ASCII-код пробела!

Действительно, можно менять регистр просто применяя XOR к ASCII-коду, с 32 (больше об этом: \myref{toupper_bit}).

Возможно ли, что пустые лакуны в файле это не нулевые байты, а скорее содержащие пробелы?
Еще раз модифицируем наш XOR-ключ (я про-XOR-ю каждый байт ключа с 32):

\begin{lstlisting}[style=custommath]
(* "32" это скаляр, и "key" это вектор, но это OK *)

In[]:= key3 = BitXor[32, key]
Out[]= {112, 86, 34, 84, 81, 70, 86, 57, 67, 40, 51, 55, 84, 93, 75, \
57, 67, 77, 82, 70, 46, 89, 83, 63, 41, 85, 81, 79, 37, 36, 95, 60, \
90, 69, 40, 78, 46, 50, 92, 74, 48, 52, 72, 87, 40, 77, 58, 74, 41, \
65, 45, 67, 47, 87, 52, 73, 85, 66, 71, 86, 33, 94, 61, 65, 90, 49, \
47, 82, 78, 35, 37, 93, 93, 67, 94, 87, 70, 62, 90, 34, 85}

In[]:= DecryptBlock[blk_] := BitXor[key3, blk]
\end{lstlisting}

И снова дешифруем входной файл:

\begin{figure}[H]
\centering
\myincludegraphics{ff/XOR/mask_1/mc_decrypted.png}
\caption{Дешифрованный файл в Midnight Commander, последняя попытка}
\end{figure}

(Расшифрованный файл доступен здесь:
\url{\GitHubBlobMasterURL/ff/XOR/mask_1/files/decrypted.dat.bz2}.)

Несомненно, это корректный исходный файл.
Да, и мы видим числа в начале каждого блока. Должно быть это и есть источник некорректного XOR-ключа.
Как выходит, самый встречающийся 81-байтный блок в файле это блок заполненный пробелами и содержащий символ \q{1} на месте
второго байта.
Действительно, как-то так получилось что многие блоки здесь перемежаются с этим блоком.
Может быть это что-то вроде выравнивания (padding) для коротких фраз/сообщений?
Другой часто встречающийся 81-байтный блок также заполнен пробелами, но с другой цифрой, следовательно,
они отличаются только вторым байтом.

Вот и всё! Теперь мы можем написать утилиту для зашифрования файла назад, и, может быть, модифицировать его перед этим

Файл для Mathematica можно скачать здесь:\\
\url{\GitHubBlobMasterURL/ff/XOR/mask_1/files/XOR_mask_1.nb}.

Итог: XOR-шифрование не надежно вообще. Вероятно, разработчик игры хотел просто скрыть внутренности игры от игрока,
ничего более серьезного.
Все же, шифрование вроде этого крайне популярно вследствии его простоты, так что многие реверс инженеры обычно хорошо
с этим знакомы.


\subsection{Простое шифрование используя XOR-маску}
\label{XOR_mask_1}

Я нашел одну старую игру в стиле interactive fiction в архиве \emph{if-archive}\footnote{\url{http://www.ifarchive.org/}}:

\begin{lstlisting}
The New Castle v3.5 - Text/Adventure Game
in the style of the original Infocom (tm)
type games, Zork, Collosal Cave (Adventure),
etc.  Can you solve the mystery of the
abandoned castle?
Shareware from Software Customization.
Software Customization [ASP] Version 3.5 Feb. 2000
\end{lstlisting}

Можно скачать здесь: \url{\GitHubBlobMasterURL/ff/XOR/mask_1/files/newcastle.tgz}.

Там внутри есть файл (с названием \emph{castle.dbf}), который явно зашифрован, но не настоящим криптоалгоритмом,
и оне сжат, это что-то куда проще.
Я бы даже не стал измерять уровень энтропии (\myref{entropy}) этого файла, потому что я итак уверен, что он низкий.
Вот как он выглядит в Midnight Commander:

\begin{figure}[H]
\centering
\myincludegraphics{ff/XOR/mask_1/mc_encrypted.png}
\caption{Зашифрованный файл в Midnight Commander}
\end{figure}

Зашифрованный файл можно скачать здесь:
\url{\GitHubBlobMasterURL/ff/XOR/mask_1/files/castle.dbf.bz2}.

Можно ли расшифровать его без доступа к программе, используя просто этот файл?

Тут явно просматривается повторяющаяся строка. 
Если использовалось простое шифрование с XOR-маской, такие повторяющиеся строки это явное свидетельство,
потому что, вероятно, тут были длинные лакуны с нулевыми байтами, которые, в свою очередь, присутствуют
во мноигих исполняемых файлах, и в остальных бинарных файлах.

\myindex{UNIX!xxd}
Вот дам начала этого файла используя утилиту \emph{xxd} из UNIX:

\lstinputlisting{ff/XOR/mask_1/xxd_result.txt}

Давайте держаться за повторяющуюся строку \TT{iubgv}.
Глядя на этот дамп, мы можем легко увидеть, что период повторений этой строки это 0x51 или 81.
Вероятно, 81 это длина блока?
Длина файла 1658961, и она может быть поделена на 81 без остатка (и тогда там 20481 блоков).

Теперь я буду использовать Mathematica для анализа, есть ли тут повторяющиеся 81-байтные блоки в файле?
Я разделю входной файл на 81-байтные блоки и затем использую ф-цию
\emph{Tally[]}\footnote{\url{https://reference.wolfram.com/language/ref/Tally.html}}
которая просто считает, сколько раз каждый элемент встретился во входном списке.
Вывод Tally не отсортирован, так что я также добавлю ф-цию \emph{Sort[]} для сортировки его по кол-ву вхождений
в нисходящем порядке.

\begin{lstlisting}[style=custommath]
input = BinaryReadList["/home/dennis/.../castle.dbf"];

blocks = Partition[input, 81];

stat = Sort[Tally[blocks], #1[[2]] > #2[[2]] &]
\end{lstlisting}

И вот вывод:

\begin{lstlisting}[style=custommath]
{{{80, 103, 2, 116, 113, 102, 118, 25, 99, 8, 19, 23, 116, 125, 107, 
   25, 99, 109, 114, 102, 14, 121, 115, 31, 9, 117, 113, 111, 5, 4, 
   127, 28, 122, 101, 8, 110, 14, 18, 124, 106, 16, 20, 104, 119, 8, 
   109, 26, 106, 9, 97, 13, 99, 15, 119, 20, 105, 117, 98, 103, 118, 
   1, 126, 29, 97, 122, 17, 15, 114, 110, 3, 5, 125, 125, 99, 126, 
   119, 102, 30, 122, 2, 117}, 1739}, 
{{80, 100, 2, 116, 113, 102, 118, 25, 99, 8, 19, 23, 116, 
   125, 107, 25, 99, 109, 114, 102, 14, 121, 115, 31, 9, 117, 113, 
   111, 5, 4, 127, 28, 122, 101, 8, 110, 14, 18, 124, 106, 16, 20, 
   104, 119, 8, 109, 26, 106, 9, 97, 13, 99, 15, 119, 20, 105, 117, 
   98, 103, 118, 1, 126, 29, 97, 122, 17, 15, 114, 110, 3, 5, 125, 
   125, 99, 126, 119, 102, 30, 122, 2, 117}, 1422}, 
{{80, 101, 2, 116, 113, 102, 118, 25, 99, 8, 19, 23, 116, 
   125, 107, 25, 99, 109, 114, 102, 14, 121, 115, 31, 9, 117, 113, 
   111, 5, 4, 127, 28, 122, 101, 8, 110, 14, 18, 124, 106, 16, 20, 
   104, 119, 8, 109, 26, 106, 9, 97, 13, 99, 15, 119, 20, 105, 117, 
   98, 103, 118, 1, 126, 29, 97, 122, 17, 15, 114, 110, 3, 5, 125, 
   125, 99, 126, 119, 102, 30, 122, 2, 117}, 1012},
{{80, 120, 2, 116, 113, 102, 118, 25, 99, 8, 19, 23, 116, 
   125, 107, 25, 99, 109, 114, 102, 14, 121, 115, 31, 9, 117, 113, 
   111, 5, 4, 127, 28, 122, 101, 8, 110, 14, 18, 124, 106, 16, 20, 
   104, 119, 8, 109, 26, 106, 9, 97, 13, 99, 15, 119, 20, 105, 117, 
   98, 103, 118, 1, 126, 29, 97, 122, 17, 15, 114, 110, 3, 5, 125, 
   125, 99, 126, 119, 102, 30, 122, 2, 117}, 377},

...

{{80, 2, 74, 49, 113, 21, 62, 88, 39, 71, 68, 23, 63, 51, 36, 78, 48, 
   108, 114, 102, 14, 121, 115, 31, 9, 117, 113, 111, 5, 4, 127, 28, 
   122, 101, 8, 110, 14, 18, 124, 106, 16, 20, 104, 119, 8, 109, 26, 
   106, 9, 97, 13, 99, 15, 119, 20, 105, 117, 98, 103, 118, 1, 126, 
   29, 97, 122, 17, 15, 114, 110, 3, 5, 125, 125, 99, 126, 119, 102, 
   30, 122, 2, 117}, 1},
{{80, 1, 74, 59, 113, 45, 56, 86, 52, 91, 19, 64, 60, 60, 63, 
   25, 38, 59, 59, 42, 14, 53, 38, 77, 66, 38, 113, 38, 75, 4, 43, 84,
    63, 101, 64, 43, 79, 64, 40, 57, 16, 91, 46, 119, 69, 40, 84, 117,
    9, 97, 13, 99, 15, 119, 20, 105, 117, 98, 103, 118, 1, 126, 29, 
   97, 122, 17, 15, 114, 110, 3, 5, 125, 125, 99, 126, 119, 102, 30, 
   122, 2, 117}, 1},
{{80, 2, 74, 49, 113, 49, 51, 92, 39, 8, 92, 81, 116, 62, 57, 
   80, 46, 40, 114, 36, 75, 56, 33, 76, 9, 55, 56, 59, 81, 65, 45, 28,
    60, 55, 93, 39, 90, 28, 124, 106, 16, 20, 104, 119, 8, 109, 26, 
   106, 9, 97, 13, 99, 15, 119, 20, 105, 117, 98, 103, 118, 1, 126, 
   29, 97, 122, 17, 15, 114, 110, 3, 5, 125, 125, 99, 126, 119, 102, 
   30, 122, 2, 117}, 1}}
\end{lstlisting}

Вывод Tally это список пар, каждая пара это 81-байтный блок и количество раз, сколько он встретился в файле.
Мы видим, что наиболее частно встречающийся блок это первый, он встретился 1739 раз.
Второй встретился 1422 раза. Есть и другие: 1012 раза, 377 раз, итд.
81-байтные блоки, встреченные лишь один раз, находятся в конце вывода.

Попробуем сравнить эти блоки. Первый и второй.
Есть ли в Mathematica ф-ция для сравнения списков/массивов?
Наверняка есть, но в педагогических целях, я буду использоват операцию XOR для сравнения.
Действительно: если байты во входных массивах равны друг другу, результат операции XOR это 0.
Если не равны, результат будет ненулевой.

Сравним первый блок (встречается 1739 раз) и второй (встречается 1422 раз):

\begin{lstlisting}[style=custommath]
In[]:= BitXor[stat[[1]][[1]], stat[[2]][[1]]]
Out[]= {0, 3, 0, 0, 0, 0, 0, 0, 0, 0, 0, 0, 0, 0, 0, 0, 0, 0, 0, \
0, 0, 0, 0, 0, 0, 0, 0, 0, 0, 0, 0, 0, 0, 0, 0, 0, 0, 0, 0, 0, 0, 0, \
0, 0, 0, 0, 0, 0, 0, 0, 0, 0, 0, 0, 0, 0, 0, 0, 0, 0, 0, 0, 0, 0, 0, \
0, 0, 0, 0, 0, 0, 0, 0, 0, 0, 0, 0, 0, 0, 0, 0}
\end{lstlisting}

Они отличаются только вторым байтом.

Сравним второй блок (встречается 1422 раза) и третий (встречается 1012 раз):

\begin{lstlisting}[style=custommath]
In[]:= BitXor[stat[[2]][[1]], stat[[3]][[1]]]
Out[]= {0, 1, 0, 0, 0, 0, 0, 0, 0, 0, 0, 0, 0, 0, 0, 0, 0, 0, 0, \
0, 0, 0, 0, 0, 0, 0, 0, 0, 0, 0, 0, 0, 0, 0, 0, 0, 0, 0, 0, 0, 0, 0, \
0, 0, 0, 0, 0, 0, 0, 0, 0, 0, 0, 0, 0, 0, 0, 0, 0, 0, 0, 0, 0, 0, 0, \
0, 0, 0, 0, 0, 0, 0, 0, 0, 0, 0, 0, 0, 0, 0, 0}
\end{lstlisting}

Они тоже отличаются только вторым байтом.

Так или иначе, попробуем использовать самый встречающийся блок как XOR-ключ и попробуем расшифровать первые 4 81-байтных
блока в файле:

\begin{lstlisting}[style=custommath]
In[]:= key = stat[[1]][[1]]
Out[]= {80, 103, 2, 116, 113, 102, 118, 25, 99, 8, 19, 23, 116, \
125, 107, 25, 99, 109, 114, 102, 14, 121, 115, 31, 9, 117, 113, 111, \
5, 4, 127, 28, 122, 101, 8, 110, 14, 18, 124, 106, 16, 20, 104, 119, \
8, 109, 26, 106, 9, 97, 13, 99, 15, 119, 20, 105, 117, 98, 103, 118, \
1, 126, 29, 97, 122, 17, 15, 114, 110, 3, 5, 125, 125, 99, 126, 119, \
102, 30, 122, 2, 117}

In[]:= ToASCII[val_] := If[val == 0, " ", FromCharacterCode[val, "PrintableASCII"]]

In[]:= DecryptBlockASCII[blk_] := Map[ToASCII[#] &, BitXor[key, blk]]

In[]:= DecryptBlockASCII[blocks[[1]]]
Out[]= {" ", " ", " ", " ", " ", " ", " ", " ", " ", " ", " ", " \
", " ", " ", " ", " ", " ", " ", " ", " ", " ", " ", " ", " ", " ", " \
", " ", " ", " ", " ", " ", " ", " ", " ", " ", " ", " ", " ", " ", " \
", " ", " ", " ", " ", " ", " ", " ", " ", " ", " ", " ", " ", " ", " \
", " ", " ", " ", " ", " ", " ", " ", " ", " ", " ", " ", " ", " ", " \
", " ", " ", " ", " ", " ", " ", " ", " ", " ", " ", " ", " ", " "}

In[]:= DecryptBlockASCII[blocks[[2]]]
Out[]= {" ", "e", "H", "E", " ", "W", "E", "E", "D", " ", "O", \
"F", " ", "C", "R", "I", "M", "E", " ", "B", "E", "A", "R", "S", " ", \
"B", "I", "T", "T", "E", "R", " ", "F", "R", "U", "I", "T", "?", \
" ", " ", " ", " ", " ", " ", " ", " ", " ", " ", " ", " ", " ", " ", \
" ", " ", " ", " ", " ", " ", " ", " ", " ", " ", " ", " ", " ", " ", \
" ", " ", " ", " ", " ", " ", " ", " ", " ", " ", " ", " ", " ", " ", \
" "}

In[]:= DecryptBlockASCII[blocks[[3]]]
Out[]= {" ", "?", " ", " ", " ", " ", " ", " ", " ", " ", " \
", " ", " ", " ", " ", " ", " ", " ", " ", " ", " ", " ", " ", " ", " \
", " ", " ", " ", " ", " ", " ", " ", " ", " ", " ", " ", " ", " ", " \
", " ", " ", " ", " ", " ", " ", " ", " ", " ", " ", " ", " ", " ", " \
", " ", " ", " ", " ", " ", " ", " ", " ", " ", " ", " ", " ", " ", " \
", " ", " ", " ", " ", " ", " ", " ", " ", " ", " ", " ", " ", " ", " \
"}

In[]:= DecryptBlockASCII[blocks[[4]]]
Out[]= {" ", "f", "H", "O", " ", "K", "N", "O", "W", "S", " ", \
"W", "H", "A", "T", " ", "E", "V", "I", "L", " ", "L", "U", "R", "K", \
"S", " ", "I", "N", " ", "T", "H", "E", " ", "H", "E", "A", "R", "T", \
"S", " ", "O", "F", " ", "M", "E", "N", "?", " ", " ", " ", " ", \
" ", " ", " ", " ", " ", " ", " ", " ", " ", " ", " ", " ", " ", " ", \
" ", " ", " ", " ", " ", " ", " ", " ", " ", " ", " ", " ", " ", " ", \
" "}
\end{lstlisting}

(Я заменил непечатаемые символы на \q{?}.)

Мы видим что первый и третий блоки пустые (или почти пустые),
но второй и четвертый имеют ясно различимые английские слова/фразы.
Похоже что наше предположение насчет ключа верно (как минимум частично).
Это означает, что самый встречающийся 81-байтный блок в файле находится в местах лакун с нулевыми байтами
или что-то в этом роде.

Попробуем расшифровать весь файл:

\begin{lstlisting}[style=custommath]
DecryptBlock[blk_] := BitXor[key, blk]

decrypted = Map[DecryptBlock[#] &, blocks];

BinaryWrite["/home/dennis/.../tmp", Flatten[decrypted]]

Close["/home/dennis/.../tmp"]
\end{lstlisting}

\begin{figure}[H]
\centering
\myincludegraphics{ff/XOR/mask_1/mc_decrypted1.png}
\caption{Расшифрованный файл в Midnight Commander, первая попытка}
\end{figure}

Выглядит как английские фразы для какой-то игры, но что-то не так.
Прежде всего, регистр инвертирован: фразы и некоторые слова начинаются со строчных букв,
в то время как остальные буквы заглавные.
Также, некоторые фразы начинаются с не тех букв.
Посмотрите на самую первую фразу: \q{eHE WEED OF CRIME BEARS BITTER FRUIT}.
Что такое \q{eHE}? Разве не \q{tHE} тут должно быть?
Возможно ли что наш ключ для дешифрования имеет неверный байт в этом месте?

Посмотрим снова на второй блок в файле, на ключ и на результат дешифрования:

\begin{lstlisting}[style=custommath]
In[]:= blocks[[2]]
Out[]= {80, 2, 74, 49, 113, 49, 51, 92, 39, 8, 92, 81, 116, 62, \
57, 80, 46, 40, 114, 36, 75, 56, 33, 76, 9, 55, 56, 59, 81, 65, 45, \
28, 60, 55, 93, 39, 90, 28, 124, 106, 16, 20, 104, 119, 8, 109, 26, \
106, 9, 97, 13, 99, 15, 119, 20, 105, 117, 98, 103, 118, 1, 126, 29, \
97, 122, 17, 15, 114, 110, 3, 5, 125, 125, 99, 126, 119, 102, 30, \
122, 2, 117}

In[]:= key
Out[]= {80, 103, 2, 116, 113, 102, 118, 25, 99, 8, 19, 23, 116, \
125, 107, 25, 99, 109, 114, 102, 14, 121, 115, 31, 9, 117, 113, 111, \
5, 4, 127, 28, 122, 101, 8, 110, 14, 18, 124, 106, 16, 20, 104, 119, \
8, 109, 26, 106, 9, 97, 13, 99, 15, 119, 20, 105, 117, 98, 103, 118, \
1, 126, 29, 97, 122, 17, 15, 114, 110, 3, 5, 125, 125, 99, 126, 119, \
102, 30, 122, 2, 117}

In[]:= BitXor[key, blocks[[2]]]
Out[]= {0, 101, 72, 69, 0, 87, 69, 69, 68, 0, 79, 70, 0, 67, 82, \
73, 77, 69, 0, 66, 69, 65, 82, 83, 0, 66, 73, 84, 84, 69, 82, 0, 70, \
82, 85, 73, 84, 14, 0, 0, 0, 0, 0, 0, 0, 0, 0, 0, 0, 0, 0, 0, 0, 0, \
0, 0, 0, 0, 0, 0, 0, 0, 0, 0, 0, 0, 0, 0, 0, 0, 0, 0, 0, 0, 0, 0, 0, \
0, 0, 0, 0}
\end{lstlisting}

Зашифрованный байт это 2, байт из ключа это 103, $2 \oplus 103=101$ и 101 это ASCII-код символа \q{e}.
Чему должен равнятся этот байт ключа, чтобы ASCII-код был 116 (для символа  \q{t})?
$2 \oplus 116=118$, присвоим 118 второму байту в ключе \dots

\begin{lstlisting}[style=custommath]
key = {80, 118, 2, 116, 113, 102, 118, 25, 99, 8, 19, 23, 116, 125, 
  107, 25, 99, 109, 114, 102, 14, 121, 115, 31, 9, 117, 113, 111, 5, 
  4, 127, 28, 122, 101, 8, 110, 14, 18, 124, 106, 16, 20, 104, 119, 8,
   109, 26, 106, 9, 97, 13, 99, 15, 119, 20, 105, 117, 98, 103, 118, 
  1, 126, 29, 97, 122, 17, 15, 114, 110, 3, 5, 125, 125, 99, 126, 119,
   102, 30, 122, 2, 117}
\end{lstlisting}

\dots и снова дешифруем весь файл.

\begin{figure}[H]
\centering
\myincludegraphics{ff/XOR/mask_1/mc_decrypted2.png}
\caption{Дешифрованный файл в Midnight Commander, вторая попытка}
\end{figure}

Ух ты, теперь грамматика корректна, и все фразы начинаются с корректных букв.
Но все таки, регистр подозрителен.
С чего бы разработчику игры записывать их в такой манере?
Может быть наш ключ все еще неправилен?

% TODO ASCII table somewhere in the book
Изучая таблицу ASCII мы можем заметить что ASCII-коды для букв в верхнем и нижнем регистре отличаются только на один бит
(6-й бит, если считать с первого, 0b100000):

\begin{figure}[H]
\centering
\includegraphics[width=0.7\textwidth]{ascii.png}
\caption{7-битная таблица \ac{ASCII} в Emacs}
\end{figure}

6-й бит, выставленный в нулевом байте, В десятичном виде это будет 32.
Но 32 это ASCII-код пробела!

Действительно, можно менять регистр просто применяя XOR к ASCII-коду, с 32 (больше об этом: \myref{toupper_bit}).

Возможно ли, что пустые лакуны в файле это не нулевые байты, а скорее содержащие пробелы?
Еще раз модифицируем наш XOR-ключ (я про-XOR-ю каждый байт ключа с 32):

\begin{lstlisting}[style=custommath]
(* "32" это скаляр, и "key" это вектор, но это OK *)

In[]:= key3 = BitXor[32, key]
Out[]= {112, 86, 34, 84, 81, 70, 86, 57, 67, 40, 51, 55, 84, 93, 75, \
57, 67, 77, 82, 70, 46, 89, 83, 63, 41, 85, 81, 79, 37, 36, 95, 60, \
90, 69, 40, 78, 46, 50, 92, 74, 48, 52, 72, 87, 40, 77, 58, 74, 41, \
65, 45, 67, 47, 87, 52, 73, 85, 66, 71, 86, 33, 94, 61, 65, 90, 49, \
47, 82, 78, 35, 37, 93, 93, 67, 94, 87, 70, 62, 90, 34, 85}

In[]:= DecryptBlock[blk_] := BitXor[key3, blk]
\end{lstlisting}

И снова дешифруем входной файл:

\begin{figure}[H]
\centering
\myincludegraphics{ff/XOR/mask_1/mc_decrypted.png}
\caption{Дешифрованный файл в Midnight Commander, последняя попытка}
\end{figure}

(Расшифрованный файл доступен здесь:
\url{\GitHubBlobMasterURL/ff/XOR/mask_1/files/decrypted.dat.bz2}.)

Несомненно, это корректный исходный файл.
Да, и мы видим числа в начале каждого блока. Должно быть это и есть источник некорректного XOR-ключа.
Как выходит, самый встречающийся 81-байтный блок в файле это блок заполненный пробелами и содержащий символ \q{1} на месте
второго байта.
Действительно, как-то так получилось что многие блоки здесь перемежаются с этим блоком.
Может быть это что-то вроде выравнивания (padding) для коротких фраз/сообщений?
Другой часто встречающийся 81-байтный блок также заполнен пробелами, но с другой цифрой, следовательно,
они отличаются только вторым байтом.

Вот и всё! Теперь мы можем написать утилиту для зашифрования файла назад, и, может быть, модифицировать его перед этим

Файл для Mathematica можно скачать здесь:\\
\url{\GitHubBlobMasterURL/ff/XOR/mask_1/files/XOR_mask_1.nb}.

Итог: XOR-шифрование не надежно вообще. Вероятно, разработчик игры хотел просто скрыть внутренности игры от игрока,
ничего более серьезного.
Все же, шифрование вроде этого крайне популярно вследствии его простоты, так что многие реверс инженеры обычно хорошо
с этим знакомы.


\mysection{\MinesweeperWinXPExampleChapterName}
\label{minesweeper_winxp}
\myindex{Windows!Windows XP}

Для тех, кто не очень хорошо играет в Сапёра (Minesweeper), можно попробовать найти все скрытые мины в отладчике.

\myindex{\CStandardLibrary!rand()}
\myindex{Windows!PDB}
Как мы знаем, Сапёр располагает мины случайным образом, так что там должен быть генератор случайных чисел
или вызов стандартной функции Си \TT{rand()}.

Вот что хорошо в реверсинге продуктов от Microsoft, так это то что часто есть \gls{PDB}-файл со всеми
символами (имена функций, и~т.д.).

Когда мы загружаем \TT{winmine.exe} в \IDA, она скачивает 
\gls{PDB} файл именно для этого исполняемого файла и добавляет все имена.

И вот оно, только один вызов \TT{rand()} в этой функции:

\lstinputlisting[style=customasmx86]{examples/minesweeper/tmp1.lst}

Так её назвала \IDA и это было имя данное ей разработчиками Сапёра.

Функция очень простая:

\begin{lstlisting}[style=customc]
int Rnd(int limit)
{
    return rand() % limit;
};
\end{lstlisting}

(В \gls{PDB}-файле не было имени \q{limit}; это мы назвали этот аргумент так, вручную.)

Так что она возвращает случайное число в пределах от нуля до заданного предела.

\TT{Rnd()} вызывается только из одного места, это функция с названием \TT{StartGame()}, 
и как видно, это именно тот код, что расставляет мины:

\begin{lstlisting}[style=customasmx86]
.text:010036C7                 push    _xBoxMac
.text:010036CD                 call    _Rnd@4          ; Rnd(x)
.text:010036D2                 push    _yBoxMac
.text:010036D8                 mov     esi, eax
.text:010036DA                 inc     esi
.text:010036DB                 call    _Rnd@4          ; Rnd(x)
.text:010036E0                 inc     eax
.text:010036E1                 mov     ecx, eax
.text:010036E3                 shl     ecx, 5          ; ECX=ECX*32
.text:010036E6                 test    _rgBlk[ecx+esi], 80h
.text:010036EE                 jnz     short loc_10036C7
.text:010036F0                 shl     eax, 5          ; EAX=EAX*32
.text:010036F3                 lea     eax, _rgBlk[eax+esi]
.text:010036FA                 or      byte ptr [eax], 80h
.text:010036FD                 dec     _cBombStart
.text:01003703                 jnz     short loc_10036C7
\end{lstlisting}

Сапёр позволяет задать размеры доски, так что X (xBoxMac) и Y (yBoxMac) это глобальные переменные.

Они передаются в \TT{Rnd()} и генерируются случайные координаты.
Мина устанавливается инструкцией \TT{OR} на \TT{0x010036FA}. 
И если она уже была установлена до этого 
(это возможно, если пара функций \TT{Rnd()} 
сгенерирует пару, которая уже была сгенерирована), 
тогда \TT{TEST} и \TT{JNZ} на \TT{0x010036E6} 
перейдет на повторную генерацию пары.

\TT{cBombStart} это глобальная переменная, содержащая количество мин. Так что это цикл.

Ширина двухмерного массива это 32 (мы можем это вывести, глядя на инструкцию \TT{SHL}, которая умножает
одну из координат на 32).

Размер глобального массива \TT{rgBlk} 
можно легко узнать по разнице между меткой \TT{rgBlk} 
в сегменте данных и следующей известной меткой. 
Это 0x360 (864):

\begin{lstlisting}[style=customasmx86]
.data:01005340 _rgBlk          db 360h dup(?)          ; DATA XREF: MainWndProc(x,x,x,x)+574
.data:01005340                                         ; DisplayBlk(x,x)+23
.data:010056A0 _Preferences    dd ?                    ; DATA XREF: FixMenus()+2
...
\end{lstlisting}

$864/32=27$.

Так что размер массива $27*32$?
Это близко к тому что мы знаем: если попытаемся установить размер доски в установках Сапёра на $100*100$, то он установит размер $24*30$.
Так что это максимальный размер доски здесь.
И размер массива фиксирован для доски любого размера.

Посмотрим на всё это в \olly.
Запустим Сапёр, присоединим (attach) \olly к нему и увидим содержимое памяти по адресу где массив \TT{rgBlk} (\TT{0x01005340})%

\footnote{Все адреса здесь для Сапёра под Windows XP SP3 English. 
Они могут отличаться для других сервис-паков.}.

Так что у нас выходит такой дамп памяти массива:

\lstinputlisting[style=customasmx86]{examples/minesweeper/1.lst}

\olly, как и любой другой шестнадцатеричный редактор, показывает 16 байт на строку.
Так что каждая 32-байтная строка массива занимает ровно 2 строки.

Это уровень для начинающих (доска 9*9).

Тут еще какая-то квадратная структура, заметная визуально (байты 0x10).

Нажмем \q{Run} в \olly чтобы разморозить процесс Сапёра, потом нажмем в случайное место окна Сапёра, попадаемся на мине, но теперь
видны все мины:

\begin{figure}[H]
\centering
\myincludegraphicsSmall{examples/minesweeper/1.png}
\caption{Мины}
\label{fig:minesweeper1}
\end{figure}

Сравнивая места с минами и дамп, мы можем обнаружить что 0x10 это граница, 0x0F --- пустой блок, 
0x8F --- мина.
Вероятно 0x10 это т.н., \emph{sentinel value}.

Теперь добавим комментариев и также заключим все байты 0x8F в квадратные скобки:%

\lstinputlisting[style=customasmx86]{examples/minesweeper/2.lst}

Теперь уберем все байты связанные с границами (0x10) и всё что за ними:%

\lstinputlisting[style=customasmx86]{examples/minesweeper/3.lst}

Да, это всё мины, теперь это очень хорошо видно, в сравнении со скриншотом.

\clearpage
Вот что интересно, это то что мы можем модифицировать массив прямо в \olly.%

Уберем все мины заменив все байты 0x8F на 0x0F, и вот что получится в Сапёре:

\begin{figure}[H]
\centering
\myincludegraphicsSmall{examples/minesweeper/3.png}
\caption{Все мины убраны в отладчике}
\label{fig:minesweeper3}
\end{figure}

Также уберем их все и добавим их в первом ряду: 

\begin{figure}[H]
\centering
\myincludegraphicsSmall{examples/minesweeper/2.png}
\caption{Мины, установленные в отладчике}
\label{fig:minesweeper2}
\end{figure}

Отладчик не очень удобен для подсматривания (а это была наша изначальная цель), так что напишем маленькую
утилиту для показа содержимого доски:

\lstinputlisting[style=customc]{examples/minesweeper/minesweeper_cheater.c}

Просто установите \ac{PID}
\footnote{PID можно увидеть в Task Manager 
(это можно включить в \q{View $\rightarrow$ Select Columns})} 
и адрес массива (\TT{0x01005340} для Windows XP SP3 English) 
и она покажет его
\footnote{Скомпилированная версия здесь: 
\href{http://go.yurichev.com/17165}{beginners.re}}.

Она подключается к win32-процессу по \ac{PID}-у и просто читает из памяти процесса по этому адресу.

\subsection{Автоматический поиск массива}

Задавать адрес каждый раз при запуске нашей утилиты, это неудобно.
К тому же, разные версии ``Сапёра'' могут иметь этот массив по разным адресам.
Зная, что всегда есть рамка (байты 0x10), массив легко найти в памяти:

\lstinputlisting[style=customc]{examples/minesweeper/cheater2_fragment.c}

Полный исходный код: \url{\RepoURL/examples/minesweeper/minesweeper_cheater2.c}.

\subsection{\Exercises}

\begin{itemize}

\item
Почему байты описывающие границы (0x10) (или \emph{sentinel value}) присутствуют вообще?
Зачем они нужны, если они вообще не видимы в интерфейсе Сапёра?
Как можно обойтись без них?

\item
Как выясняется, здесь больше возможных значений (для открытых блоков, для тех на которых игрок установил
флажок, и~т.д.).
	
Попробуйте найти значение каждого.

\item Измените мою утилиту так, чтобы она в запущенном процессе Сапёра убирала все мины, 
или расставляла их в соответствии с каким-то заданным шаблоном.

\end{itemize}

\mysection{Хакаем часы в Windows}

Иногда я устраиваю первоапрельские пранки для моих сотрудников.

Посмотрим, можем ли мы сделать что-то с часами в Windows?
Можем ли мы их заставить идти в обратную сторону?

Прежде всего, когда вы кликаете на часы/время в строке состояния (\emph{status bar}),\\
запускается модуль \emph{C:\textbackslash{}WINDOWS\textbackslash{}SYSTEM32\textbackslash{}TIMEDATE.CPL},
а это обычный \ac{PE}-файл.

Посмотрим, как отрисовываются стрелки?
Когда я открываю этот файл (из Windows 7) в Resource Hacker, здесь есть разные виды циферблата, но нет стрелок:

\begin{figure}[H]
\centering
\myincludegraphics{examples/timedate/reshack.png}
\caption{Resource Hacker}
\end{figure}

ОК, что мы знаем? Как рисовать стрелку часов? Они все начинаются в середине круга и заканчиваются на его границе.
Следовательно, нам нужно расчитать координаты точки на границе круга.
Из школьной математики мы можем вспомнить, что для рисования круга нужно использовать ф-ции синуса/косинуса, или
хотя бы квадратного корня.
Такого в \emph{TIMEDATE.CPL} нет, по крайней мере на первый взгляд.
Но, благодаря отладочным PDB-файлам от Microsoft, я могу найти ф-цию с названием \emph{CAnalogClock::DrawHand()}, которая
вызывает \emph{Gdiplus::Graphics::DrawLine()} минимум дважды.

Вот её код:

\lstinputlisting[style=customasmx86]{examples/timedate/1.lst}

\myindex{Windows!Win32!MulDiv()}
Мы можем увидеть что аргументы \emph{DrawLine()} зависят от результата ф-ции \emph{MulDiv()}
и таблицы \emph{table[]} (название дал я),
которая содержит 8-байтные элементы (посмотрите на второй операнд \INS{LEA}).

Что внутри table[]?

\lstinputlisting[style=customasmx86]{examples/timedate/2.lst}

Доступ к ней есть только из ф-ции \emph{DrawHand()}.
У ней 120 32-битных слов или 60 32-битных пар \dots подождите, 60?
Посмотрим ближе на эти значения.
Прежде всего, я затру нулями первые 6 пар или 12 32-битных слов, и затем я положу пропатченный \emph{TIMEDATE.CPL}
в \emph{C:\textbackslash{}WINDOWS\textbackslash{}SYSTEM32}.
(Вам, возможно, придется установить владельца файла *TIMEDATE.CPL* равным вашей первичной пользовательской учетной
записи (вместо \emph{TrustedInstaller}),
а также загрузиться в безопасном режиме с командной строкой, чтобы скопировать этот файл, который обычно залоченный.)

\begin{figure}[H]
\centering
\includegraphics[width=0.5\textwidth]{examples/timedate/6_pairs_zeroed.png}
\caption{Attempt to run}
\end{figure}

Теперь, когда стрелка находится на 0..5 секундах/минутах, она невидимая! Хотя, противоположная (короткая) часть секундной
стрелки видима, и двигается.
Когда любая стрелка за пределами этой области, она видима, как обычно.

\myindex{Mathematica}
Посмотрим на эту таблицу при помощи Mathematica.
Я скопировал таблицу из \emph{TIMEDATE.CPL} в файл \emph{tbl} (480 байт).
Будем считать, что это знаковые значения, потому что половина элементов меньше нуля (0FFFFE0C1h, итд.).
Если бы эти значения были бы беззнаковыми, они были бы подозрительно большими.

\begin{lstlisting}[style=custommath]
In[]:= tbl = BinaryReadList["~/.../tbl", "Integer32"]

Out[]= {0, -7999, 836, -7956, 1663, -7825, 2472, -7608, 3253, -7308, 3999, \
-6928, 4702, -6472, 5353, -5945, 5945, -5353, 6472, -4702, 6928, \
-4000, 7308, -3253, 7608, -2472, 7825, -1663, 7956, -836, 8000, 0, \
7956, 836, 7825, 1663, 7608, 2472, 7308, 3253, 6928, 4000, 6472, \
4702, 5945, 5353, 5353, 5945, 4702, 6472, 3999, 6928, 3253, 7308, \
2472, 7608, 1663, 7825, 836, 7956, 0, 7999, -836, 7956, -1663, 7825, \
-2472, 7608, -3253, 7308, -4000, 6928, -4702, 6472, -5353, 5945, \
-5945, 5353, -6472, 4702, -6928, 3999, -7308, 3253, -7608, 2472, \
-7825, 1663, -7956, 836, -7999, 0, -7956, -836, -7825, -1663, -7608, \
-2472, -7308, -3253, -6928, -4000, -6472, -4702, -5945, -5353, -5353, \
-5945, -4702, -6472, -3999, -6928, -3253, -7308, -2472, -7608, -1663, \
-7825, -836, -7956}

In[]:= Length[tbl]
Out[]= 120
\end{lstlisting}

Будем считать два последовательно идущих 32-битных значения как пару:

\begin{lstlisting}[style=custommath]
In[]:= pairs = Partition[tbl, 2]
Out[]= {{0, -7999}, {836, -7956}, {1663, -7825}, {2472, -7608}, \
{3253, -7308}, {3999, -6928}, {4702, -6472}, {5353, -5945}, {5945, \
-5353}, {6472, -4702}, {6928, -4000}, {7308, -3253}, {7608, -2472}, \
{7825, -1663}, {7956, -836}, {8000, 0}, {7956, 836}, {7825, 
1663}, {7608, 2472}, {7308, 3253}, {6928, 4000}, {6472, 
4702}, {5945, 5353}, {5353, 5945}, {4702, 6472}, {3999, 
6928}, {3253, 7308}, {2472, 7608}, {1663, 7825}, {836, 7956}, {0, 
7999}, {-836, 7956}, {-1663, 7825}, {-2472, 7608}, {-3253, 
7308}, {-4000, 6928}, {-4702, 6472}, {-5353, 5945}, {-5945, 
5353}, {-6472, 4702}, {-6928, 3999}, {-7308, 3253}, {-7608, 
2472}, {-7825, 1663}, {-7956, 836}, {-7999, 
0}, {-7956, -836}, {-7825, -1663}, {-7608, -2472}, {-7308, -3253}, \
{-6928, -4000}, {-6472, -4702}, {-5945, -5353}, {-5353, -5945}, \
{-4702, -6472}, {-3999, -6928}, {-3253, -7308}, {-2472, -7608}, \
{-1663, -7825}, {-836, -7956}}

In[]:= Length[pairs]
Out[]= 60
\end{lstlisting}

Попробуем считать каждую пару как координату X/Y и нарисуем все 60 пар, а также первые 15 пар:

\begin{figure}[H]
\centering
\myincludegraphics{examples/timedate/math.png}
\caption{Mathematica}
\end{figure}

Ну теперь это кое-что!
Каждая пара это просто координата.
Первые 15 пар это координаты $\frac{1}{4}$ круга.

Видимо, разработчики в Microsoft расчитали все координаты предварительно и сохранили их в таблице.
\myindex{Memoization}
Это распространенная, хотя и немного олдскульная практика -- доступ к предвычисленным значениям в таблице быстрее, чем вызывать относительно медленные ф-ции
синуса/косинуса\footnote{Сегодня это называют \emph{memoization}}.
В наше время операции синуса/косинуса уже не такие \emph{дорогие}.

Теперь понятно, почему когда я затер первые 6 пар, стрелки были невидимы в этой области: на самом деле, стрелки рисовались,
просто их длина была нулевой, потому что стрелка начиналась в координатах 0:0, и там же и заканчивалась.

\subsubsection{Пранк (шутка)}

Учитывая всё это, можем ли мы заставить стрелки идти в обратную сторону?
На самом деле, это просто, нужно просто развернуть таблицу, так что каждая стрелка,
вместо отображения на месте нулевой секунды, рисовалась бы на месте 59-й секунды.

Когда-то давным давно я сделал патчер, в в самом начале 2000-х, для Windows 2000.
Трудно поверить, но он всё еще работает и для Windows 7, видимо, таблица с тех пор не менялась!

Исходник патчера: \url{\RepoURL/examples/timedate/time_pt.c}.

Теперь видно как стрелки идут назад:

\begin{figure}[H]
\centering
\includegraphics[width=0.5\textwidth]{examples/timedate/counterclockwise.png}
\caption{Now it works}
\end{figure}

В этой книге, конечно, нет анимации, но если присмотритесь, увидите, что на самом деле стрелки показывают корректное время,
но весь циферблат повернут вертикально, как если бы мы видели его изнутри часов.

\subsubsection{Утекшие исходники Windows 2000}

Так что я сделал патчер, потом утекли исходники Windows 2000 (я не могу заставить вас поверить мне, конечно).
Посмотрим на исходный код этой ф-ции и таблицы.\\
Нужный файл это \emph{win2k/private/shell/cpls/utc/clock.c}:

\begin{lstlisting}[style=customc]
//
//  Array containing the sine and cosine values for hand positions.
//
POINT rCircleTable[] =
{
    { 0,     -7999},
    { 836,   -7956},
    { 1663,  -7825},
    { 2472,  -7608},
    { 3253,  -7308},
...
    { -4702, -6472},
    { -3999, -6928},
    { -3253, -7308},
    { -2472, -7608},
    { -1663, -7825},
    { -836 , -7956},
};

////////////////////////////////////////////////////////////////////////////
//
//  DrawHand
//
//  Draws the hands of the clock.
//
////////////////////////////////////////////////////////////////////////////

void DrawHand(
    HDC hDC,
    int pos,
    HPEN hPen,
    int scale,
    int patMode,
    PCLOCKSTR np)
{
    LPPOINT lppt;
    int radius;

    MoveTo(hDC, np->clockCenter.x, np->clockCenter.y);
    radius = MulDiv(np->clockRadius, scale, 100);
    lppt = rCircleTable + pos;
    SetROP2(hDC, patMode);
    SelectObject(hDC, hPen);

    LineTo( hDC,
            np->clockCenter.x + MulDiv(lppt->x, radius, 8000),
            np->clockCenter.y + MulDiv(lppt->y, radius, 8000) );
}
\end{lstlisting}

Теперь всё ясно: координаты были предвычислены, как если бы циферблат был размером $2 \cdot 8000$,
а затем он масштабируется до радиуса текущего циферблата используя ф-цию \emph{MulDiv()}.

Структура POINT\footnote{\url{https://msdn.microsoft.com/en-us/library/windows/desktop/dd162805(v=vs.85).aspx}}
это структура из двух 32-битных значений, первое это \emph{x}, второе это \emph{y}.


%\subsection{Простое шифрование используя XOR-маску}
\label{XOR_mask_1}

Я нашел одну старую игру в стиле interactive fiction в архиве \emph{if-archive}\footnote{\url{http://www.ifarchive.org/}}:

\begin{lstlisting}
The New Castle v3.5 - Text/Adventure Game
in the style of the original Infocom (tm)
type games, Zork, Collosal Cave (Adventure),
etc.  Can you solve the mystery of the
abandoned castle?
Shareware from Software Customization.
Software Customization [ASP] Version 3.5 Feb. 2000
\end{lstlisting}

Можно скачать здесь: \url{\GitHubBlobMasterURL/ff/XOR/mask_1/files/newcastle.tgz}.

Там внутри есть файл (с названием \emph{castle.dbf}), который явно зашифрован, но не настоящим криптоалгоритмом,
и оне сжат, это что-то куда проще.
Я бы даже не стал измерять уровень энтропии (\myref{entropy}) этого файла, потому что я итак уверен, что он низкий.
Вот как он выглядит в Midnight Commander:

\begin{figure}[H]
\centering
\myincludegraphics{ff/XOR/mask_1/mc_encrypted.png}
\caption{Зашифрованный файл в Midnight Commander}
\end{figure}

Зашифрованный файл можно скачать здесь:
\url{\GitHubBlobMasterURL/ff/XOR/mask_1/files/castle.dbf.bz2}.

Можно ли расшифровать его без доступа к программе, используя просто этот файл?

Тут явно просматривается повторяющаяся строка. 
Если использовалось простое шифрование с XOR-маской, такие повторяющиеся строки это явное свидетельство,
потому что, вероятно, тут были длинные лакуны с нулевыми байтами, которые, в свою очередь, присутствуют
во мноигих исполняемых файлах, и в остальных бинарных файлах.

\myindex{UNIX!xxd}
Вот дам начала этого файла используя утилиту \emph{xxd} из UNIX:

\lstinputlisting{ff/XOR/mask_1/xxd_result.txt}

Давайте держаться за повторяющуюся строку \TT{iubgv}.
Глядя на этот дамп, мы можем легко увидеть, что период повторений этой строки это 0x51 или 81.
Вероятно, 81 это длина блока?
Длина файла 1658961, и она может быть поделена на 81 без остатка (и тогда там 20481 блоков).

Теперь я буду использовать Mathematica для анализа, есть ли тут повторяющиеся 81-байтные блоки в файле?
Я разделю входной файл на 81-байтные блоки и затем использую ф-цию
\emph{Tally[]}\footnote{\url{https://reference.wolfram.com/language/ref/Tally.html}}
которая просто считает, сколько раз каждый элемент встретился во входном списке.
Вывод Tally не отсортирован, так что я также добавлю ф-цию \emph{Sort[]} для сортировки его по кол-ву вхождений
в нисходящем порядке.

\begin{lstlisting}[style=custommath]
input = BinaryReadList["/home/dennis/.../castle.dbf"];

blocks = Partition[input, 81];

stat = Sort[Tally[blocks], #1[[2]] > #2[[2]] &]
\end{lstlisting}

И вот вывод:

\begin{lstlisting}[style=custommath]
{{{80, 103, 2, 116, 113, 102, 118, 25, 99, 8, 19, 23, 116, 125, 107, 
   25, 99, 109, 114, 102, 14, 121, 115, 31, 9, 117, 113, 111, 5, 4, 
   127, 28, 122, 101, 8, 110, 14, 18, 124, 106, 16, 20, 104, 119, 8, 
   109, 26, 106, 9, 97, 13, 99, 15, 119, 20, 105, 117, 98, 103, 118, 
   1, 126, 29, 97, 122, 17, 15, 114, 110, 3, 5, 125, 125, 99, 126, 
   119, 102, 30, 122, 2, 117}, 1739}, 
{{80, 100, 2, 116, 113, 102, 118, 25, 99, 8, 19, 23, 116, 
   125, 107, 25, 99, 109, 114, 102, 14, 121, 115, 31, 9, 117, 113, 
   111, 5, 4, 127, 28, 122, 101, 8, 110, 14, 18, 124, 106, 16, 20, 
   104, 119, 8, 109, 26, 106, 9, 97, 13, 99, 15, 119, 20, 105, 117, 
   98, 103, 118, 1, 126, 29, 97, 122, 17, 15, 114, 110, 3, 5, 125, 
   125, 99, 126, 119, 102, 30, 122, 2, 117}, 1422}, 
{{80, 101, 2, 116, 113, 102, 118, 25, 99, 8, 19, 23, 116, 
   125, 107, 25, 99, 109, 114, 102, 14, 121, 115, 31, 9, 117, 113, 
   111, 5, 4, 127, 28, 122, 101, 8, 110, 14, 18, 124, 106, 16, 20, 
   104, 119, 8, 109, 26, 106, 9, 97, 13, 99, 15, 119, 20, 105, 117, 
   98, 103, 118, 1, 126, 29, 97, 122, 17, 15, 114, 110, 3, 5, 125, 
   125, 99, 126, 119, 102, 30, 122, 2, 117}, 1012},
{{80, 120, 2, 116, 113, 102, 118, 25, 99, 8, 19, 23, 116, 
   125, 107, 25, 99, 109, 114, 102, 14, 121, 115, 31, 9, 117, 113, 
   111, 5, 4, 127, 28, 122, 101, 8, 110, 14, 18, 124, 106, 16, 20, 
   104, 119, 8, 109, 26, 106, 9, 97, 13, 99, 15, 119, 20, 105, 117, 
   98, 103, 118, 1, 126, 29, 97, 122, 17, 15, 114, 110, 3, 5, 125, 
   125, 99, 126, 119, 102, 30, 122, 2, 117}, 377},

...

{{80, 2, 74, 49, 113, 21, 62, 88, 39, 71, 68, 23, 63, 51, 36, 78, 48, 
   108, 114, 102, 14, 121, 115, 31, 9, 117, 113, 111, 5, 4, 127, 28, 
   122, 101, 8, 110, 14, 18, 124, 106, 16, 20, 104, 119, 8, 109, 26, 
   106, 9, 97, 13, 99, 15, 119, 20, 105, 117, 98, 103, 118, 1, 126, 
   29, 97, 122, 17, 15, 114, 110, 3, 5, 125, 125, 99, 126, 119, 102, 
   30, 122, 2, 117}, 1},
{{80, 1, 74, 59, 113, 45, 56, 86, 52, 91, 19, 64, 60, 60, 63, 
   25, 38, 59, 59, 42, 14, 53, 38, 77, 66, 38, 113, 38, 75, 4, 43, 84,
    63, 101, 64, 43, 79, 64, 40, 57, 16, 91, 46, 119, 69, 40, 84, 117,
    9, 97, 13, 99, 15, 119, 20, 105, 117, 98, 103, 118, 1, 126, 29, 
   97, 122, 17, 15, 114, 110, 3, 5, 125, 125, 99, 126, 119, 102, 30, 
   122, 2, 117}, 1},
{{80, 2, 74, 49, 113, 49, 51, 92, 39, 8, 92, 81, 116, 62, 57, 
   80, 46, 40, 114, 36, 75, 56, 33, 76, 9, 55, 56, 59, 81, 65, 45, 28,
    60, 55, 93, 39, 90, 28, 124, 106, 16, 20, 104, 119, 8, 109, 26, 
   106, 9, 97, 13, 99, 15, 119, 20, 105, 117, 98, 103, 118, 1, 126, 
   29, 97, 122, 17, 15, 114, 110, 3, 5, 125, 125, 99, 126, 119, 102, 
   30, 122, 2, 117}, 1}}
\end{lstlisting}

Вывод Tally это список пар, каждая пара это 81-байтный блок и количество раз, сколько он встретился в файле.
Мы видим, что наиболее частно встречающийся блок это первый, он встретился 1739 раз.
Второй встретился 1422 раза. Есть и другие: 1012 раза, 377 раз, итд.
81-байтные блоки, встреченные лишь один раз, находятся в конце вывода.

Попробуем сравнить эти блоки. Первый и второй.
Есть ли в Mathematica ф-ция для сравнения списков/массивов?
Наверняка есть, но в педагогических целях, я буду использоват операцию XOR для сравнения.
Действительно: если байты во входных массивах равны друг другу, результат операции XOR это 0.
Если не равны, результат будет ненулевой.

Сравним первый блок (встречается 1739 раз) и второй (встречается 1422 раз):

\begin{lstlisting}[style=custommath]
In[]:= BitXor[stat[[1]][[1]], stat[[2]][[1]]]
Out[]= {0, 3, 0, 0, 0, 0, 0, 0, 0, 0, 0, 0, 0, 0, 0, 0, 0, 0, 0, \
0, 0, 0, 0, 0, 0, 0, 0, 0, 0, 0, 0, 0, 0, 0, 0, 0, 0, 0, 0, 0, 0, 0, \
0, 0, 0, 0, 0, 0, 0, 0, 0, 0, 0, 0, 0, 0, 0, 0, 0, 0, 0, 0, 0, 0, 0, \
0, 0, 0, 0, 0, 0, 0, 0, 0, 0, 0, 0, 0, 0, 0, 0}
\end{lstlisting}

Они отличаются только вторым байтом.

Сравним второй блок (встречается 1422 раза) и третий (встречается 1012 раз):

\begin{lstlisting}[style=custommath]
In[]:= BitXor[stat[[2]][[1]], stat[[3]][[1]]]
Out[]= {0, 1, 0, 0, 0, 0, 0, 0, 0, 0, 0, 0, 0, 0, 0, 0, 0, 0, 0, \
0, 0, 0, 0, 0, 0, 0, 0, 0, 0, 0, 0, 0, 0, 0, 0, 0, 0, 0, 0, 0, 0, 0, \
0, 0, 0, 0, 0, 0, 0, 0, 0, 0, 0, 0, 0, 0, 0, 0, 0, 0, 0, 0, 0, 0, 0, \
0, 0, 0, 0, 0, 0, 0, 0, 0, 0, 0, 0, 0, 0, 0, 0}
\end{lstlisting}

Они тоже отличаются только вторым байтом.

Так или иначе, попробуем использовать самый встречающийся блок как XOR-ключ и попробуем расшифровать первые 4 81-байтных
блока в файле:

\begin{lstlisting}[style=custommath]
In[]:= key = stat[[1]][[1]]
Out[]= {80, 103, 2, 116, 113, 102, 118, 25, 99, 8, 19, 23, 116, \
125, 107, 25, 99, 109, 114, 102, 14, 121, 115, 31, 9, 117, 113, 111, \
5, 4, 127, 28, 122, 101, 8, 110, 14, 18, 124, 106, 16, 20, 104, 119, \
8, 109, 26, 106, 9, 97, 13, 99, 15, 119, 20, 105, 117, 98, 103, 118, \
1, 126, 29, 97, 122, 17, 15, 114, 110, 3, 5, 125, 125, 99, 126, 119, \
102, 30, 122, 2, 117}

In[]:= ToASCII[val_] := If[val == 0, " ", FromCharacterCode[val, "PrintableASCII"]]

In[]:= DecryptBlockASCII[blk_] := Map[ToASCII[#] &, BitXor[key, blk]]

In[]:= DecryptBlockASCII[blocks[[1]]]
Out[]= {" ", " ", " ", " ", " ", " ", " ", " ", " ", " ", " ", " \
", " ", " ", " ", " ", " ", " ", " ", " ", " ", " ", " ", " ", " ", " \
", " ", " ", " ", " ", " ", " ", " ", " ", " ", " ", " ", " ", " ", " \
", " ", " ", " ", " ", " ", " ", " ", " ", " ", " ", " ", " ", " ", " \
", " ", " ", " ", " ", " ", " ", " ", " ", " ", " ", " ", " ", " ", " \
", " ", " ", " ", " ", " ", " ", " ", " ", " ", " ", " ", " ", " "}

In[]:= DecryptBlockASCII[blocks[[2]]]
Out[]= {" ", "e", "H", "E", " ", "W", "E", "E", "D", " ", "O", \
"F", " ", "C", "R", "I", "M", "E", " ", "B", "E", "A", "R", "S", " ", \
"B", "I", "T", "T", "E", "R", " ", "F", "R", "U", "I", "T", "?", \
" ", " ", " ", " ", " ", " ", " ", " ", " ", " ", " ", " ", " ", " ", \
" ", " ", " ", " ", " ", " ", " ", " ", " ", " ", " ", " ", " ", " ", \
" ", " ", " ", " ", " ", " ", " ", " ", " ", " ", " ", " ", " ", " ", \
" "}

In[]:= DecryptBlockASCII[blocks[[3]]]
Out[]= {" ", "?", " ", " ", " ", " ", " ", " ", " ", " ", " \
", " ", " ", " ", " ", " ", " ", " ", " ", " ", " ", " ", " ", " ", " \
", " ", " ", " ", " ", " ", " ", " ", " ", " ", " ", " ", " ", " ", " \
", " ", " ", " ", " ", " ", " ", " ", " ", " ", " ", " ", " ", " ", " \
", " ", " ", " ", " ", " ", " ", " ", " ", " ", " ", " ", " ", " ", " \
", " ", " ", " ", " ", " ", " ", " ", " ", " ", " ", " ", " ", " ", " \
"}

In[]:= DecryptBlockASCII[blocks[[4]]]
Out[]= {" ", "f", "H", "O", " ", "K", "N", "O", "W", "S", " ", \
"W", "H", "A", "T", " ", "E", "V", "I", "L", " ", "L", "U", "R", "K", \
"S", " ", "I", "N", " ", "T", "H", "E", " ", "H", "E", "A", "R", "T", \
"S", " ", "O", "F", " ", "M", "E", "N", "?", " ", " ", " ", " ", \
" ", " ", " ", " ", " ", " ", " ", " ", " ", " ", " ", " ", " ", " ", \
" ", " ", " ", " ", " ", " ", " ", " ", " ", " ", " ", " ", " ", " ", \
" "}
\end{lstlisting}

(Я заменил непечатаемые символы на \q{?}.)

Мы видим что первый и третий блоки пустые (или почти пустые),
но второй и четвертый имеют ясно различимые английские слова/фразы.
Похоже что наше предположение насчет ключа верно (как минимум частично).
Это означает, что самый встречающийся 81-байтный блок в файле находится в местах лакун с нулевыми байтами
или что-то в этом роде.

Попробуем расшифровать весь файл:

\begin{lstlisting}[style=custommath]
DecryptBlock[blk_] := BitXor[key, blk]

decrypted = Map[DecryptBlock[#] &, blocks];

BinaryWrite["/home/dennis/.../tmp", Flatten[decrypted]]

Close["/home/dennis/.../tmp"]
\end{lstlisting}

\begin{figure}[H]
\centering
\myincludegraphics{ff/XOR/mask_1/mc_decrypted1.png}
\caption{Расшифрованный файл в Midnight Commander, первая попытка}
\end{figure}

Выглядит как английские фразы для какой-то игры, но что-то не так.
Прежде всего, регистр инвертирован: фразы и некоторые слова начинаются со строчных букв,
в то время как остальные буквы заглавные.
Также, некоторые фразы начинаются с не тех букв.
Посмотрите на самую первую фразу: \q{eHE WEED OF CRIME BEARS BITTER FRUIT}.
Что такое \q{eHE}? Разве не \q{tHE} тут должно быть?
Возможно ли что наш ключ для дешифрования имеет неверный байт в этом месте?

Посмотрим снова на второй блок в файле, на ключ и на результат дешифрования:

\begin{lstlisting}[style=custommath]
In[]:= blocks[[2]]
Out[]= {80, 2, 74, 49, 113, 49, 51, 92, 39, 8, 92, 81, 116, 62, \
57, 80, 46, 40, 114, 36, 75, 56, 33, 76, 9, 55, 56, 59, 81, 65, 45, \
28, 60, 55, 93, 39, 90, 28, 124, 106, 16, 20, 104, 119, 8, 109, 26, \
106, 9, 97, 13, 99, 15, 119, 20, 105, 117, 98, 103, 118, 1, 126, 29, \
97, 122, 17, 15, 114, 110, 3, 5, 125, 125, 99, 126, 119, 102, 30, \
122, 2, 117}

In[]:= key
Out[]= {80, 103, 2, 116, 113, 102, 118, 25, 99, 8, 19, 23, 116, \
125, 107, 25, 99, 109, 114, 102, 14, 121, 115, 31, 9, 117, 113, 111, \
5, 4, 127, 28, 122, 101, 8, 110, 14, 18, 124, 106, 16, 20, 104, 119, \
8, 109, 26, 106, 9, 97, 13, 99, 15, 119, 20, 105, 117, 98, 103, 118, \
1, 126, 29, 97, 122, 17, 15, 114, 110, 3, 5, 125, 125, 99, 126, 119, \
102, 30, 122, 2, 117}

In[]:= BitXor[key, blocks[[2]]]
Out[]= {0, 101, 72, 69, 0, 87, 69, 69, 68, 0, 79, 70, 0, 67, 82, \
73, 77, 69, 0, 66, 69, 65, 82, 83, 0, 66, 73, 84, 84, 69, 82, 0, 70, \
82, 85, 73, 84, 14, 0, 0, 0, 0, 0, 0, 0, 0, 0, 0, 0, 0, 0, 0, 0, 0, \
0, 0, 0, 0, 0, 0, 0, 0, 0, 0, 0, 0, 0, 0, 0, 0, 0, 0, 0, 0, 0, 0, 0, \
0, 0, 0, 0}
\end{lstlisting}

Зашифрованный байт это 2, байт из ключа это 103, $2 \oplus 103=101$ и 101 это ASCII-код символа \q{e}.
Чему должен равнятся этот байт ключа, чтобы ASCII-код был 116 (для символа  \q{t})?
$2 \oplus 116=118$, присвоим 118 второму байту в ключе \dots

\begin{lstlisting}[style=custommath]
key = {80, 118, 2, 116, 113, 102, 118, 25, 99, 8, 19, 23, 116, 125, 
  107, 25, 99, 109, 114, 102, 14, 121, 115, 31, 9, 117, 113, 111, 5, 
  4, 127, 28, 122, 101, 8, 110, 14, 18, 124, 106, 16, 20, 104, 119, 8,
   109, 26, 106, 9, 97, 13, 99, 15, 119, 20, 105, 117, 98, 103, 118, 
  1, 126, 29, 97, 122, 17, 15, 114, 110, 3, 5, 125, 125, 99, 126, 119,
   102, 30, 122, 2, 117}
\end{lstlisting}

\dots и снова дешифруем весь файл.

\begin{figure}[H]
\centering
\myincludegraphics{ff/XOR/mask_1/mc_decrypted2.png}
\caption{Дешифрованный файл в Midnight Commander, вторая попытка}
\end{figure}

Ух ты, теперь грамматика корректна, и все фразы начинаются с корректных букв.
Но все таки, регистр подозрителен.
С чего бы разработчику игры записывать их в такой манере?
Может быть наш ключ все еще неправилен?

% TODO ASCII table somewhere in the book
Изучая таблицу ASCII мы можем заметить что ASCII-коды для букв в верхнем и нижнем регистре отличаются только на один бит
(6-й бит, если считать с первого, 0b100000):

\begin{figure}[H]
\centering
\includegraphics[width=0.7\textwidth]{ascii.png}
\caption{7-битная таблица \ac{ASCII} в Emacs}
\end{figure}

6-й бит, выставленный в нулевом байте, В десятичном виде это будет 32.
Но 32 это ASCII-код пробела!

Действительно, можно менять регистр просто применяя XOR к ASCII-коду, с 32 (больше об этом: \myref{toupper_bit}).

Возможно ли, что пустые лакуны в файле это не нулевые байты, а скорее содержащие пробелы?
Еще раз модифицируем наш XOR-ключ (я про-XOR-ю каждый байт ключа с 32):

\begin{lstlisting}[style=custommath]
(* "32" это скаляр, и "key" это вектор, но это OK *)

In[]:= key3 = BitXor[32, key]
Out[]= {112, 86, 34, 84, 81, 70, 86, 57, 67, 40, 51, 55, 84, 93, 75, \
57, 67, 77, 82, 70, 46, 89, 83, 63, 41, 85, 81, 79, 37, 36, 95, 60, \
90, 69, 40, 78, 46, 50, 92, 74, 48, 52, 72, 87, 40, 77, 58, 74, 41, \
65, 45, 67, 47, 87, 52, 73, 85, 66, 71, 86, 33, 94, 61, 65, 90, 49, \
47, 82, 78, 35, 37, 93, 93, 67, 94, 87, 70, 62, 90, 34, 85}

In[]:= DecryptBlock[blk_] := BitXor[key3, blk]
\end{lstlisting}

И снова дешифруем входной файл:

\begin{figure}[H]
\centering
\myincludegraphics{ff/XOR/mask_1/mc_decrypted.png}
\caption{Дешифрованный файл в Midnight Commander, последняя попытка}
\end{figure}

(Расшифрованный файл доступен здесь:
\url{\GitHubBlobMasterURL/ff/XOR/mask_1/files/decrypted.dat.bz2}.)

Несомненно, это корректный исходный файл.
Да, и мы видим числа в начале каждого блока. Должно быть это и есть источник некорректного XOR-ключа.
Как выходит, самый встречающийся 81-байтный блок в файле это блок заполненный пробелами и содержащий символ \q{1} на месте
второго байта.
Действительно, как-то так получилось что многие блоки здесь перемежаются с этим блоком.
Может быть это что-то вроде выравнивания (padding) для коротких фраз/сообщений?
Другой часто встречающийся 81-байтный блок также заполнен пробелами, но с другой цифрой, следовательно,
они отличаются только вторым байтом.

Вот и всё! Теперь мы можем написать утилиту для зашифрования файла назад, и, может быть, модифицировать его перед этим

Файл для Mathematica можно скачать здесь:\\
\url{\GitHubBlobMasterURL/ff/XOR/mask_1/files/XOR_mask_1.nb}.

Итог: XOR-шифрование не надежно вообще. Вероятно, разработчик игры хотел просто скрыть внутренности игры от игрока,
ничего более серьезного.
Все же, шифрование вроде этого крайне популярно вследствии его простоты, так что многие реверс инженеры обычно хорошо
с этим знакомы.


\EN{\input{patterns/016_empty_redux/main_EN}}%
\FR{\input{patterns/016_empty_redux/main_FR}}


% I never liked this part:
% \input{examples/qr9/qr9_RU}

% TODO: OpenSSL tool, URLs, etc
\mysection{Случай с зашифрованной БД \#1}
\label{encrypted_DB1}

(Эта часть впервые появилась в моем блоге 26-Aug-2015.
Обсуждение: \url{https://news.ycombinator.com/item?id=10128684}.)

\subsection{Base64 и энтропия}

\myindex{XML}
Мне достался \ac{XML}-файл, содержащий некоторые зашифрованные данные.
Вероятно, что-то связанное с заказми и/или с информацией о клиентах.

\begin{lstlisting}
<?xml version = "1.0" encoding = "UTF-8"?>
<Orders>
	<Order>
		<OrderID>1</OrderID>
		<Data>yjmxhXUbhB/5MV45chPsXZWAJwIh1S0aD9lFn3XuJMSxJ3/E+UE3hsnH</Data>
	</Order>
	<Order>
		<OrderID>2</OrderID>
		<Data>0KGe/wnypFBjsy+U0C2P9fC5nDZP3XDZLMPCRaiBw9OjIk6Tu5U=</Data>
	</Order>
	<Order>
		<OrderID>3</OrderID>
		<Data>mqkXfdzvQKvEArdzh+zD9oETVGBFvcTBLs2ph1b5bYddExzp</Data>
	</Order>
	<Order>
		<OrderID>4</OrderID>
		<Data>FCx6JhIDqnESyT3HAepyE1BJ3cJd7wCk+APCRUeuNtZdpCvQ2MR/7kLXtfUHuA==</Data>
	</Order>
...
\end{lstlisting}

Файл доступен \href{https://raw.githubusercontent.com/DennisYurichev/RE-for-beginners/master/examples/encrypted_DB1/encrypted.xml}{здесь}.

\myindex{base64}
Это явно данные закодированные в base64, потому что все строки состоят из латинских символов, цифр,
и символов плюс (+) и слэш (/).
Могут быть еще два выравнивающих символа (=), но они никогда не встречаются в середине строки.
Зная эти свойства base64, такие строки легко распозновать.

Попробуем декодировать эти блоки и вычислить их энтропии (\myref{entropy}) при помощи Wolfram Mathematica:

\begin{lstlisting}
In[]:= ListOfBase64Strings =
  Map[First[#[[3]]] &, Cases[Import["encrypted.xml"], XMLElement["Data", _, _], Infinity]];

In[]:= BinaryStrings =
  Map[ImportString[#, {"Base64", "String"}] &, ListOfBase64Strings];

In[]:= Entropies = Map[N[Entropy[2, #]] &, BinaryStrings];

In[]:= Variance[Entropies]
Out[]= 0.0238614
\end{lstlisting}

\myindex{Variance}
Разброс (variance) низкий.
Это означает, что значения энтропии не очень отличаются друг от друга.
Это видно на графике:

\begin{lstlisting}
In[]:= ListPlot[Entropies]
\end{lstlisting}

\begin{figure}[H]
\centering
\myincludegraphics{examples/encrypted_DB1/entropy.png}
\end{figure}

Большинство значений между 5.0 и 5.4.
Это свидетельство того что данные сжаты и/или зашифрованы.

Чтобы понять разброс (variance), подсчитаем энтропии всех строк в книге Конана Дойля \emph{The Hound of the Baskervilles}:

\begin{lstlisting}
In[]:= BaskervillesLines = Import["http://www.gutenberg.org/cache/epub/2852/pg2852.txt", "List"];

In[]:= EntropiesT = Map[N[Entropy[2, #]] &, BaskervillesLines];

In[]:= Variance[EntropiesT]
Out[]= 2.73883

In[]:= ListPlot[EntropiesT]
\end{lstlisting}

\begin{figure}[H]
\centering
\myincludegraphics{examples/encrypted_DB1/conan_doyle.png}
\end{figure}

Большинство значений находится вокруг 4, но есть также м\'{е}ньшие значения, и они повлияли на конечное значение разброса.

Вероятно, самые короткие строки имеют м\'{е}ньшую энтропию, попробуем короткую строку из книги Конан Дойля:

\begin{lstlisting}
In[]:= Entropy[2, "Yes, sir."] // N
Out[]= 2.9477
\end{lstlisting}

Попробуем еще м\'{е}ньшую:

\begin{lstlisting}
In[]:= Entropy[2, "Yes"] // N
Out[]= 1.58496

In[]:= Entropy[2, "No"] // N
Out[]= 1.
\end{lstlisting}

\subsection{Данные сжаты?}

ОК, наши данные сжаты и/или зашифрованы.
Сжаты ли? Почти все компрессоры данных помещают некоторый заголовок в начале, сигнатуру или что-то вроде этого.
Как видим, здесь ничего такого нет в начале каждого блока.
Все еще возможно что это какой-то самодельный компрессор, но они очень редки.
С другой стороны, самодельные криптоалгоритмы попадаются часто, потому что их куда легче заставить работать.
\myindex{memfrob()}
\myindex{ROT13}
Даже примитивные криптосистемы без ключей, как \emph{memfrob()}\footnote{\url{http://linux.die.net/man/3/memfrob}}
и ROT13 нормально работают без ошибок.
А чтобы написать свой компрессор с нуля, используя только фантазию и воображение, так что он будет работать без ошибок,
это серьезная задача.
Некоторые программисты реализуют ф-ции сжатия данных по учебникам, но это также редкость.
Наиболее популярные способы это:
\myindex{zlib}
1) просто взять опен-сорсную библиотеку вроде zlib;
2) скопипастить что-то откуда-то.
Опен-сорсные алгоритмы сжатия данных обычно добавляют какой-то заголовок, и точно так же делают алгоритмы с сайтов вроде
\url{http://www.codeproject.com/}.

\subsection{Данные зашифрованы?}

Основные алгоритмы шифрования обрабатывают данные блоками. DES --- по 8 байт, AES --- по 16 байт.
Если входной буфер не делится без остатка на длину блока, он дополняется нулями (или еще чем-то), так что зашифрованные
данные будут выровнены по размеру блока этого алгоритма шифрования.
Это не наш случай.

Используя Wolfram Mathematica, я проанализировал длины блоков:

\begin{lstlisting}
In[]:= Counts[Map[StringLength[#] &, BinaryStrings]]
Out[]= <|42 -> 1858, 38 -> 1235, 36 -> 699, 46 -> 1151, 40 -> 1784,
 44 -> 1558, 50 -> 366, 34 -> 291, 32 -> 74, 56 -> 15, 48 -> 716,
 30 -> 13, 52 -> 156, 54 -> 71, 60 -> 3, 58 -> 6, 28 -> 4|>
\end{lstlisting}

1858 блоков имеют длину 42 байта, 1235 блоков имеют длину 38 байт, итд.

Я сделал график:

\begin{lstlisting}
ListPlot[Counts[Map[StringLength[#] &, BinaryStrings]]]
\end{lstlisting}

\begin{figure}[H]
\centering
\myincludegraphics{examples/encrypted_DB1/lengths.png}
\end{figure}

Так что большинство блоков имеют размер между $\textasciitilde{}36$ и $\textasciitilde{}48$.
Вот еще что стоит отметить: длины всех блоков четные.
Нет ни одного блока с нечетной длиной.

Хотя, существуют потоковые шифры, которые работают на уровне байт, или даже на уровне бит.

\subsection{CryptoPP}
\myindex{CryptoPP}

Программа, при помощи которой можно листать зашифрованную базу написана на C\# и код на .NET сильно обфусцирован.
Тем не менее, имеется DLL с кодом для x86, который, после краткого рассмотрения,
имеет части из популярной опен-сорсной библиотеки CryptoPP!
(Я просто нашел внутри строки \q{CryptoPP}.)
Теперь легко найти все ф-ции внутри DLL, потому что библиотека CryptoPP опен-сорсная.

\myindex{AES}
Библиотека CryptoPP имеет множество ф-ций шифрования, включая AES (AKA Rijndael).
Современные x86-процессоры имеют AES-инструкции вроде \INS{AESENC}, \INS{AESDEC} и \INS{AESKEYGENASSIST}
\footnote{\url{https://en.wikipedia.org/wiki/AES_instruction_set}}.
Они не производят полного шифрования/дешифрования, но они делают б\'{о}льшую часть работы.
И новые версии CryptoPP используют их.
Например, здесь:
\href{https://github.com/mmoss/cryptopp/blob/2772f7b57182b31a41659b48d5f35a7b6cedd34d/src/rijndael.cpp#L1034}{1},
\href{https://github.com/mmoss/cryptopp/blob/2772f7b57182b31a41659b48d5f35a7b6cedd34d/src/rijndael.cpp#L1000}{2}.
\myindex{x86!\Instructions!AESENC}
\myindex{x86!\Instructions!AESDEC}
\myindex{tracer}
К моему удивлению, во время дешифрования, инструкция, \INS{AESENC} исполняется, а \INS{AESDEC} --- нет
(я это проверил при помощи моей утилиты tracer, но можно использовать любой отладчик).
Я проверил, поддерживает ли мой процессор AES-инструкции. Некоторые процессоры Intel i3 не поддерживают.
И если нет, библиотека CryptoPP применяет ф-ции AES реализованные старым способом
\footnote{\url{https://github.com/mmoss/cryptopp/blob/2772f7b57182b31a41659b48d5f35a7b6cedd34d/src/rijndael.cpp#L355}}.
Но мой процессор поддерживает их.
Почему \INS{AESDEC} не исполняется?
Почему программа использует шифрование AES чтобы дешифровать БД?

ОК, найти ф-цию шифрования блока это не проблема.
Она называется \\
\emph{CryptoPP::Rijndael::Enc::ProcessAndXorBlock}:
\href{https://github.com/mmoss/cryptopp/blob/2772f7b57182b31a41659b48d5f35a7b6cedd34d/src/rijndael.cpp#L349}{src},
и она может вызывать другую ф-цию: \\
\emph{Rijndael::Enc::AdvancedProcessBlocks()}
\href{https://github.com/mmoss/cryptopp/blob/2772f7b57182b31a41659b48d5f35a7b6cedd34d/src/rijndael.cpp#L1179}{src},
которая, в свою очередь, может вызывать две ф-ции (
\href{https://github.com/mmoss/cryptopp/blob/2772f7b57182b31a41659b48d5f35a7b6cedd34d/src/rijndael.cpp#L1000}{AESNI\_Enc\_Block}
and
\href{https://github.com/mmoss/cryptopp/blob/2772f7b57182b31a41659b48d5f35a7b6cedd34d/src/rijndael.cpp#L1012}{AESNI\_Enc\_4\_Blocks}
)
которые имеют инструкции \INS{AESENC}.

Так что, судя по внутренностям CryptoPP, \\
\emph{CryptoPP::Rijndael::Enc::ProcessAndXorBlock()} шифрует один 16-байтный блок.
Попробуем установить брякпоинт на ней и посмотрим, что происходит во время дешифрования.
Я снова использую мою простую утилиту tracer.
Сейчас программа должна дешифровать первый блок.
О, и кстати, вот первый блок сконвертированный из кодировки base64 в шестнадцатеричный вид, будем держать его под рукой:

\lstinputlisting{examples/encrypted_DB1/1.lst}

А еще вот аргументы ф-ции из исходных файлов CryptoPP:

\begin{lstlisting}
size_t Rijndael::Enc::AdvancedProcessBlocks(const byte *inBlocks, const byte *xorBlocks, byte *outBlocks, size_t length, word32 flags);
\end{lstlisting}

Так что у него 5 аргументов. Возможные флаги это:

\begin{lstlisting}
enum {BT_InBlockIsCounter=1, BT_DontIncrementInOutPointers=2, BT_XorInput=4, BT_ReverseDirection=8, BT_AllowParallel=16} FlagsForAdvancedProcessBlocks;
\end{lstlisting}

ОК, запускаем tracer на ф-ции \emph{ProcessAndXorBlock()}:

\lstinputlisting{examples/encrypted_DB1/2.lst}

Тут мы можем увидить входы в ф-цию \emph{ProcessAndXorBlock()}, и выходы из нее.

Это вывод из ф-ции во время первого вызова:

\begin{lstlisting}
00000000: C7 39 4E 7B 33 1B D6 1F-B8 31 10 39 39 13 A5 5D ".9N{3....1.99..]"
\end{lstlisting}

Затем ф-ция \emph{ProcessAndXorBlock()} вызывается с блоком нулевого размера, но с флагом 8 (\emph{BT\_ReverseDirection}).

Второй вызов:

\begin{lstlisting}
00000000: 45 00 20 00 4A 00 4F 00-48 00 4E 00 53 00 00 00 "E. .J.O.H.N.S..."
\end{lstlisting}

Ох, тут есть знакомая нам строка!

Третий вызов:

\begin{lstlisting}
00000000: B1 27 7F E4 9F 01 E3 81-CF C6 12 FB B9 7C F1 BC ".'...........|.."
\end{lstlisting}

Первый вывод очень похож на первые 16 байт зашифрованного буфера.

Вывод первого вызова \emph{ProcessAndXorBlock()}:

\begin{lstlisting}
00000000: C7 39 4E 7B 33 1B D6 1F-B8 31 10 39 39 13 A5 5D ".9N{3....1.99..]"
\end{lstlisting}

Первые 16 байт зашифрованного буфера:

\begin{lstlisting}
00000000: CA 39 B1 85 75 1B 84 1F F9 31 5E 39 72 13 EC 5D  .9..u....1^9r..]
\end{lstlisting}

Тут слишком много одинаковых байт!
Как так получается, что результат шифрования AES может быть очень похож на шифрованный буфер в то время как это
не шифрование, а скорее дешифрование?

\subsection{Режим обратной связи по шифротексту}

\myindex{Режим обратной связи по шифротексту}
\myindex{XOR}
Ответ это \ac{CFB}:
в этом режиме, алгоритм AES используются не как алгоритм шифрования, а как устройство для генерации случайных данных
с криптографической стойкостью.
Само шифрование производится используя простую операцию XOR.

Вот алгоритм шифрования (иллюстрации взяты из Wikipedia):

\begin{figure}[H]
\centering
\myincludegraphics{examples/encrypted_DB1/601px-CFB_encryption.png}
\end{figure}

И дешифрования:

\begin{figure}[H]
\centering
\myincludegraphics{examples/encrypted_DB1/601px-CFB_decryption.png}
\label{fig:CFB_decryption}
\end{figure}

Посмотрим: операция шифрования в AES генерирует 16 байт (или 128 бит) \emph{случайных} данных,
которые можно использовать во время применения операции XOR, но кто заставляет нас использовать все 16 байт?
Если на последней итерации у нас 1 байт данных, давайте про-XOR-им 1 байт данных с 1 байтом сгенерированных
\emph{случайных} данных?
Это приводит к важному свойству режима \ac{CFB}: данные не нужно выравнивать, данные произвольного размера
могут быть зашифрованы и дешифрованы.

О, и вот почему все шифрованные блоки не выровнены.
И вот почему инструкция \INS{AESDEC} никогда не вызывается.

Давайте попробуем дешифровать первый блок вручную, используя Питон.
Режим \ac{CFB} также использует \ac{IV}, как \emph{seed} для \ac{CSPRNG}.
В нашем случае, \ac{IV} это блок, который шифруется на первой итерации:

\begin{lstlisting}
0038B920: 01 00 00 00 FF FF FF FF-79 C1 69 0B 67 C1 04 7D "........y.i.g..}"
\end{lstlisting}

О, и нам нужно также восстановить ключ шифрования.
\myindex{x86!\Instructions!AESKEYGENASSIST}
В DLL есть \INS{AESKEYGENASSIST}, и она вызывается, и используется в ф-ции \\
\emph{Rijndael::Base::UncheckedSetKey()}:
\href{https://github.com/mmoss/cryptopp/blob/2772f7b57182b31a41659b48d5f35a7b6cedd34d/src/rijndael.cpp#L198}{src}.
Её легко найти в IDA и установить брякпойнт. Посмотрим:

\begin{lstlisting}
... tracer.exe -l:filename.exe bpf=filename.exe!0x435c30,args:3,dump_args:0x10

Warning: no tracer.cfg file.
PID=2068|New process software.exe
no module registered with image base 0x77320000
no module registered with image base 0x76e20000
no module registered with image base 0x77320000
no module registered with image base 0x77220000
Warning: unknown (to us) INT3 breakpoint at ntdll.dll!LdrVerifyImageMatchesChecksum+0x96c (0x776c103b)
(0) software.exe!0x435c30(0x15e8000, 0x10, 0x14f808) (called from software.exe!.text+0x22fa1 (0x13d3fa1))
Argument 1/3
015E8000: CD C5 7E AD 28 5F 6D E1-CE 8F CC 29 B1 21 88 8E "..~.(_m....).!.."
Argument 3/3
0014F808: 38 82 58 01 C8 B9 46 00-01 D1 3C 01 00 F8 14 00 "8.X...F...<....."
Argument 3/3 +0x0: software.exe!.rdata+0x5238
Argument 3/3 +0x8: software.exe!.text+0x1c101
(0) software.exe!0x435c30() -> 0x13c2801
PID=2068|Process software.exe exited. ExitCode=0 (0x0)
\end{lstlisting}

Так вот это ключ: \emph{CD C5 7E AD 28 5F 6D E1-CE 8F CC 29 B1 21 88 8E}.

Во время ручного дешифрования мы получаем это:

\begin{lstlisting}
00000000: 0D 00 FF FE 46 00 52 00  41 00 4E 00 4B 00 49 00  ....F.R.A.N.K.I.
00000010: 45 00 20 00 4A 00 4F 00  48 00 4E 00 53 00 66 66  E. .J.O.H.N.S.ff
00000020: 66 66 66 9E 61 40 D4 07  06 01                    fff.a@....
\end{lstlisting}

Теперь это что-то читаемое!
И теперь мы видим, почему было так много одинаковых байт во время первой итерации дешифрования:
потому что в оригинальном тексте так много нулевых байт!

Дешифруем второй блок:

\begin{lstlisting}
00000000: 17 98 D0 84 3A E9 72 4F  DB 82 3F AD E9 3E 2A A8  ....:.rO..?..>*.
00000010: 41 00 52 00 52 00 4F 00  4E 00 CD CC CC CC CC CC  A.R.R.O.N.......
00000020: 1B 40 D4 07 06 01                                 .@....
\end{lstlisting}

Третий, четвертый и пятый:

\begin{lstlisting}
00000000: 5D 90 59 06 EF F4 96 B4  7C 33 A7 4A BE FF 66 AB  ].Y.....|3.J..f.
00000010: 49 00 47 00 47 00 53 00  00 00 00 00 00 C0 65 40  I.G.G.S.......e@
00000020: D4 07 06 01                                       ....
\end{lstlisting}

\begin{lstlisting}
00000000: D3 15 34 5D 21 18 7C 6E  AA F8 2D FE 38 F9 D7 4E  ..4]!.|n..-.8..N
00000010: 41 00 20 00 44 00 4F 00  48 00 45 00 52 00 54 00  A. .D.O.H.E.R.T.
00000020: 59 00 48 E1 7A 14 AE FF  68 40 D4 07 06 02        Y.H.z...h@....
\end{lstlisting}

\begin{lstlisting}
00000000: 1E 8B 90 0A 17 7B C5 52  31 6C 4E 2F DE 1B 27 19  .....{.R1lN...'.
00000010: 41 00 52 00 43 00 55 00  53 00 00 00 00 00 00 60  A.R.C.U.S.......
00000020: 66 40 D4 07 06 03                                 f@....
\end{lstlisting}

Все блоки, похоже, дешифруются корректно, но не первые 16 байт.

\subsection{Инициализирующий вектор}

Что влияет на первые 16 байт?

Вернемся снова к алгоритму дешифрования \ac{CFB}: \myref{fig:CFB_decryption}.

Мы видим что \ac{IV} может влиять на первую операцию дешифрования, но не на вторую,
потому что во время второй итерации используется шифротекст от первой итерации, и в случае дешифрования,
он такой же, не важно, какой был \ac{IV}!

Так что, вероятно, \ac{IV} каждый раз разный.
Используя мой tracer, я буду смотреть на первый вход во время дешифрования второго блока \ac{XML}-файла:

\begin{lstlisting}
0038B920: 02 00 00 00 FE FF FF FF-79 C1 69 0B 67 C1 04 7D "........y.i.g..}"
\end{lstlisting}

\dots third:

\begin{lstlisting}
0038B920: 03 00 00 00 FD FF FF FF-79 C1 69 0B 67 C1 04 7D "........y.i.g..}"
\end{lstlisting}

Похоже, первый и пятый байт каждый раз меняется.
Я в итоге разобрался, что первое 32-битное число это просто OrderID из \ac{XML}-файла,
и второе 32-битное число это тоже OrderID, но с отрицательным знаком. Остальные 8 байт не меняются.
И вот я расшифровал всю БД:
\url{https://raw.githubusercontent.com/DennisYurichev/RE-for-beginners/master/examples/encrypted_DB1/decrypted.full.txt}.

Питоновский скрипт, который я использовал:
\url{\GitHubBlobMasterURL/examples/encrypted_DB1/decrypt_blocks.py}.

Вероятно, автор хотел чтобы каждый блок шифровался немного иначе, так что он/она использовал OrderID как часть ключа.
А еще можно было бы делать разный ключ для AES вместо \ac{IV}.

Так что теперь мы знаем, что \ac{IV} влияет только на первый блок во время дешифрования в режиме \ac{CFB},
это его особенность.
Остальные блоки можно дешифровать не зная \ac{IV}, но используя ключ.

ОК, но почему режим \ac{CFB}? Очевидно, потому что самый первый пример на AES в CryptoPP wiki
использует режим \ac{CFB}:
\url{http://www.cryptopp.com/wiki/Advanced_Encryption_Standard#Encrypting_and_Decrypting_Using_AES}.
Вероятно, разработчик выбрал его из-за простоты:
пример может шифровать/дешифровать текстовые строки произвольной длины, без выравнивания.

Очень похоже что автор этой программы просто скопипастил пример из старницы в CryptoPP wiki.
Многие программисты так и делают.

Разница только в том что в примере в CryptoPP wiki \ac{IV} выбирается случайно, в то время как подобный индетерминизм
не был допустимым для автора программы, которую мы сейчас разбираем,
так что они решили инициализировать \ac{IV} используя ID заказа.

Теперь мы можем идти дальше, анализировать значение каждого байта дешифрованного блока.

\subsection{Структура буфера}

Возьмем первые 4 байта дешифрованных блоков:

\begin{lstlisting}
00000000: 0D 00 FF FE 46 00 52 00  41 00 4E 00 4B 00 49 00  ....F.R.A.N.K.I.
00000010: 45 00 20 00 4A 00 4F 00  48 00 4E 00 53 00 66 66  E. .J.O.H.N.S.ff
00000020: 66 66 66 9E 61 40 D4 07  06 01                    fff.a@....

00000000: 0B 00 FF FE 4C 00 4F 00  52 00 49 00 20 00 42 00  ....L.O.R.I. .B.
00000010: 41 00 52 00 52 00 4F 00  4E 00 CD CC CC CC CC CC  A.R.R.O.N.......
00000020: 1B 40 D4 07 06 01                                 .@....

00000000: 0A 00 FF FE 47 00 41 00  52 00 59 00 20 00 42 00  ....G.A.R.Y. .B.
00000010: 49 00 47 00 47 00 53 00  00 00 00 00 00 C0 65 40  I.G.G.S.......e@
00000020: D4 07 06 01                                       ....

00000000: 0F 00 FF FE 4D 00 45 00  4C 00 49 00 4E 00 44 00  ....M.E.L.I.N.D.
00000010: 41 00 20 00 44 00 4F 00  48 00 45 00 52 00 54 00  A. .D.O.H.E.R.T.
00000020: 59 00 48 E1 7A 14 AE FF  68 40 D4 07 06 02        Y.H.z...h@....
\end{lstlisting}

Легко увидеть строки закодированные в UTF-16, это имена и фамилии.
Первый байт (или 16-битное слово) похоже это просто длина строки, мы можем проверить это визуально.
\emph{FF FE} это, похоже, \ac{BOM} в Уникоде.

После каждой строки есть еще 12 байт.

Используя этот скрипт
(\url{\GitHubBlobMasterURL/examples/encrypted_DB1/dump_buffer_rest.py})
я получил случайную выборку из \emph{хвостов}:

\lstinputlisting{examples/encrypted_DB1/tails.lst}

Видим что байты 0x40 и 0x07 присутствуют в каждом \emph{хвосте}.
Самый последний байт всегда в пределах 1..0x1F (1..31), я проверил.
Предпоследний байт всегда в пределах 1..0xC (1..12).
Ух, это выглядит как дата!
Год может быть представлен как 16-битное значение, и может быть последние 4 байта это дата (16 бит для года, 8 бит для
месяца и еще 8 для дня)?
0x7DD это 2013, 0x7D5 это 2005, итд. Похоже нормально. Это дата.
Там есть еще 8 байт.
Судя по тому факту что БД называется \emph{orders} (заказы), может быть здесь присутствует сумма?
Я сделал попытку интерпретировать их как числа с плавающей точкой двойной точности в формате IEEE 754, и вывести все значения!

Некоторые:

\begin{lstlisting}
71.0
134.0
51.95
53.0
121.99
96.95
98.95
15.95
85.95
184.99
94.95
29.95
85.0
36.0
130.99
115.95
87.99
127.95
114.0
150.95
\end{lstlisting}

Похоже на правду!

Теперь мы можем вывест имена, суммы и даты.

\begin{lstlisting}
plain:
00000000: 0D 00 FF FE 46 00 52 00  41 00 4E 00 4B 00 49 00  ....F.R.A.N.K.I.
00000010: 45 00 20 00 4A 00 4F 00  48 00 4E 00 53 00 66 66  E. .J.O.H.N.S.ff
00000020: 66 66 66 9E 61 40 D4 07  06 01                    fff.a@....
OrderID= 1 name= FRANKIE JOHNS sum= 140.95 date= 2004 / 6 / 1

plain:
00000000: 0B 00 FF FE 4C 00 4F 00  52 00 49 00 20 00 42 00  ....L.O.R.I. .B.
00000010: 41 00 52 00 52 00 4F 00  4E 00 CD CC CC CC CC CC  A.R.R.O.N.......
00000020: 1B 40 D4 07 06 01                                 .@....
OrderID= 2 name= LORI BARRON sum= 6.95 date= 2004 / 6 / 1

plain:
00000000: 0A 00 FF FE 47 00 41 00  52 00 59 00 20 00 42 00  ....G.A.R.Y. .B.
00000010: 49 00 47 00 47 00 53 00  00 00 00 00 00 C0 65 40  I.G.G.S.......e@
00000020: D4 07 06 01                                       ....
OrderID= 3 name= GARY BIGGS sum= 174.0 date= 2004 / 6 / 1

plain:
00000000: 0F 00 FF FE 4D 00 45 00  4C 00 49 00 4E 00 44 00  ....M.E.L.I.N.D.
00000010: 41 00 20 00 44 00 4F 00  48 00 45 00 52 00 54 00  A. .D.O.H.E.R.T.
00000020: 59 00 48 E1 7A 14 AE FF  68 40 D4 07 06 02        Y.H.z...h@....
OrderID= 4 name= MELINDA DOHERTY sum= 199.99 date= 2004 / 6 / 2

plain:
00000000: 0B 00 FF FE 4C 00 45 00  4E 00 41 00 20 00 4D 00  ....L.E.N.A. .M.
00000010: 41 00 52 00 43 00 55 00  53 00 00 00 00 00 00 60  A.R.C.U.S.......
00000020: 66 40 D4 07 06 03                                 f@....
OrderID= 5 name= LENA MARCUS sum= 179.0 date= 2004 / 6 / 3
\end{lstlisting}

См. еще: \url{https://raw.githubusercontent.com/DennisYurichev/RE-for-beginners/master/examples/encrypted_DB1/decrypted.full.with_data.txt}.
Или отфильтрованные: \url{\GitHubBlobMasterURL/examples/encrypted_DB1/decrypted.short.txt}.
Похоже всё корректно.

Это что-то вроде сериализации в \ac{OOP}, т.е., запаковка значений с разными типами в бинарный буфер для хранения и/или
передачи.

\subsection{Шум в конце}

Остался только один вопрос, иногда \emph{хвост} длиннее:

\begin{lstlisting}
00000000: 0E 00 FF FE 54 00 48 00  45 00 52 00 45 00 53 00  ....T.H.E.R.E.S.
00000010: 45 00 20 00 54 00 55 00  54 00 54 00 4C 00 45 00  E. .T.U.T.T.L.E.
00000020: 66 66 66 66 66 1E 63 40  D4 07 07 1A 00 07 07 19  fffff.c@........
OrderID= 172 name= THERESE TUTTLE sum= 152.95 date= 2004 / 7 / 26
\end{lstlisting}

(Байты \emph{00 07 07 19} не используются и являются балластом.)

\begin{lstlisting}
00000000: 0C 00 FF FE 4D 00 45 00  4C 00 41 00 4E 00 49 00  ....M.E.L.A.N.I.
00000010: 45 00 20 00 4B 00 49 00  52 00 4B 00 00 00 00 00  E. .K.I.R.K.....
00000020: 00 20 64 40 D4 07 09 02  00 02                    . d@......
OrderID= 286 name= MELANIE KIRK sum= 161.0 date= 2004 / 9 / 2
\end{lstlisting}

(\emph{00 02} не используются.)

После близкого рассмотрения мы можем видеть, что шум в конце \emph{хвоста} просто остался от предыдущего
шифрования!

Вот два идущих подряд буфера:

\begin{lstlisting}
00000000: 10 00 FF FE 42 00 4F 00  4E 00 4E 00 49 00 45 00  ....B.O.N.N.I.E.
00000010: 20 00 47 00 4F 00 4C 00  44 00 53 00 54 00 45 00   .G.O.L.D.S.T.E.
00000020: 49 00 4E 00 9A 99 99 99  99 79 46 40 D4 07 07 19  I.N......yF@....
OrderID= 171 name= BONNIE GOLDSTEIN sum= 44.95 date= 2004 / 7 / 25

00000000: 0E 00 FF FE 54 00 48 00  45 00 52 00 45 00 53 00  ....T.H.E.R.E.S.
00000010: 45 00 20 00 54 00 55 00  54 00 54 00 4C 00 45 00  E. .T.U.T.T.L.E.
00000020: 66 66 66 66 66 1E 63 40  D4 07 07 1A 00 07 07 19  fffff.c@........
OrderID= 172 name= THERESE TUTTLE sum= 152.95 date= 2004 / 7 / 26
\end{lstlisting}

(Последние байты \emph{07 07 19} скопированы из предыдущего незашифрованного буфера.)

Еще два подряд идущих буфера:

\begin{lstlisting}
00000000: 0D 00 FF FE 4C 00 4F 00  52 00 45 00 4E 00 45 00  ....L.O.R.E.N.E.
00000010: 20 00 4F 00 54 00 4F 00  4F 00 4C 00 45 00 CD CC   .O.T.O.O.L.E...
00000020: CC CC CC 3C 5E 40 D4 07  09 02                    ...<^@....
OrderID= 285 name= LORENE OTOOLE sum= 120.95 date= 2004 / 9 / 2

00000000: 0C 00 FF FE 4D 00 45 00  4C 00 41 00 4E 00 49 00  ....M.E.L.A.N.I.
00000010: 45 00 20 00 4B 00 49 00  52 00 4B 00 00 00 00 00  E. .K.I.R.K.....
00000020: 00 20 64 40 D4 07 09 02  00 02                    . d@......
OrderID= 286 name= MELANIE KIRK sum= 161.0 date= 2004 / 9 / 2
\end{lstlisting}

Последний байт 02 был скопирован из предыдущего незашифрованного буфера.

Возможно, что буфер использующися для шифрования глобальный и/или не очищается перед каждым шифрованием.
Размер последнего буфера тоже как-то хаотично меняется, тем не менее, ошибка не была отловлена потому что она не влияет
на процесс дешифрования, который просто игнорирует шум в конце.
Эта распространенная ошибка.
\myindex{OpenSSL}
\myindex{Heartbleed}
Она была даже в OpenSSL (ошибка Heartbleed).

\subsection{Вывод}

Итог:
каждый практикующий реверс-инженер должен быть знаком с основными алгоритмами шифрования, а также с основными режимами
шифрования.
Некоторые книги об этом: \myref{crypto_books}.

\emph{Зашифрованное} содержимое БД было искусственно мною создано ради демонстрации.
Я использовал наиболее популярные имена и фамилии в США, отсюда: \url{http://stackoverflow.com/questions/1803628/raw-list-of-person-names},
и скомбинировал их случайным образом.
Даты и суммы были сгенерированы случайным образом.

Все файлы использованные в этой части здесь:
\url{\GitHubTreeMasterURL/examples/encrypted_DB1}.

Тем не менее, многие особенности как здесь, я наблюдал в настоящем ПО.
Этот пример основан на них.

\subsection{Post Scriptum: перебор всех \ac{IV}}

Пример, который вы только что видели, был искусственно создан, но основан на настоящем ПО которое я разбирал.
Когда я над ним работал, я в начале заметил, что \ac{IV} генерируется используя некоторые 32-битное число,
и я не мог найти связь между этим числом и OrderID.
Так что я приготовился использовать полный перебор, который тут действительно возможен.

Это не проблема перебрать все 32-битные значения и попробовать каждое как основу для \ac{IV}.
Затем вы дешифруете первый 16-байтный блок и проверяете нулевые байты, которые всегда находятся на одних и тех же местах.

\mysection{Разгон майнера биткоинов Cointerra}
\myindex{Bitcoin}
\myindex{BeagleBone}

Был такой майнер биткоинов Cointerra, выглядящий так:

\begin{figure}[H]
\centering
\myincludegraphics{examples/bitcoin_miner/board.jpg}
\caption{Board}
\end{figure}

И была также (возможно утекшая) утилита\footnote{Можно скачать здесь: \url{\RepoURL/examples/bitcoin_miner/files/cointool-overclock}}
которая могла выставлять тактовую частоту платы.
Она запускается на дополнительной плате BeagleBone на ARM с Linux (маленькая плата внизу фотографии).

И у автора (этих строк) однажды спросили, можно ли хакнуть эту утилиту и посмотреть, какие частоты можно выставлять, и какие нет.
И можно ли твикнуть её?

Утилиту нужно запускать так: \TT{./cointool-overclock 0 0 900}, где 900 это частота в МГц.
Если частота слишком большая, утилита выведет ошибку \q{Error with arguments} и закончит работу.

Вот фрагмент кода вокруг ссылки на текстовую строку \q{Error with arguments}:

\begin{lstlisting}[style=customasmARM]

...

.text:0000ABC4         STR      R3, [R11,#var_28]
.text:0000ABC8         MOV      R3, #optind
.text:0000ABD0         LDR      R3, [R3]
.text:0000ABD4         ADD      R3, R3, #1
.text:0000ABD8         MOV      R3, R3,LSL#2
.text:0000ABDC         LDR      R2, [R11,#argv]
.text:0000ABE0         ADD      R3, R2, R3
.text:0000ABE4         LDR      R3, [R3]
.text:0000ABE8         MOV      R0, R3  ; nptr
.text:0000ABEC         MOV      R1, #0  ; endptr
.text:0000ABF0         MOV      R2, #0  ; base
.text:0000ABF4         BL       strtoll
.text:0000ABF8         MOV      R2, R0
.text:0000ABFC         MOV      R3, R1
.text:0000AC00         MOV      R3, R2
.text:0000AC04         STR      R3, [R11,#var_2C]
.text:0000AC08         MOV      R3, #optind
.text:0000AC10         LDR      R3, [R3]
.text:0000AC14         ADD      R3, R3, #2
.text:0000AC18         MOV      R3, R3,LSL#2
.text:0000AC1C         LDR      R2, [R11,#argv]
.text:0000AC20         ADD      R3, R2, R3
.text:0000AC24         LDR      R3, [R3]
.text:0000AC28         MOV      R0, R3  ; nptr
.text:0000AC2C         MOV      R1, #0  ; endptr
.text:0000AC30         MOV      R2, #0  ; base
.text:0000AC34         BL       strtoll
.text:0000AC38         MOV      R2, R0
.text:0000AC3C         MOV      R3, R1
.text:0000AC40         MOV      R3, R2
.text:0000AC44         STR      R3, [R11,#third_argument]
.text:0000AC48         LDR      R3, [R11,#var_28]
.text:0000AC4C         CMP      R3, #0
.text:0000AC50         BLT      errors_with_arguments
.text:0000AC54         LDR      R3, [R11,#var_28]
.text:0000AC58         CMP      R3, #1
.text:0000AC5C         BGT      errors_with_arguments
.text:0000AC60         LDR      R3, [R11,#var_2C]
.text:0000AC64         CMP      R3, #0
.text:0000AC68         BLT      errors_with_arguments
.text:0000AC6C         LDR      R3, [R11,#var_2C]
.text:0000AC70         CMP      R3, #3
.text:0000AC74         BGT      errors_with_arguments
.text:0000AC78         LDR      R3, [R11,#third_argument]
.text:0000AC7C         CMP      R3, #0x31
.text:0000AC80         BLE      errors_with_arguments
.text:0000AC84         LDR      R2, [R11,#third_argument]
.text:0000AC88         MOV      R3, #950
.text:0000AC8C         CMP      R2, R3
.text:0000AC90         BGT      errors_with_arguments
.text:0000AC94         LDR      R2, [R11,#third_argument]
.text:0000AC98         MOV      R3, #0x51EB851F
.text:0000ACA0         SMULL    R1, R3, R3, R2
.text:0000ACA4         MOV      R1, R3,ASR#4
.text:0000ACA8         MOV      R3, R2,ASR#31
.text:0000ACAC         RSB      R3, R3, R1
.text:0000ACB0         MOV      R1, #50
.text:0000ACB4         MUL      R3, R1, R3
.text:0000ACB8         RSB      R3, R3, R2
.text:0000ACBC         CMP      R3, #0
.text:0000ACC0         BEQ      loc_ACEC
.text:0000ACC4
.text:0000ACC4 errors_with_arguments
.text:0000ACC4                                         
.text:0000ACC4         LDR      R3, [R11,#argv]
.text:0000ACC8         LDR      R3, [R3]
.text:0000ACCC         MOV      R0, R3  ; path
.text:0000ACD0         BL       __xpg_basename
.text:0000ACD4         MOV      R3, R0
.text:0000ACD8         MOV      R0, #aSErrorWithArgu ; format
.text:0000ACE0         MOV      R1, R3
.text:0000ACE4         BL       printf
.text:0000ACE8         B        loc_ADD4
.text:0000ACEC ; ------------------------------------------------------------
.text:0000ACEC
.text:0000ACEC loc_ACEC                 ; CODE XREF: main+66C
.text:0000ACEC         LDR      R2, [R11,#third_argument]
.text:0000ACF0         MOV      R3, #499
.text:0000ACF4         CMP      R2, R3
.text:0000ACF8         BGT      loc_AD08
.text:0000ACFC         MOV      R3, #0x64
.text:0000AD00         STR      R3, [R11,#unk_constant]
.text:0000AD04         B        jump_to_write_power
.text:0000AD08 ; ------------------------------------------------------------
.text:0000AD08
.text:0000AD08 loc_AD08                 ; CODE XREF: main+6A4
.text:0000AD08         LDR      R2, [R11,#third_argument]
.text:0000AD0C         MOV      R3, #799
.text:0000AD10         CMP      R2, R3
.text:0000AD14         BGT      loc_AD24
.text:0000AD18         MOV      R3, #0x5F
.text:0000AD1C         STR      R3, [R11,#unk_constant]
.text:0000AD20         B        jump_to_write_power
.text:0000AD24 ; ------------------------------------------------------------
.text:0000AD24
.text:0000AD24 loc_AD24                 ; CODE XREF: main+6C0
.text:0000AD24         LDR      R2, [R11,#third_argument]
.text:0000AD28         MOV      R3, #899
.text:0000AD2C         CMP      R2, R3
.text:0000AD30         BGT      loc_AD40
.text:0000AD34         MOV      R3, #0x5A
.text:0000AD38         STR      R3, [R11,#unk_constant]
.text:0000AD3C         B        jump_to_write_power
.text:0000AD40 ; ------------------------------------------------------------
.text:0000AD40
.text:0000AD40 loc_AD40                 ; CODE XREF: main+6DC
.text:0000AD40         LDR      R2, [R11,#third_argument]
.text:0000AD44         MOV      R3, #999
.text:0000AD48         CMP      R2, R3
.text:0000AD4C         BGT      loc_AD5C
.text:0000AD50         MOV      R3, #0x55
.text:0000AD54         STR      R3, [R11,#unk_constant]
.text:0000AD58         B        jump_to_write_power
.text:0000AD5C ; ------------------------------------------------------------
.text:0000AD5C
.text:0000AD5C loc_AD5C                 ; CODE XREF: main+6F8
.text:0000AD5C         LDR      R2, [R11,#third_argument]
.text:0000AD60         MOV      R3, #1099
.text:0000AD64         CMP      R2, R3
.text:0000AD68         BGT      jump_to_write_power
.text:0000AD6C         MOV      R3, #0x50
.text:0000AD70         STR      R3, [R11,#unk_constant]
.text:0000AD74
.text:0000AD74 jump_to_write_power                     ; CODE XREF: main+6B0
.text:0000AD74                                         ; main+6CC ...
.text:0000AD74         LDR      R3, [R11,#var_28]
.text:0000AD78         UXTB     R1, R3
.text:0000AD7C         LDR      R3, [R11,#var_2C]
.text:0000AD80         UXTB     R2, R3
.text:0000AD84         LDR      R3, [R11,#unk_constant]
.text:0000AD88         UXTB     R3, R3
.text:0000AD8C         LDR      R0, [R11,#third_argument]
.text:0000AD90         UXTH     R0, R0
.text:0000AD94         STR      R0, [SP,#0x44+var_44]
.text:0000AD98         LDR      R0, [R11,#var_24]
.text:0000AD9C         BL       write_power
.text:0000ADA0         LDR      R0, [R11,#var_24]
.text:0000ADA4         MOV      R1, #0x5A
.text:0000ADA8         BL       read_loop
.text:0000ADAC         B        loc_ADD4

...

.rodata:0000B378 aSErrorWithArgu DCB "%s: Error with arguments",0xA,0 ; DATA XREF: main+684

...

\end{lstlisting}

Имена ф-ций присутствовали в отладочной информации в оригинальном исполняемом файле,
такие как \TT{write\_power}, \TT{read\_loop}.
Но имена меткам внутри ф-ции дал я.

\myindex{UNIX!getopt}
\myindex{strtoll()}
Имя \TT{optind} звучит знакомо. Это библиотека \emph{getopt} из *NIX предназначенная для парсинга командной строки ---
и это то, что внутри и происходит.
Затем, третий аргумент (где передается значение частоты) конвертируется из строку в число используя вызов ф-ции \emph{strtoll()}.

Значение затем сравнивается с разными константами.
На 0xACEC есть проверка, меньше ли оно или равно 499, и если это так, то 0x64 будет передано в ф-цию
\TT{write\_power()} (которая посылает команду через USB используя \TT{send\_msg()}).
Если значение больше 499, происходит переход на 0xAD08.

На 0xAD08 есть проверка, меньше ли оно или равно 799. Если это так, то 0x5F передается в ф-цию \TT{write\_power()}.

Есть еще проверки: на 899 на 0xAD24, на 0x999 на 0xAD40, и наконец, на 1099 на 0xAD5C.
Если входная частота меньше или равна 1099, 0x50 (на 0xAD6C) будет передано в ф-цию \TT{write\_power()}.
И тут что-то вроде баги.
Если значение все еще больше 1099, само значение будет передано в ф-цию \TT{write\_power()}.
Но с другой стороны это не бага, потому что мы не можем попасть сюда: значение в начале проверяется с 950 на 0xAC88,
и если оно больше, выводится сообщение об ошибке и утилита заканчивает работу.

Вот таблица между частотами в МГц и значениями передаваемыми в ф-цию \TT{write\_power()}:

\begin{center}
\begin{longtable}{ | l | l | l | }
\hline
\HeaderColor МГц & \HeaderColor шестнадцатеричное представление & \HeaderColor десятичное \\
\hline
499MHz & 0x64 & 100 \\
\hline
799MHz & 0x5f & 95 \\
\hline
899MHz & 0x5a & 90 \\
\hline
999MHz & 0x55 & 85 \\
\hline
1099MHz & 0x50 & 80 \\
\hline
\end{longtable}
\end{center}

Как видно, значение передаваемое в плату постепенно уменьшается с ростом частоты.

Видно что значение в 950МГц это жесткий предел, по крайней мере в этой утилите. Можно ли её обмануть?

Вернемся к этому фрагменту кода:

\begin{lstlisting}[style=customasmARM]
.text:0000AC84      LDR     R2, [R11,#third_argument]
.text:0000AC88      MOV     R3, #950
.text:0000AC8C      CMP     R2, R3
.text:0000AC90      BGT     errors_with_arguments ; Я пропатчил здесь на 00 00 00 00
\end{lstlisting}

Нам нужно как-то запретить инструкцию перехода \INS{BGT} на 0xAC90. И это ARM в режиме ARM, потому что, как мы видим,
все адреса увеличиваются на 4, т.е., длина каждой инструкции это 4 байта.
Инструкция \TT{NOP} (нет операции) в режиме ARM это просто 4 нулевых байта: \TT{00 00 00 00}.
Так что, записывая 4 нуля по адресу 0xAC90 (или по физическому смещению в файле: 0x2C90) мы можем выключить
эту проверку.

Теперь можно выставлять частоты вплоть до 1050МГц. И даже больше, но из-за ошибки, если входное значение больше 1099,
значение в МГц, \emph{как есть}, будет передано в плату, что неправильно.

Дальше я не разбирался, но если бы продолжил, я бы уменьшал значение передаваемое в ф-цию \TT{write\_power()}.

Теперь страшный фрагмент кода, который я в начале пропустил:

\lstinputlisting[style=customasmARM]{examples/bitcoin_miner/tmp1.lst}

Здесь используется деление через умножение, и константа 0x51EB851F.
Я написал для себя простой программистский калькулятор\footnote{\url{https://github.com/DennisYurichev/progcalc}}.
И там есть возможность вычислять обратное число по модулю.

\begin{lstlisting}
modinv32(0x51EB851F)
Warning, result is not integer: 3.125000
(unsigned) dec: 3 hex: 0x3 bin: 11
\end{lstlisting}

Это значит что инструкция \INS{SMULL} на 0xACA0 просто делит 3-й аргумент на 3.125.
На самом деле, все что делает ф-ция \TT{modinv32()} в моем калькуляторе, это:

\[
\frac{1}{\frac{input}{2^{32}}} = \frac{2^{32}}{input}
\]

Потом там есть дополнительные сдвиги и теперь мы видим что 3-й аргумент просто делится на 50.
И затем умножается снова на 50.
Зачем?
Это простейшая проверка, можно ли делить входное значение на 50 без остатка.
Если значение этого выражения ненулевое, $x$ не может быть разделено на 50 без остатка:

\[
x-((\frac{x}{50}) \cdot 50)
\]

На самом деле, это простой способ вычисления остатка от деления.

И затем, если остаток ненулевой, выводится сообщение об ошибке.
Так что эта утилита берет значения частотв вроде 850, 900, 950, 1000, итд, но не 855 или 911.

Вот и всё! Если вы делаете что-то такое, имейте ввиду, что это может испортить вашу плату, как и в случае разгона
чипов вроде \ac{CPU}, \ac{GPU}, итд.
Если у вас есть плата Cointerra, делайте всё это на свой собственный риск!


%\subsection{Простое шифрование используя XOR-маску}
\label{XOR_mask_1}

Я нашел одну старую игру в стиле interactive fiction в архиве \emph{if-archive}\footnote{\url{http://www.ifarchive.org/}}:

\begin{lstlisting}
The New Castle v3.5 - Text/Adventure Game
in the style of the original Infocom (tm)
type games, Zork, Collosal Cave (Adventure),
etc.  Can you solve the mystery of the
abandoned castle?
Shareware from Software Customization.
Software Customization [ASP] Version 3.5 Feb. 2000
\end{lstlisting}

Можно скачать здесь: \url{\GitHubBlobMasterURL/ff/XOR/mask_1/files/newcastle.tgz}.

Там внутри есть файл (с названием \emph{castle.dbf}), который явно зашифрован, но не настоящим криптоалгоритмом,
и оне сжат, это что-то куда проще.
Я бы даже не стал измерять уровень энтропии (\myref{entropy}) этого файла, потому что я итак уверен, что он низкий.
Вот как он выглядит в Midnight Commander:

\begin{figure}[H]
\centering
\myincludegraphics{ff/XOR/mask_1/mc_encrypted.png}
\caption{Зашифрованный файл в Midnight Commander}
\end{figure}

Зашифрованный файл можно скачать здесь:
\url{\GitHubBlobMasterURL/ff/XOR/mask_1/files/castle.dbf.bz2}.

Можно ли расшифровать его без доступа к программе, используя просто этот файл?

Тут явно просматривается повторяющаяся строка. 
Если использовалось простое шифрование с XOR-маской, такие повторяющиеся строки это явное свидетельство,
потому что, вероятно, тут были длинные лакуны с нулевыми байтами, которые, в свою очередь, присутствуют
во мноигих исполняемых файлах, и в остальных бинарных файлах.

\myindex{UNIX!xxd}
Вот дам начала этого файла используя утилиту \emph{xxd} из UNIX:

\lstinputlisting{ff/XOR/mask_1/xxd_result.txt}

Давайте держаться за повторяющуюся строку \TT{iubgv}.
Глядя на этот дамп, мы можем легко увидеть, что период повторений этой строки это 0x51 или 81.
Вероятно, 81 это длина блока?
Длина файла 1658961, и она может быть поделена на 81 без остатка (и тогда там 20481 блоков).

Теперь я буду использовать Mathematica для анализа, есть ли тут повторяющиеся 81-байтные блоки в файле?
Я разделю входной файл на 81-байтные блоки и затем использую ф-цию
\emph{Tally[]}\footnote{\url{https://reference.wolfram.com/language/ref/Tally.html}}
которая просто считает, сколько раз каждый элемент встретился во входном списке.
Вывод Tally не отсортирован, так что я также добавлю ф-цию \emph{Sort[]} для сортировки его по кол-ву вхождений
в нисходящем порядке.

\begin{lstlisting}[style=custommath]
input = BinaryReadList["/home/dennis/.../castle.dbf"];

blocks = Partition[input, 81];

stat = Sort[Tally[blocks], #1[[2]] > #2[[2]] &]
\end{lstlisting}

И вот вывод:

\begin{lstlisting}[style=custommath]
{{{80, 103, 2, 116, 113, 102, 118, 25, 99, 8, 19, 23, 116, 125, 107, 
   25, 99, 109, 114, 102, 14, 121, 115, 31, 9, 117, 113, 111, 5, 4, 
   127, 28, 122, 101, 8, 110, 14, 18, 124, 106, 16, 20, 104, 119, 8, 
   109, 26, 106, 9, 97, 13, 99, 15, 119, 20, 105, 117, 98, 103, 118, 
   1, 126, 29, 97, 122, 17, 15, 114, 110, 3, 5, 125, 125, 99, 126, 
   119, 102, 30, 122, 2, 117}, 1739}, 
{{80, 100, 2, 116, 113, 102, 118, 25, 99, 8, 19, 23, 116, 
   125, 107, 25, 99, 109, 114, 102, 14, 121, 115, 31, 9, 117, 113, 
   111, 5, 4, 127, 28, 122, 101, 8, 110, 14, 18, 124, 106, 16, 20, 
   104, 119, 8, 109, 26, 106, 9, 97, 13, 99, 15, 119, 20, 105, 117, 
   98, 103, 118, 1, 126, 29, 97, 122, 17, 15, 114, 110, 3, 5, 125, 
   125, 99, 126, 119, 102, 30, 122, 2, 117}, 1422}, 
{{80, 101, 2, 116, 113, 102, 118, 25, 99, 8, 19, 23, 116, 
   125, 107, 25, 99, 109, 114, 102, 14, 121, 115, 31, 9, 117, 113, 
   111, 5, 4, 127, 28, 122, 101, 8, 110, 14, 18, 124, 106, 16, 20, 
   104, 119, 8, 109, 26, 106, 9, 97, 13, 99, 15, 119, 20, 105, 117, 
   98, 103, 118, 1, 126, 29, 97, 122, 17, 15, 114, 110, 3, 5, 125, 
   125, 99, 126, 119, 102, 30, 122, 2, 117}, 1012},
{{80, 120, 2, 116, 113, 102, 118, 25, 99, 8, 19, 23, 116, 
   125, 107, 25, 99, 109, 114, 102, 14, 121, 115, 31, 9, 117, 113, 
   111, 5, 4, 127, 28, 122, 101, 8, 110, 14, 18, 124, 106, 16, 20, 
   104, 119, 8, 109, 26, 106, 9, 97, 13, 99, 15, 119, 20, 105, 117, 
   98, 103, 118, 1, 126, 29, 97, 122, 17, 15, 114, 110, 3, 5, 125, 
   125, 99, 126, 119, 102, 30, 122, 2, 117}, 377},

...

{{80, 2, 74, 49, 113, 21, 62, 88, 39, 71, 68, 23, 63, 51, 36, 78, 48, 
   108, 114, 102, 14, 121, 115, 31, 9, 117, 113, 111, 5, 4, 127, 28, 
   122, 101, 8, 110, 14, 18, 124, 106, 16, 20, 104, 119, 8, 109, 26, 
   106, 9, 97, 13, 99, 15, 119, 20, 105, 117, 98, 103, 118, 1, 126, 
   29, 97, 122, 17, 15, 114, 110, 3, 5, 125, 125, 99, 126, 119, 102, 
   30, 122, 2, 117}, 1},
{{80, 1, 74, 59, 113, 45, 56, 86, 52, 91, 19, 64, 60, 60, 63, 
   25, 38, 59, 59, 42, 14, 53, 38, 77, 66, 38, 113, 38, 75, 4, 43, 84,
    63, 101, 64, 43, 79, 64, 40, 57, 16, 91, 46, 119, 69, 40, 84, 117,
    9, 97, 13, 99, 15, 119, 20, 105, 117, 98, 103, 118, 1, 126, 29, 
   97, 122, 17, 15, 114, 110, 3, 5, 125, 125, 99, 126, 119, 102, 30, 
   122, 2, 117}, 1},
{{80, 2, 74, 49, 113, 49, 51, 92, 39, 8, 92, 81, 116, 62, 57, 
   80, 46, 40, 114, 36, 75, 56, 33, 76, 9, 55, 56, 59, 81, 65, 45, 28,
    60, 55, 93, 39, 90, 28, 124, 106, 16, 20, 104, 119, 8, 109, 26, 
   106, 9, 97, 13, 99, 15, 119, 20, 105, 117, 98, 103, 118, 1, 126, 
   29, 97, 122, 17, 15, 114, 110, 3, 5, 125, 125, 99, 126, 119, 102, 
   30, 122, 2, 117}, 1}}
\end{lstlisting}

Вывод Tally это список пар, каждая пара это 81-байтный блок и количество раз, сколько он встретился в файле.
Мы видим, что наиболее частно встречающийся блок это первый, он встретился 1739 раз.
Второй встретился 1422 раза. Есть и другие: 1012 раза, 377 раз, итд.
81-байтные блоки, встреченные лишь один раз, находятся в конце вывода.

Попробуем сравнить эти блоки. Первый и второй.
Есть ли в Mathematica ф-ция для сравнения списков/массивов?
Наверняка есть, но в педагогических целях, я буду использоват операцию XOR для сравнения.
Действительно: если байты во входных массивах равны друг другу, результат операции XOR это 0.
Если не равны, результат будет ненулевой.

Сравним первый блок (встречается 1739 раз) и второй (встречается 1422 раз):

\begin{lstlisting}[style=custommath]
In[]:= BitXor[stat[[1]][[1]], stat[[2]][[1]]]
Out[]= {0, 3, 0, 0, 0, 0, 0, 0, 0, 0, 0, 0, 0, 0, 0, 0, 0, 0, 0, \
0, 0, 0, 0, 0, 0, 0, 0, 0, 0, 0, 0, 0, 0, 0, 0, 0, 0, 0, 0, 0, 0, 0, \
0, 0, 0, 0, 0, 0, 0, 0, 0, 0, 0, 0, 0, 0, 0, 0, 0, 0, 0, 0, 0, 0, 0, \
0, 0, 0, 0, 0, 0, 0, 0, 0, 0, 0, 0, 0, 0, 0, 0}
\end{lstlisting}

Они отличаются только вторым байтом.

Сравним второй блок (встречается 1422 раза) и третий (встречается 1012 раз):

\begin{lstlisting}[style=custommath]
In[]:= BitXor[stat[[2]][[1]], stat[[3]][[1]]]
Out[]= {0, 1, 0, 0, 0, 0, 0, 0, 0, 0, 0, 0, 0, 0, 0, 0, 0, 0, 0, \
0, 0, 0, 0, 0, 0, 0, 0, 0, 0, 0, 0, 0, 0, 0, 0, 0, 0, 0, 0, 0, 0, 0, \
0, 0, 0, 0, 0, 0, 0, 0, 0, 0, 0, 0, 0, 0, 0, 0, 0, 0, 0, 0, 0, 0, 0, \
0, 0, 0, 0, 0, 0, 0, 0, 0, 0, 0, 0, 0, 0, 0, 0}
\end{lstlisting}

Они тоже отличаются только вторым байтом.

Так или иначе, попробуем использовать самый встречающийся блок как XOR-ключ и попробуем расшифровать первые 4 81-байтных
блока в файле:

\begin{lstlisting}[style=custommath]
In[]:= key = stat[[1]][[1]]
Out[]= {80, 103, 2, 116, 113, 102, 118, 25, 99, 8, 19, 23, 116, \
125, 107, 25, 99, 109, 114, 102, 14, 121, 115, 31, 9, 117, 113, 111, \
5, 4, 127, 28, 122, 101, 8, 110, 14, 18, 124, 106, 16, 20, 104, 119, \
8, 109, 26, 106, 9, 97, 13, 99, 15, 119, 20, 105, 117, 98, 103, 118, \
1, 126, 29, 97, 122, 17, 15, 114, 110, 3, 5, 125, 125, 99, 126, 119, \
102, 30, 122, 2, 117}

In[]:= ToASCII[val_] := If[val == 0, " ", FromCharacterCode[val, "PrintableASCII"]]

In[]:= DecryptBlockASCII[blk_] := Map[ToASCII[#] &, BitXor[key, blk]]

In[]:= DecryptBlockASCII[blocks[[1]]]
Out[]= {" ", " ", " ", " ", " ", " ", " ", " ", " ", " ", " ", " \
", " ", " ", " ", " ", " ", " ", " ", " ", " ", " ", " ", " ", " ", " \
", " ", " ", " ", " ", " ", " ", " ", " ", " ", " ", " ", " ", " ", " \
", " ", " ", " ", " ", " ", " ", " ", " ", " ", " ", " ", " ", " ", " \
", " ", " ", " ", " ", " ", " ", " ", " ", " ", " ", " ", " ", " ", " \
", " ", " ", " ", " ", " ", " ", " ", " ", " ", " ", " ", " ", " "}

In[]:= DecryptBlockASCII[blocks[[2]]]
Out[]= {" ", "e", "H", "E", " ", "W", "E", "E", "D", " ", "O", \
"F", " ", "C", "R", "I", "M", "E", " ", "B", "E", "A", "R", "S", " ", \
"B", "I", "T", "T", "E", "R", " ", "F", "R", "U", "I", "T", "?", \
" ", " ", " ", " ", " ", " ", " ", " ", " ", " ", " ", " ", " ", " ", \
" ", " ", " ", " ", " ", " ", " ", " ", " ", " ", " ", " ", " ", " ", \
" ", " ", " ", " ", " ", " ", " ", " ", " ", " ", " ", " ", " ", " ", \
" "}

In[]:= DecryptBlockASCII[blocks[[3]]]
Out[]= {" ", "?", " ", " ", " ", " ", " ", " ", " ", " ", " \
", " ", " ", " ", " ", " ", " ", " ", " ", " ", " ", " ", " ", " ", " \
", " ", " ", " ", " ", " ", " ", " ", " ", " ", " ", " ", " ", " ", " \
", " ", " ", " ", " ", " ", " ", " ", " ", " ", " ", " ", " ", " ", " \
", " ", " ", " ", " ", " ", " ", " ", " ", " ", " ", " ", " ", " ", " \
", " ", " ", " ", " ", " ", " ", " ", " ", " ", " ", " ", " ", " ", " \
"}

In[]:= DecryptBlockASCII[blocks[[4]]]
Out[]= {" ", "f", "H", "O", " ", "K", "N", "O", "W", "S", " ", \
"W", "H", "A", "T", " ", "E", "V", "I", "L", " ", "L", "U", "R", "K", \
"S", " ", "I", "N", " ", "T", "H", "E", " ", "H", "E", "A", "R", "T", \
"S", " ", "O", "F", " ", "M", "E", "N", "?", " ", " ", " ", " ", \
" ", " ", " ", " ", " ", " ", " ", " ", " ", " ", " ", " ", " ", " ", \
" ", " ", " ", " ", " ", " ", " ", " ", " ", " ", " ", " ", " ", " ", \
" "}
\end{lstlisting}

(Я заменил непечатаемые символы на \q{?}.)

Мы видим что первый и третий блоки пустые (или почти пустые),
но второй и четвертый имеют ясно различимые английские слова/фразы.
Похоже что наше предположение насчет ключа верно (как минимум частично).
Это означает, что самый встречающийся 81-байтный блок в файле находится в местах лакун с нулевыми байтами
или что-то в этом роде.

Попробуем расшифровать весь файл:

\begin{lstlisting}[style=custommath]
DecryptBlock[blk_] := BitXor[key, blk]

decrypted = Map[DecryptBlock[#] &, blocks];

BinaryWrite["/home/dennis/.../tmp", Flatten[decrypted]]

Close["/home/dennis/.../tmp"]
\end{lstlisting}

\begin{figure}[H]
\centering
\myincludegraphics{ff/XOR/mask_1/mc_decrypted1.png}
\caption{Расшифрованный файл в Midnight Commander, первая попытка}
\end{figure}

Выглядит как английские фразы для какой-то игры, но что-то не так.
Прежде всего, регистр инвертирован: фразы и некоторые слова начинаются со строчных букв,
в то время как остальные буквы заглавные.
Также, некоторые фразы начинаются с не тех букв.
Посмотрите на самую первую фразу: \q{eHE WEED OF CRIME BEARS BITTER FRUIT}.
Что такое \q{eHE}? Разве не \q{tHE} тут должно быть?
Возможно ли что наш ключ для дешифрования имеет неверный байт в этом месте?

Посмотрим снова на второй блок в файле, на ключ и на результат дешифрования:

\begin{lstlisting}[style=custommath]
In[]:= blocks[[2]]
Out[]= {80, 2, 74, 49, 113, 49, 51, 92, 39, 8, 92, 81, 116, 62, \
57, 80, 46, 40, 114, 36, 75, 56, 33, 76, 9, 55, 56, 59, 81, 65, 45, \
28, 60, 55, 93, 39, 90, 28, 124, 106, 16, 20, 104, 119, 8, 109, 26, \
106, 9, 97, 13, 99, 15, 119, 20, 105, 117, 98, 103, 118, 1, 126, 29, \
97, 122, 17, 15, 114, 110, 3, 5, 125, 125, 99, 126, 119, 102, 30, \
122, 2, 117}

In[]:= key
Out[]= {80, 103, 2, 116, 113, 102, 118, 25, 99, 8, 19, 23, 116, \
125, 107, 25, 99, 109, 114, 102, 14, 121, 115, 31, 9, 117, 113, 111, \
5, 4, 127, 28, 122, 101, 8, 110, 14, 18, 124, 106, 16, 20, 104, 119, \
8, 109, 26, 106, 9, 97, 13, 99, 15, 119, 20, 105, 117, 98, 103, 118, \
1, 126, 29, 97, 122, 17, 15, 114, 110, 3, 5, 125, 125, 99, 126, 119, \
102, 30, 122, 2, 117}

In[]:= BitXor[key, blocks[[2]]]
Out[]= {0, 101, 72, 69, 0, 87, 69, 69, 68, 0, 79, 70, 0, 67, 82, \
73, 77, 69, 0, 66, 69, 65, 82, 83, 0, 66, 73, 84, 84, 69, 82, 0, 70, \
82, 85, 73, 84, 14, 0, 0, 0, 0, 0, 0, 0, 0, 0, 0, 0, 0, 0, 0, 0, 0, \
0, 0, 0, 0, 0, 0, 0, 0, 0, 0, 0, 0, 0, 0, 0, 0, 0, 0, 0, 0, 0, 0, 0, \
0, 0, 0, 0}
\end{lstlisting}

Зашифрованный байт это 2, байт из ключа это 103, $2 \oplus 103=101$ и 101 это ASCII-код символа \q{e}.
Чему должен равнятся этот байт ключа, чтобы ASCII-код был 116 (для символа  \q{t})?
$2 \oplus 116=118$, присвоим 118 второму байту в ключе \dots

\begin{lstlisting}[style=custommath]
key = {80, 118, 2, 116, 113, 102, 118, 25, 99, 8, 19, 23, 116, 125, 
  107, 25, 99, 109, 114, 102, 14, 121, 115, 31, 9, 117, 113, 111, 5, 
  4, 127, 28, 122, 101, 8, 110, 14, 18, 124, 106, 16, 20, 104, 119, 8,
   109, 26, 106, 9, 97, 13, 99, 15, 119, 20, 105, 117, 98, 103, 118, 
  1, 126, 29, 97, 122, 17, 15, 114, 110, 3, 5, 125, 125, 99, 126, 119,
   102, 30, 122, 2, 117}
\end{lstlisting}

\dots и снова дешифруем весь файл.

\begin{figure}[H]
\centering
\myincludegraphics{ff/XOR/mask_1/mc_decrypted2.png}
\caption{Дешифрованный файл в Midnight Commander, вторая попытка}
\end{figure}

Ух ты, теперь грамматика корректна, и все фразы начинаются с корректных букв.
Но все таки, регистр подозрителен.
С чего бы разработчику игры записывать их в такой манере?
Может быть наш ключ все еще неправилен?

% TODO ASCII table somewhere in the book
Изучая таблицу ASCII мы можем заметить что ASCII-коды для букв в верхнем и нижнем регистре отличаются только на один бит
(6-й бит, если считать с первого, 0b100000):

\begin{figure}[H]
\centering
\includegraphics[width=0.7\textwidth]{ascii.png}
\caption{7-битная таблица \ac{ASCII} в Emacs}
\end{figure}

6-й бит, выставленный в нулевом байте, В десятичном виде это будет 32.
Но 32 это ASCII-код пробела!

Действительно, можно менять регистр просто применяя XOR к ASCII-коду, с 32 (больше об этом: \myref{toupper_bit}).

Возможно ли, что пустые лакуны в файле это не нулевые байты, а скорее содержащие пробелы?
Еще раз модифицируем наш XOR-ключ (я про-XOR-ю каждый байт ключа с 32):

\begin{lstlisting}[style=custommath]
(* "32" это скаляр, и "key" это вектор, но это OK *)

In[]:= key3 = BitXor[32, key]
Out[]= {112, 86, 34, 84, 81, 70, 86, 57, 67, 40, 51, 55, 84, 93, 75, \
57, 67, 77, 82, 70, 46, 89, 83, 63, 41, 85, 81, 79, 37, 36, 95, 60, \
90, 69, 40, 78, 46, 50, 92, 74, 48, 52, 72, 87, 40, 77, 58, 74, 41, \
65, 45, 67, 47, 87, 52, 73, 85, 66, 71, 86, 33, 94, 61, 65, 90, 49, \
47, 82, 78, 35, 37, 93, 93, 67, 94, 87, 70, 62, 90, 34, 85}

In[]:= DecryptBlock[blk_] := BitXor[key3, blk]
\end{lstlisting}

И снова дешифруем входной файл:

\begin{figure}[H]
\centering
\myincludegraphics{ff/XOR/mask_1/mc_decrypted.png}
\caption{Дешифрованный файл в Midnight Commander, последняя попытка}
\end{figure}

(Расшифрованный файл доступен здесь:
\url{\GitHubBlobMasterURL/ff/XOR/mask_1/files/decrypted.dat.bz2}.)

Несомненно, это корректный исходный файл.
Да, и мы видим числа в начале каждого блока. Должно быть это и есть источник некорректного XOR-ключа.
Как выходит, самый встречающийся 81-байтный блок в файле это блок заполненный пробелами и содержащий символ \q{1} на месте
второго байта.
Действительно, как-то так получилось что многие блоки здесь перемежаются с этим блоком.
Может быть это что-то вроде выравнивания (padding) для коротких фраз/сообщений?
Другой часто встречающийся 81-байтный блок также заполнен пробелами, но с другой цифрой, следовательно,
они отличаются только вторым байтом.

Вот и всё! Теперь мы можем написать утилиту для зашифрования файла назад, и, может быть, модифицировать его перед этим

Файл для Mathematica можно скачать здесь:\\
\url{\GitHubBlobMasterURL/ff/XOR/mask_1/files/XOR_mask_1.nb}.

Итог: XOR-шифрование не надежно вообще. Вероятно, разработчик игры хотел просто скрыть внутренности игры от игрока,
ничего более серьезного.
Все же, шифрование вроде этого крайне популярно вследствии его простоты, так что многие реверс инженеры обычно хорошо
с этим знакомы.


\EN{\input{patterns/016_empty_redux/main_EN}}%
\FR{\input{patterns/016_empty_redux/main_FR}}

\EN{\input{patterns/016_empty_redux/main_EN}}%
\FR{\input{patterns/016_empty_redux/main_FR}}

\mysection{Вручную написанный на ассемблере код}

\subsection{Тестовый файл EICAR}
\label{subsec:EICAR}

\myindex{MS-DOS}
\myindex{EICAR}
Этот .COM-файл предназначен для тестирования антивирусов, его можно запустить в MS-DOS
и он выведет такую строку: \q{EICAR-STANDARD-ANTIVIRUS-TEST-FILE!}.
% FIXME1 \myref{} -> about .COM files

Он примечателен тем, что он полностью состоит только из печатных ASCII-символов, следовательно, его можно
набрать в любом текстовом редакторе:

\begin{lstlisting}
X5O!P%@AP[4\PZX54(P^)7CC)7}$EICAR-STANDARD-ANTIVIRUS-TEST-FILE!$H+H*
\end{lstlisting}

Попробуем его разобрать:

\lstinputlisting[style=customasmx86]{examples/handcoding/EICAR_RU.lst}

Добавим везде комментарии, показывающие состояние регистров и стека после каждой инструкции.

Собственно, все эти инструкции нужны только для того чтобы исполнить следующий код:

\begin{lstlisting}[style=customasmx86]
B4 09     MOV AH, 9
BA 1C 01  MOV DX, 11Ch
CD 21     INT 21h
CD 20     INT 20h
\end{lstlisting}

\myindex{x86!\Instructions!INT}
\TT{INT 21h} с функцией 9 (переданной в \TT{AH}) просто выводит строку, адрес которой передан в \TT{DS:DX}.
Кстати, строка должна быть завершена символом '\$'.
Надо полагать, это наследие \gls{CP/M} 
и эта функция в DOS осталась для совместимости.
\TT{INT 20h} возвращает управление в DOS.

Но, как видно, далеко не все опкоды этих инструкций печатные.
Так что основная часть EICAR-файла это:

\begin{itemize}
\item подготовка нужных значений регистров (AH и DX);
\item подготовка в памяти опкодов для INT 21 и INT 20;
\item исполнение INT 21 и INT 20.
\end{itemize}

\myindex{Shellcode}
Кстати, подобная техника широко используется для создания шеллкодов, 
где нужно создать x86-код, который будет нужно передать в виде текстовой строки.

Здесь также список всех x86-инструкций с печатаемыми опкодами: \myref{printable_x86_opcodes}.

\EN{\input{patterns/016_empty_redux/main_EN}}%
\FR{\input{patterns/016_empty_redux/main_FR}}

\mysection{Как я переписывал 100 килобайт x86-кода на чистый Си}

То была DLL-ка с секцией кода ~100 килобайт, она брала на вход многоканальный сигнал и выдавала другой многоканальный сигнал.
Там много всего было связано с обработкой сигналов.
Внутри было очень много FPU-кода.
Написано по-олдскульному, так, как писали в то время, когда передача параметров через аргументы ф-ций была дорогой,
и потому использовалось много глобальных переменных и массивов, почти всё хранилось в них, а ф-ции, напротив, имели сравнительно
мало аргументов, если вообще.
Функции большие, их было около ста.

Тесты были, много.

\myindex{Hex-Rays}
Проблема была в том, что функции слишком большие и Hex-Rays неизменно выдавал немного неверный код.
Нужно было очень внимательно всё чистить вручную.
В процессе работы, я нашел в нем каких-то ошибок: \ref{hex_rays}.

Все 100 ф-ций декомпилировать сразу нельзя --- где-то будут ошибки, тесты не пройдут, и где вы будете искать эти ошибки?
Приходится переписывать по чуть-чуть.

В DLL-ке есть некая корневая ф-ция, скажем, \verb|ProcessMain()|.
Я переписываю её на Си при помощи Hex-Rays, она запускается из обычного .exe-процесса.
Все ф-ции из DLL-ки, которые вызываются далее, у меня вызывались через указатели на ф-ции.
DLL-ка загружена, и пока они все там.

\ac{ASLR} отключил, и DLL-ка каждый раз грузится по одному и тому же адресу, потому и адреса всех ф-ций одни и те же.
Важно, что и адреса глобальных массивов тоже одни и те же

Затем переписываю ф-ции, вызывающиеся непосредственно из ProcessMain(), затем еще ниже, итд.
Таким образом, ф-ции я постепенно перетаскивал из DLL в свою .exe.
Каждый раз тестируя.

Много раз бывало и так --- ф-ция слишком большая, например, несколько килобайт x86-кода, и после декомпиляции в Си,
там что-то косячит, и неизвестно где.
Из IDA я экспортировал её листинг в текст на ассемблере и компилировал при помощи обычного ассемблера (ML в MSVC).
Она компилируется в .obj-файл и прикомпилируется к главной .exe, и пока всё ОК.
Затем я делил эту ф-цию на более мелкие, здорово пригодился (когда бы еще?) опыт написания программ на чистом ассемблере в середине
90-х (руки до сих пор помнят).
Если всё работает, более мелкие ф-ции постепенно переписывал на Си при помощи Hex-Rays, в то время как "головная" ф-ция более
высокого уровня всё еще на ассемблере.

Интересно, что было много глобальных массивов, но границы между ними были сильно размыты.
Но я вижу что есть какой-то большой кусок в секции \verb|.data|, где лежит всё подряд.
Дошел до стадии, когда на Си переписано уже всё, а все обращения к массивам происходят по адресам внутри секции .data в 
подгружаемой DLL-ке, впрочем, там почти не было констант.
Затем, чтобы совсем отказаться от DLL-ки, я сделал большой глобальный "кусок" уже у себя на Си, и вся работа с массивами
шла через мой "кусок", при том, что все массивы всё еще не были отделены друг от друга.

Вот реальный фрагмент оттуда, как было в начале.
Значение --- это адрес в .data-секции в DLL-ке:

\begin{lstlisting}
int *a_val511=0x1002B588;
int *a_val483=0x1002B590;
int *a_val481=0x1002B5B8;
int *a_val515=0x1002B6E4;
...
\end{lstlisting}

И все обращения происходят через указатели.

Потом я сделал "кусок":

\begin{lstlisting}
char lump[0x1000000];

/* 0x1002B588 */int *a_val511=(int*)&lump[0x2B588];
/* 0x1002B590 */int *a_val483=(int*)&lump[0x2B590];
/* 0x1002B5B8 */int *a_val481=(int*)&lump[0x2B5B8];
/* 0x1002B6E4 */int *a_val515=(int*)&lump[0x2B6E4];
...
\end{lstlisting}

DLL-ку теперь можно было наконец-то отцепить и разбираться с границами массивов.
\myindex{Pin}
Этот процесс я хотел немного автоматизировать и использовал для этого Pin.
Я написал утилиту, которая показывала, по каким адресам в глобальном "куске" были обращения
из каждого адреса. Точнее, в каких пределах?
Так стало проще видеть границы массивов.

\myindex{Z3}
\myindex{Mathematica}
"На войне все средства хороши", так что я доходил и до того, что использовал Mathematica и Z3 для сокращения слишком длинных
выражений (Hex-Rays не всё может оптимизировать): \\
\url{https://github.com/DennisYurichev/SAT_SMT_by_example/blob/master/proofs/simplify_EN.tex}.

Очень хорошим тестом было пересобрать всё под Linux при помощи GCC и заставить работать --- как всегда, это было нелегко.
Плюс, чтобы работало корректно и под x86 и под x64.


\subsection{Простое шифрование используя XOR-маску}
\label{XOR_mask_1}

Я нашел одну старую игру в стиле interactive fiction в архиве \emph{if-archive}\footnote{\url{http://www.ifarchive.org/}}:

\begin{lstlisting}
The New Castle v3.5 - Text/Adventure Game
in the style of the original Infocom (tm)
type games, Zork, Collosal Cave (Adventure),
etc.  Can you solve the mystery of the
abandoned castle?
Shareware from Software Customization.
Software Customization [ASP] Version 3.5 Feb. 2000
\end{lstlisting}

Можно скачать здесь: \url{\GitHubBlobMasterURL/ff/XOR/mask_1/files/newcastle.tgz}.

Там внутри есть файл (с названием \emph{castle.dbf}), который явно зашифрован, но не настоящим криптоалгоритмом,
и оне сжат, это что-то куда проще.
Я бы даже не стал измерять уровень энтропии (\myref{entropy}) этого файла, потому что я итак уверен, что он низкий.
Вот как он выглядит в Midnight Commander:

\begin{figure}[H]
\centering
\myincludegraphics{ff/XOR/mask_1/mc_encrypted.png}
\caption{Зашифрованный файл в Midnight Commander}
\end{figure}

Зашифрованный файл можно скачать здесь:
\url{\GitHubBlobMasterURL/ff/XOR/mask_1/files/castle.dbf.bz2}.

Можно ли расшифровать его без доступа к программе, используя просто этот файл?

Тут явно просматривается повторяющаяся строка. 
Если использовалось простое шифрование с XOR-маской, такие повторяющиеся строки это явное свидетельство,
потому что, вероятно, тут были длинные лакуны с нулевыми байтами, которые, в свою очередь, присутствуют
во мноигих исполняемых файлах, и в остальных бинарных файлах.

\myindex{UNIX!xxd}
Вот дам начала этого файла используя утилиту \emph{xxd} из UNIX:

\lstinputlisting{ff/XOR/mask_1/xxd_result.txt}

Давайте держаться за повторяющуюся строку \TT{iubgv}.
Глядя на этот дамп, мы можем легко увидеть, что период повторений этой строки это 0x51 или 81.
Вероятно, 81 это длина блока?
Длина файла 1658961, и она может быть поделена на 81 без остатка (и тогда там 20481 блоков).

Теперь я буду использовать Mathematica для анализа, есть ли тут повторяющиеся 81-байтные блоки в файле?
Я разделю входной файл на 81-байтные блоки и затем использую ф-цию
\emph{Tally[]}\footnote{\url{https://reference.wolfram.com/language/ref/Tally.html}}
которая просто считает, сколько раз каждый элемент встретился во входном списке.
Вывод Tally не отсортирован, так что я также добавлю ф-цию \emph{Sort[]} для сортировки его по кол-ву вхождений
в нисходящем порядке.

\begin{lstlisting}[style=custommath]
input = BinaryReadList["/home/dennis/.../castle.dbf"];

blocks = Partition[input, 81];

stat = Sort[Tally[blocks], #1[[2]] > #2[[2]] &]
\end{lstlisting}

И вот вывод:

\begin{lstlisting}[style=custommath]
{{{80, 103, 2, 116, 113, 102, 118, 25, 99, 8, 19, 23, 116, 125, 107, 
   25, 99, 109, 114, 102, 14, 121, 115, 31, 9, 117, 113, 111, 5, 4, 
   127, 28, 122, 101, 8, 110, 14, 18, 124, 106, 16, 20, 104, 119, 8, 
   109, 26, 106, 9, 97, 13, 99, 15, 119, 20, 105, 117, 98, 103, 118, 
   1, 126, 29, 97, 122, 17, 15, 114, 110, 3, 5, 125, 125, 99, 126, 
   119, 102, 30, 122, 2, 117}, 1739}, 
{{80, 100, 2, 116, 113, 102, 118, 25, 99, 8, 19, 23, 116, 
   125, 107, 25, 99, 109, 114, 102, 14, 121, 115, 31, 9, 117, 113, 
   111, 5, 4, 127, 28, 122, 101, 8, 110, 14, 18, 124, 106, 16, 20, 
   104, 119, 8, 109, 26, 106, 9, 97, 13, 99, 15, 119, 20, 105, 117, 
   98, 103, 118, 1, 126, 29, 97, 122, 17, 15, 114, 110, 3, 5, 125, 
   125, 99, 126, 119, 102, 30, 122, 2, 117}, 1422}, 
{{80, 101, 2, 116, 113, 102, 118, 25, 99, 8, 19, 23, 116, 
   125, 107, 25, 99, 109, 114, 102, 14, 121, 115, 31, 9, 117, 113, 
   111, 5, 4, 127, 28, 122, 101, 8, 110, 14, 18, 124, 106, 16, 20, 
   104, 119, 8, 109, 26, 106, 9, 97, 13, 99, 15, 119, 20, 105, 117, 
   98, 103, 118, 1, 126, 29, 97, 122, 17, 15, 114, 110, 3, 5, 125, 
   125, 99, 126, 119, 102, 30, 122, 2, 117}, 1012},
{{80, 120, 2, 116, 113, 102, 118, 25, 99, 8, 19, 23, 116, 
   125, 107, 25, 99, 109, 114, 102, 14, 121, 115, 31, 9, 117, 113, 
   111, 5, 4, 127, 28, 122, 101, 8, 110, 14, 18, 124, 106, 16, 20, 
   104, 119, 8, 109, 26, 106, 9, 97, 13, 99, 15, 119, 20, 105, 117, 
   98, 103, 118, 1, 126, 29, 97, 122, 17, 15, 114, 110, 3, 5, 125, 
   125, 99, 126, 119, 102, 30, 122, 2, 117}, 377},

...

{{80, 2, 74, 49, 113, 21, 62, 88, 39, 71, 68, 23, 63, 51, 36, 78, 48, 
   108, 114, 102, 14, 121, 115, 31, 9, 117, 113, 111, 5, 4, 127, 28, 
   122, 101, 8, 110, 14, 18, 124, 106, 16, 20, 104, 119, 8, 109, 26, 
   106, 9, 97, 13, 99, 15, 119, 20, 105, 117, 98, 103, 118, 1, 126, 
   29, 97, 122, 17, 15, 114, 110, 3, 5, 125, 125, 99, 126, 119, 102, 
   30, 122, 2, 117}, 1},
{{80, 1, 74, 59, 113, 45, 56, 86, 52, 91, 19, 64, 60, 60, 63, 
   25, 38, 59, 59, 42, 14, 53, 38, 77, 66, 38, 113, 38, 75, 4, 43, 84,
    63, 101, 64, 43, 79, 64, 40, 57, 16, 91, 46, 119, 69, 40, 84, 117,
    9, 97, 13, 99, 15, 119, 20, 105, 117, 98, 103, 118, 1, 126, 29, 
   97, 122, 17, 15, 114, 110, 3, 5, 125, 125, 99, 126, 119, 102, 30, 
   122, 2, 117}, 1},
{{80, 2, 74, 49, 113, 49, 51, 92, 39, 8, 92, 81, 116, 62, 57, 
   80, 46, 40, 114, 36, 75, 56, 33, 76, 9, 55, 56, 59, 81, 65, 45, 28,
    60, 55, 93, 39, 90, 28, 124, 106, 16, 20, 104, 119, 8, 109, 26, 
   106, 9, 97, 13, 99, 15, 119, 20, 105, 117, 98, 103, 118, 1, 126, 
   29, 97, 122, 17, 15, 114, 110, 3, 5, 125, 125, 99, 126, 119, 102, 
   30, 122, 2, 117}, 1}}
\end{lstlisting}

Вывод Tally это список пар, каждая пара это 81-байтный блок и количество раз, сколько он встретился в файле.
Мы видим, что наиболее частно встречающийся блок это первый, он встретился 1739 раз.
Второй встретился 1422 раза. Есть и другие: 1012 раза, 377 раз, итд.
81-байтные блоки, встреченные лишь один раз, находятся в конце вывода.

Попробуем сравнить эти блоки. Первый и второй.
Есть ли в Mathematica ф-ция для сравнения списков/массивов?
Наверняка есть, но в педагогических целях, я буду использоват операцию XOR для сравнения.
Действительно: если байты во входных массивах равны друг другу, результат операции XOR это 0.
Если не равны, результат будет ненулевой.

Сравним первый блок (встречается 1739 раз) и второй (встречается 1422 раз):

\begin{lstlisting}[style=custommath]
In[]:= BitXor[stat[[1]][[1]], stat[[2]][[1]]]
Out[]= {0, 3, 0, 0, 0, 0, 0, 0, 0, 0, 0, 0, 0, 0, 0, 0, 0, 0, 0, \
0, 0, 0, 0, 0, 0, 0, 0, 0, 0, 0, 0, 0, 0, 0, 0, 0, 0, 0, 0, 0, 0, 0, \
0, 0, 0, 0, 0, 0, 0, 0, 0, 0, 0, 0, 0, 0, 0, 0, 0, 0, 0, 0, 0, 0, 0, \
0, 0, 0, 0, 0, 0, 0, 0, 0, 0, 0, 0, 0, 0, 0, 0}
\end{lstlisting}

Они отличаются только вторым байтом.

Сравним второй блок (встречается 1422 раза) и третий (встречается 1012 раз):

\begin{lstlisting}[style=custommath]
In[]:= BitXor[stat[[2]][[1]], stat[[3]][[1]]]
Out[]= {0, 1, 0, 0, 0, 0, 0, 0, 0, 0, 0, 0, 0, 0, 0, 0, 0, 0, 0, \
0, 0, 0, 0, 0, 0, 0, 0, 0, 0, 0, 0, 0, 0, 0, 0, 0, 0, 0, 0, 0, 0, 0, \
0, 0, 0, 0, 0, 0, 0, 0, 0, 0, 0, 0, 0, 0, 0, 0, 0, 0, 0, 0, 0, 0, 0, \
0, 0, 0, 0, 0, 0, 0, 0, 0, 0, 0, 0, 0, 0, 0, 0}
\end{lstlisting}

Они тоже отличаются только вторым байтом.

Так или иначе, попробуем использовать самый встречающийся блок как XOR-ключ и попробуем расшифровать первые 4 81-байтных
блока в файле:

\begin{lstlisting}[style=custommath]
In[]:= key = stat[[1]][[1]]
Out[]= {80, 103, 2, 116, 113, 102, 118, 25, 99, 8, 19, 23, 116, \
125, 107, 25, 99, 109, 114, 102, 14, 121, 115, 31, 9, 117, 113, 111, \
5, 4, 127, 28, 122, 101, 8, 110, 14, 18, 124, 106, 16, 20, 104, 119, \
8, 109, 26, 106, 9, 97, 13, 99, 15, 119, 20, 105, 117, 98, 103, 118, \
1, 126, 29, 97, 122, 17, 15, 114, 110, 3, 5, 125, 125, 99, 126, 119, \
102, 30, 122, 2, 117}

In[]:= ToASCII[val_] := If[val == 0, " ", FromCharacterCode[val, "PrintableASCII"]]

In[]:= DecryptBlockASCII[blk_] := Map[ToASCII[#] &, BitXor[key, blk]]

In[]:= DecryptBlockASCII[blocks[[1]]]
Out[]= {" ", " ", " ", " ", " ", " ", " ", " ", " ", " ", " ", " \
", " ", " ", " ", " ", " ", " ", " ", " ", " ", " ", " ", " ", " ", " \
", " ", " ", " ", " ", " ", " ", " ", " ", " ", " ", " ", " ", " ", " \
", " ", " ", " ", " ", " ", " ", " ", " ", " ", " ", " ", " ", " ", " \
", " ", " ", " ", " ", " ", " ", " ", " ", " ", " ", " ", " ", " ", " \
", " ", " ", " ", " ", " ", " ", " ", " ", " ", " ", " ", " ", " "}

In[]:= DecryptBlockASCII[blocks[[2]]]
Out[]= {" ", "e", "H", "E", " ", "W", "E", "E", "D", " ", "O", \
"F", " ", "C", "R", "I", "M", "E", " ", "B", "E", "A", "R", "S", " ", \
"B", "I", "T", "T", "E", "R", " ", "F", "R", "U", "I", "T", "?", \
" ", " ", " ", " ", " ", " ", " ", " ", " ", " ", " ", " ", " ", " ", \
" ", " ", " ", " ", " ", " ", " ", " ", " ", " ", " ", " ", " ", " ", \
" ", " ", " ", " ", " ", " ", " ", " ", " ", " ", " ", " ", " ", " ", \
" "}

In[]:= DecryptBlockASCII[blocks[[3]]]
Out[]= {" ", "?", " ", " ", " ", " ", " ", " ", " ", " ", " \
", " ", " ", " ", " ", " ", " ", " ", " ", " ", " ", " ", " ", " ", " \
", " ", " ", " ", " ", " ", " ", " ", " ", " ", " ", " ", " ", " ", " \
", " ", " ", " ", " ", " ", " ", " ", " ", " ", " ", " ", " ", " ", " \
", " ", " ", " ", " ", " ", " ", " ", " ", " ", " ", " ", " ", " ", " \
", " ", " ", " ", " ", " ", " ", " ", " ", " ", " ", " ", " ", " ", " \
"}

In[]:= DecryptBlockASCII[blocks[[4]]]
Out[]= {" ", "f", "H", "O", " ", "K", "N", "O", "W", "S", " ", \
"W", "H", "A", "T", " ", "E", "V", "I", "L", " ", "L", "U", "R", "K", \
"S", " ", "I", "N", " ", "T", "H", "E", " ", "H", "E", "A", "R", "T", \
"S", " ", "O", "F", " ", "M", "E", "N", "?", " ", " ", " ", " ", \
" ", " ", " ", " ", " ", " ", " ", " ", " ", " ", " ", " ", " ", " ", \
" ", " ", " ", " ", " ", " ", " ", " ", " ", " ", " ", " ", " ", " ", \
" "}
\end{lstlisting}

(Я заменил непечатаемые символы на \q{?}.)

Мы видим что первый и третий блоки пустые (или почти пустые),
но второй и четвертый имеют ясно различимые английские слова/фразы.
Похоже что наше предположение насчет ключа верно (как минимум частично).
Это означает, что самый встречающийся 81-байтный блок в файле находится в местах лакун с нулевыми байтами
или что-то в этом роде.

Попробуем расшифровать весь файл:

\begin{lstlisting}[style=custommath]
DecryptBlock[blk_] := BitXor[key, blk]

decrypted = Map[DecryptBlock[#] &, blocks];

BinaryWrite["/home/dennis/.../tmp", Flatten[decrypted]]

Close["/home/dennis/.../tmp"]
\end{lstlisting}

\begin{figure}[H]
\centering
\myincludegraphics{ff/XOR/mask_1/mc_decrypted1.png}
\caption{Расшифрованный файл в Midnight Commander, первая попытка}
\end{figure}

Выглядит как английские фразы для какой-то игры, но что-то не так.
Прежде всего, регистр инвертирован: фразы и некоторые слова начинаются со строчных букв,
в то время как остальные буквы заглавные.
Также, некоторые фразы начинаются с не тех букв.
Посмотрите на самую первую фразу: \q{eHE WEED OF CRIME BEARS BITTER FRUIT}.
Что такое \q{eHE}? Разве не \q{tHE} тут должно быть?
Возможно ли что наш ключ для дешифрования имеет неверный байт в этом месте?

Посмотрим снова на второй блок в файле, на ключ и на результат дешифрования:

\begin{lstlisting}[style=custommath]
In[]:= blocks[[2]]
Out[]= {80, 2, 74, 49, 113, 49, 51, 92, 39, 8, 92, 81, 116, 62, \
57, 80, 46, 40, 114, 36, 75, 56, 33, 76, 9, 55, 56, 59, 81, 65, 45, \
28, 60, 55, 93, 39, 90, 28, 124, 106, 16, 20, 104, 119, 8, 109, 26, \
106, 9, 97, 13, 99, 15, 119, 20, 105, 117, 98, 103, 118, 1, 126, 29, \
97, 122, 17, 15, 114, 110, 3, 5, 125, 125, 99, 126, 119, 102, 30, \
122, 2, 117}

In[]:= key
Out[]= {80, 103, 2, 116, 113, 102, 118, 25, 99, 8, 19, 23, 116, \
125, 107, 25, 99, 109, 114, 102, 14, 121, 115, 31, 9, 117, 113, 111, \
5, 4, 127, 28, 122, 101, 8, 110, 14, 18, 124, 106, 16, 20, 104, 119, \
8, 109, 26, 106, 9, 97, 13, 99, 15, 119, 20, 105, 117, 98, 103, 118, \
1, 126, 29, 97, 122, 17, 15, 114, 110, 3, 5, 125, 125, 99, 126, 119, \
102, 30, 122, 2, 117}

In[]:= BitXor[key, blocks[[2]]]
Out[]= {0, 101, 72, 69, 0, 87, 69, 69, 68, 0, 79, 70, 0, 67, 82, \
73, 77, 69, 0, 66, 69, 65, 82, 83, 0, 66, 73, 84, 84, 69, 82, 0, 70, \
82, 85, 73, 84, 14, 0, 0, 0, 0, 0, 0, 0, 0, 0, 0, 0, 0, 0, 0, 0, 0, \
0, 0, 0, 0, 0, 0, 0, 0, 0, 0, 0, 0, 0, 0, 0, 0, 0, 0, 0, 0, 0, 0, 0, \
0, 0, 0, 0}
\end{lstlisting}

Зашифрованный байт это 2, байт из ключа это 103, $2 \oplus 103=101$ и 101 это ASCII-код символа \q{e}.
Чему должен равнятся этот байт ключа, чтобы ASCII-код был 116 (для символа  \q{t})?
$2 \oplus 116=118$, присвоим 118 второму байту в ключе \dots

\begin{lstlisting}[style=custommath]
key = {80, 118, 2, 116, 113, 102, 118, 25, 99, 8, 19, 23, 116, 125, 
  107, 25, 99, 109, 114, 102, 14, 121, 115, 31, 9, 117, 113, 111, 5, 
  4, 127, 28, 122, 101, 8, 110, 14, 18, 124, 106, 16, 20, 104, 119, 8,
   109, 26, 106, 9, 97, 13, 99, 15, 119, 20, 105, 117, 98, 103, 118, 
  1, 126, 29, 97, 122, 17, 15, 114, 110, 3, 5, 125, 125, 99, 126, 119,
   102, 30, 122, 2, 117}
\end{lstlisting}

\dots и снова дешифруем весь файл.

\begin{figure}[H]
\centering
\myincludegraphics{ff/XOR/mask_1/mc_decrypted2.png}
\caption{Дешифрованный файл в Midnight Commander, вторая попытка}
\end{figure}

Ух ты, теперь грамматика корректна, и все фразы начинаются с корректных букв.
Но все таки, регистр подозрителен.
С чего бы разработчику игры записывать их в такой манере?
Может быть наш ключ все еще неправилен?

% TODO ASCII table somewhere in the book
Изучая таблицу ASCII мы можем заметить что ASCII-коды для букв в верхнем и нижнем регистре отличаются только на один бит
(6-й бит, если считать с первого, 0b100000):

\begin{figure}[H]
\centering
\includegraphics[width=0.7\textwidth]{ascii.png}
\caption{7-битная таблица \ac{ASCII} в Emacs}
\end{figure}

6-й бит, выставленный в нулевом байте, В десятичном виде это будет 32.
Но 32 это ASCII-код пробела!

Действительно, можно менять регистр просто применяя XOR к ASCII-коду, с 32 (больше об этом: \myref{toupper_bit}).

Возможно ли, что пустые лакуны в файле это не нулевые байты, а скорее содержащие пробелы?
Еще раз модифицируем наш XOR-ключ (я про-XOR-ю каждый байт ключа с 32):

\begin{lstlisting}[style=custommath]
(* "32" это скаляр, и "key" это вектор, но это OK *)

In[]:= key3 = BitXor[32, key]
Out[]= {112, 86, 34, 84, 81, 70, 86, 57, 67, 40, 51, 55, 84, 93, 75, \
57, 67, 77, 82, 70, 46, 89, 83, 63, 41, 85, 81, 79, 37, 36, 95, 60, \
90, 69, 40, 78, 46, 50, 92, 74, 48, 52, 72, 87, 40, 77, 58, 74, 41, \
65, 45, 67, 47, 87, 52, 73, 85, 66, 71, 86, 33, 94, 61, 65, 90, 49, \
47, 82, 78, 35, 37, 93, 93, 67, 94, 87, 70, 62, 90, 34, 85}

In[]:= DecryptBlock[blk_] := BitXor[key3, blk]
\end{lstlisting}

И снова дешифруем входной файл:

\begin{figure}[H]
\centering
\myincludegraphics{ff/XOR/mask_1/mc_decrypted.png}
\caption{Дешифрованный файл в Midnight Commander, последняя попытка}
\end{figure}

(Расшифрованный файл доступен здесь:
\url{\GitHubBlobMasterURL/ff/XOR/mask_1/files/decrypted.dat.bz2}.)

Несомненно, это корректный исходный файл.
Да, и мы видим числа в начале каждого блока. Должно быть это и есть источник некорректного XOR-ключа.
Как выходит, самый встречающийся 81-байтный блок в файле это блок заполненный пробелами и содержащий символ \q{1} на месте
второго байта.
Действительно, как-то так получилось что многие блоки здесь перемежаются с этим блоком.
Может быть это что-то вроде выравнивания (padding) для коротких фраз/сообщений?
Другой часто встречающийся 81-байтный блок также заполнен пробелами, но с другой цифрой, следовательно,
они отличаются только вторым байтом.

Вот и всё! Теперь мы можем написать утилиту для зашифрования файла назад, и, может быть, модифицировать его перед этим

Файл для Mathematica можно скачать здесь:\\
\url{\GitHubBlobMasterURL/ff/XOR/mask_1/files/XOR_mask_1.nb}.

Итог: XOR-шифрование не надежно вообще. Вероятно, разработчик игры хотел просто скрыть внутренности игры от игрока,
ничего более серьезного.
Все же, шифрование вроде этого крайне популярно вследствии его простоты, так что многие реверс инженеры обычно хорошо
с этим знакомы.


% \subsection{Простое шифрование используя XOR-маску}
\label{XOR_mask_1}

Я нашел одну старую игру в стиле interactive fiction в архиве \emph{if-archive}\footnote{\url{http://www.ifarchive.org/}}:

\begin{lstlisting}
The New Castle v3.5 - Text/Adventure Game
in the style of the original Infocom (tm)
type games, Zork, Collosal Cave (Adventure),
etc.  Can you solve the mystery of the
abandoned castle?
Shareware from Software Customization.
Software Customization [ASP] Version 3.5 Feb. 2000
\end{lstlisting}

Можно скачать здесь: \url{\GitHubBlobMasterURL/ff/XOR/mask_1/files/newcastle.tgz}.

Там внутри есть файл (с названием \emph{castle.dbf}), который явно зашифрован, но не настоящим криптоалгоритмом,
и оне сжат, это что-то куда проще.
Я бы даже не стал измерять уровень энтропии (\myref{entropy}) этого файла, потому что я итак уверен, что он низкий.
Вот как он выглядит в Midnight Commander:

\begin{figure}[H]
\centering
\myincludegraphics{ff/XOR/mask_1/mc_encrypted.png}
\caption{Зашифрованный файл в Midnight Commander}
\end{figure}

Зашифрованный файл можно скачать здесь:
\url{\GitHubBlobMasterURL/ff/XOR/mask_1/files/castle.dbf.bz2}.

Можно ли расшифровать его без доступа к программе, используя просто этот файл?

Тут явно просматривается повторяющаяся строка. 
Если использовалось простое шифрование с XOR-маской, такие повторяющиеся строки это явное свидетельство,
потому что, вероятно, тут были длинные лакуны с нулевыми байтами, которые, в свою очередь, присутствуют
во мноигих исполняемых файлах, и в остальных бинарных файлах.

\myindex{UNIX!xxd}
Вот дам начала этого файла используя утилиту \emph{xxd} из UNIX:

\lstinputlisting{ff/XOR/mask_1/xxd_result.txt}

Давайте держаться за повторяющуюся строку \TT{iubgv}.
Глядя на этот дамп, мы можем легко увидеть, что период повторений этой строки это 0x51 или 81.
Вероятно, 81 это длина блока?
Длина файла 1658961, и она может быть поделена на 81 без остатка (и тогда там 20481 блоков).

Теперь я буду использовать Mathematica для анализа, есть ли тут повторяющиеся 81-байтные блоки в файле?
Я разделю входной файл на 81-байтные блоки и затем использую ф-цию
\emph{Tally[]}\footnote{\url{https://reference.wolfram.com/language/ref/Tally.html}}
которая просто считает, сколько раз каждый элемент встретился во входном списке.
Вывод Tally не отсортирован, так что я также добавлю ф-цию \emph{Sort[]} для сортировки его по кол-ву вхождений
в нисходящем порядке.

\begin{lstlisting}[style=custommath]
input = BinaryReadList["/home/dennis/.../castle.dbf"];

blocks = Partition[input, 81];

stat = Sort[Tally[blocks], #1[[2]] > #2[[2]] &]
\end{lstlisting}

И вот вывод:

\begin{lstlisting}[style=custommath]
{{{80, 103, 2, 116, 113, 102, 118, 25, 99, 8, 19, 23, 116, 125, 107, 
   25, 99, 109, 114, 102, 14, 121, 115, 31, 9, 117, 113, 111, 5, 4, 
   127, 28, 122, 101, 8, 110, 14, 18, 124, 106, 16, 20, 104, 119, 8, 
   109, 26, 106, 9, 97, 13, 99, 15, 119, 20, 105, 117, 98, 103, 118, 
   1, 126, 29, 97, 122, 17, 15, 114, 110, 3, 5, 125, 125, 99, 126, 
   119, 102, 30, 122, 2, 117}, 1739}, 
{{80, 100, 2, 116, 113, 102, 118, 25, 99, 8, 19, 23, 116, 
   125, 107, 25, 99, 109, 114, 102, 14, 121, 115, 31, 9, 117, 113, 
   111, 5, 4, 127, 28, 122, 101, 8, 110, 14, 18, 124, 106, 16, 20, 
   104, 119, 8, 109, 26, 106, 9, 97, 13, 99, 15, 119, 20, 105, 117, 
   98, 103, 118, 1, 126, 29, 97, 122, 17, 15, 114, 110, 3, 5, 125, 
   125, 99, 126, 119, 102, 30, 122, 2, 117}, 1422}, 
{{80, 101, 2, 116, 113, 102, 118, 25, 99, 8, 19, 23, 116, 
   125, 107, 25, 99, 109, 114, 102, 14, 121, 115, 31, 9, 117, 113, 
   111, 5, 4, 127, 28, 122, 101, 8, 110, 14, 18, 124, 106, 16, 20, 
   104, 119, 8, 109, 26, 106, 9, 97, 13, 99, 15, 119, 20, 105, 117, 
   98, 103, 118, 1, 126, 29, 97, 122, 17, 15, 114, 110, 3, 5, 125, 
   125, 99, 126, 119, 102, 30, 122, 2, 117}, 1012},
{{80, 120, 2, 116, 113, 102, 118, 25, 99, 8, 19, 23, 116, 
   125, 107, 25, 99, 109, 114, 102, 14, 121, 115, 31, 9, 117, 113, 
   111, 5, 4, 127, 28, 122, 101, 8, 110, 14, 18, 124, 106, 16, 20, 
   104, 119, 8, 109, 26, 106, 9, 97, 13, 99, 15, 119, 20, 105, 117, 
   98, 103, 118, 1, 126, 29, 97, 122, 17, 15, 114, 110, 3, 5, 125, 
   125, 99, 126, 119, 102, 30, 122, 2, 117}, 377},

...

{{80, 2, 74, 49, 113, 21, 62, 88, 39, 71, 68, 23, 63, 51, 36, 78, 48, 
   108, 114, 102, 14, 121, 115, 31, 9, 117, 113, 111, 5, 4, 127, 28, 
   122, 101, 8, 110, 14, 18, 124, 106, 16, 20, 104, 119, 8, 109, 26, 
   106, 9, 97, 13, 99, 15, 119, 20, 105, 117, 98, 103, 118, 1, 126, 
   29, 97, 122, 17, 15, 114, 110, 3, 5, 125, 125, 99, 126, 119, 102, 
   30, 122, 2, 117}, 1},
{{80, 1, 74, 59, 113, 45, 56, 86, 52, 91, 19, 64, 60, 60, 63, 
   25, 38, 59, 59, 42, 14, 53, 38, 77, 66, 38, 113, 38, 75, 4, 43, 84,
    63, 101, 64, 43, 79, 64, 40, 57, 16, 91, 46, 119, 69, 40, 84, 117,
    9, 97, 13, 99, 15, 119, 20, 105, 117, 98, 103, 118, 1, 126, 29, 
   97, 122, 17, 15, 114, 110, 3, 5, 125, 125, 99, 126, 119, 102, 30, 
   122, 2, 117}, 1},
{{80, 2, 74, 49, 113, 49, 51, 92, 39, 8, 92, 81, 116, 62, 57, 
   80, 46, 40, 114, 36, 75, 56, 33, 76, 9, 55, 56, 59, 81, 65, 45, 28,
    60, 55, 93, 39, 90, 28, 124, 106, 16, 20, 104, 119, 8, 109, 26, 
   106, 9, 97, 13, 99, 15, 119, 20, 105, 117, 98, 103, 118, 1, 126, 
   29, 97, 122, 17, 15, 114, 110, 3, 5, 125, 125, 99, 126, 119, 102, 
   30, 122, 2, 117}, 1}}
\end{lstlisting}

Вывод Tally это список пар, каждая пара это 81-байтный блок и количество раз, сколько он встретился в файле.
Мы видим, что наиболее частно встречающийся блок это первый, он встретился 1739 раз.
Второй встретился 1422 раза. Есть и другие: 1012 раза, 377 раз, итд.
81-байтные блоки, встреченные лишь один раз, находятся в конце вывода.

Попробуем сравнить эти блоки. Первый и второй.
Есть ли в Mathematica ф-ция для сравнения списков/массивов?
Наверняка есть, но в педагогических целях, я буду использоват операцию XOR для сравнения.
Действительно: если байты во входных массивах равны друг другу, результат операции XOR это 0.
Если не равны, результат будет ненулевой.

Сравним первый блок (встречается 1739 раз) и второй (встречается 1422 раз):

\begin{lstlisting}[style=custommath]
In[]:= BitXor[stat[[1]][[1]], stat[[2]][[1]]]
Out[]= {0, 3, 0, 0, 0, 0, 0, 0, 0, 0, 0, 0, 0, 0, 0, 0, 0, 0, 0, \
0, 0, 0, 0, 0, 0, 0, 0, 0, 0, 0, 0, 0, 0, 0, 0, 0, 0, 0, 0, 0, 0, 0, \
0, 0, 0, 0, 0, 0, 0, 0, 0, 0, 0, 0, 0, 0, 0, 0, 0, 0, 0, 0, 0, 0, 0, \
0, 0, 0, 0, 0, 0, 0, 0, 0, 0, 0, 0, 0, 0, 0, 0}
\end{lstlisting}

Они отличаются только вторым байтом.

Сравним второй блок (встречается 1422 раза) и третий (встречается 1012 раз):

\begin{lstlisting}[style=custommath]
In[]:= BitXor[stat[[2]][[1]], stat[[3]][[1]]]
Out[]= {0, 1, 0, 0, 0, 0, 0, 0, 0, 0, 0, 0, 0, 0, 0, 0, 0, 0, 0, \
0, 0, 0, 0, 0, 0, 0, 0, 0, 0, 0, 0, 0, 0, 0, 0, 0, 0, 0, 0, 0, 0, 0, \
0, 0, 0, 0, 0, 0, 0, 0, 0, 0, 0, 0, 0, 0, 0, 0, 0, 0, 0, 0, 0, 0, 0, \
0, 0, 0, 0, 0, 0, 0, 0, 0, 0, 0, 0, 0, 0, 0, 0}
\end{lstlisting}

Они тоже отличаются только вторым байтом.

Так или иначе, попробуем использовать самый встречающийся блок как XOR-ключ и попробуем расшифровать первые 4 81-байтных
блока в файле:

\begin{lstlisting}[style=custommath]
In[]:= key = stat[[1]][[1]]
Out[]= {80, 103, 2, 116, 113, 102, 118, 25, 99, 8, 19, 23, 116, \
125, 107, 25, 99, 109, 114, 102, 14, 121, 115, 31, 9, 117, 113, 111, \
5, 4, 127, 28, 122, 101, 8, 110, 14, 18, 124, 106, 16, 20, 104, 119, \
8, 109, 26, 106, 9, 97, 13, 99, 15, 119, 20, 105, 117, 98, 103, 118, \
1, 126, 29, 97, 122, 17, 15, 114, 110, 3, 5, 125, 125, 99, 126, 119, \
102, 30, 122, 2, 117}

In[]:= ToASCII[val_] := If[val == 0, " ", FromCharacterCode[val, "PrintableASCII"]]

In[]:= DecryptBlockASCII[blk_] := Map[ToASCII[#] &, BitXor[key, blk]]

In[]:= DecryptBlockASCII[blocks[[1]]]
Out[]= {" ", " ", " ", " ", " ", " ", " ", " ", " ", " ", " ", " \
", " ", " ", " ", " ", " ", " ", " ", " ", " ", " ", " ", " ", " ", " \
", " ", " ", " ", " ", " ", " ", " ", " ", " ", " ", " ", " ", " ", " \
", " ", " ", " ", " ", " ", " ", " ", " ", " ", " ", " ", " ", " ", " \
", " ", " ", " ", " ", " ", " ", " ", " ", " ", " ", " ", " ", " ", " \
", " ", " ", " ", " ", " ", " ", " ", " ", " ", " ", " ", " ", " "}

In[]:= DecryptBlockASCII[blocks[[2]]]
Out[]= {" ", "e", "H", "E", " ", "W", "E", "E", "D", " ", "O", \
"F", " ", "C", "R", "I", "M", "E", " ", "B", "E", "A", "R", "S", " ", \
"B", "I", "T", "T", "E", "R", " ", "F", "R", "U", "I", "T", "?", \
" ", " ", " ", " ", " ", " ", " ", " ", " ", " ", " ", " ", " ", " ", \
" ", " ", " ", " ", " ", " ", " ", " ", " ", " ", " ", " ", " ", " ", \
" ", " ", " ", " ", " ", " ", " ", " ", " ", " ", " ", " ", " ", " ", \
" "}

In[]:= DecryptBlockASCII[blocks[[3]]]
Out[]= {" ", "?", " ", " ", " ", " ", " ", " ", " ", " ", " \
", " ", " ", " ", " ", " ", " ", " ", " ", " ", " ", " ", " ", " ", " \
", " ", " ", " ", " ", " ", " ", " ", " ", " ", " ", " ", " ", " ", " \
", " ", " ", " ", " ", " ", " ", " ", " ", " ", " ", " ", " ", " ", " \
", " ", " ", " ", " ", " ", " ", " ", " ", " ", " ", " ", " ", " ", " \
", " ", " ", " ", " ", " ", " ", " ", " ", " ", " ", " ", " ", " ", " \
"}

In[]:= DecryptBlockASCII[blocks[[4]]]
Out[]= {" ", "f", "H", "O", " ", "K", "N", "O", "W", "S", " ", \
"W", "H", "A", "T", " ", "E", "V", "I", "L", " ", "L", "U", "R", "K", \
"S", " ", "I", "N", " ", "T", "H", "E", " ", "H", "E", "A", "R", "T", \
"S", " ", "O", "F", " ", "M", "E", "N", "?", " ", " ", " ", " ", \
" ", " ", " ", " ", " ", " ", " ", " ", " ", " ", " ", " ", " ", " ", \
" ", " ", " ", " ", " ", " ", " ", " ", " ", " ", " ", " ", " ", " ", \
" "}
\end{lstlisting}

(Я заменил непечатаемые символы на \q{?}.)

Мы видим что первый и третий блоки пустые (или почти пустые),
но второй и четвертый имеют ясно различимые английские слова/фразы.
Похоже что наше предположение насчет ключа верно (как минимум частично).
Это означает, что самый встречающийся 81-байтный блок в файле находится в местах лакун с нулевыми байтами
или что-то в этом роде.

Попробуем расшифровать весь файл:

\begin{lstlisting}[style=custommath]
DecryptBlock[blk_] := BitXor[key, blk]

decrypted = Map[DecryptBlock[#] &, blocks];

BinaryWrite["/home/dennis/.../tmp", Flatten[decrypted]]

Close["/home/dennis/.../tmp"]
\end{lstlisting}

\begin{figure}[H]
\centering
\myincludegraphics{ff/XOR/mask_1/mc_decrypted1.png}
\caption{Расшифрованный файл в Midnight Commander, первая попытка}
\end{figure}

Выглядит как английские фразы для какой-то игры, но что-то не так.
Прежде всего, регистр инвертирован: фразы и некоторые слова начинаются со строчных букв,
в то время как остальные буквы заглавные.
Также, некоторые фразы начинаются с не тех букв.
Посмотрите на самую первую фразу: \q{eHE WEED OF CRIME BEARS BITTER FRUIT}.
Что такое \q{eHE}? Разве не \q{tHE} тут должно быть?
Возможно ли что наш ключ для дешифрования имеет неверный байт в этом месте?

Посмотрим снова на второй блок в файле, на ключ и на результат дешифрования:

\begin{lstlisting}[style=custommath]
In[]:= blocks[[2]]
Out[]= {80, 2, 74, 49, 113, 49, 51, 92, 39, 8, 92, 81, 116, 62, \
57, 80, 46, 40, 114, 36, 75, 56, 33, 76, 9, 55, 56, 59, 81, 65, 45, \
28, 60, 55, 93, 39, 90, 28, 124, 106, 16, 20, 104, 119, 8, 109, 26, \
106, 9, 97, 13, 99, 15, 119, 20, 105, 117, 98, 103, 118, 1, 126, 29, \
97, 122, 17, 15, 114, 110, 3, 5, 125, 125, 99, 126, 119, 102, 30, \
122, 2, 117}

In[]:= key
Out[]= {80, 103, 2, 116, 113, 102, 118, 25, 99, 8, 19, 23, 116, \
125, 107, 25, 99, 109, 114, 102, 14, 121, 115, 31, 9, 117, 113, 111, \
5, 4, 127, 28, 122, 101, 8, 110, 14, 18, 124, 106, 16, 20, 104, 119, \
8, 109, 26, 106, 9, 97, 13, 99, 15, 119, 20, 105, 117, 98, 103, 118, \
1, 126, 29, 97, 122, 17, 15, 114, 110, 3, 5, 125, 125, 99, 126, 119, \
102, 30, 122, 2, 117}

In[]:= BitXor[key, blocks[[2]]]
Out[]= {0, 101, 72, 69, 0, 87, 69, 69, 68, 0, 79, 70, 0, 67, 82, \
73, 77, 69, 0, 66, 69, 65, 82, 83, 0, 66, 73, 84, 84, 69, 82, 0, 70, \
82, 85, 73, 84, 14, 0, 0, 0, 0, 0, 0, 0, 0, 0, 0, 0, 0, 0, 0, 0, 0, \
0, 0, 0, 0, 0, 0, 0, 0, 0, 0, 0, 0, 0, 0, 0, 0, 0, 0, 0, 0, 0, 0, 0, \
0, 0, 0, 0}
\end{lstlisting}

Зашифрованный байт это 2, байт из ключа это 103, $2 \oplus 103=101$ и 101 это ASCII-код символа \q{e}.
Чему должен равнятся этот байт ключа, чтобы ASCII-код был 116 (для символа  \q{t})?
$2 \oplus 116=118$, присвоим 118 второму байту в ключе \dots

\begin{lstlisting}[style=custommath]
key = {80, 118, 2, 116, 113, 102, 118, 25, 99, 8, 19, 23, 116, 125, 
  107, 25, 99, 109, 114, 102, 14, 121, 115, 31, 9, 117, 113, 111, 5, 
  4, 127, 28, 122, 101, 8, 110, 14, 18, 124, 106, 16, 20, 104, 119, 8,
   109, 26, 106, 9, 97, 13, 99, 15, 119, 20, 105, 117, 98, 103, 118, 
  1, 126, 29, 97, 122, 17, 15, 114, 110, 3, 5, 125, 125, 99, 126, 119,
   102, 30, 122, 2, 117}
\end{lstlisting}

\dots и снова дешифруем весь файл.

\begin{figure}[H]
\centering
\myincludegraphics{ff/XOR/mask_1/mc_decrypted2.png}
\caption{Дешифрованный файл в Midnight Commander, вторая попытка}
\end{figure}

Ух ты, теперь грамматика корректна, и все фразы начинаются с корректных букв.
Но все таки, регистр подозрителен.
С чего бы разработчику игры записывать их в такой манере?
Может быть наш ключ все еще неправилен?

% TODO ASCII table somewhere in the book
Изучая таблицу ASCII мы можем заметить что ASCII-коды для букв в верхнем и нижнем регистре отличаются только на один бит
(6-й бит, если считать с первого, 0b100000):

\begin{figure}[H]
\centering
\includegraphics[width=0.7\textwidth]{ascii.png}
\caption{7-битная таблица \ac{ASCII} в Emacs}
\end{figure}

6-й бит, выставленный в нулевом байте, В десятичном виде это будет 32.
Но 32 это ASCII-код пробела!

Действительно, можно менять регистр просто применяя XOR к ASCII-коду, с 32 (больше об этом: \myref{toupper_bit}).

Возможно ли, что пустые лакуны в файле это не нулевые байты, а скорее содержащие пробелы?
Еще раз модифицируем наш XOR-ключ (я про-XOR-ю каждый байт ключа с 32):

\begin{lstlisting}[style=custommath]
(* "32" это скаляр, и "key" это вектор, но это OK *)

In[]:= key3 = BitXor[32, key]
Out[]= {112, 86, 34, 84, 81, 70, 86, 57, 67, 40, 51, 55, 84, 93, 75, \
57, 67, 77, 82, 70, 46, 89, 83, 63, 41, 85, 81, 79, 37, 36, 95, 60, \
90, 69, 40, 78, 46, 50, 92, 74, 48, 52, 72, 87, 40, 77, 58, 74, 41, \
65, 45, 67, 47, 87, 52, 73, 85, 66, 71, 86, 33, 94, 61, 65, 90, 49, \
47, 82, 78, 35, 37, 93, 93, 67, 94, 87, 70, 62, 90, 34, 85}

In[]:= DecryptBlock[blk_] := BitXor[key3, blk]
\end{lstlisting}

И снова дешифруем входной файл:

\begin{figure}[H]
\centering
\myincludegraphics{ff/XOR/mask_1/mc_decrypted.png}
\caption{Дешифрованный файл в Midnight Commander, последняя попытка}
\end{figure}

(Расшифрованный файл доступен здесь:
\url{\GitHubBlobMasterURL/ff/XOR/mask_1/files/decrypted.dat.bz2}.)

Несомненно, это корректный исходный файл.
Да, и мы видим числа в начале каждого блока. Должно быть это и есть источник некорректного XOR-ключа.
Как выходит, самый встречающийся 81-байтный блок в файле это блок заполненный пробелами и содержащий символ \q{1} на месте
второго байта.
Действительно, как-то так получилось что многие блоки здесь перемежаются с этим блоком.
Может быть это что-то вроде выравнивания (padding) для коротких фраз/сообщений?
Другой часто встречающийся 81-байтный блок также заполнен пробелами, но с другой цифрой, следовательно,
они отличаются только вторым байтом.

Вот и всё! Теперь мы можем написать утилиту для зашифрования файла назад, и, может быть, модифицировать его перед этим

Файл для Mathematica можно скачать здесь:\\
\url{\GitHubBlobMasterURL/ff/XOR/mask_1/files/XOR_mask_1.nb}.

Итог: XOR-шифрование не надежно вообще. Вероятно, разработчик игры хотел просто скрыть внутренности игры от игрока,
ничего более серьезного.
Все же, шифрование вроде этого крайне популярно вследствии его простоты, так что многие реверс инженеры обычно хорошо
с этим знакомы.



\mysection{Другие примеры}

Здесь также был пример с Z3 и ручной декомпиляцией.
Он перемещен сюда:
\url{https://yurichev.com/writings/SAT_SMT_by_example.pdf}.

}
%\DE{\myparagraph{\NonOptimizing MSVC}

MSVC 2010 erzeugt den folgenden Code:

\lstinputlisting[caption=\NonOptimizing MSVC
2010,style=customasmx86]{patterns/12_FPU/3_comparison/x86/MSVC/MSVC_DE.asm}

\myindex{x86!\Instructions!FLD}

Der Befehl \FLD lädt \GTT{\_b} nach \ST{0}.

\label{Czero_etc}
\newcommand{\Czero}{\GTT{C0}\xspace}
\newcommand{\Ctwo}{\GTT{C2}\xspace}
\newcommand{\Cthree}{\GTT{C3}\xspace}
\newcommand{\CThreeBits}{\Cthree/\Ctwo/\Czero}

\myindex{x86!\Instructions!FCOMP}
\FCOMP verlgeicht den Wert in \ST{0} mit dem Wert, der sich in \GTT{\_a}
befindet und setzt die \CThreeBits im FPU Status Register entsprechend.
Das Statusregister ist ein 16-Bit-Register, das den aktueller Zustand der FPU
abbildet.

Nachdem die Bits gesetzt worden sind, nimmer der \FCOMP Befehl auch eine
Variable vom Stack. Dieses Verhalten unterscheidet ihn von \FCOM, der einfach
zwei Werte vergleicht und den Stack unangetastet lässt.

Leider verfügen CPUs vor Intel P6\footnote{Intel P6 ist Pentium Pro, Pentium II,
etc.}über keinerlei bedingte Sprungbefehle, die die \CThreeBits prüfen.

After the bits are set, the \FCOMP instruction also pops one variable from the stack. 
This is what distinguishes it from \FCOM, which is just compares values, leaving the stack in the same state.
Vielleicht ist diese Tatsache historisch begründet (man erinnere sich: die FPU
war früher ein eigener Chip).\\
Moderne CPUs, beginnend mit Intel P6 haben \FCOMI/\FCOMIP/\FUCOMI/\FUCOMIP
Befehle~--welche im Prinzip das gleiche tun, aber die \ZF/\PF/\CF Flags der CPU
verändern können.

\myindex{x86!\Instructions!FNSTSW}
Der \FNSTSW Befehl kopiert das FPU Statusregister nach \AX.
\CThreeBits werden an den Stellen 14/10/8 abgelegt, sie befinden sich im \AX
Register an den gleichen Stellen und sie werden alle in höherwertigen Teil von
\AX{}~---\AH{} abgelegt.

\begin{itemize}
\item Falls in unserem Beispiel $b>a$, dann werden die \CThreeBits Bits wie
folgt gesetzt: 0, 0, 0.
\item Falls $a>b$, dann ist das Bitmuster: 0, 0, 1.
\item Falls $a=b$, dann ist das Bitmuster: 1, 0, 0.
\item

Wenn das Ergebnis (z.B. im Fehlerfall) ungeordnet ist, dann werden die Bits wie
folgt gesetzt: 1,1,1.
\end{itemize}
% TODO: table here?
So werden die \CThreeBits Bits im \AX Register angeordnet:

\input{C3_in_AX}

So werden die \CThreeBits Bits im \AH Register angeordnet:

\input{C3_in_AH}
Nach der Ausführung von \INS{test ah, 5}\footnote{5=101b} werden nur die \Czero
und \Ctwo Bits (an den Stellen 0 und 2) betrachtet, alle übrigen Bits werden
einfach überlesen.

\label{parity_flag}
\myindex{x86!\Registers!\Flags!Parity flag}
Werfen wir nun einen Blick auf ein anderes bemerkenswertes historisches
Überbleibsel: das \emph{parity flag}.

Dieses Flag wird auf 1 gesetzt, falls die Anzahl der Einsen im Ergebnis der
letzten Berechnung gerade ist und auf 1, falls dies nicht der Fall ist.

Schlagen wir in der Wikipedia nach\footnote{\WikipediaParityFlag}:

%TODO Quotation has been translated from English wiki article, since the
% correspondig German article doesn't offer such information.
\begin{framed}
\begin{quotation}
Ein guter Grund das Parity Flag abzufragen, hat tatsächlich gar nichts mit
Parität zu tun. Die FPU hat vier Bedingungsflags (C0 bis C3), aber diese können
nicht direkt abgefragt werden, sondern müssen zunächst in das Flags Register
kopiert werden. Wenn dies geschieht, wird C0 im Carry Flag abgelegt, C2 im
Parity Flag und C3 im Zero Flag.
Das C2 Flag ist gesetzt, wenn z.B. unvergleichbare Fließkommawerte (NaN oder
nicht unterstütztes Format) über der \FUCOM Befehl miteinander verglichen
werden.\textit{(Übersetzung aus der englischen Wikipedia.)}
\end{quotation}
\end{framed}

Wie in der Wikipedia dargestellt wird das Parity Flag manchmal im FPU Code
verwendet; schauen wir uns genauer an wie das funktioniert.

\myindex{x86!\Instructions!JP}
Das \PF Flag wird auf 1 gesetzt, wenn sowohl \Czero als auch \Ctwo beide 0 oder
beide 1 sind. In diesem Fall wird der nachfolgende Sprung \JP(\emph{jump if
PF==1}) ausgeführt.
Wenn wir die Werte der \CThreeBits in den unterschiedlichen Fällen betrachten,
dann sehen wir, dass der bedingte Sprung \JP in zwei Fällen ausgeführt wird:
wenn $b>a$ oder wenn $a=b$ (das \Cthree Bit wird hier nicht betrachtet, da es
durch den Befehl \INS{test ah,5}) gelöscht wurde).

Der Rest ist leicht nachvollziehbar.
Denn der bedingte Sprung ausgeführt wurde, lädt \FLD den Wert von \GTT{\_b} nach
\ST{0} und wenn nicht, wird der Wert von \GTT{\_a} dorthin geladen.

\mysubparagraph{Was ist mit der Abfrage von \Ctwo?}
Das \Ctwo Flag wird im Fehlerfall (\gls{NaN}, etc.) gesetzt, aber unser Code
prüft dies nicht. 
Wenn sich der Programmierer für FPU Fehler interessiert, muss er zusätzliche
Abfragen hinzufügen.

\input{patterns/12_FPU/3_comparison/x86/MSVC/olly_DE.tex}
}
\FR{\chapter{Études de cas}

\input{examples/Knuth_FR}

% sections here
\mysection{Fonction presque vide}
\label{Boolector}
\myindex{Boolector}
\myindex{x86!\Instructions!JMP}

Ceci est un morceau de code réel que j'ai trouvé dans Boolector\footnote{\url{https://boolector.github.io/}}:

\lstinputlisting[style=customc]{patterns/025_almost_empty/boolectormain.c}

Pourquoi quelqu'un ferait-il comme ça?
Je ne sais pas mais mon hypothèse est que \verb|boolector_main()| peut être compilée
dans une sorte de DLL ou bibliothèque dynamique, et appelée depuis une suite de test.
Certainement qu'une suite de test peut préparer les variables argc/argv comme
le ferait \ac{CRT}.

Il est intéressant de voir comment c'est compilé:

\lstinputlisting[caption=GCC 8.2 x64 \NonOptimizing (\assemblyOutput),style=customasmx86]{patterns/025_almost_empty/boolectormain_O0.s}

Ceci est OK, le prologue (non optimisé) déplace inutilement deux arguments,
\INS{CALL}, épilogue, \INS{RET}.
Mais regardons la version optimisée:

\lstinputlisting[caption=GCC 8.2 x64 \Optimizing (\assemblyOutput),style=customasmx86]{patterns/025_almost_empty/boolectormain_O3.s}

Aussi simple que ça: la pile et les registres ne sont pas touchés et \verb|boolector_main()|
a le même ensemble d'arguments.
Donc, tout ce que nous avons à faire est de passer l'exécution à une autre adresse.

Ceci est proche d'une \glslink{thunk function}{fonction thunk}.

Nous verons queelque chose de plus avancé plus tard: \myref{ARM_B_to_printf}, \myref{JMP_instead_of_RET}.

\mysection{Fonction presque vide}
\label{Boolector}
\myindex{Boolector}
\myindex{x86!\Instructions!JMP}

Ceci est un morceau de code réel que j'ai trouvé dans Boolector\footnote{\url{https://boolector.github.io/}}:

\lstinputlisting[style=customc]{patterns/025_almost_empty/boolectormain.c}

Pourquoi quelqu'un ferait-il comme ça?
Je ne sais pas mais mon hypothèse est que \verb|boolector_main()| peut être compilée
dans une sorte de DLL ou bibliothèque dynamique, et appelée depuis une suite de test.
Certainement qu'une suite de test peut préparer les variables argc/argv comme
le ferait \ac{CRT}.

Il est intéressant de voir comment c'est compilé:

\lstinputlisting[caption=GCC 8.2 x64 \NonOptimizing (\assemblyOutput),style=customasmx86]{patterns/025_almost_empty/boolectormain_O0.s}

Ceci est OK, le prologue (non optimisé) déplace inutilement deux arguments,
\INS{CALL}, épilogue, \INS{RET}.
Mais regardons la version optimisée:

\lstinputlisting[caption=GCC 8.2 x64 \Optimizing (\assemblyOutput),style=customasmx86]{patterns/025_almost_empty/boolectormain_O3.s}

Aussi simple que ça: la pile et les registres ne sont pas touchés et \verb|boolector_main()|
a le même ensemble d'arguments.
Donc, tout ce que nous avons à faire est de passer l'exécution à une autre adresse.

Ceci est proche d'une \glslink{thunk function}{fonction thunk}.

Nous verons queelque chose de plus avancé plus tard: \myref{ARM_B_to_printf}, \myref{JMP_instead_of_RET}.

\mysection{\MinesweeperWinXPExampleChapterName}
\label{minesweeper_winxp}
\myindex{Windows!Windows XP}

Pour ceux qui ne sont pas très bons avec le jeu démineur, nous pouvons essayer de
révéler les mines cachées dans le débogueur.

\myindex{\CStandardLibrary!rand()}
\myindex{Windows!PDB}

Comme on le sait, le démineur place des mines aléatoirement, donc il doit y avoir
une sorte de générateur de nombre aléatoire ou un appel à la fonction C standard
\TT{rand()}.

Ce qui est vraiment cool en rétro-ingénierant des produits Microsoft c'est qu'il
y a les fichiers \gls{PDB} avec les symboles (nom de fonctions, etc.)
Lorsque nous chargeons \TT{winmine.exe} dans \IDA, il télécharge le fichier \gls{PDB}
exact pour cet exécutable et affiche tous les noms.

Donc le voici, le seul appel à \TT{rand()} est cette fonction:

\lstinputlisting[style=customasmx86]{examples/minesweeper/tmp1.lst}

\IDA l'a appelé ainsi, et c'est le nom que lui ont donné les développeurs du démineur.

La fonction est très simple:

\begin{lstlisting}[style=customc]
int Rnd(int limit)
{
    return rand() % limit;
};
\end{lstlisting}

(Il n'y a pas de nom \q{limit} dans le fichier \gls{PDB}; nous avons nommé manuellement
les arguments comme ceci.)

Donc elle renvoie une valeur aléatoire entre 0 et la limite spécifiée.

\TT{Rnd()} est appelée depuis un seul endroit, la fonction appelée \TT{StartGame()},
et il semble bien que ce soit exactement le code qui place les mines:

\begin{lstlisting}[style=customasmx86]
.text:010036C7                 push    _xBoxMac
.text:010036CD                 call    _Rnd@4          ; Rnd(x)
.text:010036D2                 push    _yBoxMac
.text:010036D8                 mov     esi, eax
.text:010036DA                 inc     esi
.text:010036DB                 call    _Rnd@4          ; Rnd(x)
.text:010036E0                 inc     eax
.text:010036E1                 mov     ecx, eax
.text:010036E3                 shl     ecx, 5          ; ECX=ECX*32
.text:010036E6                 test    _rgBlk[ecx+esi], 80h
.text:010036EE                 jnz     short loc_10036C7
.text:010036F0                 shl     eax, 5          ; EAX=EAX*32
.text:010036F3                 lea     eax, _rgBlk[eax+esi]
.text:010036FA                 or      byte ptr [eax], 80h
.text:010036FD                 dec     _cBombStart
.text:01003703                 jnz     short loc_10036C7
\end{lstlisting}

Le démineur vous permet de définir la taille du plateau, donc les dimensions X (xBoxMac)
et Y (yBoxMac) du plateau sont des variables globales.
Elles sont passées à \TT{Rnd()} et des coordonnées aléatoires sont générées.
Une mine est placée par l'instruction \TT{OR} en \TT{0x010036FA}.
Et si une mine y a déjà été placée avant (il est possible que la fonction \TT{Rnd()}
génère une paire de coordonnées qui a déjà été générée), alors les instructions \TT{TEST}
et \TT{JNZ} en \TT{0x010036E6} bouclent sur la routine de génération.

\TT{cBombStart} est la variable globale contenant le nombre total de mines. Donc
ceci est une boucle.

La largeur du tableau est 32 (nous pouvons conclure ceci en regardant l'instruction
\TT{SHL}, qui multiplie l'une des coordonnées par 32).

La taille du tableau global \TT{rgBlk} peut facilement être déduite par la différence
entre le label \TT{rgBlk} dans le segment de données et le label suivant. Il s'agit
de 0x360 (864):

\begin{lstlisting}[style=customasmx86]
.data:01005340 _rgBlk          db 360h dup(?)          ; DATA XREF: MainWndProc(x,x,x,x)+574
.data:01005340                                         ; DisplayBlk(x,x)+23
.data:010056A0 _Preferences    dd ?                    ; DATA XREF: FixMenus()+2
...
\end{lstlisting}

$864/32=27$.

Donc, la taille du tableau est-elle $27*32$?
C'est proche de ce que nous savons: lorsque nous essayons de définir la taille du
plateau à $100*100$ dans les préférences du démineur, il corrige à une taille de
plateau de $24*30$.
Donc ceci est la taille maximale du plateau.
Et le tableau a une taille fixe, pour toutes les tailles de plateau.

REgardons tout ceci dans \olly.
Nous allons lancer le démineur, lui attacher \olly et nous allons pouvoir voir le
contenu de la mémoire à l'adresse du tableau \TT{rgBlk} (\TT{0x01005340})\footnote{Toutes
les adresses ici sont pour le démineur de Windows XP SP3 English. Elles peuvent être
différentes pour d'autres services packs.}.
Donc nous avons ceci à l'emplacement mémoire du tableau:

\lstinputlisting[style=customasmx86]{examples/minesweeper/1.lst}

\olly, comme tout autre éditeur hexadécimal, affiche 16 octets par ligne.
Donc chaque ligne de tableau de 32-octet occupe exactement 2 lignes ici.

Ceci est le niveau débutant (plateau de 9*9).

Il y a une sorte de structure carré que l'on voit ici (octets 0x10).

Nous cliquons \q{Run} dans \olly pour débloquer le processus du démineur, puis nous
cliquons au hasard dans la fenêtre du démineur et nous tombons sur une mine, mais
maintenant, toutes les mines sont visibles:

\begin{figure}[H]
\centering
\myincludegraphicsSmall{examples/minesweeper/1.png}
\caption{Mines}
\label{fig:minesweeper1}
\end{figure}

En comparant les emplacements des mines et le dump, nous pouvons en conclure que
0x10 correspond au bord, 0x0F---bloc vide, 0x8F---mine.
Peut-être que 0x10 est simplement une \emph{valeur sentinelle}.

Maintenant nous allons ajouter des commentaires et aussi mettre tous les octets à
0x8F entre parenthèses droites:

\lstinputlisting[style=customasmx86]{examples/minesweeper/2.lst}

Maintenant nous allons supprimer tous les \emph{octet de bord} (0x10) et ce qu'il
y a après:

\lstinputlisting[style=customasmx86]{examples/minesweeper/3.lst}

Oui, ce sont des mines, maintenant ça peut être vu clairement et comparé avec la
copie d'écran.

\clearpage
Ce qui est intéressant, c'est que nous pouvons modifier le tableau directement dans
\olly.
Nous pouvons supprimer toutes les mines en changeant les octets à 0x8F par 0x0F,
et voici ce que nous obtenons dans le démineur:

\begin{figure}[H]
\centering
\myincludegraphicsSmall{examples/minesweeper/3.png}
\caption{Toutes les mines sont supprimées depuis le débogueur}
\label{fig:minesweeper3}
\end{figure}

Nous pouvons aussi toutes les déplacer à la première ligne:

\begin{figure}[H]
\centering
\myincludegraphicsSmall{examples/minesweeper/2.png}
\caption{Mines mises dans le débogueur}
\label{fig:minesweeper2}
\end{figure}

Bon, le débogueur n'est pas très pratique pour espionner (ce qui est notre but),
donc nous allons écrire un petit utilitaire pour afficher le contenu du plateau:

\lstinputlisting[style=customc]{examples/minesweeper/minesweeper_cheater.c}

Simplement donner le \ac{PID}
\footnote{Le PID peut être vu dans le Task Manager
(l'activer avec \q{View $\rightarrow$ Select Columns})}
et l'adresse du tableau (\TT{0x01005340} pour Windows XP SP3 English)
et il l'affichera
\footnote{L'exécutable compilé est ici:
\href{http://go.yurichev.com/17165}{beginners.re}}.

Il s'attache à un processus win32 par le \ac{PID} et lit la mémoire du processus
à l'adresse.

\subsection{Trouver la grille automatiquement}

C'est pénible de mettre l'adresse à chaque fois que nous lançons notre utilitaire.
Aussi, différentes versions du démineur peuvent avoir le tableau à des adresses différentes.
Sachant qu'il a toujours un bord (octets 0x10), nous pouvons le trouver facilement
en mémoire:

\lstinputlisting[style=customc]{examples/minesweeper/cheater2_fragment.c}

Code source complet: \url{\RepoURL/examples/minesweeper/minesweeper_cheater2.c}.

\subsection{\Exercises}

\begin{itemize}

\item 
Pourquoi est-ce que les \emph{octets de bord} (ou \emph{valeurs sentinelles}) (0x10)
existent dans le tableau?

À quoi servent-elles si elles ne sont pas visibles dans l'interface du démineur?
Comment est-ce qu'il pourrait fonctionner sans elles?

\item 
Comme on s'en doute, il y a plus de valeurs possible (pour les blocs ouverts, ceux
flagués par l'utilisateur, etc.).
Essayez de trouver la signification de chacune d'elles.

\item 
Modifiez mon utilitaire afin qu'il puisse supprimer toutes les mines ou qu'il les
place suivant un schéma fixé de votre choix dans le démineur.

\end{itemize}

\mysection{Hacker l'horloge de Windows}

Parfois je fais des poissons d'avril à mes collègues.

Cherchons si nous pourrions faire quelque chose avec l'horloge de Windows?
Pouvons-nous la forcer à tourner à l'envers?

Tout d'abord, lorsque l'on clique sur date/time dans la barre d'état,\\
le module \emph{C:\textbackslash{}WINDOWS\textbackslash{}SYSTEM32\textbackslash{}TIMEDATE.CPL}
est exécuté, qui est un fichier exécutable \ac{PE} habituel.

Voyons d'abord comment il affiche les aiguilles.
Lorsque j'ouvre le fichier (de Windows 7) dans Resource Hacker, il y a le fond de
l'horloge, mais sans aiguille:

\begin{figure}[H]
\centering
\myincludegraphics{examples/timedate/reshack.png}
\caption{Resource Hacker}
\end{figure}

Ok, que savons-nous? Comment afficher une aiguille? Elles commencent au milieu du
cercle, s'arrêtent sur son bord.
De ce fait, nous devons calculer les coordonnées d'un point sur le bord d'un cercle.
Des mathématiques scolaires, nous pouvons nous rappeler que nous devons utiliser
les fonctions sinus/cosinus pour dessiner un cercle, ou au moins la racine carré.
Il n'y a pas de telles choses dans \emph{TIMEDATE.CPL}, au moins à première vue.
Mais grâce au fichier PDB de débogage de Microsoft, je peux trouver une fonction
appelée \emph{CAnalogClock::DrawHand()}, qui appelle \emph{Gdiplus::Graphics::DrawLine()}
au moins deux fois.

Voici le code:

\lstinputlisting[style=customasmx86]{examples/timedate/1.lst}

\myindex{Windows!Win32!MulDiv()}
Nous voyons que les arguments de \emph{DrawLine()} dépendent du résultat de la fonction
\emph{MulDiv()} et d'une table \emph{table[]} (le nom est mien), qui a des éléments
de 8-octets (regardez le second opérande de \INS{LEA}).

Qu'y a-t-il dans table[]?

\lstinputlisting[style=customasmx86]{examples/timedate/2.lst}

Elle n'est référencée que depuis la fonction \emph{DrawHand()}.
Elle a 120 mots de 32-bit ou 60 paires 32-bit... attendez, 60?
Regardons ces valeurs de plus près.
Tout d'abord, je vais remplacer 6 paires ou 12 mots de 32-bit par des zéros, et je
vais mettre le fichier \emph{TIMEDATE.CPL} modifié dans \emph{C:\textbackslash{}WINDOWS\textbackslash{}SYSTEM32}.
(Vous pourriez devoir changer le propriétaire du fichier *TIMEDATE.CPL* pour votre
compte utilisateur primaire (au lieu de \emph{TrustedInstaller}), et donc, démarrer
en mode sans échec avec la ligne de commande afin de pouvoir copier le fichier, qui
est en général bloqué.)

\begin{figure}[H]
\centering
\includegraphics[width=0.5\textwidth]{examples/timedate/6_pairs_zeroed.png}
\caption{Tentative d'exécution}
\end{figure}

Maintenant lorsqu'une aiguilles est située dans 0..5 secondes/minutes, elle est invisible!
Toutefois, la partie opposée (plus courte) de la seconde aiguille est visible et
bouge.
Lorsqu'une aiguille est en dehors de cette partie, elle est visible comme d'habitude.

\myindex{Mathematica}
Regardons d' encore plus près la table dans Mathematica.
J'ai copié/collé la table de \emph{TIMEDATE.CPL} dans un fichier \emph{tbl} (480 octets).
Nous tenons pour acquis le fait que ce sont des valeurs signées, car la moitié des
éléments sont inférieurs à zéro (0FFFFE0C1h, etc.).
Si ces valeurs étaient non signées, elles seraient étrangement grandes.

\begin{lstlisting}[style=custommath]
In[]:= tbl = BinaryReadList["~/.../tbl", "Integer32"]

Out[]= {0, -7999, 836, -7956, 1663, -7825, 2472, -7608, 3253, -7308, 3999, \
-6928, 4702, -6472, 5353, -5945, 5945, -5353, 6472, -4702, 6928, \
-4000, 7308, -3253, 7608, -2472, 7825, -1663, 7956, -836, 8000, 0, \
7956, 836, 7825, 1663, 7608, 2472, 7308, 3253, 6928, 4000, 6472, \
4702, 5945, 5353, 5353, 5945, 4702, 6472, 3999, 6928, 3253, 7308, \
2472, 7608, 1663, 7825, 836, 7956, 0, 7999, -836, 7956, -1663, 7825, \
-2472, 7608, -3253, 7308, -4000, 6928, -4702, 6472, -5353, 5945, \
-5945, 5353, -6472, 4702, -6928, 3999, -7308, 3253, -7608, 2472, \
-7825, 1663, -7956, 836, -7999, 0, -7956, -836, -7825, -1663, -7608, \
-2472, -7308, -3253, -6928, -4000, -6472, -4702, -5945, -5353, -5353, \
-5945, -4702, -6472, -3999, -6928, -3253, -7308, -2472, -7608, -1663, \
-7825, -836, -7956}

In[]:= Length[tbl]
Out[]= 120
\end{lstlisting}

Traitons deux valeurs consécutives comme une paire:

\begin{lstlisting}[style=custommath]
In[]:= pairs = Partition[tbl, 2]
Out[]= {{0, -7999}, {836, -7956}, {1663, -7825}, {2472, -7608}, \
{3253, -7308}, {3999, -6928}, {4702, -6472}, {5353, -5945}, {5945, \
-5353}, {6472, -4702}, {6928, -4000}, {7308, -3253}, {7608, -2472}, \
{7825, -1663}, {7956, -836}, {8000, 0}, {7956, 836}, {7825, 
1663}, {7608, 2472}, {7308, 3253}, {6928, 4000}, {6472, 
4702}, {5945, 5353}, {5353, 5945}, {4702, 6472}, {3999, 
6928}, {3253, 7308}, {2472, 7608}, {1663, 7825}, {836, 7956}, {0, 
7999}, {-836, 7956}, {-1663, 7825}, {-2472, 7608}, {-3253, 
7308}, {-4000, 6928}, {-4702, 6472}, {-5353, 5945}, {-5945, 
5353}, {-6472, 4702}, {-6928, 3999}, {-7308, 3253}, {-7608, 
2472}, {-7825, 1663}, {-7956, 836}, {-7999, 
0}, {-7956, -836}, {-7825, -1663}, {-7608, -2472}, {-7308, -3253}, \
{-6928, -4000}, {-6472, -4702}, {-5945, -5353}, {-5353, -5945}, \
{-4702, -6472}, {-3999, -6928}, {-3253, -7308}, {-2472, -7608}, \
{-1663, -7825}, {-836, -7956}}

In[]:= Length[pairs]
Out[]= 60
\end{lstlisting}

Essayons de traiter chaque paire comme des coordonnées X/Y et dessinons les 60 paires,
et aussi les 15 premières paires:

\begin{figure}[H]
\centering
\myincludegraphics{examples/timedate/math.png}
\caption{Mathematica}
\end{figure}

Ça donne quelque chose!
Chaque paire est juste une coordonnée.
Les 15 premières paires sont les coordonnées pour $\frac{1}{4}$ de cercle.

Peut-être que les développeurs de Microsoft ont pré-calculé toutes les coordonnées
et les ont mises dans une table.
myindex{Memoization}
Ceci est une pratique très répandue, quoique désuète -- l'accès à une table précalculée
est plus rapide que d'appeler les fonctions sinus/cosinus relativement lente\footnote{Aujourd'hui
ceci est appelé la \emph{memoïsation}}.
Les opérations sinus/cosinus ne sont plus aussi couteuses...

Maintenant, je comprends pourquoi lorsque j'ai effacé les 6 premières paires, les
aiguilles étaient invisibles dans cette zone: en fait, les aiguilles étaient dessinées,
elles avaient juste une longueur de zéro, car elles commençaient et finissaient en (0,0).

\subsubsection{La blague}

Sachant tout cela, comment serait-il possible de forcer les aiguilles à tourner à
l'envers?
En fait, ceci est simple, nous devons seulement tourner la table, afin que chaque
aiguille, au lieu d'être dessinée à l'index 0, le soit à l'index 59.

J'ai créé le modificateur il y a longtemps, au tout début des années 2000, pour Windows 2000.
Difficile à croire, il fonctionne toujours pour Windows 7, peut-être que la table
n'a pas changé depuis lors!

Code source du modificateur: \url{\RepoURL/examples/timedate/time_pt.c}.

Maintenant, je peux voir les aiguilles tourner à l'envers:

\begin{figure}[H]
\centering
\includegraphics[width=0.5\textwidth]{examples/timedate/counterclockwise.png}
\caption{Maintenant ça fonctionne}
\end{figure}

Bon, il n'y a pas d'animation dans ce livre, mais si vous y regardez de plus près,
vous pouvez voir que les aiguilles affichent en fait l'heure correcte, mais que la
surface entière de l'horloge est tournée verticalement, comme si nous la voyons depuis
l'intérieur de l'horloge.

\subsubsection{Code source divulgué de Windows 2000}

Donc, j'ai écrit le modificateur et ensuite le code source de Windows 2000 a fuité
(je ne peux toutefois pas vous obligez à me croire).
Jettons un coup d'\oe{}il au code source de cette fonction et à la table.\\
Le fichier est \emph{win2k/private/shell/cpls/utc/clock.c}:

\begin{lstlisting}[style=customc]
//
//  Array containing the sine and cosine values for hand positions.
//
POINT rCircleTable[] =
{
    { 0,     -7999},
    { 836,   -7956},
    { 1663,  -7825},
    { 2472,  -7608},
    { 3253,  -7308},
...
    { -4702, -6472},
    { -3999, -6928},
    { -3253, -7308},
    { -2472, -7608},
    { -1663, -7825},
    { -836 , -7956},
};

////////////////////////////////////////////////////////////////////////////
//
//  DrawHand
//
//  Draws the hands of the clock.
//
////////////////////////////////////////////////////////////////////////////

void DrawHand(
    HDC hDC,
    int pos,
    HPEN hPen,
    int scale,
    int patMode,
    PCLOCKSTR np)
{
    LPPOINT lppt;
    int radius;

    MoveTo(hDC, np->clockCenter.x, np->clockCenter.y);
    radius = MulDiv(np->clockRadius, scale, 100);
    lppt = rCircleTable + pos;
    SetROP2(hDC, patMode);
    SelectObject(hDC, hPen);

    LineTo( hDC,
            np->clockCenter.x + MulDiv(lppt->x, radius, 8000),
            np->clockCenter.y + MulDiv(lppt->y, radius, 8000) );
}
\end{lstlisting}

Maintenant, c'est clair: les coordonnées sont pré-calculées comme si la surface de
l'horloge avait une hauteur et une largeur de $2 \cdot 8000$, et ensuite elles sont
adaptées au rayon actuel de l'horloge en utilisant la fonction \emph{MulDiv()}.

La structure POINT\footnote{\url{https://msdn.microsoft.com/en-us/library/windows/desktop/dd162805(v=vs.85).aspx}}
est une structure de deux valeurs 32-bit, la première est \emph{x}, la seconde \emph{y}.


%\mysection{Solitaire (Windows 7): blagues}

\input{examples/solitaire/51/main_FR}
\input{examples/solitaire/53/main_FR}


\mysection{Fonction presque vide}
\label{Boolector}
\myindex{Boolector}
\myindex{x86!\Instructions!JMP}

Ceci est un morceau de code réel que j'ai trouvé dans Boolector\footnote{\url{https://boolector.github.io/}}:

\lstinputlisting[style=customc]{patterns/025_almost_empty/boolectormain.c}

Pourquoi quelqu'un ferait-il comme ça?
Je ne sais pas mais mon hypothèse est que \verb|boolector_main()| peut être compilée
dans une sorte de DLL ou bibliothèque dynamique, et appelée depuis une suite de test.
Certainement qu'une suite de test peut préparer les variables argc/argv comme
le ferait \ac{CRT}.

Il est intéressant de voir comment c'est compilé:

\lstinputlisting[caption=GCC 8.2 x64 \NonOptimizing (\assemblyOutput),style=customasmx86]{patterns/025_almost_empty/boolectormain_O0.s}

Ceci est OK, le prologue (non optimisé) déplace inutilement deux arguments,
\INS{CALL}, épilogue, \INS{RET}.
Mais regardons la version optimisée:

\lstinputlisting[caption=GCC 8.2 x64 \Optimizing (\assemblyOutput),style=customasmx86]{patterns/025_almost_empty/boolectormain_O3.s}

Aussi simple que ça: la pile et les registres ne sont pas touchés et \verb|boolector_main()|
a le même ensemble d'arguments.
Donc, tout ce que nous avons à faire est de passer l'exécution à une autre adresse.

Ceci est proche d'une \glslink{thunk function}{fonction thunk}.

Nous verons queelque chose de plus avancé plus tard: \myref{ARM_B_to_printf}, \myref{JMP_instead_of_RET}.

\input{examples/qr9/qr9_FR}
% TODO translate
% TODO: OpenSSL tool, URLs, etc
\mysection{Cas de base de données chiffrée \#1}
\label{encrypted_DB1}

(Cette partie est apparue initialement dans mon blog le 26 août 2015.
Discussion: \url{https://news.ycombinator.com/item?id=10128684}.)

\subsection{Base64 et entropie}

\myindex{XML}
J'ai un fichier \ac{XML} contenant des données chiffrées.
Peut-être est-ce relatif à des commandes et/ou des information clients.

\begin{lstlisting}
<?xml version = "1.0" encoding = "UTF-8"?>
<Orders>
	<Order>
		<OrderID>1</OrderID>
		<Data>yjmxhXUbhB/5MV45chPsXZWAJwIh1S0aD9lFn3XuJMSxJ3/E+UE3hsnH</Data>
	</Order>
	<Order>
		<OrderID>2</OrderID>
		<Data>0KGe/wnypFBjsy+U0C2P9fC5nDZP3XDZLMPCRaiBw9OjIk6Tu5U=</Data>
	</Order>
	<Order>
		<OrderID>3</OrderID>
		<Data>mqkXfdzvQKvEArdzh+zD9oETVGBFvcTBLs2ph1b5bYddExzp</Data>
	</Order>
	<Order>
		<OrderID>4</OrderID>
		<Data>FCx6JhIDqnESyT3HAepyE1BJ3cJd7wCk+APCRUeuNtZdpCvQ2MR/7kLXtfUHuA==</Data>
	</Order>
...
\end{lstlisting}

Le fichier est disponible \href{\RepoURL/examples/encrypted_DB1/encrypted.xml}{ici}.

\myindex{base64}
Ce sont clairement des données encodées en base64, car toutes les chaînes consistent
en des caractères Latin, chiffres, plus (+) et symbole slash (/).
Il peut y avoir 1 ou 2 symboles de remplissage (=), mais ils ne se trouvent jamais
au milieu d'une chaîne.
Gardez à l'esprit ces propriétés du base64, il est très facile de les reconnaître.

Décodons les et calculons l'entropie (\myref{entropy}) de ces blocs dans Wolfram Mathematica:

\begin{lstlisting}
In[]:= ListOfBase64Strings =
  Map[First[#[[3]]] &, Cases[Import["encrypted.xml"], XMLElement["Data", _, _], Infinity]];

In[]:= BinaryStrings =
  Map[ImportString[#, {"Base64", "String"}] &, ListOfBase64Strings];

In[]:= Entropies = Map[N[Entropy[2, #]] &, BinaryStrings];

In[]:= Variance[Entropies]
Out[]= 0.0238614
\end{lstlisting}

\myindex{Variance}
La variance est basse.
Cela signifie que l'entropie des valeurs ne sont pas très différentes les unes des autres.
Ceci est visible sur le graphique:

\begin{lstlisting}
In[]:= ListPlot[Entropies]
\end{lstlisting}

\begin{figure}[H]
\centering
\myincludegraphics{examples/encrypted_DB1/entropy.png}
\end{figure}

La plupart des valeurs sont entre 5.0 et 5.4.
Ceci est un signe que les données sont compressées et/ou chiffrées

Pour comprendre la variance, calculons l'entropie de toutes les liens du livre de
Conan Doyle \emph{The Hound of the Baskervilles}:

\begin{lstlisting}
In[]:= BaskervillesLines = Import["http://www.gutenberg.org/cache/epub/2852/pg2852.txt", "List"];

In[]:= EntropiesT = Map[N[Entropy[2, #]] &, BaskervillesLines];

In[]:= Variance[EntropiesT]
Out[]= 2.73883

In[]:= ListPlot[EntropiesT]
\end{lstlisting}

\begin{figure}[H]
\centering
\myincludegraphics{examples/encrypted_DB1/conan_doyle.png}
\end{figure}

La plupart des valeurs sont regroupées autour de 4, mais il y a aussi des valeurs
qui sont plus petites, et elles influencent la valeur finale de la variance.

Peut-être que les chaînes courtes ont une entropie plus petite, prenons les chaînes
courtes du livre de Conan Doyle.

\begin{lstlisting}
In[]:= Entropy[2, "Yes, sir."] // N
Out[]= 2.9477
\end{lstlisting}

Essayons encore plus petit:

\begin{lstlisting}
In[]:= Entropy[2, "Yes"] // N
Out[]= 1.58496

In[]:= Entropy[2, "No"] // N
Out[]= 1.
\end{lstlisting}

\subsection{Est-ce que les données sont compressées?}

OK, donc nos données sont compressées et/ou chiffrées.
Sont-elles compressées? Presque tous les compresseurs de données ajoutent un entête
au début, une signature ou quelque chose comme ça.
Comme on peut le voir, il n'y a pas de motifs communs au début de chaque bloc.
Il est toujours possible qu'il s'agisse d'un compresseur de données écrit à la main,
mais c'est très rare.
D'un autre côté, les algorithmes de chiffrement maison sont plus répandus, car il
est facile d'en faire un.
\myindex{memfrob()}
\myindex{ROT13}
Même des systèmes de chiffrement sans clef primitifs comme \emph{memfrob()}\footnote{\url{http://linux.die.net/man/3/memfrob}}
et ROT13 fonctionnent bien sans erreur.
C'est un gros défi d'écrire un compresseur depuis zéro, en utilisant seulement sa
fantaisie et son imagination de façon à ce qu'il n'ait pas de bugs évidents.
Certains programmeurs implémentent des fonctions de compression de données en lisant
des livres, mais ceci est aussi rare.
Les deux moyens les plus fréquents sont:
\myindex{zlib}
1) utiliser simplement la bibliothèque open-source zlib;
2) copier/coller quelque chose de quelque part.
Les algorithmes de compression open-source mettent en général une sorte d'en-tête,
ainsi que les algorithmes de sites comme \url{http://www.codeproject.com/}.

\subsection{Est-ce que les données sont chiffrées?}

Les algorithmes majeurs de chiffrement de données traitent les données par bloc.
DES---8 octets, AES---16 octets. 
Si le buffer en entrée n'est pas divisible par la taille du bloc, des zéros sont
ajoutés (ou quelque chose d'autre), afin que les donnés chiffrées soient alignées
sur la taille du bloc de l'algorithme.
Ce n'est pas notre cas.

En utilisant Wolfram Mathematica, j'ai analysé la longueur des blocs:

\begin{lstlisting}
In[]:= Counts[Map[StringLength[#] &, BinaryStrings]]
Out[]= <|42 -> 1858, 38 -> 1235, 36 -> 699, 46 -> 1151, 40 -> 1784,
 44 -> 1558, 50 -> 366, 34 -> 291, 32 -> 74, 56 -> 15, 48 -> 716,
 30 -> 13, 52 -> 156, 54 -> 71, 60 -> 3, 58 -> 6, 28 -> 4|>
\end{lstlisting}

1858 blocs ont une taille de 42 octets, 1235 blocs ont une taille de 38 octets, etc.

J'ai fait un graphe:

\begin{lstlisting}
ListPlot[Counts[Map[StringLength[#] &, BinaryStrings]]]
\end{lstlisting}

\begin{figure}[H]
\centering
\myincludegraphics{examples/encrypted_DB1/lengths.png}
\end{figure}

Donc, la plupart des blocs ont une taille entre $\textasciitilde{}36$ et $\textasciitilde{}48$.
Il y a un autre chose à remarquer: tous les blocs ont une taille paire.
Pas un bloc n'a une taille impaire.

Il y a, toutefois, des flux de chiffrement qui opèrent au niveau de l'octet ou même
du bit.

\subsection{CryptoPP}
\myindex{CryptoPP}

Le programme qui peut parcourir cette base de données chiffrées est écrit en C\#
et le code .NET est fortement obscurci.
Néanmoins, il y a une DLL avec du code x86, qui, après un bref examen, contient des
parties de la bibliothèque open-source connue CryptoPP!
(J'ai juste repéré des chaînes \q{CryptoPP} dedans.)
Maintenant, c'est très facile de trouver toutes les fonctions à l'intérieur de la
DLL car la bibliothèque CryptoPP est open-source.

\myindex{AES}
La bibliothèque CryptoPP contient beaucoup de fonctions de chiffrement, AES inclus (AKA Rijndael).
Les CPUs x86 récents possèdent des instructions dédiées à AES comme \INS{AESENC}, \INS{AESDEC}
et \INS{AESKEYGENASSIST}\footnote{\url{https://en.wikipedia.org/wiki/AES_instruction_set}}.
Elles ne font pas le chiffrement/déchiffrement complètement, mais elles font une
part significative du travail.
Et les nouvelles versions de CryptoPP les utilisent.
Par exemple, ici:
\href{https://github.com/mmoss/cryptopp/blob/2772f7b57182b31a41659b48d5f35a7b6cedd34d/src/rijndael.cpp#L1034}{1},
\href{https://github.com/mmoss/cryptopp/blob/2772f7b57182b31a41659b48d5f35a7b6cedd34d/src/rijndael.cpp#L1000}{2}.
\myindex{x86!\Instructions!AESENC}
\myindex{x86!\Instructions!AESDEC}
\myindex{tracer}
À ma surprise, lors du déchiffrement, \INS{AESENC} est exécutée, tandis que \INS{AESDEC}
ne l'est pas (j'ai vérifié avec mon utilitaire tracer, mais n'importe quel débogueur
peut être utilisé).
J'ai vérifié, si mon CPU supporte réellement les instructions AES. Certains CPUs
Intel i3 ne les supportent pas.
Et si non, la bibliothèque CryptoPP se rabat sur les fonctions implémentées de l'ancienne façon
\footnote{\url{https://github.com/mmoss/cryptopp/blob/2772f7b57182b31a41659b48d5f35a7b6cedd34d/src/rijndael.cpp#L355}}.
Mais mon CPU les supporte.
Pourquoi \INS{AESDEC} n'est pas exécuté?
Pourquoi le programme utilise le chiffrement AES pour déchiffrer la base de données?

OK, ce n'est pas un problème de trouver la fonction qui chiffre les blocs.
Elle est appelée \\
\emph{CryptoPP::Rijndael::Enc::ProcessAndXorBlock}:
\href{https://github.com/mmoss/cryptopp/blob/2772f7b57182b31a41659b48d5f35a7b6cedd34d/src/rijndael.cpp#L349}{src},
et elle peut être appelée depuis une autre fonction: \\
\emph{Rijndael::Enc::AdvancedProcessBlocks()}
\href{https://github.com/mmoss/cryptopp/blob/2772f7b57182b31a41659b48d5f35a7b6cedd34d/src/rijndael.cpp#L1179}{src},
qui, à son tour, appelle les deux fonctions: (
\href{https://github.com/mmoss/cryptopp/blob/2772f7b57182b31a41659b48d5f35a7b6cedd34d/src/rijndael.cpp#L1000}{AESNI\_Enc\_Block}
et
\href{https://github.com/mmoss/cryptopp/blob/2772f7b57182b31a41659b48d5f35a7b6cedd34d/src/rijndael.cpp#L1012}{AESNI\_Enc\_4\_Blocks}
)
qui ont les instructions  \INS{AESENC}.

Donc, a en juger par les entrailles de CryptoPP \\
\emph{CryptoPP::Rijndael::Enc::ProcessAndXorBlock()} chiffre un bloc 16-octet.
Mettons un point d'arrêt dessus et voyons ce qui se produit pendant le déchiffrement.
J'utilise à nouveau mon petit outil tracer.
Le logiciel doit déchiffrer le premier bloc de données maintenant.
Oh, à propos, voici le premier bloc de données converti de l'encodage en base64 vers
des données hexadécimale, faisons le manuellement:

\lstinputlisting{examples/encrypted_DB1/1.lst}

Voici les arguments de la fonction d'après les fichiers sources de CryptoPP:

\begin{lstlisting}
size_t Rijndael::Enc::AdvancedProcessBlocks(const byte *inBlocks, const byte *xorBlocks, byte *outBlocks, size_t length, word32 flags);
\end{lstlisting}

Donc, il y a 5 arguments. Les flags possibles sont:

\begin{lstlisting}
enum {BT_InBlockIsCounter=1, BT_DontIncrementInOutPointers=2, BT_XorInput=4, BT_ReverseDirection=8, BT_AllowParallel=16} FlagsForAdvancedProcessBlocks;
\end{lstlisting}

OK, lançons tracer sur la fonction \emph{ProcessAndXorBlock()}:

\lstinputlisting{examples/encrypted_DB1/2.lst}

Ici nous pouvons voir l'entrée de la fonction \emph{ProcessAndXorBlock()}, et sa sortie.

Ceci est la sortie de la fonction lors du premier appel:

\begin{lstlisting}
00000000: C7 39 4E 7B 33 1B D6 1F-B8 31 10 39 39 13 A5 5D ".9N{3....1.99..]"
\end{lstlisting}

Puis la fonction \emph{ProcessAndXorBlock()} est appelée avec un bloc de longueur
zéro, mais avec le flag 8 (\emph{BT\_ReverseDirection}).

Second appel:

\begin{lstlisting}
00000000: 45 00 20 00 4A 00 4F 00-48 00 4E 00 53 00 00 00 "E. .J.O.H.N.S..."
\end{lstlisting}

Maintenant, il y a des chaînes qui nous sont familières!

Troisième appel:

\begin{lstlisting}
00000000: B1 27 7F E4 9F 01 E3 81-CF C6 12 FB B9 7C F1 BC ".'...........|.."
\end{lstlisting}

La première sortie est très similaire aux 16 premiers octets du buffer chiffré.

Sortie du premier appel à \emph{ProcessAndXorBlock()}:

\begin{lstlisting}
00000000: C7 39 4E 7B 33 1B D6 1F-B8 31 10 39 39 13 A5 5D ".9N{3....1.99..]"
\end{lstlisting}

16 premiers octets du buffer chiffré:

\begin{lstlisting}
00000000: CA 39 B1 85 75 1B 84 1F F9 31 5E 39 72 13 EC 5D  .9..u....1^9r..]
\end{lstlisting}

Il y a trop d'octets égaux!
Comment le résultat du chiffrement AES peut-il être aussi similaire au buffer chiffré
alors que ceci n'est pas du chiffrement mais bien du déchiffrement?!

\subsection{Mode Cipher Feedback}

\myindex{Cipher Feedback mode}
\myindex{XOR}
La réponse est \ac{CFB}:
Dans ce mode, l'algorithme AES n'est pas utilisé comme un algorithme de chiffrement,
mais comme un dispositif qui génère des données aléatoires cryptographiquement sûres.
Le chiffrement effectif est obtenu en utilisant une simple opération XOR.

Voici l'algorithme de chiffrement (les images proviennent de Wikipédia):

\begin{figure}[H]
\centering
\myincludegraphics{examples/encrypted_DB1/601px-CFB_encryption.png}
\end{figure}

Et le déchiffrement:

\begin{figure}[H]
\centering
\myincludegraphics{examples/encrypted_DB1/601px-CFB_decryption.png}
\label{fig:CFB_decryption}
\end{figure}

Maintenant regardons: le chiffrement AES génère 16 octets (ou 128 bits) de données
\emph{aléatoires} destinées à être utilisées lors du XOR, qui nous oblige à utiliser
tous les 16 octets?
Si à la dernière itération nous n'avons qu'un octet de données, nous ne chiffrons
qu'un octet avec un octet de données \emph{aléatoires} générée.
Ceci conduit à une propriété importante du mode \ac{CFB}: les données ne doivent
pas être adaptées à une taille, des données de taille arbitraire peuvent être chiffrées
et déchiffrées.

Oh, c'est pour ça que les blocs chiffrés ne sont pas complétés.
Et c'est pourquoi l'instruction \INS{AESDEC} n'est jamais appelée.

Essayons de déchiffrer le premier bloc manuellement, en utilisant Python.
Le mode \ac{CFB} utilise aussi un \ac{IV}, comme \emph{semence} pour \ac{CSPRNG}.
Dans notre cas, l'\ac{IV} est le bloc qui est chiffré à la première itération:

\begin{lstlisting}
0038B920: 01 00 00 00 FF FF FF FF-79 C1 69 0B 67 C1 04 7D "........y.i.g..}"
\end{lstlisting}

Oh, et nous devons aussi retrouver la clef de chiffrement.
\myindex{x86!\Instructions!AESKEYGENASSIST}
Il y a \INS{AESKEYGENASSIST} dans la DLL, et elle est appelée, et elle est utilisée dans la fonction \\
\href{https://github.com/mmoss/cryptopp/blob/2772f7b57182b31a41659b48d5f35a7b6cedd34d/src/rijndael.cpp#L198}{src}.
C'est facile de la trouver dans \IDA et de mettre un point d'arrêt. Voyons:

\begin{lstlisting}
... tracer.exe -l:filename.exe bpf=filename.exe!0x435c30,args:3,dump_args:0x10

Warning: no tracer.cfg file.
PID=2068|New process software.exe
no module registered with image base 0x77320000
no module registered with image base 0x76e20000
no module registered with image base 0x77320000
no module registered with image base 0x77220000
Warning: unknown (to us) INT3 breakpoint at ntdll.dll!LdrVerifyImageMatchesChecksum+0x96c (0x776c103b)
(0) software.exe!0x435c30(0x15e8000, 0x10, 0x14f808) (called from software.exe!.text+0x22fa1 (0x13d3fa1))
Argument 1/3
015E8000: CD C5 7E AD 28 5F 6D E1-CE 8F CC 29 B1 21 88 8E "..~.(_m....).!.."
Argument 3/3
0014F808: 38 82 58 01 C8 B9 46 00-01 D1 3C 01 00 F8 14 00 "8.X...F...<....."
Argument 3/3 +0x0: software.exe!.rdata+0x5238
Argument 3/3 +0x8: software.exe!.text+0x1c101
(0) software.exe!0x435c30() -> 0x13c2801
PID=2068|Process software.exe exited. ExitCode=0 (0x0)
\end{lstlisting}

Donc, ceci est la clef: \emph{CD C5 7E AD 28 5F 6D E1-CE 8F CC 29 B1 21 88 8E}.

Durant le déchiffrement manuel, nous obtenons ceci:

\begin{lstlisting}
00000000: 0D 00 FF FE 46 00 52 00  41 00 4E 00 4B 00 49 00  ....F.R.A.N.K.I.
00000010: 45 00 20 00 4A 00 4F 00  48 00 4E 00 53 00 66 66  E. .J.O.H.N.S.ff
00000020: 66 66 66 9E 61 40 D4 07  06 01                    fff.a@....
\end{lstlisting}

Maintenant, c'est quelque chose de lisible!
Et nous comprenons pourquoi il y avait autant d'octets égaux dans la première
itération de déchiffrement:
car le text en clair a beaucoup d'octet à zéro!
Déchiffrons le second bloc:

\begin{lstlisting}
00000000: 17 98 D0 84 3A E9 72 4F  DB 82 3F AD E9 3E 2A A8  ....:.rO..?..>*.
00000010: 41 00 52 00 52 00 4F 00  4E 00 CD CC CC CC CC CC  A.R.R.O.N.......
00000020: 1B 40 D4 07 06 01                                 .@....
\end{lstlisting}

Les troisième, quatrième et cinquième:

\begin{lstlisting}
00000000: 5D 90 59 06 EF F4 96 B4  7C 33 A7 4A BE FF 66 AB  ].Y.....|3.J..f.
00000010: 49 00 47 00 47 00 53 00  00 00 00 00 00 C0 65 40  I.G.G.S.......e@
00000020: D4 07 06 01                                       ....
\end{lstlisting}

\begin{lstlisting}
00000000: D3 15 34 5D 21 18 7C 6E  AA F8 2D FE 38 F9 D7 4E  ..4]!.|n..-.8..N
00000010: 41 00 20 00 44 00 4F 00  48 00 45 00 52 00 54 00  A. .D.O.H.E.R.T.
00000020: 59 00 48 E1 7A 14 AE FF  68 40 D4 07 06 02        Y.H.z...h@....
\end{lstlisting}

\begin{lstlisting}
00000000: 1E 8B 90 0A 17 7B C5 52  31 6C 4E 2F DE 1B 27 19  .....{.R1lN...'.
00000010: 41 00 52 00 43 00 55 00  53 00 00 00 00 00 00 60  A.R.C.U.S.......
00000020: 66 40 D4 07 06 03                                 f@....
\end{lstlisting}

Tous les blocs déchiffrés semblent correct, à l'exception des 16 premiers octets.

\subsection{Initializing Vector}

Qu'est-ce qui peut affecter les 16 premiers octets?

Revenons à nouveau à l'algorithme de déchiffrement \ac{CFB}: \myref{fig:CFB_decryption}.

Nous pouvons voir que l'\ac{IV} peut affecter le déchiffrement de la première opération
de déchiffrement, mais pas la seconde, car lors de la seconde itération, le texte
chiffré de la première itération est utilisé, et en cas de déchiffrement, c'est le
même, quelque soit l'\ac{IV}!

Donc, l'\ac{IV} est sans doute différent à chaque fois.
En utilisant mon tracer, j'ai regardé la première entrée lors du déchiffrement du
second bloc du fichier \ac{XML}:

\begin{lstlisting}
0038B920: 02 00 00 00 FE FF FF FF-79 C1 69 0B 67 C1 04 7D "........y.i.g..}"
\end{lstlisting}

\dots troisième:

\begin{lstlisting}
0038B920: 03 00 00 00 FD FF FF FF-79 C1 69 0B 67 C1 04 7D "........y.i.g..}"
\end{lstlisting}

Il semble que le premier et le cinquième octet changent à chaque fois.
J'en ai finalement conclu que le premier entier 32-bit est simplement OrderID du fichier
\ac{XML}, et le second entier 32-bit est aussi OrderID, mais multiplié par -1. Tous
les 8 autres octets sont les mêmes pour chaque opération.
Maintenant, j'ai déchiffré la base de données entière:
\url{\RepoURL/examples/encrypted_DB1/decrypted.full.txt}.

Le script Python utilisé pour ceci est:
\url{\RepoURL/examples/encrypted_DB1/decrypt_blocks.py}.

Peut-être que l'auteur voulait chiffrer chaque bloc différemment, donc il a utilisé
OrderID comme une partie de la clef.
Il aurait aussi été possible de créer une clef AES différente, au lieu de l'\ac{IV}.

Dinc maintenant nous savons que l'\ac{IV} affecte seulement le premier bloc lors
du déchiffrement en mode \ac{CFB}, ceci en est une caractéristique.
Tous les autres blocs peuvent être déchiffrés sans connaître l'\ac{IV}, mais en utilisant
la clef.

OK, donc pourquoi le mode \ac{CFB}? Apparemment, parce que le tout premier exemple
sur le wiki de CryptoPP utilise le mode \ac{CFB}:
\url{http://www.cryptopp.com/wiki/Advanced_Encryption_Standard#Encrypting_and_Decrypting_Using_AES}.
On peut aussi supposer que le développeur l'a choisi pour sa simplicité:
l'exemple peut chiffrer/déchiffrer des chaînes de texte de longueur arbitraire, sans
remplissage.

Il est aussi probable que l'auteur du programme a juste copié/collé l'exemple depuis
la page wiki de CryptoPP.
Beaucoup de programmeurs font ça.

La seule différence est que l'\ac{IV} est choisi aléatoirement dans l'exemple du
wiki de CryptoPP, alors que cet indéterminisme n'était pas permis aux programmeurs
du logiciel que nous disséquons maintenant, donc ils ont choisi d'initialiser l'\ac{IV}
en utilisant OrderID.

Nous pouvons maintenant procéder à l'analyse du cas de chaque octet dans le bloc
déchiffré.

\subsection{Structure du buffer}

Prenons les quatre premier bloc déchiffrés:

\begin{lstlisting}
00000000: 0D 00 FF FE 46 00 52 00  41 00 4E 00 4B 00 49 00  ....F.R.A.N.K.I.
00000010: 45 00 20 00 4A 00 4F 00  48 00 4E 00 53 00 66 66  E. .J.O.H.N.S.ff
00000020: 66 66 66 9E 61 40 D4 07  06 01                    fff.a@....

00000000: 0B 00 FF FE 4C 00 4F 00  52 00 49 00 20 00 42 00  ....L.O.R.I. .B.
00000010: 41 00 52 00 52 00 4F 00  4E 00 CD CC CC CC CC CC  A.R.R.O.N.......
00000020: 1B 40 D4 07 06 01                                 .@....

00000000: 0A 00 FF FE 47 00 41 00  52 00 59 00 20 00 42 00  ....G.A.R.Y. .B.
00000010: 49 00 47 00 47 00 53 00  00 00 00 00 00 C0 65 40  I.G.G.S.......e@
00000020: D4 07 06 01                                       ....

00000000: 0F 00 FF FE 4D 00 45 00  4C 00 49 00 4E 00 44 00  ....M.E.L.I.N.D.
00000010: 41 00 20 00 44 00 4F 00  48 00 45 00 52 00 54 00  A. .D.O.H.E.R.T.
00000020: 59 00 48 E1 7A 14 AE FF  68 40 D4 07 06 02        Y.H.z...h@....
\end{lstlisting}

On voit clairement des chaînes de textes encodées en UTF-16, ce sont les noms et
noms de famille.
Le premier octet (ou mot de 16-bit) semble être la longueur de la chaîne, nous pouvons
vérifier visuellement.
\emph{FF FE} semble être le \ac{BOM} Unicode.

Il y a 12 autres octets après chaque chaîne.

En utilisant ce script
(\url{\RepoURL/examples/encrypted_DB1/dump_buffer_rest.py})
j'ai obtenu une sélection aléatoire de \emph{fins} (de bloc):

\lstinputlisting{examples/encrypted_DB1/tails.lst}

Nous voyons tout d'abord que les octets 0x40 et 0x07 sont présent dans chaque \emph{fin}.
Le tout dernier octet est toujours dans l'intervalle 1..0x1F (1..31), j'ai vérifié.
Le pénultième octet est toujours dans l'intervalle 1..0xC (1..12).
Ouah, ça ressemble à une date!
L'année peut être représentée comme une valeur 16-bit, et peut-être que les 4 derniers
octets sont une date (16 bits pour l'année, 8 bits pour le mois et les 8 restants
pour le jour)?
0x7DD est 2013, 0x7D5 est 2005, etc. Ça semble juste. Ceci est une date.
Il y a 8 octets supplémentaires.
À en juger par le fait que ceci est une base de données appelée \emph{orders}, peut-être
s'agit-il d'une sorte de somme ici?
J'ai essayé de les interpréter comme des réels en double précision IEEE 754 et ai
affiché toutes les valeurs!

Certaines sont:

\begin{lstlisting}
71.0
134.0
51.95
53.0
121.99
96.95
98.95
15.95
85.95
184.99
94.95
29.95
85.0
36.0
130.99
115.95
87.99
127.95
114.0
150.95
\end{lstlisting}

Ça ressemble à des nombres réels

Maintenant, nous pouvons afficher les noms, sommes et dates.

\begin{lstlisting}
plain:
00000000: 0D 00 FF FE 46 00 52 00  41 00 4E 00 4B 00 49 00  ....F.R.A.N.K.I.
00000010: 45 00 20 00 4A 00 4F 00  48 00 4E 00 53 00 66 66  E. .J.O.H.N.S.ff
00000020: 66 66 66 9E 61 40 D4 07  06 01                    fff.a@....
OrderID= 1 name= FRANKIE JOHNS sum= 140.95 date= 2004 / 6 / 1

plain:
00000000: 0B 00 FF FE 4C 00 4F 00  52 00 49 00 20 00 42 00  ....L.O.R.I. .B.
00000010: 41 00 52 00 52 00 4F 00  4E 00 CD CC CC CC CC CC  A.R.R.O.N.......
00000020: 1B 40 D4 07 06 01                                 .@....
OrderID= 2 name= LORI BARRON sum= 6.95 date= 2004 / 6 / 1

plain:
00000000: 0A 00 FF FE 47 00 41 00  52 00 59 00 20 00 42 00  ....G.A.R.Y. .B.
00000010: 49 00 47 00 47 00 53 00  00 00 00 00 00 C0 65 40  I.G.G.S.......e@
00000020: D4 07 06 01                                       ....
OrderID= 3 name= GARY BIGGS sum= 174.0 date= 2004 / 6 / 1

plain:
00000000: 0F 00 FF FE 4D 00 45 00  4C 00 49 00 4E 00 44 00  ....M.E.L.I.N.D.
00000010: 41 00 20 00 44 00 4F 00  48 00 45 00 52 00 54 00  A. .D.O.H.E.R.T.
00000020: 59 00 48 E1 7A 14 AE FF  68 40 D4 07 06 02        Y.H.z...h@....
OrderID= 4 name= MELINDA DOHERTY sum= 199.99 date= 2004 / 6 / 2

plain:
00000000: 0B 00 FF FE 4C 00 45 00  4E 00 41 00 20 00 4D 00  ....L.E.N.A. .M.
00000010: 41 00 52 00 43 00 55 00  53 00 00 00 00 00 00 60  A.R.C.U.S.......
00000020: 66 40 D4 07 06 03                                 f@....
OrderID= 5 name= LENA MARCUS sum= 179.0 date= 2004 / 6 / 3
\end{lstlisting}

En voir plus: \url{\RepoURL/examples/encrypted_DB1/decrypted.full.with_data.txt}.
Ou filtré: \url{\RepoURL/examples/encrypted_DB1/decrypted.short.txt}.
Ça semble correct.

Ceci est une sorte de sérialisation \ac{OOP}, i.e., stockant différents types de
valeurs dans un buffer binaire pour le stocker et/ou le transmettre.

\subsection{Bruit en fin de buffer}

La seule question qui reste est que, parfois, la \emph{fin} est plus longue:

\begin{lstlisting}
00000000: 0E 00 FF FE 54 00 48 00  45 00 52 00 45 00 53 00  ....T.H.E.R.E.S.
00000010: 45 00 20 00 54 00 55 00  54 00 54 00 4C 00 45 00  E. .T.U.T.T.L.E.
00000020: 66 66 66 66 66 1E 63 40  D4 07 07 1A 00 07 07 19  fffff.c@........
OrderID= 172 name= THERESE TUTTLE sum= 152.95 date= 2004 / 7 / 26
\end{lstlisting}

(Les octets \emph{00 07 07 19} ne sont pas utilisés et servent de remplissage.)

\begin{lstlisting}
00000000: 0C 00 FF FE 4D 00 45 00  4C 00 41 00 4E 00 49 00  ....M.E.L.A.N.I.
00000010: 45 00 20 00 4B 00 49 00  52 00 4B 00 00 00 00 00  E. .K.I.R.K.....
00000020: 00 20 64 40 D4 07 09 02  00 02                    . d@......
OrderID= 286 name= MELANIE KIRK sum= 161.0 date= 2004 / 9 / 2
\end{lstlisting}

(\emph{00 02} ne sont pas utilisés.)

Après un examen rigoureux, on peut voir que le but à la fin de la \emph{fin} est
juste le reste d'un chiffrement précédent!

Voici deux buffers consécutifs:

\begin{lstlisting}
00000000: 10 00 FF FE 42 00 4F 00  4E 00 4E 00 49 00 45 00  ....B.O.N.N.I.E.
00000010: 20 00 47 00 4F 00 4C 00  44 00 53 00 54 00 45 00   .G.O.L.D.S.T.E.
00000020: 49 00 4E 00 9A 99 99 99  99 79 46 40 D4 07 07 19  I.N......yF@....
OrderID= 171 name= BONNIE GOLDSTEIN sum= 44.95 date= 2004 / 7 / 25

00000000: 0E 00 FF FE 54 00 48 00  45 00 52 00 45 00 53 00  ....T.H.E.R.E.S.
00000010: 45 00 20 00 54 00 55 00  54 00 54 00 4C 00 45 00  E. .T.U.T.T.L.E.
00000020: 66 66 66 66 66 1E 63 40  D4 07 07 1A 00 07 07 19  fffff.c@........
OrderID= 172 name= THERESE TUTTLE sum= 152.95 date= 2004 / 7 / 26
\end{lstlisting}

(Les derniers octets \emph{07 07 19} sont copiés du buffer précédent.) 

Un autre exemple de deux buffers consécutifs:

\begin{lstlisting}
00000000: 0D 00 FF FE 4C 00 4F 00  52 00 45 00 4E 00 45 00  ....L.O.R.E.N.E.
00000010: 20 00 4F 00 54 00 4F 00  4F 00 4C 00 45 00 CD CC   .O.T.O.O.L.E...
00000020: CC CC CC 3C 5E 40 D4 07  09 02                    ...<^@....
OrderID= 285 name= LORENE OTOOLE sum= 120.95 date= 2004 / 9 / 2

00000000: 0C 00 FF FE 4D 00 45 00  4C 00 41 00 4E 00 49 00  ....M.E.L.A.N.I.
00000010: 45 00 20 00 4B 00 49 00  52 00 4B 00 00 00 00 00  E. .K.I.R.K.....
00000020: 00 20 64 40 D4 07 09 02  00 02                    . d@......
OrderID= 286 name= MELANIE KIRK sum= 161.0 date= 2004 / 9 / 2
\end{lstlisting}

Le dernier octet 02 a été copié du buffer en texte clair précédent.

C'est possible si le buffer utilisé lors du chiffrement est global et/ou s'il n'est
pas mis à zéro entre chaque chiffrement.
La taille du buffer final est aussi chaotique, néanmoins, le bogue reste sans conséquence
car il n'affecte pas le processus de déchiffrement, qui ignore le bruit à la fin.
C'est une erreur courante.
\myindex{OpenSSL}
\myindex{Heartbleed}
Il était présent dans OpenSSL (Heartbleed bug).

\subsection{Conclusion}

Résumé:
Chaque rétro-ingénieur pratiquant doit être familier avec la majorité des algorithmes
ainsi que la majorité des modes de chiffrement.
Quelques livres à ce sujet: \myref{crypto_books}.

Le contenu \emph{chiffré} de la base de données a été artificiellement construit
par moi, pour les besoins de la démonstration.
J'ai obtenu les nom et noms de famille les plus répandus au USA ici: \url{http://stackoverflow.com/questions/1803628/raw-list-of-person-names},
et les ai combiné aléatoirement.
Les dates et montants ont aussi été générés aléatoirement.

Tous les fichiers utilisés dans cette partie sont ici: \url{\RepoURL/examples/encrypted_DB1}.

Néanmoins, j'ai observé de telles caractéristiques dans des logiciels réels.
Cet exemple est basé dessus.

\subsection{Post Scriptum: brute-force \ac{IV}}

Le cas que vous venez de voir a été construit artificiellement, mais il est basé
sur des logiciels réels que j'ai rétro-ingénièré.
Lorsque j'ai travaillé dessus, j'ai d'abord remarqué que l'\ac{IV} avait été généré
en utilisant un nombre 32-bit, et je n'ai pas été capable de trouver un lien entre
cette valeur et OrderID.
Donc j'ai utilisé le brute-force, ce qui est aussi possible ici.

Ce n'est pas un problème d'énumérer toutes les valeurs 32-bit et d'essayer chacune
d'elle comme base pour l'\ac{IV}.
Ensuite vous déchiffrez le premier bloc de 16 octets et vérifiez les octets à zéro,
qui sont toujours à des places fixes.

\mysection{Overclocker le mineur de Bitcoin Cointerra}
\myindex{Bitcoin}
\myindex{BeagleBone}

Il y avait le mineur de Bitcoin Cointerra, ressemblant à ceci:

\begin{figure}[H]
\centering
\myincludegraphics{examples/bitcoin_miner/board.jpg}
\caption{Carte}
\end{figure}

Et il y avait aussi (peut-être leaké) l'utilitaire\footnote{Peut être téléchargé ici: \url{\RepoURL/examples/bitcoin_miner/files/cointool-overclock}}
qui peut définir la fréquence d'horloge pour la carte.
Il fonctionne sur une carte additionnelle BeagleBone Linux ARM (petite carte en bas
de l'image).

Et on m'avait demandé une fois s'il est possible de modifier cet utilitaire pour voir
quelles sont les fréquences qui peuvent être définies, et celles qui ne peuvent pas
l'être.
Et est-il possible de l'ajuster?

L'utilitaire doit être exécuté comme cela: \TT{./cointool-overclock 0 0 900}, où 900
est la fréquence en MHz.
Si la fréquence est trop grande, l'utilitaire affiche \q{Error with arguments} et
se termine.

Ceci est le morceau de code autour de la référence à la chaîne de texte \q{Error with arguments}:

\begin{lstlisting}[style=customasmARM]

...

.text:0000ABC4         STR      R3, [R11,#var_28]
.text:0000ABC8         MOV      R3, #optind
.text:0000ABD0         LDR      R3, [R3]
.text:0000ABD4         ADD      R3, R3, #1
.text:0000ABD8         MOV      R3, R3,LSL#2
.text:0000ABDC         LDR      R2, [R11,#argv]
.text:0000ABE0         ADD      R3, R2, R3
.text:0000ABE4         LDR      R3, [R3]
.text:0000ABE8         MOV      R0, R3  ; nptr
.text:0000ABEC         MOV      R1, #0  ; endptr
.text:0000ABF0         MOV      R2, #0  ; base
.text:0000ABF4         BL       strtoll
.text:0000ABF8         MOV      R2, R0
.text:0000ABFC         MOV      R3, R1
.text:0000AC00         MOV      R3, R2
.text:0000AC04         STR      R3, [R11,#var_2C]
.text:0000AC08         MOV      R3, #optind
.text:0000AC10         LDR      R3, [R3]
.text:0000AC14         ADD      R3, R3, #2
.text:0000AC18         MOV      R3, R3,LSL#2
.text:0000AC1C         LDR      R2, [R11,#argv]
.text:0000AC20         ADD      R3, R2, R3
.text:0000AC24         LDR      R3, [R3]
.text:0000AC28         MOV      R0, R3  ; nptr
.text:0000AC2C         MOV      R1, #0  ; endptr
.text:0000AC30         MOV      R2, #0  ; base
.text:0000AC34         BL       strtoll
.text:0000AC38         MOV      R2, R0
.text:0000AC3C         MOV      R3, R1
.text:0000AC40         MOV      R3, R2
.text:0000AC44         STR      R3, [R11,#third_argument]
.text:0000AC48         LDR      R3, [R11,#var_28]
.text:0000AC4C         CMP      R3, #0
.text:0000AC50         BLT      errors_with_arguments
.text:0000AC54         LDR      R3, [R11,#var_28]
.text:0000AC58         CMP      R3, #1
.text:0000AC5C         BGT      errors_with_arguments
.text:0000AC60         LDR      R3, [R11,#var_2C]
.text:0000AC64         CMP      R3, #0
.text:0000AC68         BLT      errors_with_arguments
.text:0000AC6C         LDR      R3, [R11,#var_2C]
.text:0000AC70         CMP      R3, #3
.text:0000AC74         BGT      errors_with_arguments
.text:0000AC78         LDR      R3, [R11,#third_argument]
.text:0000AC7C         CMP      R3, #0x31
.text:0000AC80         BLE      errors_with_arguments
.text:0000AC84         LDR      R2, [R11,#third_argument]
.text:0000AC88         MOV      R3, #950
.text:0000AC8C         CMP      R2, R3
.text:0000AC90         BGT      errors_with_arguments
.text:0000AC94         LDR      R2, [R11,#third_argument]
.text:0000AC98         MOV      R3, #0x51EB851F
.text:0000ACA0         SMULL    R1, R3, R3, R2
.text:0000ACA4         MOV      R1, R3,ASR#4
.text:0000ACA8         MOV      R3, R2,ASR#31
.text:0000ACAC         RSB      R3, R3, R1
.text:0000ACB0         MOV      R1, #50
.text:0000ACB4         MUL      R3, R1, R3
.text:0000ACB8         RSB      R3, R3, R2
.text:0000ACBC         CMP      R3, #0
.text:0000ACC0         BEQ      loc_ACEC
.text:0000ACC4
.text:0000ACC4 errors_with_arguments
.text:0000ACC4                                         
.text:0000ACC4         LDR      R3, [R11,#argv]
.text:0000ACC8         LDR      R3, [R3]
.text:0000ACCC         MOV      R0, R3  ; path
.text:0000ACD0         BL       __xpg_basename
.text:0000ACD4         MOV      R3, R0
.text:0000ACD8         MOV      R0, #aSErrorWithArgu ; format
.text:0000ACE0         MOV      R1, R3
.text:0000ACE4         BL       printf
.text:0000ACE8         B        loc_ADD4
.text:0000ACEC ; ------------------------------------------------------------
.text:0000ACEC
.text:0000ACEC loc_ACEC                 ; CODE XREF: main+66C
.text:0000ACEC         LDR      R2, [R11,#third_argument]
.text:0000ACF0         MOV      R3, #499
.text:0000ACF4         CMP      R2, R3
.text:0000ACF8         BGT      loc_AD08
.text:0000ACFC         MOV      R3, #0x64
.text:0000AD00         STR      R3, [R11,#unk_constant]
.text:0000AD04         B        jump_to_write_power
.text:0000AD08 ; ------------------------------------------------------------
.text:0000AD08
.text:0000AD08 loc_AD08                 ; CODE XREF: main+6A4
.text:0000AD08         LDR      R2, [R11,#third_argument]
.text:0000AD0C         MOV      R3, #799
.text:0000AD10         CMP      R2, R3
.text:0000AD14         BGT      loc_AD24
.text:0000AD18         MOV      R3, #0x5F
.text:0000AD1C         STR      R3, [R11,#unk_constant]
.text:0000AD20         B        jump_to_write_power
.text:0000AD24 ; ------------------------------------------------------------
.text:0000AD24
.text:0000AD24 loc_AD24                 ; CODE XREF: main+6C0
.text:0000AD24         LDR      R2, [R11,#third_argument]
.text:0000AD28         MOV      R3, #899
.text:0000AD2C         CMP      R2, R3
.text:0000AD30         BGT      loc_AD40
.text:0000AD34         MOV      R3, #0x5A
.text:0000AD38         STR      R3, [R11,#unk_constant]
.text:0000AD3C         B        jump_to_write_power
.text:0000AD40 ; ------------------------------------------------------------
.text:0000AD40
.text:0000AD40 loc_AD40                 ; CODE XREF: main+6DC
.text:0000AD40         LDR      R2, [R11,#third_argument]
.text:0000AD44         MOV      R3, #999
.text:0000AD48         CMP      R2, R3
.text:0000AD4C         BGT      loc_AD5C
.text:0000AD50         MOV      R3, #0x55
.text:0000AD54         STR      R3, [R11,#unk_constant]
.text:0000AD58         B        jump_to_write_power
.text:0000AD5C ; ------------------------------------------------------------
.text:0000AD5C
.text:0000AD5C loc_AD5C                 ; CODE XREF: main+6F8
.text:0000AD5C         LDR      R2, [R11,#third_argument]
.text:0000AD60         MOV      R3, #1099
.text:0000AD64         CMP      R2, R3
.text:0000AD68         BGT      jump_to_write_power
.text:0000AD6C         MOV      R3, #0x50
.text:0000AD70         STR      R3, [R11,#unk_constant]
.text:0000AD74
.text:0000AD74 jump_to_write_power                     ; CODE XREF: main+6B0
.text:0000AD74                                         ; main+6CC ...
.text:0000AD74         LDR      R3, [R11,#var_28]
.text:0000AD78         UXTB     R1, R3
.text:0000AD7C         LDR      R3, [R11,#var_2C]
.text:0000AD80         UXTB     R2, R3
.text:0000AD84         LDR      R3, [R11,#unk_constant]
.text:0000AD88         UXTB     R3, R3
.text:0000AD8C         LDR      R0, [R11,#third_argument]
.text:0000AD90         UXTH     R0, R0
.text:0000AD94         STR      R0, [SP,#0x44+var_44]
.text:0000AD98         LDR      R0, [R11,#var_24]
.text:0000AD9C         BL       write_power
.text:0000ADA0         LDR      R0, [R11,#var_24]
.text:0000ADA4         MOV      R1, #0x5A
.text:0000ADA8         BL       read_loop
.text:0000ADAC         B        loc_ADD4

...

.rodata:0000B378 aSErrorWithArgu DCB "%s: Error with arguments",0xA,0 ; DATA XREF: main+684

...

\end{lstlisting}

Les noms de fonctions étaient présents dans les informations de débogage du binaire
original, comme \TT{write\_power}, \TT{read\_loop}.
Mais j'ai nommé les labels à l'intérieur des fonctions.

\myindex{UNIX!getopt}
\myindex{strtoll()}
Le nom \TT{optind} semble familier. Il provient de la bibliothèque *NIX \emph{getopt}
qui sert à traiter les arguments de la ligne de commande---bien, c'est exactement
ce qui se passe dans ce code.
Ensuite, le 3ème argument (où la valeur de la fréquence est passée) est converti
d'une chaîne vers un nombre en utilisant un appel à la fonction \emph{strtoll()}.

La valeur est ensuite encore comparée par rapport à diverses constantes.
En 0xACEC, elle est testée, si elle est inférieure ou égale à 499, et si c'est le
cas, 0x64 est passé à la fonction \TT{write\_power()} (qui envoie une commande par
USB en utilisant \TT{send\_msg()}).
Si elle est plus grande que 499, un saut en 0xAD08 se produit.

En 0xAD08 on teste si elle est inférieure ou égale à 799. En cas de succès 0x5F est
alors passé à la fonction \TT{write\_power()}.

Il y a d'autres tests: par rapport à 899 en 0xAD24, à 0x999 en 0xAD40 et enfin,
à 1099 en 0xAD5C.
Si la fréquence est inférieure ou égale à 1099, 0x50 est passé (en 0xAD6C) à la fonction
\TT{write\_power()}.
Et il y a une sorte de bug.
Si la valeur est encore plus grande que 1099, la valeur elle-même est passée à la
fonction \TT{write\_power()}.
Oh, ce n'est pas un bug, car nous ne pouvons pas arriver là: la valeur est d'abord
comparée à 950 en 0xAC88, et si elle est plus grande, un message d'erreur est
affiché et l'utilitaire s'arrête.

Maintenant, la table des fréquences en MHz et la valeur passée à la fonction \TT{write\_power()}:

\begin{center}
\begin{longtable}{ | l | l | l | }
\hline
\HeaderColor MHz & \HeaderColor héxadecimal & \HeaderColor décimal \\
\hline
499MHz & 0x64 & 100 \\
\hline
799MHz & 0x5f & 95 \\
\hline
899MHz & 0x5a & 90 \\
\hline
999MHz & 0x55 & 85 \\
\hline
1099MHz & 0x50 & 80 \\
\hline
\end{longtable}
\end{center}

Il semble que la valeur passée à la carte décroît lorsque la fréquence croît.

Maintenant, nous voyons que la valeur de 950MHz est codée en dur, au moins dans cet
utilitaire. Pouvons-nous le truquer?

Retournons à ce morceau de code:

\begin{lstlisting}[style=customasmARM]
.text:0000AC84      LDR     R2, [R11,#third_argument]
.text:0000AC88      MOV     R3, #950
.text:0000AC8C      CMP     R2, R3
.text:0000AC90      BGT     errors_with_arguments ; j'ai modifié ici en 00 00 00 00
\end{lstlisting}

Nous devons désactiver l'instruction de branchement \INS{BGT} en 0xAC90. Et ceci est
du ARM en mode ARM, car, comme on le voit, toutes les adresses augmentent par 4,
i.e, chaque instruction a une taille de 4 octets.
L'instruction \TT{NOP} (no operation) en mode ARM est juste quatre octets à zéro:
\TT{00 00 00 00}.
Donc en écrivant quatre octets à zéro à l'adresse 0xAC90 (ou à l'offset 0x2C90 dans
le fichier), nous pouvons désactiver le test.

Maintenant, il est possible de définir la fréquence jusqu'à 1050MHz. Et même plus,
mais, à cause du bug, si la valeur en entrée est plus grande que 1099, la valeur
\emph{telle quelle} en MHz sera passée à la carte, ce qui est incorrect.

Je ne suis  pas allé plus loin, mais si je devais, j'essayerai de diminuer la valeur
qui est passée à la fonction \TT{write\_power()}.

Maintenant, le morceau de code effrayant que j'ai passé en premier:

\lstinputlisting[style=customasmARM]{examples/bitcoin_miner/tmp1.lst}

La division via la multiplication est utilisée ici, et la constante est 0x51EB851F.
Je me suis écrit un petit calculateur pour programmeur\footnote{\url{https://github.com/DennisYurichev/progcalc}}.
Et il est capable de calculer le modulo inverse.

\begin{lstlisting}
modinv32(0x51EB851F)
Warning, result is not integer: 3.125000
(unsigned) dec: 3 hex: 0x3 bin: 11
\end{lstlisting}

Cela signifie que l'instruction \INS{SMULL} en 0xACA0 divise le 3ème argument par 3.125.
En fait, tout ce que la fonction \TT{modinv32()} de mon calculateur fait est ceci:

\[
\frac{1}{\frac{input}{2^{32}}} = \frac{2^{32}}{input}
\]

Ensuite il y a es décalages additionnels et maintenant nous voyons que le 3ème argument
est simplement divisé par 50.
Et ensuite il à nouveau multiplié par 50.
Pourquoi?
Ceci est un simple test, pour savoir si la valeur entrée est divisible par 50.
Si la valeur de cette expression est non nulle, $x$ n'est pas divisible par 50:

\[
x-((\frac{x}{50}) \cdot 50)
\]

Ceci est en fait une manière simple de calculer le reste de la division.

Et alors, si le reste est non nul, un message d'erreur est affiché.
Donc cet utilitaire prend des fréquences comme 850, 900, 950, 1000, etc., mais pas 855 ou 911.

C'est ça! Si vous faites quelque chose comme ça, soyez avertis que vous pouvez endommager
votre carte, tout comme en cas d'overclocking d'autres éléments comme les \ac{CPU}s,
\ac{GPU}s, etc.
Si vous avez une carte Cointerra, faites ceci à votre propre risque!


\mysection{Casser le simple exécutable cryptor}

J'ai un fichier exécutable qui est chiffré par un chiffrement relativement simple.
\href{\GitHubBlobMasterURL/examples/simple_exec_crypto/files/cipher.bin}{Il est ici}
(seule la section exécutable est laissée ici).

Tout d'abord, tout ce que fait la fonction de chiffrement, c'est d'ajouter l'index
de la position dans le buffer à l'octet.
Voici comment ça peut être implémenté en Python:

\begin{lstlisting}[caption=Python script,style=custompy]
#!/usr/bin/env python
def e(i, k):
    return chr ((ord(i)+k) % 256)

def encrypt(buf):
    return e(buf[0], 0)+ e(buf[1], 1)+ e(buf[2], 2) + e(buf[3], 3)+ e(buf[4], 4)+ e(buf[5], 5)+ e(buf[6], 6)+ e(buf[7], 7)+
           e(buf[8], 8)+ e(buf[9], 9)+ e(buf[10], 10)+ e(buf[11], 11)+ e(buf[12], 12)+ e(buf[13], 13)+ e(buf[14], 14)+ e(buf[15], 15)
\end{lstlisting}

Ainsi, si vous chiffrez un buffer avec 16 zéros, vous obtiendrez \emph{0, 1, 2, 3 ... 12, 13, 14, 15}.

\myindex{Propagating Cipher Block Chaining}
La Propagating Cipher Block Chaining (PCBC) est aussi utilisée, voici comment elle
fonctionne:

\begin{figure}[H]
\centering
\myincludegraphics{examples/simple_exec_crypto/601px-PCBC_encryption.png}
\caption{Chiffrement avec Propagating Cipher Block Chaining (l'image provient d'un article Wikipédia)}
\end{figure}

Le problème est qu'il est trop ennuyant de retrouver l'IV (Initialization Vector)
à chaque fois.
La force brute n'est pas une option, car l'IV est trop long (16 octets).
Voyons s'il est possible de recouvrer l'IV pour un fichier binaire exécutable arbitraire?

Essayons la simple analyse de fréquence.
Ceci est du code exécutable 32-bit x86, donc collectons des statistiques sur les
octets et les opcodes les plus fréquents.
J'ai essayé le fichier géant oracle.exe d' Oracle RDBMS version 11.2 pour windows
x86 et j'ai trouvé que l'octet le plus fréquent (pas de surprise) est zéro (~10\%).
L'octet suivant le plus fréquent est (encore une fois, sans surprise) 0xFF (~5\%).
Le suivant est 0x8B (~5\%).

\myindex{x86!\Instructions!MOV}
0x8B est l'opcode de \INS{MOV}, ceci est en effet l'une des instructions x86 les
plus fréquentes.
Maintenant, que dire de la popularité de l'octet zéro?
Si le compilateur doit encoder une valeur plus grande que 127, il doit utiliser un
déplacement 32-bit au lieu d'un de 8-bit, mais les grandes valeurs sont très rares,
donc il est complété par des zéros.
\myindex{x86!\Instructions!LEA}
\myindex{x86!\Instructions!PUSH}
\myindex{x86!\Instructions!CALL}
C'est le cas au moins avec \INS{LEA}, \INS{MOV}, \INS{PUSH}, \INS{CALL}.

Par exemple:

\begin{lstlisting}[style=customasmx86]
8D B0 28 01 00 00                 lea     esi, [eax+128h]
8D BF 40 38 00 00                 lea     edi, [edi+3840h]
\end{lstlisting}

Les déplacements plus grand que 127 sont très fréquents, mais ils excèdent rarement
0x10000 (en effet, des buffers mémoire/structures aussi grands sont aussi rares).

Même chose avec \INS{MOV}, les grandes constantes sont rares, les plus utilisées sont
0, 1, 10, 100, $2^n$, et ainsi de suite.
Le compilateur doit compléter les petites constantes avec des zéros pour les encoder
comme des valeurs 32-bit:

\begin{lstlisting}[style=customasmx86]
BF 02 00 00 00                    mov     edi, 2
BF 01 00 00 00                    mov     edi, 1
\end{lstlisting}

Maintenant parlons des octets 00 et FF combinés: les sauts (conditionnels inclus)
et appels peuvent transférer le flux d'exécution en avant ou en arrière, mais très
souvent, dans les limites du module exécutable courant.
Si c'est en avant, le déplacement n'est pas très grand et il y a des zéros ajoutés.
Si c'est en arrière, le déplacement est représenté par une valeur négative, donc
complétée par des octets FF.
Par exemple, transfert du flux d'exécution en avant:

\begin{lstlisting}[style=customasmx86]
E8 43 0C 00 00                    call    _function1
E8 5C 00 00 00                    call    _function2
0F 84 F0 0A 00 00                 jz      loc_4F09A0
0F 84 EB 00 00 00                 jz      loc_4EFBB8
\end{lstlisting}

En arrière:

\begin{lstlisting}[style=customasmx86]
E8 79 0C FE FF                    call    _function1
E8 F4 16 FF FF                    call    _function2
0F 84 F8 FB FF FF                 jz      loc_8212BC
0F 84 06 FD FF FF                 jz      loc_FF1E7D
\end{lstlisting}

L'octet FF se rencontre aussi très souvent dans des déplacements négatifs, comme
ceux-ci:

\begin{lstlisting}[style=customasmx86]
8D 85 1E FF FF FF                 lea     eax, [ebp-0E2h]
8D 95 F8 5C FF FF                 lea     edx, [ebp-0A308h]
\end{lstlisting}

Jusqu'ici, tout va bien. Maintenant nous devons essayer diverses clefs 16-octet, déchiffrer
la section exécutable et mesurer les occurences des octets 00, FF et 8B.
Gardons en vue la façon dont le déchiffrement PCBC fonctionne:

\begin{figure}[H]
\centering
\myincludegraphics{examples/simple_exec_crypto/640px-PCBC_decryption.png}
\caption{Propagating Cipher Block Chaining decryption (l'image provient d'un article Wikipédia)}
\end{figure}

La bonne nouvelle est que nous n'avons pas vraiment besoin de déchiffrer l'ensemble
des données, mais seulement slice par slice, ceci est exactement comment j'ai procédé
dans mon exemple précédent: \myref{XOR_mask_2}.

Maintenant j'essaye tous les octets possible (0..255) pour chaque octet dans la clef
et je prends l'octet produisant le plus grande nombre d'octets 00/FF/8B dans le slice
déchiffré:

\begin{lstlisting}[style=custompy]
#!/usr/bin/env python
import sys, hexdump, array, string, operator

KEY_LEN=16

def chunks(l, n):
    # split n by l-byte chunks
    # https://stackoverflow.com/q/312443
    n = max(1, n)
    return [l[i:i + n] for i in range(0, len(l), n)]

def read_file(fname):
    file=open(fname, mode='rb')
    content=file.read()
    file.close()
    return content

def decrypt_byte (c, key):
    return chr((ord(c)-key) % 256)

def XOR_PCBC_step (IV, buf, k):
    prev=IV
    rt=""
    for c in buf:
	new_c=decrypt_byte(c, k)
        plain=chr(ord(new_c)^ord(prev))
	prev=chr(ord(c)^ord(plain))
	rt=rt+plain
    return rt

each_Nth_byte=[""]*KEY_LEN

content=read_file(sys.argv[1])
# split input by 16-byte chunks:
all_chunks=chunks(content, KEY_LEN)
for c in all_chunks:
    for i in range(KEY_LEN):
        each_Nth_byte[i]=each_Nth_byte[i] + c[i]

# try each byte of key
for N in range(KEY_LEN):
    print "N=", N
    stat={}
    for i in range(256):
        tmp_key=chr(i)
	tmp=XOR_PCBC_step(tmp_key,each_Nth_byte[N], N)
        # count 0, FFs and 8Bs in decrypted buffer:
	important_bytes=tmp.count('\x00')+tmp.count('\xFF')+tmp.count('\x8B')
	stat[i]=important_bytes
    sorted_stat = sorted(stat.iteritems(), key=operator.itemgetter(1), reverse=True)
    print sorted_stat[0]
\end{lstlisting}

(Le code source peut être téléchargé
\href{\GitHubBlobMasterURL/examples/simple_exec_crypto/files/decrypt.py}{ici}.)

Je le lance et voici une clef pour laquelle le nombre d'octets 00/FF/8B dans le buffer
déchiffré est maximum:

\begin{lstlisting}
N= 0
(147, 1224)
N= 1
(94, 1327)
N= 2
(252, 1223)
N= 3
(218, 1266)
N= 4
(38, 1209)
N= 5
(192, 1378)
N= 6
(199, 1204)
N= 7
(213, 1332)
N= 8
(225, 1251)
N= 9
(112, 1223)
N= 10
(143, 1177)
N= 11
(108, 1286)
N= 12
(10, 1164)
N= 13
(3, 1271)
N= 14
(128, 1253)
N= 15
(232, 1330)
\end{lstlisting}

Écrivons un utilitaire de déchiffrement avec la clef obtenue:

\begin{lstlisting}[style=custompy]
#!/usr/bin/env python
import sys, hexdump, array

def xor_strings(s,t):
    # \verb|https://en.wikipedia.org/wiki/XOR_cipher#Example_implementation|
    """xor two strings together"""
    return "".join(chr(ord(a)^ord(b)) for a,b in zip(s,t))

IV=array.array('B', [147, 94, 252, 218, 38, 192, 199, 213, 225, 112, 143, 108, 10, 3, 128, 232]).tostring()

def chunks(l, n):
    n = max(1, n)
    return [l[i:i + n] for i in range(0, len(l), n)]

def read_file(fname):
    file=open(fname, mode='rb')
    content=file.read()
    file.close()
    return content

def decrypt_byte(i, k):
    return chr ((ord(i)-k) % 256)

def decrypt(buf):
    return "".join(decrypt_byte(buf[i], i) for i in range(16))

fout=open(sys.argv[2], mode='wb')

prev=IV
content=read_file(sys.argv[1])
tmp=chunks(content, 16)
for c in tmp:
    new_c=decrypt(c)
    p=xor_strings (new_c, prev)
    prev=xor_strings(c, p)
    fout.write(p)
fout.close()
\end{lstlisting}

(Le code source peut être téléchargé
\href{\GitHubBlobMasterURL/examples/simple_exec_crypto/files/decrypt2.py}
{ici}.)

Vérifions le fichier résultant:

\lstinputlisting{examples/simple_exec_crypto/objdump_result.txt}

Oui, ceci semble être un morceau correctement désassemblé de code x86.
Le fichier déchiffré entier peut être téléchargé
\href{\GitHubBlobMasterURL/examples/simple_exec_crypto/files/decrypted.bin}{ici}.

En fait, ceci est la section text du regedit.exe de Windows 7.
Mais cet exemple est basé sur un cas réel que j'ai rencontré, seul l'exécutable est
différent (et la clef), l'algorithme est le même.

\subsection{Autres idées à prendre en considération}

Et si j'avais échoué avec cette simple analyse des fréquences?
Il y a d'autres idées sur la façon de mesurer l'exactitude de code x86 déchiffré/décompressé:

\begin{itemize}

\item Les compilateurs modernes alignent les fonctions sur une limite de 0x10.
Donc l'espace libre avant est rempli avec de NOPs (0x90) ou d'autres instructions
avec des opcodes connus: \myref{sec:npad}.

\item Peut-être que le pattern le plus fréquent dans tout langage d'assemblage est
l'appel de fonction:\\
\TT{PUSH chain / CALL / ADD ESP, X}.
Cette séquence peut facilement être détectée et trouvée.
J'ai même collecté des statistiques sur le nombre moyen d'arguments des fonctions: \myref{args_stat}.
(Ainsi, ceci est la longueur moyenne d'une chaîne PUSH.)

\end{itemize}

En savoir plus sur le code désassemblé incorrectement/correctement: \myref{ISA_detect}.


\EN{\input{patterns/016_empty_redux/main_EN}}%
\FR{\input{patterns/016_empty_redux/main_FR}}

\EN{\input{patterns/016_empty_redux/main_EN}}%
\FR{\input{patterns/016_empty_redux/main_FR}}

\mysection{Fonction presque vide}
\label{Boolector}
\myindex{Boolector}
\myindex{x86!\Instructions!JMP}

Ceci est un morceau de code réel que j'ai trouvé dans Boolector\footnote{\url{https://boolector.github.io/}}:

\lstinputlisting[style=customc]{patterns/025_almost_empty/boolectormain.c}

Pourquoi quelqu'un ferait-il comme ça?
Je ne sais pas mais mon hypothèse est que \verb|boolector_main()| peut être compilée
dans une sorte de DLL ou bibliothèque dynamique, et appelée depuis une suite de test.
Certainement qu'une suite de test peut préparer les variables argc/argv comme
le ferait \ac{CRT}.

Il est intéressant de voir comment c'est compilé:

\lstinputlisting[caption=GCC 8.2 x64 \NonOptimizing (\assemblyOutput),style=customasmx86]{patterns/025_almost_empty/boolectormain_O0.s}

Ceci est OK, le prologue (non optimisé) déplace inutilement deux arguments,
\INS{CALL}, épilogue, \INS{RET}.
Mais regardons la version optimisée:

\lstinputlisting[caption=GCC 8.2 x64 \Optimizing (\assemblyOutput),style=customasmx86]{patterns/025_almost_empty/boolectormain_O3.s}

Aussi simple que ça: la pile et les registres ne sont pas touchés et \verb|boolector_main()|
a le même ensemble d'arguments.
Donc, tout ce que nous avons à faire est de passer l'exécution à une autre adresse.

Ceci est proche d'une \glslink{thunk function}{fonction thunk}.

Nous verons queelque chose de plus avancé plus tard: \myref{ARM_B_to_printf}, \myref{JMP_instead_of_RET}.

\EN{\input{patterns/016_empty_redux/main_EN}}%
\FR{\input{patterns/016_empty_redux/main_FR}}

%\mysection{Fonction presque vide}
\label{Boolector}
\myindex{Boolector}
\myindex{x86!\Instructions!JMP}

Ceci est un morceau de code réel que j'ai trouvé dans Boolector\footnote{\url{https://boolector.github.io/}}:

\lstinputlisting[style=customc]{patterns/025_almost_empty/boolectormain.c}

Pourquoi quelqu'un ferait-il comme ça?
Je ne sais pas mais mon hypothèse est que \verb|boolector_main()| peut être compilée
dans une sorte de DLL ou bibliothèque dynamique, et appelée depuis une suite de test.
Certainement qu'une suite de test peut préparer les variables argc/argv comme
le ferait \ac{CRT}.

Il est intéressant de voir comment c'est compilé:

\lstinputlisting[caption=GCC 8.2 x64 \NonOptimizing (\assemblyOutput),style=customasmx86]{patterns/025_almost_empty/boolectormain_O0.s}

Ceci est OK, le prologue (non optimisé) déplace inutilement deux arguments,
\INS{CALL}, épilogue, \INS{RET}.
Mais regardons la version optimisée:

\lstinputlisting[caption=GCC 8.2 x64 \Optimizing (\assemblyOutput),style=customasmx86]{patterns/025_almost_empty/boolectormain_O3.s}

Aussi simple que ça: la pile et les registres ne sont pas touchés et \verb|boolector_main()|
a le même ensemble d'arguments.
Donc, tout ce que nous avons à faire est de passer l'exécution à une autre adresse.

Ceci est proche d'une \glslink{thunk function}{fonction thunk}.

Nous verons queelque chose de plus avancé plus tard: \myref{ARM_B_to_printf}, \myref{JMP_instead_of_RET}.

%% TODO \mysection{Fonction presque vide}
\label{Boolector}
\myindex{Boolector}
\myindex{x86!\Instructions!JMP}

Ceci est un morceau de code réel que j'ai trouvé dans Boolector\footnote{\url{https://boolector.github.io/}}:

\lstinputlisting[style=customc]{patterns/025_almost_empty/boolectormain.c}

Pourquoi quelqu'un ferait-il comme ça?
Je ne sais pas mais mon hypothèse est que \verb|boolector_main()| peut être compilée
dans une sorte de DLL ou bibliothèque dynamique, et appelée depuis une suite de test.
Certainement qu'une suite de test peut préparer les variables argc/argv comme
le ferait \ac{CRT}.

Il est intéressant de voir comment c'est compilé:

\lstinputlisting[caption=GCC 8.2 x64 \NonOptimizing (\assemblyOutput),style=customasmx86]{patterns/025_almost_empty/boolectormain_O0.s}

Ceci est OK, le prologue (non optimisé) déplace inutilement deux arguments,
\INS{CALL}, épilogue, \INS{RET}.
Mais regardons la version optimisée:

\lstinputlisting[caption=GCC 8.2 x64 \Optimizing (\assemblyOutput),style=customasmx86]{patterns/025_almost_empty/boolectormain_O3.s}

Aussi simple que ça: la pile et les registres ne sont pas touchés et \verb|boolector_main()|
a le même ensemble d'arguments.
Donc, tout ce que nous avons à faire est de passer l'exécution à une autre adresse.

Ceci est proche d'une \glslink{thunk function}{fonction thunk}.

Nous verons queelque chose de plus avancé plus tard: \myref{ARM_B_to_printf}, \myref{JMP_instead_of_RET}.


\mysection{Autres exemples}

Un exemple à propos de Z3 et de la décompilation manuelle se trouvait ici.
Il est (temporairement) déplacé là:
\url{https://yurichev.com/writings/SAT_SMT_by_example.pdf}.

}
\RU{\chapter{Примеры разбора закрытых (проприетарных) форматов файлов}}
\EN{\chapter{Examples of reversing proprietary file formats}}
\DE{\chapter{Beispiele für das Reverse Engineering proprietärer Dateiformate}}
\FR{\chapter{Exemples de Reverse Engineering de format de fichier propriétaire}}

% sections
\mysection{\RU{Примитивное XOR-шифрование}\EN{Primitive XOR-encryption}\FR{Chiffrement primitif avec XOR}\DEph{}}
\label{simple_XOR_encryption}

\ifdefined\RUSSIAN
В русскоязычной литературе также используется термин \emph{гаммирование}.
\fi

% subsections
\EN{\input{ff/XOR/simplest/main_EN}}
\FR{\input{ff/XOR/simplest/main_FR}}
\DE{\input{ff/XOR/simplest/main_DE}}

\EN{\input{ff/XOR/ng/main_EN}}
\RU{\input{ff/XOR/ng/main_RU}}
\FR{\input{ff/XOR/ng/main_FR}}

\EN{\input{ff/XOR/4byte/main_EN}}
\RU{\input{ff/XOR/4byte/main_RU}}
\FR{\input{ff/XOR/4byte/main_FR}}

\EN{\input{ff/XOR/mask_1/main_EN}}
\RU{\input{ff/XOR/mask_1/main_RU}}
\FR{\input{ff/XOR/mask_1/main_FR}}

\EN{\input{ff/XOR/mask_2/main_EN}}
\RU{\input{ff/XOR/mask_2/main_RU}}
\FR{\input{ff/XOR/mask_2/main_FR}}

\ifdefined\RUSSIAN
\subsection{Домашнее задание}

Очень древняя текстовая игра под MS-DOS конца 80-х.
Чтобы скрыть информацию об игре от игрока, файлы данных, скорее всего, чем-то про-XOR-ены:
\url{https://beginners.re/homework/XOR_crypto_1/destiny.zip}.
Попробуйте разобраться...
\fi

\ifdefined\ENGLISH
\subsection{Homework}

An ancient text adventure for MS-DOS, developed in the end of 1980's.
To conceal game information from player, data files, most likely, XOR-ed with something:
\url{https://beginners.re/homework/XOR_crypto_1/destiny.zip}.
Try to get into...
\fi

\ifdefined\FRENCH
\subsection{Devoir}

Un ancien jeu d'aventure en texte pour MS-DOS, développé à la fin des années 1980.
Pour cacher les informations du jeu aux joueurs, les fichiers de données sont,
le plus probablement, XORé avec quelque chose:
\url{https://beginners.re/homework/XOR_crypto_1/destiny.zip}.
Essayez d'y rentrer\dots
\fi



\EN{% TODO translate
\mysection{Breaking simple executable cryptor}

I've got an executable file which is encrypted by relatively simple encryption.
\href{\GitHubBlobMasterURL/examples/simple_exec_crypto/files/cipher.bin}{Here is it} (only executable section is left here).

First, all encryption function does is just adds number of position in buffer to the byte.
Here is how this can be encoded in Python:

\begin{lstlisting}[caption=Python script,style=custompy]
#!/usr/bin/env python
def e(i, k):
    return chr ((ord(i)+k) % 256)

def encrypt(buf):
    return e(buf[0], 0)+ e(buf[1], 1)+ e(buf[2], 2) + e(buf[3], 3)+ e(buf[4], 4)+ e(buf[5], 5)+ e(buf[6], 6)+ e(buf[7], 7)+
           e(buf[8], 8)+ e(buf[9], 9)+ e(buf[10], 10)+ e(buf[11], 11)+ e(buf[12], 12)+ e(buf[13], 13)+ e(buf[14], 14)+ e(buf[15], 15)
\end{lstlisting}

Hence, if you encrypt buffer with 16 zeros, you'll get \emph{0, 1, 2, 3 ... 12, 13, 14, 15}.

\myindex{Propagating Cipher Block Chaining}
Propagating Cipher Block Chaining (PCBC) is also used, here is how it works:

\begin{figure}[H]
\centering
\myincludegraphics{examples/simple_exec_crypto/601px-PCBC_encryption.png}
\caption{Propagating Cipher Block Chaining encryption (image is taken from Wikipedia article)}
\end{figure}

The problem is that it's too boring to recover IV (Initialization Vector) each time.
Brute-force is also not an option, because IV is too long (16 bytes).
Let's see, if it's possible to recover IV for arbitrary encrypted executable file?

Let's try simple frequency analysis.
This is 32-bit x86 executable code, so let's gather statistics about most frequent bytes and opcodes.
I tried huge oracle.exe file from Oracle RDBMS version 11.2 for windows x86 and I've found that the most frequent byte (no surprise) is zero (~10\%).
The next most frequent byte is (again, no surprise) 0xFF (~5\%).
The next is 0x8B (~5\%).

\myindex{x86!\Instructions!MOV}
0x8B is opcode for \INS{MOV}, this is indeed one of the most busy x86 instructions.
Now what about popularity of zero byte?
If compiler needs to encode value bigger than 127, it has to use 32-bit displacement instead of 8-bit one, but large values are very rare,
so it is padded by zeros.
\myindex{x86!\Instructions!LEA}
\myindex{x86!\Instructions!PUSH}
\myindex{x86!\Instructions!CALL}
This is at least in \INS{LEA}, \INS{MOV}, \INS{PUSH}, \INS{CALL}.

For example:

\begin{lstlisting}[style=customasmx86]
8D B0 28 01 00 00                 lea     esi, [eax+128h]
8D BF 40 38 00 00                 lea     edi, [edi+3840h]
\end{lstlisting}

Displacements bigger than 127 are very popular, but they are rarely exceeds 0x10000
(indeed, such large memory buffers/structures are also rare).

Same story with \INS{MOV}, large constants are rare, the most heavily used are 0, 1, 10, 100, $2^n$, and so on.
Compiler has to pad small constants by zeros to represent them as 32-bit values:

\begin{lstlisting}[style=customasmx86]
BF 02 00 00 00                    mov     edi, 2
BF 01 00 00 00                    mov     edi, 1
\end{lstlisting}

Now about 00 and FF bytes combined: jumps (including conditional) and calls can pass execution flow forward or backwards, but very often,
within the limits of the current executable module.
If forward, displacement is not very big and also padded with zeros.
If backwards, displacement is represented as negative value, so padded with FF bytes.
For example, transfer execution flow forward:

\begin{lstlisting}[style=customasmx86]
E8 43 0C 00 00                    call    _function1
E8 5C 00 00 00                    call    _function2
0F 84 F0 0A 00 00                 jz      loc_4F09A0
0F 84 EB 00 00 00                 jz      loc_4EFBB8
\end{lstlisting}

Backwards:

\begin{lstlisting}[style=customasmx86]
E8 79 0C FE FF                    call    _function1
E8 F4 16 FF FF                    call    _function2
0F 84 F8 FB FF FF                 jz      loc_8212BC
0F 84 06 FD FF FF                 jz      loc_FF1E7D
\end{lstlisting}

FF byte is also very often occurred in negative displacements like these:

\begin{lstlisting}[style=customasmx86]
8D 85 1E FF FF FF                 lea     eax, [ebp-0E2h]
8D 95 F8 5C FF FF                 lea     edx, [ebp-0A308h]
\end{lstlisting}

So far so good. Now we have to try various 16-byte keys, decrypt executable section and measure how often 00, FF and 8B bytes are occurred.
Let's also keep in sight how PCBC decryption works:

\begin{figure}[H]
\centering
\myincludegraphics{examples/simple_exec_crypto/640px-PCBC_decryption.png}
\caption{Propagating Cipher Block Chaining decryption (image is taken from Wikipedia article)}
\end{figure}

The good news is that we don't really have to decrypt whole piece of data, but only slice by slice, this is exactly how I did in my previous example: \myref{XOR_mask_2}.

Now I'm trying all possible bytes (0..255) for each byte in key and just pick the byte producing maximal amount of 00/FF/8B bytes in a decrypted slice:

\begin{lstlisting}[style=custompy]
#!/usr/bin/env python
import sys, hexdump, array, string, operator

KEY_LEN=16

def chunks(l, n):
    # split n by l-byte chunks
    # https://stackoverflow.com/q/312443
    n = max(1, n)
    return [l[i:i + n] for i in range(0, len(l), n)]

def read_file(fname):
    file=open(fname, mode='rb')
    content=file.read()
    file.close()
    return content

def decrypt_byte (c, key):
    return chr((ord(c)-key) % 256)

def XOR_PCBC_step (IV, buf, k):
    prev=IV
    rt=""
    for c in buf:
	new_c=decrypt_byte(c, k)
        plain=chr(ord(new_c)^ord(prev))
	prev=chr(ord(c)^ord(plain))
	rt=rt+plain
    return rt

each_Nth_byte=[""]*KEY_LEN

content=read_file(sys.argv[1])
# split input by 16-byte chunks:
all_chunks=chunks(content, KEY_LEN)
for c in all_chunks:
    for i in range(KEY_LEN):
        each_Nth_byte[i]=each_Nth_byte[i] + c[i]

# try each byte of key
for N in range(KEY_LEN):
    print "N=", N
    stat={}
    for i in range(256):
        tmp_key=chr(i)
	tmp=XOR_PCBC_step(tmp_key,each_Nth_byte[N], N)
        # count 0, FFs and 8Bs in decrypted buffer:
	important_bytes=tmp.count('\x00')+tmp.count('\xFF')+tmp.count('\x8B')
	stat[i]=important_bytes
    sorted_stat = sorted(stat.iteritems(), key=operator.itemgetter(1), reverse=True)
    print sorted_stat[0]
\end{lstlisting}

(Source code can be downloaded \href{\GitHubBlobMasterURL/examples/simple_exec_crypto/files/decrypt.py}{here}.)

I run it and here is a key for which 00/FF/8B bytes presence in decrypted buffer is maximal:

\begin{lstlisting}
N= 0
(147, 1224)
N= 1
(94, 1327)
N= 2
(252, 1223)
N= 3
(218, 1266)
N= 4
(38, 1209)
N= 5
(192, 1378)
N= 6
(199, 1204)
N= 7
(213, 1332)
N= 8
(225, 1251)
N= 9
(112, 1223)
N= 10
(143, 1177)
N= 11
(108, 1286)
N= 12
(10, 1164)
N= 13
(3, 1271)
N= 14
(128, 1253)
N= 15
(232, 1330)
\end{lstlisting}

Let's write decryption utility with the key we got:

\begin{lstlisting}[style=custompy]
#!/usr/bin/env python
import sys, hexdump, array

def xor_strings(s,t):
    # \verb|https://en.wikipedia.org/wiki/XOR_cipher#Example_implementation|
    """xor two strings together"""
    return "".join(chr(ord(a)^ord(b)) for a,b in zip(s,t))

IV=array.array('B', [147, 94, 252, 218, 38, 192, 199, 213, 225, 112, 143, 108, 10, 3, 128, 232]).tostring()

def chunks(l, n):
    n = max(1, n)
    return [l[i:i + n] for i in range(0, len(l), n)]

def read_file(fname):
    file=open(fname, mode='rb')
    content=file.read()
    file.close()
    return content

def decrypt_byte(i, k):
    return chr ((ord(i)-k) % 256)

def decrypt(buf):
    return "".join(decrypt_byte(buf[i], i) for i in range(16))

fout=open(sys.argv[2], mode='wb')

prev=IV
content=read_file(sys.argv[1])
tmp=chunks(content, 16)
for c in tmp:
    new_c=decrypt(c)
    p=xor_strings (new_c, prev)
    prev=xor_strings(c, p)
    fout.write(p)
fout.close()
\end{lstlisting}

(Source code can be downloaded \href{\GitHubBlobMasterURL/examples/simple_exec_crypto/files/decrypt2.py}{here}.)

Let's check resulting file:

\lstinputlisting{examples/simple_exec_crypto/objdump_result.txt}

Yes, this is seems correctly disassembled piece of x86 code.
The whole decryped file can be downloaded \href{\GitHubBlobMasterURL/examples/simple_exec_crypto/files/decrypted.bin}{here}.

In fact, this is text section from regedit.exe from Windows 7.
But this example is based on a real case I encountered, so just executable is different (and key), algorithm is the same.

\subsection{Other ideas to consider}

What if I would fail with such simple frequency analysis?
There are other ideas on how to measure correctness of decrypted/decompressed x86 code:

\begin{itemize}

\item Many modern compilers aligns functions on 0x10 border.
So the space left before is filled with NOPs (0x90) or other NOP instructions with known opcodes: \myref{sec:npad}.

\item Perhaps, the most frequent pattern in any assembly language is function call:\\
\TT{PUSH chain / CALL / ADD ESP, X}.
This sequence can easily detected and found.
I've even gathered statistics about average number of function arguments: \myref{args_stat}.
(Hence, this is average length of PUSH chain.)

\end{itemize}

Read more about incorrectly/correctly disassembled code: \myref{ISA_detect}.
}
\RU{\subsection{Простое шифрование используя XOR-маску}
\label{XOR_mask_1}

Я нашел одну старую игру в стиле interactive fiction в архиве \emph{if-archive}\footnote{\url{http://www.ifarchive.org/}}:

\begin{lstlisting}
The New Castle v3.5 - Text/Adventure Game
in the style of the original Infocom (tm)
type games, Zork, Collosal Cave (Adventure),
etc.  Can you solve the mystery of the
abandoned castle?
Shareware from Software Customization.
Software Customization [ASP] Version 3.5 Feb. 2000
\end{lstlisting}

Можно скачать здесь: \url{\GitHubBlobMasterURL/ff/XOR/mask_1/files/newcastle.tgz}.

Там внутри есть файл (с названием \emph{castle.dbf}), который явно зашифрован, но не настоящим криптоалгоритмом,
и оне сжат, это что-то куда проще.
Я бы даже не стал измерять уровень энтропии (\myref{entropy}) этого файла, потому что я итак уверен, что он низкий.
Вот как он выглядит в Midnight Commander:

\begin{figure}[H]
\centering
\myincludegraphics{ff/XOR/mask_1/mc_encrypted.png}
\caption{Зашифрованный файл в Midnight Commander}
\end{figure}

Зашифрованный файл можно скачать здесь:
\url{\GitHubBlobMasterURL/ff/XOR/mask_1/files/castle.dbf.bz2}.

Можно ли расшифровать его без доступа к программе, используя просто этот файл?

Тут явно просматривается повторяющаяся строка. 
Если использовалось простое шифрование с XOR-маской, такие повторяющиеся строки это явное свидетельство,
потому что, вероятно, тут были длинные лакуны с нулевыми байтами, которые, в свою очередь, присутствуют
во мноигих исполняемых файлах, и в остальных бинарных файлах.

\myindex{UNIX!xxd}
Вот дам начала этого файла используя утилиту \emph{xxd} из UNIX:

\lstinputlisting{ff/XOR/mask_1/xxd_result.txt}

Давайте держаться за повторяющуюся строку \TT{iubgv}.
Глядя на этот дамп, мы можем легко увидеть, что период повторений этой строки это 0x51 или 81.
Вероятно, 81 это длина блока?
Длина файла 1658961, и она может быть поделена на 81 без остатка (и тогда там 20481 блоков).

Теперь я буду использовать Mathematica для анализа, есть ли тут повторяющиеся 81-байтные блоки в файле?
Я разделю входной файл на 81-байтные блоки и затем использую ф-цию
\emph{Tally[]}\footnote{\url{https://reference.wolfram.com/language/ref/Tally.html}}
которая просто считает, сколько раз каждый элемент встретился во входном списке.
Вывод Tally не отсортирован, так что я также добавлю ф-цию \emph{Sort[]} для сортировки его по кол-ву вхождений
в нисходящем порядке.

\begin{lstlisting}[style=custommath]
input = BinaryReadList["/home/dennis/.../castle.dbf"];

blocks = Partition[input, 81];

stat = Sort[Tally[blocks], #1[[2]] > #2[[2]] &]
\end{lstlisting}

И вот вывод:

\begin{lstlisting}[style=custommath]
{{{80, 103, 2, 116, 113, 102, 118, 25, 99, 8, 19, 23, 116, 125, 107, 
   25, 99, 109, 114, 102, 14, 121, 115, 31, 9, 117, 113, 111, 5, 4, 
   127, 28, 122, 101, 8, 110, 14, 18, 124, 106, 16, 20, 104, 119, 8, 
   109, 26, 106, 9, 97, 13, 99, 15, 119, 20, 105, 117, 98, 103, 118, 
   1, 126, 29, 97, 122, 17, 15, 114, 110, 3, 5, 125, 125, 99, 126, 
   119, 102, 30, 122, 2, 117}, 1739}, 
{{80, 100, 2, 116, 113, 102, 118, 25, 99, 8, 19, 23, 116, 
   125, 107, 25, 99, 109, 114, 102, 14, 121, 115, 31, 9, 117, 113, 
   111, 5, 4, 127, 28, 122, 101, 8, 110, 14, 18, 124, 106, 16, 20, 
   104, 119, 8, 109, 26, 106, 9, 97, 13, 99, 15, 119, 20, 105, 117, 
   98, 103, 118, 1, 126, 29, 97, 122, 17, 15, 114, 110, 3, 5, 125, 
   125, 99, 126, 119, 102, 30, 122, 2, 117}, 1422}, 
{{80, 101, 2, 116, 113, 102, 118, 25, 99, 8, 19, 23, 116, 
   125, 107, 25, 99, 109, 114, 102, 14, 121, 115, 31, 9, 117, 113, 
   111, 5, 4, 127, 28, 122, 101, 8, 110, 14, 18, 124, 106, 16, 20, 
   104, 119, 8, 109, 26, 106, 9, 97, 13, 99, 15, 119, 20, 105, 117, 
   98, 103, 118, 1, 126, 29, 97, 122, 17, 15, 114, 110, 3, 5, 125, 
   125, 99, 126, 119, 102, 30, 122, 2, 117}, 1012},
{{80, 120, 2, 116, 113, 102, 118, 25, 99, 8, 19, 23, 116, 
   125, 107, 25, 99, 109, 114, 102, 14, 121, 115, 31, 9, 117, 113, 
   111, 5, 4, 127, 28, 122, 101, 8, 110, 14, 18, 124, 106, 16, 20, 
   104, 119, 8, 109, 26, 106, 9, 97, 13, 99, 15, 119, 20, 105, 117, 
   98, 103, 118, 1, 126, 29, 97, 122, 17, 15, 114, 110, 3, 5, 125, 
   125, 99, 126, 119, 102, 30, 122, 2, 117}, 377},

...

{{80, 2, 74, 49, 113, 21, 62, 88, 39, 71, 68, 23, 63, 51, 36, 78, 48, 
   108, 114, 102, 14, 121, 115, 31, 9, 117, 113, 111, 5, 4, 127, 28, 
   122, 101, 8, 110, 14, 18, 124, 106, 16, 20, 104, 119, 8, 109, 26, 
   106, 9, 97, 13, 99, 15, 119, 20, 105, 117, 98, 103, 118, 1, 126, 
   29, 97, 122, 17, 15, 114, 110, 3, 5, 125, 125, 99, 126, 119, 102, 
   30, 122, 2, 117}, 1},
{{80, 1, 74, 59, 113, 45, 56, 86, 52, 91, 19, 64, 60, 60, 63, 
   25, 38, 59, 59, 42, 14, 53, 38, 77, 66, 38, 113, 38, 75, 4, 43, 84,
    63, 101, 64, 43, 79, 64, 40, 57, 16, 91, 46, 119, 69, 40, 84, 117,
    9, 97, 13, 99, 15, 119, 20, 105, 117, 98, 103, 118, 1, 126, 29, 
   97, 122, 17, 15, 114, 110, 3, 5, 125, 125, 99, 126, 119, 102, 30, 
   122, 2, 117}, 1},
{{80, 2, 74, 49, 113, 49, 51, 92, 39, 8, 92, 81, 116, 62, 57, 
   80, 46, 40, 114, 36, 75, 56, 33, 76, 9, 55, 56, 59, 81, 65, 45, 28,
    60, 55, 93, 39, 90, 28, 124, 106, 16, 20, 104, 119, 8, 109, 26, 
   106, 9, 97, 13, 99, 15, 119, 20, 105, 117, 98, 103, 118, 1, 126, 
   29, 97, 122, 17, 15, 114, 110, 3, 5, 125, 125, 99, 126, 119, 102, 
   30, 122, 2, 117}, 1}}
\end{lstlisting}

Вывод Tally это список пар, каждая пара это 81-байтный блок и количество раз, сколько он встретился в файле.
Мы видим, что наиболее частно встречающийся блок это первый, он встретился 1739 раз.
Второй встретился 1422 раза. Есть и другие: 1012 раза, 377 раз, итд.
81-байтные блоки, встреченные лишь один раз, находятся в конце вывода.

Попробуем сравнить эти блоки. Первый и второй.
Есть ли в Mathematica ф-ция для сравнения списков/массивов?
Наверняка есть, но в педагогических целях, я буду использоват операцию XOR для сравнения.
Действительно: если байты во входных массивах равны друг другу, результат операции XOR это 0.
Если не равны, результат будет ненулевой.

Сравним первый блок (встречается 1739 раз) и второй (встречается 1422 раз):

\begin{lstlisting}[style=custommath]
In[]:= BitXor[stat[[1]][[1]], stat[[2]][[1]]]
Out[]= {0, 3, 0, 0, 0, 0, 0, 0, 0, 0, 0, 0, 0, 0, 0, 0, 0, 0, 0, \
0, 0, 0, 0, 0, 0, 0, 0, 0, 0, 0, 0, 0, 0, 0, 0, 0, 0, 0, 0, 0, 0, 0, \
0, 0, 0, 0, 0, 0, 0, 0, 0, 0, 0, 0, 0, 0, 0, 0, 0, 0, 0, 0, 0, 0, 0, \
0, 0, 0, 0, 0, 0, 0, 0, 0, 0, 0, 0, 0, 0, 0, 0}
\end{lstlisting}

Они отличаются только вторым байтом.

Сравним второй блок (встречается 1422 раза) и третий (встречается 1012 раз):

\begin{lstlisting}[style=custommath]
In[]:= BitXor[stat[[2]][[1]], stat[[3]][[1]]]
Out[]= {0, 1, 0, 0, 0, 0, 0, 0, 0, 0, 0, 0, 0, 0, 0, 0, 0, 0, 0, \
0, 0, 0, 0, 0, 0, 0, 0, 0, 0, 0, 0, 0, 0, 0, 0, 0, 0, 0, 0, 0, 0, 0, \
0, 0, 0, 0, 0, 0, 0, 0, 0, 0, 0, 0, 0, 0, 0, 0, 0, 0, 0, 0, 0, 0, 0, \
0, 0, 0, 0, 0, 0, 0, 0, 0, 0, 0, 0, 0, 0, 0, 0}
\end{lstlisting}

Они тоже отличаются только вторым байтом.

Так или иначе, попробуем использовать самый встречающийся блок как XOR-ключ и попробуем расшифровать первые 4 81-байтных
блока в файле:

\begin{lstlisting}[style=custommath]
In[]:= key = stat[[1]][[1]]
Out[]= {80, 103, 2, 116, 113, 102, 118, 25, 99, 8, 19, 23, 116, \
125, 107, 25, 99, 109, 114, 102, 14, 121, 115, 31, 9, 117, 113, 111, \
5, 4, 127, 28, 122, 101, 8, 110, 14, 18, 124, 106, 16, 20, 104, 119, \
8, 109, 26, 106, 9, 97, 13, 99, 15, 119, 20, 105, 117, 98, 103, 118, \
1, 126, 29, 97, 122, 17, 15, 114, 110, 3, 5, 125, 125, 99, 126, 119, \
102, 30, 122, 2, 117}

In[]:= ToASCII[val_] := If[val == 0, " ", FromCharacterCode[val, "PrintableASCII"]]

In[]:= DecryptBlockASCII[blk_] := Map[ToASCII[#] &, BitXor[key, blk]]

In[]:= DecryptBlockASCII[blocks[[1]]]
Out[]= {" ", " ", " ", " ", " ", " ", " ", " ", " ", " ", " ", " \
", " ", " ", " ", " ", " ", " ", " ", " ", " ", " ", " ", " ", " ", " \
", " ", " ", " ", " ", " ", " ", " ", " ", " ", " ", " ", " ", " ", " \
", " ", " ", " ", " ", " ", " ", " ", " ", " ", " ", " ", " ", " ", " \
", " ", " ", " ", " ", " ", " ", " ", " ", " ", " ", " ", " ", " ", " \
", " ", " ", " ", " ", " ", " ", " ", " ", " ", " ", " ", " ", " "}

In[]:= DecryptBlockASCII[blocks[[2]]]
Out[]= {" ", "e", "H", "E", " ", "W", "E", "E", "D", " ", "O", \
"F", " ", "C", "R", "I", "M", "E", " ", "B", "E", "A", "R", "S", " ", \
"B", "I", "T", "T", "E", "R", " ", "F", "R", "U", "I", "T", "?", \
" ", " ", " ", " ", " ", " ", " ", " ", " ", " ", " ", " ", " ", " ", \
" ", " ", " ", " ", " ", " ", " ", " ", " ", " ", " ", " ", " ", " ", \
" ", " ", " ", " ", " ", " ", " ", " ", " ", " ", " ", " ", " ", " ", \
" "}

In[]:= DecryptBlockASCII[blocks[[3]]]
Out[]= {" ", "?", " ", " ", " ", " ", " ", " ", " ", " ", " \
", " ", " ", " ", " ", " ", " ", " ", " ", " ", " ", " ", " ", " ", " \
", " ", " ", " ", " ", " ", " ", " ", " ", " ", " ", " ", " ", " ", " \
", " ", " ", " ", " ", " ", " ", " ", " ", " ", " ", " ", " ", " ", " \
", " ", " ", " ", " ", " ", " ", " ", " ", " ", " ", " ", " ", " ", " \
", " ", " ", " ", " ", " ", " ", " ", " ", " ", " ", " ", " ", " ", " \
"}

In[]:= DecryptBlockASCII[blocks[[4]]]
Out[]= {" ", "f", "H", "O", " ", "K", "N", "O", "W", "S", " ", \
"W", "H", "A", "T", " ", "E", "V", "I", "L", " ", "L", "U", "R", "K", \
"S", " ", "I", "N", " ", "T", "H", "E", " ", "H", "E", "A", "R", "T", \
"S", " ", "O", "F", " ", "M", "E", "N", "?", " ", " ", " ", " ", \
" ", " ", " ", " ", " ", " ", " ", " ", " ", " ", " ", " ", " ", " ", \
" ", " ", " ", " ", " ", " ", " ", " ", " ", " ", " ", " ", " ", " ", \
" "}
\end{lstlisting}

(Я заменил непечатаемые символы на \q{?}.)

Мы видим что первый и третий блоки пустые (или почти пустые),
но второй и четвертый имеют ясно различимые английские слова/фразы.
Похоже что наше предположение насчет ключа верно (как минимум частично).
Это означает, что самый встречающийся 81-байтный блок в файле находится в местах лакун с нулевыми байтами
или что-то в этом роде.

Попробуем расшифровать весь файл:

\begin{lstlisting}[style=custommath]
DecryptBlock[blk_] := BitXor[key, blk]

decrypted = Map[DecryptBlock[#] &, blocks];

BinaryWrite["/home/dennis/.../tmp", Flatten[decrypted]]

Close["/home/dennis/.../tmp"]
\end{lstlisting}

\begin{figure}[H]
\centering
\myincludegraphics{ff/XOR/mask_1/mc_decrypted1.png}
\caption{Расшифрованный файл в Midnight Commander, первая попытка}
\end{figure}

Выглядит как английские фразы для какой-то игры, но что-то не так.
Прежде всего, регистр инвертирован: фразы и некоторые слова начинаются со строчных букв,
в то время как остальные буквы заглавные.
Также, некоторые фразы начинаются с не тех букв.
Посмотрите на самую первую фразу: \q{eHE WEED OF CRIME BEARS BITTER FRUIT}.
Что такое \q{eHE}? Разве не \q{tHE} тут должно быть?
Возможно ли что наш ключ для дешифрования имеет неверный байт в этом месте?

Посмотрим снова на второй блок в файле, на ключ и на результат дешифрования:

\begin{lstlisting}[style=custommath]
In[]:= blocks[[2]]
Out[]= {80, 2, 74, 49, 113, 49, 51, 92, 39, 8, 92, 81, 116, 62, \
57, 80, 46, 40, 114, 36, 75, 56, 33, 76, 9, 55, 56, 59, 81, 65, 45, \
28, 60, 55, 93, 39, 90, 28, 124, 106, 16, 20, 104, 119, 8, 109, 26, \
106, 9, 97, 13, 99, 15, 119, 20, 105, 117, 98, 103, 118, 1, 126, 29, \
97, 122, 17, 15, 114, 110, 3, 5, 125, 125, 99, 126, 119, 102, 30, \
122, 2, 117}

In[]:= key
Out[]= {80, 103, 2, 116, 113, 102, 118, 25, 99, 8, 19, 23, 116, \
125, 107, 25, 99, 109, 114, 102, 14, 121, 115, 31, 9, 117, 113, 111, \
5, 4, 127, 28, 122, 101, 8, 110, 14, 18, 124, 106, 16, 20, 104, 119, \
8, 109, 26, 106, 9, 97, 13, 99, 15, 119, 20, 105, 117, 98, 103, 118, \
1, 126, 29, 97, 122, 17, 15, 114, 110, 3, 5, 125, 125, 99, 126, 119, \
102, 30, 122, 2, 117}

In[]:= BitXor[key, blocks[[2]]]
Out[]= {0, 101, 72, 69, 0, 87, 69, 69, 68, 0, 79, 70, 0, 67, 82, \
73, 77, 69, 0, 66, 69, 65, 82, 83, 0, 66, 73, 84, 84, 69, 82, 0, 70, \
82, 85, 73, 84, 14, 0, 0, 0, 0, 0, 0, 0, 0, 0, 0, 0, 0, 0, 0, 0, 0, \
0, 0, 0, 0, 0, 0, 0, 0, 0, 0, 0, 0, 0, 0, 0, 0, 0, 0, 0, 0, 0, 0, 0, \
0, 0, 0, 0}
\end{lstlisting}

Зашифрованный байт это 2, байт из ключа это 103, $2 \oplus 103=101$ и 101 это ASCII-код символа \q{e}.
Чему должен равнятся этот байт ключа, чтобы ASCII-код был 116 (для символа  \q{t})?
$2 \oplus 116=118$, присвоим 118 второму байту в ключе \dots

\begin{lstlisting}[style=custommath]
key = {80, 118, 2, 116, 113, 102, 118, 25, 99, 8, 19, 23, 116, 125, 
  107, 25, 99, 109, 114, 102, 14, 121, 115, 31, 9, 117, 113, 111, 5, 
  4, 127, 28, 122, 101, 8, 110, 14, 18, 124, 106, 16, 20, 104, 119, 8,
   109, 26, 106, 9, 97, 13, 99, 15, 119, 20, 105, 117, 98, 103, 118, 
  1, 126, 29, 97, 122, 17, 15, 114, 110, 3, 5, 125, 125, 99, 126, 119,
   102, 30, 122, 2, 117}
\end{lstlisting}

\dots и снова дешифруем весь файл.

\begin{figure}[H]
\centering
\myincludegraphics{ff/XOR/mask_1/mc_decrypted2.png}
\caption{Дешифрованный файл в Midnight Commander, вторая попытка}
\end{figure}

Ух ты, теперь грамматика корректна, и все фразы начинаются с корректных букв.
Но все таки, регистр подозрителен.
С чего бы разработчику игры записывать их в такой манере?
Может быть наш ключ все еще неправилен?

% TODO ASCII table somewhere in the book
Изучая таблицу ASCII мы можем заметить что ASCII-коды для букв в верхнем и нижнем регистре отличаются только на один бит
(6-й бит, если считать с первого, 0b100000):

\begin{figure}[H]
\centering
\includegraphics[width=0.7\textwidth]{ascii.png}
\caption{7-битная таблица \ac{ASCII} в Emacs}
\end{figure}

6-й бит, выставленный в нулевом байте, В десятичном виде это будет 32.
Но 32 это ASCII-код пробела!

Действительно, можно менять регистр просто применяя XOR к ASCII-коду, с 32 (больше об этом: \myref{toupper_bit}).

Возможно ли, что пустые лакуны в файле это не нулевые байты, а скорее содержащие пробелы?
Еще раз модифицируем наш XOR-ключ (я про-XOR-ю каждый байт ключа с 32):

\begin{lstlisting}[style=custommath]
(* "32" это скаляр, и "key" это вектор, но это OK *)

In[]:= key3 = BitXor[32, key]
Out[]= {112, 86, 34, 84, 81, 70, 86, 57, 67, 40, 51, 55, 84, 93, 75, \
57, 67, 77, 82, 70, 46, 89, 83, 63, 41, 85, 81, 79, 37, 36, 95, 60, \
90, 69, 40, 78, 46, 50, 92, 74, 48, 52, 72, 87, 40, 77, 58, 74, 41, \
65, 45, 67, 47, 87, 52, 73, 85, 66, 71, 86, 33, 94, 61, 65, 90, 49, \
47, 82, 78, 35, 37, 93, 93, 67, 94, 87, 70, 62, 90, 34, 85}

In[]:= DecryptBlock[blk_] := BitXor[key3, blk]
\end{lstlisting}

И снова дешифруем входной файл:

\begin{figure}[H]
\centering
\myincludegraphics{ff/XOR/mask_1/mc_decrypted.png}
\caption{Дешифрованный файл в Midnight Commander, последняя попытка}
\end{figure}

(Расшифрованный файл доступен здесь:
\url{\GitHubBlobMasterURL/ff/XOR/mask_1/files/decrypted.dat.bz2}.)

Несомненно, это корректный исходный файл.
Да, и мы видим числа в начале каждого блока. Должно быть это и есть источник некорректного XOR-ключа.
Как выходит, самый встречающийся 81-байтный блок в файле это блок заполненный пробелами и содержащий символ \q{1} на месте
второго байта.
Действительно, как-то так получилось что многие блоки здесь перемежаются с этим блоком.
Может быть это что-то вроде выравнивания (padding) для коротких фраз/сообщений?
Другой часто встречающийся 81-байтный блок также заполнен пробелами, но с другой цифрой, следовательно,
они отличаются только вторым байтом.

Вот и всё! Теперь мы можем написать утилиту для зашифрования файла назад, и, может быть, модифицировать его перед этим

Файл для Mathematica можно скачать здесь:\\
\url{\GitHubBlobMasterURL/ff/XOR/mask_1/files/XOR_mask_1.nb}.

Итог: XOR-шифрование не надежно вообще. Вероятно, разработчик игры хотел просто скрыть внутренности игры от игрока,
ничего более серьезного.
Все же, шифрование вроде этого крайне популярно вследствии его простоты, так что многие реверс инженеры обычно хорошо
с этим знакомы.

}

\EN{% TODO translate
\mysection{Breaking simple executable cryptor}

I've got an executable file which is encrypted by relatively simple encryption.
\href{\GitHubBlobMasterURL/examples/simple_exec_crypto/files/cipher.bin}{Here is it} (only executable section is left here).

First, all encryption function does is just adds number of position in buffer to the byte.
Here is how this can be encoded in Python:

\begin{lstlisting}[caption=Python script,style=custompy]
#!/usr/bin/env python
def e(i, k):
    return chr ((ord(i)+k) % 256)

def encrypt(buf):
    return e(buf[0], 0)+ e(buf[1], 1)+ e(buf[2], 2) + e(buf[3], 3)+ e(buf[4], 4)+ e(buf[5], 5)+ e(buf[6], 6)+ e(buf[7], 7)+
           e(buf[8], 8)+ e(buf[9], 9)+ e(buf[10], 10)+ e(buf[11], 11)+ e(buf[12], 12)+ e(buf[13], 13)+ e(buf[14], 14)+ e(buf[15], 15)
\end{lstlisting}

Hence, if you encrypt buffer with 16 zeros, you'll get \emph{0, 1, 2, 3 ... 12, 13, 14, 15}.

\myindex{Propagating Cipher Block Chaining}
Propagating Cipher Block Chaining (PCBC) is also used, here is how it works:

\begin{figure}[H]
\centering
\myincludegraphics{examples/simple_exec_crypto/601px-PCBC_encryption.png}
\caption{Propagating Cipher Block Chaining encryption (image is taken from Wikipedia article)}
\end{figure}

The problem is that it's too boring to recover IV (Initialization Vector) each time.
Brute-force is also not an option, because IV is too long (16 bytes).
Let's see, if it's possible to recover IV for arbitrary encrypted executable file?

Let's try simple frequency analysis.
This is 32-bit x86 executable code, so let's gather statistics about most frequent bytes and opcodes.
I tried huge oracle.exe file from Oracle RDBMS version 11.2 for windows x86 and I've found that the most frequent byte (no surprise) is zero (~10\%).
The next most frequent byte is (again, no surprise) 0xFF (~5\%).
The next is 0x8B (~5\%).

\myindex{x86!\Instructions!MOV}
0x8B is opcode for \INS{MOV}, this is indeed one of the most busy x86 instructions.
Now what about popularity of zero byte?
If compiler needs to encode value bigger than 127, it has to use 32-bit displacement instead of 8-bit one, but large values are very rare,
so it is padded by zeros.
\myindex{x86!\Instructions!LEA}
\myindex{x86!\Instructions!PUSH}
\myindex{x86!\Instructions!CALL}
This is at least in \INS{LEA}, \INS{MOV}, \INS{PUSH}, \INS{CALL}.

For example:

\begin{lstlisting}[style=customasmx86]
8D B0 28 01 00 00                 lea     esi, [eax+128h]
8D BF 40 38 00 00                 lea     edi, [edi+3840h]
\end{lstlisting}

Displacements bigger than 127 are very popular, but they are rarely exceeds 0x10000
(indeed, such large memory buffers/structures are also rare).

Same story with \INS{MOV}, large constants are rare, the most heavily used are 0, 1, 10, 100, $2^n$, and so on.
Compiler has to pad small constants by zeros to represent them as 32-bit values:

\begin{lstlisting}[style=customasmx86]
BF 02 00 00 00                    mov     edi, 2
BF 01 00 00 00                    mov     edi, 1
\end{lstlisting}

Now about 00 and FF bytes combined: jumps (including conditional) and calls can pass execution flow forward or backwards, but very often,
within the limits of the current executable module.
If forward, displacement is not very big and also padded with zeros.
If backwards, displacement is represented as negative value, so padded with FF bytes.
For example, transfer execution flow forward:

\begin{lstlisting}[style=customasmx86]
E8 43 0C 00 00                    call    _function1
E8 5C 00 00 00                    call    _function2
0F 84 F0 0A 00 00                 jz      loc_4F09A0
0F 84 EB 00 00 00                 jz      loc_4EFBB8
\end{lstlisting}

Backwards:

\begin{lstlisting}[style=customasmx86]
E8 79 0C FE FF                    call    _function1
E8 F4 16 FF FF                    call    _function2
0F 84 F8 FB FF FF                 jz      loc_8212BC
0F 84 06 FD FF FF                 jz      loc_FF1E7D
\end{lstlisting}

FF byte is also very often occurred in negative displacements like these:

\begin{lstlisting}[style=customasmx86]
8D 85 1E FF FF FF                 lea     eax, [ebp-0E2h]
8D 95 F8 5C FF FF                 lea     edx, [ebp-0A308h]
\end{lstlisting}

So far so good. Now we have to try various 16-byte keys, decrypt executable section and measure how often 00, FF and 8B bytes are occurred.
Let's also keep in sight how PCBC decryption works:

\begin{figure}[H]
\centering
\myincludegraphics{examples/simple_exec_crypto/640px-PCBC_decryption.png}
\caption{Propagating Cipher Block Chaining decryption (image is taken from Wikipedia article)}
\end{figure}

The good news is that we don't really have to decrypt whole piece of data, but only slice by slice, this is exactly how I did in my previous example: \myref{XOR_mask_2}.

Now I'm trying all possible bytes (0..255) for each byte in key and just pick the byte producing maximal amount of 00/FF/8B bytes in a decrypted slice:

\begin{lstlisting}[style=custompy]
#!/usr/bin/env python
import sys, hexdump, array, string, operator

KEY_LEN=16

def chunks(l, n):
    # split n by l-byte chunks
    # https://stackoverflow.com/q/312443
    n = max(1, n)
    return [l[i:i + n] for i in range(0, len(l), n)]

def read_file(fname):
    file=open(fname, mode='rb')
    content=file.read()
    file.close()
    return content

def decrypt_byte (c, key):
    return chr((ord(c)-key) % 256)

def XOR_PCBC_step (IV, buf, k):
    prev=IV
    rt=""
    for c in buf:
	new_c=decrypt_byte(c, k)
        plain=chr(ord(new_c)^ord(prev))
	prev=chr(ord(c)^ord(plain))
	rt=rt+plain
    return rt

each_Nth_byte=[""]*KEY_LEN

content=read_file(sys.argv[1])
# split input by 16-byte chunks:
all_chunks=chunks(content, KEY_LEN)
for c in all_chunks:
    for i in range(KEY_LEN):
        each_Nth_byte[i]=each_Nth_byte[i] + c[i]

# try each byte of key
for N in range(KEY_LEN):
    print "N=", N
    stat={}
    for i in range(256):
        tmp_key=chr(i)
	tmp=XOR_PCBC_step(tmp_key,each_Nth_byte[N], N)
        # count 0, FFs and 8Bs in decrypted buffer:
	important_bytes=tmp.count('\x00')+tmp.count('\xFF')+tmp.count('\x8B')
	stat[i]=important_bytes
    sorted_stat = sorted(stat.iteritems(), key=operator.itemgetter(1), reverse=True)
    print sorted_stat[0]
\end{lstlisting}

(Source code can be downloaded \href{\GitHubBlobMasterURL/examples/simple_exec_crypto/files/decrypt.py}{here}.)

I run it and here is a key for which 00/FF/8B bytes presence in decrypted buffer is maximal:

\begin{lstlisting}
N= 0
(147, 1224)
N= 1
(94, 1327)
N= 2
(252, 1223)
N= 3
(218, 1266)
N= 4
(38, 1209)
N= 5
(192, 1378)
N= 6
(199, 1204)
N= 7
(213, 1332)
N= 8
(225, 1251)
N= 9
(112, 1223)
N= 10
(143, 1177)
N= 11
(108, 1286)
N= 12
(10, 1164)
N= 13
(3, 1271)
N= 14
(128, 1253)
N= 15
(232, 1330)
\end{lstlisting}

Let's write decryption utility with the key we got:

\begin{lstlisting}[style=custompy]
#!/usr/bin/env python
import sys, hexdump, array

def xor_strings(s,t):
    # \verb|https://en.wikipedia.org/wiki/XOR_cipher#Example_implementation|
    """xor two strings together"""
    return "".join(chr(ord(a)^ord(b)) for a,b in zip(s,t))

IV=array.array('B', [147, 94, 252, 218, 38, 192, 199, 213, 225, 112, 143, 108, 10, 3, 128, 232]).tostring()

def chunks(l, n):
    n = max(1, n)
    return [l[i:i + n] for i in range(0, len(l), n)]

def read_file(fname):
    file=open(fname, mode='rb')
    content=file.read()
    file.close()
    return content

def decrypt_byte(i, k):
    return chr ((ord(i)-k) % 256)

def decrypt(buf):
    return "".join(decrypt_byte(buf[i], i) for i in range(16))

fout=open(sys.argv[2], mode='wb')

prev=IV
content=read_file(sys.argv[1])
tmp=chunks(content, 16)
for c in tmp:
    new_c=decrypt(c)
    p=xor_strings (new_c, prev)
    prev=xor_strings(c, p)
    fout.write(p)
fout.close()
\end{lstlisting}

(Source code can be downloaded \href{\GitHubBlobMasterURL/examples/simple_exec_crypto/files/decrypt2.py}{here}.)

Let's check resulting file:

\lstinputlisting{examples/simple_exec_crypto/objdump_result.txt}

Yes, this is seems correctly disassembled piece of x86 code.
The whole decryped file can be downloaded \href{\GitHubBlobMasterURL/examples/simple_exec_crypto/files/decrypted.bin}{here}.

In fact, this is text section from regedit.exe from Windows 7.
But this example is based on a real case I encountered, so just executable is different (and key), algorithm is the same.

\subsection{Other ideas to consider}

What if I would fail with such simple frequency analysis?
There are other ideas on how to measure correctness of decrypted/decompressed x86 code:

\begin{itemize}

\item Many modern compilers aligns functions on 0x10 border.
So the space left before is filled with NOPs (0x90) or other NOP instructions with known opcodes: \myref{sec:npad}.

\item Perhaps, the most frequent pattern in any assembly language is function call:\\
\TT{PUSH chain / CALL / ADD ESP, X}.
This sequence can easily detected and found.
I've even gathered statistics about average number of function arguments: \myref{args_stat}.
(Hence, this is average length of PUSH chain.)

\end{itemize}

Read more about incorrectly/correctly disassembled code: \myref{ISA_detect}.
}
\RU{\subsection{Простое шифрование используя XOR-маску}
\label{XOR_mask_1}

Я нашел одну старую игру в стиле interactive fiction в архиве \emph{if-archive}\footnote{\url{http://www.ifarchive.org/}}:

\begin{lstlisting}
The New Castle v3.5 - Text/Adventure Game
in the style of the original Infocom (tm)
type games, Zork, Collosal Cave (Adventure),
etc.  Can you solve the mystery of the
abandoned castle?
Shareware from Software Customization.
Software Customization [ASP] Version 3.5 Feb. 2000
\end{lstlisting}

Можно скачать здесь: \url{\GitHubBlobMasterURL/ff/XOR/mask_1/files/newcastle.tgz}.

Там внутри есть файл (с названием \emph{castle.dbf}), который явно зашифрован, но не настоящим криптоалгоритмом,
и оне сжат, это что-то куда проще.
Я бы даже не стал измерять уровень энтропии (\myref{entropy}) этого файла, потому что я итак уверен, что он низкий.
Вот как он выглядит в Midnight Commander:

\begin{figure}[H]
\centering
\myincludegraphics{ff/XOR/mask_1/mc_encrypted.png}
\caption{Зашифрованный файл в Midnight Commander}
\end{figure}

Зашифрованный файл можно скачать здесь:
\url{\GitHubBlobMasterURL/ff/XOR/mask_1/files/castle.dbf.bz2}.

Можно ли расшифровать его без доступа к программе, используя просто этот файл?

Тут явно просматривается повторяющаяся строка. 
Если использовалось простое шифрование с XOR-маской, такие повторяющиеся строки это явное свидетельство,
потому что, вероятно, тут были длинные лакуны с нулевыми байтами, которые, в свою очередь, присутствуют
во мноигих исполняемых файлах, и в остальных бинарных файлах.

\myindex{UNIX!xxd}
Вот дам начала этого файла используя утилиту \emph{xxd} из UNIX:

\lstinputlisting{ff/XOR/mask_1/xxd_result.txt}

Давайте держаться за повторяющуюся строку \TT{iubgv}.
Глядя на этот дамп, мы можем легко увидеть, что период повторений этой строки это 0x51 или 81.
Вероятно, 81 это длина блока?
Длина файла 1658961, и она может быть поделена на 81 без остатка (и тогда там 20481 блоков).

Теперь я буду использовать Mathematica для анализа, есть ли тут повторяющиеся 81-байтные блоки в файле?
Я разделю входной файл на 81-байтные блоки и затем использую ф-цию
\emph{Tally[]}\footnote{\url{https://reference.wolfram.com/language/ref/Tally.html}}
которая просто считает, сколько раз каждый элемент встретился во входном списке.
Вывод Tally не отсортирован, так что я также добавлю ф-цию \emph{Sort[]} для сортировки его по кол-ву вхождений
в нисходящем порядке.

\begin{lstlisting}[style=custommath]
input = BinaryReadList["/home/dennis/.../castle.dbf"];

blocks = Partition[input, 81];

stat = Sort[Tally[blocks], #1[[2]] > #2[[2]] &]
\end{lstlisting}

И вот вывод:

\begin{lstlisting}[style=custommath]
{{{80, 103, 2, 116, 113, 102, 118, 25, 99, 8, 19, 23, 116, 125, 107, 
   25, 99, 109, 114, 102, 14, 121, 115, 31, 9, 117, 113, 111, 5, 4, 
   127, 28, 122, 101, 8, 110, 14, 18, 124, 106, 16, 20, 104, 119, 8, 
   109, 26, 106, 9, 97, 13, 99, 15, 119, 20, 105, 117, 98, 103, 118, 
   1, 126, 29, 97, 122, 17, 15, 114, 110, 3, 5, 125, 125, 99, 126, 
   119, 102, 30, 122, 2, 117}, 1739}, 
{{80, 100, 2, 116, 113, 102, 118, 25, 99, 8, 19, 23, 116, 
   125, 107, 25, 99, 109, 114, 102, 14, 121, 115, 31, 9, 117, 113, 
   111, 5, 4, 127, 28, 122, 101, 8, 110, 14, 18, 124, 106, 16, 20, 
   104, 119, 8, 109, 26, 106, 9, 97, 13, 99, 15, 119, 20, 105, 117, 
   98, 103, 118, 1, 126, 29, 97, 122, 17, 15, 114, 110, 3, 5, 125, 
   125, 99, 126, 119, 102, 30, 122, 2, 117}, 1422}, 
{{80, 101, 2, 116, 113, 102, 118, 25, 99, 8, 19, 23, 116, 
   125, 107, 25, 99, 109, 114, 102, 14, 121, 115, 31, 9, 117, 113, 
   111, 5, 4, 127, 28, 122, 101, 8, 110, 14, 18, 124, 106, 16, 20, 
   104, 119, 8, 109, 26, 106, 9, 97, 13, 99, 15, 119, 20, 105, 117, 
   98, 103, 118, 1, 126, 29, 97, 122, 17, 15, 114, 110, 3, 5, 125, 
   125, 99, 126, 119, 102, 30, 122, 2, 117}, 1012},
{{80, 120, 2, 116, 113, 102, 118, 25, 99, 8, 19, 23, 116, 
   125, 107, 25, 99, 109, 114, 102, 14, 121, 115, 31, 9, 117, 113, 
   111, 5, 4, 127, 28, 122, 101, 8, 110, 14, 18, 124, 106, 16, 20, 
   104, 119, 8, 109, 26, 106, 9, 97, 13, 99, 15, 119, 20, 105, 117, 
   98, 103, 118, 1, 126, 29, 97, 122, 17, 15, 114, 110, 3, 5, 125, 
   125, 99, 126, 119, 102, 30, 122, 2, 117}, 377},

...

{{80, 2, 74, 49, 113, 21, 62, 88, 39, 71, 68, 23, 63, 51, 36, 78, 48, 
   108, 114, 102, 14, 121, 115, 31, 9, 117, 113, 111, 5, 4, 127, 28, 
   122, 101, 8, 110, 14, 18, 124, 106, 16, 20, 104, 119, 8, 109, 26, 
   106, 9, 97, 13, 99, 15, 119, 20, 105, 117, 98, 103, 118, 1, 126, 
   29, 97, 122, 17, 15, 114, 110, 3, 5, 125, 125, 99, 126, 119, 102, 
   30, 122, 2, 117}, 1},
{{80, 1, 74, 59, 113, 45, 56, 86, 52, 91, 19, 64, 60, 60, 63, 
   25, 38, 59, 59, 42, 14, 53, 38, 77, 66, 38, 113, 38, 75, 4, 43, 84,
    63, 101, 64, 43, 79, 64, 40, 57, 16, 91, 46, 119, 69, 40, 84, 117,
    9, 97, 13, 99, 15, 119, 20, 105, 117, 98, 103, 118, 1, 126, 29, 
   97, 122, 17, 15, 114, 110, 3, 5, 125, 125, 99, 126, 119, 102, 30, 
   122, 2, 117}, 1},
{{80, 2, 74, 49, 113, 49, 51, 92, 39, 8, 92, 81, 116, 62, 57, 
   80, 46, 40, 114, 36, 75, 56, 33, 76, 9, 55, 56, 59, 81, 65, 45, 28,
    60, 55, 93, 39, 90, 28, 124, 106, 16, 20, 104, 119, 8, 109, 26, 
   106, 9, 97, 13, 99, 15, 119, 20, 105, 117, 98, 103, 118, 1, 126, 
   29, 97, 122, 17, 15, 114, 110, 3, 5, 125, 125, 99, 126, 119, 102, 
   30, 122, 2, 117}, 1}}
\end{lstlisting}

Вывод Tally это список пар, каждая пара это 81-байтный блок и количество раз, сколько он встретился в файле.
Мы видим, что наиболее частно встречающийся блок это первый, он встретился 1739 раз.
Второй встретился 1422 раза. Есть и другие: 1012 раза, 377 раз, итд.
81-байтные блоки, встреченные лишь один раз, находятся в конце вывода.

Попробуем сравнить эти блоки. Первый и второй.
Есть ли в Mathematica ф-ция для сравнения списков/массивов?
Наверняка есть, но в педагогических целях, я буду использоват операцию XOR для сравнения.
Действительно: если байты во входных массивах равны друг другу, результат операции XOR это 0.
Если не равны, результат будет ненулевой.

Сравним первый блок (встречается 1739 раз) и второй (встречается 1422 раз):

\begin{lstlisting}[style=custommath]
In[]:= BitXor[stat[[1]][[1]], stat[[2]][[1]]]
Out[]= {0, 3, 0, 0, 0, 0, 0, 0, 0, 0, 0, 0, 0, 0, 0, 0, 0, 0, 0, \
0, 0, 0, 0, 0, 0, 0, 0, 0, 0, 0, 0, 0, 0, 0, 0, 0, 0, 0, 0, 0, 0, 0, \
0, 0, 0, 0, 0, 0, 0, 0, 0, 0, 0, 0, 0, 0, 0, 0, 0, 0, 0, 0, 0, 0, 0, \
0, 0, 0, 0, 0, 0, 0, 0, 0, 0, 0, 0, 0, 0, 0, 0}
\end{lstlisting}

Они отличаются только вторым байтом.

Сравним второй блок (встречается 1422 раза) и третий (встречается 1012 раз):

\begin{lstlisting}[style=custommath]
In[]:= BitXor[stat[[2]][[1]], stat[[3]][[1]]]
Out[]= {0, 1, 0, 0, 0, 0, 0, 0, 0, 0, 0, 0, 0, 0, 0, 0, 0, 0, 0, \
0, 0, 0, 0, 0, 0, 0, 0, 0, 0, 0, 0, 0, 0, 0, 0, 0, 0, 0, 0, 0, 0, 0, \
0, 0, 0, 0, 0, 0, 0, 0, 0, 0, 0, 0, 0, 0, 0, 0, 0, 0, 0, 0, 0, 0, 0, \
0, 0, 0, 0, 0, 0, 0, 0, 0, 0, 0, 0, 0, 0, 0, 0}
\end{lstlisting}

Они тоже отличаются только вторым байтом.

Так или иначе, попробуем использовать самый встречающийся блок как XOR-ключ и попробуем расшифровать первые 4 81-байтных
блока в файле:

\begin{lstlisting}[style=custommath]
In[]:= key = stat[[1]][[1]]
Out[]= {80, 103, 2, 116, 113, 102, 118, 25, 99, 8, 19, 23, 116, \
125, 107, 25, 99, 109, 114, 102, 14, 121, 115, 31, 9, 117, 113, 111, \
5, 4, 127, 28, 122, 101, 8, 110, 14, 18, 124, 106, 16, 20, 104, 119, \
8, 109, 26, 106, 9, 97, 13, 99, 15, 119, 20, 105, 117, 98, 103, 118, \
1, 126, 29, 97, 122, 17, 15, 114, 110, 3, 5, 125, 125, 99, 126, 119, \
102, 30, 122, 2, 117}

In[]:= ToASCII[val_] := If[val == 0, " ", FromCharacterCode[val, "PrintableASCII"]]

In[]:= DecryptBlockASCII[blk_] := Map[ToASCII[#] &, BitXor[key, blk]]

In[]:= DecryptBlockASCII[blocks[[1]]]
Out[]= {" ", " ", " ", " ", " ", " ", " ", " ", " ", " ", " ", " \
", " ", " ", " ", " ", " ", " ", " ", " ", " ", " ", " ", " ", " ", " \
", " ", " ", " ", " ", " ", " ", " ", " ", " ", " ", " ", " ", " ", " \
", " ", " ", " ", " ", " ", " ", " ", " ", " ", " ", " ", " ", " ", " \
", " ", " ", " ", " ", " ", " ", " ", " ", " ", " ", " ", " ", " ", " \
", " ", " ", " ", " ", " ", " ", " ", " ", " ", " ", " ", " ", " "}

In[]:= DecryptBlockASCII[blocks[[2]]]
Out[]= {" ", "e", "H", "E", " ", "W", "E", "E", "D", " ", "O", \
"F", " ", "C", "R", "I", "M", "E", " ", "B", "E", "A", "R", "S", " ", \
"B", "I", "T", "T", "E", "R", " ", "F", "R", "U", "I", "T", "?", \
" ", " ", " ", " ", " ", " ", " ", " ", " ", " ", " ", " ", " ", " ", \
" ", " ", " ", " ", " ", " ", " ", " ", " ", " ", " ", " ", " ", " ", \
" ", " ", " ", " ", " ", " ", " ", " ", " ", " ", " ", " ", " ", " ", \
" "}

In[]:= DecryptBlockASCII[blocks[[3]]]
Out[]= {" ", "?", " ", " ", " ", " ", " ", " ", " ", " ", " \
", " ", " ", " ", " ", " ", " ", " ", " ", " ", " ", " ", " ", " ", " \
", " ", " ", " ", " ", " ", " ", " ", " ", " ", " ", " ", " ", " ", " \
", " ", " ", " ", " ", " ", " ", " ", " ", " ", " ", " ", " ", " ", " \
", " ", " ", " ", " ", " ", " ", " ", " ", " ", " ", " ", " ", " ", " \
", " ", " ", " ", " ", " ", " ", " ", " ", " ", " ", " ", " ", " ", " \
"}

In[]:= DecryptBlockASCII[blocks[[4]]]
Out[]= {" ", "f", "H", "O", " ", "K", "N", "O", "W", "S", " ", \
"W", "H", "A", "T", " ", "E", "V", "I", "L", " ", "L", "U", "R", "K", \
"S", " ", "I", "N", " ", "T", "H", "E", " ", "H", "E", "A", "R", "T", \
"S", " ", "O", "F", " ", "M", "E", "N", "?", " ", " ", " ", " ", \
" ", " ", " ", " ", " ", " ", " ", " ", " ", " ", " ", " ", " ", " ", \
" ", " ", " ", " ", " ", " ", " ", " ", " ", " ", " ", " ", " ", " ", \
" "}
\end{lstlisting}

(Я заменил непечатаемые символы на \q{?}.)

Мы видим что первый и третий блоки пустые (или почти пустые),
но второй и четвертый имеют ясно различимые английские слова/фразы.
Похоже что наше предположение насчет ключа верно (как минимум частично).
Это означает, что самый встречающийся 81-байтный блок в файле находится в местах лакун с нулевыми байтами
или что-то в этом роде.

Попробуем расшифровать весь файл:

\begin{lstlisting}[style=custommath]
DecryptBlock[blk_] := BitXor[key, blk]

decrypted = Map[DecryptBlock[#] &, blocks];

BinaryWrite["/home/dennis/.../tmp", Flatten[decrypted]]

Close["/home/dennis/.../tmp"]
\end{lstlisting}

\begin{figure}[H]
\centering
\myincludegraphics{ff/XOR/mask_1/mc_decrypted1.png}
\caption{Расшифрованный файл в Midnight Commander, первая попытка}
\end{figure}

Выглядит как английские фразы для какой-то игры, но что-то не так.
Прежде всего, регистр инвертирован: фразы и некоторые слова начинаются со строчных букв,
в то время как остальные буквы заглавные.
Также, некоторые фразы начинаются с не тех букв.
Посмотрите на самую первую фразу: \q{eHE WEED OF CRIME BEARS BITTER FRUIT}.
Что такое \q{eHE}? Разве не \q{tHE} тут должно быть?
Возможно ли что наш ключ для дешифрования имеет неверный байт в этом месте?

Посмотрим снова на второй блок в файле, на ключ и на результат дешифрования:

\begin{lstlisting}[style=custommath]
In[]:= blocks[[2]]
Out[]= {80, 2, 74, 49, 113, 49, 51, 92, 39, 8, 92, 81, 116, 62, \
57, 80, 46, 40, 114, 36, 75, 56, 33, 76, 9, 55, 56, 59, 81, 65, 45, \
28, 60, 55, 93, 39, 90, 28, 124, 106, 16, 20, 104, 119, 8, 109, 26, \
106, 9, 97, 13, 99, 15, 119, 20, 105, 117, 98, 103, 118, 1, 126, 29, \
97, 122, 17, 15, 114, 110, 3, 5, 125, 125, 99, 126, 119, 102, 30, \
122, 2, 117}

In[]:= key
Out[]= {80, 103, 2, 116, 113, 102, 118, 25, 99, 8, 19, 23, 116, \
125, 107, 25, 99, 109, 114, 102, 14, 121, 115, 31, 9, 117, 113, 111, \
5, 4, 127, 28, 122, 101, 8, 110, 14, 18, 124, 106, 16, 20, 104, 119, \
8, 109, 26, 106, 9, 97, 13, 99, 15, 119, 20, 105, 117, 98, 103, 118, \
1, 126, 29, 97, 122, 17, 15, 114, 110, 3, 5, 125, 125, 99, 126, 119, \
102, 30, 122, 2, 117}

In[]:= BitXor[key, blocks[[2]]]
Out[]= {0, 101, 72, 69, 0, 87, 69, 69, 68, 0, 79, 70, 0, 67, 82, \
73, 77, 69, 0, 66, 69, 65, 82, 83, 0, 66, 73, 84, 84, 69, 82, 0, 70, \
82, 85, 73, 84, 14, 0, 0, 0, 0, 0, 0, 0, 0, 0, 0, 0, 0, 0, 0, 0, 0, \
0, 0, 0, 0, 0, 0, 0, 0, 0, 0, 0, 0, 0, 0, 0, 0, 0, 0, 0, 0, 0, 0, 0, \
0, 0, 0, 0}
\end{lstlisting}

Зашифрованный байт это 2, байт из ключа это 103, $2 \oplus 103=101$ и 101 это ASCII-код символа \q{e}.
Чему должен равнятся этот байт ключа, чтобы ASCII-код был 116 (для символа  \q{t})?
$2 \oplus 116=118$, присвоим 118 второму байту в ключе \dots

\begin{lstlisting}[style=custommath]
key = {80, 118, 2, 116, 113, 102, 118, 25, 99, 8, 19, 23, 116, 125, 
  107, 25, 99, 109, 114, 102, 14, 121, 115, 31, 9, 117, 113, 111, 5, 
  4, 127, 28, 122, 101, 8, 110, 14, 18, 124, 106, 16, 20, 104, 119, 8,
   109, 26, 106, 9, 97, 13, 99, 15, 119, 20, 105, 117, 98, 103, 118, 
  1, 126, 29, 97, 122, 17, 15, 114, 110, 3, 5, 125, 125, 99, 126, 119,
   102, 30, 122, 2, 117}
\end{lstlisting}

\dots и снова дешифруем весь файл.

\begin{figure}[H]
\centering
\myincludegraphics{ff/XOR/mask_1/mc_decrypted2.png}
\caption{Дешифрованный файл в Midnight Commander, вторая попытка}
\end{figure}

Ух ты, теперь грамматика корректна, и все фразы начинаются с корректных букв.
Но все таки, регистр подозрителен.
С чего бы разработчику игры записывать их в такой манере?
Может быть наш ключ все еще неправилен?

% TODO ASCII table somewhere in the book
Изучая таблицу ASCII мы можем заметить что ASCII-коды для букв в верхнем и нижнем регистре отличаются только на один бит
(6-й бит, если считать с первого, 0b100000):

\begin{figure}[H]
\centering
\includegraphics[width=0.7\textwidth]{ascii.png}
\caption{7-битная таблица \ac{ASCII} в Emacs}
\end{figure}

6-й бит, выставленный в нулевом байте, В десятичном виде это будет 32.
Но 32 это ASCII-код пробела!

Действительно, можно менять регистр просто применяя XOR к ASCII-коду, с 32 (больше об этом: \myref{toupper_bit}).

Возможно ли, что пустые лакуны в файле это не нулевые байты, а скорее содержащие пробелы?
Еще раз модифицируем наш XOR-ключ (я про-XOR-ю каждый байт ключа с 32):

\begin{lstlisting}[style=custommath]
(* "32" это скаляр, и "key" это вектор, но это OK *)

In[]:= key3 = BitXor[32, key]
Out[]= {112, 86, 34, 84, 81, 70, 86, 57, 67, 40, 51, 55, 84, 93, 75, \
57, 67, 77, 82, 70, 46, 89, 83, 63, 41, 85, 81, 79, 37, 36, 95, 60, \
90, 69, 40, 78, 46, 50, 92, 74, 48, 52, 72, 87, 40, 77, 58, 74, 41, \
65, 45, 67, 47, 87, 52, 73, 85, 66, 71, 86, 33, 94, 61, 65, 90, 49, \
47, 82, 78, 35, 37, 93, 93, 67, 94, 87, 70, 62, 90, 34, 85}

In[]:= DecryptBlock[blk_] := BitXor[key3, blk]
\end{lstlisting}

И снова дешифруем входной файл:

\begin{figure}[H]
\centering
\myincludegraphics{ff/XOR/mask_1/mc_decrypted.png}
\caption{Дешифрованный файл в Midnight Commander, последняя попытка}
\end{figure}

(Расшифрованный файл доступен здесь:
\url{\GitHubBlobMasterURL/ff/XOR/mask_1/files/decrypted.dat.bz2}.)

Несомненно, это корректный исходный файл.
Да, и мы видим числа в начале каждого блока. Должно быть это и есть источник некорректного XOR-ключа.
Как выходит, самый встречающийся 81-байтный блок в файле это блок заполненный пробелами и содержащий символ \q{1} на месте
второго байта.
Действительно, как-то так получилось что многие блоки здесь перемежаются с этим блоком.
Может быть это что-то вроде выравнивания (padding) для коротких фраз/сообщений?
Другой часто встречающийся 81-байтный блок также заполнен пробелами, но с другой цифрой, следовательно,
они отличаются только вторым байтом.

Вот и всё! Теперь мы можем написать утилиту для зашифрования файла назад, и, может быть, модифицировать его перед этим

Файл для Mathematica можно скачать здесь:\\
\url{\GitHubBlobMasterURL/ff/XOR/mask_1/files/XOR_mask_1.nb}.

Итог: XOR-шифрование не надежно вообще. Вероятно, разработчик игры хотел просто скрыть внутренности игры от игрока,
ничего более серьезного.
Все же, шифрование вроде этого крайне популярно вследствии его простоты, так что многие реверс инженеры обычно хорошо
с этим знакомы.

}
\FR{\mysection{Fichier de sauvegarde du jeu Millenium}
\label{Millenium_DOS_game}
\myindex{MS-DOS}

\q{Millenium Return to Earth} est un ancien jeu DOS (1991), qui vous permet d'extraire
des ressources, de construire des vaisseaux, de les équiper et de les envoyer sur d'autres
planêtes, et ainsi de suite\footnote{Il peut être téléchargé librement \href{http://go.yurichev.com/17316}{ici}}.

Comme beaucoup d'autres jeux, il vous permet de sauvegarder l'état du jeu dans un fichier.

Regardons si l'on peut y trouver quelque chose.

\clearpage
Donc, il y a des mines dans le jeu.
Sur certaines planêtes, les mines rapportent plus vite, sur d'autres, moins vite. 
L'ensemble des ressources est aussi différent.

Ici, nous pouvons voir quelles ressources sont actuellement extraites.

\begin{figure}[H]
\centering
\myincludegraphics{ff/millenium/1.png}
\caption{Mine: état 1}
\label{fig:mill_1}
\end{figure}

Sauvegardons l'état du jeu.
C'est un fichier de 9538 octets.

Attendons quelques \q{jours} dans le jeu, et maintenant, nous avons plus de ressources
extraites des mines.

\begin{figure}[H]
\centering
\myincludegraphics{ff/millenium/2.png}
\caption{Mine: état 2}
\label{fig:mill_2}
\end{figure}

Sauvegardons à nouveau l'état du jeu.

Maintenant, essayons juste de comparer au niveau binaire les fichiers de sauvegarde
en utilisant le simple utilitaire DOS/Windows FC:

\lstinputlisting{ff/millenium/fc_result.txt}

La sortie est incomplète ici, il y a plus de différences, mais j'ai tronqué le résultat
pour montrer ce qu'il y a de plus intéressant.

Dans le premier état, nous avons 14 \q{unités} d'hydrogène et 102 \q{unités} d'oxygène.

Nous avons respectivement 22 et 155 \q{unités} dans le second état.
Si ces valeurs sont sauvées dans le fichier de sauvegarde, nous devrions les voir
dans la différence.
Et en effet, nous les voyons.
Il y a 0x0E (14) à la position 0xBDA et cette valeur est à 0x16 (22) dans la nouvelle
version du fichier.
Ceci est probablement l'hydrogène.
Il y a 0x66 (102) à la position 0xBDC dans la vieille version et x9B (155) dans la
nouvelle version du fichier.
Il semble que ça soit l'oxygène.

Les deux fichiers sont disponibles sur le site web pour ceux qui veulent les inspecter
(ou expérimenter) plus:
\href{http://go.yurichev.com/17212}{beginners.re}.

\clearpage
Voici la nouvelle version du fichier ouverte dans Hiew, j'ai marqué les valeurs relatives
aux ressources extraites dans le jeu:

\begin{figure}[H]
\centering
\myincludegraphics{ff/millenium/hiew3.png}
\caption{Hiew: état 1}
\label{fig:mill_hiew3}
\end{figure}

Vérifions chacune d'elles.

Ce sont clairement des valeurs 16-bits: ce n'est pas étonnant pour un logiciel DOS
16-bit où le type \Tint fait 16-bit.

\clearpage
Vérifions nos hypothèses.
Nous allons écrire la valeur 1234 (0x4D2) à la première position (ceci doit être
l'hydrogène):

\begin{figure}[H]
\centering
\myincludegraphics{ff/millenium/hiew4.png}
\caption{Hiew: écrivons 1234 (0x4D2) ici}
\label{fig:mill_hiew4}
\end{figure}

Puis nous chargeons le fichier modifié dans le jeu et jettons un coup d'\oe{}il aux
statistiques des mines:

\begin{figure}[H]
\centering
\myincludegraphics{ff/millenium/5.png}
\caption{Vérifions la valeur pour l'hydrogène}
\label{fig:mill_5}
\end{figure}

Donc oui, c'est bien ça.

\clearpage
Maintenant essayons de finir le jeu le plus vite possible, mettons les valeurs maximales
partout:

\begin{figure}[H]
\centering
\myincludegraphics{ff/millenium/hiew7.png}
\caption{Hiew: mettons les valeurs maximales}
\label{fig:mill_hiew7}
\end{figure}

0xFFFF représente 65535, donc oui, nous avons maintenant beaucoup de ressources:

\begin{figure}[H]
\centering
\myincludegraphics{ff/millenium/6.png}
\caption{Toutes les ressources sont en effet à 65535 (0xFFFF)}
\label{fig:mill_6}
\end{figure}

\clearpage
Laissons passer quelques \q{jours} dans le jeu et oups!
Nous avons un niveau plus bas pour quelques ressources:

\begin{figure}[H]
\centering
\myincludegraphics{ff/millenium/8.png}
\caption{Dépassement des variables de ressource}
\label{fig:mill_8}
\end{figure}

C'est juste un dépassement.

Le développeur du jeu n'a probablement pas pensé à un niveau aussi élevé de ressources,
donc il n'a pas dû mettre des tests de dépassement, mais les mines \q{travaillent}
dans le jeu, des ressources sont extraites, c'est pourquoi il y a des dépassements.
Apparemment, c'est une mauvaise idée d'être aussi avide.

Il y a sans doute beaucoup plus de valeurs sauvées dans ce fichier.

Ceci est donc une méthode très simple de tricher dans les jeux.
Les fichiers des meilleurs scores peuvent souvent être modifiés comme ceci.

Plus d'informations sur la comparaison des fichiers et des snapshots de mémoire:
\myref{snapshots_comparing}.
}

\EN{% TODO translate
\mysection{Breaking simple executable cryptor}

I've got an executable file which is encrypted by relatively simple encryption.
\href{\GitHubBlobMasterURL/examples/simple_exec_crypto/files/cipher.bin}{Here is it} (only executable section is left here).

First, all encryption function does is just adds number of position in buffer to the byte.
Here is how this can be encoded in Python:

\begin{lstlisting}[caption=Python script,style=custompy]
#!/usr/bin/env python
def e(i, k):
    return chr ((ord(i)+k) % 256)

def encrypt(buf):
    return e(buf[0], 0)+ e(buf[1], 1)+ e(buf[2], 2) + e(buf[3], 3)+ e(buf[4], 4)+ e(buf[5], 5)+ e(buf[6], 6)+ e(buf[7], 7)+
           e(buf[8], 8)+ e(buf[9], 9)+ e(buf[10], 10)+ e(buf[11], 11)+ e(buf[12], 12)+ e(buf[13], 13)+ e(buf[14], 14)+ e(buf[15], 15)
\end{lstlisting}

Hence, if you encrypt buffer with 16 zeros, you'll get \emph{0, 1, 2, 3 ... 12, 13, 14, 15}.

\myindex{Propagating Cipher Block Chaining}
Propagating Cipher Block Chaining (PCBC) is also used, here is how it works:

\begin{figure}[H]
\centering
\myincludegraphics{examples/simple_exec_crypto/601px-PCBC_encryption.png}
\caption{Propagating Cipher Block Chaining encryption (image is taken from Wikipedia article)}
\end{figure}

The problem is that it's too boring to recover IV (Initialization Vector) each time.
Brute-force is also not an option, because IV is too long (16 bytes).
Let's see, if it's possible to recover IV for arbitrary encrypted executable file?

Let's try simple frequency analysis.
This is 32-bit x86 executable code, so let's gather statistics about most frequent bytes and opcodes.
I tried huge oracle.exe file from Oracle RDBMS version 11.2 for windows x86 and I've found that the most frequent byte (no surprise) is zero (~10\%).
The next most frequent byte is (again, no surprise) 0xFF (~5\%).
The next is 0x8B (~5\%).

\myindex{x86!\Instructions!MOV}
0x8B is opcode for \INS{MOV}, this is indeed one of the most busy x86 instructions.
Now what about popularity of zero byte?
If compiler needs to encode value bigger than 127, it has to use 32-bit displacement instead of 8-bit one, but large values are very rare,
so it is padded by zeros.
\myindex{x86!\Instructions!LEA}
\myindex{x86!\Instructions!PUSH}
\myindex{x86!\Instructions!CALL}
This is at least in \INS{LEA}, \INS{MOV}, \INS{PUSH}, \INS{CALL}.

For example:

\begin{lstlisting}[style=customasmx86]
8D B0 28 01 00 00                 lea     esi, [eax+128h]
8D BF 40 38 00 00                 lea     edi, [edi+3840h]
\end{lstlisting}

Displacements bigger than 127 are very popular, but they are rarely exceeds 0x10000
(indeed, such large memory buffers/structures are also rare).

Same story with \INS{MOV}, large constants are rare, the most heavily used are 0, 1, 10, 100, $2^n$, and so on.
Compiler has to pad small constants by zeros to represent them as 32-bit values:

\begin{lstlisting}[style=customasmx86]
BF 02 00 00 00                    mov     edi, 2
BF 01 00 00 00                    mov     edi, 1
\end{lstlisting}

Now about 00 and FF bytes combined: jumps (including conditional) and calls can pass execution flow forward or backwards, but very often,
within the limits of the current executable module.
If forward, displacement is not very big and also padded with zeros.
If backwards, displacement is represented as negative value, so padded with FF bytes.
For example, transfer execution flow forward:

\begin{lstlisting}[style=customasmx86]
E8 43 0C 00 00                    call    _function1
E8 5C 00 00 00                    call    _function2
0F 84 F0 0A 00 00                 jz      loc_4F09A0
0F 84 EB 00 00 00                 jz      loc_4EFBB8
\end{lstlisting}

Backwards:

\begin{lstlisting}[style=customasmx86]
E8 79 0C FE FF                    call    _function1
E8 F4 16 FF FF                    call    _function2
0F 84 F8 FB FF FF                 jz      loc_8212BC
0F 84 06 FD FF FF                 jz      loc_FF1E7D
\end{lstlisting}

FF byte is also very often occurred in negative displacements like these:

\begin{lstlisting}[style=customasmx86]
8D 85 1E FF FF FF                 lea     eax, [ebp-0E2h]
8D 95 F8 5C FF FF                 lea     edx, [ebp-0A308h]
\end{lstlisting}

So far so good. Now we have to try various 16-byte keys, decrypt executable section and measure how often 00, FF and 8B bytes are occurred.
Let's also keep in sight how PCBC decryption works:

\begin{figure}[H]
\centering
\myincludegraphics{examples/simple_exec_crypto/640px-PCBC_decryption.png}
\caption{Propagating Cipher Block Chaining decryption (image is taken from Wikipedia article)}
\end{figure}

The good news is that we don't really have to decrypt whole piece of data, but only slice by slice, this is exactly how I did in my previous example: \myref{XOR_mask_2}.

Now I'm trying all possible bytes (0..255) for each byte in key and just pick the byte producing maximal amount of 00/FF/8B bytes in a decrypted slice:

\begin{lstlisting}[style=custompy]
#!/usr/bin/env python
import sys, hexdump, array, string, operator

KEY_LEN=16

def chunks(l, n):
    # split n by l-byte chunks
    # https://stackoverflow.com/q/312443
    n = max(1, n)
    return [l[i:i + n] for i in range(0, len(l), n)]

def read_file(fname):
    file=open(fname, mode='rb')
    content=file.read()
    file.close()
    return content

def decrypt_byte (c, key):
    return chr((ord(c)-key) % 256)

def XOR_PCBC_step (IV, buf, k):
    prev=IV
    rt=""
    for c in buf:
	new_c=decrypt_byte(c, k)
        plain=chr(ord(new_c)^ord(prev))
	prev=chr(ord(c)^ord(plain))
	rt=rt+plain
    return rt

each_Nth_byte=[""]*KEY_LEN

content=read_file(sys.argv[1])
# split input by 16-byte chunks:
all_chunks=chunks(content, KEY_LEN)
for c in all_chunks:
    for i in range(KEY_LEN):
        each_Nth_byte[i]=each_Nth_byte[i] + c[i]

# try each byte of key
for N in range(KEY_LEN):
    print "N=", N
    stat={}
    for i in range(256):
        tmp_key=chr(i)
	tmp=XOR_PCBC_step(tmp_key,each_Nth_byte[N], N)
        # count 0, FFs and 8Bs in decrypted buffer:
	important_bytes=tmp.count('\x00')+tmp.count('\xFF')+tmp.count('\x8B')
	stat[i]=important_bytes
    sorted_stat = sorted(stat.iteritems(), key=operator.itemgetter(1), reverse=True)
    print sorted_stat[0]
\end{lstlisting}

(Source code can be downloaded \href{\GitHubBlobMasterURL/examples/simple_exec_crypto/files/decrypt.py}{here}.)

I run it and here is a key for which 00/FF/8B bytes presence in decrypted buffer is maximal:

\begin{lstlisting}
N= 0
(147, 1224)
N= 1
(94, 1327)
N= 2
(252, 1223)
N= 3
(218, 1266)
N= 4
(38, 1209)
N= 5
(192, 1378)
N= 6
(199, 1204)
N= 7
(213, 1332)
N= 8
(225, 1251)
N= 9
(112, 1223)
N= 10
(143, 1177)
N= 11
(108, 1286)
N= 12
(10, 1164)
N= 13
(3, 1271)
N= 14
(128, 1253)
N= 15
(232, 1330)
\end{lstlisting}

Let's write decryption utility with the key we got:

\begin{lstlisting}[style=custompy]
#!/usr/bin/env python
import sys, hexdump, array

def xor_strings(s,t):
    # \verb|https://en.wikipedia.org/wiki/XOR_cipher#Example_implementation|
    """xor two strings together"""
    return "".join(chr(ord(a)^ord(b)) for a,b in zip(s,t))

IV=array.array('B', [147, 94, 252, 218, 38, 192, 199, 213, 225, 112, 143, 108, 10, 3, 128, 232]).tostring()

def chunks(l, n):
    n = max(1, n)
    return [l[i:i + n] for i in range(0, len(l), n)]

def read_file(fname):
    file=open(fname, mode='rb')
    content=file.read()
    file.close()
    return content

def decrypt_byte(i, k):
    return chr ((ord(i)-k) % 256)

def decrypt(buf):
    return "".join(decrypt_byte(buf[i], i) for i in range(16))

fout=open(sys.argv[2], mode='wb')

prev=IV
content=read_file(sys.argv[1])
tmp=chunks(content, 16)
for c in tmp:
    new_c=decrypt(c)
    p=xor_strings (new_c, prev)
    prev=xor_strings(c, p)
    fout.write(p)
fout.close()
\end{lstlisting}

(Source code can be downloaded \href{\GitHubBlobMasterURL/examples/simple_exec_crypto/files/decrypt2.py}{here}.)

Let's check resulting file:

\lstinputlisting{examples/simple_exec_crypto/objdump_result.txt}

Yes, this is seems correctly disassembled piece of x86 code.
The whole decryped file can be downloaded \href{\GitHubBlobMasterURL/examples/simple_exec_crypto/files/decrypted.bin}{here}.

In fact, this is text section from regedit.exe from Windows 7.
But this example is based on a real case I encountered, so just executable is different (and key), algorithm is the same.

\subsection{Other ideas to consider}

What if I would fail with such simple frequency analysis?
There are other ideas on how to measure correctness of decrypted/decompressed x86 code:

\begin{itemize}

\item Many modern compilers aligns functions on 0x10 border.
So the space left before is filled with NOPs (0x90) or other NOP instructions with known opcodes: \myref{sec:npad}.

\item Perhaps, the most frequent pattern in any assembly language is function call:\\
\TT{PUSH chain / CALL / ADD ESP, X}.
This sequence can easily detected and found.
I've even gathered statistics about average number of function arguments: \myref{args_stat}.
(Hence, this is average length of PUSH chain.)

\end{itemize}

Read more about incorrectly/correctly disassembled code: \myref{ISA_detect}.
}
\RU{\subsection{Простое шифрование используя XOR-маску}
\label{XOR_mask_1}

Я нашел одну старую игру в стиле interactive fiction в архиве \emph{if-archive}\footnote{\url{http://www.ifarchive.org/}}:

\begin{lstlisting}
The New Castle v3.5 - Text/Adventure Game
in the style of the original Infocom (tm)
type games, Zork, Collosal Cave (Adventure),
etc.  Can you solve the mystery of the
abandoned castle?
Shareware from Software Customization.
Software Customization [ASP] Version 3.5 Feb. 2000
\end{lstlisting}

Можно скачать здесь: \url{\GitHubBlobMasterURL/ff/XOR/mask_1/files/newcastle.tgz}.

Там внутри есть файл (с названием \emph{castle.dbf}), который явно зашифрован, но не настоящим криптоалгоритмом,
и оне сжат, это что-то куда проще.
Я бы даже не стал измерять уровень энтропии (\myref{entropy}) этого файла, потому что я итак уверен, что он низкий.
Вот как он выглядит в Midnight Commander:

\begin{figure}[H]
\centering
\myincludegraphics{ff/XOR/mask_1/mc_encrypted.png}
\caption{Зашифрованный файл в Midnight Commander}
\end{figure}

Зашифрованный файл можно скачать здесь:
\url{\GitHubBlobMasterURL/ff/XOR/mask_1/files/castle.dbf.bz2}.

Можно ли расшифровать его без доступа к программе, используя просто этот файл?

Тут явно просматривается повторяющаяся строка. 
Если использовалось простое шифрование с XOR-маской, такие повторяющиеся строки это явное свидетельство,
потому что, вероятно, тут были длинные лакуны с нулевыми байтами, которые, в свою очередь, присутствуют
во мноигих исполняемых файлах, и в остальных бинарных файлах.

\myindex{UNIX!xxd}
Вот дам начала этого файла используя утилиту \emph{xxd} из UNIX:

\lstinputlisting{ff/XOR/mask_1/xxd_result.txt}

Давайте держаться за повторяющуюся строку \TT{iubgv}.
Глядя на этот дамп, мы можем легко увидеть, что период повторений этой строки это 0x51 или 81.
Вероятно, 81 это длина блока?
Длина файла 1658961, и она может быть поделена на 81 без остатка (и тогда там 20481 блоков).

Теперь я буду использовать Mathematica для анализа, есть ли тут повторяющиеся 81-байтные блоки в файле?
Я разделю входной файл на 81-байтные блоки и затем использую ф-цию
\emph{Tally[]}\footnote{\url{https://reference.wolfram.com/language/ref/Tally.html}}
которая просто считает, сколько раз каждый элемент встретился во входном списке.
Вывод Tally не отсортирован, так что я также добавлю ф-цию \emph{Sort[]} для сортировки его по кол-ву вхождений
в нисходящем порядке.

\begin{lstlisting}[style=custommath]
input = BinaryReadList["/home/dennis/.../castle.dbf"];

blocks = Partition[input, 81];

stat = Sort[Tally[blocks], #1[[2]] > #2[[2]] &]
\end{lstlisting}

И вот вывод:

\begin{lstlisting}[style=custommath]
{{{80, 103, 2, 116, 113, 102, 118, 25, 99, 8, 19, 23, 116, 125, 107, 
   25, 99, 109, 114, 102, 14, 121, 115, 31, 9, 117, 113, 111, 5, 4, 
   127, 28, 122, 101, 8, 110, 14, 18, 124, 106, 16, 20, 104, 119, 8, 
   109, 26, 106, 9, 97, 13, 99, 15, 119, 20, 105, 117, 98, 103, 118, 
   1, 126, 29, 97, 122, 17, 15, 114, 110, 3, 5, 125, 125, 99, 126, 
   119, 102, 30, 122, 2, 117}, 1739}, 
{{80, 100, 2, 116, 113, 102, 118, 25, 99, 8, 19, 23, 116, 
   125, 107, 25, 99, 109, 114, 102, 14, 121, 115, 31, 9, 117, 113, 
   111, 5, 4, 127, 28, 122, 101, 8, 110, 14, 18, 124, 106, 16, 20, 
   104, 119, 8, 109, 26, 106, 9, 97, 13, 99, 15, 119, 20, 105, 117, 
   98, 103, 118, 1, 126, 29, 97, 122, 17, 15, 114, 110, 3, 5, 125, 
   125, 99, 126, 119, 102, 30, 122, 2, 117}, 1422}, 
{{80, 101, 2, 116, 113, 102, 118, 25, 99, 8, 19, 23, 116, 
   125, 107, 25, 99, 109, 114, 102, 14, 121, 115, 31, 9, 117, 113, 
   111, 5, 4, 127, 28, 122, 101, 8, 110, 14, 18, 124, 106, 16, 20, 
   104, 119, 8, 109, 26, 106, 9, 97, 13, 99, 15, 119, 20, 105, 117, 
   98, 103, 118, 1, 126, 29, 97, 122, 17, 15, 114, 110, 3, 5, 125, 
   125, 99, 126, 119, 102, 30, 122, 2, 117}, 1012},
{{80, 120, 2, 116, 113, 102, 118, 25, 99, 8, 19, 23, 116, 
   125, 107, 25, 99, 109, 114, 102, 14, 121, 115, 31, 9, 117, 113, 
   111, 5, 4, 127, 28, 122, 101, 8, 110, 14, 18, 124, 106, 16, 20, 
   104, 119, 8, 109, 26, 106, 9, 97, 13, 99, 15, 119, 20, 105, 117, 
   98, 103, 118, 1, 126, 29, 97, 122, 17, 15, 114, 110, 3, 5, 125, 
   125, 99, 126, 119, 102, 30, 122, 2, 117}, 377},

...

{{80, 2, 74, 49, 113, 21, 62, 88, 39, 71, 68, 23, 63, 51, 36, 78, 48, 
   108, 114, 102, 14, 121, 115, 31, 9, 117, 113, 111, 5, 4, 127, 28, 
   122, 101, 8, 110, 14, 18, 124, 106, 16, 20, 104, 119, 8, 109, 26, 
   106, 9, 97, 13, 99, 15, 119, 20, 105, 117, 98, 103, 118, 1, 126, 
   29, 97, 122, 17, 15, 114, 110, 3, 5, 125, 125, 99, 126, 119, 102, 
   30, 122, 2, 117}, 1},
{{80, 1, 74, 59, 113, 45, 56, 86, 52, 91, 19, 64, 60, 60, 63, 
   25, 38, 59, 59, 42, 14, 53, 38, 77, 66, 38, 113, 38, 75, 4, 43, 84,
    63, 101, 64, 43, 79, 64, 40, 57, 16, 91, 46, 119, 69, 40, 84, 117,
    9, 97, 13, 99, 15, 119, 20, 105, 117, 98, 103, 118, 1, 126, 29, 
   97, 122, 17, 15, 114, 110, 3, 5, 125, 125, 99, 126, 119, 102, 30, 
   122, 2, 117}, 1},
{{80, 2, 74, 49, 113, 49, 51, 92, 39, 8, 92, 81, 116, 62, 57, 
   80, 46, 40, 114, 36, 75, 56, 33, 76, 9, 55, 56, 59, 81, 65, 45, 28,
    60, 55, 93, 39, 90, 28, 124, 106, 16, 20, 104, 119, 8, 109, 26, 
   106, 9, 97, 13, 99, 15, 119, 20, 105, 117, 98, 103, 118, 1, 126, 
   29, 97, 122, 17, 15, 114, 110, 3, 5, 125, 125, 99, 126, 119, 102, 
   30, 122, 2, 117}, 1}}
\end{lstlisting}

Вывод Tally это список пар, каждая пара это 81-байтный блок и количество раз, сколько он встретился в файле.
Мы видим, что наиболее частно встречающийся блок это первый, он встретился 1739 раз.
Второй встретился 1422 раза. Есть и другие: 1012 раза, 377 раз, итд.
81-байтные блоки, встреченные лишь один раз, находятся в конце вывода.

Попробуем сравнить эти блоки. Первый и второй.
Есть ли в Mathematica ф-ция для сравнения списков/массивов?
Наверняка есть, но в педагогических целях, я буду использоват операцию XOR для сравнения.
Действительно: если байты во входных массивах равны друг другу, результат операции XOR это 0.
Если не равны, результат будет ненулевой.

Сравним первый блок (встречается 1739 раз) и второй (встречается 1422 раз):

\begin{lstlisting}[style=custommath]
In[]:= BitXor[stat[[1]][[1]], stat[[2]][[1]]]
Out[]= {0, 3, 0, 0, 0, 0, 0, 0, 0, 0, 0, 0, 0, 0, 0, 0, 0, 0, 0, \
0, 0, 0, 0, 0, 0, 0, 0, 0, 0, 0, 0, 0, 0, 0, 0, 0, 0, 0, 0, 0, 0, 0, \
0, 0, 0, 0, 0, 0, 0, 0, 0, 0, 0, 0, 0, 0, 0, 0, 0, 0, 0, 0, 0, 0, 0, \
0, 0, 0, 0, 0, 0, 0, 0, 0, 0, 0, 0, 0, 0, 0, 0}
\end{lstlisting}

Они отличаются только вторым байтом.

Сравним второй блок (встречается 1422 раза) и третий (встречается 1012 раз):

\begin{lstlisting}[style=custommath]
In[]:= BitXor[stat[[2]][[1]], stat[[3]][[1]]]
Out[]= {0, 1, 0, 0, 0, 0, 0, 0, 0, 0, 0, 0, 0, 0, 0, 0, 0, 0, 0, \
0, 0, 0, 0, 0, 0, 0, 0, 0, 0, 0, 0, 0, 0, 0, 0, 0, 0, 0, 0, 0, 0, 0, \
0, 0, 0, 0, 0, 0, 0, 0, 0, 0, 0, 0, 0, 0, 0, 0, 0, 0, 0, 0, 0, 0, 0, \
0, 0, 0, 0, 0, 0, 0, 0, 0, 0, 0, 0, 0, 0, 0, 0}
\end{lstlisting}

Они тоже отличаются только вторым байтом.

Так или иначе, попробуем использовать самый встречающийся блок как XOR-ключ и попробуем расшифровать первые 4 81-байтных
блока в файле:

\begin{lstlisting}[style=custommath]
In[]:= key = stat[[1]][[1]]
Out[]= {80, 103, 2, 116, 113, 102, 118, 25, 99, 8, 19, 23, 116, \
125, 107, 25, 99, 109, 114, 102, 14, 121, 115, 31, 9, 117, 113, 111, \
5, 4, 127, 28, 122, 101, 8, 110, 14, 18, 124, 106, 16, 20, 104, 119, \
8, 109, 26, 106, 9, 97, 13, 99, 15, 119, 20, 105, 117, 98, 103, 118, \
1, 126, 29, 97, 122, 17, 15, 114, 110, 3, 5, 125, 125, 99, 126, 119, \
102, 30, 122, 2, 117}

In[]:= ToASCII[val_] := If[val == 0, " ", FromCharacterCode[val, "PrintableASCII"]]

In[]:= DecryptBlockASCII[blk_] := Map[ToASCII[#] &, BitXor[key, blk]]

In[]:= DecryptBlockASCII[blocks[[1]]]
Out[]= {" ", " ", " ", " ", " ", " ", " ", " ", " ", " ", " ", " \
", " ", " ", " ", " ", " ", " ", " ", " ", " ", " ", " ", " ", " ", " \
", " ", " ", " ", " ", " ", " ", " ", " ", " ", " ", " ", " ", " ", " \
", " ", " ", " ", " ", " ", " ", " ", " ", " ", " ", " ", " ", " ", " \
", " ", " ", " ", " ", " ", " ", " ", " ", " ", " ", " ", " ", " ", " \
", " ", " ", " ", " ", " ", " ", " ", " ", " ", " ", " ", " ", " "}

In[]:= DecryptBlockASCII[blocks[[2]]]
Out[]= {" ", "e", "H", "E", " ", "W", "E", "E", "D", " ", "O", \
"F", " ", "C", "R", "I", "M", "E", " ", "B", "E", "A", "R", "S", " ", \
"B", "I", "T", "T", "E", "R", " ", "F", "R", "U", "I", "T", "?", \
" ", " ", " ", " ", " ", " ", " ", " ", " ", " ", " ", " ", " ", " ", \
" ", " ", " ", " ", " ", " ", " ", " ", " ", " ", " ", " ", " ", " ", \
" ", " ", " ", " ", " ", " ", " ", " ", " ", " ", " ", " ", " ", " ", \
" "}

In[]:= DecryptBlockASCII[blocks[[3]]]
Out[]= {" ", "?", " ", " ", " ", " ", " ", " ", " ", " ", " \
", " ", " ", " ", " ", " ", " ", " ", " ", " ", " ", " ", " ", " ", " \
", " ", " ", " ", " ", " ", " ", " ", " ", " ", " ", " ", " ", " ", " \
", " ", " ", " ", " ", " ", " ", " ", " ", " ", " ", " ", " ", " ", " \
", " ", " ", " ", " ", " ", " ", " ", " ", " ", " ", " ", " ", " ", " \
", " ", " ", " ", " ", " ", " ", " ", " ", " ", " ", " ", " ", " ", " \
"}

In[]:= DecryptBlockASCII[blocks[[4]]]
Out[]= {" ", "f", "H", "O", " ", "K", "N", "O", "W", "S", " ", \
"W", "H", "A", "T", " ", "E", "V", "I", "L", " ", "L", "U", "R", "K", \
"S", " ", "I", "N", " ", "T", "H", "E", " ", "H", "E", "A", "R", "T", \
"S", " ", "O", "F", " ", "M", "E", "N", "?", " ", " ", " ", " ", \
" ", " ", " ", " ", " ", " ", " ", " ", " ", " ", " ", " ", " ", " ", \
" ", " ", " ", " ", " ", " ", " ", " ", " ", " ", " ", " ", " ", " ", \
" "}
\end{lstlisting}

(Я заменил непечатаемые символы на \q{?}.)

Мы видим что первый и третий блоки пустые (или почти пустые),
но второй и четвертый имеют ясно различимые английские слова/фразы.
Похоже что наше предположение насчет ключа верно (как минимум частично).
Это означает, что самый встречающийся 81-байтный блок в файле находится в местах лакун с нулевыми байтами
или что-то в этом роде.

Попробуем расшифровать весь файл:

\begin{lstlisting}[style=custommath]
DecryptBlock[blk_] := BitXor[key, blk]

decrypted = Map[DecryptBlock[#] &, blocks];

BinaryWrite["/home/dennis/.../tmp", Flatten[decrypted]]

Close["/home/dennis/.../tmp"]
\end{lstlisting}

\begin{figure}[H]
\centering
\myincludegraphics{ff/XOR/mask_1/mc_decrypted1.png}
\caption{Расшифрованный файл в Midnight Commander, первая попытка}
\end{figure}

Выглядит как английские фразы для какой-то игры, но что-то не так.
Прежде всего, регистр инвертирован: фразы и некоторые слова начинаются со строчных букв,
в то время как остальные буквы заглавные.
Также, некоторые фразы начинаются с не тех букв.
Посмотрите на самую первую фразу: \q{eHE WEED OF CRIME BEARS BITTER FRUIT}.
Что такое \q{eHE}? Разве не \q{tHE} тут должно быть?
Возможно ли что наш ключ для дешифрования имеет неверный байт в этом месте?

Посмотрим снова на второй блок в файле, на ключ и на результат дешифрования:

\begin{lstlisting}[style=custommath]
In[]:= blocks[[2]]
Out[]= {80, 2, 74, 49, 113, 49, 51, 92, 39, 8, 92, 81, 116, 62, \
57, 80, 46, 40, 114, 36, 75, 56, 33, 76, 9, 55, 56, 59, 81, 65, 45, \
28, 60, 55, 93, 39, 90, 28, 124, 106, 16, 20, 104, 119, 8, 109, 26, \
106, 9, 97, 13, 99, 15, 119, 20, 105, 117, 98, 103, 118, 1, 126, 29, \
97, 122, 17, 15, 114, 110, 3, 5, 125, 125, 99, 126, 119, 102, 30, \
122, 2, 117}

In[]:= key
Out[]= {80, 103, 2, 116, 113, 102, 118, 25, 99, 8, 19, 23, 116, \
125, 107, 25, 99, 109, 114, 102, 14, 121, 115, 31, 9, 117, 113, 111, \
5, 4, 127, 28, 122, 101, 8, 110, 14, 18, 124, 106, 16, 20, 104, 119, \
8, 109, 26, 106, 9, 97, 13, 99, 15, 119, 20, 105, 117, 98, 103, 118, \
1, 126, 29, 97, 122, 17, 15, 114, 110, 3, 5, 125, 125, 99, 126, 119, \
102, 30, 122, 2, 117}

In[]:= BitXor[key, blocks[[2]]]
Out[]= {0, 101, 72, 69, 0, 87, 69, 69, 68, 0, 79, 70, 0, 67, 82, \
73, 77, 69, 0, 66, 69, 65, 82, 83, 0, 66, 73, 84, 84, 69, 82, 0, 70, \
82, 85, 73, 84, 14, 0, 0, 0, 0, 0, 0, 0, 0, 0, 0, 0, 0, 0, 0, 0, 0, \
0, 0, 0, 0, 0, 0, 0, 0, 0, 0, 0, 0, 0, 0, 0, 0, 0, 0, 0, 0, 0, 0, 0, \
0, 0, 0, 0}
\end{lstlisting}

Зашифрованный байт это 2, байт из ключа это 103, $2 \oplus 103=101$ и 101 это ASCII-код символа \q{e}.
Чему должен равнятся этот байт ключа, чтобы ASCII-код был 116 (для символа  \q{t})?
$2 \oplus 116=118$, присвоим 118 второму байту в ключе \dots

\begin{lstlisting}[style=custommath]
key = {80, 118, 2, 116, 113, 102, 118, 25, 99, 8, 19, 23, 116, 125, 
  107, 25, 99, 109, 114, 102, 14, 121, 115, 31, 9, 117, 113, 111, 5, 
  4, 127, 28, 122, 101, 8, 110, 14, 18, 124, 106, 16, 20, 104, 119, 8,
   109, 26, 106, 9, 97, 13, 99, 15, 119, 20, 105, 117, 98, 103, 118, 
  1, 126, 29, 97, 122, 17, 15, 114, 110, 3, 5, 125, 125, 99, 126, 119,
   102, 30, 122, 2, 117}
\end{lstlisting}

\dots и снова дешифруем весь файл.

\begin{figure}[H]
\centering
\myincludegraphics{ff/XOR/mask_1/mc_decrypted2.png}
\caption{Дешифрованный файл в Midnight Commander, вторая попытка}
\end{figure}

Ух ты, теперь грамматика корректна, и все фразы начинаются с корректных букв.
Но все таки, регистр подозрителен.
С чего бы разработчику игры записывать их в такой манере?
Может быть наш ключ все еще неправилен?

% TODO ASCII table somewhere in the book
Изучая таблицу ASCII мы можем заметить что ASCII-коды для букв в верхнем и нижнем регистре отличаются только на один бит
(6-й бит, если считать с первого, 0b100000):

\begin{figure}[H]
\centering
\includegraphics[width=0.7\textwidth]{ascii.png}
\caption{7-битная таблица \ac{ASCII} в Emacs}
\end{figure}

6-й бит, выставленный в нулевом байте, В десятичном виде это будет 32.
Но 32 это ASCII-код пробела!

Действительно, можно менять регистр просто применяя XOR к ASCII-коду, с 32 (больше об этом: \myref{toupper_bit}).

Возможно ли, что пустые лакуны в файле это не нулевые байты, а скорее содержащие пробелы?
Еще раз модифицируем наш XOR-ключ (я про-XOR-ю каждый байт ключа с 32):

\begin{lstlisting}[style=custommath]
(* "32" это скаляр, и "key" это вектор, но это OK *)

In[]:= key3 = BitXor[32, key]
Out[]= {112, 86, 34, 84, 81, 70, 86, 57, 67, 40, 51, 55, 84, 93, 75, \
57, 67, 77, 82, 70, 46, 89, 83, 63, 41, 85, 81, 79, 37, 36, 95, 60, \
90, 69, 40, 78, 46, 50, 92, 74, 48, 52, 72, 87, 40, 77, 58, 74, 41, \
65, 45, 67, 47, 87, 52, 73, 85, 66, 71, 86, 33, 94, 61, 65, 90, 49, \
47, 82, 78, 35, 37, 93, 93, 67, 94, 87, 70, 62, 90, 34, 85}

In[]:= DecryptBlock[blk_] := BitXor[key3, blk]
\end{lstlisting}

И снова дешифруем входной файл:

\begin{figure}[H]
\centering
\myincludegraphics{ff/XOR/mask_1/mc_decrypted.png}
\caption{Дешифрованный файл в Midnight Commander, последняя попытка}
\end{figure}

(Расшифрованный файл доступен здесь:
\url{\GitHubBlobMasterURL/ff/XOR/mask_1/files/decrypted.dat.bz2}.)

Несомненно, это корректный исходный файл.
Да, и мы видим числа в начале каждого блока. Должно быть это и есть источник некорректного XOR-ключа.
Как выходит, самый встречающийся 81-байтный блок в файле это блок заполненный пробелами и содержащий символ \q{1} на месте
второго байта.
Действительно, как-то так получилось что многие блоки здесь перемежаются с этим блоком.
Может быть это что-то вроде выравнивания (padding) для коротких фраз/сообщений?
Другой часто встречающийся 81-байтный блок также заполнен пробелами, но с другой цифрой, следовательно,
они отличаются только вторым байтом.

Вот и всё! Теперь мы можем написать утилиту для зашифрования файла назад, и, может быть, модифицировать его перед этим

Файл для Mathematica можно скачать здесь:\\
\url{\GitHubBlobMasterURL/ff/XOR/mask_1/files/XOR_mask_1.nb}.

Итог: XOR-шифрование не надежно вообще. Вероятно, разработчик игры хотел просто скрыть внутренности игры от игрока,
ничего более серьезного.
Все же, шифрование вроде этого крайне популярно вследствии его простоты, так что многие реверс инженеры обычно хорошо
с этим знакомы.

}
\FR{\mysection{Fonction presque vide}
\label{Boolector}
\myindex{Boolector}
\myindex{x86!\Instructions!JMP}

Ceci est un morceau de code réel que j'ai trouvé dans Boolector\footnote{\url{https://boolector.github.io/}}:

\lstinputlisting[style=customc]{patterns/025_almost_empty/boolectormain.c}

Pourquoi quelqu'un ferait-il comme ça?
Je ne sais pas mais mon hypothèse est que \verb|boolector_main()| peut être compilée
dans une sorte de DLL ou bibliothèque dynamique, et appelée depuis une suite de test.
Certainement qu'une suite de test peut préparer les variables argc/argv comme
le ferait \ac{CRT}.

Il est intéressant de voir comment c'est compilé:

\lstinputlisting[caption=GCC 8.2 x64 \NonOptimizing (\assemblyOutput),style=customasmx86]{patterns/025_almost_empty/boolectormain_O0.s}

Ceci est OK, le prologue (non optimisé) déplace inutilement deux arguments,
\INS{CALL}, épilogue, \INS{RET}.
Mais regardons la version optimisée:

\lstinputlisting[caption=GCC 8.2 x64 \Optimizing (\assemblyOutput),style=customasmx86]{patterns/025_almost_empty/boolectormain_O3.s}

Aussi simple que ça: la pile et les registres ne sont pas touchés et \verb|boolector_main()|
a le même ensemble d'arguments.
Donc, tout ce que nous avons à faire est de passer l'exécution à une autre adresse.

Ceci est proche d'une \glslink{thunk function}{fonction thunk}.

Nous verons queelque chose de plus avancé plus tard: \myref{ARM_B_to_printf}, \myref{JMP_instead_of_RET}.
}

\EN{% TODO translate
\mysection{Breaking simple executable cryptor}

I've got an executable file which is encrypted by relatively simple encryption.
\href{\GitHubBlobMasterURL/examples/simple_exec_crypto/files/cipher.bin}{Here is it} (only executable section is left here).

First, all encryption function does is just adds number of position in buffer to the byte.
Here is how this can be encoded in Python:

\begin{lstlisting}[caption=Python script,style=custompy]
#!/usr/bin/env python
def e(i, k):
    return chr ((ord(i)+k) % 256)

def encrypt(buf):
    return e(buf[0], 0)+ e(buf[1], 1)+ e(buf[2], 2) + e(buf[3], 3)+ e(buf[4], 4)+ e(buf[5], 5)+ e(buf[6], 6)+ e(buf[7], 7)+
           e(buf[8], 8)+ e(buf[9], 9)+ e(buf[10], 10)+ e(buf[11], 11)+ e(buf[12], 12)+ e(buf[13], 13)+ e(buf[14], 14)+ e(buf[15], 15)
\end{lstlisting}

Hence, if you encrypt buffer with 16 zeros, you'll get \emph{0, 1, 2, 3 ... 12, 13, 14, 15}.

\myindex{Propagating Cipher Block Chaining}
Propagating Cipher Block Chaining (PCBC) is also used, here is how it works:

\begin{figure}[H]
\centering
\myincludegraphics{examples/simple_exec_crypto/601px-PCBC_encryption.png}
\caption{Propagating Cipher Block Chaining encryption (image is taken from Wikipedia article)}
\end{figure}

The problem is that it's too boring to recover IV (Initialization Vector) each time.
Brute-force is also not an option, because IV is too long (16 bytes).
Let's see, if it's possible to recover IV for arbitrary encrypted executable file?

Let's try simple frequency analysis.
This is 32-bit x86 executable code, so let's gather statistics about most frequent bytes and opcodes.
I tried huge oracle.exe file from Oracle RDBMS version 11.2 for windows x86 and I've found that the most frequent byte (no surprise) is zero (~10\%).
The next most frequent byte is (again, no surprise) 0xFF (~5\%).
The next is 0x8B (~5\%).

\myindex{x86!\Instructions!MOV}
0x8B is opcode for \INS{MOV}, this is indeed one of the most busy x86 instructions.
Now what about popularity of zero byte?
If compiler needs to encode value bigger than 127, it has to use 32-bit displacement instead of 8-bit one, but large values are very rare,
so it is padded by zeros.
\myindex{x86!\Instructions!LEA}
\myindex{x86!\Instructions!PUSH}
\myindex{x86!\Instructions!CALL}
This is at least in \INS{LEA}, \INS{MOV}, \INS{PUSH}, \INS{CALL}.

For example:

\begin{lstlisting}[style=customasmx86]
8D B0 28 01 00 00                 lea     esi, [eax+128h]
8D BF 40 38 00 00                 lea     edi, [edi+3840h]
\end{lstlisting}

Displacements bigger than 127 are very popular, but they are rarely exceeds 0x10000
(indeed, such large memory buffers/structures are also rare).

Same story with \INS{MOV}, large constants are rare, the most heavily used are 0, 1, 10, 100, $2^n$, and so on.
Compiler has to pad small constants by zeros to represent them as 32-bit values:

\begin{lstlisting}[style=customasmx86]
BF 02 00 00 00                    mov     edi, 2
BF 01 00 00 00                    mov     edi, 1
\end{lstlisting}

Now about 00 and FF bytes combined: jumps (including conditional) and calls can pass execution flow forward or backwards, but very often,
within the limits of the current executable module.
If forward, displacement is not very big and also padded with zeros.
If backwards, displacement is represented as negative value, so padded with FF bytes.
For example, transfer execution flow forward:

\begin{lstlisting}[style=customasmx86]
E8 43 0C 00 00                    call    _function1
E8 5C 00 00 00                    call    _function2
0F 84 F0 0A 00 00                 jz      loc_4F09A0
0F 84 EB 00 00 00                 jz      loc_4EFBB8
\end{lstlisting}

Backwards:

\begin{lstlisting}[style=customasmx86]
E8 79 0C FE FF                    call    _function1
E8 F4 16 FF FF                    call    _function2
0F 84 F8 FB FF FF                 jz      loc_8212BC
0F 84 06 FD FF FF                 jz      loc_FF1E7D
\end{lstlisting}

FF byte is also very often occurred in negative displacements like these:

\begin{lstlisting}[style=customasmx86]
8D 85 1E FF FF FF                 lea     eax, [ebp-0E2h]
8D 95 F8 5C FF FF                 lea     edx, [ebp-0A308h]
\end{lstlisting}

So far so good. Now we have to try various 16-byte keys, decrypt executable section and measure how often 00, FF and 8B bytes are occurred.
Let's also keep in sight how PCBC decryption works:

\begin{figure}[H]
\centering
\myincludegraphics{examples/simple_exec_crypto/640px-PCBC_decryption.png}
\caption{Propagating Cipher Block Chaining decryption (image is taken from Wikipedia article)}
\end{figure}

The good news is that we don't really have to decrypt whole piece of data, but only slice by slice, this is exactly how I did in my previous example: \myref{XOR_mask_2}.

Now I'm trying all possible bytes (0..255) for each byte in key and just pick the byte producing maximal amount of 00/FF/8B bytes in a decrypted slice:

\begin{lstlisting}[style=custompy]
#!/usr/bin/env python
import sys, hexdump, array, string, operator

KEY_LEN=16

def chunks(l, n):
    # split n by l-byte chunks
    # https://stackoverflow.com/q/312443
    n = max(1, n)
    return [l[i:i + n] for i in range(0, len(l), n)]

def read_file(fname):
    file=open(fname, mode='rb')
    content=file.read()
    file.close()
    return content

def decrypt_byte (c, key):
    return chr((ord(c)-key) % 256)

def XOR_PCBC_step (IV, buf, k):
    prev=IV
    rt=""
    for c in buf:
	new_c=decrypt_byte(c, k)
        plain=chr(ord(new_c)^ord(prev))
	prev=chr(ord(c)^ord(plain))
	rt=rt+plain
    return rt

each_Nth_byte=[""]*KEY_LEN

content=read_file(sys.argv[1])
# split input by 16-byte chunks:
all_chunks=chunks(content, KEY_LEN)
for c in all_chunks:
    for i in range(KEY_LEN):
        each_Nth_byte[i]=each_Nth_byte[i] + c[i]

# try each byte of key
for N in range(KEY_LEN):
    print "N=", N
    stat={}
    for i in range(256):
        tmp_key=chr(i)
	tmp=XOR_PCBC_step(tmp_key,each_Nth_byte[N], N)
        # count 0, FFs and 8Bs in decrypted buffer:
	important_bytes=tmp.count('\x00')+tmp.count('\xFF')+tmp.count('\x8B')
	stat[i]=important_bytes
    sorted_stat = sorted(stat.iteritems(), key=operator.itemgetter(1), reverse=True)
    print sorted_stat[0]
\end{lstlisting}

(Source code can be downloaded \href{\GitHubBlobMasterURL/examples/simple_exec_crypto/files/decrypt.py}{here}.)

I run it and here is a key for which 00/FF/8B bytes presence in decrypted buffer is maximal:

\begin{lstlisting}
N= 0
(147, 1224)
N= 1
(94, 1327)
N= 2
(252, 1223)
N= 3
(218, 1266)
N= 4
(38, 1209)
N= 5
(192, 1378)
N= 6
(199, 1204)
N= 7
(213, 1332)
N= 8
(225, 1251)
N= 9
(112, 1223)
N= 10
(143, 1177)
N= 11
(108, 1286)
N= 12
(10, 1164)
N= 13
(3, 1271)
N= 14
(128, 1253)
N= 15
(232, 1330)
\end{lstlisting}

Let's write decryption utility with the key we got:

\begin{lstlisting}[style=custompy]
#!/usr/bin/env python
import sys, hexdump, array

def xor_strings(s,t):
    # \verb|https://en.wikipedia.org/wiki/XOR_cipher#Example_implementation|
    """xor two strings together"""
    return "".join(chr(ord(a)^ord(b)) for a,b in zip(s,t))

IV=array.array('B', [147, 94, 252, 218, 38, 192, 199, 213, 225, 112, 143, 108, 10, 3, 128, 232]).tostring()

def chunks(l, n):
    n = max(1, n)
    return [l[i:i + n] for i in range(0, len(l), n)]

def read_file(fname):
    file=open(fname, mode='rb')
    content=file.read()
    file.close()
    return content

def decrypt_byte(i, k):
    return chr ((ord(i)-k) % 256)

def decrypt(buf):
    return "".join(decrypt_byte(buf[i], i) for i in range(16))

fout=open(sys.argv[2], mode='wb')

prev=IV
content=read_file(sys.argv[1])
tmp=chunks(content, 16)
for c in tmp:
    new_c=decrypt(c)
    p=xor_strings (new_c, prev)
    prev=xor_strings(c, p)
    fout.write(p)
fout.close()
\end{lstlisting}

(Source code can be downloaded \href{\GitHubBlobMasterURL/examples/simple_exec_crypto/files/decrypt2.py}{here}.)

Let's check resulting file:

\lstinputlisting{examples/simple_exec_crypto/objdump_result.txt}

Yes, this is seems correctly disassembled piece of x86 code.
The whole decryped file can be downloaded \href{\GitHubBlobMasterURL/examples/simple_exec_crypto/files/decrypted.bin}{here}.

In fact, this is text section from regedit.exe from Windows 7.
But this example is based on a real case I encountered, so just executable is different (and key), algorithm is the same.

\subsection{Other ideas to consider}

What if I would fail with such simple frequency analysis?
There are other ideas on how to measure correctness of decrypted/decompressed x86 code:

\begin{itemize}

\item Many modern compilers aligns functions on 0x10 border.
So the space left before is filled with NOPs (0x90) or other NOP instructions with known opcodes: \myref{sec:npad}.

\item Perhaps, the most frequent pattern in any assembly language is function call:\\
\TT{PUSH chain / CALL / ADD ESP, X}.
This sequence can easily detected and found.
I've even gathered statistics about average number of function arguments: \myref{args_stat}.
(Hence, this is average length of PUSH chain.)

\end{itemize}

Read more about incorrectly/correctly disassembled code: \myref{ISA_detect}.
}
\RU{\subsection{Простое шифрование используя XOR-маску}
\label{XOR_mask_1}

Я нашел одну старую игру в стиле interactive fiction в архиве \emph{if-archive}\footnote{\url{http://www.ifarchive.org/}}:

\begin{lstlisting}
The New Castle v3.5 - Text/Adventure Game
in the style of the original Infocom (tm)
type games, Zork, Collosal Cave (Adventure),
etc.  Can you solve the mystery of the
abandoned castle?
Shareware from Software Customization.
Software Customization [ASP] Version 3.5 Feb. 2000
\end{lstlisting}

Можно скачать здесь: \url{\GitHubBlobMasterURL/ff/XOR/mask_1/files/newcastle.tgz}.

Там внутри есть файл (с названием \emph{castle.dbf}), который явно зашифрован, но не настоящим криптоалгоритмом,
и оне сжат, это что-то куда проще.
Я бы даже не стал измерять уровень энтропии (\myref{entropy}) этого файла, потому что я итак уверен, что он низкий.
Вот как он выглядит в Midnight Commander:

\begin{figure}[H]
\centering
\myincludegraphics{ff/XOR/mask_1/mc_encrypted.png}
\caption{Зашифрованный файл в Midnight Commander}
\end{figure}

Зашифрованный файл можно скачать здесь:
\url{\GitHubBlobMasterURL/ff/XOR/mask_1/files/castle.dbf.bz2}.

Можно ли расшифровать его без доступа к программе, используя просто этот файл?

Тут явно просматривается повторяющаяся строка. 
Если использовалось простое шифрование с XOR-маской, такие повторяющиеся строки это явное свидетельство,
потому что, вероятно, тут были длинные лакуны с нулевыми байтами, которые, в свою очередь, присутствуют
во мноигих исполняемых файлах, и в остальных бинарных файлах.

\myindex{UNIX!xxd}
Вот дам начала этого файла используя утилиту \emph{xxd} из UNIX:

\lstinputlisting{ff/XOR/mask_1/xxd_result.txt}

Давайте держаться за повторяющуюся строку \TT{iubgv}.
Глядя на этот дамп, мы можем легко увидеть, что период повторений этой строки это 0x51 или 81.
Вероятно, 81 это длина блока?
Длина файла 1658961, и она может быть поделена на 81 без остатка (и тогда там 20481 блоков).

Теперь я буду использовать Mathematica для анализа, есть ли тут повторяющиеся 81-байтные блоки в файле?
Я разделю входной файл на 81-байтные блоки и затем использую ф-цию
\emph{Tally[]}\footnote{\url{https://reference.wolfram.com/language/ref/Tally.html}}
которая просто считает, сколько раз каждый элемент встретился во входном списке.
Вывод Tally не отсортирован, так что я также добавлю ф-цию \emph{Sort[]} для сортировки его по кол-ву вхождений
в нисходящем порядке.

\begin{lstlisting}[style=custommath]
input = BinaryReadList["/home/dennis/.../castle.dbf"];

blocks = Partition[input, 81];

stat = Sort[Tally[blocks], #1[[2]] > #2[[2]] &]
\end{lstlisting}

И вот вывод:

\begin{lstlisting}[style=custommath]
{{{80, 103, 2, 116, 113, 102, 118, 25, 99, 8, 19, 23, 116, 125, 107, 
   25, 99, 109, 114, 102, 14, 121, 115, 31, 9, 117, 113, 111, 5, 4, 
   127, 28, 122, 101, 8, 110, 14, 18, 124, 106, 16, 20, 104, 119, 8, 
   109, 26, 106, 9, 97, 13, 99, 15, 119, 20, 105, 117, 98, 103, 118, 
   1, 126, 29, 97, 122, 17, 15, 114, 110, 3, 5, 125, 125, 99, 126, 
   119, 102, 30, 122, 2, 117}, 1739}, 
{{80, 100, 2, 116, 113, 102, 118, 25, 99, 8, 19, 23, 116, 
   125, 107, 25, 99, 109, 114, 102, 14, 121, 115, 31, 9, 117, 113, 
   111, 5, 4, 127, 28, 122, 101, 8, 110, 14, 18, 124, 106, 16, 20, 
   104, 119, 8, 109, 26, 106, 9, 97, 13, 99, 15, 119, 20, 105, 117, 
   98, 103, 118, 1, 126, 29, 97, 122, 17, 15, 114, 110, 3, 5, 125, 
   125, 99, 126, 119, 102, 30, 122, 2, 117}, 1422}, 
{{80, 101, 2, 116, 113, 102, 118, 25, 99, 8, 19, 23, 116, 
   125, 107, 25, 99, 109, 114, 102, 14, 121, 115, 31, 9, 117, 113, 
   111, 5, 4, 127, 28, 122, 101, 8, 110, 14, 18, 124, 106, 16, 20, 
   104, 119, 8, 109, 26, 106, 9, 97, 13, 99, 15, 119, 20, 105, 117, 
   98, 103, 118, 1, 126, 29, 97, 122, 17, 15, 114, 110, 3, 5, 125, 
   125, 99, 126, 119, 102, 30, 122, 2, 117}, 1012},
{{80, 120, 2, 116, 113, 102, 118, 25, 99, 8, 19, 23, 116, 
   125, 107, 25, 99, 109, 114, 102, 14, 121, 115, 31, 9, 117, 113, 
   111, 5, 4, 127, 28, 122, 101, 8, 110, 14, 18, 124, 106, 16, 20, 
   104, 119, 8, 109, 26, 106, 9, 97, 13, 99, 15, 119, 20, 105, 117, 
   98, 103, 118, 1, 126, 29, 97, 122, 17, 15, 114, 110, 3, 5, 125, 
   125, 99, 126, 119, 102, 30, 122, 2, 117}, 377},

...

{{80, 2, 74, 49, 113, 21, 62, 88, 39, 71, 68, 23, 63, 51, 36, 78, 48, 
   108, 114, 102, 14, 121, 115, 31, 9, 117, 113, 111, 5, 4, 127, 28, 
   122, 101, 8, 110, 14, 18, 124, 106, 16, 20, 104, 119, 8, 109, 26, 
   106, 9, 97, 13, 99, 15, 119, 20, 105, 117, 98, 103, 118, 1, 126, 
   29, 97, 122, 17, 15, 114, 110, 3, 5, 125, 125, 99, 126, 119, 102, 
   30, 122, 2, 117}, 1},
{{80, 1, 74, 59, 113, 45, 56, 86, 52, 91, 19, 64, 60, 60, 63, 
   25, 38, 59, 59, 42, 14, 53, 38, 77, 66, 38, 113, 38, 75, 4, 43, 84,
    63, 101, 64, 43, 79, 64, 40, 57, 16, 91, 46, 119, 69, 40, 84, 117,
    9, 97, 13, 99, 15, 119, 20, 105, 117, 98, 103, 118, 1, 126, 29, 
   97, 122, 17, 15, 114, 110, 3, 5, 125, 125, 99, 126, 119, 102, 30, 
   122, 2, 117}, 1},
{{80, 2, 74, 49, 113, 49, 51, 92, 39, 8, 92, 81, 116, 62, 57, 
   80, 46, 40, 114, 36, 75, 56, 33, 76, 9, 55, 56, 59, 81, 65, 45, 28,
    60, 55, 93, 39, 90, 28, 124, 106, 16, 20, 104, 119, 8, 109, 26, 
   106, 9, 97, 13, 99, 15, 119, 20, 105, 117, 98, 103, 118, 1, 126, 
   29, 97, 122, 17, 15, 114, 110, 3, 5, 125, 125, 99, 126, 119, 102, 
   30, 122, 2, 117}, 1}}
\end{lstlisting}

Вывод Tally это список пар, каждая пара это 81-байтный блок и количество раз, сколько он встретился в файле.
Мы видим, что наиболее частно встречающийся блок это первый, он встретился 1739 раз.
Второй встретился 1422 раза. Есть и другие: 1012 раза, 377 раз, итд.
81-байтные блоки, встреченные лишь один раз, находятся в конце вывода.

Попробуем сравнить эти блоки. Первый и второй.
Есть ли в Mathematica ф-ция для сравнения списков/массивов?
Наверняка есть, но в педагогических целях, я буду использоват операцию XOR для сравнения.
Действительно: если байты во входных массивах равны друг другу, результат операции XOR это 0.
Если не равны, результат будет ненулевой.

Сравним первый блок (встречается 1739 раз) и второй (встречается 1422 раз):

\begin{lstlisting}[style=custommath]
In[]:= BitXor[stat[[1]][[1]], stat[[2]][[1]]]
Out[]= {0, 3, 0, 0, 0, 0, 0, 0, 0, 0, 0, 0, 0, 0, 0, 0, 0, 0, 0, \
0, 0, 0, 0, 0, 0, 0, 0, 0, 0, 0, 0, 0, 0, 0, 0, 0, 0, 0, 0, 0, 0, 0, \
0, 0, 0, 0, 0, 0, 0, 0, 0, 0, 0, 0, 0, 0, 0, 0, 0, 0, 0, 0, 0, 0, 0, \
0, 0, 0, 0, 0, 0, 0, 0, 0, 0, 0, 0, 0, 0, 0, 0}
\end{lstlisting}

Они отличаются только вторым байтом.

Сравним второй блок (встречается 1422 раза) и третий (встречается 1012 раз):

\begin{lstlisting}[style=custommath]
In[]:= BitXor[stat[[2]][[1]], stat[[3]][[1]]]
Out[]= {0, 1, 0, 0, 0, 0, 0, 0, 0, 0, 0, 0, 0, 0, 0, 0, 0, 0, 0, \
0, 0, 0, 0, 0, 0, 0, 0, 0, 0, 0, 0, 0, 0, 0, 0, 0, 0, 0, 0, 0, 0, 0, \
0, 0, 0, 0, 0, 0, 0, 0, 0, 0, 0, 0, 0, 0, 0, 0, 0, 0, 0, 0, 0, 0, 0, \
0, 0, 0, 0, 0, 0, 0, 0, 0, 0, 0, 0, 0, 0, 0, 0}
\end{lstlisting}

Они тоже отличаются только вторым байтом.

Так или иначе, попробуем использовать самый встречающийся блок как XOR-ключ и попробуем расшифровать первые 4 81-байтных
блока в файле:

\begin{lstlisting}[style=custommath]
In[]:= key = stat[[1]][[1]]
Out[]= {80, 103, 2, 116, 113, 102, 118, 25, 99, 8, 19, 23, 116, \
125, 107, 25, 99, 109, 114, 102, 14, 121, 115, 31, 9, 117, 113, 111, \
5, 4, 127, 28, 122, 101, 8, 110, 14, 18, 124, 106, 16, 20, 104, 119, \
8, 109, 26, 106, 9, 97, 13, 99, 15, 119, 20, 105, 117, 98, 103, 118, \
1, 126, 29, 97, 122, 17, 15, 114, 110, 3, 5, 125, 125, 99, 126, 119, \
102, 30, 122, 2, 117}

In[]:= ToASCII[val_] := If[val == 0, " ", FromCharacterCode[val, "PrintableASCII"]]

In[]:= DecryptBlockASCII[blk_] := Map[ToASCII[#] &, BitXor[key, blk]]

In[]:= DecryptBlockASCII[blocks[[1]]]
Out[]= {" ", " ", " ", " ", " ", " ", " ", " ", " ", " ", " ", " \
", " ", " ", " ", " ", " ", " ", " ", " ", " ", " ", " ", " ", " ", " \
", " ", " ", " ", " ", " ", " ", " ", " ", " ", " ", " ", " ", " ", " \
", " ", " ", " ", " ", " ", " ", " ", " ", " ", " ", " ", " ", " ", " \
", " ", " ", " ", " ", " ", " ", " ", " ", " ", " ", " ", " ", " ", " \
", " ", " ", " ", " ", " ", " ", " ", " ", " ", " ", " ", " ", " "}

In[]:= DecryptBlockASCII[blocks[[2]]]
Out[]= {" ", "e", "H", "E", " ", "W", "E", "E", "D", " ", "O", \
"F", " ", "C", "R", "I", "M", "E", " ", "B", "E", "A", "R", "S", " ", \
"B", "I", "T", "T", "E", "R", " ", "F", "R", "U", "I", "T", "?", \
" ", " ", " ", " ", " ", " ", " ", " ", " ", " ", " ", " ", " ", " ", \
" ", " ", " ", " ", " ", " ", " ", " ", " ", " ", " ", " ", " ", " ", \
" ", " ", " ", " ", " ", " ", " ", " ", " ", " ", " ", " ", " ", " ", \
" "}

In[]:= DecryptBlockASCII[blocks[[3]]]
Out[]= {" ", "?", " ", " ", " ", " ", " ", " ", " ", " ", " \
", " ", " ", " ", " ", " ", " ", " ", " ", " ", " ", " ", " ", " ", " \
", " ", " ", " ", " ", " ", " ", " ", " ", " ", " ", " ", " ", " ", " \
", " ", " ", " ", " ", " ", " ", " ", " ", " ", " ", " ", " ", " ", " \
", " ", " ", " ", " ", " ", " ", " ", " ", " ", " ", " ", " ", " ", " \
", " ", " ", " ", " ", " ", " ", " ", " ", " ", " ", " ", " ", " ", " \
"}

In[]:= DecryptBlockASCII[blocks[[4]]]
Out[]= {" ", "f", "H", "O", " ", "K", "N", "O", "W", "S", " ", \
"W", "H", "A", "T", " ", "E", "V", "I", "L", " ", "L", "U", "R", "K", \
"S", " ", "I", "N", " ", "T", "H", "E", " ", "H", "E", "A", "R", "T", \
"S", " ", "O", "F", " ", "M", "E", "N", "?", " ", " ", " ", " ", \
" ", " ", " ", " ", " ", " ", " ", " ", " ", " ", " ", " ", " ", " ", \
" ", " ", " ", " ", " ", " ", " ", " ", " ", " ", " ", " ", " ", " ", \
" "}
\end{lstlisting}

(Я заменил непечатаемые символы на \q{?}.)

Мы видим что первый и третий блоки пустые (или почти пустые),
но второй и четвертый имеют ясно различимые английские слова/фразы.
Похоже что наше предположение насчет ключа верно (как минимум частично).
Это означает, что самый встречающийся 81-байтный блок в файле находится в местах лакун с нулевыми байтами
или что-то в этом роде.

Попробуем расшифровать весь файл:

\begin{lstlisting}[style=custommath]
DecryptBlock[blk_] := BitXor[key, blk]

decrypted = Map[DecryptBlock[#] &, blocks];

BinaryWrite["/home/dennis/.../tmp", Flatten[decrypted]]

Close["/home/dennis/.../tmp"]
\end{lstlisting}

\begin{figure}[H]
\centering
\myincludegraphics{ff/XOR/mask_1/mc_decrypted1.png}
\caption{Расшифрованный файл в Midnight Commander, первая попытка}
\end{figure}

Выглядит как английские фразы для какой-то игры, но что-то не так.
Прежде всего, регистр инвертирован: фразы и некоторые слова начинаются со строчных букв,
в то время как остальные буквы заглавные.
Также, некоторые фразы начинаются с не тех букв.
Посмотрите на самую первую фразу: \q{eHE WEED OF CRIME BEARS BITTER FRUIT}.
Что такое \q{eHE}? Разве не \q{tHE} тут должно быть?
Возможно ли что наш ключ для дешифрования имеет неверный байт в этом месте?

Посмотрим снова на второй блок в файле, на ключ и на результат дешифрования:

\begin{lstlisting}[style=custommath]
In[]:= blocks[[2]]
Out[]= {80, 2, 74, 49, 113, 49, 51, 92, 39, 8, 92, 81, 116, 62, \
57, 80, 46, 40, 114, 36, 75, 56, 33, 76, 9, 55, 56, 59, 81, 65, 45, \
28, 60, 55, 93, 39, 90, 28, 124, 106, 16, 20, 104, 119, 8, 109, 26, \
106, 9, 97, 13, 99, 15, 119, 20, 105, 117, 98, 103, 118, 1, 126, 29, \
97, 122, 17, 15, 114, 110, 3, 5, 125, 125, 99, 126, 119, 102, 30, \
122, 2, 117}

In[]:= key
Out[]= {80, 103, 2, 116, 113, 102, 118, 25, 99, 8, 19, 23, 116, \
125, 107, 25, 99, 109, 114, 102, 14, 121, 115, 31, 9, 117, 113, 111, \
5, 4, 127, 28, 122, 101, 8, 110, 14, 18, 124, 106, 16, 20, 104, 119, \
8, 109, 26, 106, 9, 97, 13, 99, 15, 119, 20, 105, 117, 98, 103, 118, \
1, 126, 29, 97, 122, 17, 15, 114, 110, 3, 5, 125, 125, 99, 126, 119, \
102, 30, 122, 2, 117}

In[]:= BitXor[key, blocks[[2]]]
Out[]= {0, 101, 72, 69, 0, 87, 69, 69, 68, 0, 79, 70, 0, 67, 82, \
73, 77, 69, 0, 66, 69, 65, 82, 83, 0, 66, 73, 84, 84, 69, 82, 0, 70, \
82, 85, 73, 84, 14, 0, 0, 0, 0, 0, 0, 0, 0, 0, 0, 0, 0, 0, 0, 0, 0, \
0, 0, 0, 0, 0, 0, 0, 0, 0, 0, 0, 0, 0, 0, 0, 0, 0, 0, 0, 0, 0, 0, 0, \
0, 0, 0, 0}
\end{lstlisting}

Зашифрованный байт это 2, байт из ключа это 103, $2 \oplus 103=101$ и 101 это ASCII-код символа \q{e}.
Чему должен равнятся этот байт ключа, чтобы ASCII-код был 116 (для символа  \q{t})?
$2 \oplus 116=118$, присвоим 118 второму байту в ключе \dots

\begin{lstlisting}[style=custommath]
key = {80, 118, 2, 116, 113, 102, 118, 25, 99, 8, 19, 23, 116, 125, 
  107, 25, 99, 109, 114, 102, 14, 121, 115, 31, 9, 117, 113, 111, 5, 
  4, 127, 28, 122, 101, 8, 110, 14, 18, 124, 106, 16, 20, 104, 119, 8,
   109, 26, 106, 9, 97, 13, 99, 15, 119, 20, 105, 117, 98, 103, 118, 
  1, 126, 29, 97, 122, 17, 15, 114, 110, 3, 5, 125, 125, 99, 126, 119,
   102, 30, 122, 2, 117}
\end{lstlisting}

\dots и снова дешифруем весь файл.

\begin{figure}[H]
\centering
\myincludegraphics{ff/XOR/mask_1/mc_decrypted2.png}
\caption{Дешифрованный файл в Midnight Commander, вторая попытка}
\end{figure}

Ух ты, теперь грамматика корректна, и все фразы начинаются с корректных букв.
Но все таки, регистр подозрителен.
С чего бы разработчику игры записывать их в такой манере?
Может быть наш ключ все еще неправилен?

% TODO ASCII table somewhere in the book
Изучая таблицу ASCII мы можем заметить что ASCII-коды для букв в верхнем и нижнем регистре отличаются только на один бит
(6-й бит, если считать с первого, 0b100000):

\begin{figure}[H]
\centering
\includegraphics[width=0.7\textwidth]{ascii.png}
\caption{7-битная таблица \ac{ASCII} в Emacs}
\end{figure}

6-й бит, выставленный в нулевом байте, В десятичном виде это будет 32.
Но 32 это ASCII-код пробела!

Действительно, можно менять регистр просто применяя XOR к ASCII-коду, с 32 (больше об этом: \myref{toupper_bit}).

Возможно ли, что пустые лакуны в файле это не нулевые байты, а скорее содержащие пробелы?
Еще раз модифицируем наш XOR-ключ (я про-XOR-ю каждый байт ключа с 32):

\begin{lstlisting}[style=custommath]
(* "32" это скаляр, и "key" это вектор, но это OK *)

In[]:= key3 = BitXor[32, key]
Out[]= {112, 86, 34, 84, 81, 70, 86, 57, 67, 40, 51, 55, 84, 93, 75, \
57, 67, 77, 82, 70, 46, 89, 83, 63, 41, 85, 81, 79, 37, 36, 95, 60, \
90, 69, 40, 78, 46, 50, 92, 74, 48, 52, 72, 87, 40, 77, 58, 74, 41, \
65, 45, 67, 47, 87, 52, 73, 85, 66, 71, 86, 33, 94, 61, 65, 90, 49, \
47, 82, 78, 35, 37, 93, 93, 67, 94, 87, 70, 62, 90, 34, 85}

In[]:= DecryptBlock[blk_] := BitXor[key3, blk]
\end{lstlisting}

И снова дешифруем входной файл:

\begin{figure}[H]
\centering
\myincludegraphics{ff/XOR/mask_1/mc_decrypted.png}
\caption{Дешифрованный файл в Midnight Commander, последняя попытка}
\end{figure}

(Расшифрованный файл доступен здесь:
\url{\GitHubBlobMasterURL/ff/XOR/mask_1/files/decrypted.dat.bz2}.)

Несомненно, это корректный исходный файл.
Да, и мы видим числа в начале каждого блока. Должно быть это и есть источник некорректного XOR-ключа.
Как выходит, самый встречающийся 81-байтный блок в файле это блок заполненный пробелами и содержащий символ \q{1} на месте
второго байта.
Действительно, как-то так получилось что многие блоки здесь перемежаются с этим блоком.
Может быть это что-то вроде выравнивания (padding) для коротких фраз/сообщений?
Другой часто встречающийся 81-байтный блок также заполнен пробелами, но с другой цифрой, следовательно,
они отличаются только вторым байтом.

Вот и всё! Теперь мы можем написать утилиту для зашифрования файла назад, и, может быть, модифицировать его перед этим

Файл для Mathematica можно скачать здесь:\\
\url{\GitHubBlobMasterURL/ff/XOR/mask_1/files/XOR_mask_1.nb}.

Итог: XOR-шифрование не надежно вообще. Вероятно, разработчик игры хотел просто скрыть внутренности игры от игрока,
ничего более серьезного.
Все же, шифрование вроде этого крайне популярно вследствии его простоты, так что многие реверс инженеры обычно хорошо
с этим знакомы.

}

\EN{% TODO translate
\mysection{Breaking simple executable cryptor}

I've got an executable file which is encrypted by relatively simple encryption.
\href{\GitHubBlobMasterURL/examples/simple_exec_crypto/files/cipher.bin}{Here is it} (only executable section is left here).

First, all encryption function does is just adds number of position in buffer to the byte.
Here is how this can be encoded in Python:

\begin{lstlisting}[caption=Python script,style=custompy]
#!/usr/bin/env python
def e(i, k):
    return chr ((ord(i)+k) % 256)

def encrypt(buf):
    return e(buf[0], 0)+ e(buf[1], 1)+ e(buf[2], 2) + e(buf[3], 3)+ e(buf[4], 4)+ e(buf[5], 5)+ e(buf[6], 6)+ e(buf[7], 7)+
           e(buf[8], 8)+ e(buf[9], 9)+ e(buf[10], 10)+ e(buf[11], 11)+ e(buf[12], 12)+ e(buf[13], 13)+ e(buf[14], 14)+ e(buf[15], 15)
\end{lstlisting}

Hence, if you encrypt buffer with 16 zeros, you'll get \emph{0, 1, 2, 3 ... 12, 13, 14, 15}.

\myindex{Propagating Cipher Block Chaining}
Propagating Cipher Block Chaining (PCBC) is also used, here is how it works:

\begin{figure}[H]
\centering
\myincludegraphics{examples/simple_exec_crypto/601px-PCBC_encryption.png}
\caption{Propagating Cipher Block Chaining encryption (image is taken from Wikipedia article)}
\end{figure}

The problem is that it's too boring to recover IV (Initialization Vector) each time.
Brute-force is also not an option, because IV is too long (16 bytes).
Let's see, if it's possible to recover IV for arbitrary encrypted executable file?

Let's try simple frequency analysis.
This is 32-bit x86 executable code, so let's gather statistics about most frequent bytes and opcodes.
I tried huge oracle.exe file from Oracle RDBMS version 11.2 for windows x86 and I've found that the most frequent byte (no surprise) is zero (~10\%).
The next most frequent byte is (again, no surprise) 0xFF (~5\%).
The next is 0x8B (~5\%).

\myindex{x86!\Instructions!MOV}
0x8B is opcode for \INS{MOV}, this is indeed one of the most busy x86 instructions.
Now what about popularity of zero byte?
If compiler needs to encode value bigger than 127, it has to use 32-bit displacement instead of 8-bit one, but large values are very rare,
so it is padded by zeros.
\myindex{x86!\Instructions!LEA}
\myindex{x86!\Instructions!PUSH}
\myindex{x86!\Instructions!CALL}
This is at least in \INS{LEA}, \INS{MOV}, \INS{PUSH}, \INS{CALL}.

For example:

\begin{lstlisting}[style=customasmx86]
8D B0 28 01 00 00                 lea     esi, [eax+128h]
8D BF 40 38 00 00                 lea     edi, [edi+3840h]
\end{lstlisting}

Displacements bigger than 127 are very popular, but they are rarely exceeds 0x10000
(indeed, such large memory buffers/structures are also rare).

Same story with \INS{MOV}, large constants are rare, the most heavily used are 0, 1, 10, 100, $2^n$, and so on.
Compiler has to pad small constants by zeros to represent them as 32-bit values:

\begin{lstlisting}[style=customasmx86]
BF 02 00 00 00                    mov     edi, 2
BF 01 00 00 00                    mov     edi, 1
\end{lstlisting}

Now about 00 and FF bytes combined: jumps (including conditional) and calls can pass execution flow forward or backwards, but very often,
within the limits of the current executable module.
If forward, displacement is not very big and also padded with zeros.
If backwards, displacement is represented as negative value, so padded with FF bytes.
For example, transfer execution flow forward:

\begin{lstlisting}[style=customasmx86]
E8 43 0C 00 00                    call    _function1
E8 5C 00 00 00                    call    _function2
0F 84 F0 0A 00 00                 jz      loc_4F09A0
0F 84 EB 00 00 00                 jz      loc_4EFBB8
\end{lstlisting}

Backwards:

\begin{lstlisting}[style=customasmx86]
E8 79 0C FE FF                    call    _function1
E8 F4 16 FF FF                    call    _function2
0F 84 F8 FB FF FF                 jz      loc_8212BC
0F 84 06 FD FF FF                 jz      loc_FF1E7D
\end{lstlisting}

FF byte is also very often occurred in negative displacements like these:

\begin{lstlisting}[style=customasmx86]
8D 85 1E FF FF FF                 lea     eax, [ebp-0E2h]
8D 95 F8 5C FF FF                 lea     edx, [ebp-0A308h]
\end{lstlisting}

So far so good. Now we have to try various 16-byte keys, decrypt executable section and measure how often 00, FF and 8B bytes are occurred.
Let's also keep in sight how PCBC decryption works:

\begin{figure}[H]
\centering
\myincludegraphics{examples/simple_exec_crypto/640px-PCBC_decryption.png}
\caption{Propagating Cipher Block Chaining decryption (image is taken from Wikipedia article)}
\end{figure}

The good news is that we don't really have to decrypt whole piece of data, but only slice by slice, this is exactly how I did in my previous example: \myref{XOR_mask_2}.

Now I'm trying all possible bytes (0..255) for each byte in key and just pick the byte producing maximal amount of 00/FF/8B bytes in a decrypted slice:

\begin{lstlisting}[style=custompy]
#!/usr/bin/env python
import sys, hexdump, array, string, operator

KEY_LEN=16

def chunks(l, n):
    # split n by l-byte chunks
    # https://stackoverflow.com/q/312443
    n = max(1, n)
    return [l[i:i + n] for i in range(0, len(l), n)]

def read_file(fname):
    file=open(fname, mode='rb')
    content=file.read()
    file.close()
    return content

def decrypt_byte (c, key):
    return chr((ord(c)-key) % 256)

def XOR_PCBC_step (IV, buf, k):
    prev=IV
    rt=""
    for c in buf:
	new_c=decrypt_byte(c, k)
        plain=chr(ord(new_c)^ord(prev))
	prev=chr(ord(c)^ord(plain))
	rt=rt+plain
    return rt

each_Nth_byte=[""]*KEY_LEN

content=read_file(sys.argv[1])
# split input by 16-byte chunks:
all_chunks=chunks(content, KEY_LEN)
for c in all_chunks:
    for i in range(KEY_LEN):
        each_Nth_byte[i]=each_Nth_byte[i] + c[i]

# try each byte of key
for N in range(KEY_LEN):
    print "N=", N
    stat={}
    for i in range(256):
        tmp_key=chr(i)
	tmp=XOR_PCBC_step(tmp_key,each_Nth_byte[N], N)
        # count 0, FFs and 8Bs in decrypted buffer:
	important_bytes=tmp.count('\x00')+tmp.count('\xFF')+tmp.count('\x8B')
	stat[i]=important_bytes
    sorted_stat = sorted(stat.iteritems(), key=operator.itemgetter(1), reverse=True)
    print sorted_stat[0]
\end{lstlisting}

(Source code can be downloaded \href{\GitHubBlobMasterURL/examples/simple_exec_crypto/files/decrypt.py}{here}.)

I run it and here is a key for which 00/FF/8B bytes presence in decrypted buffer is maximal:

\begin{lstlisting}
N= 0
(147, 1224)
N= 1
(94, 1327)
N= 2
(252, 1223)
N= 3
(218, 1266)
N= 4
(38, 1209)
N= 5
(192, 1378)
N= 6
(199, 1204)
N= 7
(213, 1332)
N= 8
(225, 1251)
N= 9
(112, 1223)
N= 10
(143, 1177)
N= 11
(108, 1286)
N= 12
(10, 1164)
N= 13
(3, 1271)
N= 14
(128, 1253)
N= 15
(232, 1330)
\end{lstlisting}

Let's write decryption utility with the key we got:

\begin{lstlisting}[style=custompy]
#!/usr/bin/env python
import sys, hexdump, array

def xor_strings(s,t):
    # \verb|https://en.wikipedia.org/wiki/XOR_cipher#Example_implementation|
    """xor two strings together"""
    return "".join(chr(ord(a)^ord(b)) for a,b in zip(s,t))

IV=array.array('B', [147, 94, 252, 218, 38, 192, 199, 213, 225, 112, 143, 108, 10, 3, 128, 232]).tostring()

def chunks(l, n):
    n = max(1, n)
    return [l[i:i + n] for i in range(0, len(l), n)]

def read_file(fname):
    file=open(fname, mode='rb')
    content=file.read()
    file.close()
    return content

def decrypt_byte(i, k):
    return chr ((ord(i)-k) % 256)

def decrypt(buf):
    return "".join(decrypt_byte(buf[i], i) for i in range(16))

fout=open(sys.argv[2], mode='wb')

prev=IV
content=read_file(sys.argv[1])
tmp=chunks(content, 16)
for c in tmp:
    new_c=decrypt(c)
    p=xor_strings (new_c, prev)
    prev=xor_strings(c, p)
    fout.write(p)
fout.close()
\end{lstlisting}

(Source code can be downloaded \href{\GitHubBlobMasterURL/examples/simple_exec_crypto/files/decrypt2.py}{here}.)

Let's check resulting file:

\lstinputlisting{examples/simple_exec_crypto/objdump_result.txt}

Yes, this is seems correctly disassembled piece of x86 code.
The whole decryped file can be downloaded \href{\GitHubBlobMasterURL/examples/simple_exec_crypto/files/decrypted.bin}{here}.

In fact, this is text section from regedit.exe from Windows 7.
But this example is based on a real case I encountered, so just executable is different (and key), algorithm is the same.

\subsection{Other ideas to consider}

What if I would fail with such simple frequency analysis?
There are other ideas on how to measure correctness of decrypted/decompressed x86 code:

\begin{itemize}

\item Many modern compilers aligns functions on 0x10 border.
So the space left before is filled with NOPs (0x90) or other NOP instructions with known opcodes: \myref{sec:npad}.

\item Perhaps, the most frequent pattern in any assembly language is function call:\\
\TT{PUSH chain / CALL / ADD ESP, X}.
This sequence can easily detected and found.
I've even gathered statistics about average number of function arguments: \myref{args_stat}.
(Hence, this is average length of PUSH chain.)

\end{itemize}

Read more about incorrectly/correctly disassembled code: \myref{ISA_detect}.
}
\RU{\subsection{Простое шифрование используя XOR-маску}
\label{XOR_mask_1}

Я нашел одну старую игру в стиле interactive fiction в архиве \emph{if-archive}\footnote{\url{http://www.ifarchive.org/}}:

\begin{lstlisting}
The New Castle v3.5 - Text/Adventure Game
in the style of the original Infocom (tm)
type games, Zork, Collosal Cave (Adventure),
etc.  Can you solve the mystery of the
abandoned castle?
Shareware from Software Customization.
Software Customization [ASP] Version 3.5 Feb. 2000
\end{lstlisting}

Можно скачать здесь: \url{\GitHubBlobMasterURL/ff/XOR/mask_1/files/newcastle.tgz}.

Там внутри есть файл (с названием \emph{castle.dbf}), который явно зашифрован, но не настоящим криптоалгоритмом,
и оне сжат, это что-то куда проще.
Я бы даже не стал измерять уровень энтропии (\myref{entropy}) этого файла, потому что я итак уверен, что он низкий.
Вот как он выглядит в Midnight Commander:

\begin{figure}[H]
\centering
\myincludegraphics{ff/XOR/mask_1/mc_encrypted.png}
\caption{Зашифрованный файл в Midnight Commander}
\end{figure}

Зашифрованный файл можно скачать здесь:
\url{\GitHubBlobMasterURL/ff/XOR/mask_1/files/castle.dbf.bz2}.

Можно ли расшифровать его без доступа к программе, используя просто этот файл?

Тут явно просматривается повторяющаяся строка. 
Если использовалось простое шифрование с XOR-маской, такие повторяющиеся строки это явное свидетельство,
потому что, вероятно, тут были длинные лакуны с нулевыми байтами, которые, в свою очередь, присутствуют
во мноигих исполняемых файлах, и в остальных бинарных файлах.

\myindex{UNIX!xxd}
Вот дам начала этого файла используя утилиту \emph{xxd} из UNIX:

\lstinputlisting{ff/XOR/mask_1/xxd_result.txt}

Давайте держаться за повторяющуюся строку \TT{iubgv}.
Глядя на этот дамп, мы можем легко увидеть, что период повторений этой строки это 0x51 или 81.
Вероятно, 81 это длина блока?
Длина файла 1658961, и она может быть поделена на 81 без остатка (и тогда там 20481 блоков).

Теперь я буду использовать Mathematica для анализа, есть ли тут повторяющиеся 81-байтные блоки в файле?
Я разделю входной файл на 81-байтные блоки и затем использую ф-цию
\emph{Tally[]}\footnote{\url{https://reference.wolfram.com/language/ref/Tally.html}}
которая просто считает, сколько раз каждый элемент встретился во входном списке.
Вывод Tally не отсортирован, так что я также добавлю ф-цию \emph{Sort[]} для сортировки его по кол-ву вхождений
в нисходящем порядке.

\begin{lstlisting}[style=custommath]
input = BinaryReadList["/home/dennis/.../castle.dbf"];

blocks = Partition[input, 81];

stat = Sort[Tally[blocks], #1[[2]] > #2[[2]] &]
\end{lstlisting}

И вот вывод:

\begin{lstlisting}[style=custommath]
{{{80, 103, 2, 116, 113, 102, 118, 25, 99, 8, 19, 23, 116, 125, 107, 
   25, 99, 109, 114, 102, 14, 121, 115, 31, 9, 117, 113, 111, 5, 4, 
   127, 28, 122, 101, 8, 110, 14, 18, 124, 106, 16, 20, 104, 119, 8, 
   109, 26, 106, 9, 97, 13, 99, 15, 119, 20, 105, 117, 98, 103, 118, 
   1, 126, 29, 97, 122, 17, 15, 114, 110, 3, 5, 125, 125, 99, 126, 
   119, 102, 30, 122, 2, 117}, 1739}, 
{{80, 100, 2, 116, 113, 102, 118, 25, 99, 8, 19, 23, 116, 
   125, 107, 25, 99, 109, 114, 102, 14, 121, 115, 31, 9, 117, 113, 
   111, 5, 4, 127, 28, 122, 101, 8, 110, 14, 18, 124, 106, 16, 20, 
   104, 119, 8, 109, 26, 106, 9, 97, 13, 99, 15, 119, 20, 105, 117, 
   98, 103, 118, 1, 126, 29, 97, 122, 17, 15, 114, 110, 3, 5, 125, 
   125, 99, 126, 119, 102, 30, 122, 2, 117}, 1422}, 
{{80, 101, 2, 116, 113, 102, 118, 25, 99, 8, 19, 23, 116, 
   125, 107, 25, 99, 109, 114, 102, 14, 121, 115, 31, 9, 117, 113, 
   111, 5, 4, 127, 28, 122, 101, 8, 110, 14, 18, 124, 106, 16, 20, 
   104, 119, 8, 109, 26, 106, 9, 97, 13, 99, 15, 119, 20, 105, 117, 
   98, 103, 118, 1, 126, 29, 97, 122, 17, 15, 114, 110, 3, 5, 125, 
   125, 99, 126, 119, 102, 30, 122, 2, 117}, 1012},
{{80, 120, 2, 116, 113, 102, 118, 25, 99, 8, 19, 23, 116, 
   125, 107, 25, 99, 109, 114, 102, 14, 121, 115, 31, 9, 117, 113, 
   111, 5, 4, 127, 28, 122, 101, 8, 110, 14, 18, 124, 106, 16, 20, 
   104, 119, 8, 109, 26, 106, 9, 97, 13, 99, 15, 119, 20, 105, 117, 
   98, 103, 118, 1, 126, 29, 97, 122, 17, 15, 114, 110, 3, 5, 125, 
   125, 99, 126, 119, 102, 30, 122, 2, 117}, 377},

...

{{80, 2, 74, 49, 113, 21, 62, 88, 39, 71, 68, 23, 63, 51, 36, 78, 48, 
   108, 114, 102, 14, 121, 115, 31, 9, 117, 113, 111, 5, 4, 127, 28, 
   122, 101, 8, 110, 14, 18, 124, 106, 16, 20, 104, 119, 8, 109, 26, 
   106, 9, 97, 13, 99, 15, 119, 20, 105, 117, 98, 103, 118, 1, 126, 
   29, 97, 122, 17, 15, 114, 110, 3, 5, 125, 125, 99, 126, 119, 102, 
   30, 122, 2, 117}, 1},
{{80, 1, 74, 59, 113, 45, 56, 86, 52, 91, 19, 64, 60, 60, 63, 
   25, 38, 59, 59, 42, 14, 53, 38, 77, 66, 38, 113, 38, 75, 4, 43, 84,
    63, 101, 64, 43, 79, 64, 40, 57, 16, 91, 46, 119, 69, 40, 84, 117,
    9, 97, 13, 99, 15, 119, 20, 105, 117, 98, 103, 118, 1, 126, 29, 
   97, 122, 17, 15, 114, 110, 3, 5, 125, 125, 99, 126, 119, 102, 30, 
   122, 2, 117}, 1},
{{80, 2, 74, 49, 113, 49, 51, 92, 39, 8, 92, 81, 116, 62, 57, 
   80, 46, 40, 114, 36, 75, 56, 33, 76, 9, 55, 56, 59, 81, 65, 45, 28,
    60, 55, 93, 39, 90, 28, 124, 106, 16, 20, 104, 119, 8, 109, 26, 
   106, 9, 97, 13, 99, 15, 119, 20, 105, 117, 98, 103, 118, 1, 126, 
   29, 97, 122, 17, 15, 114, 110, 3, 5, 125, 125, 99, 126, 119, 102, 
   30, 122, 2, 117}, 1}}
\end{lstlisting}

Вывод Tally это список пар, каждая пара это 81-байтный блок и количество раз, сколько он встретился в файле.
Мы видим, что наиболее частно встречающийся блок это первый, он встретился 1739 раз.
Второй встретился 1422 раза. Есть и другие: 1012 раза, 377 раз, итд.
81-байтные блоки, встреченные лишь один раз, находятся в конце вывода.

Попробуем сравнить эти блоки. Первый и второй.
Есть ли в Mathematica ф-ция для сравнения списков/массивов?
Наверняка есть, но в педагогических целях, я буду использоват операцию XOR для сравнения.
Действительно: если байты во входных массивах равны друг другу, результат операции XOR это 0.
Если не равны, результат будет ненулевой.

Сравним первый блок (встречается 1739 раз) и второй (встречается 1422 раз):

\begin{lstlisting}[style=custommath]
In[]:= BitXor[stat[[1]][[1]], stat[[2]][[1]]]
Out[]= {0, 3, 0, 0, 0, 0, 0, 0, 0, 0, 0, 0, 0, 0, 0, 0, 0, 0, 0, \
0, 0, 0, 0, 0, 0, 0, 0, 0, 0, 0, 0, 0, 0, 0, 0, 0, 0, 0, 0, 0, 0, 0, \
0, 0, 0, 0, 0, 0, 0, 0, 0, 0, 0, 0, 0, 0, 0, 0, 0, 0, 0, 0, 0, 0, 0, \
0, 0, 0, 0, 0, 0, 0, 0, 0, 0, 0, 0, 0, 0, 0, 0}
\end{lstlisting}

Они отличаются только вторым байтом.

Сравним второй блок (встречается 1422 раза) и третий (встречается 1012 раз):

\begin{lstlisting}[style=custommath]
In[]:= BitXor[stat[[2]][[1]], stat[[3]][[1]]]
Out[]= {0, 1, 0, 0, 0, 0, 0, 0, 0, 0, 0, 0, 0, 0, 0, 0, 0, 0, 0, \
0, 0, 0, 0, 0, 0, 0, 0, 0, 0, 0, 0, 0, 0, 0, 0, 0, 0, 0, 0, 0, 0, 0, \
0, 0, 0, 0, 0, 0, 0, 0, 0, 0, 0, 0, 0, 0, 0, 0, 0, 0, 0, 0, 0, 0, 0, \
0, 0, 0, 0, 0, 0, 0, 0, 0, 0, 0, 0, 0, 0, 0, 0}
\end{lstlisting}

Они тоже отличаются только вторым байтом.

Так или иначе, попробуем использовать самый встречающийся блок как XOR-ключ и попробуем расшифровать первые 4 81-байтных
блока в файле:

\begin{lstlisting}[style=custommath]
In[]:= key = stat[[1]][[1]]
Out[]= {80, 103, 2, 116, 113, 102, 118, 25, 99, 8, 19, 23, 116, \
125, 107, 25, 99, 109, 114, 102, 14, 121, 115, 31, 9, 117, 113, 111, \
5, 4, 127, 28, 122, 101, 8, 110, 14, 18, 124, 106, 16, 20, 104, 119, \
8, 109, 26, 106, 9, 97, 13, 99, 15, 119, 20, 105, 117, 98, 103, 118, \
1, 126, 29, 97, 122, 17, 15, 114, 110, 3, 5, 125, 125, 99, 126, 119, \
102, 30, 122, 2, 117}

In[]:= ToASCII[val_] := If[val == 0, " ", FromCharacterCode[val, "PrintableASCII"]]

In[]:= DecryptBlockASCII[blk_] := Map[ToASCII[#] &, BitXor[key, blk]]

In[]:= DecryptBlockASCII[blocks[[1]]]
Out[]= {" ", " ", " ", " ", " ", " ", " ", " ", " ", " ", " ", " \
", " ", " ", " ", " ", " ", " ", " ", " ", " ", " ", " ", " ", " ", " \
", " ", " ", " ", " ", " ", " ", " ", " ", " ", " ", " ", " ", " ", " \
", " ", " ", " ", " ", " ", " ", " ", " ", " ", " ", " ", " ", " ", " \
", " ", " ", " ", " ", " ", " ", " ", " ", " ", " ", " ", " ", " ", " \
", " ", " ", " ", " ", " ", " ", " ", " ", " ", " ", " ", " ", " "}

In[]:= DecryptBlockASCII[blocks[[2]]]
Out[]= {" ", "e", "H", "E", " ", "W", "E", "E", "D", " ", "O", \
"F", " ", "C", "R", "I", "M", "E", " ", "B", "E", "A", "R", "S", " ", \
"B", "I", "T", "T", "E", "R", " ", "F", "R", "U", "I", "T", "?", \
" ", " ", " ", " ", " ", " ", " ", " ", " ", " ", " ", " ", " ", " ", \
" ", " ", " ", " ", " ", " ", " ", " ", " ", " ", " ", " ", " ", " ", \
" ", " ", " ", " ", " ", " ", " ", " ", " ", " ", " ", " ", " ", " ", \
" "}

In[]:= DecryptBlockASCII[blocks[[3]]]
Out[]= {" ", "?", " ", " ", " ", " ", " ", " ", " ", " ", " \
", " ", " ", " ", " ", " ", " ", " ", " ", " ", " ", " ", " ", " ", " \
", " ", " ", " ", " ", " ", " ", " ", " ", " ", " ", " ", " ", " ", " \
", " ", " ", " ", " ", " ", " ", " ", " ", " ", " ", " ", " ", " ", " \
", " ", " ", " ", " ", " ", " ", " ", " ", " ", " ", " ", " ", " ", " \
", " ", " ", " ", " ", " ", " ", " ", " ", " ", " ", " ", " ", " ", " \
"}

In[]:= DecryptBlockASCII[blocks[[4]]]
Out[]= {" ", "f", "H", "O", " ", "K", "N", "O", "W", "S", " ", \
"W", "H", "A", "T", " ", "E", "V", "I", "L", " ", "L", "U", "R", "K", \
"S", " ", "I", "N", " ", "T", "H", "E", " ", "H", "E", "A", "R", "T", \
"S", " ", "O", "F", " ", "M", "E", "N", "?", " ", " ", " ", " ", \
" ", " ", " ", " ", " ", " ", " ", " ", " ", " ", " ", " ", " ", " ", \
" ", " ", " ", " ", " ", " ", " ", " ", " ", " ", " ", " ", " ", " ", \
" "}
\end{lstlisting}

(Я заменил непечатаемые символы на \q{?}.)

Мы видим что первый и третий блоки пустые (или почти пустые),
но второй и четвертый имеют ясно различимые английские слова/фразы.
Похоже что наше предположение насчет ключа верно (как минимум частично).
Это означает, что самый встречающийся 81-байтный блок в файле находится в местах лакун с нулевыми байтами
или что-то в этом роде.

Попробуем расшифровать весь файл:

\begin{lstlisting}[style=custommath]
DecryptBlock[blk_] := BitXor[key, blk]

decrypted = Map[DecryptBlock[#] &, blocks];

BinaryWrite["/home/dennis/.../tmp", Flatten[decrypted]]

Close["/home/dennis/.../tmp"]
\end{lstlisting}

\begin{figure}[H]
\centering
\myincludegraphics{ff/XOR/mask_1/mc_decrypted1.png}
\caption{Расшифрованный файл в Midnight Commander, первая попытка}
\end{figure}

Выглядит как английские фразы для какой-то игры, но что-то не так.
Прежде всего, регистр инвертирован: фразы и некоторые слова начинаются со строчных букв,
в то время как остальные буквы заглавные.
Также, некоторые фразы начинаются с не тех букв.
Посмотрите на самую первую фразу: \q{eHE WEED OF CRIME BEARS BITTER FRUIT}.
Что такое \q{eHE}? Разве не \q{tHE} тут должно быть?
Возможно ли что наш ключ для дешифрования имеет неверный байт в этом месте?

Посмотрим снова на второй блок в файле, на ключ и на результат дешифрования:

\begin{lstlisting}[style=custommath]
In[]:= blocks[[2]]
Out[]= {80, 2, 74, 49, 113, 49, 51, 92, 39, 8, 92, 81, 116, 62, \
57, 80, 46, 40, 114, 36, 75, 56, 33, 76, 9, 55, 56, 59, 81, 65, 45, \
28, 60, 55, 93, 39, 90, 28, 124, 106, 16, 20, 104, 119, 8, 109, 26, \
106, 9, 97, 13, 99, 15, 119, 20, 105, 117, 98, 103, 118, 1, 126, 29, \
97, 122, 17, 15, 114, 110, 3, 5, 125, 125, 99, 126, 119, 102, 30, \
122, 2, 117}

In[]:= key
Out[]= {80, 103, 2, 116, 113, 102, 118, 25, 99, 8, 19, 23, 116, \
125, 107, 25, 99, 109, 114, 102, 14, 121, 115, 31, 9, 117, 113, 111, \
5, 4, 127, 28, 122, 101, 8, 110, 14, 18, 124, 106, 16, 20, 104, 119, \
8, 109, 26, 106, 9, 97, 13, 99, 15, 119, 20, 105, 117, 98, 103, 118, \
1, 126, 29, 97, 122, 17, 15, 114, 110, 3, 5, 125, 125, 99, 126, 119, \
102, 30, 122, 2, 117}

In[]:= BitXor[key, blocks[[2]]]
Out[]= {0, 101, 72, 69, 0, 87, 69, 69, 68, 0, 79, 70, 0, 67, 82, \
73, 77, 69, 0, 66, 69, 65, 82, 83, 0, 66, 73, 84, 84, 69, 82, 0, 70, \
82, 85, 73, 84, 14, 0, 0, 0, 0, 0, 0, 0, 0, 0, 0, 0, 0, 0, 0, 0, 0, \
0, 0, 0, 0, 0, 0, 0, 0, 0, 0, 0, 0, 0, 0, 0, 0, 0, 0, 0, 0, 0, 0, 0, \
0, 0, 0, 0}
\end{lstlisting}

Зашифрованный байт это 2, байт из ключа это 103, $2 \oplus 103=101$ и 101 это ASCII-код символа \q{e}.
Чему должен равнятся этот байт ключа, чтобы ASCII-код был 116 (для символа  \q{t})?
$2 \oplus 116=118$, присвоим 118 второму байту в ключе \dots

\begin{lstlisting}[style=custommath]
key = {80, 118, 2, 116, 113, 102, 118, 25, 99, 8, 19, 23, 116, 125, 
  107, 25, 99, 109, 114, 102, 14, 121, 115, 31, 9, 117, 113, 111, 5, 
  4, 127, 28, 122, 101, 8, 110, 14, 18, 124, 106, 16, 20, 104, 119, 8,
   109, 26, 106, 9, 97, 13, 99, 15, 119, 20, 105, 117, 98, 103, 118, 
  1, 126, 29, 97, 122, 17, 15, 114, 110, 3, 5, 125, 125, 99, 126, 119,
   102, 30, 122, 2, 117}
\end{lstlisting}

\dots и снова дешифруем весь файл.

\begin{figure}[H]
\centering
\myincludegraphics{ff/XOR/mask_1/mc_decrypted2.png}
\caption{Дешифрованный файл в Midnight Commander, вторая попытка}
\end{figure}

Ух ты, теперь грамматика корректна, и все фразы начинаются с корректных букв.
Но все таки, регистр подозрителен.
С чего бы разработчику игры записывать их в такой манере?
Может быть наш ключ все еще неправилен?

% TODO ASCII table somewhere in the book
Изучая таблицу ASCII мы можем заметить что ASCII-коды для букв в верхнем и нижнем регистре отличаются только на один бит
(6-й бит, если считать с первого, 0b100000):

\begin{figure}[H]
\centering
\includegraphics[width=0.7\textwidth]{ascii.png}
\caption{7-битная таблица \ac{ASCII} в Emacs}
\end{figure}

6-й бит, выставленный в нулевом байте, В десятичном виде это будет 32.
Но 32 это ASCII-код пробела!

Действительно, можно менять регистр просто применяя XOR к ASCII-коду, с 32 (больше об этом: \myref{toupper_bit}).

Возможно ли, что пустые лакуны в файле это не нулевые байты, а скорее содержащие пробелы?
Еще раз модифицируем наш XOR-ключ (я про-XOR-ю каждый байт ключа с 32):

\begin{lstlisting}[style=custommath]
(* "32" это скаляр, и "key" это вектор, но это OK *)

In[]:= key3 = BitXor[32, key]
Out[]= {112, 86, 34, 84, 81, 70, 86, 57, 67, 40, 51, 55, 84, 93, 75, \
57, 67, 77, 82, 70, 46, 89, 83, 63, 41, 85, 81, 79, 37, 36, 95, 60, \
90, 69, 40, 78, 46, 50, 92, 74, 48, 52, 72, 87, 40, 77, 58, 74, 41, \
65, 45, 67, 47, 87, 52, 73, 85, 66, 71, 86, 33, 94, 61, 65, 90, 49, \
47, 82, 78, 35, 37, 93, 93, 67, 94, 87, 70, 62, 90, 34, 85}

In[]:= DecryptBlock[blk_] := BitXor[key3, blk]
\end{lstlisting}

И снова дешифруем входной файл:

\begin{figure}[H]
\centering
\myincludegraphics{ff/XOR/mask_1/mc_decrypted.png}
\caption{Дешифрованный файл в Midnight Commander, последняя попытка}
\end{figure}

(Расшифрованный файл доступен здесь:
\url{\GitHubBlobMasterURL/ff/XOR/mask_1/files/decrypted.dat.bz2}.)

Несомненно, это корректный исходный файл.
Да, и мы видим числа в начале каждого блока. Должно быть это и есть источник некорректного XOR-ключа.
Как выходит, самый встречающийся 81-байтный блок в файле это блок заполненный пробелами и содержащий символ \q{1} на месте
второго байта.
Действительно, как-то так получилось что многие блоки здесь перемежаются с этим блоком.
Может быть это что-то вроде выравнивания (padding) для коротких фраз/сообщений?
Другой часто встречающийся 81-байтный блок также заполнен пробелами, но с другой цифрой, следовательно,
они отличаются только вторым байтом.

Вот и всё! Теперь мы можем написать утилиту для зашифрования файла назад, и, может быть, модифицировать его перед этим

Файл для Mathematica можно скачать здесь:\\
\url{\GitHubBlobMasterURL/ff/XOR/mask_1/files/XOR_mask_1.nb}.

Итог: XOR-шифрование не надежно вообще. Вероятно, разработчик игры хотел просто скрыть внутренности игры от игрока,
ничего более серьезного.
Все же, шифрование вроде этого крайне популярно вследствии его простоты, так что многие реверс инженеры обычно хорошо
с этим знакомы.

}

\EN{\input{ff/exercise_EN}}
\RU{\input{ff/exercise_RU}}
\FR{\input{ff/exercise_FR}}

\mysection{\EN{Further reading}\RU{Дальнейшее чтение}\FR{Pour aller plus loin}\DEph{}}

\href{https://yurichev.com/mirrors/SSTIC2016-Article-cryptanalyse_en_boite_noire_de_chiffrement_proprietaire-capillon.pdf}{Pierre Capillon -- Black-box cryptanalysis of home-made encryption algorithms: a practical case study}.

\href{https://alexhude.github.io/2019/01/24/hacking-leica-m240.html}{How to Hack an Expensive Camera and Not Get Killed by Your Wife}.


\EN{% TODO translate
\mysection{Breaking simple executable cryptor}

I've got an executable file which is encrypted by relatively simple encryption.
\href{\GitHubBlobMasterURL/examples/simple_exec_crypto/files/cipher.bin}{Here is it} (only executable section is left here).

First, all encryption function does is just adds number of position in buffer to the byte.
Here is how this can be encoded in Python:

\begin{lstlisting}[caption=Python script,style=custompy]
#!/usr/bin/env python
def e(i, k):
    return chr ((ord(i)+k) % 256)

def encrypt(buf):
    return e(buf[0], 0)+ e(buf[1], 1)+ e(buf[2], 2) + e(buf[3], 3)+ e(buf[4], 4)+ e(buf[5], 5)+ e(buf[6], 6)+ e(buf[7], 7)+
           e(buf[8], 8)+ e(buf[9], 9)+ e(buf[10], 10)+ e(buf[11], 11)+ e(buf[12], 12)+ e(buf[13], 13)+ e(buf[14], 14)+ e(buf[15], 15)
\end{lstlisting}

Hence, if you encrypt buffer with 16 zeros, you'll get \emph{0, 1, 2, 3 ... 12, 13, 14, 15}.

\myindex{Propagating Cipher Block Chaining}
Propagating Cipher Block Chaining (PCBC) is also used, here is how it works:

\begin{figure}[H]
\centering
\myincludegraphics{examples/simple_exec_crypto/601px-PCBC_encryption.png}
\caption{Propagating Cipher Block Chaining encryption (image is taken from Wikipedia article)}
\end{figure}

The problem is that it's too boring to recover IV (Initialization Vector) each time.
Brute-force is also not an option, because IV is too long (16 bytes).
Let's see, if it's possible to recover IV for arbitrary encrypted executable file?

Let's try simple frequency analysis.
This is 32-bit x86 executable code, so let's gather statistics about most frequent bytes and opcodes.
I tried huge oracle.exe file from Oracle RDBMS version 11.2 for windows x86 and I've found that the most frequent byte (no surprise) is zero (~10\%).
The next most frequent byte is (again, no surprise) 0xFF (~5\%).
The next is 0x8B (~5\%).

\myindex{x86!\Instructions!MOV}
0x8B is opcode for \INS{MOV}, this is indeed one of the most busy x86 instructions.
Now what about popularity of zero byte?
If compiler needs to encode value bigger than 127, it has to use 32-bit displacement instead of 8-bit one, but large values are very rare,
so it is padded by zeros.
\myindex{x86!\Instructions!LEA}
\myindex{x86!\Instructions!PUSH}
\myindex{x86!\Instructions!CALL}
This is at least in \INS{LEA}, \INS{MOV}, \INS{PUSH}, \INS{CALL}.

For example:

\begin{lstlisting}[style=customasmx86]
8D B0 28 01 00 00                 lea     esi, [eax+128h]
8D BF 40 38 00 00                 lea     edi, [edi+3840h]
\end{lstlisting}

Displacements bigger than 127 are very popular, but they are rarely exceeds 0x10000
(indeed, such large memory buffers/structures are also rare).

Same story with \INS{MOV}, large constants are rare, the most heavily used are 0, 1, 10, 100, $2^n$, and so on.
Compiler has to pad small constants by zeros to represent them as 32-bit values:

\begin{lstlisting}[style=customasmx86]
BF 02 00 00 00                    mov     edi, 2
BF 01 00 00 00                    mov     edi, 1
\end{lstlisting}

Now about 00 and FF bytes combined: jumps (including conditional) and calls can pass execution flow forward or backwards, but very often,
within the limits of the current executable module.
If forward, displacement is not very big and also padded with zeros.
If backwards, displacement is represented as negative value, so padded with FF bytes.
For example, transfer execution flow forward:

\begin{lstlisting}[style=customasmx86]
E8 43 0C 00 00                    call    _function1
E8 5C 00 00 00                    call    _function2
0F 84 F0 0A 00 00                 jz      loc_4F09A0
0F 84 EB 00 00 00                 jz      loc_4EFBB8
\end{lstlisting}

Backwards:

\begin{lstlisting}[style=customasmx86]
E8 79 0C FE FF                    call    _function1
E8 F4 16 FF FF                    call    _function2
0F 84 F8 FB FF FF                 jz      loc_8212BC
0F 84 06 FD FF FF                 jz      loc_FF1E7D
\end{lstlisting}

FF byte is also very often occurred in negative displacements like these:

\begin{lstlisting}[style=customasmx86]
8D 85 1E FF FF FF                 lea     eax, [ebp-0E2h]
8D 95 F8 5C FF FF                 lea     edx, [ebp-0A308h]
\end{lstlisting}

So far so good. Now we have to try various 16-byte keys, decrypt executable section and measure how often 00, FF and 8B bytes are occurred.
Let's also keep in sight how PCBC decryption works:

\begin{figure}[H]
\centering
\myincludegraphics{examples/simple_exec_crypto/640px-PCBC_decryption.png}
\caption{Propagating Cipher Block Chaining decryption (image is taken from Wikipedia article)}
\end{figure}

The good news is that we don't really have to decrypt whole piece of data, but only slice by slice, this is exactly how I did in my previous example: \myref{XOR_mask_2}.

Now I'm trying all possible bytes (0..255) for each byte in key and just pick the byte producing maximal amount of 00/FF/8B bytes in a decrypted slice:

\begin{lstlisting}[style=custompy]
#!/usr/bin/env python
import sys, hexdump, array, string, operator

KEY_LEN=16

def chunks(l, n):
    # split n by l-byte chunks
    # https://stackoverflow.com/q/312443
    n = max(1, n)
    return [l[i:i + n] for i in range(0, len(l), n)]

def read_file(fname):
    file=open(fname, mode='rb')
    content=file.read()
    file.close()
    return content

def decrypt_byte (c, key):
    return chr((ord(c)-key) % 256)

def XOR_PCBC_step (IV, buf, k):
    prev=IV
    rt=""
    for c in buf:
	new_c=decrypt_byte(c, k)
        plain=chr(ord(new_c)^ord(prev))
	prev=chr(ord(c)^ord(plain))
	rt=rt+plain
    return rt

each_Nth_byte=[""]*KEY_LEN

content=read_file(sys.argv[1])
# split input by 16-byte chunks:
all_chunks=chunks(content, KEY_LEN)
for c in all_chunks:
    for i in range(KEY_LEN):
        each_Nth_byte[i]=each_Nth_byte[i] + c[i]

# try each byte of key
for N in range(KEY_LEN):
    print "N=", N
    stat={}
    for i in range(256):
        tmp_key=chr(i)
	tmp=XOR_PCBC_step(tmp_key,each_Nth_byte[N], N)
        # count 0, FFs and 8Bs in decrypted buffer:
	important_bytes=tmp.count('\x00')+tmp.count('\xFF')+tmp.count('\x8B')
	stat[i]=important_bytes
    sorted_stat = sorted(stat.iteritems(), key=operator.itemgetter(1), reverse=True)
    print sorted_stat[0]
\end{lstlisting}

(Source code can be downloaded \href{\GitHubBlobMasterURL/examples/simple_exec_crypto/files/decrypt.py}{here}.)

I run it and here is a key for which 00/FF/8B bytes presence in decrypted buffer is maximal:

\begin{lstlisting}
N= 0
(147, 1224)
N= 1
(94, 1327)
N= 2
(252, 1223)
N= 3
(218, 1266)
N= 4
(38, 1209)
N= 5
(192, 1378)
N= 6
(199, 1204)
N= 7
(213, 1332)
N= 8
(225, 1251)
N= 9
(112, 1223)
N= 10
(143, 1177)
N= 11
(108, 1286)
N= 12
(10, 1164)
N= 13
(3, 1271)
N= 14
(128, 1253)
N= 15
(232, 1330)
\end{lstlisting}

Let's write decryption utility with the key we got:

\begin{lstlisting}[style=custompy]
#!/usr/bin/env python
import sys, hexdump, array

def xor_strings(s,t):
    # \verb|https://en.wikipedia.org/wiki/XOR_cipher#Example_implementation|
    """xor two strings together"""
    return "".join(chr(ord(a)^ord(b)) for a,b in zip(s,t))

IV=array.array('B', [147, 94, 252, 218, 38, 192, 199, 213, 225, 112, 143, 108, 10, 3, 128, 232]).tostring()

def chunks(l, n):
    n = max(1, n)
    return [l[i:i + n] for i in range(0, len(l), n)]

def read_file(fname):
    file=open(fname, mode='rb')
    content=file.read()
    file.close()
    return content

def decrypt_byte(i, k):
    return chr ((ord(i)-k) % 256)

def decrypt(buf):
    return "".join(decrypt_byte(buf[i], i) for i in range(16))

fout=open(sys.argv[2], mode='wb')

prev=IV
content=read_file(sys.argv[1])
tmp=chunks(content, 16)
for c in tmp:
    new_c=decrypt(c)
    p=xor_strings (new_c, prev)
    prev=xor_strings(c, p)
    fout.write(p)
fout.close()
\end{lstlisting}

(Source code can be downloaded \href{\GitHubBlobMasterURL/examples/simple_exec_crypto/files/decrypt2.py}{here}.)

Let's check resulting file:

\lstinputlisting{examples/simple_exec_crypto/objdump_result.txt}

Yes, this is seems correctly disassembled piece of x86 code.
The whole decryped file can be downloaded \href{\GitHubBlobMasterURL/examples/simple_exec_crypto/files/decrypted.bin}{here}.

In fact, this is text section from regedit.exe from Windows 7.
But this example is based on a real case I encountered, so just executable is different (and key), algorithm is the same.

\subsection{Other ideas to consider}

What if I would fail with such simple frequency analysis?
There are other ideas on how to measure correctness of decrypted/decompressed x86 code:

\begin{itemize}

\item Many modern compilers aligns functions on 0x10 border.
So the space left before is filled with NOPs (0x90) or other NOP instructions with known opcodes: \myref{sec:npad}.

\item Perhaps, the most frequent pattern in any assembly language is function call:\\
\TT{PUSH chain / CALL / ADD ESP, X}.
This sequence can easily detected and found.
I've even gathered statistics about average number of function arguments: \myref{args_stat}.
(Hence, this is average length of PUSH chain.)

\end{itemize}

Read more about incorrectly/correctly disassembled code: \myref{ISA_detect}.
}%
\FR{\mysection{Une fonction vide: redux}

Revenons sur l'exemple de la fonction vide \myref{empty_func}.
Maintenant que nous connaissons le prologue et l'épilogue de fonction, ceci est
une fonction vide \myref{lst:empty_func} compilée par GCC sans optimisation:

\lstinputlisting[caption=GCC 8.2 x64 \NonOptimizing (\assemblyOutput),style=customasmx86]{patterns/016_empty_redux/1.s}

C'est \INS{RET}, mais le prologue et l'épilogue de la fonction, probablement, n'ont
pas été optimisés et laissés tels quels.
\INS{NOP} semble être un autre artefact du compilateur.
De toutes façons, la seule instruction effective ici est \INS{RET}.
Toutes les autres instructions peuvent être supprimées (ou optimisées).

}

% another place for `unsorted' sections

\chapter{\RU{Прочее}\EN{Other things}\DE{Weitere Themen}\FR{Autres sujets}\JA{その他}}

% sections:
\EN{\mysection{Executable files patching}
\label{patching}

\subsection{x86 code}
\label{x86_patching}

Frequent patching tasks are:

\myindex{x86!\Instructions!NOP}
\begin{itemize}

\item 
One of the most frequent jobs is to disable some instruction.
It is often done by filling it using byte 
\TT{0x90} (\ac{NOP}).

\item Conditional jumps, which have an opcode like \TT{74 xx} (\JZ), 
can be filled with two \ac{NOP}s.

It is also possible to disable a conditional jump by writing 0 at the second byte (\emph{jump offset}).

\myindex{x86!\Instructions!JMP}
\item 
Another frequent job is to make a conditional jump to always trigger: 
this can be done by writing \TT{0xEB} 
instead of the opcode, which stands for \JMP.

\myindex{x86!\Instructions!RET}
\myindex{stdcall}
\item A function's execution can be disabled by writing \RETN (0xC3) at its beginning.
This is true for all functions excluding \TT{stdcall} (\myref{sec:stdcall}).
While patching \TT{stdcall} functions, one has to determine the number of arguments (for example, 
by finding \RETN in this function), 
and use \RETN with a 16-bit argument (0xC2).

\myindex{x86!\Instructions!MOV}
\myindex{x86!\Instructions!XOR}
\myindex{x86!\Instructions!INC}
\item Sometimes, a disabled functions has to return 0 or 1.
This can be done by \TT{MOV EAX, 0} or \TT{MOV EAX, 1}, 
but it's slightly verbose.\\
A better way is \TT{XOR EAX, EAX} (2 bytes \TT{0x31 0xC0}) or \TT{XOR EAX, EAX / INC EAX} (3 bytes \TT{0x31 0xC0 0x40}).

\end{itemize}

A software may be protected against modifications.

This protection is often done by reading the executable code and calculating a checksum.
Therefore, 
the code must be read before protection is triggered.

This can be determined by setting a breakpoint on reading memory.

\myindex{tracer}
\tracer has the BPM option for this.

PE executable file relocs (\myref{subsec:relocs}) 
must not to be touched while patching, 
because the Windows loader may overwrite your new code.
\myindex{Hiew}
(They are grayed in Hiew, for example:
\figref{fig:scanf_ex3_hiew_1}).

As a last resort, it is possible to write jumps that circumvent the relocs, 
or you will have to edit the relocs table.

}
\RU{% FIXME секция ни к селу ни к месту!
\mysection{Модификация исполняемых файлов}
\label{patching}

\subsection{x86-код}
\label{x86_patching}

Часто необходимые задачи:

\myindex{x86!\Instructions!NOP}
\begin{itemize}

\item Часто нужно просто запретить исполнение какой-либо инструкции.
И чаще всего, это можно сделать, заполняя её байтом 
\TT{0x90} (\ac{NOP}).

\item Условные переходы, имеющие опкод вроде \TT{74 xx} (\JZ), 
так же могут быть заполнены двумя \ac{NOP}-ами.
Также возможно запретить исполнение условного перехода записав 0 во второй байт (\emph{jump offset}).

\myindex{x86!\Instructions!JMP}
\item Еще одна часто необходимая задача это сделать условный переход всегда срабатывающим: 
это возможно при помощи записи \TT{0xEB} 
вместо опкода, это значит \JMP.

\myindex{x86!\Instructions!RET}
\myindex{stdcall}
\item Исполнение функции может быть запрещено, если записать
\RETN (0xC3) в её начале.
Это справедливо для всех функций кроме \TT{stdcall} 
(\myref{sec:stdcall}).
При модификации функций \TT{stdcall}, нужно в начале определить количество аргументов 
(например, отыскав \RETN в этой функции),
и использовать \RETN с 16-битным аргументом (0xC2).

\myindex{x86!\Instructions!MOV}
\myindex{x86!\Instructions!XOR}
\myindex{x86!\Instructions!INC}
\item Иногда, запрещенная функция должна возвращать 0 или 1.
Это можно сделать при помощи \TT{MOV EAX, 0} или \TT{MOV EAX, 1}, 
но это слишком многословно.\\
Способ получше это \TT{XOR EAX, EAX} (2 байта \TT{0x31 0xC0}) или \TT{XOR EAX, EAX / INC EAX} (3 байта \TT{0x31 0xC0 0x40}).

\end{itemize}

ПО может быть защищено от модификаций.
Эта защита чаще всего реализуется путем чтения кода и вычисления контрольной суммы.
Следовательно, код должен быть прочитан перед тем как защита сработает.
Это можно определить установив точку останова на чтение памяти.

\myindex{tracer}
В \tracer имеется опция BPM для этого.

Релоки в исполняемых PE-файлах (\myref{subsec:relocs}) 
не должны быть тронуты, потому что загрузчик Windows перезапишет ваш новый код.

\myindex{Hiew}
(Они выделяются серым в Hiew, например: \figref{fig:scanf_ex3_hiew_1}).
В качестве последней меры, можно записать \JMP для обхода релока, либо же придется модифицировать таблицу
релоков.


}
\DE{\mysection{Patchen von ausführbaren Dateien}
\label{patching}

\subsection{x86-Code}
\label{x86_patching}

Häufige Aufgaben beim Patchen sind:

\myindex{x86!\Instructions!NOP}
\begin{itemize}

\item 
Eine der häufigsten Aufgaben ist das Deaktivieren bestimmter Anweisungen. Oft
wird dies durch Austauschen des Bytes durch \TT{0x90} (\ac{NOP}).

\item
Bedingte Sprünge, die den Opcode wie \TT{74 xx} (\JZ) haben, können durch
\ac{NOP}s ersetzt werden.

Es ist möglich alle bedingten Sprünge zu deaktivieren, in dem eine 0 in das
zweite Byte geschrieben wird (\emph{Sprung-Offset}).

\myindex{x86!\Instructions!JMP}
\item 
Eine weitere häufige Aufgabe ist es einen bedingten Sprung immer ausführen zu
lassen: dies kann durch Schreiben von \TT{0xEB}, was für \JMP steht, anstatt des
Opcodes erreicht werden.

\myindex{x86!\Instructions!RET}
\myindex{stdcall}
\item Die Ausführung einer Funktion kann deaktiviert werden, wenn \RETN (0xC3) an
den Anfang geschrieben wird. Dies gilt für alle Funktionen außer \TT{stdcall}
(\myref{sec:stdcall}).
Um \TT{stdcall}-Funktionen zu patchen muss die Anzahl der Argumente bekannt sein
(zum Beispiel durch Finden der \RETN-Anweisung in der Funktion) und die \RETN-Anweisung
mit einem 16-Bit-Argument (0xC2) angewendet werden.

\myindex{x86!\Instructions!MOV}
\myindex{x86!\Instructions!XOR}
\myindex{x86!\Instructions!INC}
\item Manchmal muss eine deaktivierte Funktion den Wert 0 oder 1 zurückgeben.
Dies kann durch \TT{MOV EAX, 0} oder \TT{MOV EAX, 1} erreicht werden, was aber relativ
ausführlich ist.
Ein besserer Weg ist \TT{XOR EAX, EAX} (2 Byte \TT{0x31 0xC0}) oder \TT{XOR EAX, EAX / INC EAX}
(3 Byte \TT{0x31 0xC0 0x40}).

\end{itemize}

Eine Software kann gegen Manipulation geschützt sein.

Dieser Schutz ist häufig realisiert indem der ausführbare Code gelesen und ein
passende Checksumme errechnet wird.
Aus diesem Grund muss der Code gelesen werden bevor die Schutzfunktion aktiviert
wird. Die Stelle kann durch setzen eines Breakpoints beim Lesen von Speicher
herausgefunden werden.

\myindex{tracer}
\tracer hat für diesen Zweck die BPM-Option.

Die Relocs (\myref{subsec:relocs}) in ausführbaren PE-Dateien sollten nicht
verändert werden, da der Windows-Laser den neuen, veränderten Code möglicherweise
überschreibt.
\myindex{Hiew}
(In Hiew sind die Stellen grau markiert, zum Beispiel: \figref{fig:scanf_ex3_hiew_1}).

Eine Möglichkeit ist es Sprünge zu schreiben, welche die Relocs umgehen oder die
Reloc-Tabelle muss editiert werden.
}
\FR{\mysection{Modification de fichier exécutable}
\label{patching}

\subsection{code x86}
\label{x86_patching}

Les tâches de modification courantes sont:

\myindex{x86!\Instructions!NOP}
\begin{itemize}

\item Une des tâches la plus fréquente est de désactiver une instruction en l'écrasant
avec des octets \TT{0x90} (\ac{NOP}).

\item Les branchements conditionnels qui utilisent un code instruction tel que \TT{74 xx} (\JZ),
peuvent être réécrits avec deux instructions \ac{NOP}.

Une autre technique consiste à désactiver un branchement conditionnel en écrasant le second octet
avec la valeur 0 (\emph{jump offset}).

\myindex{x86!\Instructions!JMP}
\item
Une autre tâche courante consiste à faire en sorte qu'un branchement conditionnel soit effectué
systématiquement. On y parvient en remplaçant le code instruction par \TT{0xEB} qui correspond à
l'instruction \JMP.

\myindex{x86!\Instructions!RET}
\myindex{stdcall}
\item L'exécution d'une fonction peut être désactivée en remplaçant le premier octet par \RETN (0xC3).
Les fonctions dont la convention d'appel est \TT{stdcall} (\myref{sec:stdcall}) font exception.
Pour les modifier, il faut déterminer le nombre d'arguments (par exemple en trouvant une instruction
\RETN au sein de la fonction), puis en utilisant l'instruction \RETN accompagnée d'un argument sur
deux octets (0xC2).

\myindex{x86!\Instructions!MOV}
\myindex{x86!\Instructions!XOR}
\myindex{x86!\Instructions!INC}
\item Il arrive qu'une fonction que l'on a désactivée doive retourner une valeur 0 ou 1. Certes on
peut utiliser \TT{MOV EAX, 0} ou \TT{MOV EAX, 1}, mais cela occupe un peu trop d'espace.\\
Une meilleure approche consiste à utiliser \TT{XOR EAX, EAX} (2 octets \TT{0x31 0xC0}) ou
\TT{XOR EAX, EAX / INC EAX} (3 octets \TT{0x31 0xC0 0x40}).

\end{itemize}

Un logiciel peut être protégé contre les modifications. Le plus souvent la protection consiste à
lire le code du programme (en mémoire) et à en calculer une valeur de contrôle.
Cette technique nécessite que la protection lise le code avant de pouvoir agir. Elle peut donc être
détectée en positionnant un point d'arrêt déclenché par la lecture de la mémoire contenant le code.

\myindex{tracer}
\tracer possède l'option BPM pour ce faire.

La partie du fichier au format PE qui contient les informations de relogement (\myref{subsec:relocs})
ne doivent pas être modifiées par les patchs car le chargeur Windows risquerait d'écraser les
modifications apportées.
\myindex{Hiew}
(Ces parties sont présentées sous forme grisées dans Hiew, par exemple:
\figref{fig:scanf_ex3_hiew_1}).

En dernier ressort, il est possible d'effectuer des modifications qui contournent les relogements,
ou de modifier directement la table des relogements.
}

\EN{\input{other/args_stat_EN}}
\RU{\input{other/args_stat_RU}}
\DE{\input{other/args_stat_DE}}
\FR{\input{other/args_stat_FR}}

\EN{\input{other/compiler_intrinsic_EN}}
\RU{\input{other/compiler_intrinsic_RU}}
\DE{\input{other/compiler_intrinsic_DE}}
\FR{\input{other/compiler_intrinsic_FR}}

\EN{\mysection{Compiler's anomalies}
\label{anomaly:Intel}
\myindex{\CompilerAnomaly}

\subsection{\oracle 11.2 and Intel C++ 10.1}

\myindex{Intel C++}
\myindex{\oracle}
\myindex{x86!\Instructions!JZ}

Intel C++ 10.1, which was used for \oracle 11.2 Linux86 compilation, may emit two \JZ in row,
and there are no references to the second \JZ. The second \JZ is thus meaningless.

\lstinputlisting[caption=kdli.o from libserver11.a,style=customasmx86]{other/kdli.lst}

\begin{lstlisting}[caption=from the same code,style=customasmx86]
.text:0811A2A5                   loc_811A2A5: ; CODE XREF: kdliSerLengths+11C
.text:0811A2A5                                ; kdliSerLengths+1C1
.text:0811A2A5 8B 7D 08              mov     edi, [ebp+arg_0]
.text:0811A2A8 8B 7F 10              mov     edi, [edi+10h]
.text:0811A2AB 0F B6 57 14           movzx   edx, byte ptr [edi+14h]
.text:0811A2AF F6 C2 01              test    dl, 1
.text:0811A2B2 75 3E                 jnz     short loc_811A2F2
.text:0811A2B4 83 E0 01              and     eax, 1
.text:0811A2B7 74 1F                 jz      short loc_811A2D8
.text:0811A2B9 74 37                 jz      short loc_811A2F2
.text:0811A2BB 6A 00                 push    0
.text:0811A2BD FF 71 08              push    dword ptr [ecx+8]
.text:0811A2C0 E8 5F FE FF FF        call    len2nbytes
\end{lstlisting}

It is supposedly a code generator bug that was not found by tests, because 
resulting code works correctly anyway.

Another example from \oracle 11.1.0.6.0 for win32.

\begin{lstlisting}
.text:0051FBF8 85 C0                             test    eax, eax
.text:0051FBFA 0F 84 8F 00 00 00                 jz      loc_51FC8F
.text:0051FC00 74 1D                             jz      short loc_51FC1F
\end{lstlisting}

\input{other/anomaly2_EN}
\input{other/anomaly3_EN}

\subsection{Summary}

Other compiler anomalies here in this book: 
\myref{anomaly:LLVM}, \myref{loops_iterators_loop_anomaly}, \myref{Keil_anomaly},
\myref{MSVC2013_anomaly},
\myref{MSVC_double_JMP_anomaly},
\myref{MSVC2012_anomaly}.

Such cases are demonstrated here in this book, to show that such compilers errors are possible and sometimes
one should not to rack one's brain while thinking why did the compiler generate such strange code.

}
\RU{\mysection{Аномалии компиляторов}
\label{anomaly:Intel}
\myindex{\CompilerAnomaly}

\subsection{\oracle 11.2 and Intel C++ 10.1}

\myindex{Intel C++}
\myindex{\oracle}
\myindex{x86!\Instructions!JZ}

Intel C++ 10.1 которым скомпилирован \oracle 11.2 Linux86, может сгенерировать два \JZ идущих подряд, 
причем на второй \JZ нет ссылки ниоткуда. Второй \JZ таким образом, не имеет никакого смысла.

\lstinputlisting[caption=kdli.o from libserver11.a,style=customasmx86]{other/kdli.lst}

\begin{lstlisting}[caption=оттуда же,style=customasmx86]
.text:0811A2A5                   loc_811A2A5: ; CODE XREF: kdliSerLengths+11C
.text:0811A2A5                                ; kdliSerLengths+1C1
.text:0811A2A5 8B 7D 08              mov     edi, [ebp+arg_0]
.text:0811A2A8 8B 7F 10              mov     edi, [edi+10h]
.text:0811A2AB 0F B6 57 14           movzx   edx, byte ptr [edi+14h]
.text:0811A2AF F6 C2 01              test    dl, 1
.text:0811A2B2 75 3E                 jnz     short loc_811A2F2
.text:0811A2B4 83 E0 01              and     eax, 1
.text:0811A2B7 74 1F                 jz      short loc_811A2D8
.text:0811A2B9 74 37                 jz      short loc_811A2F2
.text:0811A2BB 6A 00                 push    0
.text:0811A2BD FF 71 08              push    dword ptr [ecx+8]
.text:0811A2C0 E8 5F FE FF FF        call    len2nbytes
\end{lstlisting}

Возможно, это ошибка его кодегенератора, не выявленная тестами 
(ведь результирующий код и так работает нормально).

Еще пример из \oracle 11.1.0.6.0 для win32.

\begin{lstlisting}
.text:0051FBF8 85 C0                             test    eax, eax
.text:0051FBFA 0F 84 8F 00 00 00                 jz      loc_51FC8F
.text:0051FC00 74 1D                             jz      short loc_51FC1F
\end{lstlisting}

\input{other/anomaly2_RU}
%\input{other/anomaly3_RU}

\subsection{Итог}

Еще подобные аномалии компиляторов в этой книге: 
\myref{anomaly:LLVM}, \myref{loops_iterators_loop_anomaly}, \myref{Keil_anomaly},
\myref{MSVC2013_anomaly},
\myref{MSVC_double_JMP_anomaly},
\myref{MSVC2012_anomaly}.

В этой книге здесь приводятся подобные случаи для того, чтобы легче было понимать, 
что подобные ошибки компиляторов 
все же имеют место быть, и не следует ломать голову над тем, почему он сгенерировал такой странный код.

}
\DE{\mysection{Compiler Anomalien}
\label{anomaly:Intel}
\myindex{\CompilerAnomaly}

\subsection{\oracle 11.2 und Intel C++ 10.1}

\myindex{Intel C++}
\myindex{\oracle}
\myindex{x86!\Instructions!JZ}

Der Intel C++ 10.1-Compiler, der für \oracle 11.2 für Linux 86 genutzt wurde, kann
zwei \JZ in einer Reihe ausgeben. Es gibt keine Referenz zum zweiten \JZ. Das zweite
ist also ohne Bedeutung.

\lstinputlisting[caption=kdli.o from libserver11.a,style=customasmx86]{other/kdli.lst}

\begin{lstlisting}[caption=from the same code,style=customasmx86]
.text:0811A2A5                   loc_811A2A5: ; CODE XREF: kdliSerLengths+11C
.text:0811A2A5                                ; kdliSerLengths+1C1
.text:0811A2A5 8B 7D 08              mov     edi, [ebp+arg_0]
.text:0811A2A8 8B 7F 10              mov     edi, [edi+10h]
.text:0811A2AB 0F B6 57 14           movzx   edx, byte ptr [edi+14h]
.text:0811A2AF F6 C2 01              test    dl, 1
.text:0811A2B2 75 3E                 jnz     short loc_811A2F2
.text:0811A2B4 83 E0 01              and     eax, 1
.text:0811A2B7 74 1F                 jz      short loc_811A2D8
.text:0811A2B9 74 37                 jz      short loc_811A2F2
.text:0811A2BB 6A 00                 push    0
.text:0811A2BD FF 71 08              push    dword ptr [ecx+8]
.text:0811A2C0 E8 5F FE FF FF        call    len2nbytes
\end{lstlisting}

Dies ist vermutlich ein Fehler im Codegenerator der während der Tests nicht
gefunden wurde. Der resultierende Code funktioniert trotzdem.

% TBT
% Another example from \oracle 11.1.0.6.0 for win32.

%\begin{lstlisting}
%.text:0051FBF8 85 C0                             test    eax, eax
%.text:0051FBFA 0F 84 8F 00 00 00                 jz      loc_51FC8F
%.text:0051FC00 74 1D                             jz      short loc_51FC1F
%\end{lstlisting}

\input{other/anomaly2_DE}
%\input{other/anomaly3_DE}

\subsection{Zusammenfassung}

Andere Compiler-Anomalien in diesem Buch:
\myref{anomaly:LLVM}, \myref{loops_iterators_loop_anomaly}, \myref{Keil_anomaly},
\myref{MSVC2013_anomaly},
\myref{MSVC_double_JMP_anomaly},
\myref{MSVC2012_anomaly}.

Diese Beispiele werden in diesem Buch gezeigt, um zu verdeutlichen, das solche Fehler
in den Compilern möglich sind und es gelegentlich keinen Sinn ergibt sich den Kopf
darüber zu zerbrechen warum der Compiler diesen \q{seltsamen} Code erzeugte.
}
\FR{\mysection{Anomalies des compilateurs}
\label{anomaly:Intel}
\myindex{\CompilerAnomaly}

\subsection{\oracle 11.2 et Intel C++ 10.1}

\myindex{Intel C++}
\myindex{\oracle}
\myindex{x86!\Instructions!JZ}

Le compilateur Intel C++ 10.1, qui a été utilisé pour la compilation de \oracle
11.2 pour Linux86, émettait parfois deux instructions \JZ successives, sans que
la seconde instruction soit jamais référencée. Elle était donc inutile.

\lstinputlisting[caption=kdli.o from libserver11.a,style=customasmx86]{other/kdli.lst}

\begin{lstlisting}[caption=from the same code,style=customasmx86]
.text:0811A2A5                   loc_811A2A5: ; CODE XREF: kdliSerLengths+11C
.text:0811A2A5                                ; kdliSerLengths+1C1
.text:0811A2A5 8B 7D 08              mov     edi, [ebp+arg_0]
.text:0811A2A8 8B 7F 10              mov     edi, [edi+10h]
.text:0811A2AB 0F B6 57 14           movzx   edx, byte ptr [edi+14h]
.text:0811A2AF F6 C2 01              test    dl, 1
.text:0811A2B2 75 3E                 jnz     short loc_811A2F2
.text:0811A2B4 83 E0 01              and     eax, 1
.text:0811A2B7 74 1F                 jz      short loc_811A2D8
.text:0811A2B9 74 37                 jz      short loc_811A2F2
.text:0811A2BB 6A 00                 push    0
.text:0811A2BD FF 71 08              push    dword ptr [ecx+8]
.text:0811A2C0 E8 5F FE FF FF        call    len2nbytes
\end{lstlisting}

Il s'agit probablement d'un bug du générateur de code du compilateur qui ne fut pas
découvert durant les tests de celui-ci car le code produit fonctionnait conformément
aux résultats attendus.

Un autre exemple tiré d'\oracle 11.1.0.6.0 pour win32.

\begin{lstlisting}
.text:0051FBF8 85 C0                             test    eax, eax
.text:0051FBFA 0F 84 8F 00 00 00                 jz      loc_51FC8F
.text:0051FC00 74 1D                             jz      short loc_51FC1F
\end{lstlisting}

\input{other/anomaly2_FR}
\input{other/anomaly3_FR}

\subsection{Résumé}

Des anomalies constatées dans d'autres compilateurs figurent également dans ce livre: 
\myref{anomaly:LLVM}, \myref{loops_iterators_loop_anomaly}, \myref{Keil_anomaly},
\myref{MSVC2013_anomaly},
\myref{MSVC_double_JMP_anomaly},
\myref{MSVC2012_anomaly}.

Ces cas sont exposés dans ce livre afin de démontrer que ces compilateurs comportent
leurs propres erreurs et qu'il convient de ne pas toujours se torturer le cerveau
en tentant de comprendre pourquoi le compilateur a généré un code aussi étrange.

}

\EN{% TODO translate
\mysection{Breaking simple executable cryptor}

I've got an executable file which is encrypted by relatively simple encryption.
\href{\GitHubBlobMasterURL/examples/simple_exec_crypto/files/cipher.bin}{Here is it} (only executable section is left here).

First, all encryption function does is just adds number of position in buffer to the byte.
Here is how this can be encoded in Python:

\begin{lstlisting}[caption=Python script,style=custompy]
#!/usr/bin/env python
def e(i, k):
    return chr ((ord(i)+k) % 256)

def encrypt(buf):
    return e(buf[0], 0)+ e(buf[1], 1)+ e(buf[2], 2) + e(buf[3], 3)+ e(buf[4], 4)+ e(buf[5], 5)+ e(buf[6], 6)+ e(buf[7], 7)+
           e(buf[8], 8)+ e(buf[9], 9)+ e(buf[10], 10)+ e(buf[11], 11)+ e(buf[12], 12)+ e(buf[13], 13)+ e(buf[14], 14)+ e(buf[15], 15)
\end{lstlisting}

Hence, if you encrypt buffer with 16 zeros, you'll get \emph{0, 1, 2, 3 ... 12, 13, 14, 15}.

\myindex{Propagating Cipher Block Chaining}
Propagating Cipher Block Chaining (PCBC) is also used, here is how it works:

\begin{figure}[H]
\centering
\myincludegraphics{examples/simple_exec_crypto/601px-PCBC_encryption.png}
\caption{Propagating Cipher Block Chaining encryption (image is taken from Wikipedia article)}
\end{figure}

The problem is that it's too boring to recover IV (Initialization Vector) each time.
Brute-force is also not an option, because IV is too long (16 bytes).
Let's see, if it's possible to recover IV for arbitrary encrypted executable file?

Let's try simple frequency analysis.
This is 32-bit x86 executable code, so let's gather statistics about most frequent bytes and opcodes.
I tried huge oracle.exe file from Oracle RDBMS version 11.2 for windows x86 and I've found that the most frequent byte (no surprise) is zero (~10\%).
The next most frequent byte is (again, no surprise) 0xFF (~5\%).
The next is 0x8B (~5\%).

\myindex{x86!\Instructions!MOV}
0x8B is opcode for \INS{MOV}, this is indeed one of the most busy x86 instructions.
Now what about popularity of zero byte?
If compiler needs to encode value bigger than 127, it has to use 32-bit displacement instead of 8-bit one, but large values are very rare,
so it is padded by zeros.
\myindex{x86!\Instructions!LEA}
\myindex{x86!\Instructions!PUSH}
\myindex{x86!\Instructions!CALL}
This is at least in \INS{LEA}, \INS{MOV}, \INS{PUSH}, \INS{CALL}.

For example:

\begin{lstlisting}[style=customasmx86]
8D B0 28 01 00 00                 lea     esi, [eax+128h]
8D BF 40 38 00 00                 lea     edi, [edi+3840h]
\end{lstlisting}

Displacements bigger than 127 are very popular, but they are rarely exceeds 0x10000
(indeed, such large memory buffers/structures are also rare).

Same story with \INS{MOV}, large constants are rare, the most heavily used are 0, 1, 10, 100, $2^n$, and so on.
Compiler has to pad small constants by zeros to represent them as 32-bit values:

\begin{lstlisting}[style=customasmx86]
BF 02 00 00 00                    mov     edi, 2
BF 01 00 00 00                    mov     edi, 1
\end{lstlisting}

Now about 00 and FF bytes combined: jumps (including conditional) and calls can pass execution flow forward or backwards, but very often,
within the limits of the current executable module.
If forward, displacement is not very big and also padded with zeros.
If backwards, displacement is represented as negative value, so padded with FF bytes.
For example, transfer execution flow forward:

\begin{lstlisting}[style=customasmx86]
E8 43 0C 00 00                    call    _function1
E8 5C 00 00 00                    call    _function2
0F 84 F0 0A 00 00                 jz      loc_4F09A0
0F 84 EB 00 00 00                 jz      loc_4EFBB8
\end{lstlisting}

Backwards:

\begin{lstlisting}[style=customasmx86]
E8 79 0C FE FF                    call    _function1
E8 F4 16 FF FF                    call    _function2
0F 84 F8 FB FF FF                 jz      loc_8212BC
0F 84 06 FD FF FF                 jz      loc_FF1E7D
\end{lstlisting}

FF byte is also very often occurred in negative displacements like these:

\begin{lstlisting}[style=customasmx86]
8D 85 1E FF FF FF                 lea     eax, [ebp-0E2h]
8D 95 F8 5C FF FF                 lea     edx, [ebp-0A308h]
\end{lstlisting}

So far so good. Now we have to try various 16-byte keys, decrypt executable section and measure how often 00, FF and 8B bytes are occurred.
Let's also keep in sight how PCBC decryption works:

\begin{figure}[H]
\centering
\myincludegraphics{examples/simple_exec_crypto/640px-PCBC_decryption.png}
\caption{Propagating Cipher Block Chaining decryption (image is taken from Wikipedia article)}
\end{figure}

The good news is that we don't really have to decrypt whole piece of data, but only slice by slice, this is exactly how I did in my previous example: \myref{XOR_mask_2}.

Now I'm trying all possible bytes (0..255) for each byte in key and just pick the byte producing maximal amount of 00/FF/8B bytes in a decrypted slice:

\begin{lstlisting}[style=custompy]
#!/usr/bin/env python
import sys, hexdump, array, string, operator

KEY_LEN=16

def chunks(l, n):
    # split n by l-byte chunks
    # https://stackoverflow.com/q/312443
    n = max(1, n)
    return [l[i:i + n] for i in range(0, len(l), n)]

def read_file(fname):
    file=open(fname, mode='rb')
    content=file.read()
    file.close()
    return content

def decrypt_byte (c, key):
    return chr((ord(c)-key) % 256)

def XOR_PCBC_step (IV, buf, k):
    prev=IV
    rt=""
    for c in buf:
	new_c=decrypt_byte(c, k)
        plain=chr(ord(new_c)^ord(prev))
	prev=chr(ord(c)^ord(plain))
	rt=rt+plain
    return rt

each_Nth_byte=[""]*KEY_LEN

content=read_file(sys.argv[1])
# split input by 16-byte chunks:
all_chunks=chunks(content, KEY_LEN)
for c in all_chunks:
    for i in range(KEY_LEN):
        each_Nth_byte[i]=each_Nth_byte[i] + c[i]

# try each byte of key
for N in range(KEY_LEN):
    print "N=", N
    stat={}
    for i in range(256):
        tmp_key=chr(i)
	tmp=XOR_PCBC_step(tmp_key,each_Nth_byte[N], N)
        # count 0, FFs and 8Bs in decrypted buffer:
	important_bytes=tmp.count('\x00')+tmp.count('\xFF')+tmp.count('\x8B')
	stat[i]=important_bytes
    sorted_stat = sorted(stat.iteritems(), key=operator.itemgetter(1), reverse=True)
    print sorted_stat[0]
\end{lstlisting}

(Source code can be downloaded \href{\GitHubBlobMasterURL/examples/simple_exec_crypto/files/decrypt.py}{here}.)

I run it and here is a key for which 00/FF/8B bytes presence in decrypted buffer is maximal:

\begin{lstlisting}
N= 0
(147, 1224)
N= 1
(94, 1327)
N= 2
(252, 1223)
N= 3
(218, 1266)
N= 4
(38, 1209)
N= 5
(192, 1378)
N= 6
(199, 1204)
N= 7
(213, 1332)
N= 8
(225, 1251)
N= 9
(112, 1223)
N= 10
(143, 1177)
N= 11
(108, 1286)
N= 12
(10, 1164)
N= 13
(3, 1271)
N= 14
(128, 1253)
N= 15
(232, 1330)
\end{lstlisting}

Let's write decryption utility with the key we got:

\begin{lstlisting}[style=custompy]
#!/usr/bin/env python
import sys, hexdump, array

def xor_strings(s,t):
    # \verb|https://en.wikipedia.org/wiki/XOR_cipher#Example_implementation|
    """xor two strings together"""
    return "".join(chr(ord(a)^ord(b)) for a,b in zip(s,t))

IV=array.array('B', [147, 94, 252, 218, 38, 192, 199, 213, 225, 112, 143, 108, 10, 3, 128, 232]).tostring()

def chunks(l, n):
    n = max(1, n)
    return [l[i:i + n] for i in range(0, len(l), n)]

def read_file(fname):
    file=open(fname, mode='rb')
    content=file.read()
    file.close()
    return content

def decrypt_byte(i, k):
    return chr ((ord(i)-k) % 256)

def decrypt(buf):
    return "".join(decrypt_byte(buf[i], i) for i in range(16))

fout=open(sys.argv[2], mode='wb')

prev=IV
content=read_file(sys.argv[1])
tmp=chunks(content, 16)
for c in tmp:
    new_c=decrypt(c)
    p=xor_strings (new_c, prev)
    prev=xor_strings(c, p)
    fout.write(p)
fout.close()
\end{lstlisting}

(Source code can be downloaded \href{\GitHubBlobMasterURL/examples/simple_exec_crypto/files/decrypt2.py}{here}.)

Let's check resulting file:

\lstinputlisting{examples/simple_exec_crypto/objdump_result.txt}

Yes, this is seems correctly disassembled piece of x86 code.
The whole decryped file can be downloaded \href{\GitHubBlobMasterURL/examples/simple_exec_crypto/files/decrypted.bin}{here}.

In fact, this is text section from regedit.exe from Windows 7.
But this example is based on a real case I encountered, so just executable is different (and key), algorithm is the same.

\subsection{Other ideas to consider}

What if I would fail with such simple frequency analysis?
There are other ideas on how to measure correctness of decrypted/decompressed x86 code:

\begin{itemize}

\item Many modern compilers aligns functions on 0x10 border.
So the space left before is filled with NOPs (0x90) or other NOP instructions with known opcodes: \myref{sec:npad}.

\item Perhaps, the most frequent pattern in any assembly language is function call:\\
\TT{PUSH chain / CALL / ADD ESP, X}.
This sequence can easily detected and found.
I've even gathered statistics about average number of function arguments: \myref{args_stat}.
(Hence, this is average length of PUSH chain.)

\end{itemize}

Read more about incorrectly/correctly disassembled code: \myref{ISA_detect}.
}
\RU{\subsection{Простое шифрование используя XOR-маску}
\label{XOR_mask_1}

Я нашел одну старую игру в стиле interactive fiction в архиве \emph{if-archive}\footnote{\url{http://www.ifarchive.org/}}:

\begin{lstlisting}
The New Castle v3.5 - Text/Adventure Game
in the style of the original Infocom (tm)
type games, Zork, Collosal Cave (Adventure),
etc.  Can you solve the mystery of the
abandoned castle?
Shareware from Software Customization.
Software Customization [ASP] Version 3.5 Feb. 2000
\end{lstlisting}

Можно скачать здесь: \url{\GitHubBlobMasterURL/ff/XOR/mask_1/files/newcastle.tgz}.

Там внутри есть файл (с названием \emph{castle.dbf}), который явно зашифрован, но не настоящим криптоалгоритмом,
и оне сжат, это что-то куда проще.
Я бы даже не стал измерять уровень энтропии (\myref{entropy}) этого файла, потому что я итак уверен, что он низкий.
Вот как он выглядит в Midnight Commander:

\begin{figure}[H]
\centering
\myincludegraphics{ff/XOR/mask_1/mc_encrypted.png}
\caption{Зашифрованный файл в Midnight Commander}
\end{figure}

Зашифрованный файл можно скачать здесь:
\url{\GitHubBlobMasterURL/ff/XOR/mask_1/files/castle.dbf.bz2}.

Можно ли расшифровать его без доступа к программе, используя просто этот файл?

Тут явно просматривается повторяющаяся строка. 
Если использовалось простое шифрование с XOR-маской, такие повторяющиеся строки это явное свидетельство,
потому что, вероятно, тут были длинные лакуны с нулевыми байтами, которые, в свою очередь, присутствуют
во мноигих исполняемых файлах, и в остальных бинарных файлах.

\myindex{UNIX!xxd}
Вот дам начала этого файла используя утилиту \emph{xxd} из UNIX:

\lstinputlisting{ff/XOR/mask_1/xxd_result.txt}

Давайте держаться за повторяющуюся строку \TT{iubgv}.
Глядя на этот дамп, мы можем легко увидеть, что период повторений этой строки это 0x51 или 81.
Вероятно, 81 это длина блока?
Длина файла 1658961, и она может быть поделена на 81 без остатка (и тогда там 20481 блоков).

Теперь я буду использовать Mathematica для анализа, есть ли тут повторяющиеся 81-байтные блоки в файле?
Я разделю входной файл на 81-байтные блоки и затем использую ф-цию
\emph{Tally[]}\footnote{\url{https://reference.wolfram.com/language/ref/Tally.html}}
которая просто считает, сколько раз каждый элемент встретился во входном списке.
Вывод Tally не отсортирован, так что я также добавлю ф-цию \emph{Sort[]} для сортировки его по кол-ву вхождений
в нисходящем порядке.

\begin{lstlisting}[style=custommath]
input = BinaryReadList["/home/dennis/.../castle.dbf"];

blocks = Partition[input, 81];

stat = Sort[Tally[blocks], #1[[2]] > #2[[2]] &]
\end{lstlisting}

И вот вывод:

\begin{lstlisting}[style=custommath]
{{{80, 103, 2, 116, 113, 102, 118, 25, 99, 8, 19, 23, 116, 125, 107, 
   25, 99, 109, 114, 102, 14, 121, 115, 31, 9, 117, 113, 111, 5, 4, 
   127, 28, 122, 101, 8, 110, 14, 18, 124, 106, 16, 20, 104, 119, 8, 
   109, 26, 106, 9, 97, 13, 99, 15, 119, 20, 105, 117, 98, 103, 118, 
   1, 126, 29, 97, 122, 17, 15, 114, 110, 3, 5, 125, 125, 99, 126, 
   119, 102, 30, 122, 2, 117}, 1739}, 
{{80, 100, 2, 116, 113, 102, 118, 25, 99, 8, 19, 23, 116, 
   125, 107, 25, 99, 109, 114, 102, 14, 121, 115, 31, 9, 117, 113, 
   111, 5, 4, 127, 28, 122, 101, 8, 110, 14, 18, 124, 106, 16, 20, 
   104, 119, 8, 109, 26, 106, 9, 97, 13, 99, 15, 119, 20, 105, 117, 
   98, 103, 118, 1, 126, 29, 97, 122, 17, 15, 114, 110, 3, 5, 125, 
   125, 99, 126, 119, 102, 30, 122, 2, 117}, 1422}, 
{{80, 101, 2, 116, 113, 102, 118, 25, 99, 8, 19, 23, 116, 
   125, 107, 25, 99, 109, 114, 102, 14, 121, 115, 31, 9, 117, 113, 
   111, 5, 4, 127, 28, 122, 101, 8, 110, 14, 18, 124, 106, 16, 20, 
   104, 119, 8, 109, 26, 106, 9, 97, 13, 99, 15, 119, 20, 105, 117, 
   98, 103, 118, 1, 126, 29, 97, 122, 17, 15, 114, 110, 3, 5, 125, 
   125, 99, 126, 119, 102, 30, 122, 2, 117}, 1012},
{{80, 120, 2, 116, 113, 102, 118, 25, 99, 8, 19, 23, 116, 
   125, 107, 25, 99, 109, 114, 102, 14, 121, 115, 31, 9, 117, 113, 
   111, 5, 4, 127, 28, 122, 101, 8, 110, 14, 18, 124, 106, 16, 20, 
   104, 119, 8, 109, 26, 106, 9, 97, 13, 99, 15, 119, 20, 105, 117, 
   98, 103, 118, 1, 126, 29, 97, 122, 17, 15, 114, 110, 3, 5, 125, 
   125, 99, 126, 119, 102, 30, 122, 2, 117}, 377},

...

{{80, 2, 74, 49, 113, 21, 62, 88, 39, 71, 68, 23, 63, 51, 36, 78, 48, 
   108, 114, 102, 14, 121, 115, 31, 9, 117, 113, 111, 5, 4, 127, 28, 
   122, 101, 8, 110, 14, 18, 124, 106, 16, 20, 104, 119, 8, 109, 26, 
   106, 9, 97, 13, 99, 15, 119, 20, 105, 117, 98, 103, 118, 1, 126, 
   29, 97, 122, 17, 15, 114, 110, 3, 5, 125, 125, 99, 126, 119, 102, 
   30, 122, 2, 117}, 1},
{{80, 1, 74, 59, 113, 45, 56, 86, 52, 91, 19, 64, 60, 60, 63, 
   25, 38, 59, 59, 42, 14, 53, 38, 77, 66, 38, 113, 38, 75, 4, 43, 84,
    63, 101, 64, 43, 79, 64, 40, 57, 16, 91, 46, 119, 69, 40, 84, 117,
    9, 97, 13, 99, 15, 119, 20, 105, 117, 98, 103, 118, 1, 126, 29, 
   97, 122, 17, 15, 114, 110, 3, 5, 125, 125, 99, 126, 119, 102, 30, 
   122, 2, 117}, 1},
{{80, 2, 74, 49, 113, 49, 51, 92, 39, 8, 92, 81, 116, 62, 57, 
   80, 46, 40, 114, 36, 75, 56, 33, 76, 9, 55, 56, 59, 81, 65, 45, 28,
    60, 55, 93, 39, 90, 28, 124, 106, 16, 20, 104, 119, 8, 109, 26, 
   106, 9, 97, 13, 99, 15, 119, 20, 105, 117, 98, 103, 118, 1, 126, 
   29, 97, 122, 17, 15, 114, 110, 3, 5, 125, 125, 99, 126, 119, 102, 
   30, 122, 2, 117}, 1}}
\end{lstlisting}

Вывод Tally это список пар, каждая пара это 81-байтный блок и количество раз, сколько он встретился в файле.
Мы видим, что наиболее частно встречающийся блок это первый, он встретился 1739 раз.
Второй встретился 1422 раза. Есть и другие: 1012 раза, 377 раз, итд.
81-байтные блоки, встреченные лишь один раз, находятся в конце вывода.

Попробуем сравнить эти блоки. Первый и второй.
Есть ли в Mathematica ф-ция для сравнения списков/массивов?
Наверняка есть, но в педагогических целях, я буду использоват операцию XOR для сравнения.
Действительно: если байты во входных массивах равны друг другу, результат операции XOR это 0.
Если не равны, результат будет ненулевой.

Сравним первый блок (встречается 1739 раз) и второй (встречается 1422 раз):

\begin{lstlisting}[style=custommath]
In[]:= BitXor[stat[[1]][[1]], stat[[2]][[1]]]
Out[]= {0, 3, 0, 0, 0, 0, 0, 0, 0, 0, 0, 0, 0, 0, 0, 0, 0, 0, 0, \
0, 0, 0, 0, 0, 0, 0, 0, 0, 0, 0, 0, 0, 0, 0, 0, 0, 0, 0, 0, 0, 0, 0, \
0, 0, 0, 0, 0, 0, 0, 0, 0, 0, 0, 0, 0, 0, 0, 0, 0, 0, 0, 0, 0, 0, 0, \
0, 0, 0, 0, 0, 0, 0, 0, 0, 0, 0, 0, 0, 0, 0, 0}
\end{lstlisting}

Они отличаются только вторым байтом.

Сравним второй блок (встречается 1422 раза) и третий (встречается 1012 раз):

\begin{lstlisting}[style=custommath]
In[]:= BitXor[stat[[2]][[1]], stat[[3]][[1]]]
Out[]= {0, 1, 0, 0, 0, 0, 0, 0, 0, 0, 0, 0, 0, 0, 0, 0, 0, 0, 0, \
0, 0, 0, 0, 0, 0, 0, 0, 0, 0, 0, 0, 0, 0, 0, 0, 0, 0, 0, 0, 0, 0, 0, \
0, 0, 0, 0, 0, 0, 0, 0, 0, 0, 0, 0, 0, 0, 0, 0, 0, 0, 0, 0, 0, 0, 0, \
0, 0, 0, 0, 0, 0, 0, 0, 0, 0, 0, 0, 0, 0, 0, 0}
\end{lstlisting}

Они тоже отличаются только вторым байтом.

Так или иначе, попробуем использовать самый встречающийся блок как XOR-ключ и попробуем расшифровать первые 4 81-байтных
блока в файле:

\begin{lstlisting}[style=custommath]
In[]:= key = stat[[1]][[1]]
Out[]= {80, 103, 2, 116, 113, 102, 118, 25, 99, 8, 19, 23, 116, \
125, 107, 25, 99, 109, 114, 102, 14, 121, 115, 31, 9, 117, 113, 111, \
5, 4, 127, 28, 122, 101, 8, 110, 14, 18, 124, 106, 16, 20, 104, 119, \
8, 109, 26, 106, 9, 97, 13, 99, 15, 119, 20, 105, 117, 98, 103, 118, \
1, 126, 29, 97, 122, 17, 15, 114, 110, 3, 5, 125, 125, 99, 126, 119, \
102, 30, 122, 2, 117}

In[]:= ToASCII[val_] := If[val == 0, " ", FromCharacterCode[val, "PrintableASCII"]]

In[]:= DecryptBlockASCII[blk_] := Map[ToASCII[#] &, BitXor[key, blk]]

In[]:= DecryptBlockASCII[blocks[[1]]]
Out[]= {" ", " ", " ", " ", " ", " ", " ", " ", " ", " ", " ", " \
", " ", " ", " ", " ", " ", " ", " ", " ", " ", " ", " ", " ", " ", " \
", " ", " ", " ", " ", " ", " ", " ", " ", " ", " ", " ", " ", " ", " \
", " ", " ", " ", " ", " ", " ", " ", " ", " ", " ", " ", " ", " ", " \
", " ", " ", " ", " ", " ", " ", " ", " ", " ", " ", " ", " ", " ", " \
", " ", " ", " ", " ", " ", " ", " ", " ", " ", " ", " ", " ", " "}

In[]:= DecryptBlockASCII[blocks[[2]]]
Out[]= {" ", "e", "H", "E", " ", "W", "E", "E", "D", " ", "O", \
"F", " ", "C", "R", "I", "M", "E", " ", "B", "E", "A", "R", "S", " ", \
"B", "I", "T", "T", "E", "R", " ", "F", "R", "U", "I", "T", "?", \
" ", " ", " ", " ", " ", " ", " ", " ", " ", " ", " ", " ", " ", " ", \
" ", " ", " ", " ", " ", " ", " ", " ", " ", " ", " ", " ", " ", " ", \
" ", " ", " ", " ", " ", " ", " ", " ", " ", " ", " ", " ", " ", " ", \
" "}

In[]:= DecryptBlockASCII[blocks[[3]]]
Out[]= {" ", "?", " ", " ", " ", " ", " ", " ", " ", " ", " \
", " ", " ", " ", " ", " ", " ", " ", " ", " ", " ", " ", " ", " ", " \
", " ", " ", " ", " ", " ", " ", " ", " ", " ", " ", " ", " ", " ", " \
", " ", " ", " ", " ", " ", " ", " ", " ", " ", " ", " ", " ", " ", " \
", " ", " ", " ", " ", " ", " ", " ", " ", " ", " ", " ", " ", " ", " \
", " ", " ", " ", " ", " ", " ", " ", " ", " ", " ", " ", " ", " ", " \
"}

In[]:= DecryptBlockASCII[blocks[[4]]]
Out[]= {" ", "f", "H", "O", " ", "K", "N", "O", "W", "S", " ", \
"W", "H", "A", "T", " ", "E", "V", "I", "L", " ", "L", "U", "R", "K", \
"S", " ", "I", "N", " ", "T", "H", "E", " ", "H", "E", "A", "R", "T", \
"S", " ", "O", "F", " ", "M", "E", "N", "?", " ", " ", " ", " ", \
" ", " ", " ", " ", " ", " ", " ", " ", " ", " ", " ", " ", " ", " ", \
" ", " ", " ", " ", " ", " ", " ", " ", " ", " ", " ", " ", " ", " ", \
" "}
\end{lstlisting}

(Я заменил непечатаемые символы на \q{?}.)

Мы видим что первый и третий блоки пустые (или почти пустые),
но второй и четвертый имеют ясно различимые английские слова/фразы.
Похоже что наше предположение насчет ключа верно (как минимум частично).
Это означает, что самый встречающийся 81-байтный блок в файле находится в местах лакун с нулевыми байтами
или что-то в этом роде.

Попробуем расшифровать весь файл:

\begin{lstlisting}[style=custommath]
DecryptBlock[blk_] := BitXor[key, blk]

decrypted = Map[DecryptBlock[#] &, blocks];

BinaryWrite["/home/dennis/.../tmp", Flatten[decrypted]]

Close["/home/dennis/.../tmp"]
\end{lstlisting}

\begin{figure}[H]
\centering
\myincludegraphics{ff/XOR/mask_1/mc_decrypted1.png}
\caption{Расшифрованный файл в Midnight Commander, первая попытка}
\end{figure}

Выглядит как английские фразы для какой-то игры, но что-то не так.
Прежде всего, регистр инвертирован: фразы и некоторые слова начинаются со строчных букв,
в то время как остальные буквы заглавные.
Также, некоторые фразы начинаются с не тех букв.
Посмотрите на самую первую фразу: \q{eHE WEED OF CRIME BEARS BITTER FRUIT}.
Что такое \q{eHE}? Разве не \q{tHE} тут должно быть?
Возможно ли что наш ключ для дешифрования имеет неверный байт в этом месте?

Посмотрим снова на второй блок в файле, на ключ и на результат дешифрования:

\begin{lstlisting}[style=custommath]
In[]:= blocks[[2]]
Out[]= {80, 2, 74, 49, 113, 49, 51, 92, 39, 8, 92, 81, 116, 62, \
57, 80, 46, 40, 114, 36, 75, 56, 33, 76, 9, 55, 56, 59, 81, 65, 45, \
28, 60, 55, 93, 39, 90, 28, 124, 106, 16, 20, 104, 119, 8, 109, 26, \
106, 9, 97, 13, 99, 15, 119, 20, 105, 117, 98, 103, 118, 1, 126, 29, \
97, 122, 17, 15, 114, 110, 3, 5, 125, 125, 99, 126, 119, 102, 30, \
122, 2, 117}

In[]:= key
Out[]= {80, 103, 2, 116, 113, 102, 118, 25, 99, 8, 19, 23, 116, \
125, 107, 25, 99, 109, 114, 102, 14, 121, 115, 31, 9, 117, 113, 111, \
5, 4, 127, 28, 122, 101, 8, 110, 14, 18, 124, 106, 16, 20, 104, 119, \
8, 109, 26, 106, 9, 97, 13, 99, 15, 119, 20, 105, 117, 98, 103, 118, \
1, 126, 29, 97, 122, 17, 15, 114, 110, 3, 5, 125, 125, 99, 126, 119, \
102, 30, 122, 2, 117}

In[]:= BitXor[key, blocks[[2]]]
Out[]= {0, 101, 72, 69, 0, 87, 69, 69, 68, 0, 79, 70, 0, 67, 82, \
73, 77, 69, 0, 66, 69, 65, 82, 83, 0, 66, 73, 84, 84, 69, 82, 0, 70, \
82, 85, 73, 84, 14, 0, 0, 0, 0, 0, 0, 0, 0, 0, 0, 0, 0, 0, 0, 0, 0, \
0, 0, 0, 0, 0, 0, 0, 0, 0, 0, 0, 0, 0, 0, 0, 0, 0, 0, 0, 0, 0, 0, 0, \
0, 0, 0, 0}
\end{lstlisting}

Зашифрованный байт это 2, байт из ключа это 103, $2 \oplus 103=101$ и 101 это ASCII-код символа \q{e}.
Чему должен равнятся этот байт ключа, чтобы ASCII-код был 116 (для символа  \q{t})?
$2 \oplus 116=118$, присвоим 118 второму байту в ключе \dots

\begin{lstlisting}[style=custommath]
key = {80, 118, 2, 116, 113, 102, 118, 25, 99, 8, 19, 23, 116, 125, 
  107, 25, 99, 109, 114, 102, 14, 121, 115, 31, 9, 117, 113, 111, 5, 
  4, 127, 28, 122, 101, 8, 110, 14, 18, 124, 106, 16, 20, 104, 119, 8,
   109, 26, 106, 9, 97, 13, 99, 15, 119, 20, 105, 117, 98, 103, 118, 
  1, 126, 29, 97, 122, 17, 15, 114, 110, 3, 5, 125, 125, 99, 126, 119,
   102, 30, 122, 2, 117}
\end{lstlisting}

\dots и снова дешифруем весь файл.

\begin{figure}[H]
\centering
\myincludegraphics{ff/XOR/mask_1/mc_decrypted2.png}
\caption{Дешифрованный файл в Midnight Commander, вторая попытка}
\end{figure}

Ух ты, теперь грамматика корректна, и все фразы начинаются с корректных букв.
Но все таки, регистр подозрителен.
С чего бы разработчику игры записывать их в такой манере?
Может быть наш ключ все еще неправилен?

% TODO ASCII table somewhere in the book
Изучая таблицу ASCII мы можем заметить что ASCII-коды для букв в верхнем и нижнем регистре отличаются только на один бит
(6-й бит, если считать с первого, 0b100000):

\begin{figure}[H]
\centering
\includegraphics[width=0.7\textwidth]{ascii.png}
\caption{7-битная таблица \ac{ASCII} в Emacs}
\end{figure}

6-й бит, выставленный в нулевом байте, В десятичном виде это будет 32.
Но 32 это ASCII-код пробела!

Действительно, можно менять регистр просто применяя XOR к ASCII-коду, с 32 (больше об этом: \myref{toupper_bit}).

Возможно ли, что пустые лакуны в файле это не нулевые байты, а скорее содержащие пробелы?
Еще раз модифицируем наш XOR-ключ (я про-XOR-ю каждый байт ключа с 32):

\begin{lstlisting}[style=custommath]
(* "32" это скаляр, и "key" это вектор, но это OK *)

In[]:= key3 = BitXor[32, key]
Out[]= {112, 86, 34, 84, 81, 70, 86, 57, 67, 40, 51, 55, 84, 93, 75, \
57, 67, 77, 82, 70, 46, 89, 83, 63, 41, 85, 81, 79, 37, 36, 95, 60, \
90, 69, 40, 78, 46, 50, 92, 74, 48, 52, 72, 87, 40, 77, 58, 74, 41, \
65, 45, 67, 47, 87, 52, 73, 85, 66, 71, 86, 33, 94, 61, 65, 90, 49, \
47, 82, 78, 35, 37, 93, 93, 67, 94, 87, 70, 62, 90, 34, 85}

In[]:= DecryptBlock[blk_] := BitXor[key3, blk]
\end{lstlisting}

И снова дешифруем входной файл:

\begin{figure}[H]
\centering
\myincludegraphics{ff/XOR/mask_1/mc_decrypted.png}
\caption{Дешифрованный файл в Midnight Commander, последняя попытка}
\end{figure}

(Расшифрованный файл доступен здесь:
\url{\GitHubBlobMasterURL/ff/XOR/mask_1/files/decrypted.dat.bz2}.)

Несомненно, это корректный исходный файл.
Да, и мы видим числа в начале каждого блока. Должно быть это и есть источник некорректного XOR-ключа.
Как выходит, самый встречающийся 81-байтный блок в файле это блок заполненный пробелами и содержащий символ \q{1} на месте
второго байта.
Действительно, как-то так получилось что многие блоки здесь перемежаются с этим блоком.
Может быть это что-то вроде выравнивания (padding) для коротких фраз/сообщений?
Другой часто встречающийся 81-байтный блок также заполнен пробелами, но с другой цифрой, следовательно,
они отличаются только вторым байтом.

Вот и всё! Теперь мы можем написать утилиту для зашифрования файла назад, и, может быть, модифицировать его перед этим

Файл для Mathematica можно скачать здесь:\\
\url{\GitHubBlobMasterURL/ff/XOR/mask_1/files/XOR_mask_1.nb}.

Итог: XOR-шифрование не надежно вообще. Вероятно, разработчик игры хотел просто скрыть внутренности игры от игрока,
ничего более серьезного.
Все же, шифрование вроде этого крайне популярно вследствии его простоты, так что многие реверс инженеры обычно хорошо
с этим знакомы.

}
\DE{\myparagraph{\NonOptimizing MSVC}

MSVC 2010 erzeugt den folgenden Code:

\lstinputlisting[caption=\NonOptimizing MSVC
2010,style=customasmx86]{patterns/12_FPU/3_comparison/x86/MSVC/MSVC_DE.asm}

\myindex{x86!\Instructions!FLD}

Der Befehl \FLD lädt \GTT{\_b} nach \ST{0}.

\label{Czero_etc}
\newcommand{\Czero}{\GTT{C0}\xspace}
\newcommand{\Ctwo}{\GTT{C2}\xspace}
\newcommand{\Cthree}{\GTT{C3}\xspace}
\newcommand{\CThreeBits}{\Cthree/\Ctwo/\Czero}

\myindex{x86!\Instructions!FCOMP}
\FCOMP verlgeicht den Wert in \ST{0} mit dem Wert, der sich in \GTT{\_a}
befindet und setzt die \CThreeBits im FPU Status Register entsprechend.
Das Statusregister ist ein 16-Bit-Register, das den aktueller Zustand der FPU
abbildet.

Nachdem die Bits gesetzt worden sind, nimmer der \FCOMP Befehl auch eine
Variable vom Stack. Dieses Verhalten unterscheidet ihn von \FCOM, der einfach
zwei Werte vergleicht und den Stack unangetastet lässt.

Leider verfügen CPUs vor Intel P6\footnote{Intel P6 ist Pentium Pro, Pentium II,
etc.}über keinerlei bedingte Sprungbefehle, die die \CThreeBits prüfen.

After the bits are set, the \FCOMP instruction also pops one variable from the stack. 
This is what distinguishes it from \FCOM, which is just compares values, leaving the stack in the same state.
Vielleicht ist diese Tatsache historisch begründet (man erinnere sich: die FPU
war früher ein eigener Chip).\\
Moderne CPUs, beginnend mit Intel P6 haben \FCOMI/\FCOMIP/\FUCOMI/\FUCOMIP
Befehle~--welche im Prinzip das gleiche tun, aber die \ZF/\PF/\CF Flags der CPU
verändern können.

\myindex{x86!\Instructions!FNSTSW}
Der \FNSTSW Befehl kopiert das FPU Statusregister nach \AX.
\CThreeBits werden an den Stellen 14/10/8 abgelegt, sie befinden sich im \AX
Register an den gleichen Stellen und sie werden alle in höherwertigen Teil von
\AX{}~---\AH{} abgelegt.

\begin{itemize}
\item Falls in unserem Beispiel $b>a$, dann werden die \CThreeBits Bits wie
folgt gesetzt: 0, 0, 0.
\item Falls $a>b$, dann ist das Bitmuster: 0, 0, 1.
\item Falls $a=b$, dann ist das Bitmuster: 1, 0, 0.
\item

Wenn das Ergebnis (z.B. im Fehlerfall) ungeordnet ist, dann werden die Bits wie
folgt gesetzt: 1,1,1.
\end{itemize}
% TODO: table here?
So werden die \CThreeBits Bits im \AX Register angeordnet:

\input{C3_in_AX}

So werden die \CThreeBits Bits im \AH Register angeordnet:

\input{C3_in_AH}
Nach der Ausführung von \INS{test ah, 5}\footnote{5=101b} werden nur die \Czero
und \Ctwo Bits (an den Stellen 0 und 2) betrachtet, alle übrigen Bits werden
einfach überlesen.

\label{parity_flag}
\myindex{x86!\Registers!\Flags!Parity flag}
Werfen wir nun einen Blick auf ein anderes bemerkenswertes historisches
Überbleibsel: das \emph{parity flag}.

Dieses Flag wird auf 1 gesetzt, falls die Anzahl der Einsen im Ergebnis der
letzten Berechnung gerade ist und auf 1, falls dies nicht der Fall ist.

Schlagen wir in der Wikipedia nach\footnote{\WikipediaParityFlag}:

%TODO Quotation has been translated from English wiki article, since the
% correspondig German article doesn't offer such information.
\begin{framed}
\begin{quotation}
Ein guter Grund das Parity Flag abzufragen, hat tatsächlich gar nichts mit
Parität zu tun. Die FPU hat vier Bedingungsflags (C0 bis C3), aber diese können
nicht direkt abgefragt werden, sondern müssen zunächst in das Flags Register
kopiert werden. Wenn dies geschieht, wird C0 im Carry Flag abgelegt, C2 im
Parity Flag und C3 im Zero Flag.
Das C2 Flag ist gesetzt, wenn z.B. unvergleichbare Fließkommawerte (NaN oder
nicht unterstütztes Format) über der \FUCOM Befehl miteinander verglichen
werden.\textit{(Übersetzung aus der englischen Wikipedia.)}
\end{quotation}
\end{framed}

Wie in der Wikipedia dargestellt wird das Parity Flag manchmal im FPU Code
verwendet; schauen wir uns genauer an wie das funktioniert.

\myindex{x86!\Instructions!JP}
Das \PF Flag wird auf 1 gesetzt, wenn sowohl \Czero als auch \Ctwo beide 0 oder
beide 1 sind. In diesem Fall wird der nachfolgende Sprung \JP(\emph{jump if
PF==1}) ausgeführt.
Wenn wir die Werte der \CThreeBits in den unterschiedlichen Fällen betrachten,
dann sehen wir, dass der bedingte Sprung \JP in zwei Fällen ausgeführt wird:
wenn $b>a$ oder wenn $a=b$ (das \Cthree Bit wird hier nicht betrachtet, da es
durch den Befehl \INS{test ah,5}) gelöscht wurde).

Der Rest ist leicht nachvollziehbar.
Denn der bedingte Sprung ausgeführt wurde, lädt \FLD den Wert von \GTT{\_b} nach
\ST{0} und wenn nicht, wird der Wert von \GTT{\_a} dorthin geladen.

\mysubparagraph{Was ist mit der Abfrage von \Ctwo?}
Das \Ctwo Flag wird im Fehlerfall (\gls{NaN}, etc.) gesetzt, aber unser Code
prüft dies nicht. 
Wenn sich der Programmierer für FPU Fehler interessiert, muss er zusätzliche
Abfragen hinzufügen.

\input{patterns/12_FPU/3_comparison/x86/MSVC/olly_DE.tex}
}
\FR{\mysection{Fonction presque vide}
\label{Boolector}
\myindex{Boolector}
\myindex{x86!\Instructions!JMP}

Ceci est un morceau de code réel que j'ai trouvé dans Boolector\footnote{\url{https://boolector.github.io/}}:

\lstinputlisting[style=customc]{patterns/025_almost_empty/boolectormain.c}

Pourquoi quelqu'un ferait-il comme ça?
Je ne sais pas mais mon hypothèse est que \verb|boolector_main()| peut être compilée
dans une sorte de DLL ou bibliothèque dynamique, et appelée depuis une suite de test.
Certainement qu'une suite de test peut préparer les variables argc/argv comme
le ferait \ac{CRT}.

Il est intéressant de voir comment c'est compilé:

\lstinputlisting[caption=GCC 8.2 x64 \NonOptimizing (\assemblyOutput),style=customasmx86]{patterns/025_almost_empty/boolectormain_O0.s}

Ceci est OK, le prologue (non optimisé) déplace inutilement deux arguments,
\INS{CALL}, épilogue, \INS{RET}.
Mais regardons la version optimisée:

\lstinputlisting[caption=GCC 8.2 x64 \Optimizing (\assemblyOutput),style=customasmx86]{patterns/025_almost_empty/boolectormain_O3.s}

Aussi simple que ça: la pile et les registres ne sont pas touchés et \verb|boolector_main()|
a le même ensemble d'arguments.
Donc, tout ce que nous avons à faire est de passer l'exécution à une autre adresse.

Ceci est proche d'une \glslink{thunk function}{fonction thunk}.

Nous verons queelque chose de plus avancé plus tard: \myref{ARM_B_to_printf}, \myref{JMP_instead_of_RET}.
}

\EN{\input{other/8086mm_EN}}
\RU{\input{other/8086mm_RU}}
\DE{\input{other/8086mm_DE}}
\FR{\mysection{Modèle de mémoire du 8086}
\myindex{Intel!8086!Memory model}
\myindex{MS-DOS}
\label{8086_memory_model}

Lorsque l'on a à faire avec des programmes 16-bit pour MS-DOS ou Win16
(\myref{dongle_16bit_dos} ou \myref{win16_near_far_pointers}),
nous voyons que les pointeurs consistent en deux valeurs 16-bit.
Que signifient-elles? Eh oui, c'est encore un artefact étrange de MS-DOS et du 8086.

Le 8086/8088 était un CPU 16-bit, mais était capable d'accèder à des adresses mémoire
sur 20-bit (il était donc capable d'accèder 1MB de mémoire externe).

L'espace de la mémoire externe était divisé entre la \ac{RAM} (max 640KB), la \ac{ROM},
la fenêtre pour la mémoire vidéo, les cartes EMS, etc.

Rappelons que le 8086/8088 était en fait un descendant du CPU 8-bit 8080.

Le 8080 avait un espace mémoire de 16-bit, i.e., il pouvait seulement adresser 64KB.

Et probablement pour une raison de portage de vieux logiciels\footnote{Je ne suis
pas sûr à 100\% de ceci}, le 8086 peut supporter plusieurs fenêtres de 64KB simultanément,
situées dans l'espace d'adresse de 1MB.

C'est une sorte de virtualisation de niveau jouet.
\myindex{x86!\Registers!CS}
\myindex{x86!\Registers!DS}
\myindex{x86!\Registers!ES}
\myindex{x86!\Registers!SS}

Tous les registres 8086 sont 16-bit, donc pour adresser plus, des registres spéciaux
de segment (CS, DS, ES, SS) ont été ajoutés.

Chaque pointeur de 20-bit est calculé en utilisant les valeurs d'une paire de registres,
de segment et d'adresse (p.e. DS:BX) comme suit:

\begin{center}
$real\_address = (segment\_register \ll 4) + address\_register$
\end{center}

Par exemple, la fenêtre de \ac{RAM} graphique (\ac{EGA}, \ac{VGA}) sur les vieux
compatibles IBM-PC a une taille de 64KB.

Pour y accèder, une valeur de 0xA000 doit être stockée dans l'un des registres de
segments, p.e. dans DS.

Ainsi DS:0 adresse le premier octet de la \ac{RAM} vidéo et DS:0xFFFF ~--- le dernier
octet de RAM.

L'adresse réelle sur le bus d'adresse de 20-bit, toutefois, sera dans l'intervalle
0xA0000 à 0xAFFFF.

Le programme peut contenir des adresses codées en dur comme 0x1234, mais l'\ac{OS}
peut avoir besoin de le charger à une adresse arbitraire, donc il recalcule les valeurs
du registre de segment de façon à ce que le programme n'ait pas à se soucier de
l'endroit où il est placé dans la RAM.

Donc, tout pointeur dans le vieil environnement MS-DOS consistait en fait en un segment
d'adresse et une adresse dans ce segment, i.e., deux valeurs 16-bit. 20-bit étaient
suffisants pour cela, cependant nous devions recalculer les adresses très souvent:
passer plus d'informations par la pile semblait un meilleur rapport espace/facilité.

À propos, à cause de tout cela, il n'était pas possible d'allouer un bloc de mémoire
plus large que 64KB.

\myindex{Intel!80286}
\myindex{Intel!80386}

Les registres de segment furent réutilisés sur les 80286 comme sélecteurs, servant
a une fonction différente.

\myindex{MS-DOS!DOS extenders}

Lorsque les CPU 80386 et les ordinateurs avec plus de \ac{RAM} ont été introduits,
MS-DOS était encore populaire, donc des extensions pour DOS ont émergés: ils étaient
en fait une étape vers un \ac{OS} \q{sérieux}, basculant le CPU en mode protégé et
fournissant des \ac{API}s mémoire bien meilleures pour les programmes qui devaient
toujours fonctionner sous MS-DOS.

Des examples très populaires incluent DOS/4GW (le jeux vidéo DOOM a été compilé pour
lui), Phar Lap, PMODE.
\par
\myindex{Windows!Windows 3.x}
\myindex{Windows!Win32}

À propos, la même manière d'adresser la mémoire était utilisée dans la série 16-bit
de Windows 3.x, avant Win32.

}

\EN{\input{other/bb_reordering_EN}}
\RU{\input{other/bb_reordering_RU}}
\DE{\input{other/bb_reordering_DE}}
\FR{\input{other/bb_reordering_FR}}

\EN{\mysection{My experience with Hex-Rays 2.2.0}
\myindex{Hex-Rays}
\label{hex_rays}

\subsection{Bugs}

There are couple of bugs.

First of all, Hex-Rays is getting lost when \ac{FPU} instructions are interleaved (by compiler codegenerator) with others.

For example, this:

\begin{lstlisting}[style=customasmx86]
f               proc    near

        	lea     eax, [esp+4]
	        fild    dword ptr [eax]
                lea     eax, [esp+8]
        	fild    dword ptr [eax]
	        fabs
                fcompp
        	fnstsw  ax
	        test    ah, 1
                jz      l01

        	mov     eax, 1
	        retn
l01:
                mov     eax, 2
	        retn

f               endp
\end{lstlisting}

\dots will be correcly decompiled to:

\begin{lstlisting}[style=customc]
signed int __cdecl f(signed int a1, signed int a2)
{
  signed int result; // eax@2

  if ( fabs((double)a2) >= (double)a1 )
    result = 2;
  else
    result = 1;
  return result;
}
\end{lstlisting}

But let's comment one of the instructions at the end:

\begin{lstlisting}[style=customasmx86]
...
l01:
	        ;mov    eax, 2
        	retn
...
\end{lstlisting}

\dots we getting an obvious bug:

\begin{lstlisting}[style=customc]
void __cdecl f(char a1, char a2)
{
  fabs((double)a2);
}
\end{lstlisting}

This is another bug:

\begin{lstlisting}[style=customasmx86]
extrn f1:dword
extrn f2:dword

f               proc    near

	        fld     dword ptr [esp+4]
        	fadd    dword ptr [esp+8]
                fst     dword ptr [esp+12]
	        fcomp   ds:const_100
                fld     dword ptr [esp+16]      ; comment this instruction and it will be OK
	        fnstsw  ax
        	test    ah, 1

                jnz     short l01

	        call    f1
        	retn
l01:
	        call    f2
        	retn

f               endp

...

const_100       dd 42C80000h            ; 100.0
\end{lstlisting}

Result:

\begin{lstlisting}[style=customc]
int __cdecl f(float a1, float a2, float a3, float a4)
{
  double v5; // st7@1
  char v6; // c0@1
  int result; // eax@2

  v5 = a4;
  if ( v6 )
    result = f2(v5);
  else
    result = f1(v5);
  return result;
}
\end{lstlisting}

\emph{v6} variable has \emph{char} type and if you'll try to compile this code, compiler will warn you about variable
usage before assignment.

Another bug: \INS{FPATAN} instruction is correctly decompiled into \emph{atan2()}, but arguments are swapped.

\subsection{Odd peculiarities}

Hex-Rays too often promotes 32-bit \emph{int} to 64-bit one.
Here is example:

\begin{lstlisting}[style=customasmx86]
f               proc    near

	        mov     eax, [esp+4]
                cdq
	        xor     eax, edx
        	sub     eax, edx
                ; EAX=abs(a1)

	        sub     eax, [esp+8]
        	; EAX=EAX-a2

                ; EAX at this point somehow gets promoted to 64-bit (RAX)

	        cdq
        	xor     eax, edx
                sub     eax, edx
	        ; EAX=abs(abs(a1)-a2)

                retn

f               endp
\end{lstlisting}

Result:

\begin{lstlisting}[style=customc]
int __cdecl f(int a1, int a2)
{
  __int64 v2; // rax@1

  v2 = abs(a1) - a2;
  return (HIDWORD(v2) ^ v2) - HIDWORD(v2);
}
\end{lstlisting}

Perhaps, this is result of \INS{CDQ} instruction? I'm not sure.
Anyway, whenever you see \emph{\_\_int64} type in 32-bit code, pay attention.

This is also weird:

\begin{lstlisting}[style=customasmx86]
f               proc    near

	        mov     esi, [esp+4]

        	lea     ebx, [esi+10h]
                cmp     esi, ebx
	        jge     short l00

                cmp     esi, 1000
	        jg      short l00

                mov     eax, 2
	        retn

l00:
	        mov     eax, 1
        	retn

f               endp
\end{lstlisting}

Result:

\begin{lstlisting}[style=customc]
signed int __cdecl f(signed int a1)
{
  signed int result; // eax@3

  if ( __OFSUB__(a1, a1 + 16) ^ 1 && a1 <= 1000 )
    result = 2;
  else
    result = 1;
  return result;
}
\end{lstlisting}

The code is correct, but needs manual intervention.

Sometimes, Hex-Rays doesn't fold (or reduce) division by multiplication code:

\begin{lstlisting}[style=customasmx86]
f               proc    near

        	mov     eax, [esp+4]
	        mov     edx, 2AAAAAABh
                imul    edx
        	mov     eax, edx

	        retn

f               endp
\end{lstlisting}

Result:

\begin{lstlisting}[style=customc]
int __cdecl f(int a1)
{
  return (unsigned __int64)(715827883i64 * a1) >> 32;
}
\end{lstlisting}

This can be folded (rewritten) manually.

Many of these peculiarities can be solved by manual reordering of instructions, recompiling assembly code,
and then feeding it to Hex-Rays again.

\subsection{Silence}

\begin{lstlisting}[style=customasmx86]
extrn some_func:dword

f               proc    near

        	mov     ecx, [esp+4]
	        mov     eax, [esp+8]
                push    eax
        	call    some_func
	        add     esp, 4

                ; use ECX
        	mov     eax, ecx

	        retn

f               endp
\end{lstlisting}

Result:

\begin{lstlisting}[style=customc]
int __cdecl f(int a1, int a2)
{
  int v2; // ecx@1

  some_func(a2);
  return v2;
}
\end{lstlisting}

\emph{v2} variable (from ECX) is lost \dots
Yes, this code is incorrect (ECX value doesn't saved during call to another function),
but it would be good for Hex-Rays to give a warning.

Another one:

\begin{lstlisting}[style=customasmx86]
extrn some_func:dword

f               proc    near

	        call    some_func
        	jnz     l01

                mov     eax, 1
	        retn
l01:
	        mov     eax, 2
        	retn

f               endp
\end{lstlisting}

Result:

\begin{lstlisting}[style=customc]
signed int f()
{
  char v0; // zf@1
  signed int result; // eax@2

  some_func();
  if ( v0 )
    result = 1;
  else
    result = 2;
  return result;
}
\end{lstlisting}

Again, warning would be great.

Anyway, whenever you see variable of \emph{char} type, or variable which is used without initialization, this is clear sign
that something went wrong and needs manual intervention.

\subsection{Comma}
\myindex{\CLanguageElements!Comma}

Comma in C/C++ has a bad fame, because it can lead to a confusing code.

Quick quiz, what does this C/C++ function return?

\begin{lstlisting}[style=customc]
int f()
{
	return 1, 2;
};
\end{lstlisting}

It's 2: when compiler encounters comma-expression, it generates code which executes all sub-expressions, and
\emph{returns} value of the last sub-expression.

I've seen something like that in production code:

\begin{lstlisting}[style=customc]
if (cond)
	return global_var=123, 456; // 456 is returned
else
	return global_var=789, 321; // 321 is returned
\end{lstlisting}

Apparently, programmer wanted to make code slightly shorter without additional curly brackets.
In other words, comma allows to pack couple of expressions into one, without forming
statement/code block inside of curly brackets.

\myindex{Scheme}
\myindex{Racket}
Comma in C/C++ is close to \TT{begin} in Scheme/Racket: \url{https://docs.racket-lang.org/guide/begin.html}.

Perhaps, the only widely accepted usage of comma is in \emph{for()} statements:

\begin{lstlisting}[style=customc]
char *s="hello, world";
for(int i=0; *s; s++, i++);
// i = string length
\end{lstlisting}

Both \emph{s++} and \emph{i++} are executed at each loop iteration.

Read more: \url{https://stackoverflow.com/q/52550}.

\myindex{\CLanguageElements!Short-circuit}
I'm writing all this because Hex-Rays produces (at least in my case) code which is rich with both commas and short-circuit
expressions.
For example, this is real output from Hex-Rays:

\begin{lstlisting}[style=customc]
 if ( a >= b || (c = a, (d[a] - e) >> 2 > f) )
    {
    	...
\end{lstlisting}

This is correct, it compiles and works, and let god help you to understand it.
Here is it rewritten:

\begin{lstlisting}[style=customc]
if (cond1 || (comma_expr, cond2))
{
	...
\end{lstlisting}

Short-circuit is effective here: first \emph{cond1} is checked, if it's \emph{true}, \emph{if()} body is executed, the rest
of \emph{if()} expression is ignored completely.
If \emph{cond1} is \emph{false}, \emph{comma\_expr} is executed (in the previous example, \emph{a} gets copied to \emph{c}),
then \emph{cond2} is checked.
If \emph{cond2} is \emph{true}, \emph{if()} body gets executed, or not.
In other words, \emph{if()} body gets executed if \emph{cond1} is \emph{true} or \emph{cond2} is \emph{true},
but if the latter is \emph{true}, \emph{comma\_expr} is also executed.

Now you can see why comma is so notorious.

\textbf{A word about short-circuit.}
A common beginner's misconception is that sub-conditions are checked in some unspecified order, which is not true.
In \TT{a | b | c} expression, $a$, $b$ and $c$ gets evaluated in unspecified order, so that is why \TT{||} has also been
added to C/C++, to apply short-circuit explicitly.

\subsection{Data types}

Data types is a problem for decompilers.

Hex-Rays can be blind to arrays in local stack, if they weren't set correctly before decompilation.
Same story about global arrays.

Another problem is too big functions, where a single slot in local stack can be used by several variables
across function's execution.
It's not a rare case when a slot is used for \emph{int}-variable, then for pointer, then for \emph{float}-variable.
Hex-Rays correctly decompiles it: it creates a variable with some type, then cast it to another type in various
parts of functions.
This problem has been solved by me by manual splitting big function into several smaller.
Just make local variables as global ones, etc, etc.
And don't forget about tests.

\subsection{Long and messed expressions}

Sometimes, during rewriting, you can end up with long and hard to understand expressions in \emph{if()} constructs, like:

\begin{lstlisting}[style=customc]
if ((! (v38 && v30 <= 5 && v27 != -1)) && ((! (v38 && v30 <= 5) && v27 != -1) || (v24 >= 5 || v26)) && v25)
{
...
}
\end{lstlisting}

Wolfram Mathematica can minimize some of them, using \TT{BooleanMinimize[]} function:

\begin{lstlisting}
In[1]:= BooleanMinimize[(! (v38 && v30 <= 5 && v27 != -1)) && v38 && v30 <= 5 && v25 == 0]

Out[1]:= v38 && v25 == 0 && v27 == -1 && v30 <= 5
\end{lstlisting}

There is even better way, to find common subexpressions:

\begin{lstlisting}
In[2]:= Experimental`OptimizeExpression[(! (v38 && v30 <= 5 && 
      v27 != -1)) && ((! (v38 && v30 <= 5) && 
      v27 != -1) || (v24 >= 5 || v26)) && v25]

Out[2]= Experimental`OptimizedExpression[
 Block[{Compile`$1, Compile`$2}, Compile`$1 = v30 <= 5; 
  Compile`$2 = 
   v27 != -1; ! (v38 && Compile`$1 && 
      Compile`$2) && ((! (v38 && Compile`$1) && Compile`$2) || 
     v24 >= 5 || v26) && v25]]
\end{lstlisting}

Mathematica has added two new variables: \TT{Compile`\$1} and \TT{Compile`\$2}, values of which are to be used several times in expression.
So we can add two additional variables.

\input{other/DeMorgan_EN}

\subsection{My plan}

\begin{itemize}
\item Split big functions (and don't forget about tests).
Sometimes it's very helpful to form new functions out of big loop bodies.

\item Check/set data type of variables, arrays, etc.

\item If you see odd result, \emph{dangling} variable (which used before initialization), try to swap instructions manually,
recompile it and feed to Hex-Rays again.
\end{itemize}

\subsection{Summary}

Nevertheless, quality of Hex-Rays 2.2.0 is very, very good.
It makes life way easier.

}
\RU{\mysection{Мой опыт с Hex-Rays 2.2.0}
\myindex{Hex-Rays}
\label{hex_rays}

\subsection{Ошибки}

Есть несколько ошибок.

Прежде всего, Hex-Rays теряется, когда инструкции \ac{FPU} перемежаются (кодегенератором компилятора) с другими.

Например:

\begin{lstlisting}[style=customasmx86]
f               proc    near

        	lea     eax, [esp+4]
	        fild    dword ptr [eax]
                lea     eax, [esp+8]
        	fild    dword ptr [eax]
	        fabs
                fcompp
        	fnstsw  ax
	        test    ah, 1
                jz      l01

        	mov     eax, 1
	        retn
l01:
                mov     eax, 2
	        retn

f               endp
\end{lstlisting}

\dots будет корректно декомпилировано в:

\begin{lstlisting}[style=customc]
signed int __cdecl f(signed int a1, signed int a2)
{
  signed int result; // eax@2

  if ( fabs((double)a2) >= (double)a1 )
    result = 2;
  else
    result = 1;
  return result;
}
\end{lstlisting}

Но давайте закомментируем одну инструкцию в конце:

\begin{lstlisting}[style=customasmx86]
...
l01:
	        ;mov    eax, 2
        	retn
...
\end{lstlisting}

\dots получаем явную ошибку:

\begin{lstlisting}[style=customc]
void __cdecl f(char a1, char a2)
{
  fabs((double)a2);
}
\end{lstlisting}

Вот еще ошибка:

\begin{lstlisting}[style=customasmx86]
extrn f1:dword
extrn f2:dword

f               proc    near

	        fld     dword ptr [esp+4]
        	fadd    dword ptr [esp+8]
                fst     dword ptr [esp+12]
	        fcomp   ds:const_100
                fld     dword ptr [esp+16]      ; закомментируйте эту инструкцию, и всё будет хорошо
	        fnstsw  ax
        	test    ah, 1

                jnz     short l01

	        call    f1
        	retn
l01:
	        call    f2
        	retn

f               endp

...

const_100       dd 42C80000h            ; 100.0
\end{lstlisting}

Результат:

\begin{lstlisting}[style=customc]
int __cdecl f(float a1, float a2, float a3, float a4)
{
  double v5; // st7@1
  char v6; // c0@1
  int result; // eax@2

  v5 = a4;
  if ( v6 )
    result = f2(v5);
  else
    result = f1(v5);
  return result;
}
\end{lstlisting}

У переменной \emph{v6} тип \emph{char}, и если вы попытаетесь скомпилировать этот код, компилятор выдаст предупреждение о том,
что переменная используется перед тем, как была инициализирована.

Еще ошибка: инструкция \INS{FPATAN} корректно декомпилируется в \emph{atan2()}, но аргументы перепутаны.

\subsection{Странности}

Hex-Rays часто конвертирует 32-битный \emph{int} в 64-битный.
Вот пример:

\begin{lstlisting}[style=customasmx86]
f               proc    near

	        mov     eax, [esp+4]
                cdq
	        xor     eax, edx
        	sub     eax, edx
                ; EAX=abs(a1)

	        sub     eax, [esp+8]
        	; EAX=EAX-a2

                ; в этом месте, EAX каким-то образом был сконвертирован в 64-битный (RAX)

	        cdq
        	xor     eax, edx
                sub     eax, edx
	        ; EAX=abs(abs(a1)-a2)

                retn

f               endp
\end{lstlisting}

Результат:

\begin{lstlisting}[style=customc]
int __cdecl f(int a1, int a2)
{
  __int64 v2; // rax@1

  v2 = abs(a1) - a2;
  return (HIDWORD(v2) ^ v2) - HIDWORD(v2);
}
\end{lstlisting}

Возможно, это результат инструкции \INS{CDQ}? Я не уверен.
Так или иначе, если вы видите тип \emph{\_\_int64} вы 32-битном коде, обращайте внимание.

Это тоже странно:

\begin{lstlisting}[style=customasmx86]
f               proc    near

	        mov     esi, [esp+4]

        	lea     ebx, [esi+10h]
                cmp     esi, ebx
	        jge     short l00

                cmp     esi, 1000
	        jg      short l00

                mov     eax, 2
	        retn

l00:
	        mov     eax, 1
        	retn

f               endp
\end{lstlisting}

Результат:

\begin{lstlisting}[style=customc]
signed int __cdecl f(signed int a1)
{
  signed int result; // eax@3

  if ( __OFSUB__(a1, a1 + 16) ^ 1 && a1 <= 1000 )
    result = 2;
  else
    result = 1;
  return result;
}
\end{lstlisting}

Код корректный, но требует ручного вмешательства.

Иногда Hex-Rays не сокращает деление через умножение:

\begin{lstlisting}[style=customasmx86]
f               proc    near

        	mov     eax, [esp+4]
	        mov     edx, 2AAAAAABh
                imul    edx
        	mov     eax, edx

	        retn

f               endp
\end{lstlisting}

Результат:

\begin{lstlisting}[style=customc]
int __cdecl f(int a1)
{
  return (unsigned __int64)(715827883i64 * a1) >> 32;
}
\end{lstlisting}

Это можно сократить вручную.

Многие из этих странностей можно решить при помощи ручного переупорядочивания инструкций, перекомпиляции ассемблерного кода,
и затем подачи его снова на вход Hex-Rays.

\subsection{Безмолвие}

\begin{lstlisting}[style=customasmx86]
extrn some_func:dword

f               proc    near

        	mov     ecx, [esp+4]
	        mov     eax, [esp+8]
                push    eax
        	call    some_func
	        add     esp, 4

                ; используем ECX
        	mov     eax, ecx

	        retn

f               endp
\end{lstlisting}

Результат:

\begin{lstlisting}[style=customc]
int __cdecl f(int a1, int a2)
{
  int v2; // ecx@1

  some_func(a2);
  return v2;
}
\end{lstlisting}

Переменная \emph{v2} (из ECX) потерялась \dots
Да, код некорректный (значение в регистре ECX не сохраняется после вызова другой ф-ции),
но для Hex-Rays было бы неплохо выдать предупреждение.

Вот еще:

\begin{lstlisting}[style=customasmx86]
extrn some_func:dword

f               proc    near

	        call    some_func
        	jnz     l01

                mov     eax, 1
	        retn
l01:
	        mov     eax, 2
        	retn

f               endp
\end{lstlisting}

Результат:

\begin{lstlisting}[style=customc]
signed int f()
{
  char v0; // zf@1
  signed int result; // eax@2

  some_func();
  if ( v0 )
    result = 1;
  else
    result = 2;
  return result;
}
\end{lstlisting}

И снова, предупреждение бы помогло.

Так или иначе, если вы видите переменную типа \emph{char}, которая используется без инициализации, это явный признак того,
что что-то пошло не так и требует ручного вмешательства.

\subsection{Запятая}
\myindex{\CLanguageElements!Запятая}

Запятая в Си/Си++ имеет дурную славу, потому что она приводит к малопонятному коду.

Вот простая задача, что возвращает эта ф-ция на Си/Си++?

\begin{lstlisting}[style=customc]
int f()
{
	return 1, 2;
};
\end{lstlisting}

Это 2: когда компилятор встречает выражение с запятой, он генерирует код, исполняющий все подвыражения, и \emph{возвращает}
значение последнего подвыражения.

Я видел такое в реальном коде:

\begin{lstlisting}[style=customc]
if (cond)
	return global_var=123, 456; // возвращается 456
else
	return global_var=789, 321; // возвращается 321
\end{lstlisting}

Вероятно, автор хотел сделать код немного короче, без дополнительных фигурных скобок.
Другими словами, запятая позволяет упаковать несколько выражений в одно, без формирования
блока внутри фигурных скобок.

\myindex{Scheme}
\myindex{Racket}
Запятая в Си/Си++ близка к \TT{begin} в Scheme/Racket: \url{https://docs.racket-lang.org/guide/begin.html}.

Вероятно, есть только одно популярное и всеми одобренное использование запятой, это в выражении \emph{for()}:

\begin{lstlisting}[style=customc]
char *s="hello, world";
for(int i=0; *s; s++, i++);
// i = длина строки
\end{lstlisting}

И \emph{s++} и \emph{i++} исполняются на каждой итерации цикла.

Читайте больше об этом: \url{https://stackoverflow.com/q/52550}.

\myindex{\CLanguageElements!Short-circuit}
Я пишу всё это потому что Hex-Rays выдает код (как минимум, в моем случае) очень богатый и на запятые и на
short-circuit-выражения (\emph{короткое замыкание}).
Например, вот реальный пример работы Hex-Rays:

\begin{lstlisting}[style=customc]
 if ( a >= b || (c = a, (d[a] - e) >> 2 > f) )
    {
    	...
\end{lstlisting}

Это корректно, оно компилируется и работает, и да поможет вам бог понять, как.
Вот оно переписанное:

\begin{lstlisting}[style=customc]
if (cond1 || (comma_expr, cond2))
{
	...
\end{lstlisting}

Здесь работает \emph{short-circuit} (\emph{короткое замыкание}): в начале проверяется \emph{cond1}, и если оно истинно,
исполняется тело \emph{if()}, и остальная часть выражения \emph{if()} полностью игнорируется.
Если \emph{cond1} ложно, тогда исполняется \emph{comma\_expr} (в предыдущем примере, \emph{a} копируется в \emph{c}),
затем проверяется \emph{cond2}.
Если \emph{cond2} истинно, исполняется тело \emph{if()}, или нет.
Другими словами, тело \emph{if()} исполняется, если \emph{cond1} истинно, или если \emph{cond2} истинно,
но если последнее истинно, исполняется также \emph{comma\_expr}.

Теперь вы видите, почему запятая имеет такую славу.

\textbf{Еще о short-circuit (короткое замыкание).}
Частое заблуждение начинающих в том, что подусловия проверяются в каком-то незаданном порядке, и это не верно.
В выражении \TT{a | b | c}, $a$, $b$ и $c$ исполняются в незаданном порядке, и вот почему в Си/Си++ был также добавлен
оператор \TT{||}, чтобы явно применять \emph{short-circuit}.

\subsection{Типы данных}

Типы данных это проблема для декомпиляторов.

Hex-Rays может быть слеп к массивам в локальном стеке, если они не были корректно обозначены перед декомпиляцией.
Та же история с глобальными массивами.

Другая проблема это большие ф-ции, где один и тот же слот в локальном стеке может использоваться разными переменными
во время исполнения ф-ции.
Нередкий случай, когда слот в начале используется для \emph{int}-переменной, затем для указателя, затем для переменной типа
\emph{float}.
Hex-Rays корректно декомпилирует это: он создает переменную с каким-то типом, затем приводит её к другому типу в разных частях
ф-ции.
Эта проблема решалась мною ручным разделением ф-ции на несколько меньших.
Просто сделайте локальные переменные глобальными, итд, итд.
И не забывайте о тестах.

\subsection{Длинные и запутанные выражения}

Иногда, во время переписывания, вы можете прийти к длинным и труднопонятным выражениям в конструкциях \emph{if()}, вроде:

\begin{lstlisting}[style=customc]
if ((! (v38 && v30 <= 5 && v27 != -1)) && ((! (v38 && v30 <= 5) && v27 != -1) || (v24 >= 5 || v26)) && v25)
{
...
}
\end{lstlisting}

Wolfram Mathematica может оптимизировать некоторые из них, используя ф-цию \TT{BooleanMinimize[]}:

\begin{lstlisting}
In[1]:= BooleanMinimize[(! (v38 && v30 <= 5 && v27 != -1)) && v38 && v30 <= 5 && v25 == 0]

Out[1]:= v38 && v25 == 0 && v27 == -1 && v30 <= 5
\end{lstlisting}

Есть даже способ еще лучше, с поиском общих подвыражений:

\begin{lstlisting}
In[2]:= Experimental`OptimizeExpression[(! (v38 && v30 <= 5 &&
      v27 != -1)) && ((! (v38 && v30 <= 5) &&
      v27 != -1) || (v24 >= 5 || v26)) && v25]
Out[2]= Experimental`OptimizedExpression[
 Block[{Compile`$1, Compile`$2}, Compile`$1 = v30 <= 5;
  Compile`$2 =
   v27 != -1; ! (v38 && Compile`$1 &&
      Compile`$2) && ((! (v38 && Compile`$1) && Compile`$2) ||
     v24 >= 5 || v26) && v25]]
\end{lstlisting}

Mathematica может добавить две новых переменных: \TT{Compile`\$1} и \TT{Compile`\$2}, значения которых будут использоваться
в выражении несколько раз.
Так что мы можем добавить две дополнительных переменных.

%\input{other/DeMorgan_RU}

\subsection{Мой план}

\begin{itemize}
\item Разделить большие ф-ции (и не забывать о тестах).
Иногда очень полезно формировать новые ф-ции из тел циклов.

\item Проверяйте/устанавливайте тип данных для переменных, массивов, итд.

\item Если вы видите странный результат, \emph{висящую} переменную (которая используется перед инициализацией),
попробуйте поменять инструкции вручную, перекомпилировать и снова подать на вход Hex-Rays-у.
\end{itemize}

\subsection{Итог}

Тем не мнее, качество Hex-Rays 2.2.0 очень и очень хорошее.
Он делает жизнь легче.

}
\FR{\section{Mon expérience avec Hex-Rays 2.2.0}
\myindex{Hex-Rays}
\label{hex_rays}

\subsection{Bugs}

Il y a plusieurs bugs.

Tout d'abord, Hex-Rays est perdu lorsque des instructions \ac{FPU} sont mélangées
(par le générateur de code du compilateur) avec des autres.

Par exemple, ceci:

\begin{lstlisting}
f               proc    near

        	lea     eax, [esp+4]
	        fild    dword ptr [eax]
                lea     eax, [esp+8]
        	fild    dword ptr [eax]
	        fabs
                fcompp
        	fnstsw  ax
	        test    ah, 1
                jz      l01

        	mov     eax, 1
	        retn
l01:
                mov     eax, 2
	        retn

f               endp
\end{lstlisting}

\dots sera correctement décompilé en:

\begin{lstlisting}
signed int __cdecl f(signed int a1, signed int a2)
{
  signed int result; // eax@2

  if ( fabs((double)a2) >= (double)a1 )
    result = 2;
  else
    result = 1;
  return result;
}
\end{lstlisting}

Mais commentons une des instructions à la fin:

\begin{lstlisting}
...
l01:
	        ;mov    eax, 2
        	retn
...
\end{lstlisting}

\dots nous obtenons ce bug évident:

\begin{lstlisting}
void __cdecl f(char a1, char a2)
{
  fabs((double)a2);
}
\end{lstlisting}

Ceci est un autre bug:

\begin{lstlisting}
extrn f1:dword
extrn f2:dword

f               proc    near

	        fld     dword ptr [esp+4]
        	fadd    dword ptr [esp+8]
                fst     dword ptr [esp+12]
	        fcomp   ds:const_100
                fld     dword ptr [esp+16]      ; commenter cette instruction et ça sera OK%% comment this instruction and it will be OK
	        fnstsw  ax
        	test    ah, 1

                jnz     short l01

	        call    f1
        	retn
l01:
	        call    f2
        	retn

f               endp

...

const_100       dd 42C80000h            ; 100.0
\end{lstlisting}

Résultat:

\begin{lstlisting}
int __cdecl f(float a1, float a2, float a3, float a4)
{
  double v5; // st7@1
  char v6; // c0@1
  int result; // eax@2

  v5 = a4;
  if ( v6 )
    result = f2(v5);
  else
    result = f1(v5);
  return result;
}
\end{lstlisting}

La variable \TT{v6} a un type \TT{char} et si vous essayez de compiler ce code, le
compilateur vous avertira à propos de l'utilisation de variable avant son initialisation.

Un autre bug: l'instruction \INS{FPATAN} est correctement décompilée en \TT{atan2()},
mais les arguments sont échangés.

\subsection{Particularités bizarres}

Hex-Rays converti trop souvent des \TT{int} 32-bit en 64-bit.
Voici un exemple:

\begin{lstlisting}
f               proc    near

	        mov     eax, [esp+4]
                cdq
	        xor     eax, edx
        	sub     eax, edx
                ; EAX=abs(a1)

	        sub     eax, [esp+8]
        	; EAX=EAX-a2

                %%; EAX at this point somehow gets promoted to 64-bit (RAX)
                ; EAX à ce point est converti en 64-bit (RAX)

	        cdq
        	xor     eax, edx
                sub     eax, edx
	        ; EAX=abs(abs(a1)-a2)

                retn

f               endp
\end{lstlisting}

Résultat:

\begin{lstlisting}
int __cdecl f(int a1, int a2)
{
  __int64 v2; // rax@1

  v2 = abs(a1) - a2;
  return (HIDWORD(v2) ^ v2) - HIDWORD(v2);
}
\end{lstlisting}

Peut-être est-ce le résultat de l'instruction \INS{CDQ}? Je ne suis pas sûr.
Quoiqu'il en soit, à chaque fois que vous voyez le type \TT{\_\_int64} dans du code
32-bit, soyez attentif.

Ceci est aussi bizarre:

\begin{lstlisting}
f               proc    near

	        mov     esi, [esp+4]

        	lea     ebx, [esi+10h]
                cmp     esi, ebx
	        jge     short l00

                cmp     esi, 1000
	        jg      short l00

                mov     eax, 2
	        retn

l00:
	        mov     eax, 1
        	retn

f               endp
\end{lstlisting}

Résultat:

\begin{lstlisting}
signed int __cdecl f(signed int a1)
{
  signed int result; // eax@3

  if ( __OFSUB__(a1, a1 + 16) ^ 1 && a1 <= 1000 )
    result = 2;
  else
    result = 1;
  return result;
}
\end{lstlisting}

Le code est correct, mais il requiert une intervention manuelle.

Parfois, Hex-Rays ne remplace pas le code de la division par la multiplication:

\begin{lstlisting}
f               proc    near

        	mov     eax, [esp+4]
	        mov     edx, 2AAAAAABh
                imul    edx
        	mov     eax, edx

	        retn

f               endp
\end{lstlisting}

Résultat:

\begin{lstlisting}
int __cdecl f(int a1)
{
  return (unsigned __int64)(715827883i64 * a1) >> 32;
}
\end{lstlisting}

Ceci peut être remplacé manuellement.

Beaucoup de ces particularités peuvent être résolues en ré-arrangeant les instructions,
recompilant le code assembleur et en le renvoyant dans Hex-Rays.

\subsection{Silence}

\begin{lstlisting}
extrn some_func:dword

f               proc    near

        	mov     ecx, [esp+4]
	        mov     eax, [esp+8]
                push    eax
        	call    some_func
	        add     esp, 4

                ; use ECX
        	mov     eax, ecx

	        retn

f               endp
\end{lstlisting}

Résultat:

\begin{lstlisting}
int __cdecl f(int a1, int a2)
{
  int v2; // ecx@1

  some_func(a2);
  return v2;
}
\end{lstlisting}

La variable \TT{v2} (de ECX) est perdue \dots
Oui, ce code est incorrect (la valeur de ECX n'est pas sauvée lors de l'appel d'une
autre fonction), mais il serait bon que Hex-Rays donne un warning.

Un autre:

\begin{lstlisting}
extrn some_func:dword

f               proc    near

	        call    some_func
        	jnz     l01

                mov     eax, 1
	        retn
l01:
	        mov     eax, 2
        	retn

f               endp
\end{lstlisting}

Résultat:

\begin{lstlisting}
signed int f()
{
  char v0; // zf@1
  signed int result; // eax@2

  some_func();
  if ( v0 )
    result = 1;
  else
    result = 2;
  return result;
}
\end{lstlisting}

Encore une fois, un warning serait utile.

En tout cas, à chaque fois que vous voyez une variable de type \TT{char}, ou une
variable qui est utilisée sans initialisation, c'est un signe clair que quelque chose
s'est mal passé et nécessite une intervention manuelle.

\subsection{Virgule}
\myindex{\CLanguageElements!Comma}

La virgule en \CCpp a mauvaise presse, car elle peut conduire à du code confus.

Quiz rapide, que renvoie cette fonction \CCpp?

\begin{lstlisting}
int f()
{
	return 1, 2;
};
\end{lstlisting}

C'est 2: lorsque le compilateur rencontre une expression avec des virgules, il génère
du code qui exécute toutes les sous-expressions, et \TT{renvoie} la valeur de la
dernière.

J'ai vu quelque chose comme ça dans du code en production:

\begin{lstlisting}
if (cond)
	return global_var=123, 456; // 456 is returned
else
	return global_var=789, 321; // 321 is returned
\end{lstlisting}

Il semble que le programmeur voulait rendre le code plus court sans parenthèses supplémentaires.
Autrement dit, la virgule permet de grouper plusieurs expressions en une seule, sans
déclaration/bloc de code dans des parenthèses.

\myindex{Scheme}
\myindex{Racket}
La virgule en C/C++ est proche du \TT{begin} en Scheme/Racket: \url{https://docs.racket-lang.org/guide/begin.html}.

Peut-être que le seul usage largement accepté de la virgule est dans les déclarations
\TT{for()}:

\begin{lstlisting}
char *s="hello, world";
for(int i=0; *s; s++, i++);
; i = string lenght
\end{lstlisting}

À la fois \TT{s++} et \TT{i++} sont exécutés à chaque itération de la boucle.

Plus d'information:\\
\url{http://stackoverflow.com/questions/52550/what-does-the-comma-operator-do-in-c}.

\myindex{\CLanguageElements!Short-circuit}
J'ai écrit tout ceci cat Hex-Rays produit (au moins dans mon cas) du code qui est
riche tant en virgules qu'en expression raccourcies:
Par exemple, ceci est une sortie réelle de Hex-Rays:

\begin{lstlisting}
 if ( a >= b || (c = a, (d[a] - e) >> 2 > f) )
    {
    	...
\end{lstlisting}

Ceci est correct, compile et fonctionne, et Dieu puisse vous aider à la comprendre.
La voici récrite:

\begin{lstlisting}
if (cond1 || (comma_expr, cond2))
{
	...
\end{lstlisting}

Le raccourci est effectif ici: d'abord \TT{cond1} est testé, si c'est \TT{true},
le corps du \TT{if()} est exécuté, le reste de l'expression du \TT{if()} est complètement
ignoré.
Si \TT{cond1} est \TT{false}, \TT{comma\_expr} est exécuté (dans l'exemple précédent,
\TT{a} est copié dans \TT{c}), puis \TT{cond2} est testée.
Si \TT{cond2} est \TT{true}, le corps du \TT{if()} est exécuté, ou pas.
Autrement dit, le corps du \TT{if()} est exécuté si \TT{cond1} est \TT{true} ou si \TT{cond2} est \TT{true},
mais si ce dernier est \TT{true}, \TT{comma\_expr} est aussi exécutée.

Maintenant, vous pouvez voir pourquoi la virgule est si célèbre.

\textbf{Un mot sur les raccourcis.}
Une idée fausse répandue de débutant est que les sous-conditions sont testées dans
un ordre indéterminé, ce qui n'est pas vrai.
Dans l'expression \TT{a | b | c},  $a$, $b$ et $c$ sont évalués dans un ordre indéterminé,
donc c'est pourquoi \TT{||} a été ajouté à C/C++, pour appliquer des raccourcis explicitement.

\subsection{Types de donnée}

Les types de donnée sont un problème pour les décompilateurs.

Hex-Rays peut ne pas voir les tableaux dans la pile locale, si ils n'ont pas été déterminés
avant la décompilation. Même histoire avec les tableaux globaux.

Un autre problème se pose avec les fonctions trop grosses, où un slot unique dans
la pile locale peut être utilisé par plusieurs variables durant l'exécution de la
fonction.
Ce n'est pas un cas rare que lorsqu'un slot est utilisé pour une variable \TT{int},
puis un pointeur, puis une variable \TT{float}.
Hex-Rays le décompile correctement: il créé une variable avec le même type, puis
la caste sur un autre type dans diverses parties de la fonction.
J'ai résolu ce problème en découpant les grosses fonctions en plusieurs plus petites.
Met les variables locales comme des globales, etc., etc.
Et n'oubliez pas les tests.

\subsection{Expressions longues et confuses}

Parfois, lors de la ré-écriture, vous pouvez vous retrouvez avec des expressions
longues et difficiles à comprendre dans des constructions \TT{if()} comme:

\begin{lstlisting}
if ((! (v38 && v30 <= 5 && v27 != -1)) && ((! (v38 && v30 <= 5) && v27 != -1) || (v24 >= 5 || v26)) && v25)
{
...
}
\end{lstlisting}

Wolfram Mathematica peut minimiser certaines d'entre elles, en utilisant la fonction
\TT{BooleanMinimize[]}:

\begin{lstlisting}
In[1]:= BooleanMinimize[(! (v38 && v30 <= 5 && v27 != -1)) && v38 && v30 <= 5 && v25 == 0]

Out[1]:= v38 && v25 == 0 && v27 == -1 && v30 <= 5
\end{lstlisting}

Il y a encore une meilleure voie, pour trouver les sous-expressions communes:

\begin{lstlisting}
In[2]:= Experimental`OptimizeExpression[(! (v38 && v30 <= 5 && 
      v27 != -1)) && ((! (v38 && v30 <= 5) && 
      v27 != -1) || (v24 >= 5 || v26)) && v25]

Out[2]= Experimental`OptimizedExpression[
 Block[{Compile`$1, Compile`$2}, Compile`$1 = v30 <= 5; 
  Compile`$2 = 
   v27 != -1; ! (v38 && Compile`$1 && 
      Compile`$2) && ((! (v38 && Compile`$1) && Compile`$2) || 
     v24 >= 5 || v26) && v25]]
\end{lstlisting}

Mathematica ajoute deux nouvelles variables: \TT{Compile`\$1} et  \TT{Compile`\$2},
qui vont être ré-utilisées plusieurs fois dans l'expression.
Donc, nous pouvons ajouter deux variables supplémentaires.

%\input{other/DeMorgan_FR}

\subsection{Mon plan}

\begin{itemize}
\item Séparer les grosses fonctions (et ne pas oublier de tester).
Parfois c'est utile de créer des nouvelles fonctions à partir des corps de boucles.

\item Vérifier/tester le type des variables, tableaux, etc.

\item Si vous voyez un résultat étrange, une variable \TT{dangling} (qui est utilisée
avant son initialisation), essayez d'échanger les instructions manuellement, recompilez
et repassez-le à Hex-Rays.
\end{itemize}

\subsection{Résumé}

Quoiqu'il en soit, la qualité de Hex-Rays 2.2.0 est très, très bonne.
Il rend la vie plus facile.

}

\EN{\section{Cyclomatic complexity}

The term is used to measure complexity of a function.
Complex functions are usually evil, because they are hard to maintain, hard to test, etc.

There are several heuristics to measure it.

For example, we can find in 
Linux kernel coding style\footnote{\url{https://www.kernel.org/doc/html/v4.10/process/coding-style.html}}:

\begin{framed}
\begin{quotation}
Now, some people will claim that having 8-character indentations makes the code move too far to the right, and makes it hard to read on a 80-character terminal screen. The answer to that is that if you need more than 3 levels of indentation, you’re screwed anyway, and should fix your program.

...

Functions should be short and sweet, and do just one thing. They should fit on one or two screenfuls of text (the ISO/ANSI screen size is 80x24, as we all know), and do one thing and do that well.

The maximum length of a function is inversely proportional to the complexity and indentation level of that function. So, if you have a conceptually simple function that is just one long (but simple) case-statement, where you have to do lots of small things for a lot of different cases, it’s OK to have a longer function.

However, if you have a complex function, and you suspect that a less-than-gifted first-year high-school student might not even understand what the function is all about, you should adhere to the maximum limits all the more closely. Use helper functions with descriptive names (you can ask the compiler to in-line them if you think it’s performance-critical, and it will probably do a better job of it than you would have done).

Another measure of the function is the number of local variables. They shouldn’t exceed 5-10, or you’re doing something wrong. Re-think the function, and split it into smaller pieces. A human brain can generally easily keep track of about 7 different things, anything more and it gets confused. You know you’re brilliant, but maybe you’d like to understand what you did 2 weeks from now.
\end{quotation}
\end{framed}

In JPL Institutional Coding Standard for the C Programming Language
\footnote{\url{https://yurichev.com/mirrors/C/JPL_Coding_Standard_C.pdf}}:

\begin{framed}
\begin{quotation}
Functions should be no longer than 60 lines of text and define no more than 6 parameters.

A function should not be longer than what can be printed on a single sheet of paper in a standard reference format with one line per statement and one line per declaration. Typically, this means no more than about 60 lines of code per function. Long lists of function parameters similarly compromise code clarity and should be avoided.

Each function should be a logical unit in the code that is understandable and verifiable as a unit. It is much harder to understand a logical unit that spans multiple screens on a computer display or multiple pages when printed. Excessively long functions are often a sign of poorly structured code.
\end{quotation}
\end{framed}

Now let's back to cyclomatic complexity.

Without diving deep into graph theory: there are basic blocks and links between them.
For example, this is how IDA shows \ac{BB}s and links (as arrows).
Just click space and you'll see this: \myref{fig:ex3_IDA_1}.
Each \ac{BB} is also called vertex or node in graph theory. Each link - edge.

There are at least two popular ways to calculate cyclomatic complexity:
1) edges - nodes + 2
2) edges - nodes + number of exits (\INS{RET} instructions)

As of IDA example below, there are 4 \ac{BB}s, so that is 4 nodes. But there are also 4 links and 1 return instruction.
By 1st rule, this is 2, by the second: 1.

The bigger the number, the more complex your function and things go from bad to worse.
As you can see, additional exit (return instructions) make things even worse,
as well as additional links between nodes (including additional goto's).

I wrote the simple IDAPython script (\url{\GitHubBlobMasterURL/other/cyclomatic/cyclomatic.py}) to measure it.
Here is result for Linux kernel 4.11 (most complex functions in it):

\begin{lstlisting}
1829c0 do_check edges=937 nodes=574 rets=1 E-N+2=365 E-N+rets=364
2effe0 ext4_fill_super edges=862 nodes=568 rets=1 E-N+2=296 E-N+rets=295
5d92e0 wm5110_readable_register edges=661 nodes=369 rets=2 E-N+2=294 E-N+rets=294
277650 do_blockdev_direct_IO edges=771 nodes=507 rets=1 E-N+2=266 E-N+rets=265
10f7c0 load_module edges=711 nodes=465 rets=1 E-N+2=248 E-N+rets=247
787730 dev_ethtool edges=559 nodes=315 rets=1 E-N+2=246 E-N+rets=245
84e440 do_ipv6_setsockopt edges=468 nodes=237 rets=1 E-N+2=233 E-N+rets=232
72c3c0 mmc_init_card edges=593 nodes=365 rets=1 E-N+2=230 E-N+rets=229
...
\end{lstlisting}
( Full list: \url{\GitHubBlobMasterURL/other/cyclomatic/linux_4.11_sorted.txt} )

This is source code of some of them:
\href{https://github.com/torvalds/linux/blob/56868a460b83c0f93d339256a81064d89aadae8e/kernel/bpf/verifier.c\#L2811}{do\_check()},
\href{https://github.com/torvalds/linux/blob/0fcc3ab23d7395f58e8ab0834e7913e2e4314a83/fs/ext4/super.c\#L3358}{ext4\_fill\_super()},
\href{https://github.com/torvalds/linux/blob/86292b33d4b79ee03e2f43ea0381ef85f077c760/fs/direct-io.c\#L1107}{do\_blockdev\_direct\_IO()},
\href{https://github.com/torvalds/linux/blob/bf5f89463f5b3109a72ed13ca62b57e90213387d/arch/x86/net/bpf_jit_comp.c\#L351}{do\_jit()}.

Most complex functions in Windows 7 ntoskrnl.exe file:

\begin{lstlisting}
140569400 sub_140569400 edges=3070 nodes=1889 rets=1 E-N+2=1183 E-N+rets=1182
14007c640 MmAccessFault edges=2256 nodes=1424 rets=1 E-N+2=834 E-N+rets=833
1401a0410 FsRtlMdlReadCompleteDevEx edges=1241 nodes=752 rets=1 E-N+2=491 E-N+rets=490
14008c190 MmProbeAndLockPages edges=983 nodes=623 rets=1 E-N+2=362 E-N+rets=361
14037fd10 ExpQuerySystemInformation edges=995 nodes=671 rets=1 E-N+2=326 E-N+rets=325
140197260 MmProbeAndLockSelectedPages edges=875 nodes=551 rets=1 E-N+2=326 E-N+rets=325
140362a50 NtSetInformationProcess edges=880 nodes=586 rets=1 E-N+2=296 E-N+rets=295
....
\end{lstlisting}

( Full list: \url{\GitHubBlobMasterURL/other/cyclomatic/win7_ntoskrnl_sorted.txt} )

From a bug hunter's standpoint, complex functions are prone to have bugs, so an attention should be paid to them.

Read more about it:
\url{https://en.wikipedia.org/wiki/Cyclomatic_complexity},
\url{http://wiki.c2.com/?CyclomaticComplexityMetric}.

Measuring cyclomatic complexity in MSVS (C\#):
\url{https://blogs.msdn.microsoft.com/zainnab/2011/05/17/code-metrics-cyclomatic-complexity/}.

Couple of other Python scripts for measuring cyclomatic complexity in IDA:
\url{http://www.openrce.org/articles/full_view/11},
\url{https://github.com/mxmssh/IDAmetrics} (incl. other metrics).

GCC plugin:
\url{https://github.com/ephox-gcc-plugins/cyclomatic_complexity}.

}



\EN{\chapter{Books/blogs worth reading}

\mysection{Books and other materials}

\subsection{Reverse Engineering}

\input{RE_books}

Also, Kris Kaspersky's books.

\subsection{Windows}

\input{Win_reading}

\subsection{\CCpp}

\input{CCppBooks}

\subsection{x86 / x86-64}

\label{x86_manuals}
\begin{itemize}
\item Intel manuals\footnote{\AlsoAvailableAs \url{http://www.intel.com/content/www/us/en/processors/architectures-software-developer-manuals.html}}

\item AMD manuals\footnote{\AlsoAvailableAs \url{http://developer.amd.com/resources/developer-guides-manuals/}}

\item \AgnerFog{}\footnote{\AlsoAvailableAs \url{http://agner.org/optimize/microarchitecture.pdf}}

\item \AgnerFogCC{}\footnote{\AlsoAvailableAs \url{http://www.agner.org/optimize/calling_conventions.pdf}}

\item \IntelOptimization

\item \AMDOptimization
\end{itemize}

Somewhat outdated, but still interesting to read:

\MAbrash\footnote{\AlsoAvailableAs \url{https://github.com/jagregory/abrash-black-book}}
(he is known for his work on low-level optimization for such projects as Windows NT 3.1 and id Quake).

\subsection{ARM}

\begin{itemize}
\item ARM manuals\footnote{\AlsoAvailableAs \url{http://infocenter.arm.com/help/index.jsp?topic=/com.arm.doc.subset.architecture.reference/index.html}}

\item \ARMSevenRef

\item \ARMSixFourRefURL

\item \ARMCookBook\footnote{\AlsoAvailableAs \url{http://go.yurichev.com/17273}}
\end{itemize}

\subsection{Assembly language}

Richard Blum --- Professional Assembly Language.

\subsection{Java}

\JavaBook.

\subsection{UNIX}

\TAOUP

\subsection{Programming in general}

\begin{itemize}

\item \RobPikePractice

\item \HenryWarren.
Some people say tricks and hacks from the book are not relevant today because they were good only for \ac{RISC} \ac{CPU}s,
where branching instructions are expensive.
Nevertheless, these can help immensely to understand boolean algebra and what all the mathematics near it.

\end{itemize}

% subsection:
\input{crypto_reading}

\iffalse
\subsection{Dedication}

As the first page of this book says, ``This book is dedicated to Robert Jourdain, John Socha, Ralf Brown and Peter Abel''.
These are authors of well-known assembly language related books and references from 1980's and 1990's:

\begin{itemize}
\item Robert Jourdain -- Programmer's problem solver for the IBM PC, XT, \& AT (1986)

\item Peter Norton and John Socha -- The Peter Norton Programmer's Guide to the IBM PC (1985), Peter Norton's Assembly Language Book for the IBM PC (1989).
In fact, John Socha is a real author of these books, it can be said, he was ghostwriter.
He is also the author of Norton Commander.

\item Ralph Brown was known for ``Ralf Brown's Interrupt List''\footnote{\url{http://www.ctyme.com/rbrown.htm}}.

\item Peter Abel -- IBM PC assembly language and programming (1991)
\end{itemize}

These are outdated books, of course.
But maybe someone will recall ``those times''.
\fi

}
\ES{\input{reading_ES}}
\RU{% TODO sync with English version
\chapter{Что стоит почитать}

\mysection{Книги и прочие материалы}

\subsection{Reverse Engineering}

\input{RE_books}

(Старое, но всё равно интересное) Pavol Cerven, \emph{Crackproof Your Software: Protect Your Software Against Crackers}, (2002).

Дмитрий Скляров --- ``Искусство защиты и взлома информации''.

Также, книги Криса Касперски.

\subsection{Windows}

\input{Win_reading}

\subsection{\CCpp}

\input{CCppBooks}

\subsection{x86 / x86-64}

\label{x86_manuals}
\begin{itemize}
\item Документация от Intel\footnote{\AlsoAvailableAs \url{http://www.intel.com/content/www/us/en/processors/architectures-software-developer-manuals.html}}

\item Документация от AMD\footnote{\AlsoAvailableAs \url{http://developer.amd.com/resources/developer-guides-manuals/}}

\item \AgnerFog{}\footnote{\AlsoAvailableAs \url{http://agner.org/optimize/microarchitecture.pdf}}

\item \AgnerFogCC{}\footnote{\AlsoAvailableAs \url{http://www.agner.org/optimize/calling_conventions.pdf}}

\item \IntelOptimization

\item \AMDOptimization
\end{itemize}

Немного устарело, но всё равно интересно почитать:

\MAbrash\footnote{\AlsoAvailableAs \url{https://github.com/jagregory/abrash-black-book}}
(он известен своей работой над низкоуровневой оптимизацией в таких проектах как Windows NT 3.1 и id Quake).

\subsection{ARM}

\begin{itemize}
\item Документация от ARM\footnote{\AlsoAvailableAs \url{http://infocenter.arm.com/help/index.jsp?topic=/com.arm.doc.subset.architecture.reference/index.html}}

\item \ARMSevenRef

\item \ARMSixFourRefURL

\item \ARMCookBook\footnote{\AlsoAvailableAs \url{http://go.yurichev.com/17273}}
\end{itemize}

\subsection{Язык ассемблера}

Richard Blum --- Professional Assembly Language.

\subsection{Java}

\JavaBook.

\subsection{UNIX}

\TAOUP

\subsection{Программирование}

\begin{itemize}

\item \RobPikePractice

\item Александр Шень\footnote{\url{http://imperium.lenin.ru/~verbit/Shen.dir/shen-progra.html}}

\item \HenryWarren.
Некоторые люди говорят, что трюки и хаки из этой книги уже не нужны, потому что годились только для \ac{RISC}-процессоров,
где инструкции перехода слишком дорогие.
Тем не менее, всё это здорово помогает лучше понять булеву алгебру и всю математику рядом.

\end{itemize}

% subsection:
\input{crypto_reading}

}
\FR{\chapter{Livres/blogs qui valent le détour}

\mysection{Livres et autres matériels}

\subsection{Rétro-ingénierie}

\input{RE_books}

(Obsolète, mais toujours intéressant) Pavol Cerven, \emph{Crackproof Your Software: Protect Your Software Against Crackers}, (2002).

Également, les livres de Kris Kaspersky.

\subsection{Windows}

\input{Win_reading}

\subsection{\CCpp}

\input{CCppBooks}

\subsection{Architecture x86 / x86-64}

\label{x86_manuals}
\begin{itemize}
\item Manuels Intel\footnote{\AlsoAvailableAs \url{http://www.intel.com/content/www/us/en/processors/architectures-software-developer-manuals.html}}

\item Manuels AMD\footnote{\AlsoAvailableAs \url{http://developer.amd.com/resources/developer-guides-manuals/}}

\item \AgnerFog{}\footnote{\AlsoAvailableAs \url{http://agner.org/optimize/microarchitecture.pdf}}

\item \AgnerFogCC{}\footnote{\AlsoAvailableAs \url{http://www.agner.org/optimize/calling_conventions.pdf}}

\item \IntelOptimization

\item \AMDOptimization
\end{itemize}

Quelque peu vieux, mais toujours intéressant à lire :

\MAbrash\footnote{\AlsoAvailableAs \url{https://github.com/jagregory/abrash-black-book}}
(il est connu pour son travail sur l'optimisation bas niveau pour des projets tels que Windows NT 3.1 et id Quake).

\subsection{ARM}

\begin{itemize}
\item Manuels ARM\footnote{\AlsoAvailableAs \url{http://infocenter.arm.com/help/index.jsp?topic=/com.arm.doc.subset.architecture.reference/index.html}}

\item \ARMSevenRef

\item \ARMSixFourRefURL

\item \ARMCookBook\footnote{\AlsoAvailableAs \url{http://go.yurichev.com/17273}}
\end{itemize}

\subsection{Langage d'assemblage}

Richard Blum --- Professional Assembly Language.

\subsection{Java}

\JavaBook.

\subsection{UNIX}

\TAOUP

\subsection{Programmation en général}

\begin{itemize}

\item \RobPikePractice

\item \HenryWarren
Certaines personnes disent que les trucs et astuces de ce livre ne sont plus pertinents
aujourd'hui, car ils n'étaient valables que pour les \ac{CPU}s \ac{RISC}, où les instructions
de branchement sont coûteuses.
Néanmoins, ils peuvent énormément aider à comprendre l'algèbre booléenne et toutes les
mathématiques associées.

\end{itemize}

%subsection:
\input{crypto_reading}

}
\DE{\chapter{Bücher / Lesenswerte Blogs}

\mysection{Bücher und andere Materialien}

\subsection{Reverse Engineering}

\input{RE_books}

Ebenfalls das Buch von Kris Kaspersky.

\subsection{Windows}

\input{Win_reading}

\subsection{\CCpp}

\input{CCppBooks}

\subsection{x86 / x86-64}

\label{x86_manuals}
\begin{itemize}
\item Intel Handbücher\footnote{\AlsoAvailableAs \url{http://www.intel.com/content/www/us/en/processors/architectures-software-developer-manuals.html}}

\item AMD Handbücher\footnote{\AlsoAvailableAs \url{http://developer.amd.com/resources/developer-guides-manuals/}}

\item \AgnerFog{}\footnote{\AlsoAvailableAs \url{http://agner.org/optimize/microarchitecture.pdf}}

\item \AgnerFogCC{}\footnote{\AlsoAvailableAs \url{http://www.agner.org/optimize/calling_conventions.pdf}}

\item \IntelOptimization

\item \AMDOptimization
\end{itemize}

Etwas veraltet aber immer noch interessant zu lesen:

\MAbrash\footnote{\AlsoAvailableAs \url{https://github.com/jagregory/abrash-black-book}}
(Er ist bekannt für seine Arbeiten auf dem Gebiet der Low-Level Optimierung in Projekten wie Windows NT 3.1 und id Quake).

\subsection{ARM}

\begin{itemize}
\item ARM Handbücher\footnote{\AlsoAvailableAs \url{http://infocenter.arm.com/help/index.jsp?topic=/com.arm.doc.subset.architecture.reference/index.html}}

\item \ARMSevenRef

\item \ARMSixFourRefURL

\item \ARMCookBook\footnote{\AlsoAvailableAs \url{http://go.yurichev.com/17273}}
\end{itemize}

\subsection{Assembler}

Richard Blum --- Professional Assembly Language.

\subsection{Java}

\JavaBook.

\subsection{UNIX}

\TAOUP

\subsection{Programmierung Allgemein}

\begin{itemize}

	\item \RobPikePractice
	
	\item \HenryWarren.
	Einige Leute sagen, die Tricks und Hacks aus diesem Buch sind heute nicht mehr relevant und haben die eigentliche Bedeutung für \ac{RISC} \ac{CPU}s, bei denen Verzweigungsbefehle teuer sind.
	Nichtsdestotrotz können diese immens hilfreich sein um Bool'sche Algebra und die damit zusammenhängende Mathematik zu verstehen.
	
\end{itemize}

% subsection:
\input{crypto_reading}
}
\IT{\chapter{Libri/blog da leggere}

\mysection{Libri ed altro materiale}

\subsection{Reverse Engineering}

\input{RE_books}

Inoltre, i libri di Kris Kaspersky.

\subsection{Windows}

\input{Win_reading}

\subsection{\CCpp}

\input{CCppBooks}

\subsection{x86 / x86-64}

\label{x86_manuals}
\begin{itemize}
\item Manuali Intel\footnote{\AlsoAvailableAs \url{http://www.intel.com/content/www/us/en/processors/architectures-software-developer-manuals.html}}

\item Manuali AMD\footnote{\AlsoAvailableAs \url{http://developer.amd.com/resources/developer-guides-manuals/}}

\item \AgnerFog{}\footnote{\AlsoAvailableAs \url{http://agner.org/optimize/microarchitecture.pdf}}

\item \AgnerFogCC{}\footnote{\AlsoAvailableAs \url{http://www.agner.org/optimize/calling_conventions.pdf}}

\item \IntelOptimization

\item \AMDOptimization
\end{itemize}

Un po' datati ma sempre interessanti:

\MAbrash\footnote{\AlsoAvailableAs \url{https://github.com/jagregory/abrash-black-book}}
(è conosciuto per i suoi lavori di ottimizzazione a basso livello su progetti come Windows NT 3.1 e id Quake).

\subsection{ARM}

\begin{itemize}
\item Manuali ARM\footnote{\AlsoAvailableAs \url{http://infocenter.arm.com/help/index.jsp?topic=/com.arm.doc.subset.architecture.reference/index.html}}

\item \ARMSevenRef

\item \ARMSixFourRefURL

\item \ARMCookBook\footnote{\AlsoAvailableAs \url{http://go.yurichev.com/17273}}
\end{itemize}

\subsection{Assembly}

Richard Blum --- Professional Assembly Language.

\subsection{Java}

\JavaBook.

\subsection{UNIX}

\TAOUP

\subsection{Programmazione in generale}

\begin{itemize}

\item \RobPikePractice

\item \HenryWarren.
Alcune persone sostengono che i trucchi e gli hack di questo libro non siano più attuali adesso perchè erano validi solo per le \ac{CPU} \ac{RISC},
dove le istruzioni di branching sono costose.
Ad ogni modo, possono aiutare enormemente a comprendere l'algebra booleana e tutta la matematica coinvolta.

\end{itemize}

% subsection:
\input{crypto_reading}

\iffalse
\subsection{Dedica}

Come scritto nella prima pagina di questo libro, ``Questo libro è dedicato a Robert Jourdain, John Socha, Ralf Brown e Peter Abel''.
Si tratta di autori famosi di libri sul linguaggio assembly e di riferimento negli anni 1980 e 1990:

\begin{itemize}
\item Robert Jourdain -- Programmer's problem solver for the IBM PC, XT, \& AT (1986)

\item Peter Norton e John Socha -- The Peter Norton Programmer's Guide to the IBM PC (1985), Peter Norton's Assembly Language Book for the IBM PC (1989).
Di fatto, John Socha è un vero autore di questi libri, si può dire, era un ghostwriter.
Inoltre è autore del Norton Commander.

\item Ralph Brown era conosciuto per la ``Ralf Brown's Interrupt List''\footnote{\url{http://www.ctyme.com/rbrown.htm}}.

\item Peter Abel -- IBM PC assembly language and programming (1991)
\end{itemize}

Chiaramente si tratta di libri antiquati.
Ma magari qualcuno si ricorderà di ``quei tempi''.
\fi
}
\JA{\chapter{読むべき本/ブログ}

\mysection{本と他の資料}

\subsection{リバースエンジニアリング}

\input{RE_books}

% TBT
% (Outdated, but still interesting) Pavol Cerven, \emph{Crackproof Your Software: Protect Your Software Against Crackers}, (2002).

そして、Kris Kasperskyの本も。

\subsection{Windows}

\input{Win_reading}

\subsection{\CCpp}

\input{CCppBooks}

\subsection{x86 / x86-64}

\label{x86_manuals}
\begin{itemize}
\item Intelマニュアル\footnote{\AlsoAvailableAs \url{http://www.intel.com/content/www/us/en/processors/architectures-software-developer-manuals.html}}

\item AMDマニュアル\footnote{\AlsoAvailableAs \url{http://developer.amd.com/resources/developer-guides-manuals/}}

\item \AgnerFog{}\footnote{\AlsoAvailableAs \url{http://agner.org/optimize/microarchitecture.pdf}}

\item \AgnerFogCC{}\footnote{\AlsoAvailableAs \url{http://www.agner.org/optimize/calling_conventions.pdf}}

\item \IntelOptimization

\item \AMDOptimization
\end{itemize}

やや時代遅れですが、それでも興味深く読めます。

\MAbrash\footnote{\AlsoAvailableAs \url{https://github.com/jagregory/abrash-black-book}}
(彼は、Windows NT 3.1やid Quakeなどのプロジェクトのための低レベルの最適化に関する仕事で知られています。)

\subsection{ARM}

\begin{itemize}
\item ARMマニュアル\footnote{\AlsoAvailableAs \url{http://infocenter.arm.com/help/index.jsp?topic=/com.arm.doc.subset.architecture.reference/index.html}}

\item \ARMSevenRef

\item \ARMSixFourRefURL

\item \ARMCookBook\footnote{\AlsoAvailableAs \url{http://go.yurichev.com/17273}}
\end{itemize}

\subsection{アセンブリ言語}

Richard Blum --- Professional Assembly Language.

\subsection{Java}

\JavaBook.

\subsection{UNIX}

\TAOUP

\subsection{プログラミング一般}

\begin{itemize}

\item \RobPikePractice

\item \HenryWarren.
本からのトリックやハックは、分岐命令が高価である\ac{RISC} \ac{CPU}にのみ適していたので、
今日は関係ないと言う人もいます。 
それにもかかわらず、これらはブール代数とそれに近いすべての数学を理解するために非常に役立ちます。

\end{itemize}

% subsection:
\input{crypto_reading}
}

\EN{\chapter{Communities}

There are two excellent \ac{RE}-related subreddits on reddit.com:
\href{http://go.yurichev.com/17027}{reddit.com/r/ReverseEngineering/} and
\href{http://go.yurichev.com/17028}{reddit.com/r/remath}
(on the topics for the intersection of \ac{RE} and mathematics).

There is also a \ac{RE} part of the Stack Exchange website:
\href{http://go.yurichev.com/17029}{reverseengineering.stackexchange.com}.

On IRC there are \#\#re and \#\#asm channels on
FreeNode\footnote{\href{http://go.yurichev.com/17030}{freenode.net}}.

}
\RU{\chapter{Сообщества}

Имеются два отличных субреддита на reddit.com посвященных \ac{RE}:
\href{http://go.yurichev.com/17027}{reddit.com/r/ReverseEngineering/} и
\href{http://go.yurichev.com/17028}{reddit.com/r/remath}

Имеется также часть сайта Stack Exchange посвященная \ac{RE}:
\href{http://go.yurichev.com/17029}{reverseengineering.stackexchange.com}.

На IRC есть каналы \#\#re и \#\#asm
FreeNode\footnote{\href{http://go.yurichev.com/17030}{freenode.net}}.

}
\DE{\chapter{Communities}

Es gibt zwei exzellente Subreddits auf reddit.com mit \ac{RE}-relevanten Themen:
\href{http://go.yurichev.com/17027}{reddit.com/r/ReverseEngineering/} und
\href{http://go.yurichev.com/17028}{reddit.com/r/remath}
(für Themen mit der Schnittmenge \ac{RE} und Mathematik).

Es gibt außerdem einen \ac{RE}-relevanten Teil auf der Stack Exchange-Seite:
\href{http://go.yurichev.com/17029}{reverseengineering.stackexchange.com}.

%TBT
%On IRC there are \#\#re and \#\#asm channels on
Im IRC gibt es einen \#\#re-Kanal auf FreeNode\footnote{\href{http://go.yurichev.com/17030}{freenode.net}}.

}
\FR{\chapter{Communautés}

Il existe deux excellents subreddits liés à la \ac{RE} (rétro-ingénierie) sur reddit.com :\\
\href{http://go.yurichev.com/17027}{reddit.com/r/ReverseEngineering/} et
\href{http://go.yurichev.com/17028}{reddit.com/r/remath}
(en lien avec la liaison de la \ac{RE} et des mathématiques).

Il y a également une section sur l'\ac{RE} sur le site Stack Exchange :

\par \href{http://go.yurichev.com/17029}{reverseengineering.stackexchange.com}.

Sur IRC, il y a les canaux \#\#re et \#\#asm sur
FreeNode\footnote{\href{http://go.yurichev.com/17030}{freenode.net}}.

}
\IT{\chapter{Community}

Esistono due eccellenti subreddit riguardo il \ac{RE} su reddit.com:
\href{http://go.yurichev.com/17027}{reddit.com/r/ReverseEngineering/} e
\href{http://go.yurichev.com/17028}{reddit.com/r/remath}
(Sugli argomenti di intersezione fra \ac{RE} e matematica).

Esiste anche una sezione relativa a \ac{RE} sul sito Stack Exchange:
\href{http://go.yurichev.com/17029}{reverseengineering.stackexchange.com}.

%TBT
%On IRC there are \#\#re and \#\#asm channels on
In IRC c'è un canale \#\#re su
FreeNode\footnote{\href{http://go.yurichev.com/17030}{freenode.net}}.
}

\input{afterword}



\EN{\input{appendix/appendix}}
\RU{\input{appendix/appendix}}
\DE{\input{appendix/appendix}}
\FR{\input{appendix/appendix}}
\IT{\input{appendix/appendix}}
% TODO split
\part*{\AcronymsUsed}

\addcontentsline{toc}{part}{\AcronymsUsed}

\begin{acronym}

\RU{	\acro{OS}[ОС]{Операционная Система}
	\acro{FAQ}[ЧаВО]{Часто задаваемые вопросы}
	\acro{OOP}[ООП]{Объектно-Ориентированное Программирование}
	\acro{PL}[ЯП]{Язык Программирования}
	\acro{PRNG}[ГПСЧ]{Генератор псевдослучайных чисел}
	\acro{ROM}[ПЗУ]{Постоянное запоминающее устройство}
	\acro{ALU}[АЛУ]{Арифметико-логическое устройство}
	\acro{PID}{ID программы/процесса}
	\acro{LF}{Line feed (подача строки) (10 или '\textbackslash{}n' в \CCpp)}
	\acro{CR}{Carriage return (возврат каретки) (13 или '\textbackslash{}r' в \CCpp)}
	\acro{LIFO}{Last In First Out (последним вошел, первым вышел)}
	\acro{MSB}{Most significant bit (самый старший бит)} % NOT BYTE!
	\acro{LSB}{Least significant bit (самый младший бит)} % NOT BYTE!
	\acro{NSA}[АНБ]{Агентство национальной безопасности}
	\acro{CFB}{Режим обратной связи по шифротексту (Cipher Feedback)}
	\acro{CSPRNG}{Криптографически стойкий генератор псевдослучайных чисел (cryptographically secure pseudorandom number generator)}
}
\EN{	\acro{OS}{Operating System}
	\acro{FAQ}{Frequently Asked Questions}
	\acro{OOP}{Object-Oriented Programming}
	\acro{PL}{Programming Language}
	\acro{PRNG}{Pseudorandom Number Generator}
	\acro{ROM}{Read-Only Memory}
	\acro{ALU}{Arithmetic Logic Unit}
	\acro{PID}{Program/process ID}
	\acro{LF}{Line Feed (10 or '\textbackslash{}n' in \CCpp)}
	\acro{CR}{Carriage Return (13 or '\textbackslash{}r' in \CCpp)}
	\acro{LIFO}{Last In First Out}
	\acro{MSB}{Most Significant Bit} % NOT BYTE!
	\acro{LSB}{Least Significant Bit} % NOT BYTE!
	\acro{NSA}{National Security Agency}
	\acro{CFB}{Cipher Feedback}
	\acro{CSPRNG}{Cryptographically Secure Pseudorandom Number Generator}
	\acro{ABI}{Application Binary Interface}
}
\ES{\input{acro_ES}}
\DE{\input{acro_DE}}
\IT{	\acro{OS}{Sistema Operativo (Operating System)}
	\acro{FAQ}{Domande Frequente (Frequently Asked Questions)}
	\acro{OOP}{Programmazione ad oggetti (Object-Oriented Programming)}
	\acro{PL}{Linguaggio di programmazione (Programming Language)}
	\acro{PRNG}{Generatore di numeri pseudo-casuali (Pseudorandom Number Generator)}
	\acro{ROM}{Read-Only Memory}
	\acro{ALU}{Arithmetic Logic Unit}
	\acro{PID}{Program/process ID}
	\acro{LF}{Line Feed (10 o '\textbackslash{}n' in \CCpp)}
	\acro{CR}{Carriage Return (13 o '\textbackslash{}r' in \CCpp)}
	\acro{LIFO}{Last In First Out}
	\acro{MSB}{Most Significant Bit} % NOT BYTE!
	\acro{LSB}{Least Significant Bit} % NOT BYTE!
	\acro{NSA}{National Security Agency}
	\acro{CFB}{Cipher Feedback}
	\acro{CSPRNG}{Cryptographically Secure Pseudorandom Number Generator}
	\acro{ABI}{Application Binary Interface}
}
\NL{\input{acro_NL}}
\FR{	\acro{OS}[OS]{Système d'exploitation (Operating System)}
	\acro{FAQ}{Foire Aux Questions}
	\acro{OOP}[POO]{Programmation orientée objet}
	\acro{PL}[LP]{Langage de programmation}
	\acro{PRNG}{Nombre généré pseudo-aléatoirement}
	\acro{ROM}{Mémoire morte}
	\acro{ALU}[UAL]{Unité arithmétique et logique}
	\acro{PID}{ID d'un processus}
	\acro{LF}{Line feed (10 ou '\textbackslash{}n' en \CCpp)}
	\acro{CR}{Carriage return (13 ou '\textbackslash{}r' en \CCpp)}
	\acro{LIFO}{Dernier entré, premier sorti}
	\acro{MSB}{Bit le plus significatif} % NOT BYTE!
	\acro{LSB}{Bit le moins significatif} % NOT BYTE!
	\acro{NSA}{National Security Agency (Agence Nationale de la Sécurité)} % translation not used in French
	\acro{CFB}{Cipher Feedback}
	\acro{CSPRNG}{Cryptographically Secure Pseudorandom Number Generator (générateur de nombres pseudo-aléatoire cryptographiquement sûr)}
	\acro{ABI}{Application Binary Interface}
}
\JA{	\acro{OS}{オペレーティングシステム}
	\acro{FAQ}{よくある質問}
	\acro{OOP}{オブジェクト指向プログラミング}
	\acro{PL}{プログラミング言語}
	\acro{PRNG}{擬似乱数生成器}
	\acro{ROM}{読み取り専用メモリ}
	\acro{ALU}{算術論理ユニット}
	\acro{PID}{プログラム/プロセスID}
	\acro{LF}{ラインフィード (\CCpp で10 または '\textbackslash{}n')}
	\acro{CR}{キャリッジリターン (\CCpp で13 または '\textbackslash{}r')}
	\acro{LIFO}{後入れ先出し}
	\acro{MSB}{最上位ビット} % NOT BYTE!
	\acro{LSB}{最下位ビット} % NOT BYTE!
	\acro{NSA}{国家安全保障局}
	\acro{CFB}{暗号フィードバック}
	\acro{CSPRNG}{暗号論的擬似乱数生成器}
	\acro{ABI}{アプリケーション・バイナリー・インタフェース}
}

\acro{RA}{\ReturnAddress}
\acro{PE}{Portable Executable}
\acro{SP}{\gls{stack pointer}. SP/ESP/RSP \InENRU x86/x64. SP \InENRU ARM.}
\acro{DLL}{Dynamic-Link Library}
\acro{PC}{Program Counter. IP/EIP/RIP \InENRU x86/64. PC \InENRU ARM.}
\acro{LR}{Link Register}
\acro{IDA}{
	\RU{Интерактивный дизассемблер и отладчик, разработан \href{https://hex-rays.com/}{Hex-Rays}}%
	\EN{Interactive Disassembler and Debugger developed by \href{https://hex-rays.com/}{Hex-Rays}}%
	\IT{\ac{TBT} by \href{https://hex-rays.com/}{Hex-Rays}}%
	\ES{Desensamblador Interactivo y depurador desarrollado por \href{https://hex-rays.com/}{Hex-Rays}}%
	\NL{Interactive Disassembler en debugger ontwikkeld door \href{https://hex-rays.com}{Hex-Rays}}
	\DE{Interaktiver Disassembler und Debugger entwickelt von \href{https://hex-rays.com/}{Hex-Rays}}%
	\FR{Désassembleur interactif et débogueur développé par \href{https://hex-rays.com/}{Hex-Rays}}%
	\JA{\href{https://hex-rays.com/}{Hex-Rays} によって開発されたインタラクティブなディスアセンブラ・デバッガ}%
}
\acro{IAT}{Import Address Table}
\acro{INT}{Import Name Table}
\acro{RVA}{Relative Virtual Address}
\acro{VA}{Virtual Address}
\acro{OEP}{Original Entry Point}
\acro{MSVC}{Microsoft Visual C++}
\acro{MSVS}{Microsoft Visual Studio}
\acro{ASLR}{Address Space Layout Randomization}
\acro{MFC}{Microsoft Foundation Classes}
\acro{TLS}{Thread Local Storage}
\acro{AKA}{
        \EN{Also Known As}%
	\FR{Aussi connu sous le nom de}%
	\RU{ - (Также известный как)}%
	\ES{ - (Tambi\'en Conocido Como)}%
	\NL{ - (Ook gekend als)}%
	\ITph{}
	\JA{別名}%
	\DEph{}
}
\acro{CRT}{C Runtime library}
\acro{CPU}{Central Processing Unit}
\acro{GPU}{Graphics Processing Unit}
\acro{FPU}{Floating-Point Unit}
\acro{CISC}{Complex Instruction Set Computing}
\acro{RISC}{Reduced Instruction Set Computing}
\acro{GUI}{Graphical User Interface}
\acro{RTTI}{Run-Time Type Information}
\acro{BSS}{Block Started by Symbol}
\acro{SIMD}{Single Instruction, Multiple Data}
\acro{BSOD}{Blue Screen of Death}
\acro{DBMS}{Database Management Systems}
\acro{ISA}{Instruction Set Architecture\RU{ (Архитектура набора команд)}}
\acro{CGI}{Common Gateway Interface}
\acro{HPC}{High-Performance Computing}
\acro{SOC}{System on Chip}
\acro{SEH}{Structured Exception Handling}
\acro{ELF}{\RU{Формат исполняемых файлов, использующийся в Linux и некоторых других *NIX}
\EN{Executable File format widely used in *NIX systems including Linux}
\FR{Format de fichier exécutable couramment utilisé sur les systèmes *NIX, Linux inclus}
\JA{Linuxを含め*NIXシステムで広く使用される実行ファイルフォーマット}
\ITph{}
\DEph{}}
\acro{TIB}{Thread Information Block}
\acro{TEA}{Tiny Encryption Algorithm}
\acro{PIC}{Position Independent Code}
\acro{NAN}{Not a Number}
\acro{NOP}{No Operation}
\acro{BEQ}{(PowerPC, ARM) Branch if Equal}
\acro{BNE}{(PowerPC, ARM) Branch if Not Equal}
\acro{BLR}{(PowerPC) Branch to Link Register}
\acro{XOR}{eXclusive OR\RU{ (исключающее \q{ИЛИ})}\FR{ (OU exclusif)}}
\acro{MCU}{Microcontroller Unit}
\acro{RAM}{Random-Access Memory}
\acro{GCC}{GNU Compiler Collection}
\acro{EGA}{Enhanced Graphics Adapter}
\acro{VGA}{Video Graphics Array}
\acro{API}{Application Programming Interface}
\acro{ASCII}{American Standard Code for Information Interchange}
\acro{ASCIIZ}{ASCII Zero (\RU{ASCII-строка заканчивающаяся нулем}\EN{null-terminated ASCII string}
\FR{chaîne ASCII terminée par un octet nul (à zéro)}\JA{ヌル終端文字列})}
\acro{IA64}{Intel Architecture 64 (Itanium)}
\acro{EPIC}{Explicitly Parallel Instruction Computing}
\acro{OOE}{Out-of-Order Execution}
\acro{MSDN}{Microsoft Developer Network}
\acro{STL}{(\Cpp) Standard Template Library}
\acro{PODT}{(\Cpp) Plain Old Data Type}
\acro{HDD}{Hard Disk Drive}
\acro{VM}{Virtual Memory\RU{ (виртуальная память)}\FR{ (mémoire virtuelle)}}
\acro{WRK}{Windows Research Kernel}
\acro{GPR}{General Purpose Registers\RU{ (регистры общего пользования)}}
\acro{SSDT}{System Service Dispatch Table}
\acro{RE}{Reverse Engineering}
\acro{RAID}{Redundant Array of Independent Disks}
\acro{SSE}{Streaming SIMD Extensions}
\acro{BCD}{Binary-Coded Decimal}
\acro{BOM}{Byte Order Mark}
\acro{GDB}{GNU Debugger}
\acro{FP}{Frame Pointer}
\acro{MBR}{Master Boot Record}
\acro{JPE}{Jump Parity Even (\RU{инструкция x86}\EN{x86 instruction}\FR{instruction x86}\JA{x86命令}\DEph{})}
\acro{CIDR}{Classless Inter-Domain Routing}
\acro{STMFD}{Store Multiple Full Descending (\DEph{}\RU{инструкция ARM}\EN{ARM instruction}\FR{instruction ARM}\JA{ARM命令})}
\acro{LDMFD}{Load Multiple Full Descending (\DEph{}\RU{инструкция ARM}\EN{ARM instruction}\FR{instruction ARM}\JA{ARM命令})}
\acro{STMED}{Store Multiple Empty Descending (\DEph{}\RU{инструкция ARM}\EN{ARM instruction}\FR{instruction ARM}\JA{ARM命令})}
\acro{LDMED}{Load Multiple Empty Descending (\DEph{}\RU{инструкция ARM}\EN{ARM instruction}\FR{instruction ARM}\JA{ARM命令})}
\acro{STMFA}{Store Multiple Full Ascending (\DEph{}\RU{инструкция ARM}\EN{ARM instruction}\FR{instruction ARM}\JA{ARM命令})}
\acro{LDMFA}{Load Multiple Full Ascending (\DEph{}\RU{инструкция ARM}\EN{ARM instruction}\FR{instruction ARM}\JA{ARM命令})}
\acro{STMEA}{Store Multiple Empty Ascending (\DEph{}\RU{инструкция ARM}\EN{ARM instruction}\FR{instruction ARM}\JA{ARM命令})}
\acro{LDMEA}{Load Multiple Empty Ascending (\DEph{}\RU{инструкция ARM}\EN{ARM instruction}\FR{instruction ARM}\JA{ARM命令})}
\acro{APSR}{(ARM) Application Program Status Register}
\acro{FPSCR}{(ARM) Floating-Point Status and Control Register}
\acro{RFC}{Request for Comments}
\acro{TOS}{Top of Stack\RU{ (вершина стека)}}
\acro{LVA}{(Java) Local Variable Array\RU{ (массив локальных переменных)}}
\acro{JVM}{Java Virtual Machine}
\acro{JIT}{Just-In-Time compilation}
\acro{CDFS}{Compact Disc File System}
\acro{CD}{Compact Disc}
\acro{ADC}{Analog-to-Digital Converter}
\acro{EOF}{End of File\DEph{}\RU{ (конец файла)}\FR{ (fin de fichier)}\JA{(ファイル終端)}}
\acro{TBT}{To be Translated. The presence of this acronym in this place means that the English version has some new/modified content which is to be translated and placed right here.}
\acro{DIY}{Do It Yourself}
\acro{MMU}{Memory Management Unit}
\acro{DES}{Data Encryption Standard}
\acro{MIME}{Multipurpose Internet Mail Extensions}
\acro{DBI}{Dynamic Binary Instrumentation}
\acro{XML}{Extensible Markup Language}
\acro{JSON}{JavaScript Object Notation}
\acro{URL}{Uniform Resource Locator}
\acro{ISP}{Internet Service Provider}
\acro{IV}{Initialization Vector}
\acro{RSA}{Rivest Shamir Adleman}
\acro{CPRNG}{Cryptographically secure PseudoRandom Number Generator}
\acro{GiB}{Gibibyte}
\acro{CRC}{Cyclic redundancy check}
\acro{AES}{Advanced Encryption Standard}
\acro{GC}{Garbage Collector}
\acro{IDE}{Integrated development environment}
\acro{BB}{Basic Block}

\end{acronym}


\bookmarksetup{startatroot}

\clearpage
\phantomsection
\addcontentsline{toc}{chapter}{%
    \RU{Глоссарий}%
    \EN{Glossary}%
    \ES{Glosario}%
    \PTBRph{}%
    \DE{Glossar}%
    \PLph{}%
    \IT{Glossario}%
    \THAph{}\NLph{}%
    \FR{Glossaire}%
    \JA{用語}
    \TR{Bolum}
}
\printglossaries

\clearpage
\phantomsection
\printindex

\end{document}
